%!TEX root = ../main.tex

%%%%%%%%%%%%%%%%%%%%%%%%%%%%%%%%%%%%%%%%%%%%%%%%%%%%%%%%%%%%%%%%
% Anmerkungen zur Verwendung:
%%%%%%%%%%%%%%%%%%%%%%%%%%%%%%%%%%%%%%%%%%%%%%%%%%%%%%%%%%%%%%%%
%
% nur verwendete Akronyme werden letztlich im Abkürzungsverzeichnis des Dokuments angezeigt
% Verwendung: 
%		\ac{Abk.}   --> fügt die Abkürzung ein, beim ersten Aufruf wird zusätzlich automatisch die ausgeschriebene Version davor eingefügt bzw. in einer Fußnote (hierfür muss in header.tex \usepackage[printonlyused,footnote]{acronym} stehen) dargestellt
%		\acs{Abk.}   -->  fügt die Abkürzung ein
%		\acf{Abk.}   --> fügt die Abkürzung UND die Erklärung ein
%		\acl{Abk.}   --> fügt nur die Erklärung ein
%		\acp{Abk.}  --> gibt Plural aus (angefügtes 's'); das zusätzliche 'p' funktioniert auch bei obigen Befehlen
%	siehe auch: http://golatex.de/wiki/%5Cacronym
%
%%%%%%%%%%%%%%%%%%%%%%%%%%%%%%%%%%%%%%%%%%%%%%%%%%%%%%%%%%%%%%%%
\cleardoublepage
\addcontentsline{toc}{chapter}{Acronyms}
\thispagestyle{plain}
\section*{Acronyms}
% \vspace*{-.5cm}
% \rule{\textwidth}{.5pt}
% \vspace*{-.5cm}

% \addchap{\langabkverz}
\begin{acronym}[mmmmmm] %hier längstes Acro
% \begin{doublespacing}
% \setlength{\itemsep}{-\parsep}

\acro{CCT}{critical clearing time}
\acro{EAC}{equal area criterion}
\acro{IBB}{infinite bus bar}
% \acrodefplural{CCM}[CCMs]{Catalyst Coated Membranes}
\acro{ODE}{ordinary differential equation}
\acro{SG}{synchronous generator}
\acro{SMIB}{single machine infinite bus}
\acro{TDS}{time domain solution}
\acro{TSA}{transient stability assessment}

\end{acronym}

%%%%%%%%%%%%%%%%%%%%%%%%%%%%%%%%%%%%%%%%%%%%%%%%%%%%%%%%%%%%%%%%
\vspace{1cm}
\addcontentsline{toc}{chapter}{Symbols}	
\thispagestyle{plain}
\section*{Symbols}
% \vspace*{-.5cm}
% \rule{\textwidth}{.5pt}
% \vspace*{-.8cm}

% \addchap{Symbols}
\begin{tabbing}
    XXXXXXXXX \= XXXXXXXX \= XXXXXXXXXXXXXXXXXXXXXXXXXXXXXXXXXXXXXXXXXXXXXXXXX \kill
    $\delta$            \> $^\circ$ / deg                   \> power angle (or power angle difference) \\
    $\Delta\omega$      \> $\mathrm{\frac{1}{s}}$           \> change of rotor angular speed \\
    $A$                 \> -                                \> acceleration or deceleration area \\
    $E$                 \> V                                \> voltage of \acs{SG} or \acs{IBB} \\
    $H_\mathrm{gen}$    \> s                                \> inertia constant of a \acf{SG} \\
    $I$                 \> A                                \> current \\
    $P$                 \> W                                \> Power; electrical or mechanical \\
    $V$                 \> V                                \> voltage \\
    $X$                 \> $\mathrm{\Omega}$                \> reactance \\
    $Y$                 \> $\mathrm{\frac{1}{\Omega}}$ / S  \> admittance \\
\end{tabbing}

In the simulations and calculations the per unit system is preferred, thus using all values as per reference unit ($\mathrm{p.u.}$). Where necessary indices are used to differentiate between similar symbols with different values.