%!TEX root = ../main.tex

\cleardoublepage
\pagestyle{empty}


% \renewcommand{\abstractname}{\langabstract} % Text für Überschrift

% \begin{otherlanguage}{english} % auskommentieren, wenn Abstract auf Deutsch sein soll
% \begin{abstract}
%     \begin{description}
%         \item[Objektivität] soll sich jeder persönlichen Wertung enthalten
%         \item[Kürze] soll so kurz wie möglich sein
%         \item[Genauigkeit] soll genau die Inhalte und die Meinung der Originalarbeit wiedergeben
%     \end{description}
%     Diese etwa einseitige Zusammenfassung soll es dem Leser ermöglichen, Inhalt der Arbeit und Vorgehensweise des Autors rasch zu überblicken. Gegenstand des Abstract sind insbesondere 
%     \begin{itemize}
%         \item Problemstellung der Arbeit,
%         \item im Rahmen der Arbeit geprüfte Hypothesen bzw. beantwortete Fragen,
%         \item der Analyse zugrunde liegende Methode,
%         \item wesentliche, im Rahmen der Arbeit gewonnene Erkenntnisse,
%         \item Einschränkungen des Gültigkeitsbereichs (der Erkenntnisse) sowie nicht beantwortete Fragen. 
%     \end{itemize}
% \end{abstract}

\section*{Abstract}
% \vspace*{-.5cm}
% \rule{\textwidth}{.5pt}
% \vspace*{-.5cm}

The goal of this \arbeit~is the determination of the \acf{CCT}, looking at a \acf{SG} in a simplified \acf{SMIB} model. For this, a simple three-phase fault scenario is applied, and the generator swing equation is solved in the time domain with a Python-integrated solver. A function is implemented to calculate the \acs{CCT} numerically, the result is compared to the analytical solution. Two additional interruption scenarios are constructed, simulated and evaluated. These three cases illustrate the transient stability of a generator against an \acf{IBB}, especially in a visual context with selected plots. The numerical algorithm shows satisfying results compared to the analytical. Limitations rely on the complexity of the considered electrical network, also in the interrupting scenario, and the missing possible machine interaction. Further, the damping in the system is neglected completely.

\section*{Kurzfassung}
% \vspace*{-.5cm}
% \rule{\textwidth}{.5pt}
% \vspace*{-.5cm}

Das Ziel dieser Seminararbeit ist die Bestimmung der kritischen Fehlerklärungszeit am Beispiel eines Synchrongenerators in einem vereinfachten Einmaschinenmodell. Zu diesem Zweck wird ein einfaches dreiphasiges Fehlerszenario angewandt, und die dynamische Bewegungsgleichung des Generators wird im Zeitbereich mit einem in Python integrierten Algorithmus gelöst. Eine Funktion wird implementiert, um die kritische Fehlerklärungszeit numerisch zu berechnen, das Ergebnis wird mit der analytischen Lösung verglichen. Zwei weitere Unterbrechungsszenarien werden erstellt, simuliert und
ausgewertet. Diese drei Fälle veranschaulichen die transiente Stabilität eines Generators gegenüber einer unendlichen Sammelschiene, insbesondere in einem visuellen Kontext mit ausgewählten Diagrammen. Der numerische Algorithmus zeigt im Vergleich zum analytischen zufriedenstellende Ergebnisse. Limitationen zeigen sich in der Komplexität des betrachteten elektrischen Netzes, vor allem auch im Unterbrechungsszenario, und die fehlende mögliche Interaktion von multiplen Maschinen im Netz. Des weiteren wird die Dämpfung im System vollständig vernachlässigt.
