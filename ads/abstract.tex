%!TEX root = ../main.tex

\pagestyle{empty}


\renewcommand{\abstractname}{\langabstract} % Text für Überschrift

\begin{otherlanguage}{english} % auskommentieren, wenn Abstract auf Deutsch sein soll
\begin{abstract}

% \begin{description}
% \item[Objektivität] soll sich jeder persönlichen Wertung enthalten
% \item[Kürze] soll so kurz wie möglich sein
% \item[Genauigkeit] soll genau die Inhalte und die Meinung der Originalarbeit wiedergeben
% \end{description}

% Diese etwa einseitige Zusammenfassung soll es dem Leser ermöglichen, Inhalt der Arbeit und Vorgehensweise
% des Autors rasch zu überblicken. Gegenstand des Abstract sind insbesondere 
% \begin{itemize}
% \item Problemstellung der Arbeit,
% \item im Rahmen der Arbeit geprüfte Hypothesen bzw. beantwortete Fragen,
% \item der Analyse zugrunde liegende Methode,
% \item wesentliche, im Rahmen der Arbeit gewonnene Erkenntnisse,
% \item Einschränkungen des Gültigkeitsbereichs (der Erkenntnisse) sowie nicht beantwortete Fragen. 
% \end{itemize}

Regarding the increasing decarbonization, there are some problems using renewable energy sources, because of the time-variant accessibility. Energy storage systems are used to solve this problem and in addition are making renewable energy more flexible. A Redox-Flow-Battery is such a storage system and is the topic of an innovation and research project at Schaeffler. The goal of this work is the validation of the project scope and the further proceeding to gain technology maturity using a technic-economic analysis. After presenting the technical basics of this topic, market analysis has been carried out. Combined with patent research the state of the art can be displayed. Further, the designed stack is described, and its cost structure is deduced. With comparison to other cost structure models the distribution, main cost drivers, and potential for optimization are elaborated. A technic-economic rating of the main project goal, developing a metal-based bipolar plate, is done with the help of VDI 2225. The overall discussion is formed by these parts of the work. The result of the work is confirming the technical maturity. Subject to pending system tests and possible cost reductions, the presented Redox-Flow-Battery is already a competitive product. With a suitable strategic approach in the planning of future pre-development projects, the product can be quickly launched into a series project. These strategic considerations are part of the overall discussion. No part of this work is a systematic literature review on current developments, and the complete dimensioning of a Redox-Flow-Battery Stack or its components. In addition, the presented cost model is not technically linked, which means changes in the dimensioning cannot be considered. 

\end{abstract}
\end{otherlanguage} % auskommentieren, wenn Abstract auf Deutsch sein soll

%%%%%%%%%%%%%%%%%%%%%%%%%%%%%%%%%%%%%%%%%%%%%%%%%%%%%%%%%%%%%%%%%%%%%%%%%%%%%%%%%%%%%%%%%%%%
\newpage
%%%%%%%%%%%%%%%%%%%%%%%%%%%%%%%%%%%%%%%%%%%%%%%%%%%%%%%%%%%%%%%%%%%%%%%%%%%%%%%%%%%%%%%%%%%%

\renewcommand{\abstractname}{\langabstract} % Text für Überschrift
\begin{abstract}
Im Zuge von zunehmender Dekarbonisierung steigt der Nutzungsanteil von vornehmlich zeitvariant verfügbaren, regenerativen Energiequellen. Um diese verfügbarer zu machen und flexibler nutzen zu können, werden Energiespeichersysteme eingesetzt. Zu diesen Systemen zählen Redox-Flow-Batterien, welche bei Schaeffler Thema in einem Forschungs- und Innovationsprojekt sind. Um den Fokus in diesem Projekt zu validieren und das weitere Vorgehen hinsichtlich eines beabsichtigten, steigenden Technologiereifegrades zu prüfen, soll eine technisch-ökonomische Analyse Ziel der Arbeit sein. Um dazu Aussagen treffen zu können, wird nach dem Darlegen der Grundlagen eine Einschätzung des Marktes erarbeitet. Zusammen mit einer Patentrecherche bildet diese einen Stand der Technik ab. Der im Projekt designte Stack wird anschließend beschrieben. Eine Ableitung der Kostenstruktur gepaart mit der Analyse von alternativen Kostenmodellen ermöglicht eine Aussage über Verteilung, Kostentreiber und Ansatzpunkte zur Optimierung. Der maßgebliche Projektgegenstand, die Entwicklung einer metallischen Bipolarplatte, wird mit einer technisch-wirtschaftlichen Lösungsbewertung nach VDI 2225 validiert. Aus diesen Bestandteilen formiert sich die abschließende Gesamtdiskussion. Das Ergebnis der Arbeit ermöglicht die Bestätigung des Reifegrades der Entwicklung. Unter Vorbehalt von noch ausstehenden Systemtests und möglichen Kostenreduktionen handelt es sich bereits um ein konkurrenzfähiges Produkt. Mit einem geeigneten Ansatz in der strategischen Ausrichtung von zukünftigen Vorentwicklungsprojekten, ist so eine zügige Überführung in ein Serienprojekt denkbar. Dieses Szenario ist Bestandteil der Gesamtdiskussion. Nicht Bestandteil der Arbeit ist eine systematische Literaturrecherche zu aktuellen Entwicklungen sowie die vollständige Auslegung eines Redox-Flow-Batterie Stacks oder Komponenten dieses. Darüber hinaus ist das vorgestellte Kostenmodell nicht technisch gekoppelt, wodurch Veränderungen in der Auslegung nicht berücksichtigt werden können.
\end{abstract}