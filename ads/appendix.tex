% !TeX root = ../main.tex

\appendixtoc
% \appendix
\addcontentsline{maintoc}{chapter}{Appendix}
\ohead[]{\textsc{Appendix}}
\label{app:appendix}

% \renewcommand\thechapter{\roman{chapter}}
% \setcounter{chapter}{0}

% \pagebreak
% \includepdf[pages=-,scale=.9,pagecommand={}]{Aufgabenstellung.pdf} 
% PDF um 10% verkleinert einbinden --> Kopf- und Fußzeile  werden so korrekt dargestellt. Die Option `pages' ermöglicht es, eine bestimmte Sequenz von Seiten (z.B. 2-10 oder `-' für alle Seiten) auszuwählen.
% \pagebreak
%\includepdf[pages=-,scale=.8,pagecommand=\section*{A. eventGenerator.py}]{../appendix/eventGenerator.py.pdf}
%\includepdf[pages=-,scale=.8,pagecommand=\section*{B. sendEvents.py}]{../appendix/sendEvents.py.pdf}

%%%%%%%%%%%%%%%%%%%%%%%%%%%%%%%%%%%%%%%%%%
%%%%%%%%%%%%%%%%%%%%%%%%%%%%%%%%%%%%%%%%%%
%%%%%%%%%%%%%%%%%%%%%%%%%%%%%%%%%%%%%%%%%%
\centering

\chapter{Graphics and tables}

\section{Initial values}
\label{app:initials}

\begin{centering}
    \small
    % \caption[short]{title}
    % \vspace{12pt}
    \begin{tabularx}{\textwidth}{l|X|l|l}
        % \hline % \rowcolor{lightgray} 
        & fault 1 & fault 2 & fault 3 \\ \hline \hline
        $P_\mathrm{e,~max}$     & \multicolumn{3}{c}{1.2 p.u.} \\ \hline
        $P_\mathrm{mech}$       & 0.9 p.u. & 1.0 p.u. & 0.7 p.u. \\ \hline
        $E_\mathrm{ibb}$        & \multicolumn{3}{c}{1 p.u.} \\ \hline
        $E_\mathrm{gen}$        & \multicolumn{3}{c}{1.14 p.u.} \\ \hline
        $\delta_\mathrm{ibb}$   & 48.6 deg & 56.4 deg & 35.7 deg \\ \hline
        $\delta_\mathrm{gen}$   & \multicolumn{3}{c}{0 deg} \\ \hline

        $Y_\mathrm{bus,~init}$ & \multicolumn{3}{c}{$\begin{bmatrix} -\frac{1j}{X_\mathrm{gen}}-\frac{3j}{X_\mathrm{line}} & \frac{3j}{X_\mathrm{line}} \\ \frac{3j}{X_\mathrm{line}} & -\frac{3j}{X_\mathrm{line}}-\frac{1j}{X_\mathrm{ibb}}\end{bmatrix}$} \\ \hline

        $Y_\mathrm{bus,~init}$ & $\begin{bmatrix} -\frac{1j}{X_\mathrm{gen}}-\frac{3j}{X_\mathrm{line}} + 10^6 & \frac{3j}{X_\mathrm{line}} \\ \frac{3j}{X_\mathrm{line}} & -\frac{3j}{X_\mathrm{line}}-\frac{1j}{X_\mathrm{ibb}}\end{bmatrix}$ 
        & \multicolumn{2}{c}{$\begin{bmatrix} -\frac{1j}{X_\mathrm{gen}}-\frac{2j}{X_\mathrm{line}} & \frac{2j}{X_\mathrm{line}} \\ \frac{2j}{X_\mathrm{line}} & -\frac{2j}{X_\mathrm{line}}-\frac{1j}{X_\mathrm{ibb}}\end{bmatrix}$} \\ \hline \hline

        $t_\mathrm{start}$      & \multicolumn{3}{c}{-1 s} \\ \hline
        $t_\mathrm{end}$        & \multicolumn{2}{c|}{2 s} & 5 s \\ \hline
        $t_\mathrm{step}$       & \multicolumn{3}{c}{0.001 s} \\ \hline
        $t_\mathrm{fault,~start}$   & \multicolumn{3}{c}{0 s} \\
    \end{tabularx}
\end{centering}

\flushleft
For the modules {\itshape parameter\_comparison.py} and {\itshape comparison\_alg-vs-nonalg.py} the initial values from fault one are used, besides the described variations. \\[24pt]

\begin{minipage}[l]{.49\textwidth}
    \begin{tabular}{l|r}
        $X_\mathrm{line}$       & 1.95 p.u. \\ \hline
        $X_\mathrm{gen}$        & 0.2 p.u. \\ \hline
        $X_\mathrm{ibb}$        & 0.1 p.u. \\ 
    \end{tabular}
\end{minipage}
\begin{minipage}[r]{.49\textwidth}
    \begin{tabular}{l|r}
        $H_\mathrm{gen}$        & 3.3 s \\ \hline
        $f_\mathrm{n}$          & 50 $\mathrm{\frac{1}{s}}$ \\ \hline
        $\omega$                & 0 \\ 
    \end{tabular}
\end{minipage}

\section{Fault 1}
\label{app:fault1}

%% Creator: Matplotlib, PGF backend
%%
%% To include the figure in your LaTeX document, write
%%   \input{<filename>.pgf}
%%
%% Make sure the required packages are loaded in your preamble
%%   \usepackage{pgf}
%%
%% Also ensure that all the required font packages are loaded; for instance,
%% the lmodern package is sometimes necessary when using math font.
%%   \usepackage{lmodern}
%%
%% Figures using additional raster images can only be included by \input if
%% they are in the same directory as the main LaTeX file. For loading figures
%% from other directories you can use the `import` package
%%   \usepackage{import}
%%
%% and then include the figures with
%%   \import{<path to file>}{<filename>.pgf}
%%
%% Matplotlib used the following preamble
%%   
%%   \usepackage{fontspec}
%%   \setmainfont{Charter.ttc}[Path=\detokenize{/System/Library/Fonts/Supplemental/}]
%%   \setsansfont{DejaVuSans.ttf}[Path=\detokenize{/opt/homebrew/lib/python3.10/site-packages/matplotlib/mpl-data/fonts/ttf/}]
%%   \setmonofont{DejaVuSansMono.ttf}[Path=\detokenize{/opt/homebrew/lib/python3.10/site-packages/matplotlib/mpl-data/fonts/ttf/}]
%%   \makeatletter\@ifpackageloaded{underscore}{}{\usepackage[strings]{underscore}}\makeatother
%%
\begingroup%
\makeatletter%
\begin{pgfpicture}%
\pgfpathrectangle{\pgfpointorigin}{\pgfqpoint{6.000000in}{8.000000in}}%
\pgfusepath{use as bounding box, clip}%
\begin{pgfscope}%
\pgfsetbuttcap%
\pgfsetmiterjoin%
\definecolor{currentfill}{rgb}{1.000000,1.000000,1.000000}%
\pgfsetfillcolor{currentfill}%
\pgfsetlinewidth{0.000000pt}%
\definecolor{currentstroke}{rgb}{1.000000,1.000000,1.000000}%
\pgfsetstrokecolor{currentstroke}%
\pgfsetdash{}{0pt}%
\pgfpathmoveto{\pgfqpoint{0.000000in}{0.000000in}}%
\pgfpathlineto{\pgfqpoint{6.000000in}{0.000000in}}%
\pgfpathlineto{\pgfqpoint{6.000000in}{8.000000in}}%
\pgfpathlineto{\pgfqpoint{0.000000in}{8.000000in}}%
\pgfpathlineto{\pgfqpoint{0.000000in}{0.000000in}}%
\pgfpathclose%
\pgfusepath{fill}%
\end{pgfscope}%
\begin{pgfscope}%
\pgfsetbuttcap%
\pgfsetmiterjoin%
\definecolor{currentfill}{rgb}{1.000000,1.000000,1.000000}%
\pgfsetfillcolor{currentfill}%
\pgfsetlinewidth{0.000000pt}%
\definecolor{currentstroke}{rgb}{0.000000,0.000000,0.000000}%
\pgfsetstrokecolor{currentstroke}%
\pgfsetstrokeopacity{0.000000}%
\pgfsetdash{}{0pt}%
\pgfpathmoveto{\pgfqpoint{0.750000in}{3.960000in}}%
\pgfpathlineto{\pgfqpoint{5.400000in}{3.960000in}}%
\pgfpathlineto{\pgfqpoint{5.400000in}{7.040000in}}%
\pgfpathlineto{\pgfqpoint{0.750000in}{7.040000in}}%
\pgfpathlineto{\pgfqpoint{0.750000in}{3.960000in}}%
\pgfpathclose%
\pgfusepath{fill}%
\end{pgfscope}%
\begin{pgfscope}%
\pgfpathrectangle{\pgfqpoint{0.750000in}{3.960000in}}{\pgfqpoint{4.650000in}{3.080000in}}%
\pgfusepath{clip}%
\pgfsetbuttcap%
\pgfsetroundjoin%
\definecolor{currentfill}{rgb}{0.900000,0.900000,0.900000}%
\pgfsetfillcolor{currentfill}%
\pgfsetlinewidth{1.003750pt}%
\definecolor{currentstroke}{rgb}{0.500000,0.500000,0.500000}%
\pgfsetstrokecolor{currentstroke}%
\pgfsetdash{}{0pt}%
\pgfsys@defobject{currentmarker}{\pgfqpoint{2.005500in}{3.960087in}}{\pgfqpoint{2.452054in}{6.161131in}}{%
\pgfpathmoveto{\pgfqpoint{2.005500in}{6.161131in}}%
\pgfpathlineto{\pgfqpoint{2.005500in}{3.960092in}}%
\pgfpathlineto{\pgfqpoint{2.014613in}{3.960092in}}%
\pgfpathlineto{\pgfqpoint{2.023727in}{3.960092in}}%
\pgfpathlineto{\pgfqpoint{2.032840in}{3.960091in}}%
\pgfpathlineto{\pgfqpoint{2.041953in}{3.960091in}}%
\pgfpathlineto{\pgfqpoint{2.051067in}{3.960091in}}%
\pgfpathlineto{\pgfqpoint{2.060180in}{3.960091in}}%
\pgfpathlineto{\pgfqpoint{2.069293in}{3.960091in}}%
\pgfpathlineto{\pgfqpoint{2.078407in}{3.960091in}}%
\pgfpathlineto{\pgfqpoint{2.087520in}{3.960091in}}%
\pgfpathlineto{\pgfqpoint{2.096634in}{3.960091in}}%
\pgfpathlineto{\pgfqpoint{2.105747in}{3.960091in}}%
\pgfpathlineto{\pgfqpoint{2.114860in}{3.960091in}}%
\pgfpathlineto{\pgfqpoint{2.123974in}{3.960091in}}%
\pgfpathlineto{\pgfqpoint{2.133087in}{3.960091in}}%
\pgfpathlineto{\pgfqpoint{2.142200in}{3.960090in}}%
\pgfpathlineto{\pgfqpoint{2.151314in}{3.960090in}}%
\pgfpathlineto{\pgfqpoint{2.160427in}{3.960090in}}%
\pgfpathlineto{\pgfqpoint{2.169540in}{3.960090in}}%
\pgfpathlineto{\pgfqpoint{2.178654in}{3.960090in}}%
\pgfpathlineto{\pgfqpoint{2.187767in}{3.960090in}}%
\pgfpathlineto{\pgfqpoint{2.196880in}{3.960090in}}%
\pgfpathlineto{\pgfqpoint{2.205994in}{3.960090in}}%
\pgfpathlineto{\pgfqpoint{2.215107in}{3.960090in}}%
\pgfpathlineto{\pgfqpoint{2.224221in}{3.960090in}}%
\pgfpathlineto{\pgfqpoint{2.233334in}{3.960089in}}%
\pgfpathlineto{\pgfqpoint{2.242447in}{3.960089in}}%
\pgfpathlineto{\pgfqpoint{2.251561in}{3.960089in}}%
\pgfpathlineto{\pgfqpoint{2.260674in}{3.960089in}}%
\pgfpathlineto{\pgfqpoint{2.269787in}{3.960089in}}%
\pgfpathlineto{\pgfqpoint{2.278901in}{3.960089in}}%
\pgfpathlineto{\pgfqpoint{2.288014in}{3.960089in}}%
\pgfpathlineto{\pgfqpoint{2.297127in}{3.960089in}}%
\pgfpathlineto{\pgfqpoint{2.306241in}{3.960089in}}%
\pgfpathlineto{\pgfqpoint{2.315354in}{3.960089in}}%
\pgfpathlineto{\pgfqpoint{2.324467in}{3.960088in}}%
\pgfpathlineto{\pgfqpoint{2.333581in}{3.960088in}}%
\pgfpathlineto{\pgfqpoint{2.342694in}{3.960088in}}%
\pgfpathlineto{\pgfqpoint{2.351807in}{3.960088in}}%
\pgfpathlineto{\pgfqpoint{2.360921in}{3.960088in}}%
\pgfpathlineto{\pgfqpoint{2.370034in}{3.960088in}}%
\pgfpathlineto{\pgfqpoint{2.379148in}{3.960088in}}%
\pgfpathlineto{\pgfqpoint{2.388261in}{3.960088in}}%
\pgfpathlineto{\pgfqpoint{2.397374in}{3.960088in}}%
\pgfpathlineto{\pgfqpoint{2.406488in}{3.960088in}}%
\pgfpathlineto{\pgfqpoint{2.415601in}{3.960087in}}%
\pgfpathlineto{\pgfqpoint{2.424714in}{3.960087in}}%
\pgfpathlineto{\pgfqpoint{2.433828in}{3.960087in}}%
\pgfpathlineto{\pgfqpoint{2.442941in}{3.960087in}}%
\pgfpathlineto{\pgfqpoint{2.452054in}{3.960087in}}%
\pgfpathlineto{\pgfqpoint{2.452054in}{6.161131in}}%
\pgfpathlineto{\pgfqpoint{2.452054in}{6.161131in}}%
\pgfpathlineto{\pgfqpoint{2.442941in}{6.161131in}}%
\pgfpathlineto{\pgfqpoint{2.433828in}{6.161131in}}%
\pgfpathlineto{\pgfqpoint{2.424714in}{6.161131in}}%
\pgfpathlineto{\pgfqpoint{2.415601in}{6.161131in}}%
\pgfpathlineto{\pgfqpoint{2.406488in}{6.161131in}}%
\pgfpathlineto{\pgfqpoint{2.397374in}{6.161131in}}%
\pgfpathlineto{\pgfqpoint{2.388261in}{6.161131in}}%
\pgfpathlineto{\pgfqpoint{2.379148in}{6.161131in}}%
\pgfpathlineto{\pgfqpoint{2.370034in}{6.161131in}}%
\pgfpathlineto{\pgfqpoint{2.360921in}{6.161131in}}%
\pgfpathlineto{\pgfqpoint{2.351807in}{6.161131in}}%
\pgfpathlineto{\pgfqpoint{2.342694in}{6.161131in}}%
\pgfpathlineto{\pgfqpoint{2.333581in}{6.161131in}}%
\pgfpathlineto{\pgfqpoint{2.324467in}{6.161131in}}%
\pgfpathlineto{\pgfqpoint{2.315354in}{6.161131in}}%
\pgfpathlineto{\pgfqpoint{2.306241in}{6.161131in}}%
\pgfpathlineto{\pgfqpoint{2.297127in}{6.161131in}}%
\pgfpathlineto{\pgfqpoint{2.288014in}{6.161131in}}%
\pgfpathlineto{\pgfqpoint{2.278901in}{6.161131in}}%
\pgfpathlineto{\pgfqpoint{2.269787in}{6.161131in}}%
\pgfpathlineto{\pgfqpoint{2.260674in}{6.161131in}}%
\pgfpathlineto{\pgfqpoint{2.251561in}{6.161131in}}%
\pgfpathlineto{\pgfqpoint{2.242447in}{6.161131in}}%
\pgfpathlineto{\pgfqpoint{2.233334in}{6.161131in}}%
\pgfpathlineto{\pgfqpoint{2.224221in}{6.161131in}}%
\pgfpathlineto{\pgfqpoint{2.215107in}{6.161131in}}%
\pgfpathlineto{\pgfqpoint{2.205994in}{6.161131in}}%
\pgfpathlineto{\pgfqpoint{2.196880in}{6.161131in}}%
\pgfpathlineto{\pgfqpoint{2.187767in}{6.161131in}}%
\pgfpathlineto{\pgfqpoint{2.178654in}{6.161131in}}%
\pgfpathlineto{\pgfqpoint{2.169540in}{6.161131in}}%
\pgfpathlineto{\pgfqpoint{2.160427in}{6.161131in}}%
\pgfpathlineto{\pgfqpoint{2.151314in}{6.161131in}}%
\pgfpathlineto{\pgfqpoint{2.142200in}{6.161131in}}%
\pgfpathlineto{\pgfqpoint{2.133087in}{6.161131in}}%
\pgfpathlineto{\pgfqpoint{2.123974in}{6.161131in}}%
\pgfpathlineto{\pgfqpoint{2.114860in}{6.161131in}}%
\pgfpathlineto{\pgfqpoint{2.105747in}{6.161131in}}%
\pgfpathlineto{\pgfqpoint{2.096634in}{6.161131in}}%
\pgfpathlineto{\pgfqpoint{2.087520in}{6.161131in}}%
\pgfpathlineto{\pgfqpoint{2.078407in}{6.161131in}}%
\pgfpathlineto{\pgfqpoint{2.069293in}{6.161131in}}%
\pgfpathlineto{\pgfqpoint{2.060180in}{6.161131in}}%
\pgfpathlineto{\pgfqpoint{2.051067in}{6.161131in}}%
\pgfpathlineto{\pgfqpoint{2.041953in}{6.161131in}}%
\pgfpathlineto{\pgfqpoint{2.032840in}{6.161131in}}%
\pgfpathlineto{\pgfqpoint{2.023727in}{6.161131in}}%
\pgfpathlineto{\pgfqpoint{2.014613in}{6.161131in}}%
\pgfpathlineto{\pgfqpoint{2.005500in}{6.161131in}}%
\pgfpathlineto{\pgfqpoint{2.005500in}{6.161131in}}%
\pgfpathclose%
\pgfusepath{stroke,fill}%
}%
\begin{pgfscope}%
\pgfsys@transformshift{0.000000in}{0.000000in}%
\pgfsys@useobject{currentmarker}{}%
\end{pgfscope}%
\end{pgfscope}%
\begin{pgfscope}%
\pgfpathrectangle{\pgfqpoint{0.750000in}{3.960000in}}{\pgfqpoint{4.650000in}{3.080000in}}%
\pgfusepath{clip}%
\pgfsetbuttcap%
\pgfsetroundjoin%
\definecolor{currentfill}{rgb}{0.900000,0.900000,0.900000}%
\pgfsetfillcolor{currentfill}%
\pgfsetlinewidth{1.003750pt}%
\definecolor{currentstroke}{rgb}{0.500000,0.500000,0.500000}%
\pgfsetstrokecolor{currentstroke}%
\pgfsetdash{}{0pt}%
\pgfsys@defobject{currentmarker}{\pgfqpoint{2.452054in}{6.161131in}}{\pgfqpoint{4.144500in}{6.894840in}}{%
\pgfpathmoveto{\pgfqpoint{2.452054in}{6.161131in}}%
\pgfpathlineto{\pgfqpoint{2.452054in}{6.638730in}}%
\pgfpathlineto{\pgfqpoint{2.486594in}{6.665978in}}%
\pgfpathlineto{\pgfqpoint{2.521134in}{6.691753in}}%
\pgfpathlineto{\pgfqpoint{2.555674in}{6.716040in}}%
\pgfpathlineto{\pgfqpoint{2.590213in}{6.738827in}}%
\pgfpathlineto{\pgfqpoint{2.624753in}{6.760101in}}%
\pgfpathlineto{\pgfqpoint{2.659293in}{6.779849in}}%
\pgfpathlineto{\pgfqpoint{2.693832in}{6.798063in}}%
\pgfpathlineto{\pgfqpoint{2.728372in}{6.814731in}}%
\pgfpathlineto{\pgfqpoint{2.762912in}{6.829844in}}%
\pgfpathlineto{\pgfqpoint{2.797451in}{6.843395in}}%
\pgfpathlineto{\pgfqpoint{2.831991in}{6.855376in}}%
\pgfpathlineto{\pgfqpoint{2.866531in}{6.865780in}}%
\pgfpathlineto{\pgfqpoint{2.901071in}{6.874602in}}%
\pgfpathlineto{\pgfqpoint{2.935610in}{6.881837in}}%
\pgfpathlineto{\pgfqpoint{2.970150in}{6.887481in}}%
\pgfpathlineto{\pgfqpoint{3.004690in}{6.891531in}}%
\pgfpathlineto{\pgfqpoint{3.039229in}{6.893984in}}%
\pgfpathlineto{\pgfqpoint{3.073769in}{6.894840in}}%
\pgfpathlineto{\pgfqpoint{3.108309in}{6.894098in}}%
\pgfpathlineto{\pgfqpoint{3.142849in}{6.891758in}}%
\pgfpathlineto{\pgfqpoint{3.177388in}{6.887822in}}%
\pgfpathlineto{\pgfqpoint{3.211928in}{6.882292in}}%
\pgfpathlineto{\pgfqpoint{3.246468in}{6.875170in}}%
\pgfpathlineto{\pgfqpoint{3.281007in}{6.866461in}}%
\pgfpathlineto{\pgfqpoint{3.315547in}{6.856170in}}%
\pgfpathlineto{\pgfqpoint{3.350087in}{6.844301in}}%
\pgfpathlineto{\pgfqpoint{3.384626in}{6.830862in}}%
\pgfpathlineto{\pgfqpoint{3.419166in}{6.815859in}}%
\pgfpathlineto{\pgfqpoint{3.453706in}{6.799302in}}%
\pgfpathlineto{\pgfqpoint{3.488246in}{6.781198in}}%
\pgfpathlineto{\pgfqpoint{3.522785in}{6.761559in}}%
\pgfpathlineto{\pgfqpoint{3.557325in}{6.740393in}}%
\pgfpathlineto{\pgfqpoint{3.591865in}{6.717714in}}%
\pgfpathlineto{\pgfqpoint{3.626404in}{6.693534in}}%
\pgfpathlineto{\pgfqpoint{3.660944in}{6.667864in}}%
\pgfpathlineto{\pgfqpoint{3.695484in}{6.640721in}}%
\pgfpathlineto{\pgfqpoint{3.730024in}{6.612117in}}%
\pgfpathlineto{\pgfqpoint{3.764563in}{6.582070in}}%
\pgfpathlineto{\pgfqpoint{3.799103in}{6.550594in}}%
\pgfpathlineto{\pgfqpoint{3.833643in}{6.517708in}}%
\pgfpathlineto{\pgfqpoint{3.868182in}{6.483430in}}%
\pgfpathlineto{\pgfqpoint{3.902722in}{6.447777in}}%
\pgfpathlineto{\pgfqpoint{3.937262in}{6.410770in}}%
\pgfpathlineto{\pgfqpoint{3.971801in}{6.372428in}}%
\pgfpathlineto{\pgfqpoint{4.006341in}{6.332772in}}%
\pgfpathlineto{\pgfqpoint{4.040881in}{6.291825in}}%
\pgfpathlineto{\pgfqpoint{4.075421in}{6.249608in}}%
\pgfpathlineto{\pgfqpoint{4.109960in}{6.206144in}}%
\pgfpathlineto{\pgfqpoint{4.144500in}{6.161457in}}%
\pgfpathlineto{\pgfqpoint{4.144500in}{6.161131in}}%
\pgfpathlineto{\pgfqpoint{4.144500in}{6.161131in}}%
\pgfpathlineto{\pgfqpoint{4.109960in}{6.161131in}}%
\pgfpathlineto{\pgfqpoint{4.075421in}{6.161131in}}%
\pgfpathlineto{\pgfqpoint{4.040881in}{6.161131in}}%
\pgfpathlineto{\pgfqpoint{4.006341in}{6.161131in}}%
\pgfpathlineto{\pgfqpoint{3.971801in}{6.161131in}}%
\pgfpathlineto{\pgfqpoint{3.937262in}{6.161131in}}%
\pgfpathlineto{\pgfqpoint{3.902722in}{6.161131in}}%
\pgfpathlineto{\pgfqpoint{3.868182in}{6.161131in}}%
\pgfpathlineto{\pgfqpoint{3.833643in}{6.161131in}}%
\pgfpathlineto{\pgfqpoint{3.799103in}{6.161131in}}%
\pgfpathlineto{\pgfqpoint{3.764563in}{6.161131in}}%
\pgfpathlineto{\pgfqpoint{3.730024in}{6.161131in}}%
\pgfpathlineto{\pgfqpoint{3.695484in}{6.161131in}}%
\pgfpathlineto{\pgfqpoint{3.660944in}{6.161131in}}%
\pgfpathlineto{\pgfqpoint{3.626404in}{6.161131in}}%
\pgfpathlineto{\pgfqpoint{3.591865in}{6.161131in}}%
\pgfpathlineto{\pgfqpoint{3.557325in}{6.161131in}}%
\pgfpathlineto{\pgfqpoint{3.522785in}{6.161131in}}%
\pgfpathlineto{\pgfqpoint{3.488246in}{6.161131in}}%
\pgfpathlineto{\pgfqpoint{3.453706in}{6.161131in}}%
\pgfpathlineto{\pgfqpoint{3.419166in}{6.161131in}}%
\pgfpathlineto{\pgfqpoint{3.384626in}{6.161131in}}%
\pgfpathlineto{\pgfqpoint{3.350087in}{6.161131in}}%
\pgfpathlineto{\pgfqpoint{3.315547in}{6.161131in}}%
\pgfpathlineto{\pgfqpoint{3.281007in}{6.161131in}}%
\pgfpathlineto{\pgfqpoint{3.246468in}{6.161131in}}%
\pgfpathlineto{\pgfqpoint{3.211928in}{6.161131in}}%
\pgfpathlineto{\pgfqpoint{3.177388in}{6.161131in}}%
\pgfpathlineto{\pgfqpoint{3.142849in}{6.161131in}}%
\pgfpathlineto{\pgfqpoint{3.108309in}{6.161131in}}%
\pgfpathlineto{\pgfqpoint{3.073769in}{6.161131in}}%
\pgfpathlineto{\pgfqpoint{3.039229in}{6.161131in}}%
\pgfpathlineto{\pgfqpoint{3.004690in}{6.161131in}}%
\pgfpathlineto{\pgfqpoint{2.970150in}{6.161131in}}%
\pgfpathlineto{\pgfqpoint{2.935610in}{6.161131in}}%
\pgfpathlineto{\pgfqpoint{2.901071in}{6.161131in}}%
\pgfpathlineto{\pgfqpoint{2.866531in}{6.161131in}}%
\pgfpathlineto{\pgfqpoint{2.831991in}{6.161131in}}%
\pgfpathlineto{\pgfqpoint{2.797451in}{6.161131in}}%
\pgfpathlineto{\pgfqpoint{2.762912in}{6.161131in}}%
\pgfpathlineto{\pgfqpoint{2.728372in}{6.161131in}}%
\pgfpathlineto{\pgfqpoint{2.693832in}{6.161131in}}%
\pgfpathlineto{\pgfqpoint{2.659293in}{6.161131in}}%
\pgfpathlineto{\pgfqpoint{2.624753in}{6.161131in}}%
\pgfpathlineto{\pgfqpoint{2.590213in}{6.161131in}}%
\pgfpathlineto{\pgfqpoint{2.555674in}{6.161131in}}%
\pgfpathlineto{\pgfqpoint{2.521134in}{6.161131in}}%
\pgfpathlineto{\pgfqpoint{2.486594in}{6.161131in}}%
\pgfpathlineto{\pgfqpoint{2.452054in}{6.161131in}}%
\pgfpathlineto{\pgfqpoint{2.452054in}{6.161131in}}%
\pgfpathclose%
\pgfusepath{stroke,fill}%
}%
\begin{pgfscope}%
\pgfsys@transformshift{0.000000in}{0.000000in}%
\pgfsys@useobject{currentmarker}{}%
\end{pgfscope}%
\end{pgfscope}%
\begin{pgfscope}%
\pgfpathrectangle{\pgfqpoint{0.750000in}{3.960000in}}{\pgfqpoint{4.650000in}{3.080000in}}%
\pgfusepath{clip}%
\pgfsetrectcap%
\pgfsetroundjoin%
\pgfsetlinewidth{0.803000pt}%
\definecolor{currentstroke}{rgb}{0.690196,0.690196,0.690196}%
\pgfsetstrokecolor{currentstroke}%
\pgfsetdash{}{0pt}%
\pgfpathmoveto{\pgfqpoint{0.750000in}{3.960000in}}%
\pgfpathlineto{\pgfqpoint{0.750000in}{7.040000in}}%
\pgfusepath{stroke}%
\end{pgfscope}%
\begin{pgfscope}%
\pgfsetbuttcap%
\pgfsetroundjoin%
\definecolor{currentfill}{rgb}{0.000000,0.000000,0.000000}%
\pgfsetfillcolor{currentfill}%
\pgfsetlinewidth{0.803000pt}%
\definecolor{currentstroke}{rgb}{0.000000,0.000000,0.000000}%
\pgfsetstrokecolor{currentstroke}%
\pgfsetdash{}{0pt}%
\pgfsys@defobject{currentmarker}{\pgfqpoint{0.000000in}{-0.048611in}}{\pgfqpoint{0.000000in}{0.000000in}}{%
\pgfpathmoveto{\pgfqpoint{0.000000in}{0.000000in}}%
\pgfpathlineto{\pgfqpoint{0.000000in}{-0.048611in}}%
\pgfusepath{stroke,fill}%
}%
\begin{pgfscope}%
\pgfsys@transformshift{0.750000in}{3.960000in}%
\pgfsys@useobject{currentmarker}{}%
\end{pgfscope}%
\end{pgfscope}%
\begin{pgfscope}%
\pgfpathrectangle{\pgfqpoint{0.750000in}{3.960000in}}{\pgfqpoint{4.650000in}{3.080000in}}%
\pgfusepath{clip}%
\pgfsetrectcap%
\pgfsetroundjoin%
\pgfsetlinewidth{0.803000pt}%
\definecolor{currentstroke}{rgb}{0.690196,0.690196,0.690196}%
\pgfsetstrokecolor{currentstroke}%
\pgfsetdash{}{0pt}%
\pgfpathmoveto{\pgfqpoint{1.266667in}{3.960000in}}%
\pgfpathlineto{\pgfqpoint{1.266667in}{7.040000in}}%
\pgfusepath{stroke}%
\end{pgfscope}%
\begin{pgfscope}%
\pgfsetbuttcap%
\pgfsetroundjoin%
\definecolor{currentfill}{rgb}{0.000000,0.000000,0.000000}%
\pgfsetfillcolor{currentfill}%
\pgfsetlinewidth{0.803000pt}%
\definecolor{currentstroke}{rgb}{0.000000,0.000000,0.000000}%
\pgfsetstrokecolor{currentstroke}%
\pgfsetdash{}{0pt}%
\pgfsys@defobject{currentmarker}{\pgfqpoint{0.000000in}{-0.048611in}}{\pgfqpoint{0.000000in}{0.000000in}}{%
\pgfpathmoveto{\pgfqpoint{0.000000in}{0.000000in}}%
\pgfpathlineto{\pgfqpoint{0.000000in}{-0.048611in}}%
\pgfusepath{stroke,fill}%
}%
\begin{pgfscope}%
\pgfsys@transformshift{1.266667in}{3.960000in}%
\pgfsys@useobject{currentmarker}{}%
\end{pgfscope}%
\end{pgfscope}%
\begin{pgfscope}%
\pgfpathrectangle{\pgfqpoint{0.750000in}{3.960000in}}{\pgfqpoint{4.650000in}{3.080000in}}%
\pgfusepath{clip}%
\pgfsetrectcap%
\pgfsetroundjoin%
\pgfsetlinewidth{0.803000pt}%
\definecolor{currentstroke}{rgb}{0.690196,0.690196,0.690196}%
\pgfsetstrokecolor{currentstroke}%
\pgfsetdash{}{0pt}%
\pgfpathmoveto{\pgfqpoint{1.783333in}{3.960000in}}%
\pgfpathlineto{\pgfqpoint{1.783333in}{7.040000in}}%
\pgfusepath{stroke}%
\end{pgfscope}%
\begin{pgfscope}%
\pgfsetbuttcap%
\pgfsetroundjoin%
\definecolor{currentfill}{rgb}{0.000000,0.000000,0.000000}%
\pgfsetfillcolor{currentfill}%
\pgfsetlinewidth{0.803000pt}%
\definecolor{currentstroke}{rgb}{0.000000,0.000000,0.000000}%
\pgfsetstrokecolor{currentstroke}%
\pgfsetdash{}{0pt}%
\pgfsys@defobject{currentmarker}{\pgfqpoint{0.000000in}{-0.048611in}}{\pgfqpoint{0.000000in}{0.000000in}}{%
\pgfpathmoveto{\pgfqpoint{0.000000in}{0.000000in}}%
\pgfpathlineto{\pgfqpoint{0.000000in}{-0.048611in}}%
\pgfusepath{stroke,fill}%
}%
\begin{pgfscope}%
\pgfsys@transformshift{1.783333in}{3.960000in}%
\pgfsys@useobject{currentmarker}{}%
\end{pgfscope}%
\end{pgfscope}%
\begin{pgfscope}%
\pgfpathrectangle{\pgfqpoint{0.750000in}{3.960000in}}{\pgfqpoint{4.650000in}{3.080000in}}%
\pgfusepath{clip}%
\pgfsetrectcap%
\pgfsetroundjoin%
\pgfsetlinewidth{0.803000pt}%
\definecolor{currentstroke}{rgb}{0.690196,0.690196,0.690196}%
\pgfsetstrokecolor{currentstroke}%
\pgfsetdash{}{0pt}%
\pgfpathmoveto{\pgfqpoint{2.300000in}{3.960000in}}%
\pgfpathlineto{\pgfqpoint{2.300000in}{7.040000in}}%
\pgfusepath{stroke}%
\end{pgfscope}%
\begin{pgfscope}%
\pgfsetbuttcap%
\pgfsetroundjoin%
\definecolor{currentfill}{rgb}{0.000000,0.000000,0.000000}%
\pgfsetfillcolor{currentfill}%
\pgfsetlinewidth{0.803000pt}%
\definecolor{currentstroke}{rgb}{0.000000,0.000000,0.000000}%
\pgfsetstrokecolor{currentstroke}%
\pgfsetdash{}{0pt}%
\pgfsys@defobject{currentmarker}{\pgfqpoint{0.000000in}{-0.048611in}}{\pgfqpoint{0.000000in}{0.000000in}}{%
\pgfpathmoveto{\pgfqpoint{0.000000in}{0.000000in}}%
\pgfpathlineto{\pgfqpoint{0.000000in}{-0.048611in}}%
\pgfusepath{stroke,fill}%
}%
\begin{pgfscope}%
\pgfsys@transformshift{2.300000in}{3.960000in}%
\pgfsys@useobject{currentmarker}{}%
\end{pgfscope}%
\end{pgfscope}%
\begin{pgfscope}%
\pgfpathrectangle{\pgfqpoint{0.750000in}{3.960000in}}{\pgfqpoint{4.650000in}{3.080000in}}%
\pgfusepath{clip}%
\pgfsetrectcap%
\pgfsetroundjoin%
\pgfsetlinewidth{0.803000pt}%
\definecolor{currentstroke}{rgb}{0.690196,0.690196,0.690196}%
\pgfsetstrokecolor{currentstroke}%
\pgfsetdash{}{0pt}%
\pgfpathmoveto{\pgfqpoint{2.816667in}{3.960000in}}%
\pgfpathlineto{\pgfqpoint{2.816667in}{7.040000in}}%
\pgfusepath{stroke}%
\end{pgfscope}%
\begin{pgfscope}%
\pgfsetbuttcap%
\pgfsetroundjoin%
\definecolor{currentfill}{rgb}{0.000000,0.000000,0.000000}%
\pgfsetfillcolor{currentfill}%
\pgfsetlinewidth{0.803000pt}%
\definecolor{currentstroke}{rgb}{0.000000,0.000000,0.000000}%
\pgfsetstrokecolor{currentstroke}%
\pgfsetdash{}{0pt}%
\pgfsys@defobject{currentmarker}{\pgfqpoint{0.000000in}{-0.048611in}}{\pgfqpoint{0.000000in}{0.000000in}}{%
\pgfpathmoveto{\pgfqpoint{0.000000in}{0.000000in}}%
\pgfpathlineto{\pgfqpoint{0.000000in}{-0.048611in}}%
\pgfusepath{stroke,fill}%
}%
\begin{pgfscope}%
\pgfsys@transformshift{2.816667in}{3.960000in}%
\pgfsys@useobject{currentmarker}{}%
\end{pgfscope}%
\end{pgfscope}%
\begin{pgfscope}%
\pgfpathrectangle{\pgfqpoint{0.750000in}{3.960000in}}{\pgfqpoint{4.650000in}{3.080000in}}%
\pgfusepath{clip}%
\pgfsetrectcap%
\pgfsetroundjoin%
\pgfsetlinewidth{0.803000pt}%
\definecolor{currentstroke}{rgb}{0.690196,0.690196,0.690196}%
\pgfsetstrokecolor{currentstroke}%
\pgfsetdash{}{0pt}%
\pgfpathmoveto{\pgfqpoint{3.333333in}{3.960000in}}%
\pgfpathlineto{\pgfqpoint{3.333333in}{7.040000in}}%
\pgfusepath{stroke}%
\end{pgfscope}%
\begin{pgfscope}%
\pgfsetbuttcap%
\pgfsetroundjoin%
\definecolor{currentfill}{rgb}{0.000000,0.000000,0.000000}%
\pgfsetfillcolor{currentfill}%
\pgfsetlinewidth{0.803000pt}%
\definecolor{currentstroke}{rgb}{0.000000,0.000000,0.000000}%
\pgfsetstrokecolor{currentstroke}%
\pgfsetdash{}{0pt}%
\pgfsys@defobject{currentmarker}{\pgfqpoint{0.000000in}{-0.048611in}}{\pgfqpoint{0.000000in}{0.000000in}}{%
\pgfpathmoveto{\pgfqpoint{0.000000in}{0.000000in}}%
\pgfpathlineto{\pgfqpoint{0.000000in}{-0.048611in}}%
\pgfusepath{stroke,fill}%
}%
\begin{pgfscope}%
\pgfsys@transformshift{3.333333in}{3.960000in}%
\pgfsys@useobject{currentmarker}{}%
\end{pgfscope}%
\end{pgfscope}%
\begin{pgfscope}%
\pgfpathrectangle{\pgfqpoint{0.750000in}{3.960000in}}{\pgfqpoint{4.650000in}{3.080000in}}%
\pgfusepath{clip}%
\pgfsetrectcap%
\pgfsetroundjoin%
\pgfsetlinewidth{0.803000pt}%
\definecolor{currentstroke}{rgb}{0.690196,0.690196,0.690196}%
\pgfsetstrokecolor{currentstroke}%
\pgfsetdash{}{0pt}%
\pgfpathmoveto{\pgfqpoint{3.850000in}{3.960000in}}%
\pgfpathlineto{\pgfqpoint{3.850000in}{7.040000in}}%
\pgfusepath{stroke}%
\end{pgfscope}%
\begin{pgfscope}%
\pgfsetbuttcap%
\pgfsetroundjoin%
\definecolor{currentfill}{rgb}{0.000000,0.000000,0.000000}%
\pgfsetfillcolor{currentfill}%
\pgfsetlinewidth{0.803000pt}%
\definecolor{currentstroke}{rgb}{0.000000,0.000000,0.000000}%
\pgfsetstrokecolor{currentstroke}%
\pgfsetdash{}{0pt}%
\pgfsys@defobject{currentmarker}{\pgfqpoint{0.000000in}{-0.048611in}}{\pgfqpoint{0.000000in}{0.000000in}}{%
\pgfpathmoveto{\pgfqpoint{0.000000in}{0.000000in}}%
\pgfpathlineto{\pgfqpoint{0.000000in}{-0.048611in}}%
\pgfusepath{stroke,fill}%
}%
\begin{pgfscope}%
\pgfsys@transformshift{3.850000in}{3.960000in}%
\pgfsys@useobject{currentmarker}{}%
\end{pgfscope}%
\end{pgfscope}%
\begin{pgfscope}%
\pgfpathrectangle{\pgfqpoint{0.750000in}{3.960000in}}{\pgfqpoint{4.650000in}{3.080000in}}%
\pgfusepath{clip}%
\pgfsetrectcap%
\pgfsetroundjoin%
\pgfsetlinewidth{0.803000pt}%
\definecolor{currentstroke}{rgb}{0.690196,0.690196,0.690196}%
\pgfsetstrokecolor{currentstroke}%
\pgfsetdash{}{0pt}%
\pgfpathmoveto{\pgfqpoint{4.366667in}{3.960000in}}%
\pgfpathlineto{\pgfqpoint{4.366667in}{7.040000in}}%
\pgfusepath{stroke}%
\end{pgfscope}%
\begin{pgfscope}%
\pgfsetbuttcap%
\pgfsetroundjoin%
\definecolor{currentfill}{rgb}{0.000000,0.000000,0.000000}%
\pgfsetfillcolor{currentfill}%
\pgfsetlinewidth{0.803000pt}%
\definecolor{currentstroke}{rgb}{0.000000,0.000000,0.000000}%
\pgfsetstrokecolor{currentstroke}%
\pgfsetdash{}{0pt}%
\pgfsys@defobject{currentmarker}{\pgfqpoint{0.000000in}{-0.048611in}}{\pgfqpoint{0.000000in}{0.000000in}}{%
\pgfpathmoveto{\pgfqpoint{0.000000in}{0.000000in}}%
\pgfpathlineto{\pgfqpoint{0.000000in}{-0.048611in}}%
\pgfusepath{stroke,fill}%
}%
\begin{pgfscope}%
\pgfsys@transformshift{4.366667in}{3.960000in}%
\pgfsys@useobject{currentmarker}{}%
\end{pgfscope}%
\end{pgfscope}%
\begin{pgfscope}%
\pgfpathrectangle{\pgfqpoint{0.750000in}{3.960000in}}{\pgfqpoint{4.650000in}{3.080000in}}%
\pgfusepath{clip}%
\pgfsetrectcap%
\pgfsetroundjoin%
\pgfsetlinewidth{0.803000pt}%
\definecolor{currentstroke}{rgb}{0.690196,0.690196,0.690196}%
\pgfsetstrokecolor{currentstroke}%
\pgfsetdash{}{0pt}%
\pgfpathmoveto{\pgfqpoint{4.883333in}{3.960000in}}%
\pgfpathlineto{\pgfqpoint{4.883333in}{7.040000in}}%
\pgfusepath{stroke}%
\end{pgfscope}%
\begin{pgfscope}%
\pgfsetbuttcap%
\pgfsetroundjoin%
\definecolor{currentfill}{rgb}{0.000000,0.000000,0.000000}%
\pgfsetfillcolor{currentfill}%
\pgfsetlinewidth{0.803000pt}%
\definecolor{currentstroke}{rgb}{0.000000,0.000000,0.000000}%
\pgfsetstrokecolor{currentstroke}%
\pgfsetdash{}{0pt}%
\pgfsys@defobject{currentmarker}{\pgfqpoint{0.000000in}{-0.048611in}}{\pgfqpoint{0.000000in}{0.000000in}}{%
\pgfpathmoveto{\pgfqpoint{0.000000in}{0.000000in}}%
\pgfpathlineto{\pgfqpoint{0.000000in}{-0.048611in}}%
\pgfusepath{stroke,fill}%
}%
\begin{pgfscope}%
\pgfsys@transformshift{4.883333in}{3.960000in}%
\pgfsys@useobject{currentmarker}{}%
\end{pgfscope}%
\end{pgfscope}%
\begin{pgfscope}%
\pgfpathrectangle{\pgfqpoint{0.750000in}{3.960000in}}{\pgfqpoint{4.650000in}{3.080000in}}%
\pgfusepath{clip}%
\pgfsetrectcap%
\pgfsetroundjoin%
\pgfsetlinewidth{0.803000pt}%
\definecolor{currentstroke}{rgb}{0.690196,0.690196,0.690196}%
\pgfsetstrokecolor{currentstroke}%
\pgfsetdash{}{0pt}%
\pgfpathmoveto{\pgfqpoint{5.400000in}{3.960000in}}%
\pgfpathlineto{\pgfqpoint{5.400000in}{7.040000in}}%
\pgfusepath{stroke}%
\end{pgfscope}%
\begin{pgfscope}%
\pgfsetbuttcap%
\pgfsetroundjoin%
\definecolor{currentfill}{rgb}{0.000000,0.000000,0.000000}%
\pgfsetfillcolor{currentfill}%
\pgfsetlinewidth{0.803000pt}%
\definecolor{currentstroke}{rgb}{0.000000,0.000000,0.000000}%
\pgfsetstrokecolor{currentstroke}%
\pgfsetdash{}{0pt}%
\pgfsys@defobject{currentmarker}{\pgfqpoint{0.000000in}{-0.048611in}}{\pgfqpoint{0.000000in}{0.000000in}}{%
\pgfpathmoveto{\pgfqpoint{0.000000in}{0.000000in}}%
\pgfpathlineto{\pgfqpoint{0.000000in}{-0.048611in}}%
\pgfusepath{stroke,fill}%
}%
\begin{pgfscope}%
\pgfsys@transformshift{5.400000in}{3.960000in}%
\pgfsys@useobject{currentmarker}{}%
\end{pgfscope}%
\end{pgfscope}%
\begin{pgfscope}%
\pgfpathrectangle{\pgfqpoint{0.750000in}{3.960000in}}{\pgfqpoint{4.650000in}{3.080000in}}%
\pgfusepath{clip}%
\pgfsetrectcap%
\pgfsetroundjoin%
\pgfsetlinewidth{0.803000pt}%
\definecolor{currentstroke}{rgb}{0.690196,0.690196,0.690196}%
\pgfsetstrokecolor{currentstroke}%
\pgfsetdash{}{0pt}%
\pgfpathmoveto{\pgfqpoint{0.750000in}{3.960000in}}%
\pgfpathlineto{\pgfqpoint{5.400000in}{3.960000in}}%
\pgfusepath{stroke}%
\end{pgfscope}%
\begin{pgfscope}%
\pgfsetbuttcap%
\pgfsetroundjoin%
\definecolor{currentfill}{rgb}{0.000000,0.000000,0.000000}%
\pgfsetfillcolor{currentfill}%
\pgfsetlinewidth{0.803000pt}%
\definecolor{currentstroke}{rgb}{0.000000,0.000000,0.000000}%
\pgfsetstrokecolor{currentstroke}%
\pgfsetdash{}{0pt}%
\pgfsys@defobject{currentmarker}{\pgfqpoint{-0.048611in}{0.000000in}}{\pgfqpoint{-0.000000in}{0.000000in}}{%
\pgfpathmoveto{\pgfqpoint{-0.000000in}{0.000000in}}%
\pgfpathlineto{\pgfqpoint{-0.048611in}{0.000000in}}%
\pgfusepath{stroke,fill}%
}%
\begin{pgfscope}%
\pgfsys@transformshift{0.750000in}{3.960000in}%
\pgfsys@useobject{currentmarker}{}%
\end{pgfscope}%
\end{pgfscope}%
\begin{pgfscope}%
\definecolor{textcolor}{rgb}{0.000000,0.000000,0.000000}%
\pgfsetstrokecolor{textcolor}%
\pgfsetfillcolor{textcolor}%
\pgftext[x=0.475308in, y=3.908900in, left, base]{\color{textcolor}\rmfamily\fontsize{10.000000}{12.000000}\selectfont \(\displaystyle {0.0}\)}%
\end{pgfscope}%
\begin{pgfscope}%
\pgfpathrectangle{\pgfqpoint{0.750000in}{3.960000in}}{\pgfqpoint{4.650000in}{3.080000in}}%
\pgfusepath{clip}%
\pgfsetrectcap%
\pgfsetroundjoin%
\pgfsetlinewidth{0.803000pt}%
\definecolor{currentstroke}{rgb}{0.690196,0.690196,0.690196}%
\pgfsetstrokecolor{currentstroke}%
\pgfsetdash{}{0pt}%
\pgfpathmoveto{\pgfqpoint{0.750000in}{4.449140in}}%
\pgfpathlineto{\pgfqpoint{5.400000in}{4.449140in}}%
\pgfusepath{stroke}%
\end{pgfscope}%
\begin{pgfscope}%
\pgfsetbuttcap%
\pgfsetroundjoin%
\definecolor{currentfill}{rgb}{0.000000,0.000000,0.000000}%
\pgfsetfillcolor{currentfill}%
\pgfsetlinewidth{0.803000pt}%
\definecolor{currentstroke}{rgb}{0.000000,0.000000,0.000000}%
\pgfsetstrokecolor{currentstroke}%
\pgfsetdash{}{0pt}%
\pgfsys@defobject{currentmarker}{\pgfqpoint{-0.048611in}{0.000000in}}{\pgfqpoint{-0.000000in}{0.000000in}}{%
\pgfpathmoveto{\pgfqpoint{-0.000000in}{0.000000in}}%
\pgfpathlineto{\pgfqpoint{-0.048611in}{0.000000in}}%
\pgfusepath{stroke,fill}%
}%
\begin{pgfscope}%
\pgfsys@transformshift{0.750000in}{4.449140in}%
\pgfsys@useobject{currentmarker}{}%
\end{pgfscope}%
\end{pgfscope}%
\begin{pgfscope}%
\definecolor{textcolor}{rgb}{0.000000,0.000000,0.000000}%
\pgfsetstrokecolor{textcolor}%
\pgfsetfillcolor{textcolor}%
\pgftext[x=0.475308in, y=4.398040in, left, base]{\color{textcolor}\rmfamily\fontsize{10.000000}{12.000000}\selectfont \(\displaystyle {0.2}\)}%
\end{pgfscope}%
\begin{pgfscope}%
\pgfpathrectangle{\pgfqpoint{0.750000in}{3.960000in}}{\pgfqpoint{4.650000in}{3.080000in}}%
\pgfusepath{clip}%
\pgfsetrectcap%
\pgfsetroundjoin%
\pgfsetlinewidth{0.803000pt}%
\definecolor{currentstroke}{rgb}{0.690196,0.690196,0.690196}%
\pgfsetstrokecolor{currentstroke}%
\pgfsetdash{}{0pt}%
\pgfpathmoveto{\pgfqpoint{0.750000in}{4.938280in}}%
\pgfpathlineto{\pgfqpoint{5.400000in}{4.938280in}}%
\pgfusepath{stroke}%
\end{pgfscope}%
\begin{pgfscope}%
\pgfsetbuttcap%
\pgfsetroundjoin%
\definecolor{currentfill}{rgb}{0.000000,0.000000,0.000000}%
\pgfsetfillcolor{currentfill}%
\pgfsetlinewidth{0.803000pt}%
\definecolor{currentstroke}{rgb}{0.000000,0.000000,0.000000}%
\pgfsetstrokecolor{currentstroke}%
\pgfsetdash{}{0pt}%
\pgfsys@defobject{currentmarker}{\pgfqpoint{-0.048611in}{0.000000in}}{\pgfqpoint{-0.000000in}{0.000000in}}{%
\pgfpathmoveto{\pgfqpoint{-0.000000in}{0.000000in}}%
\pgfpathlineto{\pgfqpoint{-0.048611in}{0.000000in}}%
\pgfusepath{stroke,fill}%
}%
\begin{pgfscope}%
\pgfsys@transformshift{0.750000in}{4.938280in}%
\pgfsys@useobject{currentmarker}{}%
\end{pgfscope}%
\end{pgfscope}%
\begin{pgfscope}%
\definecolor{textcolor}{rgb}{0.000000,0.000000,0.000000}%
\pgfsetstrokecolor{textcolor}%
\pgfsetfillcolor{textcolor}%
\pgftext[x=0.475308in, y=4.887180in, left, base]{\color{textcolor}\rmfamily\fontsize{10.000000}{12.000000}\selectfont \(\displaystyle {0.4}\)}%
\end{pgfscope}%
\begin{pgfscope}%
\pgfpathrectangle{\pgfqpoint{0.750000in}{3.960000in}}{\pgfqpoint{4.650000in}{3.080000in}}%
\pgfusepath{clip}%
\pgfsetrectcap%
\pgfsetroundjoin%
\pgfsetlinewidth{0.803000pt}%
\definecolor{currentstroke}{rgb}{0.690196,0.690196,0.690196}%
\pgfsetstrokecolor{currentstroke}%
\pgfsetdash{}{0pt}%
\pgfpathmoveto{\pgfqpoint{0.750000in}{5.427421in}}%
\pgfpathlineto{\pgfqpoint{5.400000in}{5.427421in}}%
\pgfusepath{stroke}%
\end{pgfscope}%
\begin{pgfscope}%
\pgfsetbuttcap%
\pgfsetroundjoin%
\definecolor{currentfill}{rgb}{0.000000,0.000000,0.000000}%
\pgfsetfillcolor{currentfill}%
\pgfsetlinewidth{0.803000pt}%
\definecolor{currentstroke}{rgb}{0.000000,0.000000,0.000000}%
\pgfsetstrokecolor{currentstroke}%
\pgfsetdash{}{0pt}%
\pgfsys@defobject{currentmarker}{\pgfqpoint{-0.048611in}{0.000000in}}{\pgfqpoint{-0.000000in}{0.000000in}}{%
\pgfpathmoveto{\pgfqpoint{-0.000000in}{0.000000in}}%
\pgfpathlineto{\pgfqpoint{-0.048611in}{0.000000in}}%
\pgfusepath{stroke,fill}%
}%
\begin{pgfscope}%
\pgfsys@transformshift{0.750000in}{5.427421in}%
\pgfsys@useobject{currentmarker}{}%
\end{pgfscope}%
\end{pgfscope}%
\begin{pgfscope}%
\definecolor{textcolor}{rgb}{0.000000,0.000000,0.000000}%
\pgfsetstrokecolor{textcolor}%
\pgfsetfillcolor{textcolor}%
\pgftext[x=0.475308in, y=5.376321in, left, base]{\color{textcolor}\rmfamily\fontsize{10.000000}{12.000000}\selectfont \(\displaystyle {0.6}\)}%
\end{pgfscope}%
\begin{pgfscope}%
\pgfpathrectangle{\pgfqpoint{0.750000in}{3.960000in}}{\pgfqpoint{4.650000in}{3.080000in}}%
\pgfusepath{clip}%
\pgfsetrectcap%
\pgfsetroundjoin%
\pgfsetlinewidth{0.803000pt}%
\definecolor{currentstroke}{rgb}{0.690196,0.690196,0.690196}%
\pgfsetstrokecolor{currentstroke}%
\pgfsetdash{}{0pt}%
\pgfpathmoveto{\pgfqpoint{0.750000in}{5.916561in}}%
\pgfpathlineto{\pgfqpoint{5.400000in}{5.916561in}}%
\pgfusepath{stroke}%
\end{pgfscope}%
\begin{pgfscope}%
\pgfsetbuttcap%
\pgfsetroundjoin%
\definecolor{currentfill}{rgb}{0.000000,0.000000,0.000000}%
\pgfsetfillcolor{currentfill}%
\pgfsetlinewidth{0.803000pt}%
\definecolor{currentstroke}{rgb}{0.000000,0.000000,0.000000}%
\pgfsetstrokecolor{currentstroke}%
\pgfsetdash{}{0pt}%
\pgfsys@defobject{currentmarker}{\pgfqpoint{-0.048611in}{0.000000in}}{\pgfqpoint{-0.000000in}{0.000000in}}{%
\pgfpathmoveto{\pgfqpoint{-0.000000in}{0.000000in}}%
\pgfpathlineto{\pgfqpoint{-0.048611in}{0.000000in}}%
\pgfusepath{stroke,fill}%
}%
\begin{pgfscope}%
\pgfsys@transformshift{0.750000in}{5.916561in}%
\pgfsys@useobject{currentmarker}{}%
\end{pgfscope}%
\end{pgfscope}%
\begin{pgfscope}%
\definecolor{textcolor}{rgb}{0.000000,0.000000,0.000000}%
\pgfsetstrokecolor{textcolor}%
\pgfsetfillcolor{textcolor}%
\pgftext[x=0.475308in, y=5.865461in, left, base]{\color{textcolor}\rmfamily\fontsize{10.000000}{12.000000}\selectfont \(\displaystyle {0.8}\)}%
\end{pgfscope}%
\begin{pgfscope}%
\pgfpathrectangle{\pgfqpoint{0.750000in}{3.960000in}}{\pgfqpoint{4.650000in}{3.080000in}}%
\pgfusepath{clip}%
\pgfsetrectcap%
\pgfsetroundjoin%
\pgfsetlinewidth{0.803000pt}%
\definecolor{currentstroke}{rgb}{0.690196,0.690196,0.690196}%
\pgfsetstrokecolor{currentstroke}%
\pgfsetdash{}{0pt}%
\pgfpathmoveto{\pgfqpoint{0.750000in}{6.405701in}}%
\pgfpathlineto{\pgfqpoint{5.400000in}{6.405701in}}%
\pgfusepath{stroke}%
\end{pgfscope}%
\begin{pgfscope}%
\pgfsetbuttcap%
\pgfsetroundjoin%
\definecolor{currentfill}{rgb}{0.000000,0.000000,0.000000}%
\pgfsetfillcolor{currentfill}%
\pgfsetlinewidth{0.803000pt}%
\definecolor{currentstroke}{rgb}{0.000000,0.000000,0.000000}%
\pgfsetstrokecolor{currentstroke}%
\pgfsetdash{}{0pt}%
\pgfsys@defobject{currentmarker}{\pgfqpoint{-0.048611in}{0.000000in}}{\pgfqpoint{-0.000000in}{0.000000in}}{%
\pgfpathmoveto{\pgfqpoint{-0.000000in}{0.000000in}}%
\pgfpathlineto{\pgfqpoint{-0.048611in}{0.000000in}}%
\pgfusepath{stroke,fill}%
}%
\begin{pgfscope}%
\pgfsys@transformshift{0.750000in}{6.405701in}%
\pgfsys@useobject{currentmarker}{}%
\end{pgfscope}%
\end{pgfscope}%
\begin{pgfscope}%
\definecolor{textcolor}{rgb}{0.000000,0.000000,0.000000}%
\pgfsetstrokecolor{textcolor}%
\pgfsetfillcolor{textcolor}%
\pgftext[x=0.475308in, y=6.354601in, left, base]{\color{textcolor}\rmfamily\fontsize{10.000000}{12.000000}\selectfont \(\displaystyle {1.0}\)}%
\end{pgfscope}%
\begin{pgfscope}%
\pgfpathrectangle{\pgfqpoint{0.750000in}{3.960000in}}{\pgfqpoint{4.650000in}{3.080000in}}%
\pgfusepath{clip}%
\pgfsetrectcap%
\pgfsetroundjoin%
\pgfsetlinewidth{0.803000pt}%
\definecolor{currentstroke}{rgb}{0.690196,0.690196,0.690196}%
\pgfsetstrokecolor{currentstroke}%
\pgfsetdash{}{0pt}%
\pgfpathmoveto{\pgfqpoint{0.750000in}{6.894841in}}%
\pgfpathlineto{\pgfqpoint{5.400000in}{6.894841in}}%
\pgfusepath{stroke}%
\end{pgfscope}%
\begin{pgfscope}%
\pgfsetbuttcap%
\pgfsetroundjoin%
\definecolor{currentfill}{rgb}{0.000000,0.000000,0.000000}%
\pgfsetfillcolor{currentfill}%
\pgfsetlinewidth{0.803000pt}%
\definecolor{currentstroke}{rgb}{0.000000,0.000000,0.000000}%
\pgfsetstrokecolor{currentstroke}%
\pgfsetdash{}{0pt}%
\pgfsys@defobject{currentmarker}{\pgfqpoint{-0.048611in}{0.000000in}}{\pgfqpoint{-0.000000in}{0.000000in}}{%
\pgfpathmoveto{\pgfqpoint{-0.000000in}{0.000000in}}%
\pgfpathlineto{\pgfqpoint{-0.048611in}{0.000000in}}%
\pgfusepath{stroke,fill}%
}%
\begin{pgfscope}%
\pgfsys@transformshift{0.750000in}{6.894841in}%
\pgfsys@useobject{currentmarker}{}%
\end{pgfscope}%
\end{pgfscope}%
\begin{pgfscope}%
\definecolor{textcolor}{rgb}{0.000000,0.000000,0.000000}%
\pgfsetstrokecolor{textcolor}%
\pgfsetfillcolor{textcolor}%
\pgftext[x=0.475308in, y=6.843741in, left, base]{\color{textcolor}\rmfamily\fontsize{10.000000}{12.000000}\selectfont \(\displaystyle {1.2}\)}%
\end{pgfscope}%
\begin{pgfscope}%
\definecolor{textcolor}{rgb}{0.000000,0.000000,0.000000}%
\pgfsetstrokecolor{textcolor}%
\pgfsetfillcolor{textcolor}%
\pgftext[x=0.419752in,y=5.500000in,,bottom,rotate=90.000000]{\color{textcolor}\rmfamily\fontsize{10.000000}{12.000000}\selectfont power in pu}%
\end{pgfscope}%
\begin{pgfscope}%
\pgfpathrectangle{\pgfqpoint{0.750000in}{3.960000in}}{\pgfqpoint{4.650000in}{3.080000in}}%
\pgfusepath{clip}%
\pgfsetrectcap%
\pgfsetroundjoin%
\pgfsetlinewidth{2.007500pt}%
\definecolor{currentstroke}{rgb}{0.121569,0.466667,0.705882}%
\pgfsetstrokecolor{currentstroke}%
\pgfsetdash{}{0pt}%
\pgfpathmoveto{\pgfqpoint{0.750000in}{3.960000in}}%
\pgfpathlineto{\pgfqpoint{0.844898in}{4.148036in}}%
\pgfpathlineto{\pgfqpoint{0.939796in}{4.335299in}}%
\pgfpathlineto{\pgfqpoint{1.034694in}{4.521020in}}%
\pgfpathlineto{\pgfqpoint{1.129592in}{4.704436in}}%
\pgfpathlineto{\pgfqpoint{1.224490in}{4.884793in}}%
\pgfpathlineto{\pgfqpoint{1.319388in}{5.061349in}}%
\pgfpathlineto{\pgfqpoint{1.414286in}{5.233380in}}%
\pgfpathlineto{\pgfqpoint{1.509184in}{5.400178in}}%
\pgfpathlineto{\pgfqpoint{1.604082in}{5.561058in}}%
\pgfpathlineto{\pgfqpoint{1.698980in}{5.715359in}}%
\pgfpathlineto{\pgfqpoint{1.793878in}{5.862447in}}%
\pgfpathlineto{\pgfqpoint{1.888776in}{6.001718in}}%
\pgfpathlineto{\pgfqpoint{1.983673in}{6.132598in}}%
\pgfpathlineto{\pgfqpoint{2.078571in}{6.254551in}}%
\pgfpathlineto{\pgfqpoint{2.173469in}{6.367075in}}%
\pgfpathlineto{\pgfqpoint{2.268367in}{6.469708in}}%
\pgfpathlineto{\pgfqpoint{2.363265in}{6.562028in}}%
\pgfpathlineto{\pgfqpoint{2.458163in}{6.643656in}}%
\pgfpathlineto{\pgfqpoint{2.553061in}{6.714256in}}%
\pgfpathlineto{\pgfqpoint{2.647959in}{6.773538in}}%
\pgfpathlineto{\pgfqpoint{2.742857in}{6.821259in}}%
\pgfpathlineto{\pgfqpoint{2.837755in}{6.857222in}}%
\pgfpathlineto{\pgfqpoint{2.932653in}{6.881280in}}%
\pgfpathlineto{\pgfqpoint{3.027551in}{6.893333in}}%
\pgfpathlineto{\pgfqpoint{3.122449in}{6.893333in}}%
\pgfpathlineto{\pgfqpoint{3.217347in}{6.881280in}}%
\pgfpathlineto{\pgfqpoint{3.312245in}{6.857222in}}%
\pgfpathlineto{\pgfqpoint{3.407143in}{6.821259in}}%
\pgfpathlineto{\pgfqpoint{3.502041in}{6.773538in}}%
\pgfpathlineto{\pgfqpoint{3.596939in}{6.714256in}}%
\pgfpathlineto{\pgfqpoint{3.691837in}{6.643656in}}%
\pgfpathlineto{\pgfqpoint{3.786735in}{6.562028in}}%
\pgfpathlineto{\pgfqpoint{3.881633in}{6.469708in}}%
\pgfpathlineto{\pgfqpoint{3.976531in}{6.367075in}}%
\pgfpathlineto{\pgfqpoint{4.071429in}{6.254551in}}%
\pgfpathlineto{\pgfqpoint{4.166327in}{6.132598in}}%
\pgfpathlineto{\pgfqpoint{4.261224in}{6.001718in}}%
\pgfpathlineto{\pgfqpoint{4.356122in}{5.862447in}}%
\pgfpathlineto{\pgfqpoint{4.451020in}{5.715359in}}%
\pgfpathlineto{\pgfqpoint{4.545918in}{5.561058in}}%
\pgfpathlineto{\pgfqpoint{4.640816in}{5.400178in}}%
\pgfpathlineto{\pgfqpoint{4.735714in}{5.233380in}}%
\pgfpathlineto{\pgfqpoint{4.830612in}{5.061349in}}%
\pgfpathlineto{\pgfqpoint{4.925510in}{4.884793in}}%
\pgfpathlineto{\pgfqpoint{5.020408in}{4.704436in}}%
\pgfpathlineto{\pgfqpoint{5.115306in}{4.521020in}}%
\pgfpathlineto{\pgfqpoint{5.210204in}{4.335299in}}%
\pgfpathlineto{\pgfqpoint{5.305102in}{4.148036in}}%
\pgfpathlineto{\pgfqpoint{5.400000in}{3.960000in}}%
\pgfusepath{stroke}%
\end{pgfscope}%
\begin{pgfscope}%
\pgfpathrectangle{\pgfqpoint{0.750000in}{3.960000in}}{\pgfqpoint{4.650000in}{3.080000in}}%
\pgfusepath{clip}%
\pgfsetrectcap%
\pgfsetroundjoin%
\pgfsetlinewidth{2.007500pt}%
\definecolor{currentstroke}{rgb}{1.000000,0.498039,0.054902}%
\pgfsetstrokecolor{currentstroke}%
\pgfsetdash{}{0pt}%
\pgfpathmoveto{\pgfqpoint{0.750000in}{6.161131in}}%
\pgfpathlineto{\pgfqpoint{0.844898in}{6.161131in}}%
\pgfpathlineto{\pgfqpoint{0.939796in}{6.161131in}}%
\pgfpathlineto{\pgfqpoint{1.034694in}{6.161131in}}%
\pgfpathlineto{\pgfqpoint{1.129592in}{6.161131in}}%
\pgfpathlineto{\pgfqpoint{1.224490in}{6.161131in}}%
\pgfpathlineto{\pgfqpoint{1.319388in}{6.161131in}}%
\pgfpathlineto{\pgfqpoint{1.414286in}{6.161131in}}%
\pgfpathlineto{\pgfqpoint{1.509184in}{6.161131in}}%
\pgfpathlineto{\pgfqpoint{1.604082in}{6.161131in}}%
\pgfpathlineto{\pgfqpoint{1.698980in}{6.161131in}}%
\pgfpathlineto{\pgfqpoint{1.793878in}{6.161131in}}%
\pgfpathlineto{\pgfqpoint{1.888776in}{6.161131in}}%
\pgfpathlineto{\pgfqpoint{1.983673in}{6.161131in}}%
\pgfpathlineto{\pgfqpoint{2.078571in}{6.161131in}}%
\pgfpathlineto{\pgfqpoint{2.173469in}{6.161131in}}%
\pgfpathlineto{\pgfqpoint{2.268367in}{6.161131in}}%
\pgfpathlineto{\pgfqpoint{2.363265in}{6.161131in}}%
\pgfpathlineto{\pgfqpoint{2.458163in}{6.161131in}}%
\pgfpathlineto{\pgfqpoint{2.553061in}{6.161131in}}%
\pgfpathlineto{\pgfqpoint{2.647959in}{6.161131in}}%
\pgfpathlineto{\pgfqpoint{2.742857in}{6.161131in}}%
\pgfpathlineto{\pgfqpoint{2.837755in}{6.161131in}}%
\pgfpathlineto{\pgfqpoint{2.932653in}{6.161131in}}%
\pgfpathlineto{\pgfqpoint{3.027551in}{6.161131in}}%
\pgfpathlineto{\pgfqpoint{3.122449in}{6.161131in}}%
\pgfpathlineto{\pgfqpoint{3.217347in}{6.161131in}}%
\pgfpathlineto{\pgfqpoint{3.312245in}{6.161131in}}%
\pgfpathlineto{\pgfqpoint{3.407143in}{6.161131in}}%
\pgfpathlineto{\pgfqpoint{3.502041in}{6.161131in}}%
\pgfpathlineto{\pgfqpoint{3.596939in}{6.161131in}}%
\pgfpathlineto{\pgfqpoint{3.691837in}{6.161131in}}%
\pgfpathlineto{\pgfqpoint{3.786735in}{6.161131in}}%
\pgfpathlineto{\pgfqpoint{3.881633in}{6.161131in}}%
\pgfpathlineto{\pgfqpoint{3.976531in}{6.161131in}}%
\pgfpathlineto{\pgfqpoint{4.071429in}{6.161131in}}%
\pgfpathlineto{\pgfqpoint{4.166327in}{6.161131in}}%
\pgfpathlineto{\pgfqpoint{4.261224in}{6.161131in}}%
\pgfpathlineto{\pgfqpoint{4.356122in}{6.161131in}}%
\pgfpathlineto{\pgfqpoint{4.451020in}{6.161131in}}%
\pgfpathlineto{\pgfqpoint{4.545918in}{6.161131in}}%
\pgfpathlineto{\pgfqpoint{4.640816in}{6.161131in}}%
\pgfpathlineto{\pgfqpoint{4.735714in}{6.161131in}}%
\pgfpathlineto{\pgfqpoint{4.830612in}{6.161131in}}%
\pgfpathlineto{\pgfqpoint{4.925510in}{6.161131in}}%
\pgfpathlineto{\pgfqpoint{5.020408in}{6.161131in}}%
\pgfpathlineto{\pgfqpoint{5.115306in}{6.161131in}}%
\pgfpathlineto{\pgfqpoint{5.210204in}{6.161131in}}%
\pgfpathlineto{\pgfqpoint{5.305102in}{6.161131in}}%
\pgfpathlineto{\pgfqpoint{5.400000in}{6.161131in}}%
\pgfusepath{stroke}%
\end{pgfscope}%
\begin{pgfscope}%
\pgfsetrectcap%
\pgfsetmiterjoin%
\pgfsetlinewidth{0.803000pt}%
\definecolor{currentstroke}{rgb}{0.000000,0.000000,0.000000}%
\pgfsetstrokecolor{currentstroke}%
\pgfsetdash{}{0pt}%
\pgfpathmoveto{\pgfqpoint{0.750000in}{3.960000in}}%
\pgfpathlineto{\pgfqpoint{0.750000in}{7.040000in}}%
\pgfusepath{stroke}%
\end{pgfscope}%
\begin{pgfscope}%
\pgfsetrectcap%
\pgfsetmiterjoin%
\pgfsetlinewidth{0.803000pt}%
\definecolor{currentstroke}{rgb}{0.000000,0.000000,0.000000}%
\pgfsetstrokecolor{currentstroke}%
\pgfsetdash{}{0pt}%
\pgfpathmoveto{\pgfqpoint{5.400000in}{3.960000in}}%
\pgfpathlineto{\pgfqpoint{5.400000in}{7.040000in}}%
\pgfusepath{stroke}%
\end{pgfscope}%
\begin{pgfscope}%
\pgfsetrectcap%
\pgfsetmiterjoin%
\pgfsetlinewidth{0.803000pt}%
\definecolor{currentstroke}{rgb}{0.000000,0.000000,0.000000}%
\pgfsetstrokecolor{currentstroke}%
\pgfsetdash{}{0pt}%
\pgfpathmoveto{\pgfqpoint{0.750000in}{3.960000in}}%
\pgfpathlineto{\pgfqpoint{5.400000in}{3.960000in}}%
\pgfusepath{stroke}%
\end{pgfscope}%
\begin{pgfscope}%
\pgfsetrectcap%
\pgfsetmiterjoin%
\pgfsetlinewidth{0.803000pt}%
\definecolor{currentstroke}{rgb}{0.000000,0.000000,0.000000}%
\pgfsetstrokecolor{currentstroke}%
\pgfsetdash{}{0pt}%
\pgfpathmoveto{\pgfqpoint{0.750000in}{7.040000in}}%
\pgfpathlineto{\pgfqpoint{5.400000in}{7.040000in}}%
\pgfusepath{stroke}%
\end{pgfscope}%
\begin{pgfscope}%
\pgfsetbuttcap%
\pgfsetmiterjoin%
\definecolor{currentfill}{rgb}{1.000000,1.000000,1.000000}%
\pgfsetfillcolor{currentfill}%
\pgfsetfillopacity{0.800000}%
\pgfsetlinewidth{1.003750pt}%
\definecolor{currentstroke}{rgb}{0.800000,0.800000,0.800000}%
\pgfsetstrokecolor{currentstroke}%
\pgfsetstrokeopacity{0.800000}%
\pgfsetdash{}{0pt}%
\pgfpathmoveto{\pgfqpoint{2.342004in}{4.029444in}}%
\pgfpathlineto{\pgfqpoint{3.807996in}{4.029444in}}%
\pgfpathquadraticcurveto{\pgfqpoint{3.835774in}{4.029444in}}{\pgfqpoint{3.835774in}{4.057222in}}%
\pgfpathlineto{\pgfqpoint{3.835774in}{4.448267in}}%
\pgfpathquadraticcurveto{\pgfqpoint{3.835774in}{4.476045in}}{\pgfqpoint{3.807996in}{4.476045in}}%
\pgfpathlineto{\pgfqpoint{2.342004in}{4.476045in}}%
\pgfpathquadraticcurveto{\pgfqpoint{2.314226in}{4.476045in}}{\pgfqpoint{2.314226in}{4.448267in}}%
\pgfpathlineto{\pgfqpoint{2.314226in}{4.057222in}}%
\pgfpathquadraticcurveto{\pgfqpoint{2.314226in}{4.029444in}}{\pgfqpoint{2.342004in}{4.029444in}}%
\pgfpathlineto{\pgfqpoint{2.342004in}{4.029444in}}%
\pgfpathclose%
\pgfusepath{stroke,fill}%
\end{pgfscope}%
\begin{pgfscope}%
\pgfsetrectcap%
\pgfsetroundjoin%
\pgfsetlinewidth{2.007500pt}%
\definecolor{currentstroke}{rgb}{0.121569,0.466667,0.705882}%
\pgfsetstrokecolor{currentstroke}%
\pgfsetdash{}{0pt}%
\pgfpathmoveto{\pgfqpoint{2.369781in}{4.365748in}}%
\pgfpathlineto{\pgfqpoint{2.508670in}{4.365748in}}%
\pgfpathlineto{\pgfqpoint{2.647559in}{4.365748in}}%
\pgfusepath{stroke}%
\end{pgfscope}%
\begin{pgfscope}%
\definecolor{textcolor}{rgb}{0.000000,0.000000,0.000000}%
\pgfsetstrokecolor{textcolor}%
\pgfsetfillcolor{textcolor}%
\pgftext[x=2.758670in,y=4.317137in,left,base]{\color{textcolor}\rmfamily\fontsize{10.000000}{12.000000}\selectfont \(\displaystyle P_\mathrm{e}\) pre-fault}%
\end{pgfscope}%
\begin{pgfscope}%
\pgfsetrectcap%
\pgfsetroundjoin%
\pgfsetlinewidth{2.007500pt}%
\definecolor{currentstroke}{rgb}{1.000000,0.498039,0.054902}%
\pgfsetstrokecolor{currentstroke}%
\pgfsetdash{}{0pt}%
\pgfpathmoveto{\pgfqpoint{2.369781in}{4.163857in}}%
\pgfpathlineto{\pgfqpoint{2.508670in}{4.163857in}}%
\pgfpathlineto{\pgfqpoint{2.647559in}{4.163857in}}%
\pgfusepath{stroke}%
\end{pgfscope}%
\begin{pgfscope}%
\definecolor{textcolor}{rgb}{0.000000,0.000000,0.000000}%
\pgfsetstrokecolor{textcolor}%
\pgfsetfillcolor{textcolor}%
\pgftext[x=2.758670in,y=4.115246in,left,base]{\color{textcolor}\rmfamily\fontsize{10.000000}{12.000000}\selectfont \(\displaystyle P_\mathrm{T}\) of the turbine}%
\end{pgfscope}%
\begin{pgfscope}%
\pgfsetbuttcap%
\pgfsetmiterjoin%
\definecolor{currentfill}{rgb}{1.000000,1.000000,1.000000}%
\pgfsetfillcolor{currentfill}%
\pgfsetlinewidth{0.000000pt}%
\definecolor{currentstroke}{rgb}{0.000000,0.000000,0.000000}%
\pgfsetstrokecolor{currentstroke}%
\pgfsetstrokeopacity{0.000000}%
\pgfsetdash{}{0pt}%
\pgfpathmoveto{\pgfqpoint{0.750000in}{0.880000in}}%
\pgfpathlineto{\pgfqpoint{5.400000in}{0.880000in}}%
\pgfpathlineto{\pgfqpoint{5.400000in}{3.960000in}}%
\pgfpathlineto{\pgfqpoint{0.750000in}{3.960000in}}%
\pgfpathlineto{\pgfqpoint{0.750000in}{0.880000in}}%
\pgfpathclose%
\pgfusepath{fill}%
\end{pgfscope}%
\begin{pgfscope}%
\pgfpathrectangle{\pgfqpoint{0.750000in}{0.880000in}}{\pgfqpoint{4.650000in}{3.080000in}}%
\pgfusepath{clip}%
\pgfsetrectcap%
\pgfsetroundjoin%
\pgfsetlinewidth{0.803000pt}%
\definecolor{currentstroke}{rgb}{0.690196,0.690196,0.690196}%
\pgfsetstrokecolor{currentstroke}%
\pgfsetdash{}{0pt}%
\pgfpathmoveto{\pgfqpoint{0.750000in}{0.880000in}}%
\pgfpathlineto{\pgfqpoint{0.750000in}{3.960000in}}%
\pgfusepath{stroke}%
\end{pgfscope}%
\begin{pgfscope}%
\pgfsetbuttcap%
\pgfsetroundjoin%
\definecolor{currentfill}{rgb}{0.000000,0.000000,0.000000}%
\pgfsetfillcolor{currentfill}%
\pgfsetlinewidth{0.803000pt}%
\definecolor{currentstroke}{rgb}{0.000000,0.000000,0.000000}%
\pgfsetstrokecolor{currentstroke}%
\pgfsetdash{}{0pt}%
\pgfsys@defobject{currentmarker}{\pgfqpoint{0.000000in}{-0.048611in}}{\pgfqpoint{0.000000in}{0.000000in}}{%
\pgfpathmoveto{\pgfqpoint{0.000000in}{0.000000in}}%
\pgfpathlineto{\pgfqpoint{0.000000in}{-0.048611in}}%
\pgfusepath{stroke,fill}%
}%
\begin{pgfscope}%
\pgfsys@transformshift{0.750000in}{0.880000in}%
\pgfsys@useobject{currentmarker}{}%
\end{pgfscope}%
\end{pgfscope}%
\begin{pgfscope}%
\definecolor{textcolor}{rgb}{0.000000,0.000000,0.000000}%
\pgfsetstrokecolor{textcolor}%
\pgfsetfillcolor{textcolor}%
\pgftext[x=0.750000in,y=0.782778in,,top]{\color{textcolor}\rmfamily\fontsize{10.000000}{12.000000}\selectfont \(\displaystyle {0}\)}%
\end{pgfscope}%
\begin{pgfscope}%
\pgfpathrectangle{\pgfqpoint{0.750000in}{0.880000in}}{\pgfqpoint{4.650000in}{3.080000in}}%
\pgfusepath{clip}%
\pgfsetrectcap%
\pgfsetroundjoin%
\pgfsetlinewidth{0.803000pt}%
\definecolor{currentstroke}{rgb}{0.690196,0.690196,0.690196}%
\pgfsetstrokecolor{currentstroke}%
\pgfsetdash{}{0pt}%
\pgfpathmoveto{\pgfqpoint{1.266667in}{0.880000in}}%
\pgfpathlineto{\pgfqpoint{1.266667in}{3.960000in}}%
\pgfusepath{stroke}%
\end{pgfscope}%
\begin{pgfscope}%
\pgfsetbuttcap%
\pgfsetroundjoin%
\definecolor{currentfill}{rgb}{0.000000,0.000000,0.000000}%
\pgfsetfillcolor{currentfill}%
\pgfsetlinewidth{0.803000pt}%
\definecolor{currentstroke}{rgb}{0.000000,0.000000,0.000000}%
\pgfsetstrokecolor{currentstroke}%
\pgfsetdash{}{0pt}%
\pgfsys@defobject{currentmarker}{\pgfqpoint{0.000000in}{-0.048611in}}{\pgfqpoint{0.000000in}{0.000000in}}{%
\pgfpathmoveto{\pgfqpoint{0.000000in}{0.000000in}}%
\pgfpathlineto{\pgfqpoint{0.000000in}{-0.048611in}}%
\pgfusepath{stroke,fill}%
}%
\begin{pgfscope}%
\pgfsys@transformshift{1.266667in}{0.880000in}%
\pgfsys@useobject{currentmarker}{}%
\end{pgfscope}%
\end{pgfscope}%
\begin{pgfscope}%
\definecolor{textcolor}{rgb}{0.000000,0.000000,0.000000}%
\pgfsetstrokecolor{textcolor}%
\pgfsetfillcolor{textcolor}%
\pgftext[x=1.266667in,y=0.782778in,,top]{\color{textcolor}\rmfamily\fontsize{10.000000}{12.000000}\selectfont \(\displaystyle {20}\)}%
\end{pgfscope}%
\begin{pgfscope}%
\pgfpathrectangle{\pgfqpoint{0.750000in}{0.880000in}}{\pgfqpoint{4.650000in}{3.080000in}}%
\pgfusepath{clip}%
\pgfsetrectcap%
\pgfsetroundjoin%
\pgfsetlinewidth{0.803000pt}%
\definecolor{currentstroke}{rgb}{0.690196,0.690196,0.690196}%
\pgfsetstrokecolor{currentstroke}%
\pgfsetdash{}{0pt}%
\pgfpathmoveto{\pgfqpoint{1.783333in}{0.880000in}}%
\pgfpathlineto{\pgfqpoint{1.783333in}{3.960000in}}%
\pgfusepath{stroke}%
\end{pgfscope}%
\begin{pgfscope}%
\pgfsetbuttcap%
\pgfsetroundjoin%
\definecolor{currentfill}{rgb}{0.000000,0.000000,0.000000}%
\pgfsetfillcolor{currentfill}%
\pgfsetlinewidth{0.803000pt}%
\definecolor{currentstroke}{rgb}{0.000000,0.000000,0.000000}%
\pgfsetstrokecolor{currentstroke}%
\pgfsetdash{}{0pt}%
\pgfsys@defobject{currentmarker}{\pgfqpoint{0.000000in}{-0.048611in}}{\pgfqpoint{0.000000in}{0.000000in}}{%
\pgfpathmoveto{\pgfqpoint{0.000000in}{0.000000in}}%
\pgfpathlineto{\pgfqpoint{0.000000in}{-0.048611in}}%
\pgfusepath{stroke,fill}%
}%
\begin{pgfscope}%
\pgfsys@transformshift{1.783333in}{0.880000in}%
\pgfsys@useobject{currentmarker}{}%
\end{pgfscope}%
\end{pgfscope}%
\begin{pgfscope}%
\definecolor{textcolor}{rgb}{0.000000,0.000000,0.000000}%
\pgfsetstrokecolor{textcolor}%
\pgfsetfillcolor{textcolor}%
\pgftext[x=1.783333in,y=0.782778in,,top]{\color{textcolor}\rmfamily\fontsize{10.000000}{12.000000}\selectfont \(\displaystyle {40}\)}%
\end{pgfscope}%
\begin{pgfscope}%
\pgfpathrectangle{\pgfqpoint{0.750000in}{0.880000in}}{\pgfqpoint{4.650000in}{3.080000in}}%
\pgfusepath{clip}%
\pgfsetrectcap%
\pgfsetroundjoin%
\pgfsetlinewidth{0.803000pt}%
\definecolor{currentstroke}{rgb}{0.690196,0.690196,0.690196}%
\pgfsetstrokecolor{currentstroke}%
\pgfsetdash{}{0pt}%
\pgfpathmoveto{\pgfqpoint{2.300000in}{0.880000in}}%
\pgfpathlineto{\pgfqpoint{2.300000in}{3.960000in}}%
\pgfusepath{stroke}%
\end{pgfscope}%
\begin{pgfscope}%
\pgfsetbuttcap%
\pgfsetroundjoin%
\definecolor{currentfill}{rgb}{0.000000,0.000000,0.000000}%
\pgfsetfillcolor{currentfill}%
\pgfsetlinewidth{0.803000pt}%
\definecolor{currentstroke}{rgb}{0.000000,0.000000,0.000000}%
\pgfsetstrokecolor{currentstroke}%
\pgfsetdash{}{0pt}%
\pgfsys@defobject{currentmarker}{\pgfqpoint{0.000000in}{-0.048611in}}{\pgfqpoint{0.000000in}{0.000000in}}{%
\pgfpathmoveto{\pgfqpoint{0.000000in}{0.000000in}}%
\pgfpathlineto{\pgfqpoint{0.000000in}{-0.048611in}}%
\pgfusepath{stroke,fill}%
}%
\begin{pgfscope}%
\pgfsys@transformshift{2.300000in}{0.880000in}%
\pgfsys@useobject{currentmarker}{}%
\end{pgfscope}%
\end{pgfscope}%
\begin{pgfscope}%
\definecolor{textcolor}{rgb}{0.000000,0.000000,0.000000}%
\pgfsetstrokecolor{textcolor}%
\pgfsetfillcolor{textcolor}%
\pgftext[x=2.300000in,y=0.782778in,,top]{\color{textcolor}\rmfamily\fontsize{10.000000}{12.000000}\selectfont \(\displaystyle {60}\)}%
\end{pgfscope}%
\begin{pgfscope}%
\pgfpathrectangle{\pgfqpoint{0.750000in}{0.880000in}}{\pgfqpoint{4.650000in}{3.080000in}}%
\pgfusepath{clip}%
\pgfsetrectcap%
\pgfsetroundjoin%
\pgfsetlinewidth{0.803000pt}%
\definecolor{currentstroke}{rgb}{0.690196,0.690196,0.690196}%
\pgfsetstrokecolor{currentstroke}%
\pgfsetdash{}{0pt}%
\pgfpathmoveto{\pgfqpoint{2.816667in}{0.880000in}}%
\pgfpathlineto{\pgfqpoint{2.816667in}{3.960000in}}%
\pgfusepath{stroke}%
\end{pgfscope}%
\begin{pgfscope}%
\pgfsetbuttcap%
\pgfsetroundjoin%
\definecolor{currentfill}{rgb}{0.000000,0.000000,0.000000}%
\pgfsetfillcolor{currentfill}%
\pgfsetlinewidth{0.803000pt}%
\definecolor{currentstroke}{rgb}{0.000000,0.000000,0.000000}%
\pgfsetstrokecolor{currentstroke}%
\pgfsetdash{}{0pt}%
\pgfsys@defobject{currentmarker}{\pgfqpoint{0.000000in}{-0.048611in}}{\pgfqpoint{0.000000in}{0.000000in}}{%
\pgfpathmoveto{\pgfqpoint{0.000000in}{0.000000in}}%
\pgfpathlineto{\pgfqpoint{0.000000in}{-0.048611in}}%
\pgfusepath{stroke,fill}%
}%
\begin{pgfscope}%
\pgfsys@transformshift{2.816667in}{0.880000in}%
\pgfsys@useobject{currentmarker}{}%
\end{pgfscope}%
\end{pgfscope}%
\begin{pgfscope}%
\definecolor{textcolor}{rgb}{0.000000,0.000000,0.000000}%
\pgfsetstrokecolor{textcolor}%
\pgfsetfillcolor{textcolor}%
\pgftext[x=2.816667in,y=0.782778in,,top]{\color{textcolor}\rmfamily\fontsize{10.000000}{12.000000}\selectfont \(\displaystyle {80}\)}%
\end{pgfscope}%
\begin{pgfscope}%
\pgfpathrectangle{\pgfqpoint{0.750000in}{0.880000in}}{\pgfqpoint{4.650000in}{3.080000in}}%
\pgfusepath{clip}%
\pgfsetrectcap%
\pgfsetroundjoin%
\pgfsetlinewidth{0.803000pt}%
\definecolor{currentstroke}{rgb}{0.690196,0.690196,0.690196}%
\pgfsetstrokecolor{currentstroke}%
\pgfsetdash{}{0pt}%
\pgfpathmoveto{\pgfqpoint{3.333333in}{0.880000in}}%
\pgfpathlineto{\pgfqpoint{3.333333in}{3.960000in}}%
\pgfusepath{stroke}%
\end{pgfscope}%
\begin{pgfscope}%
\pgfsetbuttcap%
\pgfsetroundjoin%
\definecolor{currentfill}{rgb}{0.000000,0.000000,0.000000}%
\pgfsetfillcolor{currentfill}%
\pgfsetlinewidth{0.803000pt}%
\definecolor{currentstroke}{rgb}{0.000000,0.000000,0.000000}%
\pgfsetstrokecolor{currentstroke}%
\pgfsetdash{}{0pt}%
\pgfsys@defobject{currentmarker}{\pgfqpoint{0.000000in}{-0.048611in}}{\pgfqpoint{0.000000in}{0.000000in}}{%
\pgfpathmoveto{\pgfqpoint{0.000000in}{0.000000in}}%
\pgfpathlineto{\pgfqpoint{0.000000in}{-0.048611in}}%
\pgfusepath{stroke,fill}%
}%
\begin{pgfscope}%
\pgfsys@transformshift{3.333333in}{0.880000in}%
\pgfsys@useobject{currentmarker}{}%
\end{pgfscope}%
\end{pgfscope}%
\begin{pgfscope}%
\definecolor{textcolor}{rgb}{0.000000,0.000000,0.000000}%
\pgfsetstrokecolor{textcolor}%
\pgfsetfillcolor{textcolor}%
\pgftext[x=3.333333in,y=0.782778in,,top]{\color{textcolor}\rmfamily\fontsize{10.000000}{12.000000}\selectfont \(\displaystyle {100}\)}%
\end{pgfscope}%
\begin{pgfscope}%
\pgfpathrectangle{\pgfqpoint{0.750000in}{0.880000in}}{\pgfqpoint{4.650000in}{3.080000in}}%
\pgfusepath{clip}%
\pgfsetrectcap%
\pgfsetroundjoin%
\pgfsetlinewidth{0.803000pt}%
\definecolor{currentstroke}{rgb}{0.690196,0.690196,0.690196}%
\pgfsetstrokecolor{currentstroke}%
\pgfsetdash{}{0pt}%
\pgfpathmoveto{\pgfqpoint{3.850000in}{0.880000in}}%
\pgfpathlineto{\pgfqpoint{3.850000in}{3.960000in}}%
\pgfusepath{stroke}%
\end{pgfscope}%
\begin{pgfscope}%
\pgfsetbuttcap%
\pgfsetroundjoin%
\definecolor{currentfill}{rgb}{0.000000,0.000000,0.000000}%
\pgfsetfillcolor{currentfill}%
\pgfsetlinewidth{0.803000pt}%
\definecolor{currentstroke}{rgb}{0.000000,0.000000,0.000000}%
\pgfsetstrokecolor{currentstroke}%
\pgfsetdash{}{0pt}%
\pgfsys@defobject{currentmarker}{\pgfqpoint{0.000000in}{-0.048611in}}{\pgfqpoint{0.000000in}{0.000000in}}{%
\pgfpathmoveto{\pgfqpoint{0.000000in}{0.000000in}}%
\pgfpathlineto{\pgfqpoint{0.000000in}{-0.048611in}}%
\pgfusepath{stroke,fill}%
}%
\begin{pgfscope}%
\pgfsys@transformshift{3.850000in}{0.880000in}%
\pgfsys@useobject{currentmarker}{}%
\end{pgfscope}%
\end{pgfscope}%
\begin{pgfscope}%
\definecolor{textcolor}{rgb}{0.000000,0.000000,0.000000}%
\pgfsetstrokecolor{textcolor}%
\pgfsetfillcolor{textcolor}%
\pgftext[x=3.850000in,y=0.782778in,,top]{\color{textcolor}\rmfamily\fontsize{10.000000}{12.000000}\selectfont \(\displaystyle {120}\)}%
\end{pgfscope}%
\begin{pgfscope}%
\pgfpathrectangle{\pgfqpoint{0.750000in}{0.880000in}}{\pgfqpoint{4.650000in}{3.080000in}}%
\pgfusepath{clip}%
\pgfsetrectcap%
\pgfsetroundjoin%
\pgfsetlinewidth{0.803000pt}%
\definecolor{currentstroke}{rgb}{0.690196,0.690196,0.690196}%
\pgfsetstrokecolor{currentstroke}%
\pgfsetdash{}{0pt}%
\pgfpathmoveto{\pgfqpoint{4.366667in}{0.880000in}}%
\pgfpathlineto{\pgfqpoint{4.366667in}{3.960000in}}%
\pgfusepath{stroke}%
\end{pgfscope}%
\begin{pgfscope}%
\pgfsetbuttcap%
\pgfsetroundjoin%
\definecolor{currentfill}{rgb}{0.000000,0.000000,0.000000}%
\pgfsetfillcolor{currentfill}%
\pgfsetlinewidth{0.803000pt}%
\definecolor{currentstroke}{rgb}{0.000000,0.000000,0.000000}%
\pgfsetstrokecolor{currentstroke}%
\pgfsetdash{}{0pt}%
\pgfsys@defobject{currentmarker}{\pgfqpoint{0.000000in}{-0.048611in}}{\pgfqpoint{0.000000in}{0.000000in}}{%
\pgfpathmoveto{\pgfqpoint{0.000000in}{0.000000in}}%
\pgfpathlineto{\pgfqpoint{0.000000in}{-0.048611in}}%
\pgfusepath{stroke,fill}%
}%
\begin{pgfscope}%
\pgfsys@transformshift{4.366667in}{0.880000in}%
\pgfsys@useobject{currentmarker}{}%
\end{pgfscope}%
\end{pgfscope}%
\begin{pgfscope}%
\definecolor{textcolor}{rgb}{0.000000,0.000000,0.000000}%
\pgfsetstrokecolor{textcolor}%
\pgfsetfillcolor{textcolor}%
\pgftext[x=4.366667in,y=0.782778in,,top]{\color{textcolor}\rmfamily\fontsize{10.000000}{12.000000}\selectfont \(\displaystyle {140}\)}%
\end{pgfscope}%
\begin{pgfscope}%
\pgfpathrectangle{\pgfqpoint{0.750000in}{0.880000in}}{\pgfqpoint{4.650000in}{3.080000in}}%
\pgfusepath{clip}%
\pgfsetrectcap%
\pgfsetroundjoin%
\pgfsetlinewidth{0.803000pt}%
\definecolor{currentstroke}{rgb}{0.690196,0.690196,0.690196}%
\pgfsetstrokecolor{currentstroke}%
\pgfsetdash{}{0pt}%
\pgfpathmoveto{\pgfqpoint{4.883333in}{0.880000in}}%
\pgfpathlineto{\pgfqpoint{4.883333in}{3.960000in}}%
\pgfusepath{stroke}%
\end{pgfscope}%
\begin{pgfscope}%
\pgfsetbuttcap%
\pgfsetroundjoin%
\definecolor{currentfill}{rgb}{0.000000,0.000000,0.000000}%
\pgfsetfillcolor{currentfill}%
\pgfsetlinewidth{0.803000pt}%
\definecolor{currentstroke}{rgb}{0.000000,0.000000,0.000000}%
\pgfsetstrokecolor{currentstroke}%
\pgfsetdash{}{0pt}%
\pgfsys@defobject{currentmarker}{\pgfqpoint{0.000000in}{-0.048611in}}{\pgfqpoint{0.000000in}{0.000000in}}{%
\pgfpathmoveto{\pgfqpoint{0.000000in}{0.000000in}}%
\pgfpathlineto{\pgfqpoint{0.000000in}{-0.048611in}}%
\pgfusepath{stroke,fill}%
}%
\begin{pgfscope}%
\pgfsys@transformshift{4.883333in}{0.880000in}%
\pgfsys@useobject{currentmarker}{}%
\end{pgfscope}%
\end{pgfscope}%
\begin{pgfscope}%
\definecolor{textcolor}{rgb}{0.000000,0.000000,0.000000}%
\pgfsetstrokecolor{textcolor}%
\pgfsetfillcolor{textcolor}%
\pgftext[x=4.883333in,y=0.782778in,,top]{\color{textcolor}\rmfamily\fontsize{10.000000}{12.000000}\selectfont \(\displaystyle {160}\)}%
\end{pgfscope}%
\begin{pgfscope}%
\pgfpathrectangle{\pgfqpoint{0.750000in}{0.880000in}}{\pgfqpoint{4.650000in}{3.080000in}}%
\pgfusepath{clip}%
\pgfsetrectcap%
\pgfsetroundjoin%
\pgfsetlinewidth{0.803000pt}%
\definecolor{currentstroke}{rgb}{0.690196,0.690196,0.690196}%
\pgfsetstrokecolor{currentstroke}%
\pgfsetdash{}{0pt}%
\pgfpathmoveto{\pgfqpoint{5.400000in}{0.880000in}}%
\pgfpathlineto{\pgfqpoint{5.400000in}{3.960000in}}%
\pgfusepath{stroke}%
\end{pgfscope}%
\begin{pgfscope}%
\pgfsetbuttcap%
\pgfsetroundjoin%
\definecolor{currentfill}{rgb}{0.000000,0.000000,0.000000}%
\pgfsetfillcolor{currentfill}%
\pgfsetlinewidth{0.803000pt}%
\definecolor{currentstroke}{rgb}{0.000000,0.000000,0.000000}%
\pgfsetstrokecolor{currentstroke}%
\pgfsetdash{}{0pt}%
\pgfsys@defobject{currentmarker}{\pgfqpoint{0.000000in}{-0.048611in}}{\pgfqpoint{0.000000in}{0.000000in}}{%
\pgfpathmoveto{\pgfqpoint{0.000000in}{0.000000in}}%
\pgfpathlineto{\pgfqpoint{0.000000in}{-0.048611in}}%
\pgfusepath{stroke,fill}%
}%
\begin{pgfscope}%
\pgfsys@transformshift{5.400000in}{0.880000in}%
\pgfsys@useobject{currentmarker}{}%
\end{pgfscope}%
\end{pgfscope}%
\begin{pgfscope}%
\definecolor{textcolor}{rgb}{0.000000,0.000000,0.000000}%
\pgfsetstrokecolor{textcolor}%
\pgfsetfillcolor{textcolor}%
\pgftext[x=5.400000in,y=0.782778in,,top]{\color{textcolor}\rmfamily\fontsize{10.000000}{12.000000}\selectfont \(\displaystyle {180}\)}%
\end{pgfscope}%
\begin{pgfscope}%
\definecolor{textcolor}{rgb}{0.000000,0.000000,0.000000}%
\pgfsetstrokecolor{textcolor}%
\pgfsetfillcolor{textcolor}%
\pgftext[x=3.075000in,y=0.594776in,,top]{\color{textcolor}\rmfamily\fontsize{10.000000}{12.000000}\selectfont power angle \(\displaystyle \delta\) in deg}%
\end{pgfscope}%
\begin{pgfscope}%
\pgfpathrectangle{\pgfqpoint{0.750000in}{0.880000in}}{\pgfqpoint{4.650000in}{3.080000in}}%
\pgfusepath{clip}%
\pgfsetrectcap%
\pgfsetroundjoin%
\pgfsetlinewidth{0.803000pt}%
\definecolor{currentstroke}{rgb}{0.690196,0.690196,0.690196}%
\pgfsetstrokecolor{currentstroke}%
\pgfsetdash{}{0pt}%
\pgfpathmoveto{\pgfqpoint{0.750000in}{3.823047in}}%
\pgfpathlineto{\pgfqpoint{5.400000in}{3.823047in}}%
\pgfusepath{stroke}%
\end{pgfscope}%
\begin{pgfscope}%
\pgfsetbuttcap%
\pgfsetroundjoin%
\definecolor{currentfill}{rgb}{0.000000,0.000000,0.000000}%
\pgfsetfillcolor{currentfill}%
\pgfsetlinewidth{0.803000pt}%
\definecolor{currentstroke}{rgb}{0.000000,0.000000,0.000000}%
\pgfsetstrokecolor{currentstroke}%
\pgfsetdash{}{0pt}%
\pgfsys@defobject{currentmarker}{\pgfqpoint{-0.048611in}{0.000000in}}{\pgfqpoint{-0.000000in}{0.000000in}}{%
\pgfpathmoveto{\pgfqpoint{-0.000000in}{0.000000in}}%
\pgfpathlineto{\pgfqpoint{-0.048611in}{0.000000in}}%
\pgfusepath{stroke,fill}%
}%
\begin{pgfscope}%
\pgfsys@transformshift{0.750000in}{3.823047in}%
\pgfsys@useobject{currentmarker}{}%
\end{pgfscope}%
\end{pgfscope}%
\begin{pgfscope}%
\definecolor{textcolor}{rgb}{0.000000,0.000000,0.000000}%
\pgfsetstrokecolor{textcolor}%
\pgfsetfillcolor{textcolor}%
\pgftext[x=0.405863in, y=3.771947in, left, base]{\color{textcolor}\rmfamily\fontsize{10.000000}{12.000000}\selectfont \(\displaystyle {0.00}\)}%
\end{pgfscope}%
\begin{pgfscope}%
\pgfpathrectangle{\pgfqpoint{0.750000in}{0.880000in}}{\pgfqpoint{4.650000in}{3.080000in}}%
\pgfusepath{clip}%
\pgfsetrectcap%
\pgfsetroundjoin%
\pgfsetlinewidth{0.803000pt}%
\definecolor{currentstroke}{rgb}{0.690196,0.690196,0.690196}%
\pgfsetstrokecolor{currentstroke}%
\pgfsetdash{}{0pt}%
\pgfpathmoveto{\pgfqpoint{0.750000in}{3.480665in}}%
\pgfpathlineto{\pgfqpoint{5.400000in}{3.480665in}}%
\pgfusepath{stroke}%
\end{pgfscope}%
\begin{pgfscope}%
\pgfsetbuttcap%
\pgfsetroundjoin%
\definecolor{currentfill}{rgb}{0.000000,0.000000,0.000000}%
\pgfsetfillcolor{currentfill}%
\pgfsetlinewidth{0.803000pt}%
\definecolor{currentstroke}{rgb}{0.000000,0.000000,0.000000}%
\pgfsetstrokecolor{currentstroke}%
\pgfsetdash{}{0pt}%
\pgfsys@defobject{currentmarker}{\pgfqpoint{-0.048611in}{0.000000in}}{\pgfqpoint{-0.000000in}{0.000000in}}{%
\pgfpathmoveto{\pgfqpoint{-0.000000in}{0.000000in}}%
\pgfpathlineto{\pgfqpoint{-0.048611in}{0.000000in}}%
\pgfusepath{stroke,fill}%
}%
\begin{pgfscope}%
\pgfsys@transformshift{0.750000in}{3.480665in}%
\pgfsys@useobject{currentmarker}{}%
\end{pgfscope}%
\end{pgfscope}%
\begin{pgfscope}%
\definecolor{textcolor}{rgb}{0.000000,0.000000,0.000000}%
\pgfsetstrokecolor{textcolor}%
\pgfsetfillcolor{textcolor}%
\pgftext[x=0.405863in, y=3.429565in, left, base]{\color{textcolor}\rmfamily\fontsize{10.000000}{12.000000}\selectfont \(\displaystyle {0.25}\)}%
\end{pgfscope}%
\begin{pgfscope}%
\pgfpathrectangle{\pgfqpoint{0.750000in}{0.880000in}}{\pgfqpoint{4.650000in}{3.080000in}}%
\pgfusepath{clip}%
\pgfsetrectcap%
\pgfsetroundjoin%
\pgfsetlinewidth{0.803000pt}%
\definecolor{currentstroke}{rgb}{0.690196,0.690196,0.690196}%
\pgfsetstrokecolor{currentstroke}%
\pgfsetdash{}{0pt}%
\pgfpathmoveto{\pgfqpoint{0.750000in}{3.138283in}}%
\pgfpathlineto{\pgfqpoint{5.400000in}{3.138283in}}%
\pgfusepath{stroke}%
\end{pgfscope}%
\begin{pgfscope}%
\pgfsetbuttcap%
\pgfsetroundjoin%
\definecolor{currentfill}{rgb}{0.000000,0.000000,0.000000}%
\pgfsetfillcolor{currentfill}%
\pgfsetlinewidth{0.803000pt}%
\definecolor{currentstroke}{rgb}{0.000000,0.000000,0.000000}%
\pgfsetstrokecolor{currentstroke}%
\pgfsetdash{}{0pt}%
\pgfsys@defobject{currentmarker}{\pgfqpoint{-0.048611in}{0.000000in}}{\pgfqpoint{-0.000000in}{0.000000in}}{%
\pgfpathmoveto{\pgfqpoint{-0.000000in}{0.000000in}}%
\pgfpathlineto{\pgfqpoint{-0.048611in}{0.000000in}}%
\pgfusepath{stroke,fill}%
}%
\begin{pgfscope}%
\pgfsys@transformshift{0.750000in}{3.138283in}%
\pgfsys@useobject{currentmarker}{}%
\end{pgfscope}%
\end{pgfscope}%
\begin{pgfscope}%
\definecolor{textcolor}{rgb}{0.000000,0.000000,0.000000}%
\pgfsetstrokecolor{textcolor}%
\pgfsetfillcolor{textcolor}%
\pgftext[x=0.405863in, y=3.087183in, left, base]{\color{textcolor}\rmfamily\fontsize{10.000000}{12.000000}\selectfont \(\displaystyle {0.50}\)}%
\end{pgfscope}%
\begin{pgfscope}%
\pgfpathrectangle{\pgfqpoint{0.750000in}{0.880000in}}{\pgfqpoint{4.650000in}{3.080000in}}%
\pgfusepath{clip}%
\pgfsetrectcap%
\pgfsetroundjoin%
\pgfsetlinewidth{0.803000pt}%
\definecolor{currentstroke}{rgb}{0.690196,0.690196,0.690196}%
\pgfsetstrokecolor{currentstroke}%
\pgfsetdash{}{0pt}%
\pgfpathmoveto{\pgfqpoint{0.750000in}{2.795901in}}%
\pgfpathlineto{\pgfqpoint{5.400000in}{2.795901in}}%
\pgfusepath{stroke}%
\end{pgfscope}%
\begin{pgfscope}%
\pgfsetbuttcap%
\pgfsetroundjoin%
\definecolor{currentfill}{rgb}{0.000000,0.000000,0.000000}%
\pgfsetfillcolor{currentfill}%
\pgfsetlinewidth{0.803000pt}%
\definecolor{currentstroke}{rgb}{0.000000,0.000000,0.000000}%
\pgfsetstrokecolor{currentstroke}%
\pgfsetdash{}{0pt}%
\pgfsys@defobject{currentmarker}{\pgfqpoint{-0.048611in}{0.000000in}}{\pgfqpoint{-0.000000in}{0.000000in}}{%
\pgfpathmoveto{\pgfqpoint{-0.000000in}{0.000000in}}%
\pgfpathlineto{\pgfqpoint{-0.048611in}{0.000000in}}%
\pgfusepath{stroke,fill}%
}%
\begin{pgfscope}%
\pgfsys@transformshift{0.750000in}{2.795901in}%
\pgfsys@useobject{currentmarker}{}%
\end{pgfscope}%
\end{pgfscope}%
\begin{pgfscope}%
\definecolor{textcolor}{rgb}{0.000000,0.000000,0.000000}%
\pgfsetstrokecolor{textcolor}%
\pgfsetfillcolor{textcolor}%
\pgftext[x=0.405863in, y=2.744801in, left, base]{\color{textcolor}\rmfamily\fontsize{10.000000}{12.000000}\selectfont \(\displaystyle {0.75}\)}%
\end{pgfscope}%
\begin{pgfscope}%
\pgfpathrectangle{\pgfqpoint{0.750000in}{0.880000in}}{\pgfqpoint{4.650000in}{3.080000in}}%
\pgfusepath{clip}%
\pgfsetrectcap%
\pgfsetroundjoin%
\pgfsetlinewidth{0.803000pt}%
\definecolor{currentstroke}{rgb}{0.690196,0.690196,0.690196}%
\pgfsetstrokecolor{currentstroke}%
\pgfsetdash{}{0pt}%
\pgfpathmoveto{\pgfqpoint{0.750000in}{2.453519in}}%
\pgfpathlineto{\pgfqpoint{5.400000in}{2.453519in}}%
\pgfusepath{stroke}%
\end{pgfscope}%
\begin{pgfscope}%
\pgfsetbuttcap%
\pgfsetroundjoin%
\definecolor{currentfill}{rgb}{0.000000,0.000000,0.000000}%
\pgfsetfillcolor{currentfill}%
\pgfsetlinewidth{0.803000pt}%
\definecolor{currentstroke}{rgb}{0.000000,0.000000,0.000000}%
\pgfsetstrokecolor{currentstroke}%
\pgfsetdash{}{0pt}%
\pgfsys@defobject{currentmarker}{\pgfqpoint{-0.048611in}{0.000000in}}{\pgfqpoint{-0.000000in}{0.000000in}}{%
\pgfpathmoveto{\pgfqpoint{-0.000000in}{0.000000in}}%
\pgfpathlineto{\pgfqpoint{-0.048611in}{0.000000in}}%
\pgfusepath{stroke,fill}%
}%
\begin{pgfscope}%
\pgfsys@transformshift{0.750000in}{2.453519in}%
\pgfsys@useobject{currentmarker}{}%
\end{pgfscope}%
\end{pgfscope}%
\begin{pgfscope}%
\definecolor{textcolor}{rgb}{0.000000,0.000000,0.000000}%
\pgfsetstrokecolor{textcolor}%
\pgfsetfillcolor{textcolor}%
\pgftext[x=0.405863in, y=2.402419in, left, base]{\color{textcolor}\rmfamily\fontsize{10.000000}{12.000000}\selectfont \(\displaystyle {1.00}\)}%
\end{pgfscope}%
\begin{pgfscope}%
\pgfpathrectangle{\pgfqpoint{0.750000in}{0.880000in}}{\pgfqpoint{4.650000in}{3.080000in}}%
\pgfusepath{clip}%
\pgfsetrectcap%
\pgfsetroundjoin%
\pgfsetlinewidth{0.803000pt}%
\definecolor{currentstroke}{rgb}{0.690196,0.690196,0.690196}%
\pgfsetstrokecolor{currentstroke}%
\pgfsetdash{}{0pt}%
\pgfpathmoveto{\pgfqpoint{0.750000in}{2.111137in}}%
\pgfpathlineto{\pgfqpoint{5.400000in}{2.111137in}}%
\pgfusepath{stroke}%
\end{pgfscope}%
\begin{pgfscope}%
\pgfsetbuttcap%
\pgfsetroundjoin%
\definecolor{currentfill}{rgb}{0.000000,0.000000,0.000000}%
\pgfsetfillcolor{currentfill}%
\pgfsetlinewidth{0.803000pt}%
\definecolor{currentstroke}{rgb}{0.000000,0.000000,0.000000}%
\pgfsetstrokecolor{currentstroke}%
\pgfsetdash{}{0pt}%
\pgfsys@defobject{currentmarker}{\pgfqpoint{-0.048611in}{0.000000in}}{\pgfqpoint{-0.000000in}{0.000000in}}{%
\pgfpathmoveto{\pgfqpoint{-0.000000in}{0.000000in}}%
\pgfpathlineto{\pgfqpoint{-0.048611in}{0.000000in}}%
\pgfusepath{stroke,fill}%
}%
\begin{pgfscope}%
\pgfsys@transformshift{0.750000in}{2.111137in}%
\pgfsys@useobject{currentmarker}{}%
\end{pgfscope}%
\end{pgfscope}%
\begin{pgfscope}%
\definecolor{textcolor}{rgb}{0.000000,0.000000,0.000000}%
\pgfsetstrokecolor{textcolor}%
\pgfsetfillcolor{textcolor}%
\pgftext[x=0.405863in, y=2.060037in, left, base]{\color{textcolor}\rmfamily\fontsize{10.000000}{12.000000}\selectfont \(\displaystyle {1.25}\)}%
\end{pgfscope}%
\begin{pgfscope}%
\pgfpathrectangle{\pgfqpoint{0.750000in}{0.880000in}}{\pgfqpoint{4.650000in}{3.080000in}}%
\pgfusepath{clip}%
\pgfsetrectcap%
\pgfsetroundjoin%
\pgfsetlinewidth{0.803000pt}%
\definecolor{currentstroke}{rgb}{0.690196,0.690196,0.690196}%
\pgfsetstrokecolor{currentstroke}%
\pgfsetdash{}{0pt}%
\pgfpathmoveto{\pgfqpoint{0.750000in}{1.768755in}}%
\pgfpathlineto{\pgfqpoint{5.400000in}{1.768755in}}%
\pgfusepath{stroke}%
\end{pgfscope}%
\begin{pgfscope}%
\pgfsetbuttcap%
\pgfsetroundjoin%
\definecolor{currentfill}{rgb}{0.000000,0.000000,0.000000}%
\pgfsetfillcolor{currentfill}%
\pgfsetlinewidth{0.803000pt}%
\definecolor{currentstroke}{rgb}{0.000000,0.000000,0.000000}%
\pgfsetstrokecolor{currentstroke}%
\pgfsetdash{}{0pt}%
\pgfsys@defobject{currentmarker}{\pgfqpoint{-0.048611in}{0.000000in}}{\pgfqpoint{-0.000000in}{0.000000in}}{%
\pgfpathmoveto{\pgfqpoint{-0.000000in}{0.000000in}}%
\pgfpathlineto{\pgfqpoint{-0.048611in}{0.000000in}}%
\pgfusepath{stroke,fill}%
}%
\begin{pgfscope}%
\pgfsys@transformshift{0.750000in}{1.768755in}%
\pgfsys@useobject{currentmarker}{}%
\end{pgfscope}%
\end{pgfscope}%
\begin{pgfscope}%
\definecolor{textcolor}{rgb}{0.000000,0.000000,0.000000}%
\pgfsetstrokecolor{textcolor}%
\pgfsetfillcolor{textcolor}%
\pgftext[x=0.405863in, y=1.717655in, left, base]{\color{textcolor}\rmfamily\fontsize{10.000000}{12.000000}\selectfont \(\displaystyle {1.50}\)}%
\end{pgfscope}%
\begin{pgfscope}%
\pgfpathrectangle{\pgfqpoint{0.750000in}{0.880000in}}{\pgfqpoint{4.650000in}{3.080000in}}%
\pgfusepath{clip}%
\pgfsetrectcap%
\pgfsetroundjoin%
\pgfsetlinewidth{0.803000pt}%
\definecolor{currentstroke}{rgb}{0.690196,0.690196,0.690196}%
\pgfsetstrokecolor{currentstroke}%
\pgfsetdash{}{0pt}%
\pgfpathmoveto{\pgfqpoint{0.750000in}{1.426373in}}%
\pgfpathlineto{\pgfqpoint{5.400000in}{1.426373in}}%
\pgfusepath{stroke}%
\end{pgfscope}%
\begin{pgfscope}%
\pgfsetbuttcap%
\pgfsetroundjoin%
\definecolor{currentfill}{rgb}{0.000000,0.000000,0.000000}%
\pgfsetfillcolor{currentfill}%
\pgfsetlinewidth{0.803000pt}%
\definecolor{currentstroke}{rgb}{0.000000,0.000000,0.000000}%
\pgfsetstrokecolor{currentstroke}%
\pgfsetdash{}{0pt}%
\pgfsys@defobject{currentmarker}{\pgfqpoint{-0.048611in}{0.000000in}}{\pgfqpoint{-0.000000in}{0.000000in}}{%
\pgfpathmoveto{\pgfqpoint{-0.000000in}{0.000000in}}%
\pgfpathlineto{\pgfqpoint{-0.048611in}{0.000000in}}%
\pgfusepath{stroke,fill}%
}%
\begin{pgfscope}%
\pgfsys@transformshift{0.750000in}{1.426373in}%
\pgfsys@useobject{currentmarker}{}%
\end{pgfscope}%
\end{pgfscope}%
\begin{pgfscope}%
\definecolor{textcolor}{rgb}{0.000000,0.000000,0.000000}%
\pgfsetstrokecolor{textcolor}%
\pgfsetfillcolor{textcolor}%
\pgftext[x=0.405863in, y=1.375273in, left, base]{\color{textcolor}\rmfamily\fontsize{10.000000}{12.000000}\selectfont \(\displaystyle {1.75}\)}%
\end{pgfscope}%
\begin{pgfscope}%
\pgfpathrectangle{\pgfqpoint{0.750000in}{0.880000in}}{\pgfqpoint{4.650000in}{3.080000in}}%
\pgfusepath{clip}%
\pgfsetrectcap%
\pgfsetroundjoin%
\pgfsetlinewidth{0.803000pt}%
\definecolor{currentstroke}{rgb}{0.690196,0.690196,0.690196}%
\pgfsetstrokecolor{currentstroke}%
\pgfsetdash{}{0pt}%
\pgfpathmoveto{\pgfqpoint{0.750000in}{1.083991in}}%
\pgfpathlineto{\pgfqpoint{5.400000in}{1.083991in}}%
\pgfusepath{stroke}%
\end{pgfscope}%
\begin{pgfscope}%
\pgfsetbuttcap%
\pgfsetroundjoin%
\definecolor{currentfill}{rgb}{0.000000,0.000000,0.000000}%
\pgfsetfillcolor{currentfill}%
\pgfsetlinewidth{0.803000pt}%
\definecolor{currentstroke}{rgb}{0.000000,0.000000,0.000000}%
\pgfsetstrokecolor{currentstroke}%
\pgfsetdash{}{0pt}%
\pgfsys@defobject{currentmarker}{\pgfqpoint{-0.048611in}{0.000000in}}{\pgfqpoint{-0.000000in}{0.000000in}}{%
\pgfpathmoveto{\pgfqpoint{-0.000000in}{0.000000in}}%
\pgfpathlineto{\pgfqpoint{-0.048611in}{0.000000in}}%
\pgfusepath{stroke,fill}%
}%
\begin{pgfscope}%
\pgfsys@transformshift{0.750000in}{1.083991in}%
\pgfsys@useobject{currentmarker}{}%
\end{pgfscope}%
\end{pgfscope}%
\begin{pgfscope}%
\definecolor{textcolor}{rgb}{0.000000,0.000000,0.000000}%
\pgfsetstrokecolor{textcolor}%
\pgfsetfillcolor{textcolor}%
\pgftext[x=0.405863in, y=1.032891in, left, base]{\color{textcolor}\rmfamily\fontsize{10.000000}{12.000000}\selectfont \(\displaystyle {2.00}\)}%
\end{pgfscope}%
\begin{pgfscope}%
\definecolor{textcolor}{rgb}{0.000000,0.000000,0.000000}%
\pgfsetstrokecolor{textcolor}%
\pgfsetfillcolor{textcolor}%
\pgftext[x=0.350308in,y=2.420000in,,bottom,rotate=90.000000]{\color{textcolor}\rmfamily\fontsize{10.000000}{12.000000}\selectfont time in s}%
\end{pgfscope}%
\begin{pgfscope}%
\pgfpathrectangle{\pgfqpoint{0.750000in}{0.880000in}}{\pgfqpoint{4.650000in}{3.080000in}}%
\pgfusepath{clip}%
\pgfsetrectcap%
\pgfsetroundjoin%
\pgfsetlinewidth{1.505625pt}%
\definecolor{currentstroke}{rgb}{0.121569,0.466667,0.705882}%
\pgfsetstrokecolor{currentstroke}%
\pgfsetdash{}{0pt}%
\pgfpathmoveto{\pgfqpoint{2.005414in}{3.970000in}}%
\pgfpathlineto{\pgfqpoint{2.006624in}{3.814830in}}%
\pgfpathlineto{\pgfqpoint{2.010047in}{3.806613in}}%
\pgfpathlineto{\pgfqpoint{2.015748in}{3.798396in}}%
\pgfpathlineto{\pgfqpoint{2.025278in}{3.788809in}}%
\pgfpathlineto{\pgfqpoint{2.037905in}{3.779222in}}%
\pgfpathlineto{\pgfqpoint{2.056123in}{3.768266in}}%
\pgfpathlineto{\pgfqpoint{2.081442in}{3.755940in}}%
\pgfpathlineto{\pgfqpoint{2.111863in}{3.743615in}}%
\pgfpathlineto{\pgfqpoint{2.151640in}{3.729919in}}%
\pgfpathlineto{\pgfqpoint{2.202652in}{3.714854in}}%
\pgfpathlineto{\pgfqpoint{2.266957in}{3.698420in}}%
\pgfpathlineto{\pgfqpoint{2.340282in}{3.681986in}}%
\pgfpathlineto{\pgfqpoint{2.436865in}{3.662812in}}%
\pgfpathlineto{\pgfqpoint{2.623660in}{3.625835in}}%
\pgfpathlineto{\pgfqpoint{2.778737in}{3.592966in}}%
\pgfpathlineto{\pgfqpoint{2.910687in}{3.562837in}}%
\pgfpathlineto{\pgfqpoint{3.027011in}{3.534077in}}%
\pgfpathlineto{\pgfqpoint{3.128871in}{3.506686in}}%
\pgfpathlineto{\pgfqpoint{3.217594in}{3.480665in}}%
\pgfpathlineto{\pgfqpoint{3.298616in}{3.454644in}}%
\pgfpathlineto{\pgfqpoint{3.368474in}{3.429993in}}%
\pgfpathlineto{\pgfqpoint{3.431861in}{3.405341in}}%
\pgfpathlineto{\pgfqpoint{3.489046in}{3.380690in}}%
\pgfpathlineto{\pgfqpoint{3.537618in}{3.357408in}}%
\pgfpathlineto{\pgfqpoint{3.581159in}{3.334126in}}%
\pgfpathlineto{\pgfqpoint{3.619911in}{3.310844in}}%
\pgfpathlineto{\pgfqpoint{3.652217in}{3.288931in}}%
\pgfpathlineto{\pgfqpoint{3.680670in}{3.267019in}}%
\pgfpathlineto{\pgfqpoint{3.705440in}{3.245106in}}%
\pgfpathlineto{\pgfqpoint{3.725454in}{3.224563in}}%
\pgfpathlineto{\pgfqpoint{3.742482in}{3.204021in}}%
\pgfpathlineto{\pgfqpoint{3.756622in}{3.183478in}}%
\pgfpathlineto{\pgfqpoint{3.767961in}{3.162935in}}%
\pgfpathlineto{\pgfqpoint{3.776080in}{3.143761in}}%
\pgfpathlineto{\pgfqpoint{3.781866in}{3.124588in}}%
\pgfpathlineto{\pgfqpoint{3.785354in}{3.105415in}}%
\pgfpathlineto{\pgfqpoint{3.786565in}{3.086241in}}%
\pgfpathlineto{\pgfqpoint{3.785511in}{3.067068in}}%
\pgfpathlineto{\pgfqpoint{3.782190in}{3.047894in}}%
\pgfpathlineto{\pgfqpoint{3.776589in}{3.028721in}}%
\pgfpathlineto{\pgfqpoint{3.768685in}{3.009548in}}%
\pgfpathlineto{\pgfqpoint{3.758442in}{2.990374in}}%
\pgfpathlineto{\pgfqpoint{3.744816in}{2.969831in}}%
\pgfpathlineto{\pgfqpoint{3.728380in}{2.949288in}}%
\pgfpathlineto{\pgfqpoint{3.709047in}{2.928745in}}%
\pgfpathlineto{\pgfqpoint{3.686719in}{2.908202in}}%
\pgfpathlineto{\pgfqpoint{3.659476in}{2.886290in}}%
\pgfpathlineto{\pgfqpoint{3.628550in}{2.864378in}}%
\pgfpathlineto{\pgfqpoint{3.593780in}{2.842465in}}%
\pgfpathlineto{\pgfqpoint{3.552434in}{2.819183in}}%
\pgfpathlineto{\pgfqpoint{3.506341in}{2.795901in}}%
\pgfpathlineto{\pgfqpoint{3.455283in}{2.772619in}}%
\pgfpathlineto{\pgfqpoint{3.395571in}{2.747968in}}%
\pgfpathlineto{\pgfqpoint{3.329807in}{2.723316in}}%
\pgfpathlineto{\pgfqpoint{3.257774in}{2.698665in}}%
\pgfpathlineto{\pgfqpoint{3.174741in}{2.672644in}}%
\pgfpathlineto{\pgfqpoint{3.084381in}{2.646623in}}%
\pgfpathlineto{\pgfqpoint{2.981301in}{2.619232in}}%
\pgfpathlineto{\pgfqpoint{2.864380in}{2.590472in}}%
\pgfpathlineto{\pgfqpoint{2.726510in}{2.558973in}}%
\pgfpathlineto{\pgfqpoint{2.565967in}{2.524735in}}%
\pgfpathlineto{\pgfqpoint{2.375007in}{2.486388in}}%
\pgfpathlineto{\pgfqpoint{2.110451in}{2.435715in}}%
\pgfpathlineto{\pgfqpoint{1.659224in}{2.349435in}}%
\pgfpathlineto{\pgfqpoint{1.493083in}{2.315197in}}%
\pgfpathlineto{\pgfqpoint{1.370548in}{2.287806in}}%
\pgfpathlineto{\pgfqpoint{1.270379in}{2.263155in}}%
\pgfpathlineto{\pgfqpoint{1.190806in}{2.241242in}}%
\pgfpathlineto{\pgfqpoint{1.129377in}{2.222069in}}%
\pgfpathlineto{\pgfqpoint{1.079783in}{2.204265in}}%
\pgfpathlineto{\pgfqpoint{1.040772in}{2.187831in}}%
\pgfpathlineto{\pgfqpoint{1.010987in}{2.172766in}}%
\pgfpathlineto{\pgfqpoint{0.989038in}{2.159071in}}%
\pgfpathlineto{\pgfqpoint{0.973561in}{2.146745in}}%
\pgfpathlineto{\pgfqpoint{0.963261in}{2.135789in}}%
\pgfpathlineto{\pgfqpoint{0.956947in}{2.126202in}}%
\pgfpathlineto{\pgfqpoint{0.953166in}{2.116615in}}%
\pgfpathlineto{\pgfqpoint{0.951928in}{2.107029in}}%
\pgfpathlineto{\pgfqpoint{0.953232in}{2.097442in}}%
\pgfpathlineto{\pgfqpoint{0.957074in}{2.087855in}}%
\pgfpathlineto{\pgfqpoint{0.963439in}{2.078269in}}%
\pgfpathlineto{\pgfqpoint{0.973779in}{2.067312in}}%
\pgfpathlineto{\pgfqpoint{0.987347in}{2.056356in}}%
\pgfpathlineto{\pgfqpoint{1.006406in}{2.044030in}}%
\pgfpathlineto{\pgfqpoint{1.032186in}{2.030335in}}%
\pgfpathlineto{\pgfqpoint{1.065964in}{2.015270in}}%
\pgfpathlineto{\pgfqpoint{1.109016in}{1.998836in}}%
\pgfpathlineto{\pgfqpoint{1.162560in}{1.981032in}}%
\pgfpathlineto{\pgfqpoint{1.227677in}{1.961859in}}%
\pgfpathlineto{\pgfqpoint{1.310670in}{1.939946in}}%
\pgfpathlineto{\pgfqpoint{1.413579in}{1.915295in}}%
\pgfpathlineto{\pgfqpoint{1.544135in}{1.886535in}}%
\pgfpathlineto{\pgfqpoint{1.717257in}{1.850927in}}%
\pgfpathlineto{\pgfqpoint{2.003464in}{1.794776in}}%
\pgfpathlineto{\pgfqpoint{2.335879in}{1.729039in}}%
\pgfpathlineto{\pgfqpoint{2.525224in}{1.689323in}}%
\pgfpathlineto{\pgfqpoint{2.677693in}{1.655084in}}%
\pgfpathlineto{\pgfqpoint{2.802179in}{1.624955in}}%
\pgfpathlineto{\pgfqpoint{2.911810in}{1.596195in}}%
\pgfpathlineto{\pgfqpoint{3.007508in}{1.568804in}}%
\pgfpathlineto{\pgfqpoint{3.090406in}{1.542783in}}%
\pgfpathlineto{\pgfqpoint{3.161722in}{1.518132in}}%
\pgfpathlineto{\pgfqpoint{3.222682in}{1.494850in}}%
\pgfpathlineto{\pgfqpoint{3.277525in}{1.471568in}}%
\pgfpathlineto{\pgfqpoint{3.323659in}{1.449655in}}%
\pgfpathlineto{\pgfqpoint{3.364594in}{1.427743in}}%
\pgfpathlineto{\pgfqpoint{3.398353in}{1.407200in}}%
\pgfpathlineto{\pgfqpoint{3.427743in}{1.386657in}}%
\pgfpathlineto{\pgfqpoint{3.451317in}{1.367483in}}%
\pgfpathlineto{\pgfqpoint{3.469935in}{1.349680in}}%
\pgfpathlineto{\pgfqpoint{3.485456in}{1.331876in}}%
\pgfpathlineto{\pgfqpoint{3.497925in}{1.314072in}}%
\pgfpathlineto{\pgfqpoint{3.506761in}{1.297638in}}%
\pgfpathlineto{\pgfqpoint{3.513054in}{1.281203in}}%
\pgfpathlineto{\pgfqpoint{3.516825in}{1.264769in}}%
\pgfpathlineto{\pgfqpoint{3.518076in}{1.249704in}}%
\pgfpathlineto{\pgfqpoint{3.517224in}{1.234639in}}%
\pgfpathlineto{\pgfqpoint{3.514270in}{1.219574in}}%
\pgfpathlineto{\pgfqpoint{3.508647in}{1.203140in}}%
\pgfpathlineto{\pgfqpoint{3.500510in}{1.186706in}}%
\pgfpathlineto{\pgfqpoint{3.489844in}{1.170271in}}%
\pgfpathlineto{\pgfqpoint{3.476628in}{1.153837in}}%
\pgfpathlineto{\pgfqpoint{3.459405in}{1.136033in}}%
\pgfpathlineto{\pgfqpoint{3.439121in}{1.118229in}}%
\pgfpathlineto{\pgfqpoint{3.413806in}{1.099056in}}%
\pgfpathlineto{\pgfqpoint{3.393487in}{1.085361in}}%
\pgfpathlineto{\pgfqpoint{3.393487in}{1.085361in}}%
\pgfusepath{stroke}%
\end{pgfscope}%
\begin{pgfscope}%
\pgfpathrectangle{\pgfqpoint{0.750000in}{0.880000in}}{\pgfqpoint{4.650000in}{3.080000in}}%
\pgfusepath{clip}%
\pgfsetbuttcap%
\pgfsetroundjoin%
\pgfsetlinewidth{1.505625pt}%
\definecolor{currentstroke}{rgb}{0.121569,0.466667,0.705882}%
\pgfsetstrokecolor{currentstroke}%
\pgfsetdash{{5.550000pt}{2.400000pt}}{0.000000pt}%
\pgfpathmoveto{\pgfqpoint{0.750000in}{3.660073in}}%
\pgfpathlineto{\pgfqpoint{5.400000in}{3.660073in}}%
\pgfusepath{stroke}%
\end{pgfscope}%
\begin{pgfscope}%
\pgfsetrectcap%
\pgfsetmiterjoin%
\pgfsetlinewidth{0.803000pt}%
\definecolor{currentstroke}{rgb}{0.000000,0.000000,0.000000}%
\pgfsetstrokecolor{currentstroke}%
\pgfsetdash{}{0pt}%
\pgfpathmoveto{\pgfqpoint{0.750000in}{0.880000in}}%
\pgfpathlineto{\pgfqpoint{0.750000in}{3.960000in}}%
\pgfusepath{stroke}%
\end{pgfscope}%
\begin{pgfscope}%
\pgfsetrectcap%
\pgfsetmiterjoin%
\pgfsetlinewidth{0.803000pt}%
\definecolor{currentstroke}{rgb}{0.000000,0.000000,0.000000}%
\pgfsetstrokecolor{currentstroke}%
\pgfsetdash{}{0pt}%
\pgfpathmoveto{\pgfqpoint{5.400000in}{0.880000in}}%
\pgfpathlineto{\pgfqpoint{5.400000in}{3.960000in}}%
\pgfusepath{stroke}%
\end{pgfscope}%
\begin{pgfscope}%
\pgfsetrectcap%
\pgfsetmiterjoin%
\pgfsetlinewidth{0.803000pt}%
\definecolor{currentstroke}{rgb}{0.000000,0.000000,0.000000}%
\pgfsetstrokecolor{currentstroke}%
\pgfsetdash{}{0pt}%
\pgfpathmoveto{\pgfqpoint{0.750000in}{0.880000in}}%
\pgfpathlineto{\pgfqpoint{5.400000in}{0.880000in}}%
\pgfusepath{stroke}%
\end{pgfscope}%
\begin{pgfscope}%
\pgfsetrectcap%
\pgfsetmiterjoin%
\pgfsetlinewidth{0.803000pt}%
\definecolor{currentstroke}{rgb}{0.000000,0.000000,0.000000}%
\pgfsetstrokecolor{currentstroke}%
\pgfsetdash{}{0pt}%
\pgfpathmoveto{\pgfqpoint{0.750000in}{3.960000in}}%
\pgfpathlineto{\pgfqpoint{5.400000in}{3.960000in}}%
\pgfusepath{stroke}%
\end{pgfscope}%
\begin{pgfscope}%
\pgfsetbuttcap%
\pgfsetmiterjoin%
\definecolor{currentfill}{rgb}{1.000000,1.000000,1.000000}%
\pgfsetfillcolor{currentfill}%
\pgfsetfillopacity{0.800000}%
\pgfsetlinewidth{1.003750pt}%
\definecolor{currentstroke}{rgb}{0.800000,0.800000,0.800000}%
\pgfsetstrokecolor{currentstroke}%
\pgfsetstrokeopacity{0.800000}%
\pgfsetdash{}{0pt}%
\pgfpathmoveto{\pgfqpoint{0.847222in}{0.949444in}}%
\pgfpathlineto{\pgfqpoint{2.250936in}{0.949444in}}%
\pgfpathquadraticcurveto{\pgfqpoint{2.278714in}{0.949444in}}{\pgfqpoint{2.278714in}{0.977222in}}%
\pgfpathlineto{\pgfqpoint{2.278714in}{1.368267in}}%
\pgfpathquadraticcurveto{\pgfqpoint{2.278714in}{1.396045in}}{\pgfqpoint{2.250936in}{1.396045in}}%
\pgfpathlineto{\pgfqpoint{0.847222in}{1.396045in}}%
\pgfpathquadraticcurveto{\pgfqpoint{0.819444in}{1.396045in}}{\pgfqpoint{0.819444in}{1.368267in}}%
\pgfpathlineto{\pgfqpoint{0.819444in}{0.977222in}}%
\pgfpathquadraticcurveto{\pgfqpoint{0.819444in}{0.949444in}}{\pgfqpoint{0.847222in}{0.949444in}}%
\pgfpathlineto{\pgfqpoint{0.847222in}{0.949444in}}%
\pgfpathclose%
\pgfusepath{stroke,fill}%
\end{pgfscope}%
\begin{pgfscope}%
\pgfsetrectcap%
\pgfsetroundjoin%
\pgfsetlinewidth{1.505625pt}%
\definecolor{currentstroke}{rgb}{0.121569,0.466667,0.705882}%
\pgfsetstrokecolor{currentstroke}%
\pgfsetdash{}{0pt}%
\pgfpathmoveto{\pgfqpoint{0.875000in}{1.286901in}}%
\pgfpathlineto{\pgfqpoint{1.013889in}{1.286901in}}%
\pgfpathlineto{\pgfqpoint{1.152778in}{1.286901in}}%
\pgfusepath{stroke}%
\end{pgfscope}%
\begin{pgfscope}%
\definecolor{textcolor}{rgb}{0.000000,0.000000,0.000000}%
\pgfsetstrokecolor{textcolor}%
\pgfsetfillcolor{textcolor}%
\pgftext[x=1.263889in,y=1.238289in,left,base]{\color{textcolor}\rmfamily\fontsize{10.000000}{12.000000}\selectfont delta}%
\end{pgfscope}%
\begin{pgfscope}%
\pgfsetbuttcap%
\pgfsetroundjoin%
\pgfsetlinewidth{1.505625pt}%
\definecolor{currentstroke}{rgb}{0.121569,0.466667,0.705882}%
\pgfsetstrokecolor{currentstroke}%
\pgfsetdash{{5.550000pt}{2.400000pt}}{0.000000pt}%
\pgfpathmoveto{\pgfqpoint{0.875000in}{1.083857in}}%
\pgfpathlineto{\pgfqpoint{1.013889in}{1.083857in}}%
\pgfpathlineto{\pgfqpoint{1.152778in}{1.083857in}}%
\pgfusepath{stroke}%
\end{pgfscope}%
\begin{pgfscope}%
\definecolor{textcolor}{rgb}{0.000000,0.000000,0.000000}%
\pgfsetstrokecolor{textcolor}%
\pgfsetfillcolor{textcolor}%
\pgftext[x=1.263889in,y=1.035246in,left,base]{\color{textcolor}\rmfamily\fontsize{10.000000}{12.000000}\selectfont clearing of fault}%
\end{pgfscope}%
\begin{pgfscope}%
\definecolor{textcolor}{rgb}{0.000000,0.000000,0.000000}%
\pgfsetstrokecolor{textcolor}%
\pgfsetfillcolor{textcolor}%
\pgftext[x=3.000000in,y=7.840000in,,top]{\color{textcolor}\rmfamily\fontsize{12.000000}{14.400000}\selectfont Stable scenario - fault 1}%
\end{pgfscope}%
\end{pgfpicture}%
\makeatother%
\endgroup%


%% Creator: Matplotlib, PGF backend
%%
%% To include the figure in your LaTeX document, write
%%   \input{<filename>.pgf}
%%
%% Make sure the required packages are loaded in your preamble
%%   \usepackage{pgf}
%%
%% Also ensure that all the required font packages are loaded; for instance,
%% the lmodern package is sometimes necessary when using math font.
%%   \usepackage{lmodern}
%%
%% Figures using additional raster images can only be included by \input if
%% they are in the same directory as the main LaTeX file. For loading figures
%% from other directories you can use the `import` package
%%   \usepackage{import}
%%
%% and then include the figures with
%%   \import{<path to file>}{<filename>.pgf}
%%
%% Matplotlib used the following preamble
%%   
%%   \usepackage{fontspec}
%%   \setmainfont{Charter.ttc}[Path=\detokenize{/System/Library/Fonts/Supplemental/}]
%%   \setsansfont{DejaVuSans.ttf}[Path=\detokenize{/opt/homebrew/lib/python3.10/site-packages/matplotlib/mpl-data/fonts/ttf/}]
%%   \setmonofont{DejaVuSansMono.ttf}[Path=\detokenize{/opt/homebrew/lib/python3.10/site-packages/matplotlib/mpl-data/fonts/ttf/}]
%%   \makeatletter\@ifpackageloaded{underscore}{}{\usepackage[strings]{underscore}}\makeatother
%%
\begingroup%
\makeatletter%
\begin{pgfpicture}%
\pgfpathrectangle{\pgfpointorigin}{\pgfqpoint{6.000000in}{8.000000in}}%
\pgfusepath{use as bounding box, clip}%
\begin{pgfscope}%
\pgfsetbuttcap%
\pgfsetmiterjoin%
\definecolor{currentfill}{rgb}{1.000000,1.000000,1.000000}%
\pgfsetfillcolor{currentfill}%
\pgfsetlinewidth{0.000000pt}%
\definecolor{currentstroke}{rgb}{1.000000,1.000000,1.000000}%
\pgfsetstrokecolor{currentstroke}%
\pgfsetdash{}{0pt}%
\pgfpathmoveto{\pgfqpoint{0.000000in}{0.000000in}}%
\pgfpathlineto{\pgfqpoint{6.000000in}{0.000000in}}%
\pgfpathlineto{\pgfqpoint{6.000000in}{8.000000in}}%
\pgfpathlineto{\pgfqpoint{0.000000in}{8.000000in}}%
\pgfpathlineto{\pgfqpoint{0.000000in}{0.000000in}}%
\pgfpathclose%
\pgfusepath{fill}%
\end{pgfscope}%
\begin{pgfscope}%
\pgfsetbuttcap%
\pgfsetmiterjoin%
\definecolor{currentfill}{rgb}{1.000000,1.000000,1.000000}%
\pgfsetfillcolor{currentfill}%
\pgfsetlinewidth{0.000000pt}%
\definecolor{currentstroke}{rgb}{0.000000,0.000000,0.000000}%
\pgfsetstrokecolor{currentstroke}%
\pgfsetstrokeopacity{0.000000}%
\pgfsetdash{}{0pt}%
\pgfpathmoveto{\pgfqpoint{0.750000in}{3.960000in}}%
\pgfpathlineto{\pgfqpoint{5.400000in}{3.960000in}}%
\pgfpathlineto{\pgfqpoint{5.400000in}{7.040000in}}%
\pgfpathlineto{\pgfqpoint{0.750000in}{7.040000in}}%
\pgfpathlineto{\pgfqpoint{0.750000in}{3.960000in}}%
\pgfpathclose%
\pgfusepath{fill}%
\end{pgfscope}%
\begin{pgfscope}%
\pgfpathrectangle{\pgfqpoint{0.750000in}{3.960000in}}{\pgfqpoint{4.650000in}{3.080000in}}%
\pgfusepath{clip}%
\pgfsetbuttcap%
\pgfsetroundjoin%
\definecolor{currentfill}{rgb}{0.900000,0.900000,0.900000}%
\pgfsetfillcolor{currentfill}%
\pgfsetlinewidth{1.003750pt}%
\definecolor{currentstroke}{rgb}{0.500000,0.500000,0.500000}%
\pgfsetstrokecolor{currentstroke}%
\pgfsetdash{}{0pt}%
\pgfsys@defobject{currentmarker}{\pgfqpoint{2.064917in}{3.960080in}}{\pgfqpoint{2.440004in}{6.240196in}}{%
\pgfpathmoveto{\pgfqpoint{2.064917in}{6.240196in}}%
\pgfpathlineto{\pgfqpoint{2.064917in}{3.960084in}}%
\pgfpathlineto{\pgfqpoint{2.072572in}{3.960084in}}%
\pgfpathlineto{\pgfqpoint{2.080226in}{3.960084in}}%
\pgfpathlineto{\pgfqpoint{2.087881in}{3.960084in}}%
\pgfpathlineto{\pgfqpoint{2.095536in}{3.960084in}}%
\pgfpathlineto{\pgfqpoint{2.103191in}{3.960084in}}%
\pgfpathlineto{\pgfqpoint{2.110846in}{3.960084in}}%
\pgfpathlineto{\pgfqpoint{2.118501in}{3.960084in}}%
\pgfpathlineto{\pgfqpoint{2.126155in}{3.960084in}}%
\pgfpathlineto{\pgfqpoint{2.133810in}{3.960084in}}%
\pgfpathlineto{\pgfqpoint{2.141465in}{3.960083in}}%
\pgfpathlineto{\pgfqpoint{2.149120in}{3.960083in}}%
\pgfpathlineto{\pgfqpoint{2.156775in}{3.960083in}}%
\pgfpathlineto{\pgfqpoint{2.164430in}{3.960083in}}%
\pgfpathlineto{\pgfqpoint{2.172084in}{3.960083in}}%
\pgfpathlineto{\pgfqpoint{2.179739in}{3.960083in}}%
\pgfpathlineto{\pgfqpoint{2.187394in}{3.960083in}}%
\pgfpathlineto{\pgfqpoint{2.195049in}{3.960083in}}%
\pgfpathlineto{\pgfqpoint{2.202704in}{3.960083in}}%
\pgfpathlineto{\pgfqpoint{2.210359in}{3.960083in}}%
\pgfpathlineto{\pgfqpoint{2.218013in}{3.960083in}}%
\pgfpathlineto{\pgfqpoint{2.225668in}{3.960083in}}%
\pgfpathlineto{\pgfqpoint{2.233323in}{3.960083in}}%
\pgfpathlineto{\pgfqpoint{2.240978in}{3.960082in}}%
\pgfpathlineto{\pgfqpoint{2.248633in}{3.960082in}}%
\pgfpathlineto{\pgfqpoint{2.256288in}{3.960082in}}%
\pgfpathlineto{\pgfqpoint{2.263942in}{3.960082in}}%
\pgfpathlineto{\pgfqpoint{2.271597in}{3.960082in}}%
\pgfpathlineto{\pgfqpoint{2.279252in}{3.960082in}}%
\pgfpathlineto{\pgfqpoint{2.286907in}{3.960082in}}%
\pgfpathlineto{\pgfqpoint{2.294562in}{3.960082in}}%
\pgfpathlineto{\pgfqpoint{2.302217in}{3.960082in}}%
\pgfpathlineto{\pgfqpoint{2.309871in}{3.960082in}}%
\pgfpathlineto{\pgfqpoint{2.317526in}{3.960082in}}%
\pgfpathlineto{\pgfqpoint{2.325181in}{3.960082in}}%
\pgfpathlineto{\pgfqpoint{2.332836in}{3.960081in}}%
\pgfpathlineto{\pgfqpoint{2.340491in}{3.960081in}}%
\pgfpathlineto{\pgfqpoint{2.348146in}{3.960081in}}%
\pgfpathlineto{\pgfqpoint{2.355800in}{3.960081in}}%
\pgfpathlineto{\pgfqpoint{2.363455in}{3.960081in}}%
\pgfpathlineto{\pgfqpoint{2.371110in}{3.960081in}}%
\pgfpathlineto{\pgfqpoint{2.378765in}{3.960081in}}%
\pgfpathlineto{\pgfqpoint{2.386420in}{3.960081in}}%
\pgfpathlineto{\pgfqpoint{2.394075in}{3.960081in}}%
\pgfpathlineto{\pgfqpoint{2.401729in}{3.960081in}}%
\pgfpathlineto{\pgfqpoint{2.409384in}{3.960081in}}%
\pgfpathlineto{\pgfqpoint{2.417039in}{3.960081in}}%
\pgfpathlineto{\pgfqpoint{2.424694in}{3.960080in}}%
\pgfpathlineto{\pgfqpoint{2.432349in}{3.960080in}}%
\pgfpathlineto{\pgfqpoint{2.440004in}{3.960080in}}%
\pgfpathlineto{\pgfqpoint{2.440004in}{6.240196in}}%
\pgfpathlineto{\pgfqpoint{2.440004in}{6.240196in}}%
\pgfpathlineto{\pgfqpoint{2.432349in}{6.240196in}}%
\pgfpathlineto{\pgfqpoint{2.424694in}{6.240196in}}%
\pgfpathlineto{\pgfqpoint{2.417039in}{6.240196in}}%
\pgfpathlineto{\pgfqpoint{2.409384in}{6.240196in}}%
\pgfpathlineto{\pgfqpoint{2.401729in}{6.240196in}}%
\pgfpathlineto{\pgfqpoint{2.394075in}{6.240196in}}%
\pgfpathlineto{\pgfqpoint{2.386420in}{6.240196in}}%
\pgfpathlineto{\pgfqpoint{2.378765in}{6.240196in}}%
\pgfpathlineto{\pgfqpoint{2.371110in}{6.240196in}}%
\pgfpathlineto{\pgfqpoint{2.363455in}{6.240196in}}%
\pgfpathlineto{\pgfqpoint{2.355800in}{6.240196in}}%
\pgfpathlineto{\pgfqpoint{2.348146in}{6.240196in}}%
\pgfpathlineto{\pgfqpoint{2.340491in}{6.240196in}}%
\pgfpathlineto{\pgfqpoint{2.332836in}{6.240196in}}%
\pgfpathlineto{\pgfqpoint{2.325181in}{6.240196in}}%
\pgfpathlineto{\pgfqpoint{2.317526in}{6.240196in}}%
\pgfpathlineto{\pgfqpoint{2.309871in}{6.240196in}}%
\pgfpathlineto{\pgfqpoint{2.302217in}{6.240196in}}%
\pgfpathlineto{\pgfqpoint{2.294562in}{6.240196in}}%
\pgfpathlineto{\pgfqpoint{2.286907in}{6.240196in}}%
\pgfpathlineto{\pgfqpoint{2.279252in}{6.240196in}}%
\pgfpathlineto{\pgfqpoint{2.271597in}{6.240196in}}%
\pgfpathlineto{\pgfqpoint{2.263942in}{6.240196in}}%
\pgfpathlineto{\pgfqpoint{2.256288in}{6.240196in}}%
\pgfpathlineto{\pgfqpoint{2.248633in}{6.240196in}}%
\pgfpathlineto{\pgfqpoint{2.240978in}{6.240196in}}%
\pgfpathlineto{\pgfqpoint{2.233323in}{6.240196in}}%
\pgfpathlineto{\pgfqpoint{2.225668in}{6.240196in}}%
\pgfpathlineto{\pgfqpoint{2.218013in}{6.240196in}}%
\pgfpathlineto{\pgfqpoint{2.210359in}{6.240196in}}%
\pgfpathlineto{\pgfqpoint{2.202704in}{6.240196in}}%
\pgfpathlineto{\pgfqpoint{2.195049in}{6.240196in}}%
\pgfpathlineto{\pgfqpoint{2.187394in}{6.240196in}}%
\pgfpathlineto{\pgfqpoint{2.179739in}{6.240196in}}%
\pgfpathlineto{\pgfqpoint{2.172084in}{6.240196in}}%
\pgfpathlineto{\pgfqpoint{2.164430in}{6.240196in}}%
\pgfpathlineto{\pgfqpoint{2.156775in}{6.240196in}}%
\pgfpathlineto{\pgfqpoint{2.149120in}{6.240196in}}%
\pgfpathlineto{\pgfqpoint{2.141465in}{6.240196in}}%
\pgfpathlineto{\pgfqpoint{2.133810in}{6.240196in}}%
\pgfpathlineto{\pgfqpoint{2.126155in}{6.240196in}}%
\pgfpathlineto{\pgfqpoint{2.118501in}{6.240196in}}%
\pgfpathlineto{\pgfqpoint{2.110846in}{6.240196in}}%
\pgfpathlineto{\pgfqpoint{2.103191in}{6.240196in}}%
\pgfpathlineto{\pgfqpoint{2.095536in}{6.240196in}}%
\pgfpathlineto{\pgfqpoint{2.087881in}{6.240196in}}%
\pgfpathlineto{\pgfqpoint{2.080226in}{6.240196in}}%
\pgfpathlineto{\pgfqpoint{2.072572in}{6.240196in}}%
\pgfpathlineto{\pgfqpoint{2.064917in}{6.240196in}}%
\pgfpathlineto{\pgfqpoint{2.064917in}{6.240196in}}%
\pgfpathclose%
\pgfusepath{stroke,fill}%
}%
\begin{pgfscope}%
\pgfsys@transformshift{0.000000in}{0.000000in}%
\pgfsys@useobject{currentmarker}{}%
\end{pgfscope}%
\end{pgfscope}%
\begin{pgfscope}%
\pgfpathrectangle{\pgfqpoint{0.750000in}{3.960000in}}{\pgfqpoint{4.650000in}{3.080000in}}%
\pgfusepath{clip}%
\pgfsetbuttcap%
\pgfsetroundjoin%
\definecolor{currentfill}{rgb}{0.900000,0.900000,0.900000}%
\pgfsetfillcolor{currentfill}%
\pgfsetlinewidth{1.003750pt}%
\definecolor{currentstroke}{rgb}{0.500000,0.500000,0.500000}%
\pgfsetstrokecolor{currentstroke}%
\pgfsetdash{}{0pt}%
\pgfsys@defobject{currentmarker}{\pgfqpoint{2.440004in}{6.237573in}}{\pgfqpoint{4.085083in}{6.894836in}}{%
\pgfpathmoveto{\pgfqpoint{2.440004in}{6.240196in}}%
\pgfpathlineto{\pgfqpoint{2.440004in}{6.628879in}}%
\pgfpathlineto{\pgfqpoint{2.473577in}{6.655881in}}%
\pgfpathlineto{\pgfqpoint{2.507150in}{6.681496in}}%
\pgfpathlineto{\pgfqpoint{2.540723in}{6.705711in}}%
\pgfpathlineto{\pgfqpoint{2.574296in}{6.728513in}}%
\pgfpathlineto{\pgfqpoint{2.607869in}{6.749891in}}%
\pgfpathlineto{\pgfqpoint{2.641442in}{6.769834in}}%
\pgfpathlineto{\pgfqpoint{2.675015in}{6.788331in}}%
\pgfpathlineto{\pgfqpoint{2.708588in}{6.805373in}}%
\pgfpathlineto{\pgfqpoint{2.742161in}{6.820951in}}%
\pgfpathlineto{\pgfqpoint{2.775734in}{6.835058in}}%
\pgfpathlineto{\pgfqpoint{2.809307in}{6.847685in}}%
\pgfpathlineto{\pgfqpoint{2.842880in}{6.858826in}}%
\pgfpathlineto{\pgfqpoint{2.876453in}{6.868477in}}%
\pgfpathlineto{\pgfqpoint{2.910026in}{6.876630in}}%
\pgfpathlineto{\pgfqpoint{2.943599in}{6.883284in}}%
\pgfpathlineto{\pgfqpoint{2.977173in}{6.888433in}}%
\pgfpathlineto{\pgfqpoint{3.010746in}{6.892076in}}%
\pgfpathlineto{\pgfqpoint{3.044319in}{6.894211in}}%
\pgfpathlineto{\pgfqpoint{3.077892in}{6.894836in}}%
\pgfpathlineto{\pgfqpoint{3.111465in}{6.893951in}}%
\pgfpathlineto{\pgfqpoint{3.145038in}{6.891556in}}%
\pgfpathlineto{\pgfqpoint{3.178611in}{6.887654in}}%
\pgfpathlineto{\pgfqpoint{3.212184in}{6.882245in}}%
\pgfpathlineto{\pgfqpoint{3.245757in}{6.875333in}}%
\pgfpathlineto{\pgfqpoint{3.279330in}{6.866921in}}%
\pgfpathlineto{\pgfqpoint{3.312903in}{6.857013in}}%
\pgfpathlineto{\pgfqpoint{3.346476in}{6.845615in}}%
\pgfpathlineto{\pgfqpoint{3.380049in}{6.832733in}}%
\pgfpathlineto{\pgfqpoint{3.413622in}{6.818372in}}%
\pgfpathlineto{\pgfqpoint{3.447195in}{6.802541in}}%
\pgfpathlineto{\pgfqpoint{3.480768in}{6.785248in}}%
\pgfpathlineto{\pgfqpoint{3.514341in}{6.766501in}}%
\pgfpathlineto{\pgfqpoint{3.547914in}{6.746311in}}%
\pgfpathlineto{\pgfqpoint{3.581488in}{6.724686in}}%
\pgfpathlineto{\pgfqpoint{3.615061in}{6.701640in}}%
\pgfpathlineto{\pgfqpoint{3.648634in}{6.677183in}}%
\pgfpathlineto{\pgfqpoint{3.682207in}{6.651328in}}%
\pgfpathlineto{\pgfqpoint{3.715780in}{6.624089in}}%
\pgfpathlineto{\pgfqpoint{3.749353in}{6.595479in}}%
\pgfpathlineto{\pgfqpoint{3.782926in}{6.565513in}}%
\pgfpathlineto{\pgfqpoint{3.816499in}{6.534206in}}%
\pgfpathlineto{\pgfqpoint{3.850072in}{6.501576in}}%
\pgfpathlineto{\pgfqpoint{3.883645in}{6.467637in}}%
\pgfpathlineto{\pgfqpoint{3.917218in}{6.432409in}}%
\pgfpathlineto{\pgfqpoint{3.950791in}{6.395909in}}%
\pgfpathlineto{\pgfqpoint{3.984364in}{6.358155in}}%
\pgfpathlineto{\pgfqpoint{4.017937in}{6.319168in}}%
\pgfpathlineto{\pgfqpoint{4.051510in}{6.278967in}}%
\pgfpathlineto{\pgfqpoint{4.085083in}{6.237573in}}%
\pgfpathlineto{\pgfqpoint{4.085083in}{6.240196in}}%
\pgfpathlineto{\pgfqpoint{4.085083in}{6.240196in}}%
\pgfpathlineto{\pgfqpoint{4.051510in}{6.240196in}}%
\pgfpathlineto{\pgfqpoint{4.017937in}{6.240196in}}%
\pgfpathlineto{\pgfqpoint{3.984364in}{6.240196in}}%
\pgfpathlineto{\pgfqpoint{3.950791in}{6.240196in}}%
\pgfpathlineto{\pgfqpoint{3.917218in}{6.240196in}}%
\pgfpathlineto{\pgfqpoint{3.883645in}{6.240196in}}%
\pgfpathlineto{\pgfqpoint{3.850072in}{6.240196in}}%
\pgfpathlineto{\pgfqpoint{3.816499in}{6.240196in}}%
\pgfpathlineto{\pgfqpoint{3.782926in}{6.240196in}}%
\pgfpathlineto{\pgfqpoint{3.749353in}{6.240196in}}%
\pgfpathlineto{\pgfqpoint{3.715780in}{6.240196in}}%
\pgfpathlineto{\pgfqpoint{3.682207in}{6.240196in}}%
\pgfpathlineto{\pgfqpoint{3.648634in}{6.240196in}}%
\pgfpathlineto{\pgfqpoint{3.615061in}{6.240196in}}%
\pgfpathlineto{\pgfqpoint{3.581488in}{6.240196in}}%
\pgfpathlineto{\pgfqpoint{3.547914in}{6.240196in}}%
\pgfpathlineto{\pgfqpoint{3.514341in}{6.240196in}}%
\pgfpathlineto{\pgfqpoint{3.480768in}{6.240196in}}%
\pgfpathlineto{\pgfqpoint{3.447195in}{6.240196in}}%
\pgfpathlineto{\pgfqpoint{3.413622in}{6.240196in}}%
\pgfpathlineto{\pgfqpoint{3.380049in}{6.240196in}}%
\pgfpathlineto{\pgfqpoint{3.346476in}{6.240196in}}%
\pgfpathlineto{\pgfqpoint{3.312903in}{6.240196in}}%
\pgfpathlineto{\pgfqpoint{3.279330in}{6.240196in}}%
\pgfpathlineto{\pgfqpoint{3.245757in}{6.240196in}}%
\pgfpathlineto{\pgfqpoint{3.212184in}{6.240196in}}%
\pgfpathlineto{\pgfqpoint{3.178611in}{6.240196in}}%
\pgfpathlineto{\pgfqpoint{3.145038in}{6.240196in}}%
\pgfpathlineto{\pgfqpoint{3.111465in}{6.240196in}}%
\pgfpathlineto{\pgfqpoint{3.077892in}{6.240196in}}%
\pgfpathlineto{\pgfqpoint{3.044319in}{6.240196in}}%
\pgfpathlineto{\pgfqpoint{3.010746in}{6.240196in}}%
\pgfpathlineto{\pgfqpoint{2.977173in}{6.240196in}}%
\pgfpathlineto{\pgfqpoint{2.943599in}{6.240196in}}%
\pgfpathlineto{\pgfqpoint{2.910026in}{6.240196in}}%
\pgfpathlineto{\pgfqpoint{2.876453in}{6.240196in}}%
\pgfpathlineto{\pgfqpoint{2.842880in}{6.240196in}}%
\pgfpathlineto{\pgfqpoint{2.809307in}{6.240196in}}%
\pgfpathlineto{\pgfqpoint{2.775734in}{6.240196in}}%
\pgfpathlineto{\pgfqpoint{2.742161in}{6.240196in}}%
\pgfpathlineto{\pgfqpoint{2.708588in}{6.240196in}}%
\pgfpathlineto{\pgfqpoint{2.675015in}{6.240196in}}%
\pgfpathlineto{\pgfqpoint{2.641442in}{6.240196in}}%
\pgfpathlineto{\pgfqpoint{2.607869in}{6.240196in}}%
\pgfpathlineto{\pgfqpoint{2.574296in}{6.240196in}}%
\pgfpathlineto{\pgfqpoint{2.540723in}{6.240196in}}%
\pgfpathlineto{\pgfqpoint{2.507150in}{6.240196in}}%
\pgfpathlineto{\pgfqpoint{2.473577in}{6.240196in}}%
\pgfpathlineto{\pgfqpoint{2.440004in}{6.240196in}}%
\pgfpathlineto{\pgfqpoint{2.440004in}{6.240196in}}%
\pgfpathclose%
\pgfusepath{stroke,fill}%
}%
\begin{pgfscope}%
\pgfsys@transformshift{0.000000in}{0.000000in}%
\pgfsys@useobject{currentmarker}{}%
\end{pgfscope}%
\end{pgfscope}%
\begin{pgfscope}%
\pgfpathrectangle{\pgfqpoint{0.750000in}{3.960000in}}{\pgfqpoint{4.650000in}{3.080000in}}%
\pgfusepath{clip}%
\pgfsetrectcap%
\pgfsetroundjoin%
\pgfsetlinewidth{0.803000pt}%
\definecolor{currentstroke}{rgb}{0.690196,0.690196,0.690196}%
\pgfsetstrokecolor{currentstroke}%
\pgfsetdash{}{0pt}%
\pgfpathmoveto{\pgfqpoint{0.750000in}{3.960000in}}%
\pgfpathlineto{\pgfqpoint{0.750000in}{7.040000in}}%
\pgfusepath{stroke}%
\end{pgfscope}%
\begin{pgfscope}%
\pgfsetbuttcap%
\pgfsetroundjoin%
\definecolor{currentfill}{rgb}{0.000000,0.000000,0.000000}%
\pgfsetfillcolor{currentfill}%
\pgfsetlinewidth{0.803000pt}%
\definecolor{currentstroke}{rgb}{0.000000,0.000000,0.000000}%
\pgfsetstrokecolor{currentstroke}%
\pgfsetdash{}{0pt}%
\pgfsys@defobject{currentmarker}{\pgfqpoint{0.000000in}{-0.048611in}}{\pgfqpoint{0.000000in}{0.000000in}}{%
\pgfpathmoveto{\pgfqpoint{0.000000in}{0.000000in}}%
\pgfpathlineto{\pgfqpoint{0.000000in}{-0.048611in}}%
\pgfusepath{stroke,fill}%
}%
\begin{pgfscope}%
\pgfsys@transformshift{0.750000in}{3.960000in}%
\pgfsys@useobject{currentmarker}{}%
\end{pgfscope}%
\end{pgfscope}%
\begin{pgfscope}%
\pgfpathrectangle{\pgfqpoint{0.750000in}{3.960000in}}{\pgfqpoint{4.650000in}{3.080000in}}%
\pgfusepath{clip}%
\pgfsetrectcap%
\pgfsetroundjoin%
\pgfsetlinewidth{0.803000pt}%
\definecolor{currentstroke}{rgb}{0.690196,0.690196,0.690196}%
\pgfsetstrokecolor{currentstroke}%
\pgfsetdash{}{0pt}%
\pgfpathmoveto{\pgfqpoint{1.266667in}{3.960000in}}%
\pgfpathlineto{\pgfqpoint{1.266667in}{7.040000in}}%
\pgfusepath{stroke}%
\end{pgfscope}%
\begin{pgfscope}%
\pgfsetbuttcap%
\pgfsetroundjoin%
\definecolor{currentfill}{rgb}{0.000000,0.000000,0.000000}%
\pgfsetfillcolor{currentfill}%
\pgfsetlinewidth{0.803000pt}%
\definecolor{currentstroke}{rgb}{0.000000,0.000000,0.000000}%
\pgfsetstrokecolor{currentstroke}%
\pgfsetdash{}{0pt}%
\pgfsys@defobject{currentmarker}{\pgfqpoint{0.000000in}{-0.048611in}}{\pgfqpoint{0.000000in}{0.000000in}}{%
\pgfpathmoveto{\pgfqpoint{0.000000in}{0.000000in}}%
\pgfpathlineto{\pgfqpoint{0.000000in}{-0.048611in}}%
\pgfusepath{stroke,fill}%
}%
\begin{pgfscope}%
\pgfsys@transformshift{1.266667in}{3.960000in}%
\pgfsys@useobject{currentmarker}{}%
\end{pgfscope}%
\end{pgfscope}%
\begin{pgfscope}%
\pgfpathrectangle{\pgfqpoint{0.750000in}{3.960000in}}{\pgfqpoint{4.650000in}{3.080000in}}%
\pgfusepath{clip}%
\pgfsetrectcap%
\pgfsetroundjoin%
\pgfsetlinewidth{0.803000pt}%
\definecolor{currentstroke}{rgb}{0.690196,0.690196,0.690196}%
\pgfsetstrokecolor{currentstroke}%
\pgfsetdash{}{0pt}%
\pgfpathmoveto{\pgfqpoint{1.783333in}{3.960000in}}%
\pgfpathlineto{\pgfqpoint{1.783333in}{7.040000in}}%
\pgfusepath{stroke}%
\end{pgfscope}%
\begin{pgfscope}%
\pgfsetbuttcap%
\pgfsetroundjoin%
\definecolor{currentfill}{rgb}{0.000000,0.000000,0.000000}%
\pgfsetfillcolor{currentfill}%
\pgfsetlinewidth{0.803000pt}%
\definecolor{currentstroke}{rgb}{0.000000,0.000000,0.000000}%
\pgfsetstrokecolor{currentstroke}%
\pgfsetdash{}{0pt}%
\pgfsys@defobject{currentmarker}{\pgfqpoint{0.000000in}{-0.048611in}}{\pgfqpoint{0.000000in}{0.000000in}}{%
\pgfpathmoveto{\pgfqpoint{0.000000in}{0.000000in}}%
\pgfpathlineto{\pgfqpoint{0.000000in}{-0.048611in}}%
\pgfusepath{stroke,fill}%
}%
\begin{pgfscope}%
\pgfsys@transformshift{1.783333in}{3.960000in}%
\pgfsys@useobject{currentmarker}{}%
\end{pgfscope}%
\end{pgfscope}%
\begin{pgfscope}%
\pgfpathrectangle{\pgfqpoint{0.750000in}{3.960000in}}{\pgfqpoint{4.650000in}{3.080000in}}%
\pgfusepath{clip}%
\pgfsetrectcap%
\pgfsetroundjoin%
\pgfsetlinewidth{0.803000pt}%
\definecolor{currentstroke}{rgb}{0.690196,0.690196,0.690196}%
\pgfsetstrokecolor{currentstroke}%
\pgfsetdash{}{0pt}%
\pgfpathmoveto{\pgfqpoint{2.300000in}{3.960000in}}%
\pgfpathlineto{\pgfqpoint{2.300000in}{7.040000in}}%
\pgfusepath{stroke}%
\end{pgfscope}%
\begin{pgfscope}%
\pgfsetbuttcap%
\pgfsetroundjoin%
\definecolor{currentfill}{rgb}{0.000000,0.000000,0.000000}%
\pgfsetfillcolor{currentfill}%
\pgfsetlinewidth{0.803000pt}%
\definecolor{currentstroke}{rgb}{0.000000,0.000000,0.000000}%
\pgfsetstrokecolor{currentstroke}%
\pgfsetdash{}{0pt}%
\pgfsys@defobject{currentmarker}{\pgfqpoint{0.000000in}{-0.048611in}}{\pgfqpoint{0.000000in}{0.000000in}}{%
\pgfpathmoveto{\pgfqpoint{0.000000in}{0.000000in}}%
\pgfpathlineto{\pgfqpoint{0.000000in}{-0.048611in}}%
\pgfusepath{stroke,fill}%
}%
\begin{pgfscope}%
\pgfsys@transformshift{2.300000in}{3.960000in}%
\pgfsys@useobject{currentmarker}{}%
\end{pgfscope}%
\end{pgfscope}%
\begin{pgfscope}%
\pgfpathrectangle{\pgfqpoint{0.750000in}{3.960000in}}{\pgfqpoint{4.650000in}{3.080000in}}%
\pgfusepath{clip}%
\pgfsetrectcap%
\pgfsetroundjoin%
\pgfsetlinewidth{0.803000pt}%
\definecolor{currentstroke}{rgb}{0.690196,0.690196,0.690196}%
\pgfsetstrokecolor{currentstroke}%
\pgfsetdash{}{0pt}%
\pgfpathmoveto{\pgfqpoint{2.816667in}{3.960000in}}%
\pgfpathlineto{\pgfqpoint{2.816667in}{7.040000in}}%
\pgfusepath{stroke}%
\end{pgfscope}%
\begin{pgfscope}%
\pgfsetbuttcap%
\pgfsetroundjoin%
\definecolor{currentfill}{rgb}{0.000000,0.000000,0.000000}%
\pgfsetfillcolor{currentfill}%
\pgfsetlinewidth{0.803000pt}%
\definecolor{currentstroke}{rgb}{0.000000,0.000000,0.000000}%
\pgfsetstrokecolor{currentstroke}%
\pgfsetdash{}{0pt}%
\pgfsys@defobject{currentmarker}{\pgfqpoint{0.000000in}{-0.048611in}}{\pgfqpoint{0.000000in}{0.000000in}}{%
\pgfpathmoveto{\pgfqpoint{0.000000in}{0.000000in}}%
\pgfpathlineto{\pgfqpoint{0.000000in}{-0.048611in}}%
\pgfusepath{stroke,fill}%
}%
\begin{pgfscope}%
\pgfsys@transformshift{2.816667in}{3.960000in}%
\pgfsys@useobject{currentmarker}{}%
\end{pgfscope}%
\end{pgfscope}%
\begin{pgfscope}%
\pgfpathrectangle{\pgfqpoint{0.750000in}{3.960000in}}{\pgfqpoint{4.650000in}{3.080000in}}%
\pgfusepath{clip}%
\pgfsetrectcap%
\pgfsetroundjoin%
\pgfsetlinewidth{0.803000pt}%
\definecolor{currentstroke}{rgb}{0.690196,0.690196,0.690196}%
\pgfsetstrokecolor{currentstroke}%
\pgfsetdash{}{0pt}%
\pgfpathmoveto{\pgfqpoint{3.333333in}{3.960000in}}%
\pgfpathlineto{\pgfqpoint{3.333333in}{7.040000in}}%
\pgfusepath{stroke}%
\end{pgfscope}%
\begin{pgfscope}%
\pgfsetbuttcap%
\pgfsetroundjoin%
\definecolor{currentfill}{rgb}{0.000000,0.000000,0.000000}%
\pgfsetfillcolor{currentfill}%
\pgfsetlinewidth{0.803000pt}%
\definecolor{currentstroke}{rgb}{0.000000,0.000000,0.000000}%
\pgfsetstrokecolor{currentstroke}%
\pgfsetdash{}{0pt}%
\pgfsys@defobject{currentmarker}{\pgfqpoint{0.000000in}{-0.048611in}}{\pgfqpoint{0.000000in}{0.000000in}}{%
\pgfpathmoveto{\pgfqpoint{0.000000in}{0.000000in}}%
\pgfpathlineto{\pgfqpoint{0.000000in}{-0.048611in}}%
\pgfusepath{stroke,fill}%
}%
\begin{pgfscope}%
\pgfsys@transformshift{3.333333in}{3.960000in}%
\pgfsys@useobject{currentmarker}{}%
\end{pgfscope}%
\end{pgfscope}%
\begin{pgfscope}%
\pgfpathrectangle{\pgfqpoint{0.750000in}{3.960000in}}{\pgfqpoint{4.650000in}{3.080000in}}%
\pgfusepath{clip}%
\pgfsetrectcap%
\pgfsetroundjoin%
\pgfsetlinewidth{0.803000pt}%
\definecolor{currentstroke}{rgb}{0.690196,0.690196,0.690196}%
\pgfsetstrokecolor{currentstroke}%
\pgfsetdash{}{0pt}%
\pgfpathmoveto{\pgfqpoint{3.850000in}{3.960000in}}%
\pgfpathlineto{\pgfqpoint{3.850000in}{7.040000in}}%
\pgfusepath{stroke}%
\end{pgfscope}%
\begin{pgfscope}%
\pgfsetbuttcap%
\pgfsetroundjoin%
\definecolor{currentfill}{rgb}{0.000000,0.000000,0.000000}%
\pgfsetfillcolor{currentfill}%
\pgfsetlinewidth{0.803000pt}%
\definecolor{currentstroke}{rgb}{0.000000,0.000000,0.000000}%
\pgfsetstrokecolor{currentstroke}%
\pgfsetdash{}{0pt}%
\pgfsys@defobject{currentmarker}{\pgfqpoint{0.000000in}{-0.048611in}}{\pgfqpoint{0.000000in}{0.000000in}}{%
\pgfpathmoveto{\pgfqpoint{0.000000in}{0.000000in}}%
\pgfpathlineto{\pgfqpoint{0.000000in}{-0.048611in}}%
\pgfusepath{stroke,fill}%
}%
\begin{pgfscope}%
\pgfsys@transformshift{3.850000in}{3.960000in}%
\pgfsys@useobject{currentmarker}{}%
\end{pgfscope}%
\end{pgfscope}%
\begin{pgfscope}%
\pgfpathrectangle{\pgfqpoint{0.750000in}{3.960000in}}{\pgfqpoint{4.650000in}{3.080000in}}%
\pgfusepath{clip}%
\pgfsetrectcap%
\pgfsetroundjoin%
\pgfsetlinewidth{0.803000pt}%
\definecolor{currentstroke}{rgb}{0.690196,0.690196,0.690196}%
\pgfsetstrokecolor{currentstroke}%
\pgfsetdash{}{0pt}%
\pgfpathmoveto{\pgfqpoint{4.366667in}{3.960000in}}%
\pgfpathlineto{\pgfqpoint{4.366667in}{7.040000in}}%
\pgfusepath{stroke}%
\end{pgfscope}%
\begin{pgfscope}%
\pgfsetbuttcap%
\pgfsetroundjoin%
\definecolor{currentfill}{rgb}{0.000000,0.000000,0.000000}%
\pgfsetfillcolor{currentfill}%
\pgfsetlinewidth{0.803000pt}%
\definecolor{currentstroke}{rgb}{0.000000,0.000000,0.000000}%
\pgfsetstrokecolor{currentstroke}%
\pgfsetdash{}{0pt}%
\pgfsys@defobject{currentmarker}{\pgfqpoint{0.000000in}{-0.048611in}}{\pgfqpoint{0.000000in}{0.000000in}}{%
\pgfpathmoveto{\pgfqpoint{0.000000in}{0.000000in}}%
\pgfpathlineto{\pgfqpoint{0.000000in}{-0.048611in}}%
\pgfusepath{stroke,fill}%
}%
\begin{pgfscope}%
\pgfsys@transformshift{4.366667in}{3.960000in}%
\pgfsys@useobject{currentmarker}{}%
\end{pgfscope}%
\end{pgfscope}%
\begin{pgfscope}%
\pgfpathrectangle{\pgfqpoint{0.750000in}{3.960000in}}{\pgfqpoint{4.650000in}{3.080000in}}%
\pgfusepath{clip}%
\pgfsetrectcap%
\pgfsetroundjoin%
\pgfsetlinewidth{0.803000pt}%
\definecolor{currentstroke}{rgb}{0.690196,0.690196,0.690196}%
\pgfsetstrokecolor{currentstroke}%
\pgfsetdash{}{0pt}%
\pgfpathmoveto{\pgfqpoint{4.883333in}{3.960000in}}%
\pgfpathlineto{\pgfqpoint{4.883333in}{7.040000in}}%
\pgfusepath{stroke}%
\end{pgfscope}%
\begin{pgfscope}%
\pgfsetbuttcap%
\pgfsetroundjoin%
\definecolor{currentfill}{rgb}{0.000000,0.000000,0.000000}%
\pgfsetfillcolor{currentfill}%
\pgfsetlinewidth{0.803000pt}%
\definecolor{currentstroke}{rgb}{0.000000,0.000000,0.000000}%
\pgfsetstrokecolor{currentstroke}%
\pgfsetdash{}{0pt}%
\pgfsys@defobject{currentmarker}{\pgfqpoint{0.000000in}{-0.048611in}}{\pgfqpoint{0.000000in}{0.000000in}}{%
\pgfpathmoveto{\pgfqpoint{0.000000in}{0.000000in}}%
\pgfpathlineto{\pgfqpoint{0.000000in}{-0.048611in}}%
\pgfusepath{stroke,fill}%
}%
\begin{pgfscope}%
\pgfsys@transformshift{4.883333in}{3.960000in}%
\pgfsys@useobject{currentmarker}{}%
\end{pgfscope}%
\end{pgfscope}%
\begin{pgfscope}%
\pgfpathrectangle{\pgfqpoint{0.750000in}{3.960000in}}{\pgfqpoint{4.650000in}{3.080000in}}%
\pgfusepath{clip}%
\pgfsetrectcap%
\pgfsetroundjoin%
\pgfsetlinewidth{0.803000pt}%
\definecolor{currentstroke}{rgb}{0.690196,0.690196,0.690196}%
\pgfsetstrokecolor{currentstroke}%
\pgfsetdash{}{0pt}%
\pgfpathmoveto{\pgfqpoint{5.400000in}{3.960000in}}%
\pgfpathlineto{\pgfqpoint{5.400000in}{7.040000in}}%
\pgfusepath{stroke}%
\end{pgfscope}%
\begin{pgfscope}%
\pgfsetbuttcap%
\pgfsetroundjoin%
\definecolor{currentfill}{rgb}{0.000000,0.000000,0.000000}%
\pgfsetfillcolor{currentfill}%
\pgfsetlinewidth{0.803000pt}%
\definecolor{currentstroke}{rgb}{0.000000,0.000000,0.000000}%
\pgfsetstrokecolor{currentstroke}%
\pgfsetdash{}{0pt}%
\pgfsys@defobject{currentmarker}{\pgfqpoint{0.000000in}{-0.048611in}}{\pgfqpoint{0.000000in}{0.000000in}}{%
\pgfpathmoveto{\pgfqpoint{0.000000in}{0.000000in}}%
\pgfpathlineto{\pgfqpoint{0.000000in}{-0.048611in}}%
\pgfusepath{stroke,fill}%
}%
\begin{pgfscope}%
\pgfsys@transformshift{5.400000in}{3.960000in}%
\pgfsys@useobject{currentmarker}{}%
\end{pgfscope}%
\end{pgfscope}%
\begin{pgfscope}%
\pgfpathrectangle{\pgfqpoint{0.750000in}{3.960000in}}{\pgfqpoint{4.650000in}{3.080000in}}%
\pgfusepath{clip}%
\pgfsetrectcap%
\pgfsetroundjoin%
\pgfsetlinewidth{0.803000pt}%
\definecolor{currentstroke}{rgb}{0.690196,0.690196,0.690196}%
\pgfsetstrokecolor{currentstroke}%
\pgfsetdash{}{0pt}%
\pgfpathmoveto{\pgfqpoint{0.750000in}{3.960000in}}%
\pgfpathlineto{\pgfqpoint{5.400000in}{3.960000in}}%
\pgfusepath{stroke}%
\end{pgfscope}%
\begin{pgfscope}%
\pgfsetbuttcap%
\pgfsetroundjoin%
\definecolor{currentfill}{rgb}{0.000000,0.000000,0.000000}%
\pgfsetfillcolor{currentfill}%
\pgfsetlinewidth{0.803000pt}%
\definecolor{currentstroke}{rgb}{0.000000,0.000000,0.000000}%
\pgfsetstrokecolor{currentstroke}%
\pgfsetdash{}{0pt}%
\pgfsys@defobject{currentmarker}{\pgfqpoint{-0.048611in}{0.000000in}}{\pgfqpoint{-0.000000in}{0.000000in}}{%
\pgfpathmoveto{\pgfqpoint{-0.000000in}{0.000000in}}%
\pgfpathlineto{\pgfqpoint{-0.048611in}{0.000000in}}%
\pgfusepath{stroke,fill}%
}%
\begin{pgfscope}%
\pgfsys@transformshift{0.750000in}{3.960000in}%
\pgfsys@useobject{currentmarker}{}%
\end{pgfscope}%
\end{pgfscope}%
\begin{pgfscope}%
\definecolor{textcolor}{rgb}{0.000000,0.000000,0.000000}%
\pgfsetstrokecolor{textcolor}%
\pgfsetfillcolor{textcolor}%
\pgftext[x=0.444254in, y=3.898680in, left, base]{\color{textcolor}\rmfamily\fontsize{12.000000}{14.400000}\selectfont \(\displaystyle {0.0}\)}%
\end{pgfscope}%
\begin{pgfscope}%
\pgfpathrectangle{\pgfqpoint{0.750000in}{3.960000in}}{\pgfqpoint{4.650000in}{3.080000in}}%
\pgfusepath{clip}%
\pgfsetrectcap%
\pgfsetroundjoin%
\pgfsetlinewidth{0.803000pt}%
\definecolor{currentstroke}{rgb}{0.690196,0.690196,0.690196}%
\pgfsetstrokecolor{currentstroke}%
\pgfsetdash{}{0pt}%
\pgfpathmoveto{\pgfqpoint{0.750000in}{4.462145in}}%
\pgfpathlineto{\pgfqpoint{5.400000in}{4.462145in}}%
\pgfusepath{stroke}%
\end{pgfscope}%
\begin{pgfscope}%
\pgfsetbuttcap%
\pgfsetroundjoin%
\definecolor{currentfill}{rgb}{0.000000,0.000000,0.000000}%
\pgfsetfillcolor{currentfill}%
\pgfsetlinewidth{0.803000pt}%
\definecolor{currentstroke}{rgb}{0.000000,0.000000,0.000000}%
\pgfsetstrokecolor{currentstroke}%
\pgfsetdash{}{0pt}%
\pgfsys@defobject{currentmarker}{\pgfqpoint{-0.048611in}{0.000000in}}{\pgfqpoint{-0.000000in}{0.000000in}}{%
\pgfpathmoveto{\pgfqpoint{-0.000000in}{0.000000in}}%
\pgfpathlineto{\pgfqpoint{-0.048611in}{0.000000in}}%
\pgfusepath{stroke,fill}%
}%
\begin{pgfscope}%
\pgfsys@transformshift{0.750000in}{4.462145in}%
\pgfsys@useobject{currentmarker}{}%
\end{pgfscope}%
\end{pgfscope}%
\begin{pgfscope}%
\definecolor{textcolor}{rgb}{0.000000,0.000000,0.000000}%
\pgfsetstrokecolor{textcolor}%
\pgfsetfillcolor{textcolor}%
\pgftext[x=0.444254in, y=4.400825in, left, base]{\color{textcolor}\rmfamily\fontsize{12.000000}{14.400000}\selectfont \(\displaystyle {0.2}\)}%
\end{pgfscope}%
\begin{pgfscope}%
\pgfpathrectangle{\pgfqpoint{0.750000in}{3.960000in}}{\pgfqpoint{4.650000in}{3.080000in}}%
\pgfusepath{clip}%
\pgfsetrectcap%
\pgfsetroundjoin%
\pgfsetlinewidth{0.803000pt}%
\definecolor{currentstroke}{rgb}{0.690196,0.690196,0.690196}%
\pgfsetstrokecolor{currentstroke}%
\pgfsetdash{}{0pt}%
\pgfpathmoveto{\pgfqpoint{0.750000in}{4.964291in}}%
\pgfpathlineto{\pgfqpoint{5.400000in}{4.964291in}}%
\pgfusepath{stroke}%
\end{pgfscope}%
\begin{pgfscope}%
\pgfsetbuttcap%
\pgfsetroundjoin%
\definecolor{currentfill}{rgb}{0.000000,0.000000,0.000000}%
\pgfsetfillcolor{currentfill}%
\pgfsetlinewidth{0.803000pt}%
\definecolor{currentstroke}{rgb}{0.000000,0.000000,0.000000}%
\pgfsetstrokecolor{currentstroke}%
\pgfsetdash{}{0pt}%
\pgfsys@defobject{currentmarker}{\pgfqpoint{-0.048611in}{0.000000in}}{\pgfqpoint{-0.000000in}{0.000000in}}{%
\pgfpathmoveto{\pgfqpoint{-0.000000in}{0.000000in}}%
\pgfpathlineto{\pgfqpoint{-0.048611in}{0.000000in}}%
\pgfusepath{stroke,fill}%
}%
\begin{pgfscope}%
\pgfsys@transformshift{0.750000in}{4.964291in}%
\pgfsys@useobject{currentmarker}{}%
\end{pgfscope}%
\end{pgfscope}%
\begin{pgfscope}%
\definecolor{textcolor}{rgb}{0.000000,0.000000,0.000000}%
\pgfsetstrokecolor{textcolor}%
\pgfsetfillcolor{textcolor}%
\pgftext[x=0.444254in, y=4.902971in, left, base]{\color{textcolor}\rmfamily\fontsize{12.000000}{14.400000}\selectfont \(\displaystyle {0.4}\)}%
\end{pgfscope}%
\begin{pgfscope}%
\pgfpathrectangle{\pgfqpoint{0.750000in}{3.960000in}}{\pgfqpoint{4.650000in}{3.080000in}}%
\pgfusepath{clip}%
\pgfsetrectcap%
\pgfsetroundjoin%
\pgfsetlinewidth{0.803000pt}%
\definecolor{currentstroke}{rgb}{0.690196,0.690196,0.690196}%
\pgfsetstrokecolor{currentstroke}%
\pgfsetdash{}{0pt}%
\pgfpathmoveto{\pgfqpoint{0.750000in}{5.466436in}}%
\pgfpathlineto{\pgfqpoint{5.400000in}{5.466436in}}%
\pgfusepath{stroke}%
\end{pgfscope}%
\begin{pgfscope}%
\pgfsetbuttcap%
\pgfsetroundjoin%
\definecolor{currentfill}{rgb}{0.000000,0.000000,0.000000}%
\pgfsetfillcolor{currentfill}%
\pgfsetlinewidth{0.803000pt}%
\definecolor{currentstroke}{rgb}{0.000000,0.000000,0.000000}%
\pgfsetstrokecolor{currentstroke}%
\pgfsetdash{}{0pt}%
\pgfsys@defobject{currentmarker}{\pgfqpoint{-0.048611in}{0.000000in}}{\pgfqpoint{-0.000000in}{0.000000in}}{%
\pgfpathmoveto{\pgfqpoint{-0.000000in}{0.000000in}}%
\pgfpathlineto{\pgfqpoint{-0.048611in}{0.000000in}}%
\pgfusepath{stroke,fill}%
}%
\begin{pgfscope}%
\pgfsys@transformshift{0.750000in}{5.466436in}%
\pgfsys@useobject{currentmarker}{}%
\end{pgfscope}%
\end{pgfscope}%
\begin{pgfscope}%
\definecolor{textcolor}{rgb}{0.000000,0.000000,0.000000}%
\pgfsetstrokecolor{textcolor}%
\pgfsetfillcolor{textcolor}%
\pgftext[x=0.444254in, y=5.405116in, left, base]{\color{textcolor}\rmfamily\fontsize{12.000000}{14.400000}\selectfont \(\displaystyle {0.6}\)}%
\end{pgfscope}%
\begin{pgfscope}%
\pgfpathrectangle{\pgfqpoint{0.750000in}{3.960000in}}{\pgfqpoint{4.650000in}{3.080000in}}%
\pgfusepath{clip}%
\pgfsetrectcap%
\pgfsetroundjoin%
\pgfsetlinewidth{0.803000pt}%
\definecolor{currentstroke}{rgb}{0.690196,0.690196,0.690196}%
\pgfsetstrokecolor{currentstroke}%
\pgfsetdash{}{0pt}%
\pgfpathmoveto{\pgfqpoint{0.750000in}{5.968581in}}%
\pgfpathlineto{\pgfqpoint{5.400000in}{5.968581in}}%
\pgfusepath{stroke}%
\end{pgfscope}%
\begin{pgfscope}%
\pgfsetbuttcap%
\pgfsetroundjoin%
\definecolor{currentfill}{rgb}{0.000000,0.000000,0.000000}%
\pgfsetfillcolor{currentfill}%
\pgfsetlinewidth{0.803000pt}%
\definecolor{currentstroke}{rgb}{0.000000,0.000000,0.000000}%
\pgfsetstrokecolor{currentstroke}%
\pgfsetdash{}{0pt}%
\pgfsys@defobject{currentmarker}{\pgfqpoint{-0.048611in}{0.000000in}}{\pgfqpoint{-0.000000in}{0.000000in}}{%
\pgfpathmoveto{\pgfqpoint{-0.000000in}{0.000000in}}%
\pgfpathlineto{\pgfqpoint{-0.048611in}{0.000000in}}%
\pgfusepath{stroke,fill}%
}%
\begin{pgfscope}%
\pgfsys@transformshift{0.750000in}{5.968581in}%
\pgfsys@useobject{currentmarker}{}%
\end{pgfscope}%
\end{pgfscope}%
\begin{pgfscope}%
\definecolor{textcolor}{rgb}{0.000000,0.000000,0.000000}%
\pgfsetstrokecolor{textcolor}%
\pgfsetfillcolor{textcolor}%
\pgftext[x=0.444254in, y=5.907261in, left, base]{\color{textcolor}\rmfamily\fontsize{12.000000}{14.400000}\selectfont \(\displaystyle {0.8}\)}%
\end{pgfscope}%
\begin{pgfscope}%
\pgfpathrectangle{\pgfqpoint{0.750000in}{3.960000in}}{\pgfqpoint{4.650000in}{3.080000in}}%
\pgfusepath{clip}%
\pgfsetrectcap%
\pgfsetroundjoin%
\pgfsetlinewidth{0.803000pt}%
\definecolor{currentstroke}{rgb}{0.690196,0.690196,0.690196}%
\pgfsetstrokecolor{currentstroke}%
\pgfsetdash{}{0pt}%
\pgfpathmoveto{\pgfqpoint{0.750000in}{6.470727in}}%
\pgfpathlineto{\pgfqpoint{5.400000in}{6.470727in}}%
\pgfusepath{stroke}%
\end{pgfscope}%
\begin{pgfscope}%
\pgfsetbuttcap%
\pgfsetroundjoin%
\definecolor{currentfill}{rgb}{0.000000,0.000000,0.000000}%
\pgfsetfillcolor{currentfill}%
\pgfsetlinewidth{0.803000pt}%
\definecolor{currentstroke}{rgb}{0.000000,0.000000,0.000000}%
\pgfsetstrokecolor{currentstroke}%
\pgfsetdash{}{0pt}%
\pgfsys@defobject{currentmarker}{\pgfqpoint{-0.048611in}{0.000000in}}{\pgfqpoint{-0.000000in}{0.000000in}}{%
\pgfpathmoveto{\pgfqpoint{-0.000000in}{0.000000in}}%
\pgfpathlineto{\pgfqpoint{-0.048611in}{0.000000in}}%
\pgfusepath{stroke,fill}%
}%
\begin{pgfscope}%
\pgfsys@transformshift{0.750000in}{6.470727in}%
\pgfsys@useobject{currentmarker}{}%
\end{pgfscope}%
\end{pgfscope}%
\begin{pgfscope}%
\definecolor{textcolor}{rgb}{0.000000,0.000000,0.000000}%
\pgfsetstrokecolor{textcolor}%
\pgfsetfillcolor{textcolor}%
\pgftext[x=0.444254in, y=6.409407in, left, base]{\color{textcolor}\rmfamily\fontsize{12.000000}{14.400000}\selectfont \(\displaystyle {1.0}\)}%
\end{pgfscope}%
\begin{pgfscope}%
\pgfpathrectangle{\pgfqpoint{0.750000in}{3.960000in}}{\pgfqpoint{4.650000in}{3.080000in}}%
\pgfusepath{clip}%
\pgfsetrectcap%
\pgfsetroundjoin%
\pgfsetlinewidth{0.803000pt}%
\definecolor{currentstroke}{rgb}{0.690196,0.690196,0.690196}%
\pgfsetstrokecolor{currentstroke}%
\pgfsetdash{}{0pt}%
\pgfpathmoveto{\pgfqpoint{0.750000in}{6.972872in}}%
\pgfpathlineto{\pgfqpoint{5.400000in}{6.972872in}}%
\pgfusepath{stroke}%
\end{pgfscope}%
\begin{pgfscope}%
\pgfsetbuttcap%
\pgfsetroundjoin%
\definecolor{currentfill}{rgb}{0.000000,0.000000,0.000000}%
\pgfsetfillcolor{currentfill}%
\pgfsetlinewidth{0.803000pt}%
\definecolor{currentstroke}{rgb}{0.000000,0.000000,0.000000}%
\pgfsetstrokecolor{currentstroke}%
\pgfsetdash{}{0pt}%
\pgfsys@defobject{currentmarker}{\pgfqpoint{-0.048611in}{0.000000in}}{\pgfqpoint{-0.000000in}{0.000000in}}{%
\pgfpathmoveto{\pgfqpoint{-0.000000in}{0.000000in}}%
\pgfpathlineto{\pgfqpoint{-0.048611in}{0.000000in}}%
\pgfusepath{stroke,fill}%
}%
\begin{pgfscope}%
\pgfsys@transformshift{0.750000in}{6.972872in}%
\pgfsys@useobject{currentmarker}{}%
\end{pgfscope}%
\end{pgfscope}%
\begin{pgfscope}%
\definecolor{textcolor}{rgb}{0.000000,0.000000,0.000000}%
\pgfsetstrokecolor{textcolor}%
\pgfsetfillcolor{textcolor}%
\pgftext[x=0.444254in, y=6.911552in, left, base]{\color{textcolor}\rmfamily\fontsize{12.000000}{14.400000}\selectfont \(\displaystyle {1.2}\)}%
\end{pgfscope}%
\begin{pgfscope}%
\definecolor{textcolor}{rgb}{0.000000,0.000000,0.000000}%
\pgfsetstrokecolor{textcolor}%
\pgfsetfillcolor{textcolor}%
\pgftext[x=0.388698in,y=5.500000in,,bottom,rotate=90.000000]{\color{textcolor}\rmfamily\fontsize{12.000000}{14.400000}\selectfont power in pu}%
\end{pgfscope}%
\begin{pgfscope}%
\pgfpathrectangle{\pgfqpoint{0.750000in}{3.960000in}}{\pgfqpoint{4.650000in}{3.080000in}}%
\pgfusepath{clip}%
\pgfsetrectcap%
\pgfsetroundjoin%
\pgfsetlinewidth{2.007500pt}%
\definecolor{currentstroke}{rgb}{0.121569,0.466667,0.705882}%
\pgfsetstrokecolor{currentstroke}%
\pgfsetdash{}{0pt}%
\pgfpathmoveto{\pgfqpoint{0.750000in}{3.960000in}}%
\pgfpathlineto{\pgfqpoint{0.844898in}{4.148036in}}%
\pgfpathlineto{\pgfqpoint{0.939796in}{4.335299in}}%
\pgfpathlineto{\pgfqpoint{1.034694in}{4.521020in}}%
\pgfpathlineto{\pgfqpoint{1.129592in}{4.704436in}}%
\pgfpathlineto{\pgfqpoint{1.224490in}{4.884793in}}%
\pgfpathlineto{\pgfqpoint{1.319388in}{5.061349in}}%
\pgfpathlineto{\pgfqpoint{1.414286in}{5.233380in}}%
\pgfpathlineto{\pgfqpoint{1.509184in}{5.400178in}}%
\pgfpathlineto{\pgfqpoint{1.604082in}{5.561058in}}%
\pgfpathlineto{\pgfqpoint{1.698980in}{5.715359in}}%
\pgfpathlineto{\pgfqpoint{1.793878in}{5.862447in}}%
\pgfpathlineto{\pgfqpoint{1.888776in}{6.001718in}}%
\pgfpathlineto{\pgfqpoint{1.983673in}{6.132598in}}%
\pgfpathlineto{\pgfqpoint{2.078571in}{6.254551in}}%
\pgfpathlineto{\pgfqpoint{2.173469in}{6.367075in}}%
\pgfpathlineto{\pgfqpoint{2.268367in}{6.469708in}}%
\pgfpathlineto{\pgfqpoint{2.363265in}{6.562028in}}%
\pgfpathlineto{\pgfqpoint{2.458163in}{6.643656in}}%
\pgfpathlineto{\pgfqpoint{2.553061in}{6.714256in}}%
\pgfpathlineto{\pgfqpoint{2.647959in}{6.773538in}}%
\pgfpathlineto{\pgfqpoint{2.742857in}{6.821259in}}%
\pgfpathlineto{\pgfqpoint{2.837755in}{6.857222in}}%
\pgfpathlineto{\pgfqpoint{2.932653in}{6.881280in}}%
\pgfpathlineto{\pgfqpoint{3.027551in}{6.893333in}}%
\pgfpathlineto{\pgfqpoint{3.122449in}{6.893333in}}%
\pgfpathlineto{\pgfqpoint{3.217347in}{6.881280in}}%
\pgfpathlineto{\pgfqpoint{3.312245in}{6.857222in}}%
\pgfpathlineto{\pgfqpoint{3.407143in}{6.821259in}}%
\pgfpathlineto{\pgfqpoint{3.502041in}{6.773538in}}%
\pgfpathlineto{\pgfqpoint{3.596939in}{6.714256in}}%
\pgfpathlineto{\pgfqpoint{3.691837in}{6.643656in}}%
\pgfpathlineto{\pgfqpoint{3.786735in}{6.562028in}}%
\pgfpathlineto{\pgfqpoint{3.881633in}{6.469708in}}%
\pgfpathlineto{\pgfqpoint{3.976531in}{6.367075in}}%
\pgfpathlineto{\pgfqpoint{4.071429in}{6.254551in}}%
\pgfpathlineto{\pgfqpoint{4.166327in}{6.132598in}}%
\pgfpathlineto{\pgfqpoint{4.261224in}{6.001718in}}%
\pgfpathlineto{\pgfqpoint{4.356122in}{5.862447in}}%
\pgfpathlineto{\pgfqpoint{4.451020in}{5.715359in}}%
\pgfpathlineto{\pgfqpoint{4.545918in}{5.561058in}}%
\pgfpathlineto{\pgfqpoint{4.640816in}{5.400178in}}%
\pgfpathlineto{\pgfqpoint{4.735714in}{5.233380in}}%
\pgfpathlineto{\pgfqpoint{4.830612in}{5.061349in}}%
\pgfpathlineto{\pgfqpoint{4.925510in}{4.884793in}}%
\pgfpathlineto{\pgfqpoint{5.020408in}{4.704436in}}%
\pgfpathlineto{\pgfqpoint{5.115306in}{4.521020in}}%
\pgfpathlineto{\pgfqpoint{5.210204in}{4.335299in}}%
\pgfpathlineto{\pgfqpoint{5.305102in}{4.148036in}}%
\pgfpathlineto{\pgfqpoint{5.400000in}{3.960000in}}%
\pgfusepath{stroke}%
\end{pgfscope}%
\begin{pgfscope}%
\pgfpathrectangle{\pgfqpoint{0.750000in}{3.960000in}}{\pgfqpoint{4.650000in}{3.080000in}}%
\pgfusepath{clip}%
\pgfsetrectcap%
\pgfsetroundjoin%
\pgfsetlinewidth{2.007500pt}%
\definecolor{currentstroke}{rgb}{1.000000,0.498039,0.054902}%
\pgfsetstrokecolor{currentstroke}%
\pgfsetdash{}{0pt}%
\pgfpathmoveto{\pgfqpoint{0.750000in}{6.240196in}}%
\pgfpathlineto{\pgfqpoint{0.844898in}{6.240196in}}%
\pgfpathlineto{\pgfqpoint{0.939796in}{6.240196in}}%
\pgfpathlineto{\pgfqpoint{1.034694in}{6.240196in}}%
\pgfpathlineto{\pgfqpoint{1.129592in}{6.240196in}}%
\pgfpathlineto{\pgfqpoint{1.224490in}{6.240196in}}%
\pgfpathlineto{\pgfqpoint{1.319388in}{6.240196in}}%
\pgfpathlineto{\pgfqpoint{1.414286in}{6.240196in}}%
\pgfpathlineto{\pgfqpoint{1.509184in}{6.240196in}}%
\pgfpathlineto{\pgfqpoint{1.604082in}{6.240196in}}%
\pgfpathlineto{\pgfqpoint{1.698980in}{6.240196in}}%
\pgfpathlineto{\pgfqpoint{1.793878in}{6.240196in}}%
\pgfpathlineto{\pgfqpoint{1.888776in}{6.240196in}}%
\pgfpathlineto{\pgfqpoint{1.983673in}{6.240196in}}%
\pgfpathlineto{\pgfqpoint{2.078571in}{6.240196in}}%
\pgfpathlineto{\pgfqpoint{2.173469in}{6.240196in}}%
\pgfpathlineto{\pgfqpoint{2.268367in}{6.240196in}}%
\pgfpathlineto{\pgfqpoint{2.363265in}{6.240196in}}%
\pgfpathlineto{\pgfqpoint{2.458163in}{6.240196in}}%
\pgfpathlineto{\pgfqpoint{2.553061in}{6.240196in}}%
\pgfpathlineto{\pgfqpoint{2.647959in}{6.240196in}}%
\pgfpathlineto{\pgfqpoint{2.742857in}{6.240196in}}%
\pgfpathlineto{\pgfqpoint{2.837755in}{6.240196in}}%
\pgfpathlineto{\pgfqpoint{2.932653in}{6.240196in}}%
\pgfpathlineto{\pgfqpoint{3.027551in}{6.240196in}}%
\pgfpathlineto{\pgfqpoint{3.122449in}{6.240196in}}%
\pgfpathlineto{\pgfqpoint{3.217347in}{6.240196in}}%
\pgfpathlineto{\pgfqpoint{3.312245in}{6.240196in}}%
\pgfpathlineto{\pgfqpoint{3.407143in}{6.240196in}}%
\pgfpathlineto{\pgfqpoint{3.502041in}{6.240196in}}%
\pgfpathlineto{\pgfqpoint{3.596939in}{6.240196in}}%
\pgfpathlineto{\pgfqpoint{3.691837in}{6.240196in}}%
\pgfpathlineto{\pgfqpoint{3.786735in}{6.240196in}}%
\pgfpathlineto{\pgfqpoint{3.881633in}{6.240196in}}%
\pgfpathlineto{\pgfqpoint{3.976531in}{6.240196in}}%
\pgfpathlineto{\pgfqpoint{4.071429in}{6.240196in}}%
\pgfpathlineto{\pgfqpoint{4.166327in}{6.240196in}}%
\pgfpathlineto{\pgfqpoint{4.261224in}{6.240196in}}%
\pgfpathlineto{\pgfqpoint{4.356122in}{6.240196in}}%
\pgfpathlineto{\pgfqpoint{4.451020in}{6.240196in}}%
\pgfpathlineto{\pgfqpoint{4.545918in}{6.240196in}}%
\pgfpathlineto{\pgfqpoint{4.640816in}{6.240196in}}%
\pgfpathlineto{\pgfqpoint{4.735714in}{6.240196in}}%
\pgfpathlineto{\pgfqpoint{4.830612in}{6.240196in}}%
\pgfpathlineto{\pgfqpoint{4.925510in}{6.240196in}}%
\pgfpathlineto{\pgfqpoint{5.020408in}{6.240196in}}%
\pgfpathlineto{\pgfqpoint{5.115306in}{6.240196in}}%
\pgfpathlineto{\pgfqpoint{5.210204in}{6.240196in}}%
\pgfpathlineto{\pgfqpoint{5.305102in}{6.240196in}}%
\pgfpathlineto{\pgfqpoint{5.400000in}{6.240196in}}%
\pgfusepath{stroke}%
\end{pgfscope}%
\begin{pgfscope}%
\pgfsetrectcap%
\pgfsetmiterjoin%
\pgfsetlinewidth{0.803000pt}%
\definecolor{currentstroke}{rgb}{0.000000,0.000000,0.000000}%
\pgfsetstrokecolor{currentstroke}%
\pgfsetdash{}{0pt}%
\pgfpathmoveto{\pgfqpoint{0.750000in}{3.960000in}}%
\pgfpathlineto{\pgfqpoint{0.750000in}{7.040000in}}%
\pgfusepath{stroke}%
\end{pgfscope}%
\begin{pgfscope}%
\pgfsetrectcap%
\pgfsetmiterjoin%
\pgfsetlinewidth{0.803000pt}%
\definecolor{currentstroke}{rgb}{0.000000,0.000000,0.000000}%
\pgfsetstrokecolor{currentstroke}%
\pgfsetdash{}{0pt}%
\pgfpathmoveto{\pgfqpoint{5.400000in}{3.960000in}}%
\pgfpathlineto{\pgfqpoint{5.400000in}{7.040000in}}%
\pgfusepath{stroke}%
\end{pgfscope}%
\begin{pgfscope}%
\pgfsetrectcap%
\pgfsetmiterjoin%
\pgfsetlinewidth{0.803000pt}%
\definecolor{currentstroke}{rgb}{0.000000,0.000000,0.000000}%
\pgfsetstrokecolor{currentstroke}%
\pgfsetdash{}{0pt}%
\pgfpathmoveto{\pgfqpoint{0.750000in}{3.960000in}}%
\pgfpathlineto{\pgfqpoint{5.400000in}{3.960000in}}%
\pgfusepath{stroke}%
\end{pgfscope}%
\begin{pgfscope}%
\pgfsetrectcap%
\pgfsetmiterjoin%
\pgfsetlinewidth{0.803000pt}%
\definecolor{currentstroke}{rgb}{0.000000,0.000000,0.000000}%
\pgfsetstrokecolor{currentstroke}%
\pgfsetdash{}{0pt}%
\pgfpathmoveto{\pgfqpoint{0.750000in}{7.040000in}}%
\pgfpathlineto{\pgfqpoint{5.400000in}{7.040000in}}%
\pgfusepath{stroke}%
\end{pgfscope}%
\begin{pgfscope}%
\pgfsetbuttcap%
\pgfsetmiterjoin%
\definecolor{currentfill}{rgb}{1.000000,1.000000,1.000000}%
\pgfsetfillcolor{currentfill}%
\pgfsetfillopacity{0.800000}%
\pgfsetlinewidth{1.003750pt}%
\definecolor{currentstroke}{rgb}{0.800000,0.800000,0.800000}%
\pgfsetstrokecolor{currentstroke}%
\pgfsetstrokeopacity{0.800000}%
\pgfsetdash{}{0pt}%
\pgfpathmoveto{\pgfqpoint{2.198665in}{4.043333in}}%
\pgfpathlineto{\pgfqpoint{3.951335in}{4.043333in}}%
\pgfpathquadraticcurveto{\pgfqpoint{3.984669in}{4.043333in}}{\pgfqpoint{3.984669in}{4.076667in}}%
\pgfpathlineto{\pgfqpoint{3.984669in}{4.545921in}}%
\pgfpathquadraticcurveto{\pgfqpoint{3.984669in}{4.579254in}}{\pgfqpoint{3.951335in}{4.579254in}}%
\pgfpathlineto{\pgfqpoint{2.198665in}{4.579254in}}%
\pgfpathquadraticcurveto{\pgfqpoint{2.165331in}{4.579254in}}{\pgfqpoint{2.165331in}{4.545921in}}%
\pgfpathlineto{\pgfqpoint{2.165331in}{4.076667in}}%
\pgfpathquadraticcurveto{\pgfqpoint{2.165331in}{4.043333in}}{\pgfqpoint{2.198665in}{4.043333in}}%
\pgfpathlineto{\pgfqpoint{2.198665in}{4.043333in}}%
\pgfpathclose%
\pgfusepath{stroke,fill}%
\end{pgfscope}%
\begin{pgfscope}%
\pgfsetrectcap%
\pgfsetroundjoin%
\pgfsetlinewidth{2.007500pt}%
\definecolor{currentstroke}{rgb}{0.121569,0.466667,0.705882}%
\pgfsetstrokecolor{currentstroke}%
\pgfsetdash{}{0pt}%
\pgfpathmoveto{\pgfqpoint{2.231998in}{4.446897in}}%
\pgfpathlineto{\pgfqpoint{2.398665in}{4.446897in}}%
\pgfpathlineto{\pgfqpoint{2.565331in}{4.446897in}}%
\pgfusepath{stroke}%
\end{pgfscope}%
\begin{pgfscope}%
\definecolor{textcolor}{rgb}{0.000000,0.000000,0.000000}%
\pgfsetstrokecolor{textcolor}%
\pgfsetfillcolor{textcolor}%
\pgftext[x=2.698665in,y=4.388564in,left,base]{\color{textcolor}\rmfamily\fontsize{12.000000}{14.400000}\selectfont \(\displaystyle P_\mathrm{e}\) pre-fault}%
\end{pgfscope}%
\begin{pgfscope}%
\pgfsetrectcap%
\pgfsetroundjoin%
\pgfsetlinewidth{2.007500pt}%
\definecolor{currentstroke}{rgb}{1.000000,0.498039,0.054902}%
\pgfsetstrokecolor{currentstroke}%
\pgfsetdash{}{0pt}%
\pgfpathmoveto{\pgfqpoint{2.231998in}{4.204629in}}%
\pgfpathlineto{\pgfqpoint{2.398665in}{4.204629in}}%
\pgfpathlineto{\pgfqpoint{2.565331in}{4.204629in}}%
\pgfusepath{stroke}%
\end{pgfscope}%
\begin{pgfscope}%
\definecolor{textcolor}{rgb}{0.000000,0.000000,0.000000}%
\pgfsetstrokecolor{textcolor}%
\pgfsetfillcolor{textcolor}%
\pgftext[x=2.698665in,y=4.146295in,left,base]{\color{textcolor}\rmfamily\fontsize{12.000000}{14.400000}\selectfont \(\displaystyle P_\mathrm{T}\) of the turbine}%
\end{pgfscope}%
\begin{pgfscope}%
\pgfsetbuttcap%
\pgfsetmiterjoin%
\definecolor{currentfill}{rgb}{1.000000,1.000000,1.000000}%
\pgfsetfillcolor{currentfill}%
\pgfsetlinewidth{0.000000pt}%
\definecolor{currentstroke}{rgb}{0.000000,0.000000,0.000000}%
\pgfsetstrokecolor{currentstroke}%
\pgfsetstrokeopacity{0.000000}%
\pgfsetdash{}{0pt}%
\pgfpathmoveto{\pgfqpoint{0.750000in}{0.880000in}}%
\pgfpathlineto{\pgfqpoint{5.400000in}{0.880000in}}%
\pgfpathlineto{\pgfqpoint{5.400000in}{3.960000in}}%
\pgfpathlineto{\pgfqpoint{0.750000in}{3.960000in}}%
\pgfpathlineto{\pgfqpoint{0.750000in}{0.880000in}}%
\pgfpathclose%
\pgfusepath{fill}%
\end{pgfscope}%
\begin{pgfscope}%
\pgfpathrectangle{\pgfqpoint{0.750000in}{0.880000in}}{\pgfqpoint{4.650000in}{3.080000in}}%
\pgfusepath{clip}%
\pgfsetrectcap%
\pgfsetroundjoin%
\pgfsetlinewidth{0.803000pt}%
\definecolor{currentstroke}{rgb}{0.690196,0.690196,0.690196}%
\pgfsetstrokecolor{currentstroke}%
\pgfsetdash{}{0pt}%
\pgfpathmoveto{\pgfqpoint{0.750000in}{0.880000in}}%
\pgfpathlineto{\pgfqpoint{0.750000in}{3.960000in}}%
\pgfusepath{stroke}%
\end{pgfscope}%
\begin{pgfscope}%
\pgfsetbuttcap%
\pgfsetroundjoin%
\definecolor{currentfill}{rgb}{0.000000,0.000000,0.000000}%
\pgfsetfillcolor{currentfill}%
\pgfsetlinewidth{0.803000pt}%
\definecolor{currentstroke}{rgb}{0.000000,0.000000,0.000000}%
\pgfsetstrokecolor{currentstroke}%
\pgfsetdash{}{0pt}%
\pgfsys@defobject{currentmarker}{\pgfqpoint{0.000000in}{-0.048611in}}{\pgfqpoint{0.000000in}{0.000000in}}{%
\pgfpathmoveto{\pgfqpoint{0.000000in}{0.000000in}}%
\pgfpathlineto{\pgfqpoint{0.000000in}{-0.048611in}}%
\pgfusepath{stroke,fill}%
}%
\begin{pgfscope}%
\pgfsys@transformshift{0.750000in}{0.880000in}%
\pgfsys@useobject{currentmarker}{}%
\end{pgfscope}%
\end{pgfscope}%
\begin{pgfscope}%
\definecolor{textcolor}{rgb}{0.000000,0.000000,0.000000}%
\pgfsetstrokecolor{textcolor}%
\pgfsetfillcolor{textcolor}%
\pgftext[x=0.750000in,y=0.782778in,,top]{\color{textcolor}\rmfamily\fontsize{12.000000}{14.400000}\selectfont \(\displaystyle {0}\)}%
\end{pgfscope}%
\begin{pgfscope}%
\pgfpathrectangle{\pgfqpoint{0.750000in}{0.880000in}}{\pgfqpoint{4.650000in}{3.080000in}}%
\pgfusepath{clip}%
\pgfsetrectcap%
\pgfsetroundjoin%
\pgfsetlinewidth{0.803000pt}%
\definecolor{currentstroke}{rgb}{0.690196,0.690196,0.690196}%
\pgfsetstrokecolor{currentstroke}%
\pgfsetdash{}{0pt}%
\pgfpathmoveto{\pgfqpoint{1.266667in}{0.880000in}}%
\pgfpathlineto{\pgfqpoint{1.266667in}{3.960000in}}%
\pgfusepath{stroke}%
\end{pgfscope}%
\begin{pgfscope}%
\pgfsetbuttcap%
\pgfsetroundjoin%
\definecolor{currentfill}{rgb}{0.000000,0.000000,0.000000}%
\pgfsetfillcolor{currentfill}%
\pgfsetlinewidth{0.803000pt}%
\definecolor{currentstroke}{rgb}{0.000000,0.000000,0.000000}%
\pgfsetstrokecolor{currentstroke}%
\pgfsetdash{}{0pt}%
\pgfsys@defobject{currentmarker}{\pgfqpoint{0.000000in}{-0.048611in}}{\pgfqpoint{0.000000in}{0.000000in}}{%
\pgfpathmoveto{\pgfqpoint{0.000000in}{0.000000in}}%
\pgfpathlineto{\pgfqpoint{0.000000in}{-0.048611in}}%
\pgfusepath{stroke,fill}%
}%
\begin{pgfscope}%
\pgfsys@transformshift{1.266667in}{0.880000in}%
\pgfsys@useobject{currentmarker}{}%
\end{pgfscope}%
\end{pgfscope}%
\begin{pgfscope}%
\definecolor{textcolor}{rgb}{0.000000,0.000000,0.000000}%
\pgfsetstrokecolor{textcolor}%
\pgfsetfillcolor{textcolor}%
\pgftext[x=1.266667in,y=0.782778in,,top]{\color{textcolor}\rmfamily\fontsize{12.000000}{14.400000}\selectfont \(\displaystyle {20}\)}%
\end{pgfscope}%
\begin{pgfscope}%
\pgfpathrectangle{\pgfqpoint{0.750000in}{0.880000in}}{\pgfqpoint{4.650000in}{3.080000in}}%
\pgfusepath{clip}%
\pgfsetrectcap%
\pgfsetroundjoin%
\pgfsetlinewidth{0.803000pt}%
\definecolor{currentstroke}{rgb}{0.690196,0.690196,0.690196}%
\pgfsetstrokecolor{currentstroke}%
\pgfsetdash{}{0pt}%
\pgfpathmoveto{\pgfqpoint{1.783333in}{0.880000in}}%
\pgfpathlineto{\pgfqpoint{1.783333in}{3.960000in}}%
\pgfusepath{stroke}%
\end{pgfscope}%
\begin{pgfscope}%
\pgfsetbuttcap%
\pgfsetroundjoin%
\definecolor{currentfill}{rgb}{0.000000,0.000000,0.000000}%
\pgfsetfillcolor{currentfill}%
\pgfsetlinewidth{0.803000pt}%
\definecolor{currentstroke}{rgb}{0.000000,0.000000,0.000000}%
\pgfsetstrokecolor{currentstroke}%
\pgfsetdash{}{0pt}%
\pgfsys@defobject{currentmarker}{\pgfqpoint{0.000000in}{-0.048611in}}{\pgfqpoint{0.000000in}{0.000000in}}{%
\pgfpathmoveto{\pgfqpoint{0.000000in}{0.000000in}}%
\pgfpathlineto{\pgfqpoint{0.000000in}{-0.048611in}}%
\pgfusepath{stroke,fill}%
}%
\begin{pgfscope}%
\pgfsys@transformshift{1.783333in}{0.880000in}%
\pgfsys@useobject{currentmarker}{}%
\end{pgfscope}%
\end{pgfscope}%
\begin{pgfscope}%
\definecolor{textcolor}{rgb}{0.000000,0.000000,0.000000}%
\pgfsetstrokecolor{textcolor}%
\pgfsetfillcolor{textcolor}%
\pgftext[x=1.783333in,y=0.782778in,,top]{\color{textcolor}\rmfamily\fontsize{12.000000}{14.400000}\selectfont \(\displaystyle {40}\)}%
\end{pgfscope}%
\begin{pgfscope}%
\pgfpathrectangle{\pgfqpoint{0.750000in}{0.880000in}}{\pgfqpoint{4.650000in}{3.080000in}}%
\pgfusepath{clip}%
\pgfsetrectcap%
\pgfsetroundjoin%
\pgfsetlinewidth{0.803000pt}%
\definecolor{currentstroke}{rgb}{0.690196,0.690196,0.690196}%
\pgfsetstrokecolor{currentstroke}%
\pgfsetdash{}{0pt}%
\pgfpathmoveto{\pgfqpoint{2.300000in}{0.880000in}}%
\pgfpathlineto{\pgfqpoint{2.300000in}{3.960000in}}%
\pgfusepath{stroke}%
\end{pgfscope}%
\begin{pgfscope}%
\pgfsetbuttcap%
\pgfsetroundjoin%
\definecolor{currentfill}{rgb}{0.000000,0.000000,0.000000}%
\pgfsetfillcolor{currentfill}%
\pgfsetlinewidth{0.803000pt}%
\definecolor{currentstroke}{rgb}{0.000000,0.000000,0.000000}%
\pgfsetstrokecolor{currentstroke}%
\pgfsetdash{}{0pt}%
\pgfsys@defobject{currentmarker}{\pgfqpoint{0.000000in}{-0.048611in}}{\pgfqpoint{0.000000in}{0.000000in}}{%
\pgfpathmoveto{\pgfqpoint{0.000000in}{0.000000in}}%
\pgfpathlineto{\pgfqpoint{0.000000in}{-0.048611in}}%
\pgfusepath{stroke,fill}%
}%
\begin{pgfscope}%
\pgfsys@transformshift{2.300000in}{0.880000in}%
\pgfsys@useobject{currentmarker}{}%
\end{pgfscope}%
\end{pgfscope}%
\begin{pgfscope}%
\definecolor{textcolor}{rgb}{0.000000,0.000000,0.000000}%
\pgfsetstrokecolor{textcolor}%
\pgfsetfillcolor{textcolor}%
\pgftext[x=2.300000in,y=0.782778in,,top]{\color{textcolor}\rmfamily\fontsize{12.000000}{14.400000}\selectfont \(\displaystyle {60}\)}%
\end{pgfscope}%
\begin{pgfscope}%
\pgfpathrectangle{\pgfqpoint{0.750000in}{0.880000in}}{\pgfqpoint{4.650000in}{3.080000in}}%
\pgfusepath{clip}%
\pgfsetrectcap%
\pgfsetroundjoin%
\pgfsetlinewidth{0.803000pt}%
\definecolor{currentstroke}{rgb}{0.690196,0.690196,0.690196}%
\pgfsetstrokecolor{currentstroke}%
\pgfsetdash{}{0pt}%
\pgfpathmoveto{\pgfqpoint{2.816667in}{0.880000in}}%
\pgfpathlineto{\pgfqpoint{2.816667in}{3.960000in}}%
\pgfusepath{stroke}%
\end{pgfscope}%
\begin{pgfscope}%
\pgfsetbuttcap%
\pgfsetroundjoin%
\definecolor{currentfill}{rgb}{0.000000,0.000000,0.000000}%
\pgfsetfillcolor{currentfill}%
\pgfsetlinewidth{0.803000pt}%
\definecolor{currentstroke}{rgb}{0.000000,0.000000,0.000000}%
\pgfsetstrokecolor{currentstroke}%
\pgfsetdash{}{0pt}%
\pgfsys@defobject{currentmarker}{\pgfqpoint{0.000000in}{-0.048611in}}{\pgfqpoint{0.000000in}{0.000000in}}{%
\pgfpathmoveto{\pgfqpoint{0.000000in}{0.000000in}}%
\pgfpathlineto{\pgfqpoint{0.000000in}{-0.048611in}}%
\pgfusepath{stroke,fill}%
}%
\begin{pgfscope}%
\pgfsys@transformshift{2.816667in}{0.880000in}%
\pgfsys@useobject{currentmarker}{}%
\end{pgfscope}%
\end{pgfscope}%
\begin{pgfscope}%
\definecolor{textcolor}{rgb}{0.000000,0.000000,0.000000}%
\pgfsetstrokecolor{textcolor}%
\pgfsetfillcolor{textcolor}%
\pgftext[x=2.816667in,y=0.782778in,,top]{\color{textcolor}\rmfamily\fontsize{12.000000}{14.400000}\selectfont \(\displaystyle {80}\)}%
\end{pgfscope}%
\begin{pgfscope}%
\pgfpathrectangle{\pgfqpoint{0.750000in}{0.880000in}}{\pgfqpoint{4.650000in}{3.080000in}}%
\pgfusepath{clip}%
\pgfsetrectcap%
\pgfsetroundjoin%
\pgfsetlinewidth{0.803000pt}%
\definecolor{currentstroke}{rgb}{0.690196,0.690196,0.690196}%
\pgfsetstrokecolor{currentstroke}%
\pgfsetdash{}{0pt}%
\pgfpathmoveto{\pgfqpoint{3.333333in}{0.880000in}}%
\pgfpathlineto{\pgfqpoint{3.333333in}{3.960000in}}%
\pgfusepath{stroke}%
\end{pgfscope}%
\begin{pgfscope}%
\pgfsetbuttcap%
\pgfsetroundjoin%
\definecolor{currentfill}{rgb}{0.000000,0.000000,0.000000}%
\pgfsetfillcolor{currentfill}%
\pgfsetlinewidth{0.803000pt}%
\definecolor{currentstroke}{rgb}{0.000000,0.000000,0.000000}%
\pgfsetstrokecolor{currentstroke}%
\pgfsetdash{}{0pt}%
\pgfsys@defobject{currentmarker}{\pgfqpoint{0.000000in}{-0.048611in}}{\pgfqpoint{0.000000in}{0.000000in}}{%
\pgfpathmoveto{\pgfqpoint{0.000000in}{0.000000in}}%
\pgfpathlineto{\pgfqpoint{0.000000in}{-0.048611in}}%
\pgfusepath{stroke,fill}%
}%
\begin{pgfscope}%
\pgfsys@transformshift{3.333333in}{0.880000in}%
\pgfsys@useobject{currentmarker}{}%
\end{pgfscope}%
\end{pgfscope}%
\begin{pgfscope}%
\definecolor{textcolor}{rgb}{0.000000,0.000000,0.000000}%
\pgfsetstrokecolor{textcolor}%
\pgfsetfillcolor{textcolor}%
\pgftext[x=3.333333in,y=0.782778in,,top]{\color{textcolor}\rmfamily\fontsize{12.000000}{14.400000}\selectfont \(\displaystyle {100}\)}%
\end{pgfscope}%
\begin{pgfscope}%
\pgfpathrectangle{\pgfqpoint{0.750000in}{0.880000in}}{\pgfqpoint{4.650000in}{3.080000in}}%
\pgfusepath{clip}%
\pgfsetrectcap%
\pgfsetroundjoin%
\pgfsetlinewidth{0.803000pt}%
\definecolor{currentstroke}{rgb}{0.690196,0.690196,0.690196}%
\pgfsetstrokecolor{currentstroke}%
\pgfsetdash{}{0pt}%
\pgfpathmoveto{\pgfqpoint{3.850000in}{0.880000in}}%
\pgfpathlineto{\pgfqpoint{3.850000in}{3.960000in}}%
\pgfusepath{stroke}%
\end{pgfscope}%
\begin{pgfscope}%
\pgfsetbuttcap%
\pgfsetroundjoin%
\definecolor{currentfill}{rgb}{0.000000,0.000000,0.000000}%
\pgfsetfillcolor{currentfill}%
\pgfsetlinewidth{0.803000pt}%
\definecolor{currentstroke}{rgb}{0.000000,0.000000,0.000000}%
\pgfsetstrokecolor{currentstroke}%
\pgfsetdash{}{0pt}%
\pgfsys@defobject{currentmarker}{\pgfqpoint{0.000000in}{-0.048611in}}{\pgfqpoint{0.000000in}{0.000000in}}{%
\pgfpathmoveto{\pgfqpoint{0.000000in}{0.000000in}}%
\pgfpathlineto{\pgfqpoint{0.000000in}{-0.048611in}}%
\pgfusepath{stroke,fill}%
}%
\begin{pgfscope}%
\pgfsys@transformshift{3.850000in}{0.880000in}%
\pgfsys@useobject{currentmarker}{}%
\end{pgfscope}%
\end{pgfscope}%
\begin{pgfscope}%
\definecolor{textcolor}{rgb}{0.000000,0.000000,0.000000}%
\pgfsetstrokecolor{textcolor}%
\pgfsetfillcolor{textcolor}%
\pgftext[x=3.850000in,y=0.782778in,,top]{\color{textcolor}\rmfamily\fontsize{12.000000}{14.400000}\selectfont \(\displaystyle {120}\)}%
\end{pgfscope}%
\begin{pgfscope}%
\pgfpathrectangle{\pgfqpoint{0.750000in}{0.880000in}}{\pgfqpoint{4.650000in}{3.080000in}}%
\pgfusepath{clip}%
\pgfsetrectcap%
\pgfsetroundjoin%
\pgfsetlinewidth{0.803000pt}%
\definecolor{currentstroke}{rgb}{0.690196,0.690196,0.690196}%
\pgfsetstrokecolor{currentstroke}%
\pgfsetdash{}{0pt}%
\pgfpathmoveto{\pgfqpoint{4.366667in}{0.880000in}}%
\pgfpathlineto{\pgfqpoint{4.366667in}{3.960000in}}%
\pgfusepath{stroke}%
\end{pgfscope}%
\begin{pgfscope}%
\pgfsetbuttcap%
\pgfsetroundjoin%
\definecolor{currentfill}{rgb}{0.000000,0.000000,0.000000}%
\pgfsetfillcolor{currentfill}%
\pgfsetlinewidth{0.803000pt}%
\definecolor{currentstroke}{rgb}{0.000000,0.000000,0.000000}%
\pgfsetstrokecolor{currentstroke}%
\pgfsetdash{}{0pt}%
\pgfsys@defobject{currentmarker}{\pgfqpoint{0.000000in}{-0.048611in}}{\pgfqpoint{0.000000in}{0.000000in}}{%
\pgfpathmoveto{\pgfqpoint{0.000000in}{0.000000in}}%
\pgfpathlineto{\pgfqpoint{0.000000in}{-0.048611in}}%
\pgfusepath{stroke,fill}%
}%
\begin{pgfscope}%
\pgfsys@transformshift{4.366667in}{0.880000in}%
\pgfsys@useobject{currentmarker}{}%
\end{pgfscope}%
\end{pgfscope}%
\begin{pgfscope}%
\definecolor{textcolor}{rgb}{0.000000,0.000000,0.000000}%
\pgfsetstrokecolor{textcolor}%
\pgfsetfillcolor{textcolor}%
\pgftext[x=4.366667in,y=0.782778in,,top]{\color{textcolor}\rmfamily\fontsize{12.000000}{14.400000}\selectfont \(\displaystyle {140}\)}%
\end{pgfscope}%
\begin{pgfscope}%
\pgfpathrectangle{\pgfqpoint{0.750000in}{0.880000in}}{\pgfqpoint{4.650000in}{3.080000in}}%
\pgfusepath{clip}%
\pgfsetrectcap%
\pgfsetroundjoin%
\pgfsetlinewidth{0.803000pt}%
\definecolor{currentstroke}{rgb}{0.690196,0.690196,0.690196}%
\pgfsetstrokecolor{currentstroke}%
\pgfsetdash{}{0pt}%
\pgfpathmoveto{\pgfqpoint{4.883333in}{0.880000in}}%
\pgfpathlineto{\pgfqpoint{4.883333in}{3.960000in}}%
\pgfusepath{stroke}%
\end{pgfscope}%
\begin{pgfscope}%
\pgfsetbuttcap%
\pgfsetroundjoin%
\definecolor{currentfill}{rgb}{0.000000,0.000000,0.000000}%
\pgfsetfillcolor{currentfill}%
\pgfsetlinewidth{0.803000pt}%
\definecolor{currentstroke}{rgb}{0.000000,0.000000,0.000000}%
\pgfsetstrokecolor{currentstroke}%
\pgfsetdash{}{0pt}%
\pgfsys@defobject{currentmarker}{\pgfqpoint{0.000000in}{-0.048611in}}{\pgfqpoint{0.000000in}{0.000000in}}{%
\pgfpathmoveto{\pgfqpoint{0.000000in}{0.000000in}}%
\pgfpathlineto{\pgfqpoint{0.000000in}{-0.048611in}}%
\pgfusepath{stroke,fill}%
}%
\begin{pgfscope}%
\pgfsys@transformshift{4.883333in}{0.880000in}%
\pgfsys@useobject{currentmarker}{}%
\end{pgfscope}%
\end{pgfscope}%
\begin{pgfscope}%
\definecolor{textcolor}{rgb}{0.000000,0.000000,0.000000}%
\pgfsetstrokecolor{textcolor}%
\pgfsetfillcolor{textcolor}%
\pgftext[x=4.883333in,y=0.782778in,,top]{\color{textcolor}\rmfamily\fontsize{12.000000}{14.400000}\selectfont \(\displaystyle {160}\)}%
\end{pgfscope}%
\begin{pgfscope}%
\pgfpathrectangle{\pgfqpoint{0.750000in}{0.880000in}}{\pgfqpoint{4.650000in}{3.080000in}}%
\pgfusepath{clip}%
\pgfsetrectcap%
\pgfsetroundjoin%
\pgfsetlinewidth{0.803000pt}%
\definecolor{currentstroke}{rgb}{0.690196,0.690196,0.690196}%
\pgfsetstrokecolor{currentstroke}%
\pgfsetdash{}{0pt}%
\pgfpathmoveto{\pgfqpoint{5.400000in}{0.880000in}}%
\pgfpathlineto{\pgfqpoint{5.400000in}{3.960000in}}%
\pgfusepath{stroke}%
\end{pgfscope}%
\begin{pgfscope}%
\pgfsetbuttcap%
\pgfsetroundjoin%
\definecolor{currentfill}{rgb}{0.000000,0.000000,0.000000}%
\pgfsetfillcolor{currentfill}%
\pgfsetlinewidth{0.803000pt}%
\definecolor{currentstroke}{rgb}{0.000000,0.000000,0.000000}%
\pgfsetstrokecolor{currentstroke}%
\pgfsetdash{}{0pt}%
\pgfsys@defobject{currentmarker}{\pgfqpoint{0.000000in}{-0.048611in}}{\pgfqpoint{0.000000in}{0.000000in}}{%
\pgfpathmoveto{\pgfqpoint{0.000000in}{0.000000in}}%
\pgfpathlineto{\pgfqpoint{0.000000in}{-0.048611in}}%
\pgfusepath{stroke,fill}%
}%
\begin{pgfscope}%
\pgfsys@transformshift{5.400000in}{0.880000in}%
\pgfsys@useobject{currentmarker}{}%
\end{pgfscope}%
\end{pgfscope}%
\begin{pgfscope}%
\definecolor{textcolor}{rgb}{0.000000,0.000000,0.000000}%
\pgfsetstrokecolor{textcolor}%
\pgfsetfillcolor{textcolor}%
\pgftext[x=5.400000in,y=0.782778in,,top]{\color{textcolor}\rmfamily\fontsize{12.000000}{14.400000}\selectfont \(\displaystyle {180}\)}%
\end{pgfscope}%
\begin{pgfscope}%
\definecolor{textcolor}{rgb}{0.000000,0.000000,0.000000}%
\pgfsetstrokecolor{textcolor}%
\pgfsetfillcolor{textcolor}%
\pgftext[x=3.075000in,y=0.568287in,,top]{\color{textcolor}\rmfamily\fontsize{12.000000}{14.400000}\selectfont power angle \(\displaystyle \delta\) in deg}%
\end{pgfscope}%
\begin{pgfscope}%
\pgfpathrectangle{\pgfqpoint{0.750000in}{0.880000in}}{\pgfqpoint{4.650000in}{3.080000in}}%
\pgfusepath{clip}%
\pgfsetrectcap%
\pgfsetroundjoin%
\pgfsetlinewidth{0.803000pt}%
\definecolor{currentstroke}{rgb}{0.690196,0.690196,0.690196}%
\pgfsetstrokecolor{currentstroke}%
\pgfsetdash{}{0pt}%
\pgfpathmoveto{\pgfqpoint{0.750000in}{3.703109in}}%
\pgfpathlineto{\pgfqpoint{5.400000in}{3.703109in}}%
\pgfusepath{stroke}%
\end{pgfscope}%
\begin{pgfscope}%
\pgfsetbuttcap%
\pgfsetroundjoin%
\definecolor{currentfill}{rgb}{0.000000,0.000000,0.000000}%
\pgfsetfillcolor{currentfill}%
\pgfsetlinewidth{0.803000pt}%
\definecolor{currentstroke}{rgb}{0.000000,0.000000,0.000000}%
\pgfsetstrokecolor{currentstroke}%
\pgfsetdash{}{0pt}%
\pgfsys@defobject{currentmarker}{\pgfqpoint{-0.048611in}{0.000000in}}{\pgfqpoint{-0.000000in}{0.000000in}}{%
\pgfpathmoveto{\pgfqpoint{-0.000000in}{0.000000in}}%
\pgfpathlineto{\pgfqpoint{-0.048611in}{0.000000in}}%
\pgfusepath{stroke,fill}%
}%
\begin{pgfscope}%
\pgfsys@transformshift{0.750000in}{3.703109in}%
\pgfsys@useobject{currentmarker}{}%
\end{pgfscope}%
\end{pgfscope}%
\begin{pgfscope}%
\definecolor{textcolor}{rgb}{0.000000,0.000000,0.000000}%
\pgfsetstrokecolor{textcolor}%
\pgfsetfillcolor{textcolor}%
\pgftext[x=0.444254in, y=3.641789in, left, base]{\color{textcolor}\rmfamily\fontsize{12.000000}{14.400000}\selectfont \(\displaystyle {0.0}\)}%
\end{pgfscope}%
\begin{pgfscope}%
\pgfpathrectangle{\pgfqpoint{0.750000in}{0.880000in}}{\pgfqpoint{4.650000in}{3.080000in}}%
\pgfusepath{clip}%
\pgfsetrectcap%
\pgfsetroundjoin%
\pgfsetlinewidth{0.803000pt}%
\definecolor{currentstroke}{rgb}{0.690196,0.690196,0.690196}%
\pgfsetstrokecolor{currentstroke}%
\pgfsetdash{}{0pt}%
\pgfpathmoveto{\pgfqpoint{0.750000in}{3.189326in}}%
\pgfpathlineto{\pgfqpoint{5.400000in}{3.189326in}}%
\pgfusepath{stroke}%
\end{pgfscope}%
\begin{pgfscope}%
\pgfsetbuttcap%
\pgfsetroundjoin%
\definecolor{currentfill}{rgb}{0.000000,0.000000,0.000000}%
\pgfsetfillcolor{currentfill}%
\pgfsetlinewidth{0.803000pt}%
\definecolor{currentstroke}{rgb}{0.000000,0.000000,0.000000}%
\pgfsetstrokecolor{currentstroke}%
\pgfsetdash{}{0pt}%
\pgfsys@defobject{currentmarker}{\pgfqpoint{-0.048611in}{0.000000in}}{\pgfqpoint{-0.000000in}{0.000000in}}{%
\pgfpathmoveto{\pgfqpoint{-0.000000in}{0.000000in}}%
\pgfpathlineto{\pgfqpoint{-0.048611in}{0.000000in}}%
\pgfusepath{stroke,fill}%
}%
\begin{pgfscope}%
\pgfsys@transformshift{0.750000in}{3.189326in}%
\pgfsys@useobject{currentmarker}{}%
\end{pgfscope}%
\end{pgfscope}%
\begin{pgfscope}%
\definecolor{textcolor}{rgb}{0.000000,0.000000,0.000000}%
\pgfsetstrokecolor{textcolor}%
\pgfsetfillcolor{textcolor}%
\pgftext[x=0.444254in, y=3.128006in, left, base]{\color{textcolor}\rmfamily\fontsize{12.000000}{14.400000}\selectfont \(\displaystyle {0.2}\)}%
\end{pgfscope}%
\begin{pgfscope}%
\pgfpathrectangle{\pgfqpoint{0.750000in}{0.880000in}}{\pgfqpoint{4.650000in}{3.080000in}}%
\pgfusepath{clip}%
\pgfsetrectcap%
\pgfsetroundjoin%
\pgfsetlinewidth{0.803000pt}%
\definecolor{currentstroke}{rgb}{0.690196,0.690196,0.690196}%
\pgfsetstrokecolor{currentstroke}%
\pgfsetdash{}{0pt}%
\pgfpathmoveto{\pgfqpoint{0.750000in}{2.675543in}}%
\pgfpathlineto{\pgfqpoint{5.400000in}{2.675543in}}%
\pgfusepath{stroke}%
\end{pgfscope}%
\begin{pgfscope}%
\pgfsetbuttcap%
\pgfsetroundjoin%
\definecolor{currentfill}{rgb}{0.000000,0.000000,0.000000}%
\pgfsetfillcolor{currentfill}%
\pgfsetlinewidth{0.803000pt}%
\definecolor{currentstroke}{rgb}{0.000000,0.000000,0.000000}%
\pgfsetstrokecolor{currentstroke}%
\pgfsetdash{}{0pt}%
\pgfsys@defobject{currentmarker}{\pgfqpoint{-0.048611in}{0.000000in}}{\pgfqpoint{-0.000000in}{0.000000in}}{%
\pgfpathmoveto{\pgfqpoint{-0.000000in}{0.000000in}}%
\pgfpathlineto{\pgfqpoint{-0.048611in}{0.000000in}}%
\pgfusepath{stroke,fill}%
}%
\begin{pgfscope}%
\pgfsys@transformshift{0.750000in}{2.675543in}%
\pgfsys@useobject{currentmarker}{}%
\end{pgfscope}%
\end{pgfscope}%
\begin{pgfscope}%
\definecolor{textcolor}{rgb}{0.000000,0.000000,0.000000}%
\pgfsetstrokecolor{textcolor}%
\pgfsetfillcolor{textcolor}%
\pgftext[x=0.444254in, y=2.614223in, left, base]{\color{textcolor}\rmfamily\fontsize{12.000000}{14.400000}\selectfont \(\displaystyle {0.4}\)}%
\end{pgfscope}%
\begin{pgfscope}%
\pgfpathrectangle{\pgfqpoint{0.750000in}{0.880000in}}{\pgfqpoint{4.650000in}{3.080000in}}%
\pgfusepath{clip}%
\pgfsetrectcap%
\pgfsetroundjoin%
\pgfsetlinewidth{0.803000pt}%
\definecolor{currentstroke}{rgb}{0.690196,0.690196,0.690196}%
\pgfsetstrokecolor{currentstroke}%
\pgfsetdash{}{0pt}%
\pgfpathmoveto{\pgfqpoint{0.750000in}{2.161760in}}%
\pgfpathlineto{\pgfqpoint{5.400000in}{2.161760in}}%
\pgfusepath{stroke}%
\end{pgfscope}%
\begin{pgfscope}%
\pgfsetbuttcap%
\pgfsetroundjoin%
\definecolor{currentfill}{rgb}{0.000000,0.000000,0.000000}%
\pgfsetfillcolor{currentfill}%
\pgfsetlinewidth{0.803000pt}%
\definecolor{currentstroke}{rgb}{0.000000,0.000000,0.000000}%
\pgfsetstrokecolor{currentstroke}%
\pgfsetdash{}{0pt}%
\pgfsys@defobject{currentmarker}{\pgfqpoint{-0.048611in}{0.000000in}}{\pgfqpoint{-0.000000in}{0.000000in}}{%
\pgfpathmoveto{\pgfqpoint{-0.000000in}{0.000000in}}%
\pgfpathlineto{\pgfqpoint{-0.048611in}{0.000000in}}%
\pgfusepath{stroke,fill}%
}%
\begin{pgfscope}%
\pgfsys@transformshift{0.750000in}{2.161760in}%
\pgfsys@useobject{currentmarker}{}%
\end{pgfscope}%
\end{pgfscope}%
\begin{pgfscope}%
\definecolor{textcolor}{rgb}{0.000000,0.000000,0.000000}%
\pgfsetstrokecolor{textcolor}%
\pgfsetfillcolor{textcolor}%
\pgftext[x=0.444254in, y=2.100440in, left, base]{\color{textcolor}\rmfamily\fontsize{12.000000}{14.400000}\selectfont \(\displaystyle {0.6}\)}%
\end{pgfscope}%
\begin{pgfscope}%
\pgfpathrectangle{\pgfqpoint{0.750000in}{0.880000in}}{\pgfqpoint{4.650000in}{3.080000in}}%
\pgfusepath{clip}%
\pgfsetrectcap%
\pgfsetroundjoin%
\pgfsetlinewidth{0.803000pt}%
\definecolor{currentstroke}{rgb}{0.690196,0.690196,0.690196}%
\pgfsetstrokecolor{currentstroke}%
\pgfsetdash{}{0pt}%
\pgfpathmoveto{\pgfqpoint{0.750000in}{1.647977in}}%
\pgfpathlineto{\pgfqpoint{5.400000in}{1.647977in}}%
\pgfusepath{stroke}%
\end{pgfscope}%
\begin{pgfscope}%
\pgfsetbuttcap%
\pgfsetroundjoin%
\definecolor{currentfill}{rgb}{0.000000,0.000000,0.000000}%
\pgfsetfillcolor{currentfill}%
\pgfsetlinewidth{0.803000pt}%
\definecolor{currentstroke}{rgb}{0.000000,0.000000,0.000000}%
\pgfsetstrokecolor{currentstroke}%
\pgfsetdash{}{0pt}%
\pgfsys@defobject{currentmarker}{\pgfqpoint{-0.048611in}{0.000000in}}{\pgfqpoint{-0.000000in}{0.000000in}}{%
\pgfpathmoveto{\pgfqpoint{-0.000000in}{0.000000in}}%
\pgfpathlineto{\pgfqpoint{-0.048611in}{0.000000in}}%
\pgfusepath{stroke,fill}%
}%
\begin{pgfscope}%
\pgfsys@transformshift{0.750000in}{1.647977in}%
\pgfsys@useobject{currentmarker}{}%
\end{pgfscope}%
\end{pgfscope}%
\begin{pgfscope}%
\definecolor{textcolor}{rgb}{0.000000,0.000000,0.000000}%
\pgfsetstrokecolor{textcolor}%
\pgfsetfillcolor{textcolor}%
\pgftext[x=0.444254in, y=1.586657in, left, base]{\color{textcolor}\rmfamily\fontsize{12.000000}{14.400000}\selectfont \(\displaystyle {0.8}\)}%
\end{pgfscope}%
\begin{pgfscope}%
\pgfpathrectangle{\pgfqpoint{0.750000in}{0.880000in}}{\pgfqpoint{4.650000in}{3.080000in}}%
\pgfusepath{clip}%
\pgfsetrectcap%
\pgfsetroundjoin%
\pgfsetlinewidth{0.803000pt}%
\definecolor{currentstroke}{rgb}{0.690196,0.690196,0.690196}%
\pgfsetstrokecolor{currentstroke}%
\pgfsetdash{}{0pt}%
\pgfpathmoveto{\pgfqpoint{0.750000in}{1.134194in}}%
\pgfpathlineto{\pgfqpoint{5.400000in}{1.134194in}}%
\pgfusepath{stroke}%
\end{pgfscope}%
\begin{pgfscope}%
\pgfsetbuttcap%
\pgfsetroundjoin%
\definecolor{currentfill}{rgb}{0.000000,0.000000,0.000000}%
\pgfsetfillcolor{currentfill}%
\pgfsetlinewidth{0.803000pt}%
\definecolor{currentstroke}{rgb}{0.000000,0.000000,0.000000}%
\pgfsetstrokecolor{currentstroke}%
\pgfsetdash{}{0pt}%
\pgfsys@defobject{currentmarker}{\pgfqpoint{-0.048611in}{0.000000in}}{\pgfqpoint{-0.000000in}{0.000000in}}{%
\pgfpathmoveto{\pgfqpoint{-0.000000in}{0.000000in}}%
\pgfpathlineto{\pgfqpoint{-0.048611in}{0.000000in}}%
\pgfusepath{stroke,fill}%
}%
\begin{pgfscope}%
\pgfsys@transformshift{0.750000in}{1.134194in}%
\pgfsys@useobject{currentmarker}{}%
\end{pgfscope}%
\end{pgfscope}%
\begin{pgfscope}%
\definecolor{textcolor}{rgb}{0.000000,0.000000,0.000000}%
\pgfsetstrokecolor{textcolor}%
\pgfsetfillcolor{textcolor}%
\pgftext[x=0.444254in, y=1.072874in, left, base]{\color{textcolor}\rmfamily\fontsize{12.000000}{14.400000}\selectfont \(\displaystyle {1.0}\)}%
\end{pgfscope}%
\begin{pgfscope}%
\definecolor{textcolor}{rgb}{0.000000,0.000000,0.000000}%
\pgfsetstrokecolor{textcolor}%
\pgfsetfillcolor{textcolor}%
\pgftext[x=0.388698in,y=2.420000in,,bottom,rotate=90.000000]{\color{textcolor}\rmfamily\fontsize{12.000000}{14.400000}\selectfont time in s}%
\end{pgfscope}%
\begin{pgfscope}%
\pgfpathrectangle{\pgfqpoint{0.750000in}{0.880000in}}{\pgfqpoint{4.650000in}{3.080000in}}%
\pgfusepath{clip}%
\pgfsetrectcap%
\pgfsetroundjoin%
\pgfsetlinewidth{1.505625pt}%
\definecolor{currentstroke}{rgb}{0.121569,0.466667,0.705882}%
\pgfsetstrokecolor{currentstroke}%
\pgfsetdash{}{0pt}%
\pgfpathmoveto{\pgfqpoint{2.065441in}{3.970000in}}%
\pgfpathlineto{\pgfqpoint{2.065627in}{3.692833in}}%
\pgfpathlineto{\pgfqpoint{2.068667in}{3.677419in}}%
\pgfpathlineto{\pgfqpoint{2.074310in}{3.662006in}}%
\pgfpathlineto{\pgfqpoint{2.082553in}{3.646592in}}%
\pgfpathlineto{\pgfqpoint{2.093395in}{3.631179in}}%
\pgfpathlineto{\pgfqpoint{2.106834in}{3.615765in}}%
\pgfpathlineto{\pgfqpoint{2.125792in}{3.597783in}}%
\pgfpathlineto{\pgfqpoint{2.148279in}{3.579801in}}%
\pgfpathlineto{\pgfqpoint{2.174291in}{3.561818in}}%
\pgfpathlineto{\pgfqpoint{2.208331in}{3.541267in}}%
\pgfpathlineto{\pgfqpoint{2.246967in}{3.520716in}}%
\pgfpathlineto{\pgfqpoint{2.290194in}{3.500164in}}%
\pgfpathlineto{\pgfqpoint{2.344307in}{3.477044in}}%
\pgfpathlineto{\pgfqpoint{2.404217in}{3.453924in}}%
\pgfpathlineto{\pgfqpoint{2.477193in}{3.428235in}}%
\pgfpathlineto{\pgfqpoint{2.629968in}{3.374288in}}%
\pgfpathlineto{\pgfqpoint{2.761602in}{3.325478in}}%
\pgfpathlineto{\pgfqpoint{2.879821in}{3.279238in}}%
\pgfpathlineto{\pgfqpoint{2.985305in}{3.235566in}}%
\pgfpathlineto{\pgfqpoint{3.078951in}{3.194463in}}%
\pgfpathlineto{\pgfqpoint{3.167096in}{3.153361in}}%
\pgfpathlineto{\pgfqpoint{3.249797in}{3.112258in}}%
\pgfpathlineto{\pgfqpoint{3.322499in}{3.073725in}}%
\pgfpathlineto{\pgfqpoint{3.390684in}{3.035191in}}%
\pgfpathlineto{\pgfqpoint{3.454543in}{2.996657in}}%
\pgfpathlineto{\pgfqpoint{3.514293in}{2.958123in}}%
\pgfpathlineto{\pgfqpoint{3.570172in}{2.919590in}}%
\pgfpathlineto{\pgfqpoint{3.625796in}{2.878487in}}%
\pgfpathlineto{\pgfqpoint{3.677620in}{2.837384in}}%
\pgfpathlineto{\pgfqpoint{3.725971in}{2.796282in}}%
\pgfpathlineto{\pgfqpoint{3.773903in}{2.752610in}}%
\pgfpathlineto{\pgfqpoint{3.821216in}{2.706370in}}%
\pgfpathlineto{\pgfqpoint{3.867820in}{2.657560in}}%
\pgfpathlineto{\pgfqpoint{3.913750in}{2.606182in}}%
\pgfpathlineto{\pgfqpoint{3.961289in}{2.549666in}}%
\pgfpathlineto{\pgfqpoint{4.014511in}{2.482874in}}%
\pgfpathlineto{\pgfqpoint{4.091090in}{2.382687in}}%
\pgfpathlineto{\pgfqpoint{4.181946in}{2.264516in}}%
\pgfpathlineto{\pgfqpoint{4.234205in}{2.200294in}}%
\pgfpathlineto{\pgfqpoint{4.280987in}{2.146346in}}%
\pgfpathlineto{\pgfqpoint{4.323924in}{2.100106in}}%
\pgfpathlineto{\pgfqpoint{4.367487in}{2.056434in}}%
\pgfpathlineto{\pgfqpoint{4.411701in}{2.015332in}}%
\pgfpathlineto{\pgfqpoint{4.456476in}{1.976798in}}%
\pgfpathlineto{\pgfqpoint{4.501617in}{1.940833in}}%
\pgfpathlineto{\pgfqpoint{4.550449in}{1.904868in}}%
\pgfpathlineto{\pgfqpoint{4.599531in}{1.871473in}}%
\pgfpathlineto{\pgfqpoint{4.652658in}{1.838077in}}%
\pgfpathlineto{\pgfqpoint{4.705707in}{1.807250in}}%
\pgfpathlineto{\pgfqpoint{4.763033in}{1.776423in}}%
\pgfpathlineto{\pgfqpoint{4.825098in}{1.745596in}}%
\pgfpathlineto{\pgfqpoint{4.892412in}{1.714769in}}%
\pgfpathlineto{\pgfqpoint{4.965535in}{1.683942in}}%
\pgfpathlineto{\pgfqpoint{5.038198in}{1.655684in}}%
\pgfpathlineto{\pgfqpoint{5.116790in}{1.627426in}}%
\pgfpathlineto{\pgfqpoint{5.201886in}{1.599168in}}%
\pgfpathlineto{\pgfqpoint{5.294118in}{1.570910in}}%
\pgfpathlineto{\pgfqpoint{5.394175in}{1.542651in}}%
\pgfpathlineto{\pgfqpoint{5.410000in}{1.538389in}}%
\pgfpathlineto{\pgfqpoint{5.410000in}{1.538389in}}%
\pgfusepath{stroke}%
\end{pgfscope}%
\begin{pgfscope}%
\pgfpathrectangle{\pgfqpoint{0.750000in}{0.880000in}}{\pgfqpoint{4.650000in}{3.080000in}}%
\pgfusepath{clip}%
\pgfsetbuttcap%
\pgfsetroundjoin%
\pgfsetlinewidth{1.505625pt}%
\definecolor{currentstroke}{rgb}{0.121569,0.466667,0.705882}%
\pgfsetstrokecolor{currentstroke}%
\pgfsetdash{{5.550000pt}{2.400000pt}}{0.000000pt}%
\pgfpathmoveto{\pgfqpoint{0.750000in}{3.441079in}}%
\pgfpathlineto{\pgfqpoint{5.400000in}{3.441079in}}%
\pgfusepath{stroke}%
\end{pgfscope}%
\begin{pgfscope}%
\pgfsetrectcap%
\pgfsetmiterjoin%
\pgfsetlinewidth{0.803000pt}%
\definecolor{currentstroke}{rgb}{0.000000,0.000000,0.000000}%
\pgfsetstrokecolor{currentstroke}%
\pgfsetdash{}{0pt}%
\pgfpathmoveto{\pgfqpoint{0.750000in}{0.880000in}}%
\pgfpathlineto{\pgfqpoint{0.750000in}{3.960000in}}%
\pgfusepath{stroke}%
\end{pgfscope}%
\begin{pgfscope}%
\pgfsetrectcap%
\pgfsetmiterjoin%
\pgfsetlinewidth{0.803000pt}%
\definecolor{currentstroke}{rgb}{0.000000,0.000000,0.000000}%
\pgfsetstrokecolor{currentstroke}%
\pgfsetdash{}{0pt}%
\pgfpathmoveto{\pgfqpoint{5.400000in}{0.880000in}}%
\pgfpathlineto{\pgfqpoint{5.400000in}{3.960000in}}%
\pgfusepath{stroke}%
\end{pgfscope}%
\begin{pgfscope}%
\pgfsetrectcap%
\pgfsetmiterjoin%
\pgfsetlinewidth{0.803000pt}%
\definecolor{currentstroke}{rgb}{0.000000,0.000000,0.000000}%
\pgfsetstrokecolor{currentstroke}%
\pgfsetdash{}{0pt}%
\pgfpathmoveto{\pgfqpoint{0.750000in}{0.880000in}}%
\pgfpathlineto{\pgfqpoint{5.400000in}{0.880000in}}%
\pgfusepath{stroke}%
\end{pgfscope}%
\begin{pgfscope}%
\pgfsetrectcap%
\pgfsetmiterjoin%
\pgfsetlinewidth{0.803000pt}%
\definecolor{currentstroke}{rgb}{0.000000,0.000000,0.000000}%
\pgfsetstrokecolor{currentstroke}%
\pgfsetdash{}{0pt}%
\pgfpathmoveto{\pgfqpoint{0.750000in}{3.960000in}}%
\pgfpathlineto{\pgfqpoint{5.400000in}{3.960000in}}%
\pgfusepath{stroke}%
\end{pgfscope}%
\begin{pgfscope}%
\pgfsetbuttcap%
\pgfsetmiterjoin%
\definecolor{currentfill}{rgb}{1.000000,1.000000,1.000000}%
\pgfsetfillcolor{currentfill}%
\pgfsetfillopacity{0.800000}%
\pgfsetlinewidth{1.003750pt}%
\definecolor{currentstroke}{rgb}{0.800000,0.800000,0.800000}%
\pgfsetstrokecolor{currentstroke}%
\pgfsetstrokeopacity{0.800000}%
\pgfsetdash{}{0pt}%
\pgfpathmoveto{\pgfqpoint{0.866667in}{0.963333in}}%
\pgfpathlineto{\pgfqpoint{2.551123in}{0.963333in}}%
\pgfpathquadraticcurveto{\pgfqpoint{2.584456in}{0.963333in}}{\pgfqpoint{2.584456in}{0.996667in}}%
\pgfpathlineto{\pgfqpoint{2.584456in}{1.465921in}}%
\pgfpathquadraticcurveto{\pgfqpoint{2.584456in}{1.499254in}}{\pgfqpoint{2.551123in}{1.499254in}}%
\pgfpathlineto{\pgfqpoint{0.866667in}{1.499254in}}%
\pgfpathquadraticcurveto{\pgfqpoint{0.833333in}{1.499254in}}{\pgfqpoint{0.833333in}{1.465921in}}%
\pgfpathlineto{\pgfqpoint{0.833333in}{0.996667in}}%
\pgfpathquadraticcurveto{\pgfqpoint{0.833333in}{0.963333in}}{\pgfqpoint{0.866667in}{0.963333in}}%
\pgfpathlineto{\pgfqpoint{0.866667in}{0.963333in}}%
\pgfpathclose%
\pgfusepath{stroke,fill}%
\end{pgfscope}%
\begin{pgfscope}%
\pgfsetrectcap%
\pgfsetroundjoin%
\pgfsetlinewidth{1.505625pt}%
\definecolor{currentstroke}{rgb}{0.121569,0.466667,0.705882}%
\pgfsetstrokecolor{currentstroke}%
\pgfsetdash{}{0pt}%
\pgfpathmoveto{\pgfqpoint{0.900000in}{1.368281in}}%
\pgfpathlineto{\pgfqpoint{1.066667in}{1.368281in}}%
\pgfpathlineto{\pgfqpoint{1.233333in}{1.368281in}}%
\pgfusepath{stroke}%
\end{pgfscope}%
\begin{pgfscope}%
\definecolor{textcolor}{rgb}{0.000000,0.000000,0.000000}%
\pgfsetstrokecolor{textcolor}%
\pgfsetfillcolor{textcolor}%
\pgftext[x=1.366667in,y=1.309948in,left,base]{\color{textcolor}\rmfamily\fontsize{12.000000}{14.400000}\selectfont delta}%
\end{pgfscope}%
\begin{pgfscope}%
\pgfsetbuttcap%
\pgfsetroundjoin%
\pgfsetlinewidth{1.505625pt}%
\definecolor{currentstroke}{rgb}{0.121569,0.466667,0.705882}%
\pgfsetstrokecolor{currentstroke}%
\pgfsetdash{{5.550000pt}{2.400000pt}}{0.000000pt}%
\pgfpathmoveto{\pgfqpoint{0.900000in}{1.124629in}}%
\pgfpathlineto{\pgfqpoint{1.066667in}{1.124629in}}%
\pgfpathlineto{\pgfqpoint{1.233333in}{1.124629in}}%
\pgfusepath{stroke}%
\end{pgfscope}%
\begin{pgfscope}%
\definecolor{textcolor}{rgb}{0.000000,0.000000,0.000000}%
\pgfsetstrokecolor{textcolor}%
\pgfsetfillcolor{textcolor}%
\pgftext[x=1.366667in,y=1.066295in,left,base]{\color{textcolor}\rmfamily\fontsize{12.000000}{14.400000}\selectfont clearing of fault}%
\end{pgfscope}%
\begin{pgfscope}%
\definecolor{textcolor}{rgb}{0.000000,0.000000,0.000000}%
\pgfsetstrokecolor{textcolor}%
\pgfsetfillcolor{textcolor}%
\pgftext[x=3.000000in,y=7.840000in,,top]{\color{textcolor}\rmfamily\fontsize{14.400000}{17.280000}\selectfont Stable scenario}%
\end{pgfscope}%
\end{pgfpicture}%
\makeatother%
\endgroup%


%% Creator: Matplotlib, PGF backend
%%
%% To include the figure in your LaTeX document, write
%%   \input{<filename>.pgf}
%%
%% Make sure the required packages are loaded in your preamble
%%   \usepackage{pgf}
%%
%% Also ensure that all the required font packages are loaded; for instance,
%% the lmodern package is sometimes necessary when using math font.
%%   \usepackage{lmodern}
%%
%% Figures using additional raster images can only be included by \input if
%% they are in the same directory as the main LaTeX file. For loading figures
%% from other directories you can use the `import` package
%%   \usepackage{import}
%%
%% and then include the figures with
%%   \import{<path to file>}{<filename>.pgf}
%%
%% Matplotlib used the following preamble
%%   
%%   \usepackage{fontspec}
%%   \setmainfont{Charter.ttc}[Path=\detokenize{/System/Library/Fonts/Supplemental/}]
%%   \setsansfont{DejaVuSans.ttf}[Path=\detokenize{/opt/homebrew/lib/python3.10/site-packages/matplotlib/mpl-data/fonts/ttf/}]
%%   \setmonofont{DejaVuSansMono.ttf}[Path=\detokenize{/opt/homebrew/lib/python3.10/site-packages/matplotlib/mpl-data/fonts/ttf/}]
%%   \makeatletter\@ifpackageloaded{underscore}{}{\usepackage[strings]{underscore}}\makeatother
%%
\begingroup%
\makeatletter%
\begin{pgfpicture}%
\pgfpathrectangle{\pgfpointorigin}{\pgfqpoint{6.400000in}{4.800000in}}%
\pgfusepath{use as bounding box, clip}%
\begin{pgfscope}%
\pgfsetbuttcap%
\pgfsetmiterjoin%
\definecolor{currentfill}{rgb}{1.000000,1.000000,1.000000}%
\pgfsetfillcolor{currentfill}%
\pgfsetlinewidth{0.000000pt}%
\definecolor{currentstroke}{rgb}{1.000000,1.000000,1.000000}%
\pgfsetstrokecolor{currentstroke}%
\pgfsetdash{}{0pt}%
\pgfpathmoveto{\pgfqpoint{0.000000in}{0.000000in}}%
\pgfpathlineto{\pgfqpoint{6.400000in}{0.000000in}}%
\pgfpathlineto{\pgfqpoint{6.400000in}{4.800000in}}%
\pgfpathlineto{\pgfqpoint{0.000000in}{4.800000in}}%
\pgfpathlineto{\pgfqpoint{0.000000in}{0.000000in}}%
\pgfpathclose%
\pgfusepath{fill}%
\end{pgfscope}%
\begin{pgfscope}%
\pgfsetbuttcap%
\pgfsetmiterjoin%
\definecolor{currentfill}{rgb}{1.000000,1.000000,1.000000}%
\pgfsetfillcolor{currentfill}%
\pgfsetlinewidth{0.000000pt}%
\definecolor{currentstroke}{rgb}{0.000000,0.000000,0.000000}%
\pgfsetstrokecolor{currentstroke}%
\pgfsetstrokeopacity{0.000000}%
\pgfsetdash{}{0pt}%
\pgfpathmoveto{\pgfqpoint{0.800000in}{0.528000in}}%
\pgfpathlineto{\pgfqpoint{5.760000in}{0.528000in}}%
\pgfpathlineto{\pgfqpoint{5.760000in}{4.224000in}}%
\pgfpathlineto{\pgfqpoint{0.800000in}{4.224000in}}%
\pgfpathlineto{\pgfqpoint{0.800000in}{0.528000in}}%
\pgfpathclose%
\pgfusepath{fill}%
\end{pgfscope}%
\begin{pgfscope}%
\pgfpathrectangle{\pgfqpoint{0.800000in}{0.528000in}}{\pgfqpoint{4.960000in}{3.696000in}}%
\pgfusepath{clip}%
\pgfsetrectcap%
\pgfsetroundjoin%
\pgfsetlinewidth{0.803000pt}%
\definecolor{currentstroke}{rgb}{0.690196,0.690196,0.690196}%
\pgfsetstrokecolor{currentstroke}%
\pgfsetdash{}{0pt}%
\pgfpathmoveto{\pgfqpoint{1.025455in}{0.528000in}}%
\pgfpathlineto{\pgfqpoint{1.025455in}{4.224000in}}%
\pgfusepath{stroke}%
\end{pgfscope}%
\begin{pgfscope}%
\pgfsetbuttcap%
\pgfsetroundjoin%
\definecolor{currentfill}{rgb}{0.000000,0.000000,0.000000}%
\pgfsetfillcolor{currentfill}%
\pgfsetlinewidth{0.803000pt}%
\definecolor{currentstroke}{rgb}{0.000000,0.000000,0.000000}%
\pgfsetstrokecolor{currentstroke}%
\pgfsetdash{}{0pt}%
\pgfsys@defobject{currentmarker}{\pgfqpoint{0.000000in}{-0.048611in}}{\pgfqpoint{0.000000in}{0.000000in}}{%
\pgfpathmoveto{\pgfqpoint{0.000000in}{0.000000in}}%
\pgfpathlineto{\pgfqpoint{0.000000in}{-0.048611in}}%
\pgfusepath{stroke,fill}%
}%
\begin{pgfscope}%
\pgfsys@transformshift{1.025455in}{0.528000in}%
\pgfsys@useobject{currentmarker}{}%
\end{pgfscope}%
\end{pgfscope}%
\begin{pgfscope}%
\definecolor{textcolor}{rgb}{0.000000,0.000000,0.000000}%
\pgfsetstrokecolor{textcolor}%
\pgfsetfillcolor{textcolor}%
\pgftext[x=1.025455in,y=0.430778in,,top]{\color{textcolor}\rmfamily\fontsize{10.000000}{12.000000}\selectfont \(\displaystyle {\ensuremath{-}1.0}\)}%
\end{pgfscope}%
\begin{pgfscope}%
\pgfpathrectangle{\pgfqpoint{0.800000in}{0.528000in}}{\pgfqpoint{4.960000in}{3.696000in}}%
\pgfusepath{clip}%
\pgfsetrectcap%
\pgfsetroundjoin%
\pgfsetlinewidth{0.803000pt}%
\definecolor{currentstroke}{rgb}{0.690196,0.690196,0.690196}%
\pgfsetstrokecolor{currentstroke}%
\pgfsetdash{}{0pt}%
\pgfpathmoveto{\pgfqpoint{1.777220in}{0.528000in}}%
\pgfpathlineto{\pgfqpoint{1.777220in}{4.224000in}}%
\pgfusepath{stroke}%
\end{pgfscope}%
\begin{pgfscope}%
\pgfsetbuttcap%
\pgfsetroundjoin%
\definecolor{currentfill}{rgb}{0.000000,0.000000,0.000000}%
\pgfsetfillcolor{currentfill}%
\pgfsetlinewidth{0.803000pt}%
\definecolor{currentstroke}{rgb}{0.000000,0.000000,0.000000}%
\pgfsetstrokecolor{currentstroke}%
\pgfsetdash{}{0pt}%
\pgfsys@defobject{currentmarker}{\pgfqpoint{0.000000in}{-0.048611in}}{\pgfqpoint{0.000000in}{0.000000in}}{%
\pgfpathmoveto{\pgfqpoint{0.000000in}{0.000000in}}%
\pgfpathlineto{\pgfqpoint{0.000000in}{-0.048611in}}%
\pgfusepath{stroke,fill}%
}%
\begin{pgfscope}%
\pgfsys@transformshift{1.777220in}{0.528000in}%
\pgfsys@useobject{currentmarker}{}%
\end{pgfscope}%
\end{pgfscope}%
\begin{pgfscope}%
\definecolor{textcolor}{rgb}{0.000000,0.000000,0.000000}%
\pgfsetstrokecolor{textcolor}%
\pgfsetfillcolor{textcolor}%
\pgftext[x=1.777220in,y=0.430778in,,top]{\color{textcolor}\rmfamily\fontsize{10.000000}{12.000000}\selectfont \(\displaystyle {\ensuremath{-}0.5}\)}%
\end{pgfscope}%
\begin{pgfscope}%
\pgfpathrectangle{\pgfqpoint{0.800000in}{0.528000in}}{\pgfqpoint{4.960000in}{3.696000in}}%
\pgfusepath{clip}%
\pgfsetrectcap%
\pgfsetroundjoin%
\pgfsetlinewidth{0.803000pt}%
\definecolor{currentstroke}{rgb}{0.690196,0.690196,0.690196}%
\pgfsetstrokecolor{currentstroke}%
\pgfsetdash{}{0pt}%
\pgfpathmoveto{\pgfqpoint{2.528986in}{0.528000in}}%
\pgfpathlineto{\pgfqpoint{2.528986in}{4.224000in}}%
\pgfusepath{stroke}%
\end{pgfscope}%
\begin{pgfscope}%
\pgfsetbuttcap%
\pgfsetroundjoin%
\definecolor{currentfill}{rgb}{0.000000,0.000000,0.000000}%
\pgfsetfillcolor{currentfill}%
\pgfsetlinewidth{0.803000pt}%
\definecolor{currentstroke}{rgb}{0.000000,0.000000,0.000000}%
\pgfsetstrokecolor{currentstroke}%
\pgfsetdash{}{0pt}%
\pgfsys@defobject{currentmarker}{\pgfqpoint{0.000000in}{-0.048611in}}{\pgfqpoint{0.000000in}{0.000000in}}{%
\pgfpathmoveto{\pgfqpoint{0.000000in}{0.000000in}}%
\pgfpathlineto{\pgfqpoint{0.000000in}{-0.048611in}}%
\pgfusepath{stroke,fill}%
}%
\begin{pgfscope}%
\pgfsys@transformshift{2.528986in}{0.528000in}%
\pgfsys@useobject{currentmarker}{}%
\end{pgfscope}%
\end{pgfscope}%
\begin{pgfscope}%
\definecolor{textcolor}{rgb}{0.000000,0.000000,0.000000}%
\pgfsetstrokecolor{textcolor}%
\pgfsetfillcolor{textcolor}%
\pgftext[x=2.528986in,y=0.430778in,,top]{\color{textcolor}\rmfamily\fontsize{10.000000}{12.000000}\selectfont \(\displaystyle {0.0}\)}%
\end{pgfscope}%
\begin{pgfscope}%
\pgfpathrectangle{\pgfqpoint{0.800000in}{0.528000in}}{\pgfqpoint{4.960000in}{3.696000in}}%
\pgfusepath{clip}%
\pgfsetrectcap%
\pgfsetroundjoin%
\pgfsetlinewidth{0.803000pt}%
\definecolor{currentstroke}{rgb}{0.690196,0.690196,0.690196}%
\pgfsetstrokecolor{currentstroke}%
\pgfsetdash{}{0pt}%
\pgfpathmoveto{\pgfqpoint{3.280752in}{0.528000in}}%
\pgfpathlineto{\pgfqpoint{3.280752in}{4.224000in}}%
\pgfusepath{stroke}%
\end{pgfscope}%
\begin{pgfscope}%
\pgfsetbuttcap%
\pgfsetroundjoin%
\definecolor{currentfill}{rgb}{0.000000,0.000000,0.000000}%
\pgfsetfillcolor{currentfill}%
\pgfsetlinewidth{0.803000pt}%
\definecolor{currentstroke}{rgb}{0.000000,0.000000,0.000000}%
\pgfsetstrokecolor{currentstroke}%
\pgfsetdash{}{0pt}%
\pgfsys@defobject{currentmarker}{\pgfqpoint{0.000000in}{-0.048611in}}{\pgfqpoint{0.000000in}{0.000000in}}{%
\pgfpathmoveto{\pgfqpoint{0.000000in}{0.000000in}}%
\pgfpathlineto{\pgfqpoint{0.000000in}{-0.048611in}}%
\pgfusepath{stroke,fill}%
}%
\begin{pgfscope}%
\pgfsys@transformshift{3.280752in}{0.528000in}%
\pgfsys@useobject{currentmarker}{}%
\end{pgfscope}%
\end{pgfscope}%
\begin{pgfscope}%
\definecolor{textcolor}{rgb}{0.000000,0.000000,0.000000}%
\pgfsetstrokecolor{textcolor}%
\pgfsetfillcolor{textcolor}%
\pgftext[x=3.280752in,y=0.430778in,,top]{\color{textcolor}\rmfamily\fontsize{10.000000}{12.000000}\selectfont \(\displaystyle {0.5}\)}%
\end{pgfscope}%
\begin{pgfscope}%
\pgfpathrectangle{\pgfqpoint{0.800000in}{0.528000in}}{\pgfqpoint{4.960000in}{3.696000in}}%
\pgfusepath{clip}%
\pgfsetrectcap%
\pgfsetroundjoin%
\pgfsetlinewidth{0.803000pt}%
\definecolor{currentstroke}{rgb}{0.690196,0.690196,0.690196}%
\pgfsetstrokecolor{currentstroke}%
\pgfsetdash{}{0pt}%
\pgfpathmoveto{\pgfqpoint{4.032518in}{0.528000in}}%
\pgfpathlineto{\pgfqpoint{4.032518in}{4.224000in}}%
\pgfusepath{stroke}%
\end{pgfscope}%
\begin{pgfscope}%
\pgfsetbuttcap%
\pgfsetroundjoin%
\definecolor{currentfill}{rgb}{0.000000,0.000000,0.000000}%
\pgfsetfillcolor{currentfill}%
\pgfsetlinewidth{0.803000pt}%
\definecolor{currentstroke}{rgb}{0.000000,0.000000,0.000000}%
\pgfsetstrokecolor{currentstroke}%
\pgfsetdash{}{0pt}%
\pgfsys@defobject{currentmarker}{\pgfqpoint{0.000000in}{-0.048611in}}{\pgfqpoint{0.000000in}{0.000000in}}{%
\pgfpathmoveto{\pgfqpoint{0.000000in}{0.000000in}}%
\pgfpathlineto{\pgfqpoint{0.000000in}{-0.048611in}}%
\pgfusepath{stroke,fill}%
}%
\begin{pgfscope}%
\pgfsys@transformshift{4.032518in}{0.528000in}%
\pgfsys@useobject{currentmarker}{}%
\end{pgfscope}%
\end{pgfscope}%
\begin{pgfscope}%
\definecolor{textcolor}{rgb}{0.000000,0.000000,0.000000}%
\pgfsetstrokecolor{textcolor}%
\pgfsetfillcolor{textcolor}%
\pgftext[x=4.032518in,y=0.430778in,,top]{\color{textcolor}\rmfamily\fontsize{10.000000}{12.000000}\selectfont \(\displaystyle {1.0}\)}%
\end{pgfscope}%
\begin{pgfscope}%
\pgfpathrectangle{\pgfqpoint{0.800000in}{0.528000in}}{\pgfqpoint{4.960000in}{3.696000in}}%
\pgfusepath{clip}%
\pgfsetrectcap%
\pgfsetroundjoin%
\pgfsetlinewidth{0.803000pt}%
\definecolor{currentstroke}{rgb}{0.690196,0.690196,0.690196}%
\pgfsetstrokecolor{currentstroke}%
\pgfsetdash{}{0pt}%
\pgfpathmoveto{\pgfqpoint{4.784283in}{0.528000in}}%
\pgfpathlineto{\pgfqpoint{4.784283in}{4.224000in}}%
\pgfusepath{stroke}%
\end{pgfscope}%
\begin{pgfscope}%
\pgfsetbuttcap%
\pgfsetroundjoin%
\definecolor{currentfill}{rgb}{0.000000,0.000000,0.000000}%
\pgfsetfillcolor{currentfill}%
\pgfsetlinewidth{0.803000pt}%
\definecolor{currentstroke}{rgb}{0.000000,0.000000,0.000000}%
\pgfsetstrokecolor{currentstroke}%
\pgfsetdash{}{0pt}%
\pgfsys@defobject{currentmarker}{\pgfqpoint{0.000000in}{-0.048611in}}{\pgfqpoint{0.000000in}{0.000000in}}{%
\pgfpathmoveto{\pgfqpoint{0.000000in}{0.000000in}}%
\pgfpathlineto{\pgfqpoint{0.000000in}{-0.048611in}}%
\pgfusepath{stroke,fill}%
}%
\begin{pgfscope}%
\pgfsys@transformshift{4.784283in}{0.528000in}%
\pgfsys@useobject{currentmarker}{}%
\end{pgfscope}%
\end{pgfscope}%
\begin{pgfscope}%
\definecolor{textcolor}{rgb}{0.000000,0.000000,0.000000}%
\pgfsetstrokecolor{textcolor}%
\pgfsetfillcolor{textcolor}%
\pgftext[x=4.784283in,y=0.430778in,,top]{\color{textcolor}\rmfamily\fontsize{10.000000}{12.000000}\selectfont \(\displaystyle {1.5}\)}%
\end{pgfscope}%
\begin{pgfscope}%
\pgfpathrectangle{\pgfqpoint{0.800000in}{0.528000in}}{\pgfqpoint{4.960000in}{3.696000in}}%
\pgfusepath{clip}%
\pgfsetrectcap%
\pgfsetroundjoin%
\pgfsetlinewidth{0.803000pt}%
\definecolor{currentstroke}{rgb}{0.690196,0.690196,0.690196}%
\pgfsetstrokecolor{currentstroke}%
\pgfsetdash{}{0pt}%
\pgfpathmoveto{\pgfqpoint{5.536049in}{0.528000in}}%
\pgfpathlineto{\pgfqpoint{5.536049in}{4.224000in}}%
\pgfusepath{stroke}%
\end{pgfscope}%
\begin{pgfscope}%
\pgfsetbuttcap%
\pgfsetroundjoin%
\definecolor{currentfill}{rgb}{0.000000,0.000000,0.000000}%
\pgfsetfillcolor{currentfill}%
\pgfsetlinewidth{0.803000pt}%
\definecolor{currentstroke}{rgb}{0.000000,0.000000,0.000000}%
\pgfsetstrokecolor{currentstroke}%
\pgfsetdash{}{0pt}%
\pgfsys@defobject{currentmarker}{\pgfqpoint{0.000000in}{-0.048611in}}{\pgfqpoint{0.000000in}{0.000000in}}{%
\pgfpathmoveto{\pgfqpoint{0.000000in}{0.000000in}}%
\pgfpathlineto{\pgfqpoint{0.000000in}{-0.048611in}}%
\pgfusepath{stroke,fill}%
}%
\begin{pgfscope}%
\pgfsys@transformshift{5.536049in}{0.528000in}%
\pgfsys@useobject{currentmarker}{}%
\end{pgfscope}%
\end{pgfscope}%
\begin{pgfscope}%
\definecolor{textcolor}{rgb}{0.000000,0.000000,0.000000}%
\pgfsetstrokecolor{textcolor}%
\pgfsetfillcolor{textcolor}%
\pgftext[x=5.536049in,y=0.430778in,,top]{\color{textcolor}\rmfamily\fontsize{10.000000}{12.000000}\selectfont \(\displaystyle {2.0}\)}%
\end{pgfscope}%
\begin{pgfscope}%
\definecolor{textcolor}{rgb}{0.000000,0.000000,0.000000}%
\pgfsetstrokecolor{textcolor}%
\pgfsetfillcolor{textcolor}%
\pgftext[x=3.280000in,y=0.242776in,,top]{\color{textcolor}\rmfamily\fontsize{10.000000}{12.000000}\selectfont time in s}%
\end{pgfscope}%
\begin{pgfscope}%
\pgfpathrectangle{\pgfqpoint{0.800000in}{0.528000in}}{\pgfqpoint{4.960000in}{3.696000in}}%
\pgfusepath{clip}%
\pgfsetrectcap%
\pgfsetroundjoin%
\pgfsetlinewidth{0.803000pt}%
\definecolor{currentstroke}{rgb}{0.690196,0.690196,0.690196}%
\pgfsetstrokecolor{currentstroke}%
\pgfsetdash{}{0pt}%
\pgfpathmoveto{\pgfqpoint{0.800000in}{0.976000in}}%
\pgfpathlineto{\pgfqpoint{5.760000in}{0.976000in}}%
\pgfusepath{stroke}%
\end{pgfscope}%
\begin{pgfscope}%
\pgfsetbuttcap%
\pgfsetroundjoin%
\definecolor{currentfill}{rgb}{0.000000,0.000000,0.000000}%
\pgfsetfillcolor{currentfill}%
\pgfsetlinewidth{0.803000pt}%
\definecolor{currentstroke}{rgb}{0.000000,0.000000,0.000000}%
\pgfsetstrokecolor{currentstroke}%
\pgfsetdash{}{0pt}%
\pgfsys@defobject{currentmarker}{\pgfqpoint{-0.048611in}{0.000000in}}{\pgfqpoint{-0.000000in}{0.000000in}}{%
\pgfpathmoveto{\pgfqpoint{-0.000000in}{0.000000in}}%
\pgfpathlineto{\pgfqpoint{-0.048611in}{0.000000in}}%
\pgfusepath{stroke,fill}%
}%
\begin{pgfscope}%
\pgfsys@transformshift{0.800000in}{0.976000in}%
\pgfsys@useobject{currentmarker}{}%
\end{pgfscope}%
\end{pgfscope}%
\begin{pgfscope}%
\definecolor{textcolor}{rgb}{0.000000,0.000000,0.000000}%
\pgfsetstrokecolor{textcolor}%
\pgfsetfillcolor{textcolor}%
\pgftext[x=0.417283in, y=0.924900in, left, base]{\color{textcolor}\rmfamily\fontsize{10.000000}{12.000000}\selectfont \(\displaystyle {\ensuremath{-}1.0}\)}%
\end{pgfscope}%
\begin{pgfscope}%
\pgfpathrectangle{\pgfqpoint{0.800000in}{0.528000in}}{\pgfqpoint{4.960000in}{3.696000in}}%
\pgfusepath{clip}%
\pgfsetrectcap%
\pgfsetroundjoin%
\pgfsetlinewidth{0.803000pt}%
\definecolor{currentstroke}{rgb}{0.690196,0.690196,0.690196}%
\pgfsetstrokecolor{currentstroke}%
\pgfsetdash{}{0pt}%
\pgfpathmoveto{\pgfqpoint{0.800000in}{1.676000in}}%
\pgfpathlineto{\pgfqpoint{5.760000in}{1.676000in}}%
\pgfusepath{stroke}%
\end{pgfscope}%
\begin{pgfscope}%
\pgfsetbuttcap%
\pgfsetroundjoin%
\definecolor{currentfill}{rgb}{0.000000,0.000000,0.000000}%
\pgfsetfillcolor{currentfill}%
\pgfsetlinewidth{0.803000pt}%
\definecolor{currentstroke}{rgb}{0.000000,0.000000,0.000000}%
\pgfsetstrokecolor{currentstroke}%
\pgfsetdash{}{0pt}%
\pgfsys@defobject{currentmarker}{\pgfqpoint{-0.048611in}{0.000000in}}{\pgfqpoint{-0.000000in}{0.000000in}}{%
\pgfpathmoveto{\pgfqpoint{-0.000000in}{0.000000in}}%
\pgfpathlineto{\pgfqpoint{-0.048611in}{0.000000in}}%
\pgfusepath{stroke,fill}%
}%
\begin{pgfscope}%
\pgfsys@transformshift{0.800000in}{1.676000in}%
\pgfsys@useobject{currentmarker}{}%
\end{pgfscope}%
\end{pgfscope}%
\begin{pgfscope}%
\definecolor{textcolor}{rgb}{0.000000,0.000000,0.000000}%
\pgfsetstrokecolor{textcolor}%
\pgfsetfillcolor{textcolor}%
\pgftext[x=0.417283in, y=1.624900in, left, base]{\color{textcolor}\rmfamily\fontsize{10.000000}{12.000000}\selectfont \(\displaystyle {\ensuremath{-}0.5}\)}%
\end{pgfscope}%
\begin{pgfscope}%
\pgfpathrectangle{\pgfqpoint{0.800000in}{0.528000in}}{\pgfqpoint{4.960000in}{3.696000in}}%
\pgfusepath{clip}%
\pgfsetrectcap%
\pgfsetroundjoin%
\pgfsetlinewidth{0.803000pt}%
\definecolor{currentstroke}{rgb}{0.690196,0.690196,0.690196}%
\pgfsetstrokecolor{currentstroke}%
\pgfsetdash{}{0pt}%
\pgfpathmoveto{\pgfqpoint{0.800000in}{2.376000in}}%
\pgfpathlineto{\pgfqpoint{5.760000in}{2.376000in}}%
\pgfusepath{stroke}%
\end{pgfscope}%
\begin{pgfscope}%
\pgfsetbuttcap%
\pgfsetroundjoin%
\definecolor{currentfill}{rgb}{0.000000,0.000000,0.000000}%
\pgfsetfillcolor{currentfill}%
\pgfsetlinewidth{0.803000pt}%
\definecolor{currentstroke}{rgb}{0.000000,0.000000,0.000000}%
\pgfsetstrokecolor{currentstroke}%
\pgfsetdash{}{0pt}%
\pgfsys@defobject{currentmarker}{\pgfqpoint{-0.048611in}{0.000000in}}{\pgfqpoint{-0.000000in}{0.000000in}}{%
\pgfpathmoveto{\pgfqpoint{-0.000000in}{0.000000in}}%
\pgfpathlineto{\pgfqpoint{-0.048611in}{0.000000in}}%
\pgfusepath{stroke,fill}%
}%
\begin{pgfscope}%
\pgfsys@transformshift{0.800000in}{2.376000in}%
\pgfsys@useobject{currentmarker}{}%
\end{pgfscope}%
\end{pgfscope}%
\begin{pgfscope}%
\definecolor{textcolor}{rgb}{0.000000,0.000000,0.000000}%
\pgfsetstrokecolor{textcolor}%
\pgfsetfillcolor{textcolor}%
\pgftext[x=0.525308in, y=2.324900in, left, base]{\color{textcolor}\rmfamily\fontsize{10.000000}{12.000000}\selectfont \(\displaystyle {0.0}\)}%
\end{pgfscope}%
\begin{pgfscope}%
\pgfpathrectangle{\pgfqpoint{0.800000in}{0.528000in}}{\pgfqpoint{4.960000in}{3.696000in}}%
\pgfusepath{clip}%
\pgfsetrectcap%
\pgfsetroundjoin%
\pgfsetlinewidth{0.803000pt}%
\definecolor{currentstroke}{rgb}{0.690196,0.690196,0.690196}%
\pgfsetstrokecolor{currentstroke}%
\pgfsetdash{}{0pt}%
\pgfpathmoveto{\pgfqpoint{0.800000in}{3.076000in}}%
\pgfpathlineto{\pgfqpoint{5.760000in}{3.076000in}}%
\pgfusepath{stroke}%
\end{pgfscope}%
\begin{pgfscope}%
\pgfsetbuttcap%
\pgfsetroundjoin%
\definecolor{currentfill}{rgb}{0.000000,0.000000,0.000000}%
\pgfsetfillcolor{currentfill}%
\pgfsetlinewidth{0.803000pt}%
\definecolor{currentstroke}{rgb}{0.000000,0.000000,0.000000}%
\pgfsetstrokecolor{currentstroke}%
\pgfsetdash{}{0pt}%
\pgfsys@defobject{currentmarker}{\pgfqpoint{-0.048611in}{0.000000in}}{\pgfqpoint{-0.000000in}{0.000000in}}{%
\pgfpathmoveto{\pgfqpoint{-0.000000in}{0.000000in}}%
\pgfpathlineto{\pgfqpoint{-0.048611in}{0.000000in}}%
\pgfusepath{stroke,fill}%
}%
\begin{pgfscope}%
\pgfsys@transformshift{0.800000in}{3.076000in}%
\pgfsys@useobject{currentmarker}{}%
\end{pgfscope}%
\end{pgfscope}%
\begin{pgfscope}%
\definecolor{textcolor}{rgb}{0.000000,0.000000,0.000000}%
\pgfsetstrokecolor{textcolor}%
\pgfsetfillcolor{textcolor}%
\pgftext[x=0.525308in, y=3.024900in, left, base]{\color{textcolor}\rmfamily\fontsize{10.000000}{12.000000}\selectfont \(\displaystyle {0.5}\)}%
\end{pgfscope}%
\begin{pgfscope}%
\pgfpathrectangle{\pgfqpoint{0.800000in}{0.528000in}}{\pgfqpoint{4.960000in}{3.696000in}}%
\pgfusepath{clip}%
\pgfsetrectcap%
\pgfsetroundjoin%
\pgfsetlinewidth{0.803000pt}%
\definecolor{currentstroke}{rgb}{0.690196,0.690196,0.690196}%
\pgfsetstrokecolor{currentstroke}%
\pgfsetdash{}{0pt}%
\pgfpathmoveto{\pgfqpoint{0.800000in}{3.776000in}}%
\pgfpathlineto{\pgfqpoint{5.760000in}{3.776000in}}%
\pgfusepath{stroke}%
\end{pgfscope}%
\begin{pgfscope}%
\pgfsetbuttcap%
\pgfsetroundjoin%
\definecolor{currentfill}{rgb}{0.000000,0.000000,0.000000}%
\pgfsetfillcolor{currentfill}%
\pgfsetlinewidth{0.803000pt}%
\definecolor{currentstroke}{rgb}{0.000000,0.000000,0.000000}%
\pgfsetstrokecolor{currentstroke}%
\pgfsetdash{}{0pt}%
\pgfsys@defobject{currentmarker}{\pgfqpoint{-0.048611in}{0.000000in}}{\pgfqpoint{-0.000000in}{0.000000in}}{%
\pgfpathmoveto{\pgfqpoint{-0.000000in}{0.000000in}}%
\pgfpathlineto{\pgfqpoint{-0.048611in}{0.000000in}}%
\pgfusepath{stroke,fill}%
}%
\begin{pgfscope}%
\pgfsys@transformshift{0.800000in}{3.776000in}%
\pgfsys@useobject{currentmarker}{}%
\end{pgfscope}%
\end{pgfscope}%
\begin{pgfscope}%
\definecolor{textcolor}{rgb}{0.000000,0.000000,0.000000}%
\pgfsetstrokecolor{textcolor}%
\pgfsetfillcolor{textcolor}%
\pgftext[x=0.525308in, y=3.724900in, left, base]{\color{textcolor}\rmfamily\fontsize{10.000000}{12.000000}\selectfont \(\displaystyle {1.0}\)}%
\end{pgfscope}%
\begin{pgfscope}%
\definecolor{textcolor}{rgb}{0.000000,0.000000,0.000000}%
\pgfsetstrokecolor{textcolor}%
\pgfsetfillcolor{textcolor}%
\pgftext[x=0.361727in,y=2.376000in,,bottom,rotate=90.000000]{\color{textcolor}\rmfamily\fontsize{10.000000}{12.000000}\selectfont electrical power in \(\displaystyle \mathrm{p.u.}\)}%
\end{pgfscope}%
\begin{pgfscope}%
\pgfpathrectangle{\pgfqpoint{0.800000in}{0.528000in}}{\pgfqpoint{4.960000in}{3.696000in}}%
\pgfusepath{clip}%
\pgfsetrectcap%
\pgfsetroundjoin%
\pgfsetlinewidth{1.505625pt}%
\definecolor{currentstroke}{rgb}{0.121569,0.466667,0.705882}%
\pgfsetstrokecolor{currentstroke}%
\pgfsetdash{}{0pt}%
\pgfpathmoveto{\pgfqpoint{1.025455in}{3.636187in}}%
\pgfpathlineto{\pgfqpoint{2.527482in}{3.636173in}}%
\pgfpathlineto{\pgfqpoint{2.528986in}{2.376053in}}%
\pgfpathlineto{\pgfqpoint{2.706403in}{2.376050in}}%
\pgfpathlineto{\pgfqpoint{2.707906in}{3.908950in}}%
\pgfpathlineto{\pgfqpoint{2.724445in}{3.942698in}}%
\pgfpathlineto{\pgfqpoint{2.740984in}{3.971484in}}%
\pgfpathlineto{\pgfqpoint{2.756019in}{3.993511in}}%
\pgfpathlineto{\pgfqpoint{2.771055in}{4.011785in}}%
\pgfpathlineto{\pgfqpoint{2.786090in}{4.026523in}}%
\pgfpathlineto{\pgfqpoint{2.801125in}{4.037967in}}%
\pgfpathlineto{\pgfqpoint{2.814657in}{4.045659in}}%
\pgfpathlineto{\pgfqpoint{2.829692in}{4.051552in}}%
\pgfpathlineto{\pgfqpoint{2.844728in}{4.054911in}}%
\pgfpathlineto{\pgfqpoint{2.859763in}{4.056000in}}%
\pgfpathlineto{\pgfqpoint{2.876302in}{4.054888in}}%
\pgfpathlineto{\pgfqpoint{2.894344in}{4.051294in}}%
\pgfpathlineto{\pgfqpoint{2.913890in}{4.045061in}}%
\pgfpathlineto{\pgfqpoint{2.936443in}{4.035457in}}%
\pgfpathlineto{\pgfqpoint{2.963507in}{4.021420in}}%
\pgfpathlineto{\pgfqpoint{3.001095in}{3.999134in}}%
\pgfpathlineto{\pgfqpoint{3.112356in}{3.931509in}}%
\pgfpathlineto{\pgfqpoint{3.149945in}{3.912161in}}%
\pgfpathlineto{\pgfqpoint{3.183022in}{3.897575in}}%
\pgfpathlineto{\pgfqpoint{3.214596in}{3.886042in}}%
\pgfpathlineto{\pgfqpoint{3.244667in}{3.877370in}}%
\pgfpathlineto{\pgfqpoint{3.274738in}{3.871042in}}%
\pgfpathlineto{\pgfqpoint{3.303305in}{3.867247in}}%
\pgfpathlineto{\pgfqpoint{3.331872in}{3.865636in}}%
\pgfpathlineto{\pgfqpoint{3.360439in}{3.866214in}}%
\pgfpathlineto{\pgfqpoint{3.389006in}{3.868971in}}%
\pgfpathlineto{\pgfqpoint{3.417573in}{3.873882in}}%
\pgfpathlineto{\pgfqpoint{3.447644in}{3.881332in}}%
\pgfpathlineto{\pgfqpoint{3.477714in}{3.891043in}}%
\pgfpathlineto{\pgfqpoint{3.509289in}{3.903551in}}%
\pgfpathlineto{\pgfqpoint{3.543870in}{3.919740in}}%
\pgfpathlineto{\pgfqpoint{3.581458in}{3.939877in}}%
\pgfpathlineto{\pgfqpoint{3.628068in}{3.967511in}}%
\pgfpathlineto{\pgfqpoint{3.719783in}{4.022580in}}%
\pgfpathlineto{\pgfqpoint{3.748350in}{4.036870in}}%
\pgfpathlineto{\pgfqpoint{3.772407in}{4.046517in}}%
\pgfpathlineto{\pgfqpoint{3.793456in}{4.052529in}}%
\pgfpathlineto{\pgfqpoint{3.811498in}{4.055426in}}%
\pgfpathlineto{\pgfqpoint{3.828037in}{4.055907in}}%
\pgfpathlineto{\pgfqpoint{3.843073in}{4.054271in}}%
\pgfpathlineto{\pgfqpoint{3.858108in}{4.050414in}}%
\pgfpathlineto{\pgfqpoint{3.873143in}{4.044094in}}%
\pgfpathlineto{\pgfqpoint{3.886675in}{4.036102in}}%
\pgfpathlineto{\pgfqpoint{3.900207in}{4.025749in}}%
\pgfpathlineto{\pgfqpoint{3.915242in}{4.011277in}}%
\pgfpathlineto{\pgfqpoint{3.930277in}{3.993479in}}%
\pgfpathlineto{\pgfqpoint{3.945313in}{3.972167in}}%
\pgfpathlineto{\pgfqpoint{3.960348in}{3.947183in}}%
\pgfpathlineto{\pgfqpoint{3.976887in}{3.915316in}}%
\pgfpathlineto{\pgfqpoint{3.993426in}{3.878759in}}%
\pgfpathlineto{\pgfqpoint{4.011468in}{3.833497in}}%
\pgfpathlineto{\pgfqpoint{4.031014in}{3.778218in}}%
\pgfpathlineto{\pgfqpoint{4.052063in}{3.711731in}}%
\pgfpathlineto{\pgfqpoint{4.074616in}{3.633137in}}%
\pgfpathlineto{\pgfqpoint{4.101680in}{3.530226in}}%
\pgfpathlineto{\pgfqpoint{4.134758in}{3.395061in}}%
\pgfpathlineto{\pgfqpoint{4.233991in}{2.982969in}}%
\pgfpathlineto{\pgfqpoint{4.259551in}{2.889530in}}%
\pgfpathlineto{\pgfqpoint{4.280600in}{2.820243in}}%
\pgfpathlineto{\pgfqpoint{4.300146in}{2.763219in}}%
\pgfpathlineto{\pgfqpoint{4.318188in}{2.717553in}}%
\pgfpathlineto{\pgfqpoint{4.333224in}{2.685004in}}%
\pgfpathlineto{\pgfqpoint{4.348259in}{2.657730in}}%
\pgfpathlineto{\pgfqpoint{4.361791in}{2.637861in}}%
\pgfpathlineto{\pgfqpoint{4.373819in}{2.624018in}}%
\pgfpathlineto{\pgfqpoint{4.384344in}{2.614904in}}%
\pgfpathlineto{\pgfqpoint{4.394869in}{2.608620in}}%
\pgfpathlineto{\pgfqpoint{4.403890in}{2.605502in}}%
\pgfpathlineto{\pgfqpoint{4.412911in}{2.604483in}}%
\pgfpathlineto{\pgfqpoint{4.421932in}{2.605566in}}%
\pgfpathlineto{\pgfqpoint{4.430953in}{2.608745in}}%
\pgfpathlineto{\pgfqpoint{4.441478in}{2.615090in}}%
\pgfpathlineto{\pgfqpoint{4.452003in}{2.624249in}}%
\pgfpathlineto{\pgfqpoint{4.464031in}{2.638119in}}%
\pgfpathlineto{\pgfqpoint{4.476059in}{2.655550in}}%
\pgfpathlineto{\pgfqpoint{4.489591in}{2.679308in}}%
\pgfpathlineto{\pgfqpoint{4.504626in}{2.710676in}}%
\pgfpathlineto{\pgfqpoint{4.521165in}{2.750931in}}%
\pgfpathlineto{\pgfqpoint{4.539208in}{2.801254in}}%
\pgfpathlineto{\pgfqpoint{4.558754in}{2.862618in}}%
\pgfpathlineto{\pgfqpoint{4.581306in}{2.941108in}}%
\pgfpathlineto{\pgfqpoint{4.608370in}{3.043994in}}%
\pgfpathlineto{\pgfqpoint{4.644455in}{3.190835in}}%
\pgfpathlineto{\pgfqpoint{4.722638in}{3.511758in}}%
\pgfpathlineto{\pgfqpoint{4.751206in}{3.618648in}}%
\pgfpathlineto{\pgfqpoint{4.776766in}{3.705777in}}%
\pgfpathlineto{\pgfqpoint{4.799319in}{3.774838in}}%
\pgfpathlineto{\pgfqpoint{4.820368in}{3.832126in}}%
\pgfpathlineto{\pgfqpoint{4.839914in}{3.878909in}}%
\pgfpathlineto{\pgfqpoint{4.857956in}{3.916589in}}%
\pgfpathlineto{\pgfqpoint{4.875999in}{3.949082in}}%
\pgfpathlineto{\pgfqpoint{4.892538in}{3.974465in}}%
\pgfpathlineto{\pgfqpoint{4.909076in}{3.995839in}}%
\pgfpathlineto{\pgfqpoint{4.925615in}{4.013442in}}%
\pgfpathlineto{\pgfqpoint{4.940651in}{4.026398in}}%
\pgfpathlineto{\pgfqpoint{4.955686in}{4.036677in}}%
\pgfpathlineto{\pgfqpoint{4.970721in}{4.044507in}}%
\pgfpathlineto{\pgfqpoint{4.987260in}{4.050568in}}%
\pgfpathlineto{\pgfqpoint{5.003799in}{4.054254in}}%
\pgfpathlineto{\pgfqpoint{5.021841in}{4.055923in}}%
\pgfpathlineto{\pgfqpoint{5.041387in}{4.055388in}}%
\pgfpathlineto{\pgfqpoint{5.062437in}{4.052598in}}%
\pgfpathlineto{\pgfqpoint{5.087997in}{4.046862in}}%
\pgfpathlineto{\pgfqpoint{5.121074in}{4.036908in}}%
\pgfpathlineto{\pgfqpoint{5.253385in}{3.994272in}}%
\pgfpathlineto{\pgfqpoint{5.286463in}{3.987356in}}%
\pgfpathlineto{\pgfqpoint{5.316533in}{3.983250in}}%
\pgfpathlineto{\pgfqpoint{5.346604in}{3.981392in}}%
\pgfpathlineto{\pgfqpoint{5.376675in}{3.981852in}}%
\pgfpathlineto{\pgfqpoint{5.406745in}{3.984603in}}%
\pgfpathlineto{\pgfqpoint{5.438319in}{3.989832in}}%
\pgfpathlineto{\pgfqpoint{5.471397in}{3.997607in}}%
\pgfpathlineto{\pgfqpoint{5.510489in}{4.009223in}}%
\pgfpathlineto{\pgfqpoint{5.533042in}{4.016744in}}%
\pgfpathlineto{\pgfqpoint{5.534545in}{2.376000in}}%
\pgfpathlineto{\pgfqpoint{5.534545in}{2.376000in}}%
\pgfusepath{stroke}%
\end{pgfscope}%
\begin{pgfscope}%
\pgfpathrectangle{\pgfqpoint{0.800000in}{0.528000in}}{\pgfqpoint{4.960000in}{3.696000in}}%
\pgfusepath{clip}%
\pgfsetrectcap%
\pgfsetroundjoin%
\pgfsetlinewidth{1.505625pt}%
\definecolor{currentstroke}{rgb}{1.000000,0.498039,0.054902}%
\pgfsetstrokecolor{currentstroke}%
\pgfsetdash{}{0pt}%
\pgfpathmoveto{\pgfqpoint{1.025455in}{3.636187in}}%
\pgfpathlineto{\pgfqpoint{2.527482in}{3.636173in}}%
\pgfpathlineto{\pgfqpoint{2.528986in}{2.376053in}}%
\pgfpathlineto{\pgfqpoint{2.706403in}{2.376050in}}%
\pgfpathlineto{\pgfqpoint{2.707906in}{3.909394in}}%
\pgfpathlineto{\pgfqpoint{2.725949in}{3.948434in}}%
\pgfpathlineto{\pgfqpoint{2.740984in}{3.975906in}}%
\pgfpathlineto{\pgfqpoint{2.756019in}{3.998889in}}%
\pgfpathlineto{\pgfqpoint{2.769551in}{4.015904in}}%
\pgfpathlineto{\pgfqpoint{2.783083in}{4.029623in}}%
\pgfpathlineto{\pgfqpoint{2.796615in}{4.040238in}}%
\pgfpathlineto{\pgfqpoint{2.810146in}{4.047955in}}%
\pgfpathlineto{\pgfqpoint{2.823678in}{4.052987in}}%
\pgfpathlineto{\pgfqpoint{2.837210in}{4.055552in}}%
\pgfpathlineto{\pgfqpoint{2.850742in}{4.055868in}}%
\pgfpathlineto{\pgfqpoint{2.865777in}{4.053842in}}%
\pgfpathlineto{\pgfqpoint{2.880812in}{4.049590in}}%
\pgfpathlineto{\pgfqpoint{2.897351in}{4.042668in}}%
\pgfpathlineto{\pgfqpoint{2.916897in}{4.031907in}}%
\pgfpathlineto{\pgfqpoint{2.939450in}{4.016661in}}%
\pgfpathlineto{\pgfqpoint{2.965010in}{3.996568in}}%
\pgfpathlineto{\pgfqpoint{2.998088in}{3.967495in}}%
\pgfpathlineto{\pgfqpoint{3.046201in}{3.921849in}}%
\pgfpathlineto{\pgfqpoint{3.169490in}{3.803850in}}%
\pgfpathlineto{\pgfqpoint{3.232639in}{3.747188in}}%
\pgfpathlineto{\pgfqpoint{3.322851in}{3.669668in}}%
\pgfpathlineto{\pgfqpoint{3.398027in}{3.603951in}}%
\pgfpathlineto{\pgfqpoint{3.443133in}{3.561535in}}%
\pgfpathlineto{\pgfqpoint{3.479218in}{3.524671in}}%
\pgfpathlineto{\pgfqpoint{3.512296in}{3.487658in}}%
\pgfpathlineto{\pgfqpoint{3.542366in}{3.450540in}}%
\pgfpathlineto{\pgfqpoint{3.569430in}{3.413615in}}%
\pgfpathlineto{\pgfqpoint{3.594990in}{3.375039in}}%
\pgfpathlineto{\pgfqpoint{3.619046in}{3.334834in}}%
\pgfpathlineto{\pgfqpoint{3.643103in}{3.290196in}}%
\pgfpathlineto{\pgfqpoint{3.665656in}{3.243668in}}%
\pgfpathlineto{\pgfqpoint{3.688209in}{3.191907in}}%
\pgfpathlineto{\pgfqpoint{3.709258in}{3.138185in}}%
\pgfpathlineto{\pgfqpoint{3.730308in}{3.078510in}}%
\pgfpathlineto{\pgfqpoint{3.751357in}{3.012082in}}%
\pgfpathlineto{\pgfqpoint{3.772407in}{2.938023in}}%
\pgfpathlineto{\pgfqpoint{3.793456in}{2.855371in}}%
\pgfpathlineto{\pgfqpoint{3.814505in}{2.763094in}}%
\pgfpathlineto{\pgfqpoint{3.835555in}{2.660116in}}%
\pgfpathlineto{\pgfqpoint{3.856604in}{2.545354in}}%
\pgfpathlineto{\pgfqpoint{3.877654in}{2.417800in}}%
\pgfpathlineto{\pgfqpoint{3.898703in}{2.276624in}}%
\pgfpathlineto{\pgfqpoint{3.921256in}{2.109729in}}%
\pgfpathlineto{\pgfqpoint{3.945313in}{1.914159in}}%
\pgfpathlineto{\pgfqpoint{3.972376in}{1.675005in}}%
\pgfpathlineto{\pgfqpoint{4.053567in}{0.940555in}}%
\pgfpathlineto{\pgfqpoint{4.067099in}{0.844549in}}%
\pgfpathlineto{\pgfqpoint{4.077623in}{0.782929in}}%
\pgfpathlineto{\pgfqpoint{4.086645in}{0.741463in}}%
\pgfpathlineto{\pgfqpoint{4.094162in}{0.716292in}}%
\pgfpathlineto{\pgfqpoint{4.100176in}{0.703068in}}%
\pgfpathlineto{\pgfqpoint{4.104687in}{0.697544in}}%
\pgfpathlineto{\pgfqpoint{4.107694in}{0.696080in}}%
\pgfpathlineto{\pgfqpoint{4.110701in}{0.696471in}}%
\pgfpathlineto{\pgfqpoint{4.113708in}{0.698781in}}%
\pgfpathlineto{\pgfqpoint{4.118219in}{0.705983in}}%
\pgfpathlineto{\pgfqpoint{4.122729in}{0.717855in}}%
\pgfpathlineto{\pgfqpoint{4.128744in}{0.741282in}}%
\pgfpathlineto{\pgfqpoint{4.134758in}{0.773774in}}%
\pgfpathlineto{\pgfqpoint{4.142275in}{0.827676in}}%
\pgfpathlineto{\pgfqpoint{4.151296in}{0.912529in}}%
\pgfpathlineto{\pgfqpoint{4.160318in}{1.019846in}}%
\pgfpathlineto{\pgfqpoint{4.170842in}{1.173358in}}%
\pgfpathlineto{\pgfqpoint{4.182871in}{1.384636in}}%
\pgfpathlineto{\pgfqpoint{4.196402in}{1.663374in}}%
\pgfpathlineto{\pgfqpoint{4.214445in}{2.087434in}}%
\pgfpathlineto{\pgfqpoint{4.268572in}{3.403048in}}%
\pgfpathlineto{\pgfqpoint{4.282104in}{3.658474in}}%
\pgfpathlineto{\pgfqpoint{4.292628in}{3.818241in}}%
\pgfpathlineto{\pgfqpoint{4.301650in}{3.924377in}}%
\pgfpathlineto{\pgfqpoint{4.309167in}{3.989580in}}%
\pgfpathlineto{\pgfqpoint{4.315181in}{4.025982in}}%
\pgfpathlineto{\pgfqpoint{4.319692in}{4.043942in}}%
\pgfpathlineto{\pgfqpoint{4.324203in}{4.053850in}}%
\pgfpathlineto{\pgfqpoint{4.327210in}{4.055983in}}%
\pgfpathlineto{\pgfqpoint{4.330217in}{4.054550in}}%
\pgfpathlineto{\pgfqpoint{4.333224in}{4.049571in}}%
\pgfpathlineto{\pgfqpoint{4.337734in}{4.035510in}}%
\pgfpathlineto{\pgfqpoint{4.342245in}{4.013648in}}%
\pgfpathlineto{\pgfqpoint{4.348259in}{3.972624in}}%
\pgfpathlineto{\pgfqpoint{4.355777in}{3.902894in}}%
\pgfpathlineto{\pgfqpoint{4.364798in}{3.793560in}}%
\pgfpathlineto{\pgfqpoint{4.375323in}{3.633361in}}%
\pgfpathlineto{\pgfqpoint{4.387351in}{3.412089in}}%
\pgfpathlineto{\pgfqpoint{4.402386in}{3.088032in}}%
\pgfpathlineto{\pgfqpoint{4.421932in}{2.610678in}}%
\pgfpathlineto{\pgfqpoint{4.473052in}{1.331588in}}%
\pgfpathlineto{\pgfqpoint{4.486584in}{1.061967in}}%
\pgfpathlineto{\pgfqpoint{4.497109in}{0.894660in}}%
\pgfpathlineto{\pgfqpoint{4.504626in}{0.803081in}}%
\pgfpathlineto{\pgfqpoint{4.512144in}{0.738110in}}%
\pgfpathlineto{\pgfqpoint{4.516655in}{0.713022in}}%
\pgfpathlineto{\pgfqpoint{4.521165in}{0.698994in}}%
\pgfpathlineto{\pgfqpoint{4.524172in}{0.696014in}}%
\pgfpathlineto{\pgfqpoint{4.525676in}{0.696480in}}%
\pgfpathlineto{\pgfqpoint{4.528683in}{0.701392in}}%
\pgfpathlineto{\pgfqpoint{4.531690in}{0.711686in}}%
\pgfpathlineto{\pgfqpoint{4.536201in}{0.737378in}}%
\pgfpathlineto{\pgfqpoint{4.542215in}{0.791012in}}%
\pgfpathlineto{\pgfqpoint{4.548229in}{0.866870in}}%
\pgfpathlineto{\pgfqpoint{4.555746in}{0.992497in}}%
\pgfpathlineto{\pgfqpoint{4.564768in}{1.186394in}}%
\pgfpathlineto{\pgfqpoint{4.575292in}{1.466159in}}%
\pgfpathlineto{\pgfqpoint{4.588824in}{1.893068in}}%
\pgfpathlineto{\pgfqpoint{4.638441in}{3.540922in}}%
\pgfpathlineto{\pgfqpoint{4.648965in}{3.782745in}}%
\pgfpathlineto{\pgfqpoint{4.656483in}{3.912005in}}%
\pgfpathlineto{\pgfqpoint{4.662497in}{3.986704in}}%
\pgfpathlineto{\pgfqpoint{4.668511in}{4.034639in}}%
\pgfpathlineto{\pgfqpoint{4.673022in}{4.052626in}}%
\pgfpathlineto{\pgfqpoint{4.676029in}{4.056000in}}%
\pgfpathlineto{\pgfqpoint{4.677533in}{4.055102in}}%
\pgfpathlineto{\pgfqpoint{4.680540in}{4.048156in}}%
\pgfpathlineto{\pgfqpoint{4.683547in}{4.034390in}}%
\pgfpathlineto{\pgfqpoint{4.688057in}{4.001129in}}%
\pgfpathlineto{\pgfqpoint{4.694071in}{3.933855in}}%
\pgfpathlineto{\pgfqpoint{4.701589in}{3.814692in}}%
\pgfpathlineto{\pgfqpoint{4.710610in}{3.624512in}}%
\pgfpathlineto{\pgfqpoint{4.721135in}{3.346363in}}%
\pgfpathlineto{\pgfqpoint{4.734667in}{2.920141in}}%
\pgfpathlineto{\pgfqpoint{4.758723in}{2.069565in}}%
\pgfpathlineto{\pgfqpoint{4.778269in}{1.414434in}}%
\pgfpathlineto{\pgfqpoint{4.790297in}{1.084091in}}%
\pgfpathlineto{\pgfqpoint{4.799319in}{0.893043in}}%
\pgfpathlineto{\pgfqpoint{4.806836in}{0.779431in}}%
\pgfpathlineto{\pgfqpoint{4.812850in}{0.722139in}}%
\pgfpathlineto{\pgfqpoint{4.817361in}{0.700204in}}%
\pgfpathlineto{\pgfqpoint{4.820368in}{0.696000in}}%
\pgfpathlineto{\pgfqpoint{4.821872in}{0.697087in}}%
\pgfpathlineto{\pgfqpoint{4.824879in}{0.705717in}}%
\pgfpathlineto{\pgfqpoint{4.827886in}{0.723028in}}%
\pgfpathlineto{\pgfqpoint{4.832396in}{0.765358in}}%
\pgfpathlineto{\pgfqpoint{4.838410in}{0.852189in}}%
\pgfpathlineto{\pgfqpoint{4.844425in}{0.972893in}}%
\pgfpathlineto{\pgfqpoint{4.851942in}{1.168620in}}%
\pgfpathlineto{\pgfqpoint{4.860963in}{1.461757in}}%
\pgfpathlineto{\pgfqpoint{4.872992in}{1.928443in}}%
\pgfpathlineto{\pgfqpoint{4.912083in}{3.511136in}}%
\pgfpathlineto{\pgfqpoint{4.921105in}{3.767968in}}%
\pgfpathlineto{\pgfqpoint{4.928622in}{3.924044in}}%
\pgfpathlineto{\pgfqpoint{4.934636in}{4.006640in}}%
\pgfpathlineto{\pgfqpoint{4.939147in}{4.042639in}}%
\pgfpathlineto{\pgfqpoint{4.942154in}{4.054031in}}%
\pgfpathlineto{\pgfqpoint{4.943658in}{4.055923in}}%
\pgfpathlineto{\pgfqpoint{4.945161in}{4.055280in}}%
\pgfpathlineto{\pgfqpoint{4.948168in}{4.046409in}}%
\pgfpathlineto{\pgfqpoint{4.951175in}{4.027500in}}%
\pgfpathlineto{\pgfqpoint{4.955686in}{3.980640in}}%
\pgfpathlineto{\pgfqpoint{4.961700in}{3.884815in}}%
\pgfpathlineto{\pgfqpoint{4.969218in}{3.714997in}}%
\pgfpathlineto{\pgfqpoint{4.978239in}{3.446499in}}%
\pgfpathlineto{\pgfqpoint{4.988764in}{3.062080in}}%
\pgfpathlineto{\pgfqpoint{5.006806in}{2.301109in}}%
\pgfpathlineto{\pgfqpoint{5.027855in}{1.434250in}}%
\pgfpathlineto{\pgfqpoint{5.038380in}{1.086422in}}%
\pgfpathlineto{\pgfqpoint{5.045898in}{0.896265in}}%
\pgfpathlineto{\pgfqpoint{5.051912in}{0.786501in}}%
\pgfpathlineto{\pgfqpoint{5.056422in}{0.731561in}}%
\pgfpathlineto{\pgfqpoint{5.060933in}{0.701599in}}%
\pgfpathlineto{\pgfqpoint{5.063940in}{0.696003in}}%
\pgfpathlineto{\pgfqpoint{5.065444in}{0.697599in}}%
\pgfpathlineto{\pgfqpoint{5.068451in}{0.709669in}}%
\pgfpathlineto{\pgfqpoint{5.071458in}{0.733626in}}%
\pgfpathlineto{\pgfqpoint{5.075968in}{0.791793in}}%
\pgfpathlineto{\pgfqpoint{5.081982in}{0.910045in}}%
\pgfpathlineto{\pgfqpoint{5.089500in}{1.119639in}}%
\pgfpathlineto{\pgfqpoint{5.098521in}{1.450636in}}%
\pgfpathlineto{\pgfqpoint{5.110550in}{1.991439in}}%
\pgfpathlineto{\pgfqpoint{5.140620in}{3.407825in}}%
\pgfpathlineto{\pgfqpoint{5.149641in}{3.722793in}}%
\pgfpathlineto{\pgfqpoint{5.157159in}{3.912039in}}%
\pgfpathlineto{\pgfqpoint{5.163173in}{4.008536in}}%
\pgfpathlineto{\pgfqpoint{5.167684in}{4.046851in}}%
\pgfpathlineto{\pgfqpoint{5.170691in}{4.055804in}}%
\pgfpathlineto{\pgfqpoint{5.172194in}{4.055280in}}%
\pgfpathlineto{\pgfqpoint{5.173698in}{4.051428in}}%
\pgfpathlineto{\pgfqpoint{5.176705in}{4.033790in}}%
\pgfpathlineto{\pgfqpoint{5.181216in}{3.982876in}}%
\pgfpathlineto{\pgfqpoint{5.185726in}{3.903541in}}%
\pgfpathlineto{\pgfqpoint{5.191740in}{3.756171in}}%
\pgfpathlineto{\pgfqpoint{5.199258in}{3.511962in}}%
\pgfpathlineto{\pgfqpoint{5.209783in}{3.079967in}}%
\pgfpathlineto{\pgfqpoint{5.226322in}{2.280754in}}%
\pgfpathlineto{\pgfqpoint{5.244364in}{1.432931in}}%
\pgfpathlineto{\pgfqpoint{5.253385in}{1.092837in}}%
\pgfpathlineto{\pgfqpoint{5.260903in}{0.879545in}}%
\pgfpathlineto{\pgfqpoint{5.266917in}{0.764418in}}%
\pgfpathlineto{\pgfqpoint{5.271427in}{0.713954in}}%
\pgfpathlineto{\pgfqpoint{5.274435in}{0.698315in}}%
\pgfpathlineto{\pgfqpoint{5.275938in}{0.696025in}}%
\pgfpathlineto{\pgfqpoint{5.277442in}{0.697461in}}%
\pgfpathlineto{\pgfqpoint{5.280449in}{0.711579in}}%
\pgfpathlineto{\pgfqpoint{5.283456in}{0.740720in}}%
\pgfpathlineto{\pgfqpoint{5.287966in}{0.812366in}}%
\pgfpathlineto{\pgfqpoint{5.293980in}{0.958424in}}%
\pgfpathlineto{\pgfqpoint{5.301498in}{1.215831in}}%
\pgfpathlineto{\pgfqpoint{5.310519in}{1.616118in}}%
\pgfpathlineto{\pgfqpoint{5.324051in}{2.332491in}}%
\pgfpathlineto{\pgfqpoint{5.342093in}{3.282394in}}%
\pgfpathlineto{\pgfqpoint{5.351115in}{3.658308in}}%
\pgfpathlineto{\pgfqpoint{5.358632in}{3.885944in}}%
\pgfpathlineto{\pgfqpoint{5.364646in}{4.001592in}}%
\pgfpathlineto{\pgfqpoint{5.369157in}{4.046498in}}%
\pgfpathlineto{\pgfqpoint{5.372164in}{4.055955in}}%
\pgfpathlineto{\pgfqpoint{5.373668in}{4.054504in}}%
\pgfpathlineto{\pgfqpoint{5.376675in}{4.039287in}}%
\pgfpathlineto{\pgfqpoint{5.379682in}{4.007822in}}%
\pgfpathlineto{\pgfqpoint{5.384192in}{3.930950in}}%
\pgfpathlineto{\pgfqpoint{5.390206in}{3.776022in}}%
\pgfpathlineto{\pgfqpoint{5.397724in}{3.507086in}}%
\pgfpathlineto{\pgfqpoint{5.406745in}{3.095848in}}%
\pgfpathlineto{\pgfqpoint{5.421781in}{2.289520in}}%
\pgfpathlineto{\pgfqpoint{5.438319in}{1.427465in}}%
\pgfpathlineto{\pgfqpoint{5.447341in}{1.057495in}}%
\pgfpathlineto{\pgfqpoint{5.454858in}{0.838003in}}%
\pgfpathlineto{\pgfqpoint{5.459369in}{0.752677in}}%
\pgfpathlineto{\pgfqpoint{5.463879in}{0.705437in}}%
\pgfpathlineto{\pgfqpoint{5.466887in}{0.696006in}}%
\pgfpathlineto{\pgfqpoint{5.468390in}{0.698037in}}%
\pgfpathlineto{\pgfqpoint{5.471397in}{0.715687in}}%
\pgfpathlineto{\pgfqpoint{5.474404in}{0.751438in}}%
\pgfpathlineto{\pgfqpoint{5.478915in}{0.838503in}}%
\pgfpathlineto{\pgfqpoint{5.484929in}{1.014310in}}%
\pgfpathlineto{\pgfqpoint{5.492447in}{1.320156in}}%
\pgfpathlineto{\pgfqpoint{5.501468in}{1.786248in}}%
\pgfpathlineto{\pgfqpoint{5.533042in}{3.553058in}}%
\pgfpathlineto{\pgfqpoint{5.534545in}{2.376000in}}%
\pgfpathlineto{\pgfqpoint{5.534545in}{2.376000in}}%
\pgfusepath{stroke}%
\end{pgfscope}%
\begin{pgfscope}%
\pgfsetrectcap%
\pgfsetmiterjoin%
\pgfsetlinewidth{0.803000pt}%
\definecolor{currentstroke}{rgb}{0.000000,0.000000,0.000000}%
\pgfsetstrokecolor{currentstroke}%
\pgfsetdash{}{0pt}%
\pgfpathmoveto{\pgfqpoint{0.800000in}{0.528000in}}%
\pgfpathlineto{\pgfqpoint{0.800000in}{4.224000in}}%
\pgfusepath{stroke}%
\end{pgfscope}%
\begin{pgfscope}%
\pgfsetrectcap%
\pgfsetmiterjoin%
\pgfsetlinewidth{0.803000pt}%
\definecolor{currentstroke}{rgb}{0.000000,0.000000,0.000000}%
\pgfsetstrokecolor{currentstroke}%
\pgfsetdash{}{0pt}%
\pgfpathmoveto{\pgfqpoint{5.760000in}{0.528000in}}%
\pgfpathlineto{\pgfqpoint{5.760000in}{4.224000in}}%
\pgfusepath{stroke}%
\end{pgfscope}%
\begin{pgfscope}%
\pgfsetrectcap%
\pgfsetmiterjoin%
\pgfsetlinewidth{0.803000pt}%
\definecolor{currentstroke}{rgb}{0.000000,0.000000,0.000000}%
\pgfsetstrokecolor{currentstroke}%
\pgfsetdash{}{0pt}%
\pgfpathmoveto{\pgfqpoint{0.800000in}{0.528000in}}%
\pgfpathlineto{\pgfqpoint{5.760000in}{0.528000in}}%
\pgfusepath{stroke}%
\end{pgfscope}%
\begin{pgfscope}%
\pgfsetrectcap%
\pgfsetmiterjoin%
\pgfsetlinewidth{0.803000pt}%
\definecolor{currentstroke}{rgb}{0.000000,0.000000,0.000000}%
\pgfsetstrokecolor{currentstroke}%
\pgfsetdash{}{0pt}%
\pgfpathmoveto{\pgfqpoint{0.800000in}{4.224000in}}%
\pgfpathlineto{\pgfqpoint{5.760000in}{4.224000in}}%
\pgfusepath{stroke}%
\end{pgfscope}%
\begin{pgfscope}%
\definecolor{textcolor}{rgb}{0.000000,0.000000,0.000000}%
\pgfsetstrokecolor{textcolor}%
\pgfsetfillcolor{textcolor}%
\pgftext[x=3.280000in,y=4.307333in,,base]{\color{textcolor}\rmfamily\fontsize{12.000000}{14.400000}\selectfont Electrical power over time - fault 1}%
\end{pgfscope}%
\begin{pgfscope}%
\pgfsetbuttcap%
\pgfsetmiterjoin%
\definecolor{currentfill}{rgb}{1.000000,1.000000,1.000000}%
\pgfsetfillcolor{currentfill}%
\pgfsetfillopacity{0.800000}%
\pgfsetlinewidth{1.003750pt}%
\definecolor{currentstroke}{rgb}{0.800000,0.800000,0.800000}%
\pgfsetstrokecolor{currentstroke}%
\pgfsetstrokeopacity{0.800000}%
\pgfsetdash{}{0pt}%
\pgfpathmoveto{\pgfqpoint{0.897222in}{0.597444in}}%
\pgfpathlineto{\pgfqpoint{2.740115in}{0.597444in}}%
\pgfpathquadraticcurveto{\pgfqpoint{2.767893in}{0.597444in}}{\pgfqpoint{2.767893in}{0.625222in}}%
\pgfpathlineto{\pgfqpoint{2.767893in}{1.015114in}}%
\pgfpathquadraticcurveto{\pgfqpoint{2.767893in}{1.042892in}}{\pgfqpoint{2.740115in}{1.042892in}}%
\pgfpathlineto{\pgfqpoint{0.897222in}{1.042892in}}%
\pgfpathquadraticcurveto{\pgfqpoint{0.869444in}{1.042892in}}{\pgfqpoint{0.869444in}{1.015114in}}%
\pgfpathlineto{\pgfqpoint{0.869444in}{0.625222in}}%
\pgfpathquadraticcurveto{\pgfqpoint{0.869444in}{0.597444in}}{\pgfqpoint{0.897222in}{0.597444in}}%
\pgfpathlineto{\pgfqpoint{0.897222in}{0.597444in}}%
\pgfpathclose%
\pgfusepath{stroke,fill}%
\end{pgfscope}%
\begin{pgfscope}%
\pgfsetrectcap%
\pgfsetroundjoin%
\pgfsetlinewidth{1.505625pt}%
\definecolor{currentstroke}{rgb}{0.121569,0.466667,0.705882}%
\pgfsetstrokecolor{currentstroke}%
\pgfsetdash{}{0pt}%
\pgfpathmoveto{\pgfqpoint{0.925000in}{0.933748in}}%
\pgfpathlineto{\pgfqpoint{1.063889in}{0.933748in}}%
\pgfpathlineto{\pgfqpoint{1.202778in}{0.933748in}}%
\pgfusepath{stroke}%
\end{pgfscope}%
\begin{pgfscope}%
\definecolor{textcolor}{rgb}{0.000000,0.000000,0.000000}%
\pgfsetstrokecolor{textcolor}%
\pgfsetfillcolor{textcolor}%
\pgftext[x=1.313889in,y=0.885137in,left,base]{\color{textcolor}\rmfamily\fontsize{10.000000}{12.000000}\selectfont \(\displaystyle \Delta P\) - stable scenario}%
\end{pgfscope}%
\begin{pgfscope}%
\pgfsetrectcap%
\pgfsetroundjoin%
\pgfsetlinewidth{1.505625pt}%
\definecolor{currentstroke}{rgb}{1.000000,0.498039,0.054902}%
\pgfsetstrokecolor{currentstroke}%
\pgfsetdash{}{0pt}%
\pgfpathmoveto{\pgfqpoint{0.925000in}{0.731857in}}%
\pgfpathlineto{\pgfqpoint{1.063889in}{0.731857in}}%
\pgfpathlineto{\pgfqpoint{1.202778in}{0.731857in}}%
\pgfusepath{stroke}%
\end{pgfscope}%
\begin{pgfscope}%
\definecolor{textcolor}{rgb}{0.000000,0.000000,0.000000}%
\pgfsetstrokecolor{textcolor}%
\pgfsetfillcolor{textcolor}%
\pgftext[x=1.313889in,y=0.683246in,left,base]{\color{textcolor}\rmfamily\fontsize{10.000000}{12.000000}\selectfont \(\displaystyle \Delta P\) - unstable scenario}%
\end{pgfscope}%
\end{pgfpicture}%
\makeatother%
\endgroup%


\section{Fault 2}
\label{app:fault2}

%% Creator: Matplotlib, PGF backend
%%
%% To include the figure in your LaTeX document, write
%%   \input{<filename>.pgf}
%%
%% Make sure the required packages are loaded in your preamble
%%   \usepackage{pgf}
%%
%% Also ensure that all the required font packages are loaded; for instance,
%% the lmodern package is sometimes necessary when using math font.
%%   \usepackage{lmodern}
%%
%% Figures using additional raster images can only be included by \input if
%% they are in the same directory as the main LaTeX file. For loading figures
%% from other directories you can use the `import` package
%%   \usepackage{import}
%%
%% and then include the figures with
%%   \import{<path to file>}{<filename>.pgf}
%%
%% Matplotlib used the following preamble
%%   
%%   \usepackage{fontspec}
%%   \setmainfont{Charter.ttc}[Path=\detokenize{/System/Library/Fonts/Supplemental/}]
%%   \setsansfont{DejaVuSans.ttf}[Path=\detokenize{/opt/homebrew/lib/python3.10/site-packages/matplotlib/mpl-data/fonts/ttf/}]
%%   \setmonofont{DejaVuSansMono.ttf}[Path=\detokenize{/opt/homebrew/lib/python3.10/site-packages/matplotlib/mpl-data/fonts/ttf/}]
%%   \makeatletter\@ifpackageloaded{underscore}{}{\usepackage[strings]{underscore}}\makeatother
%%
\begingroup%
\makeatletter%
\begin{pgfpicture}%
\pgfpathrectangle{\pgfpointorigin}{\pgfqpoint{6.000000in}{8.000000in}}%
\pgfusepath{use as bounding box, clip}%
\begin{pgfscope}%
\pgfsetbuttcap%
\pgfsetmiterjoin%
\definecolor{currentfill}{rgb}{1.000000,1.000000,1.000000}%
\pgfsetfillcolor{currentfill}%
\pgfsetlinewidth{0.000000pt}%
\definecolor{currentstroke}{rgb}{1.000000,1.000000,1.000000}%
\pgfsetstrokecolor{currentstroke}%
\pgfsetdash{}{0pt}%
\pgfpathmoveto{\pgfqpoint{0.000000in}{0.000000in}}%
\pgfpathlineto{\pgfqpoint{6.000000in}{0.000000in}}%
\pgfpathlineto{\pgfqpoint{6.000000in}{8.000000in}}%
\pgfpathlineto{\pgfqpoint{0.000000in}{8.000000in}}%
\pgfpathlineto{\pgfqpoint{0.000000in}{0.000000in}}%
\pgfpathclose%
\pgfusepath{fill}%
\end{pgfscope}%
\begin{pgfscope}%
\pgfsetbuttcap%
\pgfsetmiterjoin%
\definecolor{currentfill}{rgb}{1.000000,1.000000,1.000000}%
\pgfsetfillcolor{currentfill}%
\pgfsetlinewidth{0.000000pt}%
\definecolor{currentstroke}{rgb}{0.000000,0.000000,0.000000}%
\pgfsetstrokecolor{currentstroke}%
\pgfsetstrokeopacity{0.000000}%
\pgfsetdash{}{0pt}%
\pgfpathmoveto{\pgfqpoint{0.750000in}{3.960000in}}%
\pgfpathlineto{\pgfqpoint{5.400000in}{3.960000in}}%
\pgfpathlineto{\pgfqpoint{5.400000in}{7.040000in}}%
\pgfpathlineto{\pgfqpoint{0.750000in}{7.040000in}}%
\pgfpathlineto{\pgfqpoint{0.750000in}{3.960000in}}%
\pgfpathclose%
\pgfusepath{fill}%
\end{pgfscope}%
\begin{pgfscope}%
\pgfpathrectangle{\pgfqpoint{0.750000in}{3.960000in}}{\pgfqpoint{4.650000in}{3.080000in}}%
\pgfusepath{clip}%
\pgfsetbuttcap%
\pgfsetroundjoin%
\definecolor{currentfill}{rgb}{0.900000,0.900000,0.900000}%
\pgfsetfillcolor{currentfill}%
\pgfsetlinewidth{1.003750pt}%
\definecolor{currentstroke}{rgb}{0.500000,0.500000,0.500000}%
\pgfsetstrokecolor{currentstroke}%
\pgfsetdash{}{0pt}%
\pgfsys@defobject{currentmarker}{\pgfqpoint{2.005500in}{5.441634in}}{\pgfqpoint{3.228434in}{6.161131in}}{%
\pgfpathmoveto{\pgfqpoint{2.005500in}{6.161131in}}%
\pgfpathlineto{\pgfqpoint{2.005500in}{5.441634in}}%
\pgfpathlineto{\pgfqpoint{2.030458in}{5.463448in}}%
\pgfpathlineto{\pgfqpoint{2.055416in}{5.484835in}}%
\pgfpathlineto{\pgfqpoint{2.080374in}{5.505788in}}%
\pgfpathlineto{\pgfqpoint{2.105331in}{5.526301in}}%
\pgfpathlineto{\pgfqpoint{2.130289in}{5.546369in}}%
\pgfpathlineto{\pgfqpoint{2.155247in}{5.565986in}}%
\pgfpathlineto{\pgfqpoint{2.180205in}{5.585146in}}%
\pgfpathlineto{\pgfqpoint{2.205163in}{5.603845in}}%
\pgfpathlineto{\pgfqpoint{2.230121in}{5.622076in}}%
\pgfpathlineto{\pgfqpoint{2.255078in}{5.639834in}}%
\pgfpathlineto{\pgfqpoint{2.280036in}{5.657115in}}%
\pgfpathlineto{\pgfqpoint{2.304994in}{5.673913in}}%
\pgfpathlineto{\pgfqpoint{2.329952in}{5.690224in}}%
\pgfpathlineto{\pgfqpoint{2.354910in}{5.706043in}}%
\pgfpathlineto{\pgfqpoint{2.379868in}{5.721366in}}%
\pgfpathlineto{\pgfqpoint{2.404825in}{5.736188in}}%
\pgfpathlineto{\pgfqpoint{2.429783in}{5.750505in}}%
\pgfpathlineto{\pgfqpoint{2.454741in}{5.764313in}}%
\pgfpathlineto{\pgfqpoint{2.479699in}{5.777608in}}%
\pgfpathlineto{\pgfqpoint{2.504657in}{5.790386in}}%
\pgfpathlineto{\pgfqpoint{2.529615in}{5.802643in}}%
\pgfpathlineto{\pgfqpoint{2.554572in}{5.814377in}}%
\pgfpathlineto{\pgfqpoint{2.579530in}{5.825584in}}%
\pgfpathlineto{\pgfqpoint{2.604488in}{5.836260in}}%
\pgfpathlineto{\pgfqpoint{2.629446in}{5.846403in}}%
\pgfpathlineto{\pgfqpoint{2.654404in}{5.856009in}}%
\pgfpathlineto{\pgfqpoint{2.679362in}{5.865076in}}%
\pgfpathlineto{\pgfqpoint{2.704319in}{5.873602in}}%
\pgfpathlineto{\pgfqpoint{2.729277in}{5.881584in}}%
\pgfpathlineto{\pgfqpoint{2.754235in}{5.889019in}}%
\pgfpathlineto{\pgfqpoint{2.779193in}{5.895906in}}%
\pgfpathlineto{\pgfqpoint{2.804151in}{5.902242in}}%
\pgfpathlineto{\pgfqpoint{2.829109in}{5.908026in}}%
\pgfpathlineto{\pgfqpoint{2.854066in}{5.913257in}}%
\pgfpathlineto{\pgfqpoint{2.879024in}{5.917932in}}%
\pgfpathlineto{\pgfqpoint{2.903982in}{5.922050in}}%
\pgfpathlineto{\pgfqpoint{2.928940in}{5.925611in}}%
\pgfpathlineto{\pgfqpoint{2.953898in}{5.928613in}}%
\pgfpathlineto{\pgfqpoint{2.978856in}{5.931055in}}%
\pgfpathlineto{\pgfqpoint{3.003813in}{5.932936in}}%
\pgfpathlineto{\pgfqpoint{3.028771in}{5.934257in}}%
\pgfpathlineto{\pgfqpoint{3.053729in}{5.935016in}}%
\pgfpathlineto{\pgfqpoint{3.078687in}{5.935214in}}%
\pgfpathlineto{\pgfqpoint{3.103645in}{5.934850in}}%
\pgfpathlineto{\pgfqpoint{3.128603in}{5.933925in}}%
\pgfpathlineto{\pgfqpoint{3.153560in}{5.932439in}}%
\pgfpathlineto{\pgfqpoint{3.178518in}{5.930391in}}%
\pgfpathlineto{\pgfqpoint{3.203476in}{5.927784in}}%
\pgfpathlineto{\pgfqpoint{3.228434in}{5.924617in}}%
\pgfpathlineto{\pgfqpoint{3.228434in}{6.161131in}}%
\pgfpathlineto{\pgfqpoint{3.228434in}{6.161131in}}%
\pgfpathlineto{\pgfqpoint{3.203476in}{6.161131in}}%
\pgfpathlineto{\pgfqpoint{3.178518in}{6.161131in}}%
\pgfpathlineto{\pgfqpoint{3.153560in}{6.161131in}}%
\pgfpathlineto{\pgfqpoint{3.128603in}{6.161131in}}%
\pgfpathlineto{\pgfqpoint{3.103645in}{6.161131in}}%
\pgfpathlineto{\pgfqpoint{3.078687in}{6.161131in}}%
\pgfpathlineto{\pgfqpoint{3.053729in}{6.161131in}}%
\pgfpathlineto{\pgfqpoint{3.028771in}{6.161131in}}%
\pgfpathlineto{\pgfqpoint{3.003813in}{6.161131in}}%
\pgfpathlineto{\pgfqpoint{2.978856in}{6.161131in}}%
\pgfpathlineto{\pgfqpoint{2.953898in}{6.161131in}}%
\pgfpathlineto{\pgfqpoint{2.928940in}{6.161131in}}%
\pgfpathlineto{\pgfqpoint{2.903982in}{6.161131in}}%
\pgfpathlineto{\pgfqpoint{2.879024in}{6.161131in}}%
\pgfpathlineto{\pgfqpoint{2.854066in}{6.161131in}}%
\pgfpathlineto{\pgfqpoint{2.829109in}{6.161131in}}%
\pgfpathlineto{\pgfqpoint{2.804151in}{6.161131in}}%
\pgfpathlineto{\pgfqpoint{2.779193in}{6.161131in}}%
\pgfpathlineto{\pgfqpoint{2.754235in}{6.161131in}}%
\pgfpathlineto{\pgfqpoint{2.729277in}{6.161131in}}%
\pgfpathlineto{\pgfqpoint{2.704319in}{6.161131in}}%
\pgfpathlineto{\pgfqpoint{2.679362in}{6.161131in}}%
\pgfpathlineto{\pgfqpoint{2.654404in}{6.161131in}}%
\pgfpathlineto{\pgfqpoint{2.629446in}{6.161131in}}%
\pgfpathlineto{\pgfqpoint{2.604488in}{6.161131in}}%
\pgfpathlineto{\pgfqpoint{2.579530in}{6.161131in}}%
\pgfpathlineto{\pgfqpoint{2.554572in}{6.161131in}}%
\pgfpathlineto{\pgfqpoint{2.529615in}{6.161131in}}%
\pgfpathlineto{\pgfqpoint{2.504657in}{6.161131in}}%
\pgfpathlineto{\pgfqpoint{2.479699in}{6.161131in}}%
\pgfpathlineto{\pgfqpoint{2.454741in}{6.161131in}}%
\pgfpathlineto{\pgfqpoint{2.429783in}{6.161131in}}%
\pgfpathlineto{\pgfqpoint{2.404825in}{6.161131in}}%
\pgfpathlineto{\pgfqpoint{2.379868in}{6.161131in}}%
\pgfpathlineto{\pgfqpoint{2.354910in}{6.161131in}}%
\pgfpathlineto{\pgfqpoint{2.329952in}{6.161131in}}%
\pgfpathlineto{\pgfqpoint{2.304994in}{6.161131in}}%
\pgfpathlineto{\pgfqpoint{2.280036in}{6.161131in}}%
\pgfpathlineto{\pgfqpoint{2.255078in}{6.161131in}}%
\pgfpathlineto{\pgfqpoint{2.230121in}{6.161131in}}%
\pgfpathlineto{\pgfqpoint{2.205163in}{6.161131in}}%
\pgfpathlineto{\pgfqpoint{2.180205in}{6.161131in}}%
\pgfpathlineto{\pgfqpoint{2.155247in}{6.161131in}}%
\pgfpathlineto{\pgfqpoint{2.130289in}{6.161131in}}%
\pgfpathlineto{\pgfqpoint{2.105331in}{6.161131in}}%
\pgfpathlineto{\pgfqpoint{2.080374in}{6.161131in}}%
\pgfpathlineto{\pgfqpoint{2.055416in}{6.161131in}}%
\pgfpathlineto{\pgfqpoint{2.030458in}{6.161131in}}%
\pgfpathlineto{\pgfqpoint{2.005500in}{6.161131in}}%
\pgfpathlineto{\pgfqpoint{2.005500in}{6.161131in}}%
\pgfpathclose%
\pgfusepath{stroke,fill}%
}%
\begin{pgfscope}%
\pgfsys@transformshift{0.000000in}{0.000000in}%
\pgfsys@useobject{currentmarker}{}%
\end{pgfscope}%
\end{pgfscope}%
\begin{pgfscope}%
\pgfpathrectangle{\pgfqpoint{0.750000in}{3.960000in}}{\pgfqpoint{4.650000in}{3.080000in}}%
\pgfusepath{clip}%
\pgfsetbuttcap%
\pgfsetroundjoin%
\definecolor{currentfill}{rgb}{0.900000,0.900000,0.900000}%
\pgfsetfillcolor{currentfill}%
\pgfsetlinewidth{1.003750pt}%
\definecolor{currentstroke}{rgb}{0.500000,0.500000,0.500000}%
\pgfsetstrokecolor{currentstroke}%
\pgfsetdash{}{0pt}%
\pgfsys@defobject{currentmarker}{\pgfqpoint{3.228434in}{6.161131in}}{\pgfqpoint{4.144500in}{6.879087in}}{%
\pgfpathmoveto{\pgfqpoint{3.228434in}{6.161131in}}%
\pgfpathlineto{\pgfqpoint{3.228434in}{6.879087in}}%
\pgfpathlineto{\pgfqpoint{3.247129in}{6.875018in}}%
\pgfpathlineto{\pgfqpoint{3.265824in}{6.870485in}}%
\pgfpathlineto{\pgfqpoint{3.284520in}{6.865487in}}%
\pgfpathlineto{\pgfqpoint{3.303215in}{6.860025in}}%
\pgfpathlineto{\pgfqpoint{3.321910in}{6.854101in}}%
\pgfpathlineto{\pgfqpoint{3.340605in}{6.847716in}}%
\pgfpathlineto{\pgfqpoint{3.359301in}{6.840869in}}%
\pgfpathlineto{\pgfqpoint{3.377996in}{6.833563in}}%
\pgfpathlineto{\pgfqpoint{3.396691in}{6.825799in}}%
\pgfpathlineto{\pgfqpoint{3.415386in}{6.817577in}}%
\pgfpathlineto{\pgfqpoint{3.434081in}{6.808900in}}%
\pgfpathlineto{\pgfqpoint{3.452777in}{6.799768in}}%
\pgfpathlineto{\pgfqpoint{3.471472in}{6.790183in}}%
\pgfpathlineto{\pgfqpoint{3.490167in}{6.780146in}}%
\pgfpathlineto{\pgfqpoint{3.508862in}{6.769660in}}%
\pgfpathlineto{\pgfqpoint{3.527558in}{6.758725in}}%
\pgfpathlineto{\pgfqpoint{3.546253in}{6.747344in}}%
\pgfpathlineto{\pgfqpoint{3.564948in}{6.735518in}}%
\pgfpathlineto{\pgfqpoint{3.583643in}{6.723249in}}%
\pgfpathlineto{\pgfqpoint{3.602338in}{6.710540in}}%
\pgfpathlineto{\pgfqpoint{3.621034in}{6.697392in}}%
\pgfpathlineto{\pgfqpoint{3.639729in}{6.683807in}}%
\pgfpathlineto{\pgfqpoint{3.658424in}{6.669787in}}%
\pgfpathlineto{\pgfqpoint{3.677119in}{6.655335in}}%
\pgfpathlineto{\pgfqpoint{3.695815in}{6.640453in}}%
\pgfpathlineto{\pgfqpoint{3.714510in}{6.625144in}}%
\pgfpathlineto{\pgfqpoint{3.733205in}{6.609409in}}%
\pgfpathlineto{\pgfqpoint{3.751900in}{6.593252in}}%
\pgfpathlineto{\pgfqpoint{3.770596in}{6.576675in}}%
\pgfpathlineto{\pgfqpoint{3.789291in}{6.559680in}}%
\pgfpathlineto{\pgfqpoint{3.807986in}{6.542271in}}%
\pgfpathlineto{\pgfqpoint{3.826681in}{6.524449in}}%
\pgfpathlineto{\pgfqpoint{3.845376in}{6.506218in}}%
\pgfpathlineto{\pgfqpoint{3.864072in}{6.487582in}}%
\pgfpathlineto{\pgfqpoint{3.882767in}{6.468542in}}%
\pgfpathlineto{\pgfqpoint{3.901462in}{6.449101in}}%
\pgfpathlineto{\pgfqpoint{3.920157in}{6.429264in}}%
\pgfpathlineto{\pgfqpoint{3.938853in}{6.409033in}}%
\pgfpathlineto{\pgfqpoint{3.957548in}{6.388411in}}%
\pgfpathlineto{\pgfqpoint{3.976243in}{6.367402in}}%
\pgfpathlineto{\pgfqpoint{3.994938in}{6.346008in}}%
\pgfpathlineto{\pgfqpoint{4.013633in}{6.324234in}}%
\pgfpathlineto{\pgfqpoint{4.032329in}{6.302083in}}%
\pgfpathlineto{\pgfqpoint{4.051024in}{6.279558in}}%
\pgfpathlineto{\pgfqpoint{4.069719in}{6.256663in}}%
\pgfpathlineto{\pgfqpoint{4.088414in}{6.233402in}}%
\pgfpathlineto{\pgfqpoint{4.107110in}{6.209778in}}%
\pgfpathlineto{\pgfqpoint{4.125805in}{6.185795in}}%
\pgfpathlineto{\pgfqpoint{4.144500in}{6.161457in}}%
\pgfpathlineto{\pgfqpoint{4.144500in}{6.161131in}}%
\pgfpathlineto{\pgfqpoint{4.144500in}{6.161131in}}%
\pgfpathlineto{\pgfqpoint{4.125805in}{6.161131in}}%
\pgfpathlineto{\pgfqpoint{4.107110in}{6.161131in}}%
\pgfpathlineto{\pgfqpoint{4.088414in}{6.161131in}}%
\pgfpathlineto{\pgfqpoint{4.069719in}{6.161131in}}%
\pgfpathlineto{\pgfqpoint{4.051024in}{6.161131in}}%
\pgfpathlineto{\pgfqpoint{4.032329in}{6.161131in}}%
\pgfpathlineto{\pgfqpoint{4.013633in}{6.161131in}}%
\pgfpathlineto{\pgfqpoint{3.994938in}{6.161131in}}%
\pgfpathlineto{\pgfqpoint{3.976243in}{6.161131in}}%
\pgfpathlineto{\pgfqpoint{3.957548in}{6.161131in}}%
\pgfpathlineto{\pgfqpoint{3.938853in}{6.161131in}}%
\pgfpathlineto{\pgfqpoint{3.920157in}{6.161131in}}%
\pgfpathlineto{\pgfqpoint{3.901462in}{6.161131in}}%
\pgfpathlineto{\pgfqpoint{3.882767in}{6.161131in}}%
\pgfpathlineto{\pgfqpoint{3.864072in}{6.161131in}}%
\pgfpathlineto{\pgfqpoint{3.845376in}{6.161131in}}%
\pgfpathlineto{\pgfqpoint{3.826681in}{6.161131in}}%
\pgfpathlineto{\pgfqpoint{3.807986in}{6.161131in}}%
\pgfpathlineto{\pgfqpoint{3.789291in}{6.161131in}}%
\pgfpathlineto{\pgfqpoint{3.770596in}{6.161131in}}%
\pgfpathlineto{\pgfqpoint{3.751900in}{6.161131in}}%
\pgfpathlineto{\pgfqpoint{3.733205in}{6.161131in}}%
\pgfpathlineto{\pgfqpoint{3.714510in}{6.161131in}}%
\pgfpathlineto{\pgfqpoint{3.695815in}{6.161131in}}%
\pgfpathlineto{\pgfqpoint{3.677119in}{6.161131in}}%
\pgfpathlineto{\pgfqpoint{3.658424in}{6.161131in}}%
\pgfpathlineto{\pgfqpoint{3.639729in}{6.161131in}}%
\pgfpathlineto{\pgfqpoint{3.621034in}{6.161131in}}%
\pgfpathlineto{\pgfqpoint{3.602338in}{6.161131in}}%
\pgfpathlineto{\pgfqpoint{3.583643in}{6.161131in}}%
\pgfpathlineto{\pgfqpoint{3.564948in}{6.161131in}}%
\pgfpathlineto{\pgfqpoint{3.546253in}{6.161131in}}%
\pgfpathlineto{\pgfqpoint{3.527558in}{6.161131in}}%
\pgfpathlineto{\pgfqpoint{3.508862in}{6.161131in}}%
\pgfpathlineto{\pgfqpoint{3.490167in}{6.161131in}}%
\pgfpathlineto{\pgfqpoint{3.471472in}{6.161131in}}%
\pgfpathlineto{\pgfqpoint{3.452777in}{6.161131in}}%
\pgfpathlineto{\pgfqpoint{3.434081in}{6.161131in}}%
\pgfpathlineto{\pgfqpoint{3.415386in}{6.161131in}}%
\pgfpathlineto{\pgfqpoint{3.396691in}{6.161131in}}%
\pgfpathlineto{\pgfqpoint{3.377996in}{6.161131in}}%
\pgfpathlineto{\pgfqpoint{3.359301in}{6.161131in}}%
\pgfpathlineto{\pgfqpoint{3.340605in}{6.161131in}}%
\pgfpathlineto{\pgfqpoint{3.321910in}{6.161131in}}%
\pgfpathlineto{\pgfqpoint{3.303215in}{6.161131in}}%
\pgfpathlineto{\pgfqpoint{3.284520in}{6.161131in}}%
\pgfpathlineto{\pgfqpoint{3.265824in}{6.161131in}}%
\pgfpathlineto{\pgfqpoint{3.247129in}{6.161131in}}%
\pgfpathlineto{\pgfqpoint{3.228434in}{6.161131in}}%
\pgfpathlineto{\pgfqpoint{3.228434in}{6.161131in}}%
\pgfpathclose%
\pgfusepath{stroke,fill}%
}%
\begin{pgfscope}%
\pgfsys@transformshift{0.000000in}{0.000000in}%
\pgfsys@useobject{currentmarker}{}%
\end{pgfscope}%
\end{pgfscope}%
\begin{pgfscope}%
\pgfpathrectangle{\pgfqpoint{0.750000in}{3.960000in}}{\pgfqpoint{4.650000in}{3.080000in}}%
\pgfusepath{clip}%
\pgfsetrectcap%
\pgfsetroundjoin%
\pgfsetlinewidth{0.803000pt}%
\definecolor{currentstroke}{rgb}{0.690196,0.690196,0.690196}%
\pgfsetstrokecolor{currentstroke}%
\pgfsetdash{}{0pt}%
\pgfpathmoveto{\pgfqpoint{0.750000in}{3.960000in}}%
\pgfpathlineto{\pgfqpoint{0.750000in}{7.040000in}}%
\pgfusepath{stroke}%
\end{pgfscope}%
\begin{pgfscope}%
\pgfsetbuttcap%
\pgfsetroundjoin%
\definecolor{currentfill}{rgb}{0.000000,0.000000,0.000000}%
\pgfsetfillcolor{currentfill}%
\pgfsetlinewidth{0.803000pt}%
\definecolor{currentstroke}{rgb}{0.000000,0.000000,0.000000}%
\pgfsetstrokecolor{currentstroke}%
\pgfsetdash{}{0pt}%
\pgfsys@defobject{currentmarker}{\pgfqpoint{0.000000in}{-0.048611in}}{\pgfqpoint{0.000000in}{0.000000in}}{%
\pgfpathmoveto{\pgfqpoint{0.000000in}{0.000000in}}%
\pgfpathlineto{\pgfqpoint{0.000000in}{-0.048611in}}%
\pgfusepath{stroke,fill}%
}%
\begin{pgfscope}%
\pgfsys@transformshift{0.750000in}{3.960000in}%
\pgfsys@useobject{currentmarker}{}%
\end{pgfscope}%
\end{pgfscope}%
\begin{pgfscope}%
\pgfpathrectangle{\pgfqpoint{0.750000in}{3.960000in}}{\pgfqpoint{4.650000in}{3.080000in}}%
\pgfusepath{clip}%
\pgfsetrectcap%
\pgfsetroundjoin%
\pgfsetlinewidth{0.803000pt}%
\definecolor{currentstroke}{rgb}{0.690196,0.690196,0.690196}%
\pgfsetstrokecolor{currentstroke}%
\pgfsetdash{}{0pt}%
\pgfpathmoveto{\pgfqpoint{1.266667in}{3.960000in}}%
\pgfpathlineto{\pgfqpoint{1.266667in}{7.040000in}}%
\pgfusepath{stroke}%
\end{pgfscope}%
\begin{pgfscope}%
\pgfsetbuttcap%
\pgfsetroundjoin%
\definecolor{currentfill}{rgb}{0.000000,0.000000,0.000000}%
\pgfsetfillcolor{currentfill}%
\pgfsetlinewidth{0.803000pt}%
\definecolor{currentstroke}{rgb}{0.000000,0.000000,0.000000}%
\pgfsetstrokecolor{currentstroke}%
\pgfsetdash{}{0pt}%
\pgfsys@defobject{currentmarker}{\pgfqpoint{0.000000in}{-0.048611in}}{\pgfqpoint{0.000000in}{0.000000in}}{%
\pgfpathmoveto{\pgfqpoint{0.000000in}{0.000000in}}%
\pgfpathlineto{\pgfqpoint{0.000000in}{-0.048611in}}%
\pgfusepath{stroke,fill}%
}%
\begin{pgfscope}%
\pgfsys@transformshift{1.266667in}{3.960000in}%
\pgfsys@useobject{currentmarker}{}%
\end{pgfscope}%
\end{pgfscope}%
\begin{pgfscope}%
\pgfpathrectangle{\pgfqpoint{0.750000in}{3.960000in}}{\pgfqpoint{4.650000in}{3.080000in}}%
\pgfusepath{clip}%
\pgfsetrectcap%
\pgfsetroundjoin%
\pgfsetlinewidth{0.803000pt}%
\definecolor{currentstroke}{rgb}{0.690196,0.690196,0.690196}%
\pgfsetstrokecolor{currentstroke}%
\pgfsetdash{}{0pt}%
\pgfpathmoveto{\pgfqpoint{1.783333in}{3.960000in}}%
\pgfpathlineto{\pgfqpoint{1.783333in}{7.040000in}}%
\pgfusepath{stroke}%
\end{pgfscope}%
\begin{pgfscope}%
\pgfsetbuttcap%
\pgfsetroundjoin%
\definecolor{currentfill}{rgb}{0.000000,0.000000,0.000000}%
\pgfsetfillcolor{currentfill}%
\pgfsetlinewidth{0.803000pt}%
\definecolor{currentstroke}{rgb}{0.000000,0.000000,0.000000}%
\pgfsetstrokecolor{currentstroke}%
\pgfsetdash{}{0pt}%
\pgfsys@defobject{currentmarker}{\pgfqpoint{0.000000in}{-0.048611in}}{\pgfqpoint{0.000000in}{0.000000in}}{%
\pgfpathmoveto{\pgfqpoint{0.000000in}{0.000000in}}%
\pgfpathlineto{\pgfqpoint{0.000000in}{-0.048611in}}%
\pgfusepath{stroke,fill}%
}%
\begin{pgfscope}%
\pgfsys@transformshift{1.783333in}{3.960000in}%
\pgfsys@useobject{currentmarker}{}%
\end{pgfscope}%
\end{pgfscope}%
\begin{pgfscope}%
\pgfpathrectangle{\pgfqpoint{0.750000in}{3.960000in}}{\pgfqpoint{4.650000in}{3.080000in}}%
\pgfusepath{clip}%
\pgfsetrectcap%
\pgfsetroundjoin%
\pgfsetlinewidth{0.803000pt}%
\definecolor{currentstroke}{rgb}{0.690196,0.690196,0.690196}%
\pgfsetstrokecolor{currentstroke}%
\pgfsetdash{}{0pt}%
\pgfpathmoveto{\pgfqpoint{2.300000in}{3.960000in}}%
\pgfpathlineto{\pgfqpoint{2.300000in}{7.040000in}}%
\pgfusepath{stroke}%
\end{pgfscope}%
\begin{pgfscope}%
\pgfsetbuttcap%
\pgfsetroundjoin%
\definecolor{currentfill}{rgb}{0.000000,0.000000,0.000000}%
\pgfsetfillcolor{currentfill}%
\pgfsetlinewidth{0.803000pt}%
\definecolor{currentstroke}{rgb}{0.000000,0.000000,0.000000}%
\pgfsetstrokecolor{currentstroke}%
\pgfsetdash{}{0pt}%
\pgfsys@defobject{currentmarker}{\pgfqpoint{0.000000in}{-0.048611in}}{\pgfqpoint{0.000000in}{0.000000in}}{%
\pgfpathmoveto{\pgfqpoint{0.000000in}{0.000000in}}%
\pgfpathlineto{\pgfqpoint{0.000000in}{-0.048611in}}%
\pgfusepath{stroke,fill}%
}%
\begin{pgfscope}%
\pgfsys@transformshift{2.300000in}{3.960000in}%
\pgfsys@useobject{currentmarker}{}%
\end{pgfscope}%
\end{pgfscope}%
\begin{pgfscope}%
\pgfpathrectangle{\pgfqpoint{0.750000in}{3.960000in}}{\pgfqpoint{4.650000in}{3.080000in}}%
\pgfusepath{clip}%
\pgfsetrectcap%
\pgfsetroundjoin%
\pgfsetlinewidth{0.803000pt}%
\definecolor{currentstroke}{rgb}{0.690196,0.690196,0.690196}%
\pgfsetstrokecolor{currentstroke}%
\pgfsetdash{}{0pt}%
\pgfpathmoveto{\pgfqpoint{2.816667in}{3.960000in}}%
\pgfpathlineto{\pgfqpoint{2.816667in}{7.040000in}}%
\pgfusepath{stroke}%
\end{pgfscope}%
\begin{pgfscope}%
\pgfsetbuttcap%
\pgfsetroundjoin%
\definecolor{currentfill}{rgb}{0.000000,0.000000,0.000000}%
\pgfsetfillcolor{currentfill}%
\pgfsetlinewidth{0.803000pt}%
\definecolor{currentstroke}{rgb}{0.000000,0.000000,0.000000}%
\pgfsetstrokecolor{currentstroke}%
\pgfsetdash{}{0pt}%
\pgfsys@defobject{currentmarker}{\pgfqpoint{0.000000in}{-0.048611in}}{\pgfqpoint{0.000000in}{0.000000in}}{%
\pgfpathmoveto{\pgfqpoint{0.000000in}{0.000000in}}%
\pgfpathlineto{\pgfqpoint{0.000000in}{-0.048611in}}%
\pgfusepath{stroke,fill}%
}%
\begin{pgfscope}%
\pgfsys@transformshift{2.816667in}{3.960000in}%
\pgfsys@useobject{currentmarker}{}%
\end{pgfscope}%
\end{pgfscope}%
\begin{pgfscope}%
\pgfpathrectangle{\pgfqpoint{0.750000in}{3.960000in}}{\pgfqpoint{4.650000in}{3.080000in}}%
\pgfusepath{clip}%
\pgfsetrectcap%
\pgfsetroundjoin%
\pgfsetlinewidth{0.803000pt}%
\definecolor{currentstroke}{rgb}{0.690196,0.690196,0.690196}%
\pgfsetstrokecolor{currentstroke}%
\pgfsetdash{}{0pt}%
\pgfpathmoveto{\pgfqpoint{3.333333in}{3.960000in}}%
\pgfpathlineto{\pgfqpoint{3.333333in}{7.040000in}}%
\pgfusepath{stroke}%
\end{pgfscope}%
\begin{pgfscope}%
\pgfsetbuttcap%
\pgfsetroundjoin%
\definecolor{currentfill}{rgb}{0.000000,0.000000,0.000000}%
\pgfsetfillcolor{currentfill}%
\pgfsetlinewidth{0.803000pt}%
\definecolor{currentstroke}{rgb}{0.000000,0.000000,0.000000}%
\pgfsetstrokecolor{currentstroke}%
\pgfsetdash{}{0pt}%
\pgfsys@defobject{currentmarker}{\pgfqpoint{0.000000in}{-0.048611in}}{\pgfqpoint{0.000000in}{0.000000in}}{%
\pgfpathmoveto{\pgfqpoint{0.000000in}{0.000000in}}%
\pgfpathlineto{\pgfqpoint{0.000000in}{-0.048611in}}%
\pgfusepath{stroke,fill}%
}%
\begin{pgfscope}%
\pgfsys@transformshift{3.333333in}{3.960000in}%
\pgfsys@useobject{currentmarker}{}%
\end{pgfscope}%
\end{pgfscope}%
\begin{pgfscope}%
\pgfpathrectangle{\pgfqpoint{0.750000in}{3.960000in}}{\pgfqpoint{4.650000in}{3.080000in}}%
\pgfusepath{clip}%
\pgfsetrectcap%
\pgfsetroundjoin%
\pgfsetlinewidth{0.803000pt}%
\definecolor{currentstroke}{rgb}{0.690196,0.690196,0.690196}%
\pgfsetstrokecolor{currentstroke}%
\pgfsetdash{}{0pt}%
\pgfpathmoveto{\pgfqpoint{3.850000in}{3.960000in}}%
\pgfpathlineto{\pgfqpoint{3.850000in}{7.040000in}}%
\pgfusepath{stroke}%
\end{pgfscope}%
\begin{pgfscope}%
\pgfsetbuttcap%
\pgfsetroundjoin%
\definecolor{currentfill}{rgb}{0.000000,0.000000,0.000000}%
\pgfsetfillcolor{currentfill}%
\pgfsetlinewidth{0.803000pt}%
\definecolor{currentstroke}{rgb}{0.000000,0.000000,0.000000}%
\pgfsetstrokecolor{currentstroke}%
\pgfsetdash{}{0pt}%
\pgfsys@defobject{currentmarker}{\pgfqpoint{0.000000in}{-0.048611in}}{\pgfqpoint{0.000000in}{0.000000in}}{%
\pgfpathmoveto{\pgfqpoint{0.000000in}{0.000000in}}%
\pgfpathlineto{\pgfqpoint{0.000000in}{-0.048611in}}%
\pgfusepath{stroke,fill}%
}%
\begin{pgfscope}%
\pgfsys@transformshift{3.850000in}{3.960000in}%
\pgfsys@useobject{currentmarker}{}%
\end{pgfscope}%
\end{pgfscope}%
\begin{pgfscope}%
\pgfpathrectangle{\pgfqpoint{0.750000in}{3.960000in}}{\pgfqpoint{4.650000in}{3.080000in}}%
\pgfusepath{clip}%
\pgfsetrectcap%
\pgfsetroundjoin%
\pgfsetlinewidth{0.803000pt}%
\definecolor{currentstroke}{rgb}{0.690196,0.690196,0.690196}%
\pgfsetstrokecolor{currentstroke}%
\pgfsetdash{}{0pt}%
\pgfpathmoveto{\pgfqpoint{4.366667in}{3.960000in}}%
\pgfpathlineto{\pgfqpoint{4.366667in}{7.040000in}}%
\pgfusepath{stroke}%
\end{pgfscope}%
\begin{pgfscope}%
\pgfsetbuttcap%
\pgfsetroundjoin%
\definecolor{currentfill}{rgb}{0.000000,0.000000,0.000000}%
\pgfsetfillcolor{currentfill}%
\pgfsetlinewidth{0.803000pt}%
\definecolor{currentstroke}{rgb}{0.000000,0.000000,0.000000}%
\pgfsetstrokecolor{currentstroke}%
\pgfsetdash{}{0pt}%
\pgfsys@defobject{currentmarker}{\pgfqpoint{0.000000in}{-0.048611in}}{\pgfqpoint{0.000000in}{0.000000in}}{%
\pgfpathmoveto{\pgfqpoint{0.000000in}{0.000000in}}%
\pgfpathlineto{\pgfqpoint{0.000000in}{-0.048611in}}%
\pgfusepath{stroke,fill}%
}%
\begin{pgfscope}%
\pgfsys@transformshift{4.366667in}{3.960000in}%
\pgfsys@useobject{currentmarker}{}%
\end{pgfscope}%
\end{pgfscope}%
\begin{pgfscope}%
\pgfpathrectangle{\pgfqpoint{0.750000in}{3.960000in}}{\pgfqpoint{4.650000in}{3.080000in}}%
\pgfusepath{clip}%
\pgfsetrectcap%
\pgfsetroundjoin%
\pgfsetlinewidth{0.803000pt}%
\definecolor{currentstroke}{rgb}{0.690196,0.690196,0.690196}%
\pgfsetstrokecolor{currentstroke}%
\pgfsetdash{}{0pt}%
\pgfpathmoveto{\pgfqpoint{4.883333in}{3.960000in}}%
\pgfpathlineto{\pgfqpoint{4.883333in}{7.040000in}}%
\pgfusepath{stroke}%
\end{pgfscope}%
\begin{pgfscope}%
\pgfsetbuttcap%
\pgfsetroundjoin%
\definecolor{currentfill}{rgb}{0.000000,0.000000,0.000000}%
\pgfsetfillcolor{currentfill}%
\pgfsetlinewidth{0.803000pt}%
\definecolor{currentstroke}{rgb}{0.000000,0.000000,0.000000}%
\pgfsetstrokecolor{currentstroke}%
\pgfsetdash{}{0pt}%
\pgfsys@defobject{currentmarker}{\pgfqpoint{0.000000in}{-0.048611in}}{\pgfqpoint{0.000000in}{0.000000in}}{%
\pgfpathmoveto{\pgfqpoint{0.000000in}{0.000000in}}%
\pgfpathlineto{\pgfqpoint{0.000000in}{-0.048611in}}%
\pgfusepath{stroke,fill}%
}%
\begin{pgfscope}%
\pgfsys@transformshift{4.883333in}{3.960000in}%
\pgfsys@useobject{currentmarker}{}%
\end{pgfscope}%
\end{pgfscope}%
\begin{pgfscope}%
\pgfpathrectangle{\pgfqpoint{0.750000in}{3.960000in}}{\pgfqpoint{4.650000in}{3.080000in}}%
\pgfusepath{clip}%
\pgfsetrectcap%
\pgfsetroundjoin%
\pgfsetlinewidth{0.803000pt}%
\definecolor{currentstroke}{rgb}{0.690196,0.690196,0.690196}%
\pgfsetstrokecolor{currentstroke}%
\pgfsetdash{}{0pt}%
\pgfpathmoveto{\pgfqpoint{5.400000in}{3.960000in}}%
\pgfpathlineto{\pgfqpoint{5.400000in}{7.040000in}}%
\pgfusepath{stroke}%
\end{pgfscope}%
\begin{pgfscope}%
\pgfsetbuttcap%
\pgfsetroundjoin%
\definecolor{currentfill}{rgb}{0.000000,0.000000,0.000000}%
\pgfsetfillcolor{currentfill}%
\pgfsetlinewidth{0.803000pt}%
\definecolor{currentstroke}{rgb}{0.000000,0.000000,0.000000}%
\pgfsetstrokecolor{currentstroke}%
\pgfsetdash{}{0pt}%
\pgfsys@defobject{currentmarker}{\pgfqpoint{0.000000in}{-0.048611in}}{\pgfqpoint{0.000000in}{0.000000in}}{%
\pgfpathmoveto{\pgfqpoint{0.000000in}{0.000000in}}%
\pgfpathlineto{\pgfqpoint{0.000000in}{-0.048611in}}%
\pgfusepath{stroke,fill}%
}%
\begin{pgfscope}%
\pgfsys@transformshift{5.400000in}{3.960000in}%
\pgfsys@useobject{currentmarker}{}%
\end{pgfscope}%
\end{pgfscope}%
\begin{pgfscope}%
\pgfpathrectangle{\pgfqpoint{0.750000in}{3.960000in}}{\pgfqpoint{4.650000in}{3.080000in}}%
\pgfusepath{clip}%
\pgfsetrectcap%
\pgfsetroundjoin%
\pgfsetlinewidth{0.803000pt}%
\definecolor{currentstroke}{rgb}{0.690196,0.690196,0.690196}%
\pgfsetstrokecolor{currentstroke}%
\pgfsetdash{}{0pt}%
\pgfpathmoveto{\pgfqpoint{0.750000in}{3.960000in}}%
\pgfpathlineto{\pgfqpoint{5.400000in}{3.960000in}}%
\pgfusepath{stroke}%
\end{pgfscope}%
\begin{pgfscope}%
\pgfsetbuttcap%
\pgfsetroundjoin%
\definecolor{currentfill}{rgb}{0.000000,0.000000,0.000000}%
\pgfsetfillcolor{currentfill}%
\pgfsetlinewidth{0.803000pt}%
\definecolor{currentstroke}{rgb}{0.000000,0.000000,0.000000}%
\pgfsetstrokecolor{currentstroke}%
\pgfsetdash{}{0pt}%
\pgfsys@defobject{currentmarker}{\pgfqpoint{-0.048611in}{0.000000in}}{\pgfqpoint{-0.000000in}{0.000000in}}{%
\pgfpathmoveto{\pgfqpoint{-0.000000in}{0.000000in}}%
\pgfpathlineto{\pgfqpoint{-0.048611in}{0.000000in}}%
\pgfusepath{stroke,fill}%
}%
\begin{pgfscope}%
\pgfsys@transformshift{0.750000in}{3.960000in}%
\pgfsys@useobject{currentmarker}{}%
\end{pgfscope}%
\end{pgfscope}%
\begin{pgfscope}%
\definecolor{textcolor}{rgb}{0.000000,0.000000,0.000000}%
\pgfsetstrokecolor{textcolor}%
\pgfsetfillcolor{textcolor}%
\pgftext[x=0.475308in, y=3.908900in, left, base]{\color{textcolor}\rmfamily\fontsize{10.000000}{12.000000}\selectfont \(\displaystyle {0.0}\)}%
\end{pgfscope}%
\begin{pgfscope}%
\pgfpathrectangle{\pgfqpoint{0.750000in}{3.960000in}}{\pgfqpoint{4.650000in}{3.080000in}}%
\pgfusepath{clip}%
\pgfsetrectcap%
\pgfsetroundjoin%
\pgfsetlinewidth{0.803000pt}%
\definecolor{currentstroke}{rgb}{0.690196,0.690196,0.690196}%
\pgfsetstrokecolor{currentstroke}%
\pgfsetdash{}{0pt}%
\pgfpathmoveto{\pgfqpoint{0.750000in}{4.449140in}}%
\pgfpathlineto{\pgfqpoint{5.400000in}{4.449140in}}%
\pgfusepath{stroke}%
\end{pgfscope}%
\begin{pgfscope}%
\pgfsetbuttcap%
\pgfsetroundjoin%
\definecolor{currentfill}{rgb}{0.000000,0.000000,0.000000}%
\pgfsetfillcolor{currentfill}%
\pgfsetlinewidth{0.803000pt}%
\definecolor{currentstroke}{rgb}{0.000000,0.000000,0.000000}%
\pgfsetstrokecolor{currentstroke}%
\pgfsetdash{}{0pt}%
\pgfsys@defobject{currentmarker}{\pgfqpoint{-0.048611in}{0.000000in}}{\pgfqpoint{-0.000000in}{0.000000in}}{%
\pgfpathmoveto{\pgfqpoint{-0.000000in}{0.000000in}}%
\pgfpathlineto{\pgfqpoint{-0.048611in}{0.000000in}}%
\pgfusepath{stroke,fill}%
}%
\begin{pgfscope}%
\pgfsys@transformshift{0.750000in}{4.449140in}%
\pgfsys@useobject{currentmarker}{}%
\end{pgfscope}%
\end{pgfscope}%
\begin{pgfscope}%
\definecolor{textcolor}{rgb}{0.000000,0.000000,0.000000}%
\pgfsetstrokecolor{textcolor}%
\pgfsetfillcolor{textcolor}%
\pgftext[x=0.475308in, y=4.398040in, left, base]{\color{textcolor}\rmfamily\fontsize{10.000000}{12.000000}\selectfont \(\displaystyle {0.2}\)}%
\end{pgfscope}%
\begin{pgfscope}%
\pgfpathrectangle{\pgfqpoint{0.750000in}{3.960000in}}{\pgfqpoint{4.650000in}{3.080000in}}%
\pgfusepath{clip}%
\pgfsetrectcap%
\pgfsetroundjoin%
\pgfsetlinewidth{0.803000pt}%
\definecolor{currentstroke}{rgb}{0.690196,0.690196,0.690196}%
\pgfsetstrokecolor{currentstroke}%
\pgfsetdash{}{0pt}%
\pgfpathmoveto{\pgfqpoint{0.750000in}{4.938280in}}%
\pgfpathlineto{\pgfqpoint{5.400000in}{4.938280in}}%
\pgfusepath{stroke}%
\end{pgfscope}%
\begin{pgfscope}%
\pgfsetbuttcap%
\pgfsetroundjoin%
\definecolor{currentfill}{rgb}{0.000000,0.000000,0.000000}%
\pgfsetfillcolor{currentfill}%
\pgfsetlinewidth{0.803000pt}%
\definecolor{currentstroke}{rgb}{0.000000,0.000000,0.000000}%
\pgfsetstrokecolor{currentstroke}%
\pgfsetdash{}{0pt}%
\pgfsys@defobject{currentmarker}{\pgfqpoint{-0.048611in}{0.000000in}}{\pgfqpoint{-0.000000in}{0.000000in}}{%
\pgfpathmoveto{\pgfqpoint{-0.000000in}{0.000000in}}%
\pgfpathlineto{\pgfqpoint{-0.048611in}{0.000000in}}%
\pgfusepath{stroke,fill}%
}%
\begin{pgfscope}%
\pgfsys@transformshift{0.750000in}{4.938280in}%
\pgfsys@useobject{currentmarker}{}%
\end{pgfscope}%
\end{pgfscope}%
\begin{pgfscope}%
\definecolor{textcolor}{rgb}{0.000000,0.000000,0.000000}%
\pgfsetstrokecolor{textcolor}%
\pgfsetfillcolor{textcolor}%
\pgftext[x=0.475308in, y=4.887180in, left, base]{\color{textcolor}\rmfamily\fontsize{10.000000}{12.000000}\selectfont \(\displaystyle {0.4}\)}%
\end{pgfscope}%
\begin{pgfscope}%
\pgfpathrectangle{\pgfqpoint{0.750000in}{3.960000in}}{\pgfqpoint{4.650000in}{3.080000in}}%
\pgfusepath{clip}%
\pgfsetrectcap%
\pgfsetroundjoin%
\pgfsetlinewidth{0.803000pt}%
\definecolor{currentstroke}{rgb}{0.690196,0.690196,0.690196}%
\pgfsetstrokecolor{currentstroke}%
\pgfsetdash{}{0pt}%
\pgfpathmoveto{\pgfqpoint{0.750000in}{5.427421in}}%
\pgfpathlineto{\pgfqpoint{5.400000in}{5.427421in}}%
\pgfusepath{stroke}%
\end{pgfscope}%
\begin{pgfscope}%
\pgfsetbuttcap%
\pgfsetroundjoin%
\definecolor{currentfill}{rgb}{0.000000,0.000000,0.000000}%
\pgfsetfillcolor{currentfill}%
\pgfsetlinewidth{0.803000pt}%
\definecolor{currentstroke}{rgb}{0.000000,0.000000,0.000000}%
\pgfsetstrokecolor{currentstroke}%
\pgfsetdash{}{0pt}%
\pgfsys@defobject{currentmarker}{\pgfqpoint{-0.048611in}{0.000000in}}{\pgfqpoint{-0.000000in}{0.000000in}}{%
\pgfpathmoveto{\pgfqpoint{-0.000000in}{0.000000in}}%
\pgfpathlineto{\pgfqpoint{-0.048611in}{0.000000in}}%
\pgfusepath{stroke,fill}%
}%
\begin{pgfscope}%
\pgfsys@transformshift{0.750000in}{5.427421in}%
\pgfsys@useobject{currentmarker}{}%
\end{pgfscope}%
\end{pgfscope}%
\begin{pgfscope}%
\definecolor{textcolor}{rgb}{0.000000,0.000000,0.000000}%
\pgfsetstrokecolor{textcolor}%
\pgfsetfillcolor{textcolor}%
\pgftext[x=0.475308in, y=5.376321in, left, base]{\color{textcolor}\rmfamily\fontsize{10.000000}{12.000000}\selectfont \(\displaystyle {0.6}\)}%
\end{pgfscope}%
\begin{pgfscope}%
\pgfpathrectangle{\pgfqpoint{0.750000in}{3.960000in}}{\pgfqpoint{4.650000in}{3.080000in}}%
\pgfusepath{clip}%
\pgfsetrectcap%
\pgfsetroundjoin%
\pgfsetlinewidth{0.803000pt}%
\definecolor{currentstroke}{rgb}{0.690196,0.690196,0.690196}%
\pgfsetstrokecolor{currentstroke}%
\pgfsetdash{}{0pt}%
\pgfpathmoveto{\pgfqpoint{0.750000in}{5.916561in}}%
\pgfpathlineto{\pgfqpoint{5.400000in}{5.916561in}}%
\pgfusepath{stroke}%
\end{pgfscope}%
\begin{pgfscope}%
\pgfsetbuttcap%
\pgfsetroundjoin%
\definecolor{currentfill}{rgb}{0.000000,0.000000,0.000000}%
\pgfsetfillcolor{currentfill}%
\pgfsetlinewidth{0.803000pt}%
\definecolor{currentstroke}{rgb}{0.000000,0.000000,0.000000}%
\pgfsetstrokecolor{currentstroke}%
\pgfsetdash{}{0pt}%
\pgfsys@defobject{currentmarker}{\pgfqpoint{-0.048611in}{0.000000in}}{\pgfqpoint{-0.000000in}{0.000000in}}{%
\pgfpathmoveto{\pgfqpoint{-0.000000in}{0.000000in}}%
\pgfpathlineto{\pgfqpoint{-0.048611in}{0.000000in}}%
\pgfusepath{stroke,fill}%
}%
\begin{pgfscope}%
\pgfsys@transformshift{0.750000in}{5.916561in}%
\pgfsys@useobject{currentmarker}{}%
\end{pgfscope}%
\end{pgfscope}%
\begin{pgfscope}%
\definecolor{textcolor}{rgb}{0.000000,0.000000,0.000000}%
\pgfsetstrokecolor{textcolor}%
\pgfsetfillcolor{textcolor}%
\pgftext[x=0.475308in, y=5.865461in, left, base]{\color{textcolor}\rmfamily\fontsize{10.000000}{12.000000}\selectfont \(\displaystyle {0.8}\)}%
\end{pgfscope}%
\begin{pgfscope}%
\pgfpathrectangle{\pgfqpoint{0.750000in}{3.960000in}}{\pgfqpoint{4.650000in}{3.080000in}}%
\pgfusepath{clip}%
\pgfsetrectcap%
\pgfsetroundjoin%
\pgfsetlinewidth{0.803000pt}%
\definecolor{currentstroke}{rgb}{0.690196,0.690196,0.690196}%
\pgfsetstrokecolor{currentstroke}%
\pgfsetdash{}{0pt}%
\pgfpathmoveto{\pgfqpoint{0.750000in}{6.405701in}}%
\pgfpathlineto{\pgfqpoint{5.400000in}{6.405701in}}%
\pgfusepath{stroke}%
\end{pgfscope}%
\begin{pgfscope}%
\pgfsetbuttcap%
\pgfsetroundjoin%
\definecolor{currentfill}{rgb}{0.000000,0.000000,0.000000}%
\pgfsetfillcolor{currentfill}%
\pgfsetlinewidth{0.803000pt}%
\definecolor{currentstroke}{rgb}{0.000000,0.000000,0.000000}%
\pgfsetstrokecolor{currentstroke}%
\pgfsetdash{}{0pt}%
\pgfsys@defobject{currentmarker}{\pgfqpoint{-0.048611in}{0.000000in}}{\pgfqpoint{-0.000000in}{0.000000in}}{%
\pgfpathmoveto{\pgfqpoint{-0.000000in}{0.000000in}}%
\pgfpathlineto{\pgfqpoint{-0.048611in}{0.000000in}}%
\pgfusepath{stroke,fill}%
}%
\begin{pgfscope}%
\pgfsys@transformshift{0.750000in}{6.405701in}%
\pgfsys@useobject{currentmarker}{}%
\end{pgfscope}%
\end{pgfscope}%
\begin{pgfscope}%
\definecolor{textcolor}{rgb}{0.000000,0.000000,0.000000}%
\pgfsetstrokecolor{textcolor}%
\pgfsetfillcolor{textcolor}%
\pgftext[x=0.475308in, y=6.354601in, left, base]{\color{textcolor}\rmfamily\fontsize{10.000000}{12.000000}\selectfont \(\displaystyle {1.0}\)}%
\end{pgfscope}%
\begin{pgfscope}%
\pgfpathrectangle{\pgfqpoint{0.750000in}{3.960000in}}{\pgfqpoint{4.650000in}{3.080000in}}%
\pgfusepath{clip}%
\pgfsetrectcap%
\pgfsetroundjoin%
\pgfsetlinewidth{0.803000pt}%
\definecolor{currentstroke}{rgb}{0.690196,0.690196,0.690196}%
\pgfsetstrokecolor{currentstroke}%
\pgfsetdash{}{0pt}%
\pgfpathmoveto{\pgfqpoint{0.750000in}{6.894841in}}%
\pgfpathlineto{\pgfqpoint{5.400000in}{6.894841in}}%
\pgfusepath{stroke}%
\end{pgfscope}%
\begin{pgfscope}%
\pgfsetbuttcap%
\pgfsetroundjoin%
\definecolor{currentfill}{rgb}{0.000000,0.000000,0.000000}%
\pgfsetfillcolor{currentfill}%
\pgfsetlinewidth{0.803000pt}%
\definecolor{currentstroke}{rgb}{0.000000,0.000000,0.000000}%
\pgfsetstrokecolor{currentstroke}%
\pgfsetdash{}{0pt}%
\pgfsys@defobject{currentmarker}{\pgfqpoint{-0.048611in}{0.000000in}}{\pgfqpoint{-0.000000in}{0.000000in}}{%
\pgfpathmoveto{\pgfqpoint{-0.000000in}{0.000000in}}%
\pgfpathlineto{\pgfqpoint{-0.048611in}{0.000000in}}%
\pgfusepath{stroke,fill}%
}%
\begin{pgfscope}%
\pgfsys@transformshift{0.750000in}{6.894841in}%
\pgfsys@useobject{currentmarker}{}%
\end{pgfscope}%
\end{pgfscope}%
\begin{pgfscope}%
\definecolor{textcolor}{rgb}{0.000000,0.000000,0.000000}%
\pgfsetstrokecolor{textcolor}%
\pgfsetfillcolor{textcolor}%
\pgftext[x=0.475308in, y=6.843741in, left, base]{\color{textcolor}\rmfamily\fontsize{10.000000}{12.000000}\selectfont \(\displaystyle {1.2}\)}%
\end{pgfscope}%
\begin{pgfscope}%
\definecolor{textcolor}{rgb}{0.000000,0.000000,0.000000}%
\pgfsetstrokecolor{textcolor}%
\pgfsetfillcolor{textcolor}%
\pgftext[x=0.419752in,y=5.500000in,,bottom,rotate=90.000000]{\color{textcolor}\rmfamily\fontsize{10.000000}{12.000000}\selectfont power in pu}%
\end{pgfscope}%
\begin{pgfscope}%
\pgfpathrectangle{\pgfqpoint{0.750000in}{3.960000in}}{\pgfqpoint{4.650000in}{3.080000in}}%
\pgfusepath{clip}%
\pgfsetrectcap%
\pgfsetroundjoin%
\pgfsetlinewidth{2.007500pt}%
\definecolor{currentstroke}{rgb}{0.121569,0.466667,0.705882}%
\pgfsetstrokecolor{currentstroke}%
\pgfsetdash{}{0pt}%
\pgfpathmoveto{\pgfqpoint{0.750000in}{3.960000in}}%
\pgfpathlineto{\pgfqpoint{0.844898in}{4.148036in}}%
\pgfpathlineto{\pgfqpoint{0.939796in}{4.335299in}}%
\pgfpathlineto{\pgfqpoint{1.034694in}{4.521020in}}%
\pgfpathlineto{\pgfqpoint{1.129592in}{4.704436in}}%
\pgfpathlineto{\pgfqpoint{1.224490in}{4.884793in}}%
\pgfpathlineto{\pgfqpoint{1.319388in}{5.061349in}}%
\pgfpathlineto{\pgfqpoint{1.414286in}{5.233380in}}%
\pgfpathlineto{\pgfqpoint{1.509184in}{5.400178in}}%
\pgfpathlineto{\pgfqpoint{1.604082in}{5.561058in}}%
\pgfpathlineto{\pgfqpoint{1.698980in}{5.715359in}}%
\pgfpathlineto{\pgfqpoint{1.793878in}{5.862447in}}%
\pgfpathlineto{\pgfqpoint{1.888776in}{6.001718in}}%
\pgfpathlineto{\pgfqpoint{1.983673in}{6.132598in}}%
\pgfpathlineto{\pgfqpoint{2.078571in}{6.254551in}}%
\pgfpathlineto{\pgfqpoint{2.173469in}{6.367075in}}%
\pgfpathlineto{\pgfqpoint{2.268367in}{6.469708in}}%
\pgfpathlineto{\pgfqpoint{2.363265in}{6.562028in}}%
\pgfpathlineto{\pgfqpoint{2.458163in}{6.643656in}}%
\pgfpathlineto{\pgfqpoint{2.553061in}{6.714256in}}%
\pgfpathlineto{\pgfqpoint{2.647959in}{6.773538in}}%
\pgfpathlineto{\pgfqpoint{2.742857in}{6.821259in}}%
\pgfpathlineto{\pgfqpoint{2.837755in}{6.857222in}}%
\pgfpathlineto{\pgfqpoint{2.932653in}{6.881280in}}%
\pgfpathlineto{\pgfqpoint{3.027551in}{6.893333in}}%
\pgfpathlineto{\pgfqpoint{3.122449in}{6.893333in}}%
\pgfpathlineto{\pgfqpoint{3.217347in}{6.881280in}}%
\pgfpathlineto{\pgfqpoint{3.312245in}{6.857222in}}%
\pgfpathlineto{\pgfqpoint{3.407143in}{6.821259in}}%
\pgfpathlineto{\pgfqpoint{3.502041in}{6.773538in}}%
\pgfpathlineto{\pgfqpoint{3.596939in}{6.714256in}}%
\pgfpathlineto{\pgfqpoint{3.691837in}{6.643656in}}%
\pgfpathlineto{\pgfqpoint{3.786735in}{6.562028in}}%
\pgfpathlineto{\pgfqpoint{3.881633in}{6.469708in}}%
\pgfpathlineto{\pgfqpoint{3.976531in}{6.367075in}}%
\pgfpathlineto{\pgfqpoint{4.071429in}{6.254551in}}%
\pgfpathlineto{\pgfqpoint{4.166327in}{6.132598in}}%
\pgfpathlineto{\pgfqpoint{4.261224in}{6.001718in}}%
\pgfpathlineto{\pgfqpoint{4.356122in}{5.862447in}}%
\pgfpathlineto{\pgfqpoint{4.451020in}{5.715359in}}%
\pgfpathlineto{\pgfqpoint{4.545918in}{5.561058in}}%
\pgfpathlineto{\pgfqpoint{4.640816in}{5.400178in}}%
\pgfpathlineto{\pgfqpoint{4.735714in}{5.233380in}}%
\pgfpathlineto{\pgfqpoint{4.830612in}{5.061349in}}%
\pgfpathlineto{\pgfqpoint{4.925510in}{4.884793in}}%
\pgfpathlineto{\pgfqpoint{5.020408in}{4.704436in}}%
\pgfpathlineto{\pgfqpoint{5.115306in}{4.521020in}}%
\pgfpathlineto{\pgfqpoint{5.210204in}{4.335299in}}%
\pgfpathlineto{\pgfqpoint{5.305102in}{4.148036in}}%
\pgfpathlineto{\pgfqpoint{5.400000in}{3.960000in}}%
\pgfusepath{stroke}%
\end{pgfscope}%
\begin{pgfscope}%
\pgfpathrectangle{\pgfqpoint{0.750000in}{3.960000in}}{\pgfqpoint{4.650000in}{3.080000in}}%
\pgfusepath{clip}%
\pgfsetrectcap%
\pgfsetroundjoin%
\pgfsetlinewidth{2.007500pt}%
\definecolor{currentstroke}{rgb}{1.000000,0.498039,0.054902}%
\pgfsetstrokecolor{currentstroke}%
\pgfsetdash{}{0pt}%
\pgfpathmoveto{\pgfqpoint{0.750000in}{3.960000in}}%
\pgfpathlineto{\pgfqpoint{0.844898in}{4.086553in}}%
\pgfpathlineto{\pgfqpoint{0.939796in}{4.212586in}}%
\pgfpathlineto{\pgfqpoint{1.034694in}{4.337580in}}%
\pgfpathlineto{\pgfqpoint{1.129592in}{4.461024in}}%
\pgfpathlineto{\pgfqpoint{1.224490in}{4.582408in}}%
\pgfpathlineto{\pgfqpoint{1.319388in}{4.701235in}}%
\pgfpathlineto{\pgfqpoint{1.414286in}{4.817016in}}%
\pgfpathlineto{\pgfqpoint{1.509184in}{4.929275in}}%
\pgfpathlineto{\pgfqpoint{1.604082in}{5.037552in}}%
\pgfpathlineto{\pgfqpoint{1.698980in}{5.141400in}}%
\pgfpathlineto{\pgfqpoint{1.793878in}{5.240394in}}%
\pgfpathlineto{\pgfqpoint{1.888776in}{5.334126in}}%
\pgfpathlineto{\pgfqpoint{1.983673in}{5.422212in}}%
\pgfpathlineto{\pgfqpoint{2.078571in}{5.504289in}}%
\pgfpathlineto{\pgfqpoint{2.173469in}{5.580021in}}%
\pgfpathlineto{\pgfqpoint{2.268367in}{5.649095in}}%
\pgfpathlineto{\pgfqpoint{2.363265in}{5.711229in}}%
\pgfpathlineto{\pgfqpoint{2.458163in}{5.766166in}}%
\pgfpathlineto{\pgfqpoint{2.553061in}{5.813682in}}%
\pgfpathlineto{\pgfqpoint{2.647959in}{5.853580in}}%
\pgfpathlineto{\pgfqpoint{2.742857in}{5.885697in}}%
\pgfpathlineto{\pgfqpoint{2.837755in}{5.909901in}}%
\pgfpathlineto{\pgfqpoint{2.932653in}{5.926093in}}%
\pgfpathlineto{\pgfqpoint{3.027551in}{5.934205in}}%
\pgfpathlineto{\pgfqpoint{3.122449in}{5.934205in}}%
\pgfpathlineto{\pgfqpoint{3.217347in}{5.926093in}}%
\pgfpathlineto{\pgfqpoint{3.312245in}{5.909901in}}%
\pgfpathlineto{\pgfqpoint{3.407143in}{5.885697in}}%
\pgfpathlineto{\pgfqpoint{3.502041in}{5.853580in}}%
\pgfpathlineto{\pgfqpoint{3.596939in}{5.813682in}}%
\pgfpathlineto{\pgfqpoint{3.691837in}{5.766166in}}%
\pgfpathlineto{\pgfqpoint{3.786735in}{5.711229in}}%
\pgfpathlineto{\pgfqpoint{3.881633in}{5.649095in}}%
\pgfpathlineto{\pgfqpoint{3.976531in}{5.580021in}}%
\pgfpathlineto{\pgfqpoint{4.071429in}{5.504289in}}%
\pgfpathlineto{\pgfqpoint{4.166327in}{5.422212in}}%
\pgfpathlineto{\pgfqpoint{4.261224in}{5.334126in}}%
\pgfpathlineto{\pgfqpoint{4.356122in}{5.240394in}}%
\pgfpathlineto{\pgfqpoint{4.451020in}{5.141400in}}%
\pgfpathlineto{\pgfqpoint{4.545918in}{5.037552in}}%
\pgfpathlineto{\pgfqpoint{4.640816in}{4.929275in}}%
\pgfpathlineto{\pgfqpoint{4.735714in}{4.817016in}}%
\pgfpathlineto{\pgfqpoint{4.830612in}{4.701235in}}%
\pgfpathlineto{\pgfqpoint{4.925510in}{4.582408in}}%
\pgfpathlineto{\pgfqpoint{5.020408in}{4.461024in}}%
\pgfpathlineto{\pgfqpoint{5.115306in}{4.337580in}}%
\pgfpathlineto{\pgfqpoint{5.210204in}{4.212586in}}%
\pgfpathlineto{\pgfqpoint{5.305102in}{4.086553in}}%
\pgfpathlineto{\pgfqpoint{5.400000in}{3.960000in}}%
\pgfusepath{stroke}%
\end{pgfscope}%
\begin{pgfscope}%
\pgfpathrectangle{\pgfqpoint{0.750000in}{3.960000in}}{\pgfqpoint{4.650000in}{3.080000in}}%
\pgfusepath{clip}%
\pgfsetrectcap%
\pgfsetroundjoin%
\pgfsetlinewidth{2.007500pt}%
\definecolor{currentstroke}{rgb}{0.172549,0.627451,0.172549}%
\pgfsetstrokecolor{currentstroke}%
\pgfsetdash{}{0pt}%
\pgfpathmoveto{\pgfqpoint{0.750000in}{6.161131in}}%
\pgfpathlineto{\pgfqpoint{0.844898in}{6.161131in}}%
\pgfpathlineto{\pgfqpoint{0.939796in}{6.161131in}}%
\pgfpathlineto{\pgfqpoint{1.034694in}{6.161131in}}%
\pgfpathlineto{\pgfqpoint{1.129592in}{6.161131in}}%
\pgfpathlineto{\pgfqpoint{1.224490in}{6.161131in}}%
\pgfpathlineto{\pgfqpoint{1.319388in}{6.161131in}}%
\pgfpathlineto{\pgfqpoint{1.414286in}{6.161131in}}%
\pgfpathlineto{\pgfqpoint{1.509184in}{6.161131in}}%
\pgfpathlineto{\pgfqpoint{1.604082in}{6.161131in}}%
\pgfpathlineto{\pgfqpoint{1.698980in}{6.161131in}}%
\pgfpathlineto{\pgfqpoint{1.793878in}{6.161131in}}%
\pgfpathlineto{\pgfqpoint{1.888776in}{6.161131in}}%
\pgfpathlineto{\pgfqpoint{1.983673in}{6.161131in}}%
\pgfpathlineto{\pgfqpoint{2.078571in}{6.161131in}}%
\pgfpathlineto{\pgfqpoint{2.173469in}{6.161131in}}%
\pgfpathlineto{\pgfqpoint{2.268367in}{6.161131in}}%
\pgfpathlineto{\pgfqpoint{2.363265in}{6.161131in}}%
\pgfpathlineto{\pgfqpoint{2.458163in}{6.161131in}}%
\pgfpathlineto{\pgfqpoint{2.553061in}{6.161131in}}%
\pgfpathlineto{\pgfqpoint{2.647959in}{6.161131in}}%
\pgfpathlineto{\pgfqpoint{2.742857in}{6.161131in}}%
\pgfpathlineto{\pgfqpoint{2.837755in}{6.161131in}}%
\pgfpathlineto{\pgfqpoint{2.932653in}{6.161131in}}%
\pgfpathlineto{\pgfqpoint{3.027551in}{6.161131in}}%
\pgfpathlineto{\pgfqpoint{3.122449in}{6.161131in}}%
\pgfpathlineto{\pgfqpoint{3.217347in}{6.161131in}}%
\pgfpathlineto{\pgfqpoint{3.312245in}{6.161131in}}%
\pgfpathlineto{\pgfqpoint{3.407143in}{6.161131in}}%
\pgfpathlineto{\pgfqpoint{3.502041in}{6.161131in}}%
\pgfpathlineto{\pgfqpoint{3.596939in}{6.161131in}}%
\pgfpathlineto{\pgfqpoint{3.691837in}{6.161131in}}%
\pgfpathlineto{\pgfqpoint{3.786735in}{6.161131in}}%
\pgfpathlineto{\pgfqpoint{3.881633in}{6.161131in}}%
\pgfpathlineto{\pgfqpoint{3.976531in}{6.161131in}}%
\pgfpathlineto{\pgfqpoint{4.071429in}{6.161131in}}%
\pgfpathlineto{\pgfqpoint{4.166327in}{6.161131in}}%
\pgfpathlineto{\pgfqpoint{4.261224in}{6.161131in}}%
\pgfpathlineto{\pgfqpoint{4.356122in}{6.161131in}}%
\pgfpathlineto{\pgfqpoint{4.451020in}{6.161131in}}%
\pgfpathlineto{\pgfqpoint{4.545918in}{6.161131in}}%
\pgfpathlineto{\pgfqpoint{4.640816in}{6.161131in}}%
\pgfpathlineto{\pgfqpoint{4.735714in}{6.161131in}}%
\pgfpathlineto{\pgfqpoint{4.830612in}{6.161131in}}%
\pgfpathlineto{\pgfqpoint{4.925510in}{6.161131in}}%
\pgfpathlineto{\pgfqpoint{5.020408in}{6.161131in}}%
\pgfpathlineto{\pgfqpoint{5.115306in}{6.161131in}}%
\pgfpathlineto{\pgfqpoint{5.210204in}{6.161131in}}%
\pgfpathlineto{\pgfqpoint{5.305102in}{6.161131in}}%
\pgfpathlineto{\pgfqpoint{5.400000in}{6.161131in}}%
\pgfusepath{stroke}%
\end{pgfscope}%
\begin{pgfscope}%
\pgfsetrectcap%
\pgfsetmiterjoin%
\pgfsetlinewidth{0.803000pt}%
\definecolor{currentstroke}{rgb}{0.000000,0.000000,0.000000}%
\pgfsetstrokecolor{currentstroke}%
\pgfsetdash{}{0pt}%
\pgfpathmoveto{\pgfqpoint{0.750000in}{3.960000in}}%
\pgfpathlineto{\pgfqpoint{0.750000in}{7.040000in}}%
\pgfusepath{stroke}%
\end{pgfscope}%
\begin{pgfscope}%
\pgfsetrectcap%
\pgfsetmiterjoin%
\pgfsetlinewidth{0.803000pt}%
\definecolor{currentstroke}{rgb}{0.000000,0.000000,0.000000}%
\pgfsetstrokecolor{currentstroke}%
\pgfsetdash{}{0pt}%
\pgfpathmoveto{\pgfqpoint{5.400000in}{3.960000in}}%
\pgfpathlineto{\pgfqpoint{5.400000in}{7.040000in}}%
\pgfusepath{stroke}%
\end{pgfscope}%
\begin{pgfscope}%
\pgfsetrectcap%
\pgfsetmiterjoin%
\pgfsetlinewidth{0.803000pt}%
\definecolor{currentstroke}{rgb}{0.000000,0.000000,0.000000}%
\pgfsetstrokecolor{currentstroke}%
\pgfsetdash{}{0pt}%
\pgfpathmoveto{\pgfqpoint{0.750000in}{3.960000in}}%
\pgfpathlineto{\pgfqpoint{5.400000in}{3.960000in}}%
\pgfusepath{stroke}%
\end{pgfscope}%
\begin{pgfscope}%
\pgfsetrectcap%
\pgfsetmiterjoin%
\pgfsetlinewidth{0.803000pt}%
\definecolor{currentstroke}{rgb}{0.000000,0.000000,0.000000}%
\pgfsetstrokecolor{currentstroke}%
\pgfsetdash{}{0pt}%
\pgfpathmoveto{\pgfqpoint{0.750000in}{7.040000in}}%
\pgfpathlineto{\pgfqpoint{5.400000in}{7.040000in}}%
\pgfusepath{stroke}%
\end{pgfscope}%
\begin{pgfscope}%
\pgfsetbuttcap%
\pgfsetmiterjoin%
\definecolor{currentfill}{rgb}{1.000000,1.000000,1.000000}%
\pgfsetfillcolor{currentfill}%
\pgfsetfillopacity{0.800000}%
\pgfsetlinewidth{1.003750pt}%
\definecolor{currentstroke}{rgb}{0.800000,0.800000,0.800000}%
\pgfsetstrokecolor{currentstroke}%
\pgfsetstrokeopacity{0.800000}%
\pgfsetdash{}{0pt}%
\pgfpathmoveto{\pgfqpoint{2.342004in}{4.029444in}}%
\pgfpathlineto{\pgfqpoint{3.807996in}{4.029444in}}%
\pgfpathquadraticcurveto{\pgfqpoint{3.835774in}{4.029444in}}{\pgfqpoint{3.835774in}{4.057222in}}%
\pgfpathlineto{\pgfqpoint{3.835774in}{4.651311in}}%
\pgfpathquadraticcurveto{\pgfqpoint{3.835774in}{4.679088in}}{\pgfqpoint{3.807996in}{4.679088in}}%
\pgfpathlineto{\pgfqpoint{2.342004in}{4.679088in}}%
\pgfpathquadraticcurveto{\pgfqpoint{2.314226in}{4.679088in}}{\pgfqpoint{2.314226in}{4.651311in}}%
\pgfpathlineto{\pgfqpoint{2.314226in}{4.057222in}}%
\pgfpathquadraticcurveto{\pgfqpoint{2.314226in}{4.029444in}}{\pgfqpoint{2.342004in}{4.029444in}}%
\pgfpathlineto{\pgfqpoint{2.342004in}{4.029444in}}%
\pgfpathclose%
\pgfusepath{stroke,fill}%
\end{pgfscope}%
\begin{pgfscope}%
\pgfsetrectcap%
\pgfsetroundjoin%
\pgfsetlinewidth{2.007500pt}%
\definecolor{currentstroke}{rgb}{0.121569,0.466667,0.705882}%
\pgfsetstrokecolor{currentstroke}%
\pgfsetdash{}{0pt}%
\pgfpathmoveto{\pgfqpoint{2.369781in}{4.568791in}}%
\pgfpathlineto{\pgfqpoint{2.508670in}{4.568791in}}%
\pgfpathlineto{\pgfqpoint{2.647559in}{4.568791in}}%
\pgfusepath{stroke}%
\end{pgfscope}%
\begin{pgfscope}%
\definecolor{textcolor}{rgb}{0.000000,0.000000,0.000000}%
\pgfsetstrokecolor{textcolor}%
\pgfsetfillcolor{textcolor}%
\pgftext[x=2.758670in,y=4.520180in,left,base]{\color{textcolor}\rmfamily\fontsize{10.000000}{12.000000}\selectfont \(\displaystyle P_\mathrm{e}\) pre-fault}%
\end{pgfscope}%
\begin{pgfscope}%
\pgfsetrectcap%
\pgfsetroundjoin%
\pgfsetlinewidth{2.007500pt}%
\definecolor{currentstroke}{rgb}{1.000000,0.498039,0.054902}%
\pgfsetstrokecolor{currentstroke}%
\pgfsetdash{}{0pt}%
\pgfpathmoveto{\pgfqpoint{2.369781in}{4.365748in}}%
\pgfpathlineto{\pgfqpoint{2.508670in}{4.365748in}}%
\pgfpathlineto{\pgfqpoint{2.647559in}{4.365748in}}%
\pgfusepath{stroke}%
\end{pgfscope}%
\begin{pgfscope}%
\definecolor{textcolor}{rgb}{0.000000,0.000000,0.000000}%
\pgfsetstrokecolor{textcolor}%
\pgfsetfillcolor{textcolor}%
\pgftext[x=2.758670in,y=4.317137in,left,base]{\color{textcolor}\rmfamily\fontsize{10.000000}{12.000000}\selectfont \(\displaystyle P_\mathrm{e}\) post-fault}%
\end{pgfscope}%
\begin{pgfscope}%
\pgfsetrectcap%
\pgfsetroundjoin%
\pgfsetlinewidth{2.007500pt}%
\definecolor{currentstroke}{rgb}{0.172549,0.627451,0.172549}%
\pgfsetstrokecolor{currentstroke}%
\pgfsetdash{}{0pt}%
\pgfpathmoveto{\pgfqpoint{2.369781in}{4.163857in}}%
\pgfpathlineto{\pgfqpoint{2.508670in}{4.163857in}}%
\pgfpathlineto{\pgfqpoint{2.647559in}{4.163857in}}%
\pgfusepath{stroke}%
\end{pgfscope}%
\begin{pgfscope}%
\definecolor{textcolor}{rgb}{0.000000,0.000000,0.000000}%
\pgfsetstrokecolor{textcolor}%
\pgfsetfillcolor{textcolor}%
\pgftext[x=2.758670in,y=4.115246in,left,base]{\color{textcolor}\rmfamily\fontsize{10.000000}{12.000000}\selectfont \(\displaystyle P_\mathrm{T}\) of the turbine}%
\end{pgfscope}%
\begin{pgfscope}%
\pgfsetbuttcap%
\pgfsetmiterjoin%
\definecolor{currentfill}{rgb}{1.000000,1.000000,1.000000}%
\pgfsetfillcolor{currentfill}%
\pgfsetlinewidth{0.000000pt}%
\definecolor{currentstroke}{rgb}{0.000000,0.000000,0.000000}%
\pgfsetstrokecolor{currentstroke}%
\pgfsetstrokeopacity{0.000000}%
\pgfsetdash{}{0pt}%
\pgfpathmoveto{\pgfqpoint{0.750000in}{0.880000in}}%
\pgfpathlineto{\pgfqpoint{5.400000in}{0.880000in}}%
\pgfpathlineto{\pgfqpoint{5.400000in}{3.960000in}}%
\pgfpathlineto{\pgfqpoint{0.750000in}{3.960000in}}%
\pgfpathlineto{\pgfqpoint{0.750000in}{0.880000in}}%
\pgfpathclose%
\pgfusepath{fill}%
\end{pgfscope}%
\begin{pgfscope}%
\pgfpathrectangle{\pgfqpoint{0.750000in}{0.880000in}}{\pgfqpoint{4.650000in}{3.080000in}}%
\pgfusepath{clip}%
\pgfsetrectcap%
\pgfsetroundjoin%
\pgfsetlinewidth{0.803000pt}%
\definecolor{currentstroke}{rgb}{0.690196,0.690196,0.690196}%
\pgfsetstrokecolor{currentstroke}%
\pgfsetdash{}{0pt}%
\pgfpathmoveto{\pgfqpoint{0.750000in}{0.880000in}}%
\pgfpathlineto{\pgfqpoint{0.750000in}{3.960000in}}%
\pgfusepath{stroke}%
\end{pgfscope}%
\begin{pgfscope}%
\pgfsetbuttcap%
\pgfsetroundjoin%
\definecolor{currentfill}{rgb}{0.000000,0.000000,0.000000}%
\pgfsetfillcolor{currentfill}%
\pgfsetlinewidth{0.803000pt}%
\definecolor{currentstroke}{rgb}{0.000000,0.000000,0.000000}%
\pgfsetstrokecolor{currentstroke}%
\pgfsetdash{}{0pt}%
\pgfsys@defobject{currentmarker}{\pgfqpoint{0.000000in}{-0.048611in}}{\pgfqpoint{0.000000in}{0.000000in}}{%
\pgfpathmoveto{\pgfqpoint{0.000000in}{0.000000in}}%
\pgfpathlineto{\pgfqpoint{0.000000in}{-0.048611in}}%
\pgfusepath{stroke,fill}%
}%
\begin{pgfscope}%
\pgfsys@transformshift{0.750000in}{0.880000in}%
\pgfsys@useobject{currentmarker}{}%
\end{pgfscope}%
\end{pgfscope}%
\begin{pgfscope}%
\definecolor{textcolor}{rgb}{0.000000,0.000000,0.000000}%
\pgfsetstrokecolor{textcolor}%
\pgfsetfillcolor{textcolor}%
\pgftext[x=0.750000in,y=0.782778in,,top]{\color{textcolor}\rmfamily\fontsize{10.000000}{12.000000}\selectfont \(\displaystyle {0}\)}%
\end{pgfscope}%
\begin{pgfscope}%
\pgfpathrectangle{\pgfqpoint{0.750000in}{0.880000in}}{\pgfqpoint{4.650000in}{3.080000in}}%
\pgfusepath{clip}%
\pgfsetrectcap%
\pgfsetroundjoin%
\pgfsetlinewidth{0.803000pt}%
\definecolor{currentstroke}{rgb}{0.690196,0.690196,0.690196}%
\pgfsetstrokecolor{currentstroke}%
\pgfsetdash{}{0pt}%
\pgfpathmoveto{\pgfqpoint{1.266667in}{0.880000in}}%
\pgfpathlineto{\pgfqpoint{1.266667in}{3.960000in}}%
\pgfusepath{stroke}%
\end{pgfscope}%
\begin{pgfscope}%
\pgfsetbuttcap%
\pgfsetroundjoin%
\definecolor{currentfill}{rgb}{0.000000,0.000000,0.000000}%
\pgfsetfillcolor{currentfill}%
\pgfsetlinewidth{0.803000pt}%
\definecolor{currentstroke}{rgb}{0.000000,0.000000,0.000000}%
\pgfsetstrokecolor{currentstroke}%
\pgfsetdash{}{0pt}%
\pgfsys@defobject{currentmarker}{\pgfqpoint{0.000000in}{-0.048611in}}{\pgfqpoint{0.000000in}{0.000000in}}{%
\pgfpathmoveto{\pgfqpoint{0.000000in}{0.000000in}}%
\pgfpathlineto{\pgfqpoint{0.000000in}{-0.048611in}}%
\pgfusepath{stroke,fill}%
}%
\begin{pgfscope}%
\pgfsys@transformshift{1.266667in}{0.880000in}%
\pgfsys@useobject{currentmarker}{}%
\end{pgfscope}%
\end{pgfscope}%
\begin{pgfscope}%
\definecolor{textcolor}{rgb}{0.000000,0.000000,0.000000}%
\pgfsetstrokecolor{textcolor}%
\pgfsetfillcolor{textcolor}%
\pgftext[x=1.266667in,y=0.782778in,,top]{\color{textcolor}\rmfamily\fontsize{10.000000}{12.000000}\selectfont \(\displaystyle {20}\)}%
\end{pgfscope}%
\begin{pgfscope}%
\pgfpathrectangle{\pgfqpoint{0.750000in}{0.880000in}}{\pgfqpoint{4.650000in}{3.080000in}}%
\pgfusepath{clip}%
\pgfsetrectcap%
\pgfsetroundjoin%
\pgfsetlinewidth{0.803000pt}%
\definecolor{currentstroke}{rgb}{0.690196,0.690196,0.690196}%
\pgfsetstrokecolor{currentstroke}%
\pgfsetdash{}{0pt}%
\pgfpathmoveto{\pgfqpoint{1.783333in}{0.880000in}}%
\pgfpathlineto{\pgfqpoint{1.783333in}{3.960000in}}%
\pgfusepath{stroke}%
\end{pgfscope}%
\begin{pgfscope}%
\pgfsetbuttcap%
\pgfsetroundjoin%
\definecolor{currentfill}{rgb}{0.000000,0.000000,0.000000}%
\pgfsetfillcolor{currentfill}%
\pgfsetlinewidth{0.803000pt}%
\definecolor{currentstroke}{rgb}{0.000000,0.000000,0.000000}%
\pgfsetstrokecolor{currentstroke}%
\pgfsetdash{}{0pt}%
\pgfsys@defobject{currentmarker}{\pgfqpoint{0.000000in}{-0.048611in}}{\pgfqpoint{0.000000in}{0.000000in}}{%
\pgfpathmoveto{\pgfqpoint{0.000000in}{0.000000in}}%
\pgfpathlineto{\pgfqpoint{0.000000in}{-0.048611in}}%
\pgfusepath{stroke,fill}%
}%
\begin{pgfscope}%
\pgfsys@transformshift{1.783333in}{0.880000in}%
\pgfsys@useobject{currentmarker}{}%
\end{pgfscope}%
\end{pgfscope}%
\begin{pgfscope}%
\definecolor{textcolor}{rgb}{0.000000,0.000000,0.000000}%
\pgfsetstrokecolor{textcolor}%
\pgfsetfillcolor{textcolor}%
\pgftext[x=1.783333in,y=0.782778in,,top]{\color{textcolor}\rmfamily\fontsize{10.000000}{12.000000}\selectfont \(\displaystyle {40}\)}%
\end{pgfscope}%
\begin{pgfscope}%
\pgfpathrectangle{\pgfqpoint{0.750000in}{0.880000in}}{\pgfqpoint{4.650000in}{3.080000in}}%
\pgfusepath{clip}%
\pgfsetrectcap%
\pgfsetroundjoin%
\pgfsetlinewidth{0.803000pt}%
\definecolor{currentstroke}{rgb}{0.690196,0.690196,0.690196}%
\pgfsetstrokecolor{currentstroke}%
\pgfsetdash{}{0pt}%
\pgfpathmoveto{\pgfqpoint{2.300000in}{0.880000in}}%
\pgfpathlineto{\pgfqpoint{2.300000in}{3.960000in}}%
\pgfusepath{stroke}%
\end{pgfscope}%
\begin{pgfscope}%
\pgfsetbuttcap%
\pgfsetroundjoin%
\definecolor{currentfill}{rgb}{0.000000,0.000000,0.000000}%
\pgfsetfillcolor{currentfill}%
\pgfsetlinewidth{0.803000pt}%
\definecolor{currentstroke}{rgb}{0.000000,0.000000,0.000000}%
\pgfsetstrokecolor{currentstroke}%
\pgfsetdash{}{0pt}%
\pgfsys@defobject{currentmarker}{\pgfqpoint{0.000000in}{-0.048611in}}{\pgfqpoint{0.000000in}{0.000000in}}{%
\pgfpathmoveto{\pgfqpoint{0.000000in}{0.000000in}}%
\pgfpathlineto{\pgfqpoint{0.000000in}{-0.048611in}}%
\pgfusepath{stroke,fill}%
}%
\begin{pgfscope}%
\pgfsys@transformshift{2.300000in}{0.880000in}%
\pgfsys@useobject{currentmarker}{}%
\end{pgfscope}%
\end{pgfscope}%
\begin{pgfscope}%
\definecolor{textcolor}{rgb}{0.000000,0.000000,0.000000}%
\pgfsetstrokecolor{textcolor}%
\pgfsetfillcolor{textcolor}%
\pgftext[x=2.300000in,y=0.782778in,,top]{\color{textcolor}\rmfamily\fontsize{10.000000}{12.000000}\selectfont \(\displaystyle {60}\)}%
\end{pgfscope}%
\begin{pgfscope}%
\pgfpathrectangle{\pgfqpoint{0.750000in}{0.880000in}}{\pgfqpoint{4.650000in}{3.080000in}}%
\pgfusepath{clip}%
\pgfsetrectcap%
\pgfsetroundjoin%
\pgfsetlinewidth{0.803000pt}%
\definecolor{currentstroke}{rgb}{0.690196,0.690196,0.690196}%
\pgfsetstrokecolor{currentstroke}%
\pgfsetdash{}{0pt}%
\pgfpathmoveto{\pgfqpoint{2.816667in}{0.880000in}}%
\pgfpathlineto{\pgfqpoint{2.816667in}{3.960000in}}%
\pgfusepath{stroke}%
\end{pgfscope}%
\begin{pgfscope}%
\pgfsetbuttcap%
\pgfsetroundjoin%
\definecolor{currentfill}{rgb}{0.000000,0.000000,0.000000}%
\pgfsetfillcolor{currentfill}%
\pgfsetlinewidth{0.803000pt}%
\definecolor{currentstroke}{rgb}{0.000000,0.000000,0.000000}%
\pgfsetstrokecolor{currentstroke}%
\pgfsetdash{}{0pt}%
\pgfsys@defobject{currentmarker}{\pgfqpoint{0.000000in}{-0.048611in}}{\pgfqpoint{0.000000in}{0.000000in}}{%
\pgfpathmoveto{\pgfqpoint{0.000000in}{0.000000in}}%
\pgfpathlineto{\pgfqpoint{0.000000in}{-0.048611in}}%
\pgfusepath{stroke,fill}%
}%
\begin{pgfscope}%
\pgfsys@transformshift{2.816667in}{0.880000in}%
\pgfsys@useobject{currentmarker}{}%
\end{pgfscope}%
\end{pgfscope}%
\begin{pgfscope}%
\definecolor{textcolor}{rgb}{0.000000,0.000000,0.000000}%
\pgfsetstrokecolor{textcolor}%
\pgfsetfillcolor{textcolor}%
\pgftext[x=2.816667in,y=0.782778in,,top]{\color{textcolor}\rmfamily\fontsize{10.000000}{12.000000}\selectfont \(\displaystyle {80}\)}%
\end{pgfscope}%
\begin{pgfscope}%
\pgfpathrectangle{\pgfqpoint{0.750000in}{0.880000in}}{\pgfqpoint{4.650000in}{3.080000in}}%
\pgfusepath{clip}%
\pgfsetrectcap%
\pgfsetroundjoin%
\pgfsetlinewidth{0.803000pt}%
\definecolor{currentstroke}{rgb}{0.690196,0.690196,0.690196}%
\pgfsetstrokecolor{currentstroke}%
\pgfsetdash{}{0pt}%
\pgfpathmoveto{\pgfqpoint{3.333333in}{0.880000in}}%
\pgfpathlineto{\pgfqpoint{3.333333in}{3.960000in}}%
\pgfusepath{stroke}%
\end{pgfscope}%
\begin{pgfscope}%
\pgfsetbuttcap%
\pgfsetroundjoin%
\definecolor{currentfill}{rgb}{0.000000,0.000000,0.000000}%
\pgfsetfillcolor{currentfill}%
\pgfsetlinewidth{0.803000pt}%
\definecolor{currentstroke}{rgb}{0.000000,0.000000,0.000000}%
\pgfsetstrokecolor{currentstroke}%
\pgfsetdash{}{0pt}%
\pgfsys@defobject{currentmarker}{\pgfqpoint{0.000000in}{-0.048611in}}{\pgfqpoint{0.000000in}{0.000000in}}{%
\pgfpathmoveto{\pgfqpoint{0.000000in}{0.000000in}}%
\pgfpathlineto{\pgfqpoint{0.000000in}{-0.048611in}}%
\pgfusepath{stroke,fill}%
}%
\begin{pgfscope}%
\pgfsys@transformshift{3.333333in}{0.880000in}%
\pgfsys@useobject{currentmarker}{}%
\end{pgfscope}%
\end{pgfscope}%
\begin{pgfscope}%
\definecolor{textcolor}{rgb}{0.000000,0.000000,0.000000}%
\pgfsetstrokecolor{textcolor}%
\pgfsetfillcolor{textcolor}%
\pgftext[x=3.333333in,y=0.782778in,,top]{\color{textcolor}\rmfamily\fontsize{10.000000}{12.000000}\selectfont \(\displaystyle {100}\)}%
\end{pgfscope}%
\begin{pgfscope}%
\pgfpathrectangle{\pgfqpoint{0.750000in}{0.880000in}}{\pgfqpoint{4.650000in}{3.080000in}}%
\pgfusepath{clip}%
\pgfsetrectcap%
\pgfsetroundjoin%
\pgfsetlinewidth{0.803000pt}%
\definecolor{currentstroke}{rgb}{0.690196,0.690196,0.690196}%
\pgfsetstrokecolor{currentstroke}%
\pgfsetdash{}{0pt}%
\pgfpathmoveto{\pgfqpoint{3.850000in}{0.880000in}}%
\pgfpathlineto{\pgfqpoint{3.850000in}{3.960000in}}%
\pgfusepath{stroke}%
\end{pgfscope}%
\begin{pgfscope}%
\pgfsetbuttcap%
\pgfsetroundjoin%
\definecolor{currentfill}{rgb}{0.000000,0.000000,0.000000}%
\pgfsetfillcolor{currentfill}%
\pgfsetlinewidth{0.803000pt}%
\definecolor{currentstroke}{rgb}{0.000000,0.000000,0.000000}%
\pgfsetstrokecolor{currentstroke}%
\pgfsetdash{}{0pt}%
\pgfsys@defobject{currentmarker}{\pgfqpoint{0.000000in}{-0.048611in}}{\pgfqpoint{0.000000in}{0.000000in}}{%
\pgfpathmoveto{\pgfqpoint{0.000000in}{0.000000in}}%
\pgfpathlineto{\pgfqpoint{0.000000in}{-0.048611in}}%
\pgfusepath{stroke,fill}%
}%
\begin{pgfscope}%
\pgfsys@transformshift{3.850000in}{0.880000in}%
\pgfsys@useobject{currentmarker}{}%
\end{pgfscope}%
\end{pgfscope}%
\begin{pgfscope}%
\definecolor{textcolor}{rgb}{0.000000,0.000000,0.000000}%
\pgfsetstrokecolor{textcolor}%
\pgfsetfillcolor{textcolor}%
\pgftext[x=3.850000in,y=0.782778in,,top]{\color{textcolor}\rmfamily\fontsize{10.000000}{12.000000}\selectfont \(\displaystyle {120}\)}%
\end{pgfscope}%
\begin{pgfscope}%
\pgfpathrectangle{\pgfqpoint{0.750000in}{0.880000in}}{\pgfqpoint{4.650000in}{3.080000in}}%
\pgfusepath{clip}%
\pgfsetrectcap%
\pgfsetroundjoin%
\pgfsetlinewidth{0.803000pt}%
\definecolor{currentstroke}{rgb}{0.690196,0.690196,0.690196}%
\pgfsetstrokecolor{currentstroke}%
\pgfsetdash{}{0pt}%
\pgfpathmoveto{\pgfqpoint{4.366667in}{0.880000in}}%
\pgfpathlineto{\pgfqpoint{4.366667in}{3.960000in}}%
\pgfusepath{stroke}%
\end{pgfscope}%
\begin{pgfscope}%
\pgfsetbuttcap%
\pgfsetroundjoin%
\definecolor{currentfill}{rgb}{0.000000,0.000000,0.000000}%
\pgfsetfillcolor{currentfill}%
\pgfsetlinewidth{0.803000pt}%
\definecolor{currentstroke}{rgb}{0.000000,0.000000,0.000000}%
\pgfsetstrokecolor{currentstroke}%
\pgfsetdash{}{0pt}%
\pgfsys@defobject{currentmarker}{\pgfqpoint{0.000000in}{-0.048611in}}{\pgfqpoint{0.000000in}{0.000000in}}{%
\pgfpathmoveto{\pgfqpoint{0.000000in}{0.000000in}}%
\pgfpathlineto{\pgfqpoint{0.000000in}{-0.048611in}}%
\pgfusepath{stroke,fill}%
}%
\begin{pgfscope}%
\pgfsys@transformshift{4.366667in}{0.880000in}%
\pgfsys@useobject{currentmarker}{}%
\end{pgfscope}%
\end{pgfscope}%
\begin{pgfscope}%
\definecolor{textcolor}{rgb}{0.000000,0.000000,0.000000}%
\pgfsetstrokecolor{textcolor}%
\pgfsetfillcolor{textcolor}%
\pgftext[x=4.366667in,y=0.782778in,,top]{\color{textcolor}\rmfamily\fontsize{10.000000}{12.000000}\selectfont \(\displaystyle {140}\)}%
\end{pgfscope}%
\begin{pgfscope}%
\pgfpathrectangle{\pgfqpoint{0.750000in}{0.880000in}}{\pgfqpoint{4.650000in}{3.080000in}}%
\pgfusepath{clip}%
\pgfsetrectcap%
\pgfsetroundjoin%
\pgfsetlinewidth{0.803000pt}%
\definecolor{currentstroke}{rgb}{0.690196,0.690196,0.690196}%
\pgfsetstrokecolor{currentstroke}%
\pgfsetdash{}{0pt}%
\pgfpathmoveto{\pgfqpoint{4.883333in}{0.880000in}}%
\pgfpathlineto{\pgfqpoint{4.883333in}{3.960000in}}%
\pgfusepath{stroke}%
\end{pgfscope}%
\begin{pgfscope}%
\pgfsetbuttcap%
\pgfsetroundjoin%
\definecolor{currentfill}{rgb}{0.000000,0.000000,0.000000}%
\pgfsetfillcolor{currentfill}%
\pgfsetlinewidth{0.803000pt}%
\definecolor{currentstroke}{rgb}{0.000000,0.000000,0.000000}%
\pgfsetstrokecolor{currentstroke}%
\pgfsetdash{}{0pt}%
\pgfsys@defobject{currentmarker}{\pgfqpoint{0.000000in}{-0.048611in}}{\pgfqpoint{0.000000in}{0.000000in}}{%
\pgfpathmoveto{\pgfqpoint{0.000000in}{0.000000in}}%
\pgfpathlineto{\pgfqpoint{0.000000in}{-0.048611in}}%
\pgfusepath{stroke,fill}%
}%
\begin{pgfscope}%
\pgfsys@transformshift{4.883333in}{0.880000in}%
\pgfsys@useobject{currentmarker}{}%
\end{pgfscope}%
\end{pgfscope}%
\begin{pgfscope}%
\definecolor{textcolor}{rgb}{0.000000,0.000000,0.000000}%
\pgfsetstrokecolor{textcolor}%
\pgfsetfillcolor{textcolor}%
\pgftext[x=4.883333in,y=0.782778in,,top]{\color{textcolor}\rmfamily\fontsize{10.000000}{12.000000}\selectfont \(\displaystyle {160}\)}%
\end{pgfscope}%
\begin{pgfscope}%
\pgfpathrectangle{\pgfqpoint{0.750000in}{0.880000in}}{\pgfqpoint{4.650000in}{3.080000in}}%
\pgfusepath{clip}%
\pgfsetrectcap%
\pgfsetroundjoin%
\pgfsetlinewidth{0.803000pt}%
\definecolor{currentstroke}{rgb}{0.690196,0.690196,0.690196}%
\pgfsetstrokecolor{currentstroke}%
\pgfsetdash{}{0pt}%
\pgfpathmoveto{\pgfqpoint{5.400000in}{0.880000in}}%
\pgfpathlineto{\pgfqpoint{5.400000in}{3.960000in}}%
\pgfusepath{stroke}%
\end{pgfscope}%
\begin{pgfscope}%
\pgfsetbuttcap%
\pgfsetroundjoin%
\definecolor{currentfill}{rgb}{0.000000,0.000000,0.000000}%
\pgfsetfillcolor{currentfill}%
\pgfsetlinewidth{0.803000pt}%
\definecolor{currentstroke}{rgb}{0.000000,0.000000,0.000000}%
\pgfsetstrokecolor{currentstroke}%
\pgfsetdash{}{0pt}%
\pgfsys@defobject{currentmarker}{\pgfqpoint{0.000000in}{-0.048611in}}{\pgfqpoint{0.000000in}{0.000000in}}{%
\pgfpathmoveto{\pgfqpoint{0.000000in}{0.000000in}}%
\pgfpathlineto{\pgfqpoint{0.000000in}{-0.048611in}}%
\pgfusepath{stroke,fill}%
}%
\begin{pgfscope}%
\pgfsys@transformshift{5.400000in}{0.880000in}%
\pgfsys@useobject{currentmarker}{}%
\end{pgfscope}%
\end{pgfscope}%
\begin{pgfscope}%
\definecolor{textcolor}{rgb}{0.000000,0.000000,0.000000}%
\pgfsetstrokecolor{textcolor}%
\pgfsetfillcolor{textcolor}%
\pgftext[x=5.400000in,y=0.782778in,,top]{\color{textcolor}\rmfamily\fontsize{10.000000}{12.000000}\selectfont \(\displaystyle {180}\)}%
\end{pgfscope}%
\begin{pgfscope}%
\definecolor{textcolor}{rgb}{0.000000,0.000000,0.000000}%
\pgfsetstrokecolor{textcolor}%
\pgfsetfillcolor{textcolor}%
\pgftext[x=3.075000in,y=0.594776in,,top]{\color{textcolor}\rmfamily\fontsize{10.000000}{12.000000}\selectfont power angle \(\displaystyle \delta\) in deg}%
\end{pgfscope}%
\begin{pgfscope}%
\pgfpathrectangle{\pgfqpoint{0.750000in}{0.880000in}}{\pgfqpoint{4.650000in}{3.080000in}}%
\pgfusepath{clip}%
\pgfsetrectcap%
\pgfsetroundjoin%
\pgfsetlinewidth{0.803000pt}%
\definecolor{currentstroke}{rgb}{0.690196,0.690196,0.690196}%
\pgfsetstrokecolor{currentstroke}%
\pgfsetdash{}{0pt}%
\pgfpathmoveto{\pgfqpoint{0.750000in}{3.823047in}}%
\pgfpathlineto{\pgfqpoint{5.400000in}{3.823047in}}%
\pgfusepath{stroke}%
\end{pgfscope}%
\begin{pgfscope}%
\pgfsetbuttcap%
\pgfsetroundjoin%
\definecolor{currentfill}{rgb}{0.000000,0.000000,0.000000}%
\pgfsetfillcolor{currentfill}%
\pgfsetlinewidth{0.803000pt}%
\definecolor{currentstroke}{rgb}{0.000000,0.000000,0.000000}%
\pgfsetstrokecolor{currentstroke}%
\pgfsetdash{}{0pt}%
\pgfsys@defobject{currentmarker}{\pgfqpoint{-0.048611in}{0.000000in}}{\pgfqpoint{-0.000000in}{0.000000in}}{%
\pgfpathmoveto{\pgfqpoint{-0.000000in}{0.000000in}}%
\pgfpathlineto{\pgfqpoint{-0.048611in}{0.000000in}}%
\pgfusepath{stroke,fill}%
}%
\begin{pgfscope}%
\pgfsys@transformshift{0.750000in}{3.823047in}%
\pgfsys@useobject{currentmarker}{}%
\end{pgfscope}%
\end{pgfscope}%
\begin{pgfscope}%
\definecolor{textcolor}{rgb}{0.000000,0.000000,0.000000}%
\pgfsetstrokecolor{textcolor}%
\pgfsetfillcolor{textcolor}%
\pgftext[x=0.405863in, y=3.771947in, left, base]{\color{textcolor}\rmfamily\fontsize{10.000000}{12.000000}\selectfont \(\displaystyle {0.00}\)}%
\end{pgfscope}%
\begin{pgfscope}%
\pgfpathrectangle{\pgfqpoint{0.750000in}{0.880000in}}{\pgfqpoint{4.650000in}{3.080000in}}%
\pgfusepath{clip}%
\pgfsetrectcap%
\pgfsetroundjoin%
\pgfsetlinewidth{0.803000pt}%
\definecolor{currentstroke}{rgb}{0.690196,0.690196,0.690196}%
\pgfsetstrokecolor{currentstroke}%
\pgfsetdash{}{0pt}%
\pgfpathmoveto{\pgfqpoint{0.750000in}{3.480665in}}%
\pgfpathlineto{\pgfqpoint{5.400000in}{3.480665in}}%
\pgfusepath{stroke}%
\end{pgfscope}%
\begin{pgfscope}%
\pgfsetbuttcap%
\pgfsetroundjoin%
\definecolor{currentfill}{rgb}{0.000000,0.000000,0.000000}%
\pgfsetfillcolor{currentfill}%
\pgfsetlinewidth{0.803000pt}%
\definecolor{currentstroke}{rgb}{0.000000,0.000000,0.000000}%
\pgfsetstrokecolor{currentstroke}%
\pgfsetdash{}{0pt}%
\pgfsys@defobject{currentmarker}{\pgfqpoint{-0.048611in}{0.000000in}}{\pgfqpoint{-0.000000in}{0.000000in}}{%
\pgfpathmoveto{\pgfqpoint{-0.000000in}{0.000000in}}%
\pgfpathlineto{\pgfqpoint{-0.048611in}{0.000000in}}%
\pgfusepath{stroke,fill}%
}%
\begin{pgfscope}%
\pgfsys@transformshift{0.750000in}{3.480665in}%
\pgfsys@useobject{currentmarker}{}%
\end{pgfscope}%
\end{pgfscope}%
\begin{pgfscope}%
\definecolor{textcolor}{rgb}{0.000000,0.000000,0.000000}%
\pgfsetstrokecolor{textcolor}%
\pgfsetfillcolor{textcolor}%
\pgftext[x=0.405863in, y=3.429565in, left, base]{\color{textcolor}\rmfamily\fontsize{10.000000}{12.000000}\selectfont \(\displaystyle {0.25}\)}%
\end{pgfscope}%
\begin{pgfscope}%
\pgfpathrectangle{\pgfqpoint{0.750000in}{0.880000in}}{\pgfqpoint{4.650000in}{3.080000in}}%
\pgfusepath{clip}%
\pgfsetrectcap%
\pgfsetroundjoin%
\pgfsetlinewidth{0.803000pt}%
\definecolor{currentstroke}{rgb}{0.690196,0.690196,0.690196}%
\pgfsetstrokecolor{currentstroke}%
\pgfsetdash{}{0pt}%
\pgfpathmoveto{\pgfqpoint{0.750000in}{3.138283in}}%
\pgfpathlineto{\pgfqpoint{5.400000in}{3.138283in}}%
\pgfusepath{stroke}%
\end{pgfscope}%
\begin{pgfscope}%
\pgfsetbuttcap%
\pgfsetroundjoin%
\definecolor{currentfill}{rgb}{0.000000,0.000000,0.000000}%
\pgfsetfillcolor{currentfill}%
\pgfsetlinewidth{0.803000pt}%
\definecolor{currentstroke}{rgb}{0.000000,0.000000,0.000000}%
\pgfsetstrokecolor{currentstroke}%
\pgfsetdash{}{0pt}%
\pgfsys@defobject{currentmarker}{\pgfqpoint{-0.048611in}{0.000000in}}{\pgfqpoint{-0.000000in}{0.000000in}}{%
\pgfpathmoveto{\pgfqpoint{-0.000000in}{0.000000in}}%
\pgfpathlineto{\pgfqpoint{-0.048611in}{0.000000in}}%
\pgfusepath{stroke,fill}%
}%
\begin{pgfscope}%
\pgfsys@transformshift{0.750000in}{3.138283in}%
\pgfsys@useobject{currentmarker}{}%
\end{pgfscope}%
\end{pgfscope}%
\begin{pgfscope}%
\definecolor{textcolor}{rgb}{0.000000,0.000000,0.000000}%
\pgfsetstrokecolor{textcolor}%
\pgfsetfillcolor{textcolor}%
\pgftext[x=0.405863in, y=3.087183in, left, base]{\color{textcolor}\rmfamily\fontsize{10.000000}{12.000000}\selectfont \(\displaystyle {0.50}\)}%
\end{pgfscope}%
\begin{pgfscope}%
\pgfpathrectangle{\pgfqpoint{0.750000in}{0.880000in}}{\pgfqpoint{4.650000in}{3.080000in}}%
\pgfusepath{clip}%
\pgfsetrectcap%
\pgfsetroundjoin%
\pgfsetlinewidth{0.803000pt}%
\definecolor{currentstroke}{rgb}{0.690196,0.690196,0.690196}%
\pgfsetstrokecolor{currentstroke}%
\pgfsetdash{}{0pt}%
\pgfpathmoveto{\pgfqpoint{0.750000in}{2.795901in}}%
\pgfpathlineto{\pgfqpoint{5.400000in}{2.795901in}}%
\pgfusepath{stroke}%
\end{pgfscope}%
\begin{pgfscope}%
\pgfsetbuttcap%
\pgfsetroundjoin%
\definecolor{currentfill}{rgb}{0.000000,0.000000,0.000000}%
\pgfsetfillcolor{currentfill}%
\pgfsetlinewidth{0.803000pt}%
\definecolor{currentstroke}{rgb}{0.000000,0.000000,0.000000}%
\pgfsetstrokecolor{currentstroke}%
\pgfsetdash{}{0pt}%
\pgfsys@defobject{currentmarker}{\pgfqpoint{-0.048611in}{0.000000in}}{\pgfqpoint{-0.000000in}{0.000000in}}{%
\pgfpathmoveto{\pgfqpoint{-0.000000in}{0.000000in}}%
\pgfpathlineto{\pgfqpoint{-0.048611in}{0.000000in}}%
\pgfusepath{stroke,fill}%
}%
\begin{pgfscope}%
\pgfsys@transformshift{0.750000in}{2.795901in}%
\pgfsys@useobject{currentmarker}{}%
\end{pgfscope}%
\end{pgfscope}%
\begin{pgfscope}%
\definecolor{textcolor}{rgb}{0.000000,0.000000,0.000000}%
\pgfsetstrokecolor{textcolor}%
\pgfsetfillcolor{textcolor}%
\pgftext[x=0.405863in, y=2.744801in, left, base]{\color{textcolor}\rmfamily\fontsize{10.000000}{12.000000}\selectfont \(\displaystyle {0.75}\)}%
\end{pgfscope}%
\begin{pgfscope}%
\pgfpathrectangle{\pgfqpoint{0.750000in}{0.880000in}}{\pgfqpoint{4.650000in}{3.080000in}}%
\pgfusepath{clip}%
\pgfsetrectcap%
\pgfsetroundjoin%
\pgfsetlinewidth{0.803000pt}%
\definecolor{currentstroke}{rgb}{0.690196,0.690196,0.690196}%
\pgfsetstrokecolor{currentstroke}%
\pgfsetdash{}{0pt}%
\pgfpathmoveto{\pgfqpoint{0.750000in}{2.453519in}}%
\pgfpathlineto{\pgfqpoint{5.400000in}{2.453519in}}%
\pgfusepath{stroke}%
\end{pgfscope}%
\begin{pgfscope}%
\pgfsetbuttcap%
\pgfsetroundjoin%
\definecolor{currentfill}{rgb}{0.000000,0.000000,0.000000}%
\pgfsetfillcolor{currentfill}%
\pgfsetlinewidth{0.803000pt}%
\definecolor{currentstroke}{rgb}{0.000000,0.000000,0.000000}%
\pgfsetstrokecolor{currentstroke}%
\pgfsetdash{}{0pt}%
\pgfsys@defobject{currentmarker}{\pgfqpoint{-0.048611in}{0.000000in}}{\pgfqpoint{-0.000000in}{0.000000in}}{%
\pgfpathmoveto{\pgfqpoint{-0.000000in}{0.000000in}}%
\pgfpathlineto{\pgfqpoint{-0.048611in}{0.000000in}}%
\pgfusepath{stroke,fill}%
}%
\begin{pgfscope}%
\pgfsys@transformshift{0.750000in}{2.453519in}%
\pgfsys@useobject{currentmarker}{}%
\end{pgfscope}%
\end{pgfscope}%
\begin{pgfscope}%
\definecolor{textcolor}{rgb}{0.000000,0.000000,0.000000}%
\pgfsetstrokecolor{textcolor}%
\pgfsetfillcolor{textcolor}%
\pgftext[x=0.405863in, y=2.402419in, left, base]{\color{textcolor}\rmfamily\fontsize{10.000000}{12.000000}\selectfont \(\displaystyle {1.00}\)}%
\end{pgfscope}%
\begin{pgfscope}%
\pgfpathrectangle{\pgfqpoint{0.750000in}{0.880000in}}{\pgfqpoint{4.650000in}{3.080000in}}%
\pgfusepath{clip}%
\pgfsetrectcap%
\pgfsetroundjoin%
\pgfsetlinewidth{0.803000pt}%
\definecolor{currentstroke}{rgb}{0.690196,0.690196,0.690196}%
\pgfsetstrokecolor{currentstroke}%
\pgfsetdash{}{0pt}%
\pgfpathmoveto{\pgfqpoint{0.750000in}{2.111137in}}%
\pgfpathlineto{\pgfqpoint{5.400000in}{2.111137in}}%
\pgfusepath{stroke}%
\end{pgfscope}%
\begin{pgfscope}%
\pgfsetbuttcap%
\pgfsetroundjoin%
\definecolor{currentfill}{rgb}{0.000000,0.000000,0.000000}%
\pgfsetfillcolor{currentfill}%
\pgfsetlinewidth{0.803000pt}%
\definecolor{currentstroke}{rgb}{0.000000,0.000000,0.000000}%
\pgfsetstrokecolor{currentstroke}%
\pgfsetdash{}{0pt}%
\pgfsys@defobject{currentmarker}{\pgfqpoint{-0.048611in}{0.000000in}}{\pgfqpoint{-0.000000in}{0.000000in}}{%
\pgfpathmoveto{\pgfqpoint{-0.000000in}{0.000000in}}%
\pgfpathlineto{\pgfqpoint{-0.048611in}{0.000000in}}%
\pgfusepath{stroke,fill}%
}%
\begin{pgfscope}%
\pgfsys@transformshift{0.750000in}{2.111137in}%
\pgfsys@useobject{currentmarker}{}%
\end{pgfscope}%
\end{pgfscope}%
\begin{pgfscope}%
\definecolor{textcolor}{rgb}{0.000000,0.000000,0.000000}%
\pgfsetstrokecolor{textcolor}%
\pgfsetfillcolor{textcolor}%
\pgftext[x=0.405863in, y=2.060037in, left, base]{\color{textcolor}\rmfamily\fontsize{10.000000}{12.000000}\selectfont \(\displaystyle {1.25}\)}%
\end{pgfscope}%
\begin{pgfscope}%
\pgfpathrectangle{\pgfqpoint{0.750000in}{0.880000in}}{\pgfqpoint{4.650000in}{3.080000in}}%
\pgfusepath{clip}%
\pgfsetrectcap%
\pgfsetroundjoin%
\pgfsetlinewidth{0.803000pt}%
\definecolor{currentstroke}{rgb}{0.690196,0.690196,0.690196}%
\pgfsetstrokecolor{currentstroke}%
\pgfsetdash{}{0pt}%
\pgfpathmoveto{\pgfqpoint{0.750000in}{1.768755in}}%
\pgfpathlineto{\pgfqpoint{5.400000in}{1.768755in}}%
\pgfusepath{stroke}%
\end{pgfscope}%
\begin{pgfscope}%
\pgfsetbuttcap%
\pgfsetroundjoin%
\definecolor{currentfill}{rgb}{0.000000,0.000000,0.000000}%
\pgfsetfillcolor{currentfill}%
\pgfsetlinewidth{0.803000pt}%
\definecolor{currentstroke}{rgb}{0.000000,0.000000,0.000000}%
\pgfsetstrokecolor{currentstroke}%
\pgfsetdash{}{0pt}%
\pgfsys@defobject{currentmarker}{\pgfqpoint{-0.048611in}{0.000000in}}{\pgfqpoint{-0.000000in}{0.000000in}}{%
\pgfpathmoveto{\pgfqpoint{-0.000000in}{0.000000in}}%
\pgfpathlineto{\pgfqpoint{-0.048611in}{0.000000in}}%
\pgfusepath{stroke,fill}%
}%
\begin{pgfscope}%
\pgfsys@transformshift{0.750000in}{1.768755in}%
\pgfsys@useobject{currentmarker}{}%
\end{pgfscope}%
\end{pgfscope}%
\begin{pgfscope}%
\definecolor{textcolor}{rgb}{0.000000,0.000000,0.000000}%
\pgfsetstrokecolor{textcolor}%
\pgfsetfillcolor{textcolor}%
\pgftext[x=0.405863in, y=1.717655in, left, base]{\color{textcolor}\rmfamily\fontsize{10.000000}{12.000000}\selectfont \(\displaystyle {1.50}\)}%
\end{pgfscope}%
\begin{pgfscope}%
\pgfpathrectangle{\pgfqpoint{0.750000in}{0.880000in}}{\pgfqpoint{4.650000in}{3.080000in}}%
\pgfusepath{clip}%
\pgfsetrectcap%
\pgfsetroundjoin%
\pgfsetlinewidth{0.803000pt}%
\definecolor{currentstroke}{rgb}{0.690196,0.690196,0.690196}%
\pgfsetstrokecolor{currentstroke}%
\pgfsetdash{}{0pt}%
\pgfpathmoveto{\pgfqpoint{0.750000in}{1.426373in}}%
\pgfpathlineto{\pgfqpoint{5.400000in}{1.426373in}}%
\pgfusepath{stroke}%
\end{pgfscope}%
\begin{pgfscope}%
\pgfsetbuttcap%
\pgfsetroundjoin%
\definecolor{currentfill}{rgb}{0.000000,0.000000,0.000000}%
\pgfsetfillcolor{currentfill}%
\pgfsetlinewidth{0.803000pt}%
\definecolor{currentstroke}{rgb}{0.000000,0.000000,0.000000}%
\pgfsetstrokecolor{currentstroke}%
\pgfsetdash{}{0pt}%
\pgfsys@defobject{currentmarker}{\pgfqpoint{-0.048611in}{0.000000in}}{\pgfqpoint{-0.000000in}{0.000000in}}{%
\pgfpathmoveto{\pgfqpoint{-0.000000in}{0.000000in}}%
\pgfpathlineto{\pgfqpoint{-0.048611in}{0.000000in}}%
\pgfusepath{stroke,fill}%
}%
\begin{pgfscope}%
\pgfsys@transformshift{0.750000in}{1.426373in}%
\pgfsys@useobject{currentmarker}{}%
\end{pgfscope}%
\end{pgfscope}%
\begin{pgfscope}%
\definecolor{textcolor}{rgb}{0.000000,0.000000,0.000000}%
\pgfsetstrokecolor{textcolor}%
\pgfsetfillcolor{textcolor}%
\pgftext[x=0.405863in, y=1.375273in, left, base]{\color{textcolor}\rmfamily\fontsize{10.000000}{12.000000}\selectfont \(\displaystyle {1.75}\)}%
\end{pgfscope}%
\begin{pgfscope}%
\pgfpathrectangle{\pgfqpoint{0.750000in}{0.880000in}}{\pgfqpoint{4.650000in}{3.080000in}}%
\pgfusepath{clip}%
\pgfsetrectcap%
\pgfsetroundjoin%
\pgfsetlinewidth{0.803000pt}%
\definecolor{currentstroke}{rgb}{0.690196,0.690196,0.690196}%
\pgfsetstrokecolor{currentstroke}%
\pgfsetdash{}{0pt}%
\pgfpathmoveto{\pgfqpoint{0.750000in}{1.083991in}}%
\pgfpathlineto{\pgfqpoint{5.400000in}{1.083991in}}%
\pgfusepath{stroke}%
\end{pgfscope}%
\begin{pgfscope}%
\pgfsetbuttcap%
\pgfsetroundjoin%
\definecolor{currentfill}{rgb}{0.000000,0.000000,0.000000}%
\pgfsetfillcolor{currentfill}%
\pgfsetlinewidth{0.803000pt}%
\definecolor{currentstroke}{rgb}{0.000000,0.000000,0.000000}%
\pgfsetstrokecolor{currentstroke}%
\pgfsetdash{}{0pt}%
\pgfsys@defobject{currentmarker}{\pgfqpoint{-0.048611in}{0.000000in}}{\pgfqpoint{-0.000000in}{0.000000in}}{%
\pgfpathmoveto{\pgfqpoint{-0.000000in}{0.000000in}}%
\pgfpathlineto{\pgfqpoint{-0.048611in}{0.000000in}}%
\pgfusepath{stroke,fill}%
}%
\begin{pgfscope}%
\pgfsys@transformshift{0.750000in}{1.083991in}%
\pgfsys@useobject{currentmarker}{}%
\end{pgfscope}%
\end{pgfscope}%
\begin{pgfscope}%
\definecolor{textcolor}{rgb}{0.000000,0.000000,0.000000}%
\pgfsetstrokecolor{textcolor}%
\pgfsetfillcolor{textcolor}%
\pgftext[x=0.405863in, y=1.032891in, left, base]{\color{textcolor}\rmfamily\fontsize{10.000000}{12.000000}\selectfont \(\displaystyle {2.00}\)}%
\end{pgfscope}%
\begin{pgfscope}%
\definecolor{textcolor}{rgb}{0.000000,0.000000,0.000000}%
\pgfsetstrokecolor{textcolor}%
\pgfsetfillcolor{textcolor}%
\pgftext[x=0.350308in,y=2.420000in,,bottom,rotate=90.000000]{\color{textcolor}\rmfamily\fontsize{10.000000}{12.000000}\selectfont time in s}%
\end{pgfscope}%
\begin{pgfscope}%
\pgfpathrectangle{\pgfqpoint{0.750000in}{0.880000in}}{\pgfqpoint{4.650000in}{3.080000in}}%
\pgfusepath{clip}%
\pgfsetrectcap%
\pgfsetroundjoin%
\pgfsetlinewidth{1.505625pt}%
\definecolor{currentstroke}{rgb}{0.121569,0.466667,0.705882}%
\pgfsetstrokecolor{currentstroke}%
\pgfsetdash{}{0pt}%
\pgfpathmoveto{\pgfqpoint{2.005414in}{3.970000in}}%
\pgfpathlineto{\pgfqpoint{2.006519in}{3.809352in}}%
\pgfpathlineto{\pgfqpoint{2.009623in}{3.795657in}}%
\pgfpathlineto{\pgfqpoint{2.015412in}{3.780592in}}%
\pgfpathlineto{\pgfqpoint{2.023668in}{3.765527in}}%
\pgfpathlineto{\pgfqpoint{2.034361in}{3.750462in}}%
\pgfpathlineto{\pgfqpoint{2.048764in}{3.734028in}}%
\pgfpathlineto{\pgfqpoint{2.067528in}{3.716224in}}%
\pgfpathlineto{\pgfqpoint{2.089498in}{3.698420in}}%
\pgfpathlineto{\pgfqpoint{2.116634in}{3.679247in}}%
\pgfpathlineto{\pgfqpoint{2.149563in}{3.658704in}}%
\pgfpathlineto{\pgfqpoint{2.188887in}{3.636791in}}%
\pgfpathlineto{\pgfqpoint{2.235164in}{3.613509in}}%
\pgfpathlineto{\pgfqpoint{2.292014in}{3.587488in}}%
\pgfpathlineto{\pgfqpoint{2.357254in}{3.560098in}}%
\pgfpathlineto{\pgfqpoint{2.434824in}{3.529968in}}%
\pgfpathlineto{\pgfqpoint{2.525639in}{3.497100in}}%
\pgfpathlineto{\pgfqpoint{2.634561in}{3.460122in}}%
\pgfpathlineto{\pgfqpoint{2.762787in}{3.419036in}}%
\pgfpathlineto{\pgfqpoint{2.911296in}{3.373842in}}%
\pgfpathlineto{\pgfqpoint{3.081087in}{3.324539in}}%
\pgfpathlineto{\pgfqpoint{3.347446in}{3.247845in}}%
\pgfpathlineto{\pgfqpoint{3.425956in}{3.221824in}}%
\pgfpathlineto{\pgfqpoint{3.497500in}{3.195803in}}%
\pgfpathlineto{\pgfqpoint{3.562475in}{3.169782in}}%
\pgfpathlineto{\pgfqpoint{3.621302in}{3.143761in}}%
\pgfpathlineto{\pgfqpoint{3.671751in}{3.119110in}}%
\pgfpathlineto{\pgfqpoint{3.717429in}{3.094458in}}%
\pgfpathlineto{\pgfqpoint{3.758693in}{3.069807in}}%
\pgfpathlineto{\pgfqpoint{3.795882in}{3.045155in}}%
\pgfpathlineto{\pgfqpoint{3.831073in}{3.019134in}}%
\pgfpathlineto{\pgfqpoint{3.862443in}{2.993113in}}%
\pgfpathlineto{\pgfqpoint{3.890322in}{2.967092in}}%
\pgfpathlineto{\pgfqpoint{3.915015in}{2.941071in}}%
\pgfpathlineto{\pgfqpoint{3.937872in}{2.913681in}}%
\pgfpathlineto{\pgfqpoint{3.957799in}{2.886290in}}%
\pgfpathlineto{\pgfqpoint{3.975057in}{2.858899in}}%
\pgfpathlineto{\pgfqpoint{3.990557in}{2.830139in}}%
\pgfpathlineto{\pgfqpoint{4.003602in}{2.801379in}}%
\pgfpathlineto{\pgfqpoint{4.014848in}{2.771250in}}%
\pgfpathlineto{\pgfqpoint{4.024163in}{2.739751in}}%
\pgfpathlineto{\pgfqpoint{4.031153in}{2.708251in}}%
\pgfpathlineto{\pgfqpoint{4.036112in}{2.675383in}}%
\pgfpathlineto{\pgfqpoint{4.038791in}{2.642514in}}%
\pgfpathlineto{\pgfqpoint{4.039251in}{2.609645in}}%
\pgfpathlineto{\pgfqpoint{4.037508in}{2.576777in}}%
\pgfpathlineto{\pgfqpoint{4.033535in}{2.543908in}}%
\pgfpathlineto{\pgfqpoint{4.027260in}{2.511039in}}%
\pgfpathlineto{\pgfqpoint{4.018980in}{2.479540in}}%
\pgfpathlineto{\pgfqpoint{4.008335in}{2.448041in}}%
\pgfpathlineto{\pgfqpoint{3.995774in}{2.417911in}}%
\pgfpathlineto{\pgfqpoint{3.981431in}{2.389151in}}%
\pgfpathlineto{\pgfqpoint{3.964584in}{2.360391in}}%
\pgfpathlineto{\pgfqpoint{3.945997in}{2.333001in}}%
\pgfpathlineto{\pgfqpoint{3.924696in}{2.305610in}}%
\pgfpathlineto{\pgfqpoint{3.901710in}{2.279589in}}%
\pgfpathlineto{\pgfqpoint{3.875793in}{2.253568in}}%
\pgfpathlineto{\pgfqpoint{3.846675in}{2.227547in}}%
\pgfpathlineto{\pgfqpoint{3.814059in}{2.201526in}}%
\pgfpathlineto{\pgfqpoint{3.779645in}{2.176875in}}%
\pgfpathlineto{\pgfqpoint{3.741517in}{2.152223in}}%
\pgfpathlineto{\pgfqpoint{3.699369in}{2.127572in}}%
\pgfpathlineto{\pgfqpoint{3.652882in}{2.102920in}}%
\pgfpathlineto{\pgfqpoint{3.601729in}{2.078269in}}%
\pgfpathlineto{\pgfqpoint{3.545576in}{2.053617in}}%
\pgfpathlineto{\pgfqpoint{3.480516in}{2.027596in}}%
\pgfpathlineto{\pgfqpoint{3.409144in}{2.001575in}}%
\pgfpathlineto{\pgfqpoint{3.331116in}{1.975554in}}%
\pgfpathlineto{\pgfqpoint{3.246131in}{1.949533in}}%
\pgfpathlineto{\pgfqpoint{3.148906in}{1.922142in}}%
\pgfpathlineto{\pgfqpoint{3.043560in}{1.894752in}}%
\pgfpathlineto{\pgfqpoint{2.924200in}{1.865992in}}%
\pgfpathlineto{\pgfqpoint{2.789784in}{1.835862in}}%
\pgfpathlineto{\pgfqpoint{2.632895in}{1.802993in}}%
\pgfpathlineto{\pgfqpoint{2.445216in}{1.766016in}}%
\pgfpathlineto{\pgfqpoint{2.196386in}{1.719452in}}%
\pgfpathlineto{\pgfqpoint{1.598051in}{1.608520in}}%
\pgfpathlineto{\pgfqpoint{1.442924in}{1.577021in}}%
\pgfpathlineto{\pgfqpoint{1.325028in}{1.551000in}}%
\pgfpathlineto{\pgfqpoint{1.229323in}{1.527718in}}%
\pgfpathlineto{\pgfqpoint{1.153761in}{1.507175in}}%
\pgfpathlineto{\pgfqpoint{1.091592in}{1.488002in}}%
\pgfpathlineto{\pgfqpoint{1.045240in}{1.471568in}}%
\pgfpathlineto{\pgfqpoint{1.008704in}{1.456503in}}%
\pgfpathlineto{\pgfqpoint{0.980638in}{1.442808in}}%
\pgfpathlineto{\pgfqpoint{0.957634in}{1.429112in}}%
\pgfpathlineto{\pgfqpoint{0.941361in}{1.416787in}}%
\pgfpathlineto{\pgfqpoint{0.930477in}{1.405830in}}%
\pgfpathlineto{\pgfqpoint{0.923750in}{1.396244in}}%
\pgfpathlineto{\pgfqpoint{0.919649in}{1.386657in}}%
\pgfpathlineto{\pgfqpoint{0.918183in}{1.377070in}}%
\pgfpathlineto{\pgfqpoint{0.919026in}{1.368853in}}%
\pgfpathlineto{\pgfqpoint{0.922453in}{1.359266in}}%
\pgfpathlineto{\pgfqpoint{0.928500in}{1.349680in}}%
\pgfpathlineto{\pgfqpoint{0.937148in}{1.340093in}}%
\pgfpathlineto{\pgfqpoint{0.950183in}{1.329137in}}%
\pgfpathlineto{\pgfqpoint{0.968802in}{1.316811in}}%
\pgfpathlineto{\pgfqpoint{0.991523in}{1.304485in}}%
\pgfpathlineto{\pgfqpoint{1.021444in}{1.290790in}}%
\pgfpathlineto{\pgfqpoint{1.059839in}{1.275725in}}%
\pgfpathlineto{\pgfqpoint{1.107968in}{1.259291in}}%
\pgfpathlineto{\pgfqpoint{1.167010in}{1.241487in}}%
\pgfpathlineto{\pgfqpoint{1.243334in}{1.220944in}}%
\pgfpathlineto{\pgfqpoint{1.333401in}{1.199032in}}%
\pgfpathlineto{\pgfqpoint{1.450268in}{1.173011in}}%
\pgfpathlineto{\pgfqpoint{1.596267in}{1.142881in}}%
\pgfpathlineto{\pgfqpoint{1.800542in}{1.103165in}}%
\pgfpathlineto{\pgfqpoint{1.894587in}{1.085361in}}%
\pgfpathlineto{\pgfqpoint{1.894587in}{1.085361in}}%
\pgfusepath{stroke}%
\end{pgfscope}%
\begin{pgfscope}%
\pgfpathrectangle{\pgfqpoint{0.750000in}{0.880000in}}{\pgfqpoint{4.650000in}{3.080000in}}%
\pgfusepath{clip}%
\pgfsetbuttcap%
\pgfsetroundjoin%
\pgfsetlinewidth{1.505625pt}%
\definecolor{currentstroke}{rgb}{0.121569,0.466667,0.705882}%
\pgfsetstrokecolor{currentstroke}%
\pgfsetdash{{5.550000pt}{2.400000pt}}{0.000000pt}%
\pgfpathmoveto{\pgfqpoint{0.750000in}{3.283453in}}%
\pgfpathlineto{\pgfqpoint{5.400000in}{3.283453in}}%
\pgfusepath{stroke}%
\end{pgfscope}%
\begin{pgfscope}%
\pgfsetrectcap%
\pgfsetmiterjoin%
\pgfsetlinewidth{0.803000pt}%
\definecolor{currentstroke}{rgb}{0.000000,0.000000,0.000000}%
\pgfsetstrokecolor{currentstroke}%
\pgfsetdash{}{0pt}%
\pgfpathmoveto{\pgfqpoint{0.750000in}{0.880000in}}%
\pgfpathlineto{\pgfqpoint{0.750000in}{3.960000in}}%
\pgfusepath{stroke}%
\end{pgfscope}%
\begin{pgfscope}%
\pgfsetrectcap%
\pgfsetmiterjoin%
\pgfsetlinewidth{0.803000pt}%
\definecolor{currentstroke}{rgb}{0.000000,0.000000,0.000000}%
\pgfsetstrokecolor{currentstroke}%
\pgfsetdash{}{0pt}%
\pgfpathmoveto{\pgfqpoint{5.400000in}{0.880000in}}%
\pgfpathlineto{\pgfqpoint{5.400000in}{3.960000in}}%
\pgfusepath{stroke}%
\end{pgfscope}%
\begin{pgfscope}%
\pgfsetrectcap%
\pgfsetmiterjoin%
\pgfsetlinewidth{0.803000pt}%
\definecolor{currentstroke}{rgb}{0.000000,0.000000,0.000000}%
\pgfsetstrokecolor{currentstroke}%
\pgfsetdash{}{0pt}%
\pgfpathmoveto{\pgfqpoint{0.750000in}{0.880000in}}%
\pgfpathlineto{\pgfqpoint{5.400000in}{0.880000in}}%
\pgfusepath{stroke}%
\end{pgfscope}%
\begin{pgfscope}%
\pgfsetrectcap%
\pgfsetmiterjoin%
\pgfsetlinewidth{0.803000pt}%
\definecolor{currentstroke}{rgb}{0.000000,0.000000,0.000000}%
\pgfsetstrokecolor{currentstroke}%
\pgfsetdash{}{0pt}%
\pgfpathmoveto{\pgfqpoint{0.750000in}{3.960000in}}%
\pgfpathlineto{\pgfqpoint{5.400000in}{3.960000in}}%
\pgfusepath{stroke}%
\end{pgfscope}%
\begin{pgfscope}%
\pgfsetbuttcap%
\pgfsetmiterjoin%
\definecolor{currentfill}{rgb}{1.000000,1.000000,1.000000}%
\pgfsetfillcolor{currentfill}%
\pgfsetfillopacity{0.800000}%
\pgfsetlinewidth{1.003750pt}%
\definecolor{currentstroke}{rgb}{0.800000,0.800000,0.800000}%
\pgfsetstrokecolor{currentstroke}%
\pgfsetstrokeopacity{0.800000}%
\pgfsetdash{}{0pt}%
\pgfpathmoveto{\pgfqpoint{3.899064in}{3.443955in}}%
\pgfpathlineto{\pgfqpoint{5.302778in}{3.443955in}}%
\pgfpathquadraticcurveto{\pgfqpoint{5.330556in}{3.443955in}}{\pgfqpoint{5.330556in}{3.471733in}}%
\pgfpathlineto{\pgfqpoint{5.330556in}{3.862778in}}%
\pgfpathquadraticcurveto{\pgfqpoint{5.330556in}{3.890556in}}{\pgfqpoint{5.302778in}{3.890556in}}%
\pgfpathlineto{\pgfqpoint{3.899064in}{3.890556in}}%
\pgfpathquadraticcurveto{\pgfqpoint{3.871286in}{3.890556in}}{\pgfqpoint{3.871286in}{3.862778in}}%
\pgfpathlineto{\pgfqpoint{3.871286in}{3.471733in}}%
\pgfpathquadraticcurveto{\pgfqpoint{3.871286in}{3.443955in}}{\pgfqpoint{3.899064in}{3.443955in}}%
\pgfpathlineto{\pgfqpoint{3.899064in}{3.443955in}}%
\pgfpathclose%
\pgfusepath{stroke,fill}%
\end{pgfscope}%
\begin{pgfscope}%
\pgfsetrectcap%
\pgfsetroundjoin%
\pgfsetlinewidth{1.505625pt}%
\definecolor{currentstroke}{rgb}{0.121569,0.466667,0.705882}%
\pgfsetstrokecolor{currentstroke}%
\pgfsetdash{}{0pt}%
\pgfpathmoveto{\pgfqpoint{3.926842in}{3.781411in}}%
\pgfpathlineto{\pgfqpoint{4.065731in}{3.781411in}}%
\pgfpathlineto{\pgfqpoint{4.204620in}{3.781411in}}%
\pgfusepath{stroke}%
\end{pgfscope}%
\begin{pgfscope}%
\definecolor{textcolor}{rgb}{0.000000,0.000000,0.000000}%
\pgfsetstrokecolor{textcolor}%
\pgfsetfillcolor{textcolor}%
\pgftext[x=4.315731in,y=3.732800in,left,base]{\color{textcolor}\rmfamily\fontsize{10.000000}{12.000000}\selectfont delta}%
\end{pgfscope}%
\begin{pgfscope}%
\pgfsetbuttcap%
\pgfsetroundjoin%
\pgfsetlinewidth{1.505625pt}%
\definecolor{currentstroke}{rgb}{0.121569,0.466667,0.705882}%
\pgfsetstrokecolor{currentstroke}%
\pgfsetdash{{5.550000pt}{2.400000pt}}{0.000000pt}%
\pgfpathmoveto{\pgfqpoint{3.926842in}{3.578368in}}%
\pgfpathlineto{\pgfqpoint{4.065731in}{3.578368in}}%
\pgfpathlineto{\pgfqpoint{4.204620in}{3.578368in}}%
\pgfusepath{stroke}%
\end{pgfscope}%
\begin{pgfscope}%
\definecolor{textcolor}{rgb}{0.000000,0.000000,0.000000}%
\pgfsetstrokecolor{textcolor}%
\pgfsetfillcolor{textcolor}%
\pgftext[x=4.315731in,y=3.529757in,left,base]{\color{textcolor}\rmfamily\fontsize{10.000000}{12.000000}\selectfont clearing of fault}%
\end{pgfscope}%
\begin{pgfscope}%
\definecolor{textcolor}{rgb}{0.000000,0.000000,0.000000}%
\pgfsetstrokecolor{textcolor}%
\pgfsetfillcolor{textcolor}%
\pgftext[x=3.000000in,y=7.840000in,,top]{\color{textcolor}\rmfamily\fontsize{12.000000}{14.400000}\selectfont Stable scenario - fault 2}%
\end{pgfscope}%
\end{pgfpicture}%
\makeatother%
\endgroup%


%% Creator: Matplotlib, PGF backend
%%
%% To include the figure in your LaTeX document, write
%%   \input{<filename>.pgf}
%%
%% Make sure the required packages are loaded in your preamble
%%   \usepackage{pgf}
%%
%% Also ensure that all the required font packages are loaded; for instance,
%% the lmodern package is sometimes necessary when using math font.
%%   \usepackage{lmodern}
%%
%% Figures using additional raster images can only be included by \input if
%% they are in the same directory as the main LaTeX file. For loading figures
%% from other directories you can use the `import` package
%%   \usepackage{import}
%%
%% and then include the figures with
%%   \import{<path to file>}{<filename>.pgf}
%%
%% Matplotlib used the following preamble
%%   
%%   \usepackage{fontspec}
%%   \setmainfont{Charter.ttc}[Path=\detokenize{/System/Library/Fonts/Supplemental/}]
%%   \setsansfont{DejaVuSans.ttf}[Path=\detokenize{/opt/homebrew/lib/python3.10/site-packages/matplotlib/mpl-data/fonts/ttf/}]
%%   \setmonofont{DejaVuSansMono.ttf}[Path=\detokenize{/opt/homebrew/lib/python3.10/site-packages/matplotlib/mpl-data/fonts/ttf/}]
%%   \makeatletter\@ifpackageloaded{underscore}{}{\usepackage[strings]{underscore}}\makeatother
%%
\begingroup%
\makeatletter%
\begin{pgfpicture}%
\pgfpathrectangle{\pgfpointorigin}{\pgfqpoint{6.000000in}{8.000000in}}%
\pgfusepath{use as bounding box, clip}%
\begin{pgfscope}%
\pgfsetbuttcap%
\pgfsetmiterjoin%
\definecolor{currentfill}{rgb}{1.000000,1.000000,1.000000}%
\pgfsetfillcolor{currentfill}%
\pgfsetlinewidth{0.000000pt}%
\definecolor{currentstroke}{rgb}{1.000000,1.000000,1.000000}%
\pgfsetstrokecolor{currentstroke}%
\pgfsetdash{}{0pt}%
\pgfpathmoveto{\pgfqpoint{0.000000in}{0.000000in}}%
\pgfpathlineto{\pgfqpoint{6.000000in}{0.000000in}}%
\pgfpathlineto{\pgfqpoint{6.000000in}{8.000000in}}%
\pgfpathlineto{\pgfqpoint{0.000000in}{8.000000in}}%
\pgfpathlineto{\pgfqpoint{0.000000in}{0.000000in}}%
\pgfpathclose%
\pgfusepath{fill}%
\end{pgfscope}%
\begin{pgfscope}%
\pgfsetbuttcap%
\pgfsetmiterjoin%
\definecolor{currentfill}{rgb}{1.000000,1.000000,1.000000}%
\pgfsetfillcolor{currentfill}%
\pgfsetlinewidth{0.000000pt}%
\definecolor{currentstroke}{rgb}{0.000000,0.000000,0.000000}%
\pgfsetstrokecolor{currentstroke}%
\pgfsetstrokeopacity{0.000000}%
\pgfsetdash{}{0pt}%
\pgfpathmoveto{\pgfqpoint{0.750000in}{3.960000in}}%
\pgfpathlineto{\pgfqpoint{5.400000in}{3.960000in}}%
\pgfpathlineto{\pgfqpoint{5.400000in}{7.040000in}}%
\pgfpathlineto{\pgfqpoint{0.750000in}{7.040000in}}%
\pgfpathlineto{\pgfqpoint{0.750000in}{3.960000in}}%
\pgfpathclose%
\pgfusepath{fill}%
\end{pgfscope}%
\begin{pgfscope}%
\pgfpathrectangle{\pgfqpoint{0.750000in}{3.960000in}}{\pgfqpoint{4.650000in}{3.080000in}}%
\pgfusepath{clip}%
\pgfsetbuttcap%
\pgfsetroundjoin%
\definecolor{currentfill}{rgb}{0.900000,0.900000,0.900000}%
\pgfsetfillcolor{currentfill}%
\pgfsetlinewidth{1.003750pt}%
\definecolor{currentstroke}{rgb}{0.500000,0.500000,0.500000}%
\pgfsetstrokecolor{currentstroke}%
\pgfsetdash{}{0pt}%
\pgfsys@defobject{currentmarker}{\pgfqpoint{2.005500in}{5.441634in}}{\pgfqpoint{3.228434in}{6.161131in}}{%
\pgfpathmoveto{\pgfqpoint{2.005500in}{6.161131in}}%
\pgfpathlineto{\pgfqpoint{2.005500in}{5.441634in}}%
\pgfpathlineto{\pgfqpoint{2.030458in}{5.463448in}}%
\pgfpathlineto{\pgfqpoint{2.055416in}{5.484835in}}%
\pgfpathlineto{\pgfqpoint{2.080374in}{5.505788in}}%
\pgfpathlineto{\pgfqpoint{2.105331in}{5.526301in}}%
\pgfpathlineto{\pgfqpoint{2.130289in}{5.546369in}}%
\pgfpathlineto{\pgfqpoint{2.155247in}{5.565986in}}%
\pgfpathlineto{\pgfqpoint{2.180205in}{5.585146in}}%
\pgfpathlineto{\pgfqpoint{2.205163in}{5.603845in}}%
\pgfpathlineto{\pgfqpoint{2.230121in}{5.622076in}}%
\pgfpathlineto{\pgfqpoint{2.255078in}{5.639834in}}%
\pgfpathlineto{\pgfqpoint{2.280036in}{5.657115in}}%
\pgfpathlineto{\pgfqpoint{2.304994in}{5.673913in}}%
\pgfpathlineto{\pgfqpoint{2.329952in}{5.690224in}}%
\pgfpathlineto{\pgfqpoint{2.354910in}{5.706043in}}%
\pgfpathlineto{\pgfqpoint{2.379868in}{5.721366in}}%
\pgfpathlineto{\pgfqpoint{2.404825in}{5.736188in}}%
\pgfpathlineto{\pgfqpoint{2.429783in}{5.750505in}}%
\pgfpathlineto{\pgfqpoint{2.454741in}{5.764313in}}%
\pgfpathlineto{\pgfqpoint{2.479699in}{5.777608in}}%
\pgfpathlineto{\pgfqpoint{2.504657in}{5.790386in}}%
\pgfpathlineto{\pgfqpoint{2.529615in}{5.802643in}}%
\pgfpathlineto{\pgfqpoint{2.554572in}{5.814377in}}%
\pgfpathlineto{\pgfqpoint{2.579530in}{5.825584in}}%
\pgfpathlineto{\pgfqpoint{2.604488in}{5.836260in}}%
\pgfpathlineto{\pgfqpoint{2.629446in}{5.846403in}}%
\pgfpathlineto{\pgfqpoint{2.654404in}{5.856009in}}%
\pgfpathlineto{\pgfqpoint{2.679362in}{5.865076in}}%
\pgfpathlineto{\pgfqpoint{2.704319in}{5.873602in}}%
\pgfpathlineto{\pgfqpoint{2.729277in}{5.881584in}}%
\pgfpathlineto{\pgfqpoint{2.754235in}{5.889019in}}%
\pgfpathlineto{\pgfqpoint{2.779193in}{5.895906in}}%
\pgfpathlineto{\pgfqpoint{2.804151in}{5.902242in}}%
\pgfpathlineto{\pgfqpoint{2.829109in}{5.908026in}}%
\pgfpathlineto{\pgfqpoint{2.854066in}{5.913257in}}%
\pgfpathlineto{\pgfqpoint{2.879024in}{5.917932in}}%
\pgfpathlineto{\pgfqpoint{2.903982in}{5.922050in}}%
\pgfpathlineto{\pgfqpoint{2.928940in}{5.925611in}}%
\pgfpathlineto{\pgfqpoint{2.953898in}{5.928613in}}%
\pgfpathlineto{\pgfqpoint{2.978856in}{5.931055in}}%
\pgfpathlineto{\pgfqpoint{3.003813in}{5.932936in}}%
\pgfpathlineto{\pgfqpoint{3.028771in}{5.934257in}}%
\pgfpathlineto{\pgfqpoint{3.053729in}{5.935016in}}%
\pgfpathlineto{\pgfqpoint{3.078687in}{5.935214in}}%
\pgfpathlineto{\pgfqpoint{3.103645in}{5.934850in}}%
\pgfpathlineto{\pgfqpoint{3.128603in}{5.933925in}}%
\pgfpathlineto{\pgfqpoint{3.153560in}{5.932439in}}%
\pgfpathlineto{\pgfqpoint{3.178518in}{5.930391in}}%
\pgfpathlineto{\pgfqpoint{3.203476in}{5.927784in}}%
\pgfpathlineto{\pgfqpoint{3.228434in}{5.924617in}}%
\pgfpathlineto{\pgfqpoint{3.228434in}{6.161131in}}%
\pgfpathlineto{\pgfqpoint{3.228434in}{6.161131in}}%
\pgfpathlineto{\pgfqpoint{3.203476in}{6.161131in}}%
\pgfpathlineto{\pgfqpoint{3.178518in}{6.161131in}}%
\pgfpathlineto{\pgfqpoint{3.153560in}{6.161131in}}%
\pgfpathlineto{\pgfqpoint{3.128603in}{6.161131in}}%
\pgfpathlineto{\pgfqpoint{3.103645in}{6.161131in}}%
\pgfpathlineto{\pgfqpoint{3.078687in}{6.161131in}}%
\pgfpathlineto{\pgfqpoint{3.053729in}{6.161131in}}%
\pgfpathlineto{\pgfqpoint{3.028771in}{6.161131in}}%
\pgfpathlineto{\pgfqpoint{3.003813in}{6.161131in}}%
\pgfpathlineto{\pgfqpoint{2.978856in}{6.161131in}}%
\pgfpathlineto{\pgfqpoint{2.953898in}{6.161131in}}%
\pgfpathlineto{\pgfqpoint{2.928940in}{6.161131in}}%
\pgfpathlineto{\pgfqpoint{2.903982in}{6.161131in}}%
\pgfpathlineto{\pgfqpoint{2.879024in}{6.161131in}}%
\pgfpathlineto{\pgfqpoint{2.854066in}{6.161131in}}%
\pgfpathlineto{\pgfqpoint{2.829109in}{6.161131in}}%
\pgfpathlineto{\pgfqpoint{2.804151in}{6.161131in}}%
\pgfpathlineto{\pgfqpoint{2.779193in}{6.161131in}}%
\pgfpathlineto{\pgfqpoint{2.754235in}{6.161131in}}%
\pgfpathlineto{\pgfqpoint{2.729277in}{6.161131in}}%
\pgfpathlineto{\pgfqpoint{2.704319in}{6.161131in}}%
\pgfpathlineto{\pgfqpoint{2.679362in}{6.161131in}}%
\pgfpathlineto{\pgfqpoint{2.654404in}{6.161131in}}%
\pgfpathlineto{\pgfqpoint{2.629446in}{6.161131in}}%
\pgfpathlineto{\pgfqpoint{2.604488in}{6.161131in}}%
\pgfpathlineto{\pgfqpoint{2.579530in}{6.161131in}}%
\pgfpathlineto{\pgfqpoint{2.554572in}{6.161131in}}%
\pgfpathlineto{\pgfqpoint{2.529615in}{6.161131in}}%
\pgfpathlineto{\pgfqpoint{2.504657in}{6.161131in}}%
\pgfpathlineto{\pgfqpoint{2.479699in}{6.161131in}}%
\pgfpathlineto{\pgfqpoint{2.454741in}{6.161131in}}%
\pgfpathlineto{\pgfqpoint{2.429783in}{6.161131in}}%
\pgfpathlineto{\pgfqpoint{2.404825in}{6.161131in}}%
\pgfpathlineto{\pgfqpoint{2.379868in}{6.161131in}}%
\pgfpathlineto{\pgfqpoint{2.354910in}{6.161131in}}%
\pgfpathlineto{\pgfqpoint{2.329952in}{6.161131in}}%
\pgfpathlineto{\pgfqpoint{2.304994in}{6.161131in}}%
\pgfpathlineto{\pgfqpoint{2.280036in}{6.161131in}}%
\pgfpathlineto{\pgfqpoint{2.255078in}{6.161131in}}%
\pgfpathlineto{\pgfqpoint{2.230121in}{6.161131in}}%
\pgfpathlineto{\pgfqpoint{2.205163in}{6.161131in}}%
\pgfpathlineto{\pgfqpoint{2.180205in}{6.161131in}}%
\pgfpathlineto{\pgfqpoint{2.155247in}{6.161131in}}%
\pgfpathlineto{\pgfqpoint{2.130289in}{6.161131in}}%
\pgfpathlineto{\pgfqpoint{2.105331in}{6.161131in}}%
\pgfpathlineto{\pgfqpoint{2.080374in}{6.161131in}}%
\pgfpathlineto{\pgfqpoint{2.055416in}{6.161131in}}%
\pgfpathlineto{\pgfqpoint{2.030458in}{6.161131in}}%
\pgfpathlineto{\pgfqpoint{2.005500in}{6.161131in}}%
\pgfpathlineto{\pgfqpoint{2.005500in}{6.161131in}}%
\pgfpathclose%
\pgfusepath{stroke,fill}%
}%
\begin{pgfscope}%
\pgfsys@transformshift{0.000000in}{0.000000in}%
\pgfsys@useobject{currentmarker}{}%
\end{pgfscope}%
\end{pgfscope}%
\begin{pgfscope}%
\pgfpathrectangle{\pgfqpoint{0.750000in}{3.960000in}}{\pgfqpoint{4.650000in}{3.080000in}}%
\pgfusepath{clip}%
\pgfsetbuttcap%
\pgfsetroundjoin%
\definecolor{currentfill}{rgb}{0.900000,0.900000,0.900000}%
\pgfsetfillcolor{currentfill}%
\pgfsetlinewidth{1.003750pt}%
\definecolor{currentstroke}{rgb}{0.500000,0.500000,0.500000}%
\pgfsetstrokecolor{currentstroke}%
\pgfsetdash{}{0pt}%
\pgfsys@defobject{currentmarker}{\pgfqpoint{3.228434in}{6.161131in}}{\pgfqpoint{4.144500in}{6.879087in}}{%
\pgfpathmoveto{\pgfqpoint{3.228434in}{6.161131in}}%
\pgfpathlineto{\pgfqpoint{3.228434in}{6.879087in}}%
\pgfpathlineto{\pgfqpoint{3.247129in}{6.875018in}}%
\pgfpathlineto{\pgfqpoint{3.265824in}{6.870485in}}%
\pgfpathlineto{\pgfqpoint{3.284520in}{6.865487in}}%
\pgfpathlineto{\pgfqpoint{3.303215in}{6.860025in}}%
\pgfpathlineto{\pgfqpoint{3.321910in}{6.854101in}}%
\pgfpathlineto{\pgfqpoint{3.340605in}{6.847716in}}%
\pgfpathlineto{\pgfqpoint{3.359301in}{6.840869in}}%
\pgfpathlineto{\pgfqpoint{3.377996in}{6.833563in}}%
\pgfpathlineto{\pgfqpoint{3.396691in}{6.825799in}}%
\pgfpathlineto{\pgfqpoint{3.415386in}{6.817577in}}%
\pgfpathlineto{\pgfqpoint{3.434081in}{6.808900in}}%
\pgfpathlineto{\pgfqpoint{3.452777in}{6.799768in}}%
\pgfpathlineto{\pgfqpoint{3.471472in}{6.790183in}}%
\pgfpathlineto{\pgfqpoint{3.490167in}{6.780146in}}%
\pgfpathlineto{\pgfqpoint{3.508862in}{6.769660in}}%
\pgfpathlineto{\pgfqpoint{3.527558in}{6.758725in}}%
\pgfpathlineto{\pgfqpoint{3.546253in}{6.747344in}}%
\pgfpathlineto{\pgfqpoint{3.564948in}{6.735518in}}%
\pgfpathlineto{\pgfqpoint{3.583643in}{6.723249in}}%
\pgfpathlineto{\pgfqpoint{3.602338in}{6.710540in}}%
\pgfpathlineto{\pgfqpoint{3.621034in}{6.697392in}}%
\pgfpathlineto{\pgfqpoint{3.639729in}{6.683807in}}%
\pgfpathlineto{\pgfqpoint{3.658424in}{6.669787in}}%
\pgfpathlineto{\pgfqpoint{3.677119in}{6.655335in}}%
\pgfpathlineto{\pgfqpoint{3.695815in}{6.640453in}}%
\pgfpathlineto{\pgfqpoint{3.714510in}{6.625144in}}%
\pgfpathlineto{\pgfqpoint{3.733205in}{6.609409in}}%
\pgfpathlineto{\pgfqpoint{3.751900in}{6.593252in}}%
\pgfpathlineto{\pgfqpoint{3.770596in}{6.576675in}}%
\pgfpathlineto{\pgfqpoint{3.789291in}{6.559680in}}%
\pgfpathlineto{\pgfqpoint{3.807986in}{6.542271in}}%
\pgfpathlineto{\pgfqpoint{3.826681in}{6.524449in}}%
\pgfpathlineto{\pgfqpoint{3.845376in}{6.506218in}}%
\pgfpathlineto{\pgfqpoint{3.864072in}{6.487582in}}%
\pgfpathlineto{\pgfqpoint{3.882767in}{6.468542in}}%
\pgfpathlineto{\pgfqpoint{3.901462in}{6.449101in}}%
\pgfpathlineto{\pgfqpoint{3.920157in}{6.429264in}}%
\pgfpathlineto{\pgfqpoint{3.938853in}{6.409033in}}%
\pgfpathlineto{\pgfqpoint{3.957548in}{6.388411in}}%
\pgfpathlineto{\pgfqpoint{3.976243in}{6.367402in}}%
\pgfpathlineto{\pgfqpoint{3.994938in}{6.346008in}}%
\pgfpathlineto{\pgfqpoint{4.013633in}{6.324234in}}%
\pgfpathlineto{\pgfqpoint{4.032329in}{6.302083in}}%
\pgfpathlineto{\pgfqpoint{4.051024in}{6.279558in}}%
\pgfpathlineto{\pgfqpoint{4.069719in}{6.256663in}}%
\pgfpathlineto{\pgfqpoint{4.088414in}{6.233402in}}%
\pgfpathlineto{\pgfqpoint{4.107110in}{6.209778in}}%
\pgfpathlineto{\pgfqpoint{4.125805in}{6.185795in}}%
\pgfpathlineto{\pgfqpoint{4.144500in}{6.161457in}}%
\pgfpathlineto{\pgfqpoint{4.144500in}{6.161131in}}%
\pgfpathlineto{\pgfqpoint{4.144500in}{6.161131in}}%
\pgfpathlineto{\pgfqpoint{4.125805in}{6.161131in}}%
\pgfpathlineto{\pgfqpoint{4.107110in}{6.161131in}}%
\pgfpathlineto{\pgfqpoint{4.088414in}{6.161131in}}%
\pgfpathlineto{\pgfqpoint{4.069719in}{6.161131in}}%
\pgfpathlineto{\pgfqpoint{4.051024in}{6.161131in}}%
\pgfpathlineto{\pgfqpoint{4.032329in}{6.161131in}}%
\pgfpathlineto{\pgfqpoint{4.013633in}{6.161131in}}%
\pgfpathlineto{\pgfqpoint{3.994938in}{6.161131in}}%
\pgfpathlineto{\pgfqpoint{3.976243in}{6.161131in}}%
\pgfpathlineto{\pgfqpoint{3.957548in}{6.161131in}}%
\pgfpathlineto{\pgfqpoint{3.938853in}{6.161131in}}%
\pgfpathlineto{\pgfqpoint{3.920157in}{6.161131in}}%
\pgfpathlineto{\pgfqpoint{3.901462in}{6.161131in}}%
\pgfpathlineto{\pgfqpoint{3.882767in}{6.161131in}}%
\pgfpathlineto{\pgfqpoint{3.864072in}{6.161131in}}%
\pgfpathlineto{\pgfqpoint{3.845376in}{6.161131in}}%
\pgfpathlineto{\pgfqpoint{3.826681in}{6.161131in}}%
\pgfpathlineto{\pgfqpoint{3.807986in}{6.161131in}}%
\pgfpathlineto{\pgfqpoint{3.789291in}{6.161131in}}%
\pgfpathlineto{\pgfqpoint{3.770596in}{6.161131in}}%
\pgfpathlineto{\pgfqpoint{3.751900in}{6.161131in}}%
\pgfpathlineto{\pgfqpoint{3.733205in}{6.161131in}}%
\pgfpathlineto{\pgfqpoint{3.714510in}{6.161131in}}%
\pgfpathlineto{\pgfqpoint{3.695815in}{6.161131in}}%
\pgfpathlineto{\pgfqpoint{3.677119in}{6.161131in}}%
\pgfpathlineto{\pgfqpoint{3.658424in}{6.161131in}}%
\pgfpathlineto{\pgfqpoint{3.639729in}{6.161131in}}%
\pgfpathlineto{\pgfqpoint{3.621034in}{6.161131in}}%
\pgfpathlineto{\pgfqpoint{3.602338in}{6.161131in}}%
\pgfpathlineto{\pgfqpoint{3.583643in}{6.161131in}}%
\pgfpathlineto{\pgfqpoint{3.564948in}{6.161131in}}%
\pgfpathlineto{\pgfqpoint{3.546253in}{6.161131in}}%
\pgfpathlineto{\pgfqpoint{3.527558in}{6.161131in}}%
\pgfpathlineto{\pgfqpoint{3.508862in}{6.161131in}}%
\pgfpathlineto{\pgfqpoint{3.490167in}{6.161131in}}%
\pgfpathlineto{\pgfqpoint{3.471472in}{6.161131in}}%
\pgfpathlineto{\pgfqpoint{3.452777in}{6.161131in}}%
\pgfpathlineto{\pgfqpoint{3.434081in}{6.161131in}}%
\pgfpathlineto{\pgfqpoint{3.415386in}{6.161131in}}%
\pgfpathlineto{\pgfqpoint{3.396691in}{6.161131in}}%
\pgfpathlineto{\pgfqpoint{3.377996in}{6.161131in}}%
\pgfpathlineto{\pgfqpoint{3.359301in}{6.161131in}}%
\pgfpathlineto{\pgfqpoint{3.340605in}{6.161131in}}%
\pgfpathlineto{\pgfqpoint{3.321910in}{6.161131in}}%
\pgfpathlineto{\pgfqpoint{3.303215in}{6.161131in}}%
\pgfpathlineto{\pgfqpoint{3.284520in}{6.161131in}}%
\pgfpathlineto{\pgfqpoint{3.265824in}{6.161131in}}%
\pgfpathlineto{\pgfqpoint{3.247129in}{6.161131in}}%
\pgfpathlineto{\pgfqpoint{3.228434in}{6.161131in}}%
\pgfpathlineto{\pgfqpoint{3.228434in}{6.161131in}}%
\pgfpathclose%
\pgfusepath{stroke,fill}%
}%
\begin{pgfscope}%
\pgfsys@transformshift{0.000000in}{0.000000in}%
\pgfsys@useobject{currentmarker}{}%
\end{pgfscope}%
\end{pgfscope}%
\begin{pgfscope}%
\pgfpathrectangle{\pgfqpoint{0.750000in}{3.960000in}}{\pgfqpoint{4.650000in}{3.080000in}}%
\pgfusepath{clip}%
\pgfsetrectcap%
\pgfsetroundjoin%
\pgfsetlinewidth{0.803000pt}%
\definecolor{currentstroke}{rgb}{0.690196,0.690196,0.690196}%
\pgfsetstrokecolor{currentstroke}%
\pgfsetdash{}{0pt}%
\pgfpathmoveto{\pgfqpoint{0.750000in}{3.960000in}}%
\pgfpathlineto{\pgfqpoint{0.750000in}{7.040000in}}%
\pgfusepath{stroke}%
\end{pgfscope}%
\begin{pgfscope}%
\pgfsetbuttcap%
\pgfsetroundjoin%
\definecolor{currentfill}{rgb}{0.000000,0.000000,0.000000}%
\pgfsetfillcolor{currentfill}%
\pgfsetlinewidth{0.803000pt}%
\definecolor{currentstroke}{rgb}{0.000000,0.000000,0.000000}%
\pgfsetstrokecolor{currentstroke}%
\pgfsetdash{}{0pt}%
\pgfsys@defobject{currentmarker}{\pgfqpoint{0.000000in}{-0.048611in}}{\pgfqpoint{0.000000in}{0.000000in}}{%
\pgfpathmoveto{\pgfqpoint{0.000000in}{0.000000in}}%
\pgfpathlineto{\pgfqpoint{0.000000in}{-0.048611in}}%
\pgfusepath{stroke,fill}%
}%
\begin{pgfscope}%
\pgfsys@transformshift{0.750000in}{3.960000in}%
\pgfsys@useobject{currentmarker}{}%
\end{pgfscope}%
\end{pgfscope}%
\begin{pgfscope}%
\pgfpathrectangle{\pgfqpoint{0.750000in}{3.960000in}}{\pgfqpoint{4.650000in}{3.080000in}}%
\pgfusepath{clip}%
\pgfsetrectcap%
\pgfsetroundjoin%
\pgfsetlinewidth{0.803000pt}%
\definecolor{currentstroke}{rgb}{0.690196,0.690196,0.690196}%
\pgfsetstrokecolor{currentstroke}%
\pgfsetdash{}{0pt}%
\pgfpathmoveto{\pgfqpoint{1.266667in}{3.960000in}}%
\pgfpathlineto{\pgfqpoint{1.266667in}{7.040000in}}%
\pgfusepath{stroke}%
\end{pgfscope}%
\begin{pgfscope}%
\pgfsetbuttcap%
\pgfsetroundjoin%
\definecolor{currentfill}{rgb}{0.000000,0.000000,0.000000}%
\pgfsetfillcolor{currentfill}%
\pgfsetlinewidth{0.803000pt}%
\definecolor{currentstroke}{rgb}{0.000000,0.000000,0.000000}%
\pgfsetstrokecolor{currentstroke}%
\pgfsetdash{}{0pt}%
\pgfsys@defobject{currentmarker}{\pgfqpoint{0.000000in}{-0.048611in}}{\pgfqpoint{0.000000in}{0.000000in}}{%
\pgfpathmoveto{\pgfqpoint{0.000000in}{0.000000in}}%
\pgfpathlineto{\pgfqpoint{0.000000in}{-0.048611in}}%
\pgfusepath{stroke,fill}%
}%
\begin{pgfscope}%
\pgfsys@transformshift{1.266667in}{3.960000in}%
\pgfsys@useobject{currentmarker}{}%
\end{pgfscope}%
\end{pgfscope}%
\begin{pgfscope}%
\pgfpathrectangle{\pgfqpoint{0.750000in}{3.960000in}}{\pgfqpoint{4.650000in}{3.080000in}}%
\pgfusepath{clip}%
\pgfsetrectcap%
\pgfsetroundjoin%
\pgfsetlinewidth{0.803000pt}%
\definecolor{currentstroke}{rgb}{0.690196,0.690196,0.690196}%
\pgfsetstrokecolor{currentstroke}%
\pgfsetdash{}{0pt}%
\pgfpathmoveto{\pgfqpoint{1.783333in}{3.960000in}}%
\pgfpathlineto{\pgfqpoint{1.783333in}{7.040000in}}%
\pgfusepath{stroke}%
\end{pgfscope}%
\begin{pgfscope}%
\pgfsetbuttcap%
\pgfsetroundjoin%
\definecolor{currentfill}{rgb}{0.000000,0.000000,0.000000}%
\pgfsetfillcolor{currentfill}%
\pgfsetlinewidth{0.803000pt}%
\definecolor{currentstroke}{rgb}{0.000000,0.000000,0.000000}%
\pgfsetstrokecolor{currentstroke}%
\pgfsetdash{}{0pt}%
\pgfsys@defobject{currentmarker}{\pgfqpoint{0.000000in}{-0.048611in}}{\pgfqpoint{0.000000in}{0.000000in}}{%
\pgfpathmoveto{\pgfqpoint{0.000000in}{0.000000in}}%
\pgfpathlineto{\pgfqpoint{0.000000in}{-0.048611in}}%
\pgfusepath{stroke,fill}%
}%
\begin{pgfscope}%
\pgfsys@transformshift{1.783333in}{3.960000in}%
\pgfsys@useobject{currentmarker}{}%
\end{pgfscope}%
\end{pgfscope}%
\begin{pgfscope}%
\pgfpathrectangle{\pgfqpoint{0.750000in}{3.960000in}}{\pgfqpoint{4.650000in}{3.080000in}}%
\pgfusepath{clip}%
\pgfsetrectcap%
\pgfsetroundjoin%
\pgfsetlinewidth{0.803000pt}%
\definecolor{currentstroke}{rgb}{0.690196,0.690196,0.690196}%
\pgfsetstrokecolor{currentstroke}%
\pgfsetdash{}{0pt}%
\pgfpathmoveto{\pgfqpoint{2.300000in}{3.960000in}}%
\pgfpathlineto{\pgfqpoint{2.300000in}{7.040000in}}%
\pgfusepath{stroke}%
\end{pgfscope}%
\begin{pgfscope}%
\pgfsetbuttcap%
\pgfsetroundjoin%
\definecolor{currentfill}{rgb}{0.000000,0.000000,0.000000}%
\pgfsetfillcolor{currentfill}%
\pgfsetlinewidth{0.803000pt}%
\definecolor{currentstroke}{rgb}{0.000000,0.000000,0.000000}%
\pgfsetstrokecolor{currentstroke}%
\pgfsetdash{}{0pt}%
\pgfsys@defobject{currentmarker}{\pgfqpoint{0.000000in}{-0.048611in}}{\pgfqpoint{0.000000in}{0.000000in}}{%
\pgfpathmoveto{\pgfqpoint{0.000000in}{0.000000in}}%
\pgfpathlineto{\pgfqpoint{0.000000in}{-0.048611in}}%
\pgfusepath{stroke,fill}%
}%
\begin{pgfscope}%
\pgfsys@transformshift{2.300000in}{3.960000in}%
\pgfsys@useobject{currentmarker}{}%
\end{pgfscope}%
\end{pgfscope}%
\begin{pgfscope}%
\pgfpathrectangle{\pgfqpoint{0.750000in}{3.960000in}}{\pgfqpoint{4.650000in}{3.080000in}}%
\pgfusepath{clip}%
\pgfsetrectcap%
\pgfsetroundjoin%
\pgfsetlinewidth{0.803000pt}%
\definecolor{currentstroke}{rgb}{0.690196,0.690196,0.690196}%
\pgfsetstrokecolor{currentstroke}%
\pgfsetdash{}{0pt}%
\pgfpathmoveto{\pgfqpoint{2.816667in}{3.960000in}}%
\pgfpathlineto{\pgfqpoint{2.816667in}{7.040000in}}%
\pgfusepath{stroke}%
\end{pgfscope}%
\begin{pgfscope}%
\pgfsetbuttcap%
\pgfsetroundjoin%
\definecolor{currentfill}{rgb}{0.000000,0.000000,0.000000}%
\pgfsetfillcolor{currentfill}%
\pgfsetlinewidth{0.803000pt}%
\definecolor{currentstroke}{rgb}{0.000000,0.000000,0.000000}%
\pgfsetstrokecolor{currentstroke}%
\pgfsetdash{}{0pt}%
\pgfsys@defobject{currentmarker}{\pgfqpoint{0.000000in}{-0.048611in}}{\pgfqpoint{0.000000in}{0.000000in}}{%
\pgfpathmoveto{\pgfqpoint{0.000000in}{0.000000in}}%
\pgfpathlineto{\pgfqpoint{0.000000in}{-0.048611in}}%
\pgfusepath{stroke,fill}%
}%
\begin{pgfscope}%
\pgfsys@transformshift{2.816667in}{3.960000in}%
\pgfsys@useobject{currentmarker}{}%
\end{pgfscope}%
\end{pgfscope}%
\begin{pgfscope}%
\pgfpathrectangle{\pgfqpoint{0.750000in}{3.960000in}}{\pgfqpoint{4.650000in}{3.080000in}}%
\pgfusepath{clip}%
\pgfsetrectcap%
\pgfsetroundjoin%
\pgfsetlinewidth{0.803000pt}%
\definecolor{currentstroke}{rgb}{0.690196,0.690196,0.690196}%
\pgfsetstrokecolor{currentstroke}%
\pgfsetdash{}{0pt}%
\pgfpathmoveto{\pgfqpoint{3.333333in}{3.960000in}}%
\pgfpathlineto{\pgfqpoint{3.333333in}{7.040000in}}%
\pgfusepath{stroke}%
\end{pgfscope}%
\begin{pgfscope}%
\pgfsetbuttcap%
\pgfsetroundjoin%
\definecolor{currentfill}{rgb}{0.000000,0.000000,0.000000}%
\pgfsetfillcolor{currentfill}%
\pgfsetlinewidth{0.803000pt}%
\definecolor{currentstroke}{rgb}{0.000000,0.000000,0.000000}%
\pgfsetstrokecolor{currentstroke}%
\pgfsetdash{}{0pt}%
\pgfsys@defobject{currentmarker}{\pgfqpoint{0.000000in}{-0.048611in}}{\pgfqpoint{0.000000in}{0.000000in}}{%
\pgfpathmoveto{\pgfqpoint{0.000000in}{0.000000in}}%
\pgfpathlineto{\pgfqpoint{0.000000in}{-0.048611in}}%
\pgfusepath{stroke,fill}%
}%
\begin{pgfscope}%
\pgfsys@transformshift{3.333333in}{3.960000in}%
\pgfsys@useobject{currentmarker}{}%
\end{pgfscope}%
\end{pgfscope}%
\begin{pgfscope}%
\pgfpathrectangle{\pgfqpoint{0.750000in}{3.960000in}}{\pgfqpoint{4.650000in}{3.080000in}}%
\pgfusepath{clip}%
\pgfsetrectcap%
\pgfsetroundjoin%
\pgfsetlinewidth{0.803000pt}%
\definecolor{currentstroke}{rgb}{0.690196,0.690196,0.690196}%
\pgfsetstrokecolor{currentstroke}%
\pgfsetdash{}{0pt}%
\pgfpathmoveto{\pgfqpoint{3.850000in}{3.960000in}}%
\pgfpathlineto{\pgfqpoint{3.850000in}{7.040000in}}%
\pgfusepath{stroke}%
\end{pgfscope}%
\begin{pgfscope}%
\pgfsetbuttcap%
\pgfsetroundjoin%
\definecolor{currentfill}{rgb}{0.000000,0.000000,0.000000}%
\pgfsetfillcolor{currentfill}%
\pgfsetlinewidth{0.803000pt}%
\definecolor{currentstroke}{rgb}{0.000000,0.000000,0.000000}%
\pgfsetstrokecolor{currentstroke}%
\pgfsetdash{}{0pt}%
\pgfsys@defobject{currentmarker}{\pgfqpoint{0.000000in}{-0.048611in}}{\pgfqpoint{0.000000in}{0.000000in}}{%
\pgfpathmoveto{\pgfqpoint{0.000000in}{0.000000in}}%
\pgfpathlineto{\pgfqpoint{0.000000in}{-0.048611in}}%
\pgfusepath{stroke,fill}%
}%
\begin{pgfscope}%
\pgfsys@transformshift{3.850000in}{3.960000in}%
\pgfsys@useobject{currentmarker}{}%
\end{pgfscope}%
\end{pgfscope}%
\begin{pgfscope}%
\pgfpathrectangle{\pgfqpoint{0.750000in}{3.960000in}}{\pgfqpoint{4.650000in}{3.080000in}}%
\pgfusepath{clip}%
\pgfsetrectcap%
\pgfsetroundjoin%
\pgfsetlinewidth{0.803000pt}%
\definecolor{currentstroke}{rgb}{0.690196,0.690196,0.690196}%
\pgfsetstrokecolor{currentstroke}%
\pgfsetdash{}{0pt}%
\pgfpathmoveto{\pgfqpoint{4.366667in}{3.960000in}}%
\pgfpathlineto{\pgfqpoint{4.366667in}{7.040000in}}%
\pgfusepath{stroke}%
\end{pgfscope}%
\begin{pgfscope}%
\pgfsetbuttcap%
\pgfsetroundjoin%
\definecolor{currentfill}{rgb}{0.000000,0.000000,0.000000}%
\pgfsetfillcolor{currentfill}%
\pgfsetlinewidth{0.803000pt}%
\definecolor{currentstroke}{rgb}{0.000000,0.000000,0.000000}%
\pgfsetstrokecolor{currentstroke}%
\pgfsetdash{}{0pt}%
\pgfsys@defobject{currentmarker}{\pgfqpoint{0.000000in}{-0.048611in}}{\pgfqpoint{0.000000in}{0.000000in}}{%
\pgfpathmoveto{\pgfqpoint{0.000000in}{0.000000in}}%
\pgfpathlineto{\pgfqpoint{0.000000in}{-0.048611in}}%
\pgfusepath{stroke,fill}%
}%
\begin{pgfscope}%
\pgfsys@transformshift{4.366667in}{3.960000in}%
\pgfsys@useobject{currentmarker}{}%
\end{pgfscope}%
\end{pgfscope}%
\begin{pgfscope}%
\pgfpathrectangle{\pgfqpoint{0.750000in}{3.960000in}}{\pgfqpoint{4.650000in}{3.080000in}}%
\pgfusepath{clip}%
\pgfsetrectcap%
\pgfsetroundjoin%
\pgfsetlinewidth{0.803000pt}%
\definecolor{currentstroke}{rgb}{0.690196,0.690196,0.690196}%
\pgfsetstrokecolor{currentstroke}%
\pgfsetdash{}{0pt}%
\pgfpathmoveto{\pgfqpoint{4.883333in}{3.960000in}}%
\pgfpathlineto{\pgfqpoint{4.883333in}{7.040000in}}%
\pgfusepath{stroke}%
\end{pgfscope}%
\begin{pgfscope}%
\pgfsetbuttcap%
\pgfsetroundjoin%
\definecolor{currentfill}{rgb}{0.000000,0.000000,0.000000}%
\pgfsetfillcolor{currentfill}%
\pgfsetlinewidth{0.803000pt}%
\definecolor{currentstroke}{rgb}{0.000000,0.000000,0.000000}%
\pgfsetstrokecolor{currentstroke}%
\pgfsetdash{}{0pt}%
\pgfsys@defobject{currentmarker}{\pgfqpoint{0.000000in}{-0.048611in}}{\pgfqpoint{0.000000in}{0.000000in}}{%
\pgfpathmoveto{\pgfqpoint{0.000000in}{0.000000in}}%
\pgfpathlineto{\pgfqpoint{0.000000in}{-0.048611in}}%
\pgfusepath{stroke,fill}%
}%
\begin{pgfscope}%
\pgfsys@transformshift{4.883333in}{3.960000in}%
\pgfsys@useobject{currentmarker}{}%
\end{pgfscope}%
\end{pgfscope}%
\begin{pgfscope}%
\pgfpathrectangle{\pgfqpoint{0.750000in}{3.960000in}}{\pgfqpoint{4.650000in}{3.080000in}}%
\pgfusepath{clip}%
\pgfsetrectcap%
\pgfsetroundjoin%
\pgfsetlinewidth{0.803000pt}%
\definecolor{currentstroke}{rgb}{0.690196,0.690196,0.690196}%
\pgfsetstrokecolor{currentstroke}%
\pgfsetdash{}{0pt}%
\pgfpathmoveto{\pgfqpoint{5.400000in}{3.960000in}}%
\pgfpathlineto{\pgfqpoint{5.400000in}{7.040000in}}%
\pgfusepath{stroke}%
\end{pgfscope}%
\begin{pgfscope}%
\pgfsetbuttcap%
\pgfsetroundjoin%
\definecolor{currentfill}{rgb}{0.000000,0.000000,0.000000}%
\pgfsetfillcolor{currentfill}%
\pgfsetlinewidth{0.803000pt}%
\definecolor{currentstroke}{rgb}{0.000000,0.000000,0.000000}%
\pgfsetstrokecolor{currentstroke}%
\pgfsetdash{}{0pt}%
\pgfsys@defobject{currentmarker}{\pgfqpoint{0.000000in}{-0.048611in}}{\pgfqpoint{0.000000in}{0.000000in}}{%
\pgfpathmoveto{\pgfqpoint{0.000000in}{0.000000in}}%
\pgfpathlineto{\pgfqpoint{0.000000in}{-0.048611in}}%
\pgfusepath{stroke,fill}%
}%
\begin{pgfscope}%
\pgfsys@transformshift{5.400000in}{3.960000in}%
\pgfsys@useobject{currentmarker}{}%
\end{pgfscope}%
\end{pgfscope}%
\begin{pgfscope}%
\pgfpathrectangle{\pgfqpoint{0.750000in}{3.960000in}}{\pgfqpoint{4.650000in}{3.080000in}}%
\pgfusepath{clip}%
\pgfsetrectcap%
\pgfsetroundjoin%
\pgfsetlinewidth{0.803000pt}%
\definecolor{currentstroke}{rgb}{0.690196,0.690196,0.690196}%
\pgfsetstrokecolor{currentstroke}%
\pgfsetdash{}{0pt}%
\pgfpathmoveto{\pgfqpoint{0.750000in}{3.960000in}}%
\pgfpathlineto{\pgfqpoint{5.400000in}{3.960000in}}%
\pgfusepath{stroke}%
\end{pgfscope}%
\begin{pgfscope}%
\pgfsetbuttcap%
\pgfsetroundjoin%
\definecolor{currentfill}{rgb}{0.000000,0.000000,0.000000}%
\pgfsetfillcolor{currentfill}%
\pgfsetlinewidth{0.803000pt}%
\definecolor{currentstroke}{rgb}{0.000000,0.000000,0.000000}%
\pgfsetstrokecolor{currentstroke}%
\pgfsetdash{}{0pt}%
\pgfsys@defobject{currentmarker}{\pgfqpoint{-0.048611in}{0.000000in}}{\pgfqpoint{-0.000000in}{0.000000in}}{%
\pgfpathmoveto{\pgfqpoint{-0.000000in}{0.000000in}}%
\pgfpathlineto{\pgfqpoint{-0.048611in}{0.000000in}}%
\pgfusepath{stroke,fill}%
}%
\begin{pgfscope}%
\pgfsys@transformshift{0.750000in}{3.960000in}%
\pgfsys@useobject{currentmarker}{}%
\end{pgfscope}%
\end{pgfscope}%
\begin{pgfscope}%
\definecolor{textcolor}{rgb}{0.000000,0.000000,0.000000}%
\pgfsetstrokecolor{textcolor}%
\pgfsetfillcolor{textcolor}%
\pgftext[x=0.475308in, y=3.908900in, left, base]{\color{textcolor}\rmfamily\fontsize{10.000000}{12.000000}\selectfont \(\displaystyle {0.0}\)}%
\end{pgfscope}%
\begin{pgfscope}%
\pgfpathrectangle{\pgfqpoint{0.750000in}{3.960000in}}{\pgfqpoint{4.650000in}{3.080000in}}%
\pgfusepath{clip}%
\pgfsetrectcap%
\pgfsetroundjoin%
\pgfsetlinewidth{0.803000pt}%
\definecolor{currentstroke}{rgb}{0.690196,0.690196,0.690196}%
\pgfsetstrokecolor{currentstroke}%
\pgfsetdash{}{0pt}%
\pgfpathmoveto{\pgfqpoint{0.750000in}{4.449140in}}%
\pgfpathlineto{\pgfqpoint{5.400000in}{4.449140in}}%
\pgfusepath{stroke}%
\end{pgfscope}%
\begin{pgfscope}%
\pgfsetbuttcap%
\pgfsetroundjoin%
\definecolor{currentfill}{rgb}{0.000000,0.000000,0.000000}%
\pgfsetfillcolor{currentfill}%
\pgfsetlinewidth{0.803000pt}%
\definecolor{currentstroke}{rgb}{0.000000,0.000000,0.000000}%
\pgfsetstrokecolor{currentstroke}%
\pgfsetdash{}{0pt}%
\pgfsys@defobject{currentmarker}{\pgfqpoint{-0.048611in}{0.000000in}}{\pgfqpoint{-0.000000in}{0.000000in}}{%
\pgfpathmoveto{\pgfqpoint{-0.000000in}{0.000000in}}%
\pgfpathlineto{\pgfqpoint{-0.048611in}{0.000000in}}%
\pgfusepath{stroke,fill}%
}%
\begin{pgfscope}%
\pgfsys@transformshift{0.750000in}{4.449140in}%
\pgfsys@useobject{currentmarker}{}%
\end{pgfscope}%
\end{pgfscope}%
\begin{pgfscope}%
\definecolor{textcolor}{rgb}{0.000000,0.000000,0.000000}%
\pgfsetstrokecolor{textcolor}%
\pgfsetfillcolor{textcolor}%
\pgftext[x=0.475308in, y=4.398040in, left, base]{\color{textcolor}\rmfamily\fontsize{10.000000}{12.000000}\selectfont \(\displaystyle {0.2}\)}%
\end{pgfscope}%
\begin{pgfscope}%
\pgfpathrectangle{\pgfqpoint{0.750000in}{3.960000in}}{\pgfqpoint{4.650000in}{3.080000in}}%
\pgfusepath{clip}%
\pgfsetrectcap%
\pgfsetroundjoin%
\pgfsetlinewidth{0.803000pt}%
\definecolor{currentstroke}{rgb}{0.690196,0.690196,0.690196}%
\pgfsetstrokecolor{currentstroke}%
\pgfsetdash{}{0pt}%
\pgfpathmoveto{\pgfqpoint{0.750000in}{4.938280in}}%
\pgfpathlineto{\pgfqpoint{5.400000in}{4.938280in}}%
\pgfusepath{stroke}%
\end{pgfscope}%
\begin{pgfscope}%
\pgfsetbuttcap%
\pgfsetroundjoin%
\definecolor{currentfill}{rgb}{0.000000,0.000000,0.000000}%
\pgfsetfillcolor{currentfill}%
\pgfsetlinewidth{0.803000pt}%
\definecolor{currentstroke}{rgb}{0.000000,0.000000,0.000000}%
\pgfsetstrokecolor{currentstroke}%
\pgfsetdash{}{0pt}%
\pgfsys@defobject{currentmarker}{\pgfqpoint{-0.048611in}{0.000000in}}{\pgfqpoint{-0.000000in}{0.000000in}}{%
\pgfpathmoveto{\pgfqpoint{-0.000000in}{0.000000in}}%
\pgfpathlineto{\pgfqpoint{-0.048611in}{0.000000in}}%
\pgfusepath{stroke,fill}%
}%
\begin{pgfscope}%
\pgfsys@transformshift{0.750000in}{4.938280in}%
\pgfsys@useobject{currentmarker}{}%
\end{pgfscope}%
\end{pgfscope}%
\begin{pgfscope}%
\definecolor{textcolor}{rgb}{0.000000,0.000000,0.000000}%
\pgfsetstrokecolor{textcolor}%
\pgfsetfillcolor{textcolor}%
\pgftext[x=0.475308in, y=4.887180in, left, base]{\color{textcolor}\rmfamily\fontsize{10.000000}{12.000000}\selectfont \(\displaystyle {0.4}\)}%
\end{pgfscope}%
\begin{pgfscope}%
\pgfpathrectangle{\pgfqpoint{0.750000in}{3.960000in}}{\pgfqpoint{4.650000in}{3.080000in}}%
\pgfusepath{clip}%
\pgfsetrectcap%
\pgfsetroundjoin%
\pgfsetlinewidth{0.803000pt}%
\definecolor{currentstroke}{rgb}{0.690196,0.690196,0.690196}%
\pgfsetstrokecolor{currentstroke}%
\pgfsetdash{}{0pt}%
\pgfpathmoveto{\pgfqpoint{0.750000in}{5.427421in}}%
\pgfpathlineto{\pgfqpoint{5.400000in}{5.427421in}}%
\pgfusepath{stroke}%
\end{pgfscope}%
\begin{pgfscope}%
\pgfsetbuttcap%
\pgfsetroundjoin%
\definecolor{currentfill}{rgb}{0.000000,0.000000,0.000000}%
\pgfsetfillcolor{currentfill}%
\pgfsetlinewidth{0.803000pt}%
\definecolor{currentstroke}{rgb}{0.000000,0.000000,0.000000}%
\pgfsetstrokecolor{currentstroke}%
\pgfsetdash{}{0pt}%
\pgfsys@defobject{currentmarker}{\pgfqpoint{-0.048611in}{0.000000in}}{\pgfqpoint{-0.000000in}{0.000000in}}{%
\pgfpathmoveto{\pgfqpoint{-0.000000in}{0.000000in}}%
\pgfpathlineto{\pgfqpoint{-0.048611in}{0.000000in}}%
\pgfusepath{stroke,fill}%
}%
\begin{pgfscope}%
\pgfsys@transformshift{0.750000in}{5.427421in}%
\pgfsys@useobject{currentmarker}{}%
\end{pgfscope}%
\end{pgfscope}%
\begin{pgfscope}%
\definecolor{textcolor}{rgb}{0.000000,0.000000,0.000000}%
\pgfsetstrokecolor{textcolor}%
\pgfsetfillcolor{textcolor}%
\pgftext[x=0.475308in, y=5.376321in, left, base]{\color{textcolor}\rmfamily\fontsize{10.000000}{12.000000}\selectfont \(\displaystyle {0.6}\)}%
\end{pgfscope}%
\begin{pgfscope}%
\pgfpathrectangle{\pgfqpoint{0.750000in}{3.960000in}}{\pgfqpoint{4.650000in}{3.080000in}}%
\pgfusepath{clip}%
\pgfsetrectcap%
\pgfsetroundjoin%
\pgfsetlinewidth{0.803000pt}%
\definecolor{currentstroke}{rgb}{0.690196,0.690196,0.690196}%
\pgfsetstrokecolor{currentstroke}%
\pgfsetdash{}{0pt}%
\pgfpathmoveto{\pgfqpoint{0.750000in}{5.916561in}}%
\pgfpathlineto{\pgfqpoint{5.400000in}{5.916561in}}%
\pgfusepath{stroke}%
\end{pgfscope}%
\begin{pgfscope}%
\pgfsetbuttcap%
\pgfsetroundjoin%
\definecolor{currentfill}{rgb}{0.000000,0.000000,0.000000}%
\pgfsetfillcolor{currentfill}%
\pgfsetlinewidth{0.803000pt}%
\definecolor{currentstroke}{rgb}{0.000000,0.000000,0.000000}%
\pgfsetstrokecolor{currentstroke}%
\pgfsetdash{}{0pt}%
\pgfsys@defobject{currentmarker}{\pgfqpoint{-0.048611in}{0.000000in}}{\pgfqpoint{-0.000000in}{0.000000in}}{%
\pgfpathmoveto{\pgfqpoint{-0.000000in}{0.000000in}}%
\pgfpathlineto{\pgfqpoint{-0.048611in}{0.000000in}}%
\pgfusepath{stroke,fill}%
}%
\begin{pgfscope}%
\pgfsys@transformshift{0.750000in}{5.916561in}%
\pgfsys@useobject{currentmarker}{}%
\end{pgfscope}%
\end{pgfscope}%
\begin{pgfscope}%
\definecolor{textcolor}{rgb}{0.000000,0.000000,0.000000}%
\pgfsetstrokecolor{textcolor}%
\pgfsetfillcolor{textcolor}%
\pgftext[x=0.475308in, y=5.865461in, left, base]{\color{textcolor}\rmfamily\fontsize{10.000000}{12.000000}\selectfont \(\displaystyle {0.8}\)}%
\end{pgfscope}%
\begin{pgfscope}%
\pgfpathrectangle{\pgfqpoint{0.750000in}{3.960000in}}{\pgfqpoint{4.650000in}{3.080000in}}%
\pgfusepath{clip}%
\pgfsetrectcap%
\pgfsetroundjoin%
\pgfsetlinewidth{0.803000pt}%
\definecolor{currentstroke}{rgb}{0.690196,0.690196,0.690196}%
\pgfsetstrokecolor{currentstroke}%
\pgfsetdash{}{0pt}%
\pgfpathmoveto{\pgfqpoint{0.750000in}{6.405701in}}%
\pgfpathlineto{\pgfqpoint{5.400000in}{6.405701in}}%
\pgfusepath{stroke}%
\end{pgfscope}%
\begin{pgfscope}%
\pgfsetbuttcap%
\pgfsetroundjoin%
\definecolor{currentfill}{rgb}{0.000000,0.000000,0.000000}%
\pgfsetfillcolor{currentfill}%
\pgfsetlinewidth{0.803000pt}%
\definecolor{currentstroke}{rgb}{0.000000,0.000000,0.000000}%
\pgfsetstrokecolor{currentstroke}%
\pgfsetdash{}{0pt}%
\pgfsys@defobject{currentmarker}{\pgfqpoint{-0.048611in}{0.000000in}}{\pgfqpoint{-0.000000in}{0.000000in}}{%
\pgfpathmoveto{\pgfqpoint{-0.000000in}{0.000000in}}%
\pgfpathlineto{\pgfqpoint{-0.048611in}{0.000000in}}%
\pgfusepath{stroke,fill}%
}%
\begin{pgfscope}%
\pgfsys@transformshift{0.750000in}{6.405701in}%
\pgfsys@useobject{currentmarker}{}%
\end{pgfscope}%
\end{pgfscope}%
\begin{pgfscope}%
\definecolor{textcolor}{rgb}{0.000000,0.000000,0.000000}%
\pgfsetstrokecolor{textcolor}%
\pgfsetfillcolor{textcolor}%
\pgftext[x=0.475308in, y=6.354601in, left, base]{\color{textcolor}\rmfamily\fontsize{10.000000}{12.000000}\selectfont \(\displaystyle {1.0}\)}%
\end{pgfscope}%
\begin{pgfscope}%
\pgfpathrectangle{\pgfqpoint{0.750000in}{3.960000in}}{\pgfqpoint{4.650000in}{3.080000in}}%
\pgfusepath{clip}%
\pgfsetrectcap%
\pgfsetroundjoin%
\pgfsetlinewidth{0.803000pt}%
\definecolor{currentstroke}{rgb}{0.690196,0.690196,0.690196}%
\pgfsetstrokecolor{currentstroke}%
\pgfsetdash{}{0pt}%
\pgfpathmoveto{\pgfqpoint{0.750000in}{6.894841in}}%
\pgfpathlineto{\pgfqpoint{5.400000in}{6.894841in}}%
\pgfusepath{stroke}%
\end{pgfscope}%
\begin{pgfscope}%
\pgfsetbuttcap%
\pgfsetroundjoin%
\definecolor{currentfill}{rgb}{0.000000,0.000000,0.000000}%
\pgfsetfillcolor{currentfill}%
\pgfsetlinewidth{0.803000pt}%
\definecolor{currentstroke}{rgb}{0.000000,0.000000,0.000000}%
\pgfsetstrokecolor{currentstroke}%
\pgfsetdash{}{0pt}%
\pgfsys@defobject{currentmarker}{\pgfqpoint{-0.048611in}{0.000000in}}{\pgfqpoint{-0.000000in}{0.000000in}}{%
\pgfpathmoveto{\pgfqpoint{-0.000000in}{0.000000in}}%
\pgfpathlineto{\pgfqpoint{-0.048611in}{0.000000in}}%
\pgfusepath{stroke,fill}%
}%
\begin{pgfscope}%
\pgfsys@transformshift{0.750000in}{6.894841in}%
\pgfsys@useobject{currentmarker}{}%
\end{pgfscope}%
\end{pgfscope}%
\begin{pgfscope}%
\definecolor{textcolor}{rgb}{0.000000,0.000000,0.000000}%
\pgfsetstrokecolor{textcolor}%
\pgfsetfillcolor{textcolor}%
\pgftext[x=0.475308in, y=6.843741in, left, base]{\color{textcolor}\rmfamily\fontsize{10.000000}{12.000000}\selectfont \(\displaystyle {1.2}\)}%
\end{pgfscope}%
\begin{pgfscope}%
\definecolor{textcolor}{rgb}{0.000000,0.000000,0.000000}%
\pgfsetstrokecolor{textcolor}%
\pgfsetfillcolor{textcolor}%
\pgftext[x=0.419752in,y=5.500000in,,bottom,rotate=90.000000]{\color{textcolor}\rmfamily\fontsize{10.000000}{12.000000}\selectfont power in pu}%
\end{pgfscope}%
\begin{pgfscope}%
\pgfpathrectangle{\pgfqpoint{0.750000in}{3.960000in}}{\pgfqpoint{4.650000in}{3.080000in}}%
\pgfusepath{clip}%
\pgfsetrectcap%
\pgfsetroundjoin%
\pgfsetlinewidth{2.007500pt}%
\definecolor{currentstroke}{rgb}{0.121569,0.466667,0.705882}%
\pgfsetstrokecolor{currentstroke}%
\pgfsetdash{}{0pt}%
\pgfpathmoveto{\pgfqpoint{0.750000in}{3.960000in}}%
\pgfpathlineto{\pgfqpoint{0.844898in}{4.148036in}}%
\pgfpathlineto{\pgfqpoint{0.939796in}{4.335299in}}%
\pgfpathlineto{\pgfqpoint{1.034694in}{4.521020in}}%
\pgfpathlineto{\pgfqpoint{1.129592in}{4.704436in}}%
\pgfpathlineto{\pgfqpoint{1.224490in}{4.884793in}}%
\pgfpathlineto{\pgfqpoint{1.319388in}{5.061349in}}%
\pgfpathlineto{\pgfqpoint{1.414286in}{5.233380in}}%
\pgfpathlineto{\pgfqpoint{1.509184in}{5.400178in}}%
\pgfpathlineto{\pgfqpoint{1.604082in}{5.561058in}}%
\pgfpathlineto{\pgfqpoint{1.698980in}{5.715359in}}%
\pgfpathlineto{\pgfqpoint{1.793878in}{5.862447in}}%
\pgfpathlineto{\pgfqpoint{1.888776in}{6.001718in}}%
\pgfpathlineto{\pgfqpoint{1.983673in}{6.132598in}}%
\pgfpathlineto{\pgfqpoint{2.078571in}{6.254551in}}%
\pgfpathlineto{\pgfqpoint{2.173469in}{6.367075in}}%
\pgfpathlineto{\pgfqpoint{2.268367in}{6.469708in}}%
\pgfpathlineto{\pgfqpoint{2.363265in}{6.562028in}}%
\pgfpathlineto{\pgfqpoint{2.458163in}{6.643656in}}%
\pgfpathlineto{\pgfqpoint{2.553061in}{6.714256in}}%
\pgfpathlineto{\pgfqpoint{2.647959in}{6.773538in}}%
\pgfpathlineto{\pgfqpoint{2.742857in}{6.821259in}}%
\pgfpathlineto{\pgfqpoint{2.837755in}{6.857222in}}%
\pgfpathlineto{\pgfqpoint{2.932653in}{6.881280in}}%
\pgfpathlineto{\pgfqpoint{3.027551in}{6.893333in}}%
\pgfpathlineto{\pgfqpoint{3.122449in}{6.893333in}}%
\pgfpathlineto{\pgfqpoint{3.217347in}{6.881280in}}%
\pgfpathlineto{\pgfqpoint{3.312245in}{6.857222in}}%
\pgfpathlineto{\pgfqpoint{3.407143in}{6.821259in}}%
\pgfpathlineto{\pgfqpoint{3.502041in}{6.773538in}}%
\pgfpathlineto{\pgfqpoint{3.596939in}{6.714256in}}%
\pgfpathlineto{\pgfqpoint{3.691837in}{6.643656in}}%
\pgfpathlineto{\pgfqpoint{3.786735in}{6.562028in}}%
\pgfpathlineto{\pgfqpoint{3.881633in}{6.469708in}}%
\pgfpathlineto{\pgfqpoint{3.976531in}{6.367075in}}%
\pgfpathlineto{\pgfqpoint{4.071429in}{6.254551in}}%
\pgfpathlineto{\pgfqpoint{4.166327in}{6.132598in}}%
\pgfpathlineto{\pgfqpoint{4.261224in}{6.001718in}}%
\pgfpathlineto{\pgfqpoint{4.356122in}{5.862447in}}%
\pgfpathlineto{\pgfqpoint{4.451020in}{5.715359in}}%
\pgfpathlineto{\pgfqpoint{4.545918in}{5.561058in}}%
\pgfpathlineto{\pgfqpoint{4.640816in}{5.400178in}}%
\pgfpathlineto{\pgfqpoint{4.735714in}{5.233380in}}%
\pgfpathlineto{\pgfqpoint{4.830612in}{5.061349in}}%
\pgfpathlineto{\pgfqpoint{4.925510in}{4.884793in}}%
\pgfpathlineto{\pgfqpoint{5.020408in}{4.704436in}}%
\pgfpathlineto{\pgfqpoint{5.115306in}{4.521020in}}%
\pgfpathlineto{\pgfqpoint{5.210204in}{4.335299in}}%
\pgfpathlineto{\pgfqpoint{5.305102in}{4.148036in}}%
\pgfpathlineto{\pgfqpoint{5.400000in}{3.960000in}}%
\pgfusepath{stroke}%
\end{pgfscope}%
\begin{pgfscope}%
\pgfpathrectangle{\pgfqpoint{0.750000in}{3.960000in}}{\pgfqpoint{4.650000in}{3.080000in}}%
\pgfusepath{clip}%
\pgfsetrectcap%
\pgfsetroundjoin%
\pgfsetlinewidth{2.007500pt}%
\definecolor{currentstroke}{rgb}{1.000000,0.498039,0.054902}%
\pgfsetstrokecolor{currentstroke}%
\pgfsetdash{}{0pt}%
\pgfpathmoveto{\pgfqpoint{0.750000in}{3.960000in}}%
\pgfpathlineto{\pgfqpoint{0.844898in}{4.086553in}}%
\pgfpathlineto{\pgfqpoint{0.939796in}{4.212586in}}%
\pgfpathlineto{\pgfqpoint{1.034694in}{4.337580in}}%
\pgfpathlineto{\pgfqpoint{1.129592in}{4.461024in}}%
\pgfpathlineto{\pgfqpoint{1.224490in}{4.582408in}}%
\pgfpathlineto{\pgfqpoint{1.319388in}{4.701235in}}%
\pgfpathlineto{\pgfqpoint{1.414286in}{4.817016in}}%
\pgfpathlineto{\pgfqpoint{1.509184in}{4.929275in}}%
\pgfpathlineto{\pgfqpoint{1.604082in}{5.037552in}}%
\pgfpathlineto{\pgfqpoint{1.698980in}{5.141400in}}%
\pgfpathlineto{\pgfqpoint{1.793878in}{5.240394in}}%
\pgfpathlineto{\pgfqpoint{1.888776in}{5.334126in}}%
\pgfpathlineto{\pgfqpoint{1.983673in}{5.422212in}}%
\pgfpathlineto{\pgfqpoint{2.078571in}{5.504289in}}%
\pgfpathlineto{\pgfqpoint{2.173469in}{5.580021in}}%
\pgfpathlineto{\pgfqpoint{2.268367in}{5.649095in}}%
\pgfpathlineto{\pgfqpoint{2.363265in}{5.711229in}}%
\pgfpathlineto{\pgfqpoint{2.458163in}{5.766166in}}%
\pgfpathlineto{\pgfqpoint{2.553061in}{5.813682in}}%
\pgfpathlineto{\pgfqpoint{2.647959in}{5.853580in}}%
\pgfpathlineto{\pgfqpoint{2.742857in}{5.885697in}}%
\pgfpathlineto{\pgfqpoint{2.837755in}{5.909901in}}%
\pgfpathlineto{\pgfqpoint{2.932653in}{5.926093in}}%
\pgfpathlineto{\pgfqpoint{3.027551in}{5.934205in}}%
\pgfpathlineto{\pgfqpoint{3.122449in}{5.934205in}}%
\pgfpathlineto{\pgfqpoint{3.217347in}{5.926093in}}%
\pgfpathlineto{\pgfqpoint{3.312245in}{5.909901in}}%
\pgfpathlineto{\pgfqpoint{3.407143in}{5.885697in}}%
\pgfpathlineto{\pgfqpoint{3.502041in}{5.853580in}}%
\pgfpathlineto{\pgfqpoint{3.596939in}{5.813682in}}%
\pgfpathlineto{\pgfqpoint{3.691837in}{5.766166in}}%
\pgfpathlineto{\pgfqpoint{3.786735in}{5.711229in}}%
\pgfpathlineto{\pgfqpoint{3.881633in}{5.649095in}}%
\pgfpathlineto{\pgfqpoint{3.976531in}{5.580021in}}%
\pgfpathlineto{\pgfqpoint{4.071429in}{5.504289in}}%
\pgfpathlineto{\pgfqpoint{4.166327in}{5.422212in}}%
\pgfpathlineto{\pgfqpoint{4.261224in}{5.334126in}}%
\pgfpathlineto{\pgfqpoint{4.356122in}{5.240394in}}%
\pgfpathlineto{\pgfqpoint{4.451020in}{5.141400in}}%
\pgfpathlineto{\pgfqpoint{4.545918in}{5.037552in}}%
\pgfpathlineto{\pgfqpoint{4.640816in}{4.929275in}}%
\pgfpathlineto{\pgfqpoint{4.735714in}{4.817016in}}%
\pgfpathlineto{\pgfqpoint{4.830612in}{4.701235in}}%
\pgfpathlineto{\pgfqpoint{4.925510in}{4.582408in}}%
\pgfpathlineto{\pgfqpoint{5.020408in}{4.461024in}}%
\pgfpathlineto{\pgfqpoint{5.115306in}{4.337580in}}%
\pgfpathlineto{\pgfqpoint{5.210204in}{4.212586in}}%
\pgfpathlineto{\pgfqpoint{5.305102in}{4.086553in}}%
\pgfpathlineto{\pgfqpoint{5.400000in}{3.960000in}}%
\pgfusepath{stroke}%
\end{pgfscope}%
\begin{pgfscope}%
\pgfpathrectangle{\pgfqpoint{0.750000in}{3.960000in}}{\pgfqpoint{4.650000in}{3.080000in}}%
\pgfusepath{clip}%
\pgfsetrectcap%
\pgfsetroundjoin%
\pgfsetlinewidth{2.007500pt}%
\definecolor{currentstroke}{rgb}{0.172549,0.627451,0.172549}%
\pgfsetstrokecolor{currentstroke}%
\pgfsetdash{}{0pt}%
\pgfpathmoveto{\pgfqpoint{0.750000in}{6.161131in}}%
\pgfpathlineto{\pgfqpoint{0.844898in}{6.161131in}}%
\pgfpathlineto{\pgfqpoint{0.939796in}{6.161131in}}%
\pgfpathlineto{\pgfqpoint{1.034694in}{6.161131in}}%
\pgfpathlineto{\pgfqpoint{1.129592in}{6.161131in}}%
\pgfpathlineto{\pgfqpoint{1.224490in}{6.161131in}}%
\pgfpathlineto{\pgfqpoint{1.319388in}{6.161131in}}%
\pgfpathlineto{\pgfqpoint{1.414286in}{6.161131in}}%
\pgfpathlineto{\pgfqpoint{1.509184in}{6.161131in}}%
\pgfpathlineto{\pgfqpoint{1.604082in}{6.161131in}}%
\pgfpathlineto{\pgfqpoint{1.698980in}{6.161131in}}%
\pgfpathlineto{\pgfqpoint{1.793878in}{6.161131in}}%
\pgfpathlineto{\pgfqpoint{1.888776in}{6.161131in}}%
\pgfpathlineto{\pgfqpoint{1.983673in}{6.161131in}}%
\pgfpathlineto{\pgfqpoint{2.078571in}{6.161131in}}%
\pgfpathlineto{\pgfqpoint{2.173469in}{6.161131in}}%
\pgfpathlineto{\pgfqpoint{2.268367in}{6.161131in}}%
\pgfpathlineto{\pgfqpoint{2.363265in}{6.161131in}}%
\pgfpathlineto{\pgfqpoint{2.458163in}{6.161131in}}%
\pgfpathlineto{\pgfqpoint{2.553061in}{6.161131in}}%
\pgfpathlineto{\pgfqpoint{2.647959in}{6.161131in}}%
\pgfpathlineto{\pgfqpoint{2.742857in}{6.161131in}}%
\pgfpathlineto{\pgfqpoint{2.837755in}{6.161131in}}%
\pgfpathlineto{\pgfqpoint{2.932653in}{6.161131in}}%
\pgfpathlineto{\pgfqpoint{3.027551in}{6.161131in}}%
\pgfpathlineto{\pgfqpoint{3.122449in}{6.161131in}}%
\pgfpathlineto{\pgfqpoint{3.217347in}{6.161131in}}%
\pgfpathlineto{\pgfqpoint{3.312245in}{6.161131in}}%
\pgfpathlineto{\pgfqpoint{3.407143in}{6.161131in}}%
\pgfpathlineto{\pgfqpoint{3.502041in}{6.161131in}}%
\pgfpathlineto{\pgfqpoint{3.596939in}{6.161131in}}%
\pgfpathlineto{\pgfqpoint{3.691837in}{6.161131in}}%
\pgfpathlineto{\pgfqpoint{3.786735in}{6.161131in}}%
\pgfpathlineto{\pgfqpoint{3.881633in}{6.161131in}}%
\pgfpathlineto{\pgfqpoint{3.976531in}{6.161131in}}%
\pgfpathlineto{\pgfqpoint{4.071429in}{6.161131in}}%
\pgfpathlineto{\pgfqpoint{4.166327in}{6.161131in}}%
\pgfpathlineto{\pgfqpoint{4.261224in}{6.161131in}}%
\pgfpathlineto{\pgfqpoint{4.356122in}{6.161131in}}%
\pgfpathlineto{\pgfqpoint{4.451020in}{6.161131in}}%
\pgfpathlineto{\pgfqpoint{4.545918in}{6.161131in}}%
\pgfpathlineto{\pgfqpoint{4.640816in}{6.161131in}}%
\pgfpathlineto{\pgfqpoint{4.735714in}{6.161131in}}%
\pgfpathlineto{\pgfqpoint{4.830612in}{6.161131in}}%
\pgfpathlineto{\pgfqpoint{4.925510in}{6.161131in}}%
\pgfpathlineto{\pgfqpoint{5.020408in}{6.161131in}}%
\pgfpathlineto{\pgfqpoint{5.115306in}{6.161131in}}%
\pgfpathlineto{\pgfqpoint{5.210204in}{6.161131in}}%
\pgfpathlineto{\pgfqpoint{5.305102in}{6.161131in}}%
\pgfpathlineto{\pgfqpoint{5.400000in}{6.161131in}}%
\pgfusepath{stroke}%
\end{pgfscope}%
\begin{pgfscope}%
\pgfsetrectcap%
\pgfsetmiterjoin%
\pgfsetlinewidth{0.803000pt}%
\definecolor{currentstroke}{rgb}{0.000000,0.000000,0.000000}%
\pgfsetstrokecolor{currentstroke}%
\pgfsetdash{}{0pt}%
\pgfpathmoveto{\pgfqpoint{0.750000in}{3.960000in}}%
\pgfpathlineto{\pgfqpoint{0.750000in}{7.040000in}}%
\pgfusepath{stroke}%
\end{pgfscope}%
\begin{pgfscope}%
\pgfsetrectcap%
\pgfsetmiterjoin%
\pgfsetlinewidth{0.803000pt}%
\definecolor{currentstroke}{rgb}{0.000000,0.000000,0.000000}%
\pgfsetstrokecolor{currentstroke}%
\pgfsetdash{}{0pt}%
\pgfpathmoveto{\pgfqpoint{5.400000in}{3.960000in}}%
\pgfpathlineto{\pgfqpoint{5.400000in}{7.040000in}}%
\pgfusepath{stroke}%
\end{pgfscope}%
\begin{pgfscope}%
\pgfsetrectcap%
\pgfsetmiterjoin%
\pgfsetlinewidth{0.803000pt}%
\definecolor{currentstroke}{rgb}{0.000000,0.000000,0.000000}%
\pgfsetstrokecolor{currentstroke}%
\pgfsetdash{}{0pt}%
\pgfpathmoveto{\pgfqpoint{0.750000in}{3.960000in}}%
\pgfpathlineto{\pgfqpoint{5.400000in}{3.960000in}}%
\pgfusepath{stroke}%
\end{pgfscope}%
\begin{pgfscope}%
\pgfsetrectcap%
\pgfsetmiterjoin%
\pgfsetlinewidth{0.803000pt}%
\definecolor{currentstroke}{rgb}{0.000000,0.000000,0.000000}%
\pgfsetstrokecolor{currentstroke}%
\pgfsetdash{}{0pt}%
\pgfpathmoveto{\pgfqpoint{0.750000in}{7.040000in}}%
\pgfpathlineto{\pgfqpoint{5.400000in}{7.040000in}}%
\pgfusepath{stroke}%
\end{pgfscope}%
\begin{pgfscope}%
\pgfsetbuttcap%
\pgfsetmiterjoin%
\definecolor{currentfill}{rgb}{1.000000,1.000000,1.000000}%
\pgfsetfillcolor{currentfill}%
\pgfsetfillopacity{0.800000}%
\pgfsetlinewidth{1.003750pt}%
\definecolor{currentstroke}{rgb}{0.800000,0.800000,0.800000}%
\pgfsetstrokecolor{currentstroke}%
\pgfsetstrokeopacity{0.800000}%
\pgfsetdash{}{0pt}%
\pgfpathmoveto{\pgfqpoint{2.342004in}{4.029444in}}%
\pgfpathlineto{\pgfqpoint{3.807996in}{4.029444in}}%
\pgfpathquadraticcurveto{\pgfqpoint{3.835774in}{4.029444in}}{\pgfqpoint{3.835774in}{4.057222in}}%
\pgfpathlineto{\pgfqpoint{3.835774in}{4.651311in}}%
\pgfpathquadraticcurveto{\pgfqpoint{3.835774in}{4.679088in}}{\pgfqpoint{3.807996in}{4.679088in}}%
\pgfpathlineto{\pgfqpoint{2.342004in}{4.679088in}}%
\pgfpathquadraticcurveto{\pgfqpoint{2.314226in}{4.679088in}}{\pgfqpoint{2.314226in}{4.651311in}}%
\pgfpathlineto{\pgfqpoint{2.314226in}{4.057222in}}%
\pgfpathquadraticcurveto{\pgfqpoint{2.314226in}{4.029444in}}{\pgfqpoint{2.342004in}{4.029444in}}%
\pgfpathlineto{\pgfqpoint{2.342004in}{4.029444in}}%
\pgfpathclose%
\pgfusepath{stroke,fill}%
\end{pgfscope}%
\begin{pgfscope}%
\pgfsetrectcap%
\pgfsetroundjoin%
\pgfsetlinewidth{2.007500pt}%
\definecolor{currentstroke}{rgb}{0.121569,0.466667,0.705882}%
\pgfsetstrokecolor{currentstroke}%
\pgfsetdash{}{0pt}%
\pgfpathmoveto{\pgfqpoint{2.369781in}{4.568791in}}%
\pgfpathlineto{\pgfqpoint{2.508670in}{4.568791in}}%
\pgfpathlineto{\pgfqpoint{2.647559in}{4.568791in}}%
\pgfusepath{stroke}%
\end{pgfscope}%
\begin{pgfscope}%
\definecolor{textcolor}{rgb}{0.000000,0.000000,0.000000}%
\pgfsetstrokecolor{textcolor}%
\pgfsetfillcolor{textcolor}%
\pgftext[x=2.758670in,y=4.520180in,left,base]{\color{textcolor}\rmfamily\fontsize{10.000000}{12.000000}\selectfont \(\displaystyle P_\mathrm{e}\) pre-fault}%
\end{pgfscope}%
\begin{pgfscope}%
\pgfsetrectcap%
\pgfsetroundjoin%
\pgfsetlinewidth{2.007500pt}%
\definecolor{currentstroke}{rgb}{1.000000,0.498039,0.054902}%
\pgfsetstrokecolor{currentstroke}%
\pgfsetdash{}{0pt}%
\pgfpathmoveto{\pgfqpoint{2.369781in}{4.365748in}}%
\pgfpathlineto{\pgfqpoint{2.508670in}{4.365748in}}%
\pgfpathlineto{\pgfqpoint{2.647559in}{4.365748in}}%
\pgfusepath{stroke}%
\end{pgfscope}%
\begin{pgfscope}%
\definecolor{textcolor}{rgb}{0.000000,0.000000,0.000000}%
\pgfsetstrokecolor{textcolor}%
\pgfsetfillcolor{textcolor}%
\pgftext[x=2.758670in,y=4.317137in,left,base]{\color{textcolor}\rmfamily\fontsize{10.000000}{12.000000}\selectfont \(\displaystyle P_\mathrm{e}\) post-fault}%
\end{pgfscope}%
\begin{pgfscope}%
\pgfsetrectcap%
\pgfsetroundjoin%
\pgfsetlinewidth{2.007500pt}%
\definecolor{currentstroke}{rgb}{0.172549,0.627451,0.172549}%
\pgfsetstrokecolor{currentstroke}%
\pgfsetdash{}{0pt}%
\pgfpathmoveto{\pgfqpoint{2.369781in}{4.163857in}}%
\pgfpathlineto{\pgfqpoint{2.508670in}{4.163857in}}%
\pgfpathlineto{\pgfqpoint{2.647559in}{4.163857in}}%
\pgfusepath{stroke}%
\end{pgfscope}%
\begin{pgfscope}%
\definecolor{textcolor}{rgb}{0.000000,0.000000,0.000000}%
\pgfsetstrokecolor{textcolor}%
\pgfsetfillcolor{textcolor}%
\pgftext[x=2.758670in,y=4.115246in,left,base]{\color{textcolor}\rmfamily\fontsize{10.000000}{12.000000}\selectfont \(\displaystyle P_\mathrm{T}\) of the turbine}%
\end{pgfscope}%
\begin{pgfscope}%
\pgfsetbuttcap%
\pgfsetmiterjoin%
\definecolor{currentfill}{rgb}{1.000000,1.000000,1.000000}%
\pgfsetfillcolor{currentfill}%
\pgfsetlinewidth{0.000000pt}%
\definecolor{currentstroke}{rgb}{0.000000,0.000000,0.000000}%
\pgfsetstrokecolor{currentstroke}%
\pgfsetstrokeopacity{0.000000}%
\pgfsetdash{}{0pt}%
\pgfpathmoveto{\pgfqpoint{0.750000in}{0.880000in}}%
\pgfpathlineto{\pgfqpoint{5.400000in}{0.880000in}}%
\pgfpathlineto{\pgfqpoint{5.400000in}{3.960000in}}%
\pgfpathlineto{\pgfqpoint{0.750000in}{3.960000in}}%
\pgfpathlineto{\pgfqpoint{0.750000in}{0.880000in}}%
\pgfpathclose%
\pgfusepath{fill}%
\end{pgfscope}%
\begin{pgfscope}%
\pgfpathrectangle{\pgfqpoint{0.750000in}{0.880000in}}{\pgfqpoint{4.650000in}{3.080000in}}%
\pgfusepath{clip}%
\pgfsetrectcap%
\pgfsetroundjoin%
\pgfsetlinewidth{0.803000pt}%
\definecolor{currentstroke}{rgb}{0.690196,0.690196,0.690196}%
\pgfsetstrokecolor{currentstroke}%
\pgfsetdash{}{0pt}%
\pgfpathmoveto{\pgfqpoint{0.750000in}{0.880000in}}%
\pgfpathlineto{\pgfqpoint{0.750000in}{3.960000in}}%
\pgfusepath{stroke}%
\end{pgfscope}%
\begin{pgfscope}%
\pgfsetbuttcap%
\pgfsetroundjoin%
\definecolor{currentfill}{rgb}{0.000000,0.000000,0.000000}%
\pgfsetfillcolor{currentfill}%
\pgfsetlinewidth{0.803000pt}%
\definecolor{currentstroke}{rgb}{0.000000,0.000000,0.000000}%
\pgfsetstrokecolor{currentstroke}%
\pgfsetdash{}{0pt}%
\pgfsys@defobject{currentmarker}{\pgfqpoint{0.000000in}{-0.048611in}}{\pgfqpoint{0.000000in}{0.000000in}}{%
\pgfpathmoveto{\pgfqpoint{0.000000in}{0.000000in}}%
\pgfpathlineto{\pgfqpoint{0.000000in}{-0.048611in}}%
\pgfusepath{stroke,fill}%
}%
\begin{pgfscope}%
\pgfsys@transformshift{0.750000in}{0.880000in}%
\pgfsys@useobject{currentmarker}{}%
\end{pgfscope}%
\end{pgfscope}%
\begin{pgfscope}%
\definecolor{textcolor}{rgb}{0.000000,0.000000,0.000000}%
\pgfsetstrokecolor{textcolor}%
\pgfsetfillcolor{textcolor}%
\pgftext[x=0.750000in,y=0.782778in,,top]{\color{textcolor}\rmfamily\fontsize{10.000000}{12.000000}\selectfont \(\displaystyle {0}\)}%
\end{pgfscope}%
\begin{pgfscope}%
\pgfpathrectangle{\pgfqpoint{0.750000in}{0.880000in}}{\pgfqpoint{4.650000in}{3.080000in}}%
\pgfusepath{clip}%
\pgfsetrectcap%
\pgfsetroundjoin%
\pgfsetlinewidth{0.803000pt}%
\definecolor{currentstroke}{rgb}{0.690196,0.690196,0.690196}%
\pgfsetstrokecolor{currentstroke}%
\pgfsetdash{}{0pt}%
\pgfpathmoveto{\pgfqpoint{1.266667in}{0.880000in}}%
\pgfpathlineto{\pgfqpoint{1.266667in}{3.960000in}}%
\pgfusepath{stroke}%
\end{pgfscope}%
\begin{pgfscope}%
\pgfsetbuttcap%
\pgfsetroundjoin%
\definecolor{currentfill}{rgb}{0.000000,0.000000,0.000000}%
\pgfsetfillcolor{currentfill}%
\pgfsetlinewidth{0.803000pt}%
\definecolor{currentstroke}{rgb}{0.000000,0.000000,0.000000}%
\pgfsetstrokecolor{currentstroke}%
\pgfsetdash{}{0pt}%
\pgfsys@defobject{currentmarker}{\pgfqpoint{0.000000in}{-0.048611in}}{\pgfqpoint{0.000000in}{0.000000in}}{%
\pgfpathmoveto{\pgfqpoint{0.000000in}{0.000000in}}%
\pgfpathlineto{\pgfqpoint{0.000000in}{-0.048611in}}%
\pgfusepath{stroke,fill}%
}%
\begin{pgfscope}%
\pgfsys@transformshift{1.266667in}{0.880000in}%
\pgfsys@useobject{currentmarker}{}%
\end{pgfscope}%
\end{pgfscope}%
\begin{pgfscope}%
\definecolor{textcolor}{rgb}{0.000000,0.000000,0.000000}%
\pgfsetstrokecolor{textcolor}%
\pgfsetfillcolor{textcolor}%
\pgftext[x=1.266667in,y=0.782778in,,top]{\color{textcolor}\rmfamily\fontsize{10.000000}{12.000000}\selectfont \(\displaystyle {20}\)}%
\end{pgfscope}%
\begin{pgfscope}%
\pgfpathrectangle{\pgfqpoint{0.750000in}{0.880000in}}{\pgfqpoint{4.650000in}{3.080000in}}%
\pgfusepath{clip}%
\pgfsetrectcap%
\pgfsetroundjoin%
\pgfsetlinewidth{0.803000pt}%
\definecolor{currentstroke}{rgb}{0.690196,0.690196,0.690196}%
\pgfsetstrokecolor{currentstroke}%
\pgfsetdash{}{0pt}%
\pgfpathmoveto{\pgfqpoint{1.783333in}{0.880000in}}%
\pgfpathlineto{\pgfqpoint{1.783333in}{3.960000in}}%
\pgfusepath{stroke}%
\end{pgfscope}%
\begin{pgfscope}%
\pgfsetbuttcap%
\pgfsetroundjoin%
\definecolor{currentfill}{rgb}{0.000000,0.000000,0.000000}%
\pgfsetfillcolor{currentfill}%
\pgfsetlinewidth{0.803000pt}%
\definecolor{currentstroke}{rgb}{0.000000,0.000000,0.000000}%
\pgfsetstrokecolor{currentstroke}%
\pgfsetdash{}{0pt}%
\pgfsys@defobject{currentmarker}{\pgfqpoint{0.000000in}{-0.048611in}}{\pgfqpoint{0.000000in}{0.000000in}}{%
\pgfpathmoveto{\pgfqpoint{0.000000in}{0.000000in}}%
\pgfpathlineto{\pgfqpoint{0.000000in}{-0.048611in}}%
\pgfusepath{stroke,fill}%
}%
\begin{pgfscope}%
\pgfsys@transformshift{1.783333in}{0.880000in}%
\pgfsys@useobject{currentmarker}{}%
\end{pgfscope}%
\end{pgfscope}%
\begin{pgfscope}%
\definecolor{textcolor}{rgb}{0.000000,0.000000,0.000000}%
\pgfsetstrokecolor{textcolor}%
\pgfsetfillcolor{textcolor}%
\pgftext[x=1.783333in,y=0.782778in,,top]{\color{textcolor}\rmfamily\fontsize{10.000000}{12.000000}\selectfont \(\displaystyle {40}\)}%
\end{pgfscope}%
\begin{pgfscope}%
\pgfpathrectangle{\pgfqpoint{0.750000in}{0.880000in}}{\pgfqpoint{4.650000in}{3.080000in}}%
\pgfusepath{clip}%
\pgfsetrectcap%
\pgfsetroundjoin%
\pgfsetlinewidth{0.803000pt}%
\definecolor{currentstroke}{rgb}{0.690196,0.690196,0.690196}%
\pgfsetstrokecolor{currentstroke}%
\pgfsetdash{}{0pt}%
\pgfpathmoveto{\pgfqpoint{2.300000in}{0.880000in}}%
\pgfpathlineto{\pgfqpoint{2.300000in}{3.960000in}}%
\pgfusepath{stroke}%
\end{pgfscope}%
\begin{pgfscope}%
\pgfsetbuttcap%
\pgfsetroundjoin%
\definecolor{currentfill}{rgb}{0.000000,0.000000,0.000000}%
\pgfsetfillcolor{currentfill}%
\pgfsetlinewidth{0.803000pt}%
\definecolor{currentstroke}{rgb}{0.000000,0.000000,0.000000}%
\pgfsetstrokecolor{currentstroke}%
\pgfsetdash{}{0pt}%
\pgfsys@defobject{currentmarker}{\pgfqpoint{0.000000in}{-0.048611in}}{\pgfqpoint{0.000000in}{0.000000in}}{%
\pgfpathmoveto{\pgfqpoint{0.000000in}{0.000000in}}%
\pgfpathlineto{\pgfqpoint{0.000000in}{-0.048611in}}%
\pgfusepath{stroke,fill}%
}%
\begin{pgfscope}%
\pgfsys@transformshift{2.300000in}{0.880000in}%
\pgfsys@useobject{currentmarker}{}%
\end{pgfscope}%
\end{pgfscope}%
\begin{pgfscope}%
\definecolor{textcolor}{rgb}{0.000000,0.000000,0.000000}%
\pgfsetstrokecolor{textcolor}%
\pgfsetfillcolor{textcolor}%
\pgftext[x=2.300000in,y=0.782778in,,top]{\color{textcolor}\rmfamily\fontsize{10.000000}{12.000000}\selectfont \(\displaystyle {60}\)}%
\end{pgfscope}%
\begin{pgfscope}%
\pgfpathrectangle{\pgfqpoint{0.750000in}{0.880000in}}{\pgfqpoint{4.650000in}{3.080000in}}%
\pgfusepath{clip}%
\pgfsetrectcap%
\pgfsetroundjoin%
\pgfsetlinewidth{0.803000pt}%
\definecolor{currentstroke}{rgb}{0.690196,0.690196,0.690196}%
\pgfsetstrokecolor{currentstroke}%
\pgfsetdash{}{0pt}%
\pgfpathmoveto{\pgfqpoint{2.816667in}{0.880000in}}%
\pgfpathlineto{\pgfqpoint{2.816667in}{3.960000in}}%
\pgfusepath{stroke}%
\end{pgfscope}%
\begin{pgfscope}%
\pgfsetbuttcap%
\pgfsetroundjoin%
\definecolor{currentfill}{rgb}{0.000000,0.000000,0.000000}%
\pgfsetfillcolor{currentfill}%
\pgfsetlinewidth{0.803000pt}%
\definecolor{currentstroke}{rgb}{0.000000,0.000000,0.000000}%
\pgfsetstrokecolor{currentstroke}%
\pgfsetdash{}{0pt}%
\pgfsys@defobject{currentmarker}{\pgfqpoint{0.000000in}{-0.048611in}}{\pgfqpoint{0.000000in}{0.000000in}}{%
\pgfpathmoveto{\pgfqpoint{0.000000in}{0.000000in}}%
\pgfpathlineto{\pgfqpoint{0.000000in}{-0.048611in}}%
\pgfusepath{stroke,fill}%
}%
\begin{pgfscope}%
\pgfsys@transformshift{2.816667in}{0.880000in}%
\pgfsys@useobject{currentmarker}{}%
\end{pgfscope}%
\end{pgfscope}%
\begin{pgfscope}%
\definecolor{textcolor}{rgb}{0.000000,0.000000,0.000000}%
\pgfsetstrokecolor{textcolor}%
\pgfsetfillcolor{textcolor}%
\pgftext[x=2.816667in,y=0.782778in,,top]{\color{textcolor}\rmfamily\fontsize{10.000000}{12.000000}\selectfont \(\displaystyle {80}\)}%
\end{pgfscope}%
\begin{pgfscope}%
\pgfpathrectangle{\pgfqpoint{0.750000in}{0.880000in}}{\pgfqpoint{4.650000in}{3.080000in}}%
\pgfusepath{clip}%
\pgfsetrectcap%
\pgfsetroundjoin%
\pgfsetlinewidth{0.803000pt}%
\definecolor{currentstroke}{rgb}{0.690196,0.690196,0.690196}%
\pgfsetstrokecolor{currentstroke}%
\pgfsetdash{}{0pt}%
\pgfpathmoveto{\pgfqpoint{3.333333in}{0.880000in}}%
\pgfpathlineto{\pgfqpoint{3.333333in}{3.960000in}}%
\pgfusepath{stroke}%
\end{pgfscope}%
\begin{pgfscope}%
\pgfsetbuttcap%
\pgfsetroundjoin%
\definecolor{currentfill}{rgb}{0.000000,0.000000,0.000000}%
\pgfsetfillcolor{currentfill}%
\pgfsetlinewidth{0.803000pt}%
\definecolor{currentstroke}{rgb}{0.000000,0.000000,0.000000}%
\pgfsetstrokecolor{currentstroke}%
\pgfsetdash{}{0pt}%
\pgfsys@defobject{currentmarker}{\pgfqpoint{0.000000in}{-0.048611in}}{\pgfqpoint{0.000000in}{0.000000in}}{%
\pgfpathmoveto{\pgfqpoint{0.000000in}{0.000000in}}%
\pgfpathlineto{\pgfqpoint{0.000000in}{-0.048611in}}%
\pgfusepath{stroke,fill}%
}%
\begin{pgfscope}%
\pgfsys@transformshift{3.333333in}{0.880000in}%
\pgfsys@useobject{currentmarker}{}%
\end{pgfscope}%
\end{pgfscope}%
\begin{pgfscope}%
\definecolor{textcolor}{rgb}{0.000000,0.000000,0.000000}%
\pgfsetstrokecolor{textcolor}%
\pgfsetfillcolor{textcolor}%
\pgftext[x=3.333333in,y=0.782778in,,top]{\color{textcolor}\rmfamily\fontsize{10.000000}{12.000000}\selectfont \(\displaystyle {100}\)}%
\end{pgfscope}%
\begin{pgfscope}%
\pgfpathrectangle{\pgfqpoint{0.750000in}{0.880000in}}{\pgfqpoint{4.650000in}{3.080000in}}%
\pgfusepath{clip}%
\pgfsetrectcap%
\pgfsetroundjoin%
\pgfsetlinewidth{0.803000pt}%
\definecolor{currentstroke}{rgb}{0.690196,0.690196,0.690196}%
\pgfsetstrokecolor{currentstroke}%
\pgfsetdash{}{0pt}%
\pgfpathmoveto{\pgfqpoint{3.850000in}{0.880000in}}%
\pgfpathlineto{\pgfqpoint{3.850000in}{3.960000in}}%
\pgfusepath{stroke}%
\end{pgfscope}%
\begin{pgfscope}%
\pgfsetbuttcap%
\pgfsetroundjoin%
\definecolor{currentfill}{rgb}{0.000000,0.000000,0.000000}%
\pgfsetfillcolor{currentfill}%
\pgfsetlinewidth{0.803000pt}%
\definecolor{currentstroke}{rgb}{0.000000,0.000000,0.000000}%
\pgfsetstrokecolor{currentstroke}%
\pgfsetdash{}{0pt}%
\pgfsys@defobject{currentmarker}{\pgfqpoint{0.000000in}{-0.048611in}}{\pgfqpoint{0.000000in}{0.000000in}}{%
\pgfpathmoveto{\pgfqpoint{0.000000in}{0.000000in}}%
\pgfpathlineto{\pgfqpoint{0.000000in}{-0.048611in}}%
\pgfusepath{stroke,fill}%
}%
\begin{pgfscope}%
\pgfsys@transformshift{3.850000in}{0.880000in}%
\pgfsys@useobject{currentmarker}{}%
\end{pgfscope}%
\end{pgfscope}%
\begin{pgfscope}%
\definecolor{textcolor}{rgb}{0.000000,0.000000,0.000000}%
\pgfsetstrokecolor{textcolor}%
\pgfsetfillcolor{textcolor}%
\pgftext[x=3.850000in,y=0.782778in,,top]{\color{textcolor}\rmfamily\fontsize{10.000000}{12.000000}\selectfont \(\displaystyle {120}\)}%
\end{pgfscope}%
\begin{pgfscope}%
\pgfpathrectangle{\pgfqpoint{0.750000in}{0.880000in}}{\pgfqpoint{4.650000in}{3.080000in}}%
\pgfusepath{clip}%
\pgfsetrectcap%
\pgfsetroundjoin%
\pgfsetlinewidth{0.803000pt}%
\definecolor{currentstroke}{rgb}{0.690196,0.690196,0.690196}%
\pgfsetstrokecolor{currentstroke}%
\pgfsetdash{}{0pt}%
\pgfpathmoveto{\pgfqpoint{4.366667in}{0.880000in}}%
\pgfpathlineto{\pgfqpoint{4.366667in}{3.960000in}}%
\pgfusepath{stroke}%
\end{pgfscope}%
\begin{pgfscope}%
\pgfsetbuttcap%
\pgfsetroundjoin%
\definecolor{currentfill}{rgb}{0.000000,0.000000,0.000000}%
\pgfsetfillcolor{currentfill}%
\pgfsetlinewidth{0.803000pt}%
\definecolor{currentstroke}{rgb}{0.000000,0.000000,0.000000}%
\pgfsetstrokecolor{currentstroke}%
\pgfsetdash{}{0pt}%
\pgfsys@defobject{currentmarker}{\pgfqpoint{0.000000in}{-0.048611in}}{\pgfqpoint{0.000000in}{0.000000in}}{%
\pgfpathmoveto{\pgfqpoint{0.000000in}{0.000000in}}%
\pgfpathlineto{\pgfqpoint{0.000000in}{-0.048611in}}%
\pgfusepath{stroke,fill}%
}%
\begin{pgfscope}%
\pgfsys@transformshift{4.366667in}{0.880000in}%
\pgfsys@useobject{currentmarker}{}%
\end{pgfscope}%
\end{pgfscope}%
\begin{pgfscope}%
\definecolor{textcolor}{rgb}{0.000000,0.000000,0.000000}%
\pgfsetstrokecolor{textcolor}%
\pgfsetfillcolor{textcolor}%
\pgftext[x=4.366667in,y=0.782778in,,top]{\color{textcolor}\rmfamily\fontsize{10.000000}{12.000000}\selectfont \(\displaystyle {140}\)}%
\end{pgfscope}%
\begin{pgfscope}%
\pgfpathrectangle{\pgfqpoint{0.750000in}{0.880000in}}{\pgfqpoint{4.650000in}{3.080000in}}%
\pgfusepath{clip}%
\pgfsetrectcap%
\pgfsetroundjoin%
\pgfsetlinewidth{0.803000pt}%
\definecolor{currentstroke}{rgb}{0.690196,0.690196,0.690196}%
\pgfsetstrokecolor{currentstroke}%
\pgfsetdash{}{0pt}%
\pgfpathmoveto{\pgfqpoint{4.883333in}{0.880000in}}%
\pgfpathlineto{\pgfqpoint{4.883333in}{3.960000in}}%
\pgfusepath{stroke}%
\end{pgfscope}%
\begin{pgfscope}%
\pgfsetbuttcap%
\pgfsetroundjoin%
\definecolor{currentfill}{rgb}{0.000000,0.000000,0.000000}%
\pgfsetfillcolor{currentfill}%
\pgfsetlinewidth{0.803000pt}%
\definecolor{currentstroke}{rgb}{0.000000,0.000000,0.000000}%
\pgfsetstrokecolor{currentstroke}%
\pgfsetdash{}{0pt}%
\pgfsys@defobject{currentmarker}{\pgfqpoint{0.000000in}{-0.048611in}}{\pgfqpoint{0.000000in}{0.000000in}}{%
\pgfpathmoveto{\pgfqpoint{0.000000in}{0.000000in}}%
\pgfpathlineto{\pgfqpoint{0.000000in}{-0.048611in}}%
\pgfusepath{stroke,fill}%
}%
\begin{pgfscope}%
\pgfsys@transformshift{4.883333in}{0.880000in}%
\pgfsys@useobject{currentmarker}{}%
\end{pgfscope}%
\end{pgfscope}%
\begin{pgfscope}%
\definecolor{textcolor}{rgb}{0.000000,0.000000,0.000000}%
\pgfsetstrokecolor{textcolor}%
\pgfsetfillcolor{textcolor}%
\pgftext[x=4.883333in,y=0.782778in,,top]{\color{textcolor}\rmfamily\fontsize{10.000000}{12.000000}\selectfont \(\displaystyle {160}\)}%
\end{pgfscope}%
\begin{pgfscope}%
\pgfpathrectangle{\pgfqpoint{0.750000in}{0.880000in}}{\pgfqpoint{4.650000in}{3.080000in}}%
\pgfusepath{clip}%
\pgfsetrectcap%
\pgfsetroundjoin%
\pgfsetlinewidth{0.803000pt}%
\definecolor{currentstroke}{rgb}{0.690196,0.690196,0.690196}%
\pgfsetstrokecolor{currentstroke}%
\pgfsetdash{}{0pt}%
\pgfpathmoveto{\pgfqpoint{5.400000in}{0.880000in}}%
\pgfpathlineto{\pgfqpoint{5.400000in}{3.960000in}}%
\pgfusepath{stroke}%
\end{pgfscope}%
\begin{pgfscope}%
\pgfsetbuttcap%
\pgfsetroundjoin%
\definecolor{currentfill}{rgb}{0.000000,0.000000,0.000000}%
\pgfsetfillcolor{currentfill}%
\pgfsetlinewidth{0.803000pt}%
\definecolor{currentstroke}{rgb}{0.000000,0.000000,0.000000}%
\pgfsetstrokecolor{currentstroke}%
\pgfsetdash{}{0pt}%
\pgfsys@defobject{currentmarker}{\pgfqpoint{0.000000in}{-0.048611in}}{\pgfqpoint{0.000000in}{0.000000in}}{%
\pgfpathmoveto{\pgfqpoint{0.000000in}{0.000000in}}%
\pgfpathlineto{\pgfqpoint{0.000000in}{-0.048611in}}%
\pgfusepath{stroke,fill}%
}%
\begin{pgfscope}%
\pgfsys@transformshift{5.400000in}{0.880000in}%
\pgfsys@useobject{currentmarker}{}%
\end{pgfscope}%
\end{pgfscope}%
\begin{pgfscope}%
\definecolor{textcolor}{rgb}{0.000000,0.000000,0.000000}%
\pgfsetstrokecolor{textcolor}%
\pgfsetfillcolor{textcolor}%
\pgftext[x=5.400000in,y=0.782778in,,top]{\color{textcolor}\rmfamily\fontsize{10.000000}{12.000000}\selectfont \(\displaystyle {180}\)}%
\end{pgfscope}%
\begin{pgfscope}%
\definecolor{textcolor}{rgb}{0.000000,0.000000,0.000000}%
\pgfsetstrokecolor{textcolor}%
\pgfsetfillcolor{textcolor}%
\pgftext[x=3.075000in,y=0.594776in,,top]{\color{textcolor}\rmfamily\fontsize{10.000000}{12.000000}\selectfont power angle \(\displaystyle \delta\) in deg}%
\end{pgfscope}%
\begin{pgfscope}%
\pgfpathrectangle{\pgfqpoint{0.750000in}{0.880000in}}{\pgfqpoint{4.650000in}{3.080000in}}%
\pgfusepath{clip}%
\pgfsetrectcap%
\pgfsetroundjoin%
\pgfsetlinewidth{0.803000pt}%
\definecolor{currentstroke}{rgb}{0.690196,0.690196,0.690196}%
\pgfsetstrokecolor{currentstroke}%
\pgfsetdash{}{0pt}%
\pgfpathmoveto{\pgfqpoint{0.750000in}{3.823047in}}%
\pgfpathlineto{\pgfqpoint{5.400000in}{3.823047in}}%
\pgfusepath{stroke}%
\end{pgfscope}%
\begin{pgfscope}%
\pgfsetbuttcap%
\pgfsetroundjoin%
\definecolor{currentfill}{rgb}{0.000000,0.000000,0.000000}%
\pgfsetfillcolor{currentfill}%
\pgfsetlinewidth{0.803000pt}%
\definecolor{currentstroke}{rgb}{0.000000,0.000000,0.000000}%
\pgfsetstrokecolor{currentstroke}%
\pgfsetdash{}{0pt}%
\pgfsys@defobject{currentmarker}{\pgfqpoint{-0.048611in}{0.000000in}}{\pgfqpoint{-0.000000in}{0.000000in}}{%
\pgfpathmoveto{\pgfqpoint{-0.000000in}{0.000000in}}%
\pgfpathlineto{\pgfqpoint{-0.048611in}{0.000000in}}%
\pgfusepath{stroke,fill}%
}%
\begin{pgfscope}%
\pgfsys@transformshift{0.750000in}{3.823047in}%
\pgfsys@useobject{currentmarker}{}%
\end{pgfscope}%
\end{pgfscope}%
\begin{pgfscope}%
\definecolor{textcolor}{rgb}{0.000000,0.000000,0.000000}%
\pgfsetstrokecolor{textcolor}%
\pgfsetfillcolor{textcolor}%
\pgftext[x=0.405863in, y=3.771947in, left, base]{\color{textcolor}\rmfamily\fontsize{10.000000}{12.000000}\selectfont \(\displaystyle {0.00}\)}%
\end{pgfscope}%
\begin{pgfscope}%
\pgfpathrectangle{\pgfqpoint{0.750000in}{0.880000in}}{\pgfqpoint{4.650000in}{3.080000in}}%
\pgfusepath{clip}%
\pgfsetrectcap%
\pgfsetroundjoin%
\pgfsetlinewidth{0.803000pt}%
\definecolor{currentstroke}{rgb}{0.690196,0.690196,0.690196}%
\pgfsetstrokecolor{currentstroke}%
\pgfsetdash{}{0pt}%
\pgfpathmoveto{\pgfqpoint{0.750000in}{3.480665in}}%
\pgfpathlineto{\pgfqpoint{5.400000in}{3.480665in}}%
\pgfusepath{stroke}%
\end{pgfscope}%
\begin{pgfscope}%
\pgfsetbuttcap%
\pgfsetroundjoin%
\definecolor{currentfill}{rgb}{0.000000,0.000000,0.000000}%
\pgfsetfillcolor{currentfill}%
\pgfsetlinewidth{0.803000pt}%
\definecolor{currentstroke}{rgb}{0.000000,0.000000,0.000000}%
\pgfsetstrokecolor{currentstroke}%
\pgfsetdash{}{0pt}%
\pgfsys@defobject{currentmarker}{\pgfqpoint{-0.048611in}{0.000000in}}{\pgfqpoint{-0.000000in}{0.000000in}}{%
\pgfpathmoveto{\pgfqpoint{-0.000000in}{0.000000in}}%
\pgfpathlineto{\pgfqpoint{-0.048611in}{0.000000in}}%
\pgfusepath{stroke,fill}%
}%
\begin{pgfscope}%
\pgfsys@transformshift{0.750000in}{3.480665in}%
\pgfsys@useobject{currentmarker}{}%
\end{pgfscope}%
\end{pgfscope}%
\begin{pgfscope}%
\definecolor{textcolor}{rgb}{0.000000,0.000000,0.000000}%
\pgfsetstrokecolor{textcolor}%
\pgfsetfillcolor{textcolor}%
\pgftext[x=0.405863in, y=3.429565in, left, base]{\color{textcolor}\rmfamily\fontsize{10.000000}{12.000000}\selectfont \(\displaystyle {0.25}\)}%
\end{pgfscope}%
\begin{pgfscope}%
\pgfpathrectangle{\pgfqpoint{0.750000in}{0.880000in}}{\pgfqpoint{4.650000in}{3.080000in}}%
\pgfusepath{clip}%
\pgfsetrectcap%
\pgfsetroundjoin%
\pgfsetlinewidth{0.803000pt}%
\definecolor{currentstroke}{rgb}{0.690196,0.690196,0.690196}%
\pgfsetstrokecolor{currentstroke}%
\pgfsetdash{}{0pt}%
\pgfpathmoveto{\pgfqpoint{0.750000in}{3.138283in}}%
\pgfpathlineto{\pgfqpoint{5.400000in}{3.138283in}}%
\pgfusepath{stroke}%
\end{pgfscope}%
\begin{pgfscope}%
\pgfsetbuttcap%
\pgfsetroundjoin%
\definecolor{currentfill}{rgb}{0.000000,0.000000,0.000000}%
\pgfsetfillcolor{currentfill}%
\pgfsetlinewidth{0.803000pt}%
\definecolor{currentstroke}{rgb}{0.000000,0.000000,0.000000}%
\pgfsetstrokecolor{currentstroke}%
\pgfsetdash{}{0pt}%
\pgfsys@defobject{currentmarker}{\pgfqpoint{-0.048611in}{0.000000in}}{\pgfqpoint{-0.000000in}{0.000000in}}{%
\pgfpathmoveto{\pgfqpoint{-0.000000in}{0.000000in}}%
\pgfpathlineto{\pgfqpoint{-0.048611in}{0.000000in}}%
\pgfusepath{stroke,fill}%
}%
\begin{pgfscope}%
\pgfsys@transformshift{0.750000in}{3.138283in}%
\pgfsys@useobject{currentmarker}{}%
\end{pgfscope}%
\end{pgfscope}%
\begin{pgfscope}%
\definecolor{textcolor}{rgb}{0.000000,0.000000,0.000000}%
\pgfsetstrokecolor{textcolor}%
\pgfsetfillcolor{textcolor}%
\pgftext[x=0.405863in, y=3.087183in, left, base]{\color{textcolor}\rmfamily\fontsize{10.000000}{12.000000}\selectfont \(\displaystyle {0.50}\)}%
\end{pgfscope}%
\begin{pgfscope}%
\pgfpathrectangle{\pgfqpoint{0.750000in}{0.880000in}}{\pgfqpoint{4.650000in}{3.080000in}}%
\pgfusepath{clip}%
\pgfsetrectcap%
\pgfsetroundjoin%
\pgfsetlinewidth{0.803000pt}%
\definecolor{currentstroke}{rgb}{0.690196,0.690196,0.690196}%
\pgfsetstrokecolor{currentstroke}%
\pgfsetdash{}{0pt}%
\pgfpathmoveto{\pgfqpoint{0.750000in}{2.795901in}}%
\pgfpathlineto{\pgfqpoint{5.400000in}{2.795901in}}%
\pgfusepath{stroke}%
\end{pgfscope}%
\begin{pgfscope}%
\pgfsetbuttcap%
\pgfsetroundjoin%
\definecolor{currentfill}{rgb}{0.000000,0.000000,0.000000}%
\pgfsetfillcolor{currentfill}%
\pgfsetlinewidth{0.803000pt}%
\definecolor{currentstroke}{rgb}{0.000000,0.000000,0.000000}%
\pgfsetstrokecolor{currentstroke}%
\pgfsetdash{}{0pt}%
\pgfsys@defobject{currentmarker}{\pgfqpoint{-0.048611in}{0.000000in}}{\pgfqpoint{-0.000000in}{0.000000in}}{%
\pgfpathmoveto{\pgfqpoint{-0.000000in}{0.000000in}}%
\pgfpathlineto{\pgfqpoint{-0.048611in}{0.000000in}}%
\pgfusepath{stroke,fill}%
}%
\begin{pgfscope}%
\pgfsys@transformshift{0.750000in}{2.795901in}%
\pgfsys@useobject{currentmarker}{}%
\end{pgfscope}%
\end{pgfscope}%
\begin{pgfscope}%
\definecolor{textcolor}{rgb}{0.000000,0.000000,0.000000}%
\pgfsetstrokecolor{textcolor}%
\pgfsetfillcolor{textcolor}%
\pgftext[x=0.405863in, y=2.744801in, left, base]{\color{textcolor}\rmfamily\fontsize{10.000000}{12.000000}\selectfont \(\displaystyle {0.75}\)}%
\end{pgfscope}%
\begin{pgfscope}%
\pgfpathrectangle{\pgfqpoint{0.750000in}{0.880000in}}{\pgfqpoint{4.650000in}{3.080000in}}%
\pgfusepath{clip}%
\pgfsetrectcap%
\pgfsetroundjoin%
\pgfsetlinewidth{0.803000pt}%
\definecolor{currentstroke}{rgb}{0.690196,0.690196,0.690196}%
\pgfsetstrokecolor{currentstroke}%
\pgfsetdash{}{0pt}%
\pgfpathmoveto{\pgfqpoint{0.750000in}{2.453519in}}%
\pgfpathlineto{\pgfqpoint{5.400000in}{2.453519in}}%
\pgfusepath{stroke}%
\end{pgfscope}%
\begin{pgfscope}%
\pgfsetbuttcap%
\pgfsetroundjoin%
\definecolor{currentfill}{rgb}{0.000000,0.000000,0.000000}%
\pgfsetfillcolor{currentfill}%
\pgfsetlinewidth{0.803000pt}%
\definecolor{currentstroke}{rgb}{0.000000,0.000000,0.000000}%
\pgfsetstrokecolor{currentstroke}%
\pgfsetdash{}{0pt}%
\pgfsys@defobject{currentmarker}{\pgfqpoint{-0.048611in}{0.000000in}}{\pgfqpoint{-0.000000in}{0.000000in}}{%
\pgfpathmoveto{\pgfqpoint{-0.000000in}{0.000000in}}%
\pgfpathlineto{\pgfqpoint{-0.048611in}{0.000000in}}%
\pgfusepath{stroke,fill}%
}%
\begin{pgfscope}%
\pgfsys@transformshift{0.750000in}{2.453519in}%
\pgfsys@useobject{currentmarker}{}%
\end{pgfscope}%
\end{pgfscope}%
\begin{pgfscope}%
\definecolor{textcolor}{rgb}{0.000000,0.000000,0.000000}%
\pgfsetstrokecolor{textcolor}%
\pgfsetfillcolor{textcolor}%
\pgftext[x=0.405863in, y=2.402419in, left, base]{\color{textcolor}\rmfamily\fontsize{10.000000}{12.000000}\selectfont \(\displaystyle {1.00}\)}%
\end{pgfscope}%
\begin{pgfscope}%
\pgfpathrectangle{\pgfqpoint{0.750000in}{0.880000in}}{\pgfqpoint{4.650000in}{3.080000in}}%
\pgfusepath{clip}%
\pgfsetrectcap%
\pgfsetroundjoin%
\pgfsetlinewidth{0.803000pt}%
\definecolor{currentstroke}{rgb}{0.690196,0.690196,0.690196}%
\pgfsetstrokecolor{currentstroke}%
\pgfsetdash{}{0pt}%
\pgfpathmoveto{\pgfqpoint{0.750000in}{2.111137in}}%
\pgfpathlineto{\pgfqpoint{5.400000in}{2.111137in}}%
\pgfusepath{stroke}%
\end{pgfscope}%
\begin{pgfscope}%
\pgfsetbuttcap%
\pgfsetroundjoin%
\definecolor{currentfill}{rgb}{0.000000,0.000000,0.000000}%
\pgfsetfillcolor{currentfill}%
\pgfsetlinewidth{0.803000pt}%
\definecolor{currentstroke}{rgb}{0.000000,0.000000,0.000000}%
\pgfsetstrokecolor{currentstroke}%
\pgfsetdash{}{0pt}%
\pgfsys@defobject{currentmarker}{\pgfqpoint{-0.048611in}{0.000000in}}{\pgfqpoint{-0.000000in}{0.000000in}}{%
\pgfpathmoveto{\pgfqpoint{-0.000000in}{0.000000in}}%
\pgfpathlineto{\pgfqpoint{-0.048611in}{0.000000in}}%
\pgfusepath{stroke,fill}%
}%
\begin{pgfscope}%
\pgfsys@transformshift{0.750000in}{2.111137in}%
\pgfsys@useobject{currentmarker}{}%
\end{pgfscope}%
\end{pgfscope}%
\begin{pgfscope}%
\definecolor{textcolor}{rgb}{0.000000,0.000000,0.000000}%
\pgfsetstrokecolor{textcolor}%
\pgfsetfillcolor{textcolor}%
\pgftext[x=0.405863in, y=2.060037in, left, base]{\color{textcolor}\rmfamily\fontsize{10.000000}{12.000000}\selectfont \(\displaystyle {1.25}\)}%
\end{pgfscope}%
\begin{pgfscope}%
\pgfpathrectangle{\pgfqpoint{0.750000in}{0.880000in}}{\pgfqpoint{4.650000in}{3.080000in}}%
\pgfusepath{clip}%
\pgfsetrectcap%
\pgfsetroundjoin%
\pgfsetlinewidth{0.803000pt}%
\definecolor{currentstroke}{rgb}{0.690196,0.690196,0.690196}%
\pgfsetstrokecolor{currentstroke}%
\pgfsetdash{}{0pt}%
\pgfpathmoveto{\pgfqpoint{0.750000in}{1.768755in}}%
\pgfpathlineto{\pgfqpoint{5.400000in}{1.768755in}}%
\pgfusepath{stroke}%
\end{pgfscope}%
\begin{pgfscope}%
\pgfsetbuttcap%
\pgfsetroundjoin%
\definecolor{currentfill}{rgb}{0.000000,0.000000,0.000000}%
\pgfsetfillcolor{currentfill}%
\pgfsetlinewidth{0.803000pt}%
\definecolor{currentstroke}{rgb}{0.000000,0.000000,0.000000}%
\pgfsetstrokecolor{currentstroke}%
\pgfsetdash{}{0pt}%
\pgfsys@defobject{currentmarker}{\pgfqpoint{-0.048611in}{0.000000in}}{\pgfqpoint{-0.000000in}{0.000000in}}{%
\pgfpathmoveto{\pgfqpoint{-0.000000in}{0.000000in}}%
\pgfpathlineto{\pgfqpoint{-0.048611in}{0.000000in}}%
\pgfusepath{stroke,fill}%
}%
\begin{pgfscope}%
\pgfsys@transformshift{0.750000in}{1.768755in}%
\pgfsys@useobject{currentmarker}{}%
\end{pgfscope}%
\end{pgfscope}%
\begin{pgfscope}%
\definecolor{textcolor}{rgb}{0.000000,0.000000,0.000000}%
\pgfsetstrokecolor{textcolor}%
\pgfsetfillcolor{textcolor}%
\pgftext[x=0.405863in, y=1.717655in, left, base]{\color{textcolor}\rmfamily\fontsize{10.000000}{12.000000}\selectfont \(\displaystyle {1.50}\)}%
\end{pgfscope}%
\begin{pgfscope}%
\pgfpathrectangle{\pgfqpoint{0.750000in}{0.880000in}}{\pgfqpoint{4.650000in}{3.080000in}}%
\pgfusepath{clip}%
\pgfsetrectcap%
\pgfsetroundjoin%
\pgfsetlinewidth{0.803000pt}%
\definecolor{currentstroke}{rgb}{0.690196,0.690196,0.690196}%
\pgfsetstrokecolor{currentstroke}%
\pgfsetdash{}{0pt}%
\pgfpathmoveto{\pgfqpoint{0.750000in}{1.426373in}}%
\pgfpathlineto{\pgfqpoint{5.400000in}{1.426373in}}%
\pgfusepath{stroke}%
\end{pgfscope}%
\begin{pgfscope}%
\pgfsetbuttcap%
\pgfsetroundjoin%
\definecolor{currentfill}{rgb}{0.000000,0.000000,0.000000}%
\pgfsetfillcolor{currentfill}%
\pgfsetlinewidth{0.803000pt}%
\definecolor{currentstroke}{rgb}{0.000000,0.000000,0.000000}%
\pgfsetstrokecolor{currentstroke}%
\pgfsetdash{}{0pt}%
\pgfsys@defobject{currentmarker}{\pgfqpoint{-0.048611in}{0.000000in}}{\pgfqpoint{-0.000000in}{0.000000in}}{%
\pgfpathmoveto{\pgfqpoint{-0.000000in}{0.000000in}}%
\pgfpathlineto{\pgfqpoint{-0.048611in}{0.000000in}}%
\pgfusepath{stroke,fill}%
}%
\begin{pgfscope}%
\pgfsys@transformshift{0.750000in}{1.426373in}%
\pgfsys@useobject{currentmarker}{}%
\end{pgfscope}%
\end{pgfscope}%
\begin{pgfscope}%
\definecolor{textcolor}{rgb}{0.000000,0.000000,0.000000}%
\pgfsetstrokecolor{textcolor}%
\pgfsetfillcolor{textcolor}%
\pgftext[x=0.405863in, y=1.375273in, left, base]{\color{textcolor}\rmfamily\fontsize{10.000000}{12.000000}\selectfont \(\displaystyle {1.75}\)}%
\end{pgfscope}%
\begin{pgfscope}%
\pgfpathrectangle{\pgfqpoint{0.750000in}{0.880000in}}{\pgfqpoint{4.650000in}{3.080000in}}%
\pgfusepath{clip}%
\pgfsetrectcap%
\pgfsetroundjoin%
\pgfsetlinewidth{0.803000pt}%
\definecolor{currentstroke}{rgb}{0.690196,0.690196,0.690196}%
\pgfsetstrokecolor{currentstroke}%
\pgfsetdash{}{0pt}%
\pgfpathmoveto{\pgfqpoint{0.750000in}{1.083991in}}%
\pgfpathlineto{\pgfqpoint{5.400000in}{1.083991in}}%
\pgfusepath{stroke}%
\end{pgfscope}%
\begin{pgfscope}%
\pgfsetbuttcap%
\pgfsetroundjoin%
\definecolor{currentfill}{rgb}{0.000000,0.000000,0.000000}%
\pgfsetfillcolor{currentfill}%
\pgfsetlinewidth{0.803000pt}%
\definecolor{currentstroke}{rgb}{0.000000,0.000000,0.000000}%
\pgfsetstrokecolor{currentstroke}%
\pgfsetdash{}{0pt}%
\pgfsys@defobject{currentmarker}{\pgfqpoint{-0.048611in}{0.000000in}}{\pgfqpoint{-0.000000in}{0.000000in}}{%
\pgfpathmoveto{\pgfqpoint{-0.000000in}{0.000000in}}%
\pgfpathlineto{\pgfqpoint{-0.048611in}{0.000000in}}%
\pgfusepath{stroke,fill}%
}%
\begin{pgfscope}%
\pgfsys@transformshift{0.750000in}{1.083991in}%
\pgfsys@useobject{currentmarker}{}%
\end{pgfscope}%
\end{pgfscope}%
\begin{pgfscope}%
\definecolor{textcolor}{rgb}{0.000000,0.000000,0.000000}%
\pgfsetstrokecolor{textcolor}%
\pgfsetfillcolor{textcolor}%
\pgftext[x=0.405863in, y=1.032891in, left, base]{\color{textcolor}\rmfamily\fontsize{10.000000}{12.000000}\selectfont \(\displaystyle {2.00}\)}%
\end{pgfscope}%
\begin{pgfscope}%
\definecolor{textcolor}{rgb}{0.000000,0.000000,0.000000}%
\pgfsetstrokecolor{textcolor}%
\pgfsetfillcolor{textcolor}%
\pgftext[x=0.350308in,y=2.420000in,,bottom,rotate=90.000000]{\color{textcolor}\rmfamily\fontsize{10.000000}{12.000000}\selectfont time in s}%
\end{pgfscope}%
\begin{pgfscope}%
\pgfpathrectangle{\pgfqpoint{0.750000in}{0.880000in}}{\pgfqpoint{4.650000in}{3.080000in}}%
\pgfusepath{clip}%
\pgfsetrectcap%
\pgfsetroundjoin%
\pgfsetlinewidth{1.505625pt}%
\definecolor{currentstroke}{rgb}{0.121569,0.466667,0.705882}%
\pgfsetstrokecolor{currentstroke}%
\pgfsetdash{}{0pt}%
\pgfpathmoveto{\pgfqpoint{2.005414in}{3.970000in}}%
\pgfpathlineto{\pgfqpoint{2.006519in}{3.809352in}}%
\pgfpathlineto{\pgfqpoint{2.009623in}{3.795657in}}%
\pgfpathlineto{\pgfqpoint{2.015412in}{3.780592in}}%
\pgfpathlineto{\pgfqpoint{2.023668in}{3.765527in}}%
\pgfpathlineto{\pgfqpoint{2.034361in}{3.750462in}}%
\pgfpathlineto{\pgfqpoint{2.048764in}{3.734028in}}%
\pgfpathlineto{\pgfqpoint{2.067528in}{3.716224in}}%
\pgfpathlineto{\pgfqpoint{2.089498in}{3.698420in}}%
\pgfpathlineto{\pgfqpoint{2.116634in}{3.679247in}}%
\pgfpathlineto{\pgfqpoint{2.149563in}{3.658704in}}%
\pgfpathlineto{\pgfqpoint{2.188887in}{3.636791in}}%
\pgfpathlineto{\pgfqpoint{2.235164in}{3.613509in}}%
\pgfpathlineto{\pgfqpoint{2.292014in}{3.587488in}}%
\pgfpathlineto{\pgfqpoint{2.357254in}{3.560098in}}%
\pgfpathlineto{\pgfqpoint{2.434824in}{3.529968in}}%
\pgfpathlineto{\pgfqpoint{2.525639in}{3.497100in}}%
\pgfpathlineto{\pgfqpoint{2.634561in}{3.460122in}}%
\pgfpathlineto{\pgfqpoint{2.762787in}{3.419036in}}%
\pgfpathlineto{\pgfqpoint{2.911296in}{3.373842in}}%
\pgfpathlineto{\pgfqpoint{3.081087in}{3.324539in}}%
\pgfpathlineto{\pgfqpoint{3.405465in}{3.231411in}}%
\pgfpathlineto{\pgfqpoint{3.483216in}{3.205390in}}%
\pgfpathlineto{\pgfqpoint{3.554302in}{3.179369in}}%
\pgfpathlineto{\pgfqpoint{3.619171in}{3.153348in}}%
\pgfpathlineto{\pgfqpoint{3.678291in}{3.127327in}}%
\pgfpathlineto{\pgfqpoint{3.732135in}{3.101306in}}%
\pgfpathlineto{\pgfqpoint{3.783632in}{3.073915in}}%
\pgfpathlineto{\pgfqpoint{3.830341in}{3.046525in}}%
\pgfpathlineto{\pgfqpoint{3.872783in}{3.019134in}}%
\pgfpathlineto{\pgfqpoint{3.913295in}{2.990374in}}%
\pgfpathlineto{\pgfqpoint{3.951871in}{2.960245in}}%
\pgfpathlineto{\pgfqpoint{3.988608in}{2.928745in}}%
\pgfpathlineto{\pgfqpoint{4.025109in}{2.894507in}}%
\pgfpathlineto{\pgfqpoint{4.062685in}{2.856160in}}%
\pgfpathlineto{\pgfqpoint{4.105056in}{2.809596in}}%
\pgfpathlineto{\pgfqpoint{4.252689in}{2.643884in}}%
\pgfpathlineto{\pgfqpoint{4.289179in}{2.608276in}}%
\pgfpathlineto{\pgfqpoint{4.324560in}{2.576777in}}%
\pgfpathlineto{\pgfqpoint{4.360096in}{2.548017in}}%
\pgfpathlineto{\pgfqpoint{4.397435in}{2.520626in}}%
\pgfpathlineto{\pgfqpoint{4.436666in}{2.494605in}}%
\pgfpathlineto{\pgfqpoint{4.477787in}{2.469954in}}%
\pgfpathlineto{\pgfqpoint{4.523369in}{2.445302in}}%
\pgfpathlineto{\pgfqpoint{4.571129in}{2.422020in}}%
\pgfpathlineto{\pgfqpoint{4.624127in}{2.398738in}}%
\pgfpathlineto{\pgfqpoint{4.679452in}{2.376826in}}%
\pgfpathlineto{\pgfqpoint{4.740766in}{2.354913in}}%
\pgfpathlineto{\pgfqpoint{4.808858in}{2.333001in}}%
\pgfpathlineto{\pgfqpoint{4.884622in}{2.311088in}}%
\pgfpathlineto{\pgfqpoint{4.963518in}{2.290545in}}%
\pgfpathlineto{\pgfqpoint{5.050996in}{2.270002in}}%
\pgfpathlineto{\pgfqpoint{5.148121in}{2.249460in}}%
\pgfpathlineto{\pgfqpoint{5.256098in}{2.228917in}}%
\pgfpathlineto{\pgfqpoint{5.376281in}{2.208374in}}%
\pgfpathlineto{\pgfqpoint{5.410000in}{2.202987in}}%
\pgfpathlineto{\pgfqpoint{5.410000in}{2.202987in}}%
\pgfusepath{stroke}%
\end{pgfscope}%
\begin{pgfscope}%
\pgfpathrectangle{\pgfqpoint{0.750000in}{0.880000in}}{\pgfqpoint{4.650000in}{3.080000in}}%
\pgfusepath{clip}%
\pgfsetbuttcap%
\pgfsetroundjoin%
\pgfsetlinewidth{1.505625pt}%
\definecolor{currentstroke}{rgb}{0.121569,0.466667,0.705882}%
\pgfsetstrokecolor{currentstroke}%
\pgfsetdash{{5.550000pt}{2.400000pt}}{0.000000pt}%
\pgfpathmoveto{\pgfqpoint{0.750000in}{3.283453in}}%
\pgfpathlineto{\pgfqpoint{5.400000in}{3.283453in}}%
\pgfusepath{stroke}%
\end{pgfscope}%
\begin{pgfscope}%
\pgfsetrectcap%
\pgfsetmiterjoin%
\pgfsetlinewidth{0.803000pt}%
\definecolor{currentstroke}{rgb}{0.000000,0.000000,0.000000}%
\pgfsetstrokecolor{currentstroke}%
\pgfsetdash{}{0pt}%
\pgfpathmoveto{\pgfqpoint{0.750000in}{0.880000in}}%
\pgfpathlineto{\pgfqpoint{0.750000in}{3.960000in}}%
\pgfusepath{stroke}%
\end{pgfscope}%
\begin{pgfscope}%
\pgfsetrectcap%
\pgfsetmiterjoin%
\pgfsetlinewidth{0.803000pt}%
\definecolor{currentstroke}{rgb}{0.000000,0.000000,0.000000}%
\pgfsetstrokecolor{currentstroke}%
\pgfsetdash{}{0pt}%
\pgfpathmoveto{\pgfqpoint{5.400000in}{0.880000in}}%
\pgfpathlineto{\pgfqpoint{5.400000in}{3.960000in}}%
\pgfusepath{stroke}%
\end{pgfscope}%
\begin{pgfscope}%
\pgfsetrectcap%
\pgfsetmiterjoin%
\pgfsetlinewidth{0.803000pt}%
\definecolor{currentstroke}{rgb}{0.000000,0.000000,0.000000}%
\pgfsetstrokecolor{currentstroke}%
\pgfsetdash{}{0pt}%
\pgfpathmoveto{\pgfqpoint{0.750000in}{0.880000in}}%
\pgfpathlineto{\pgfqpoint{5.400000in}{0.880000in}}%
\pgfusepath{stroke}%
\end{pgfscope}%
\begin{pgfscope}%
\pgfsetrectcap%
\pgfsetmiterjoin%
\pgfsetlinewidth{0.803000pt}%
\definecolor{currentstroke}{rgb}{0.000000,0.000000,0.000000}%
\pgfsetstrokecolor{currentstroke}%
\pgfsetdash{}{0pt}%
\pgfpathmoveto{\pgfqpoint{0.750000in}{3.960000in}}%
\pgfpathlineto{\pgfqpoint{5.400000in}{3.960000in}}%
\pgfusepath{stroke}%
\end{pgfscope}%
\begin{pgfscope}%
\pgfsetbuttcap%
\pgfsetmiterjoin%
\definecolor{currentfill}{rgb}{1.000000,1.000000,1.000000}%
\pgfsetfillcolor{currentfill}%
\pgfsetfillopacity{0.800000}%
\pgfsetlinewidth{1.003750pt}%
\definecolor{currentstroke}{rgb}{0.800000,0.800000,0.800000}%
\pgfsetstrokecolor{currentstroke}%
\pgfsetstrokeopacity{0.800000}%
\pgfsetdash{}{0pt}%
\pgfpathmoveto{\pgfqpoint{3.899064in}{3.443955in}}%
\pgfpathlineto{\pgfqpoint{5.302778in}{3.443955in}}%
\pgfpathquadraticcurveto{\pgfqpoint{5.330556in}{3.443955in}}{\pgfqpoint{5.330556in}{3.471733in}}%
\pgfpathlineto{\pgfqpoint{5.330556in}{3.862778in}}%
\pgfpathquadraticcurveto{\pgfqpoint{5.330556in}{3.890556in}}{\pgfqpoint{5.302778in}{3.890556in}}%
\pgfpathlineto{\pgfqpoint{3.899064in}{3.890556in}}%
\pgfpathquadraticcurveto{\pgfqpoint{3.871286in}{3.890556in}}{\pgfqpoint{3.871286in}{3.862778in}}%
\pgfpathlineto{\pgfqpoint{3.871286in}{3.471733in}}%
\pgfpathquadraticcurveto{\pgfqpoint{3.871286in}{3.443955in}}{\pgfqpoint{3.899064in}{3.443955in}}%
\pgfpathlineto{\pgfqpoint{3.899064in}{3.443955in}}%
\pgfpathclose%
\pgfusepath{stroke,fill}%
\end{pgfscope}%
\begin{pgfscope}%
\pgfsetrectcap%
\pgfsetroundjoin%
\pgfsetlinewidth{1.505625pt}%
\definecolor{currentstroke}{rgb}{0.121569,0.466667,0.705882}%
\pgfsetstrokecolor{currentstroke}%
\pgfsetdash{}{0pt}%
\pgfpathmoveto{\pgfqpoint{3.926842in}{3.781411in}}%
\pgfpathlineto{\pgfqpoint{4.065731in}{3.781411in}}%
\pgfpathlineto{\pgfqpoint{4.204620in}{3.781411in}}%
\pgfusepath{stroke}%
\end{pgfscope}%
\begin{pgfscope}%
\definecolor{textcolor}{rgb}{0.000000,0.000000,0.000000}%
\pgfsetstrokecolor{textcolor}%
\pgfsetfillcolor{textcolor}%
\pgftext[x=4.315731in,y=3.732800in,left,base]{\color{textcolor}\rmfamily\fontsize{10.000000}{12.000000}\selectfont delta}%
\end{pgfscope}%
\begin{pgfscope}%
\pgfsetbuttcap%
\pgfsetroundjoin%
\pgfsetlinewidth{1.505625pt}%
\definecolor{currentstroke}{rgb}{0.121569,0.466667,0.705882}%
\pgfsetstrokecolor{currentstroke}%
\pgfsetdash{{5.550000pt}{2.400000pt}}{0.000000pt}%
\pgfpathmoveto{\pgfqpoint{3.926842in}{3.578368in}}%
\pgfpathlineto{\pgfqpoint{4.065731in}{3.578368in}}%
\pgfpathlineto{\pgfqpoint{4.204620in}{3.578368in}}%
\pgfusepath{stroke}%
\end{pgfscope}%
\begin{pgfscope}%
\definecolor{textcolor}{rgb}{0.000000,0.000000,0.000000}%
\pgfsetstrokecolor{textcolor}%
\pgfsetfillcolor{textcolor}%
\pgftext[x=4.315731in,y=3.529757in,left,base]{\color{textcolor}\rmfamily\fontsize{10.000000}{12.000000}\selectfont clearing of fault}%
\end{pgfscope}%
\begin{pgfscope}%
\definecolor{textcolor}{rgb}{0.000000,0.000000,0.000000}%
\pgfsetstrokecolor{textcolor}%
\pgfsetfillcolor{textcolor}%
\pgftext[x=3.000000in,y=7.840000in,,top]{\color{textcolor}\rmfamily\fontsize{12.000000}{14.400000}\selectfont Unstable scenario - fault 2}%
\end{pgfscope}%
\end{pgfpicture}%
\makeatother%
\endgroup%


%% Creator: Matplotlib, PGF backend
%%
%% To include the figure in your LaTeX document, write
%%   \input{<filename>.pgf}
%%
%% Make sure the required packages are loaded in your preamble
%%   \usepackage{pgf}
%%
%% Also ensure that all the required font packages are loaded; for instance,
%% the lmodern package is sometimes necessary when using math font.
%%   \usepackage{lmodern}
%%
%% Figures using additional raster images can only be included by \input if
%% they are in the same directory as the main LaTeX file. For loading figures
%% from other directories you can use the `import` package
%%   \usepackage{import}
%%
%% and then include the figures with
%%   \import{<path to file>}{<filename>.pgf}
%%
%% Matplotlib used the following preamble
%%   
%%   \usepackage{fontspec}
%%   \setmainfont{Charter.ttc}[Path=\detokenize{/System/Library/Fonts/Supplemental/}]
%%   \setsansfont{DejaVuSans.ttf}[Path=\detokenize{/opt/homebrew/lib/python3.10/site-packages/matplotlib/mpl-data/fonts/ttf/}]
%%   \setmonofont{DejaVuSansMono.ttf}[Path=\detokenize{/opt/homebrew/lib/python3.10/site-packages/matplotlib/mpl-data/fonts/ttf/}]
%%   \makeatletter\@ifpackageloaded{underscore}{}{\usepackage[strings]{underscore}}\makeatother
%%
\begingroup%
\makeatletter%
\begin{pgfpicture}%
\pgfpathrectangle{\pgfpointorigin}{\pgfqpoint{6.400000in}{4.800000in}}%
\pgfusepath{use as bounding box, clip}%
\begin{pgfscope}%
\pgfsetbuttcap%
\pgfsetmiterjoin%
\definecolor{currentfill}{rgb}{1.000000,1.000000,1.000000}%
\pgfsetfillcolor{currentfill}%
\pgfsetlinewidth{0.000000pt}%
\definecolor{currentstroke}{rgb}{1.000000,1.000000,1.000000}%
\pgfsetstrokecolor{currentstroke}%
\pgfsetdash{}{0pt}%
\pgfpathmoveto{\pgfqpoint{0.000000in}{0.000000in}}%
\pgfpathlineto{\pgfqpoint{6.400000in}{0.000000in}}%
\pgfpathlineto{\pgfqpoint{6.400000in}{4.800000in}}%
\pgfpathlineto{\pgfqpoint{0.000000in}{4.800000in}}%
\pgfpathlineto{\pgfqpoint{0.000000in}{0.000000in}}%
\pgfpathclose%
\pgfusepath{fill}%
\end{pgfscope}%
\begin{pgfscope}%
\pgfsetbuttcap%
\pgfsetmiterjoin%
\definecolor{currentfill}{rgb}{1.000000,1.000000,1.000000}%
\pgfsetfillcolor{currentfill}%
\pgfsetlinewidth{0.000000pt}%
\definecolor{currentstroke}{rgb}{0.000000,0.000000,0.000000}%
\pgfsetstrokecolor{currentstroke}%
\pgfsetstrokeopacity{0.000000}%
\pgfsetdash{}{0pt}%
\pgfpathmoveto{\pgfqpoint{0.800000in}{0.528000in}}%
\pgfpathlineto{\pgfqpoint{5.760000in}{0.528000in}}%
\pgfpathlineto{\pgfqpoint{5.760000in}{4.224000in}}%
\pgfpathlineto{\pgfqpoint{0.800000in}{4.224000in}}%
\pgfpathlineto{\pgfqpoint{0.800000in}{0.528000in}}%
\pgfpathclose%
\pgfusepath{fill}%
\end{pgfscope}%
\begin{pgfscope}%
\pgfpathrectangle{\pgfqpoint{0.800000in}{0.528000in}}{\pgfqpoint{4.960000in}{3.696000in}}%
\pgfusepath{clip}%
\pgfsetrectcap%
\pgfsetroundjoin%
\pgfsetlinewidth{0.803000pt}%
\definecolor{currentstroke}{rgb}{0.690196,0.690196,0.690196}%
\pgfsetstrokecolor{currentstroke}%
\pgfsetdash{}{0pt}%
\pgfpathmoveto{\pgfqpoint{1.043857in}{0.528000in}}%
\pgfpathlineto{\pgfqpoint{1.043857in}{4.224000in}}%
\pgfusepath{stroke}%
\end{pgfscope}%
\begin{pgfscope}%
\pgfsetbuttcap%
\pgfsetroundjoin%
\definecolor{currentfill}{rgb}{0.000000,0.000000,0.000000}%
\pgfsetfillcolor{currentfill}%
\pgfsetlinewidth{0.803000pt}%
\definecolor{currentstroke}{rgb}{0.000000,0.000000,0.000000}%
\pgfsetstrokecolor{currentstroke}%
\pgfsetdash{}{0pt}%
\pgfsys@defobject{currentmarker}{\pgfqpoint{0.000000in}{-0.048611in}}{\pgfqpoint{0.000000in}{0.000000in}}{%
\pgfpathmoveto{\pgfqpoint{0.000000in}{0.000000in}}%
\pgfpathlineto{\pgfqpoint{0.000000in}{-0.048611in}}%
\pgfusepath{stroke,fill}%
}%
\begin{pgfscope}%
\pgfsys@transformshift{1.043857in}{0.528000in}%
\pgfsys@useobject{currentmarker}{}%
\end{pgfscope}%
\end{pgfscope}%
\begin{pgfscope}%
\definecolor{textcolor}{rgb}{0.000000,0.000000,0.000000}%
\pgfsetstrokecolor{textcolor}%
\pgfsetfillcolor{textcolor}%
\pgftext[x=1.043857in,y=0.430778in,,top]{\color{textcolor}\rmfamily\fontsize{10.000000}{12.000000}\selectfont \(\displaystyle {\ensuremath{-}1.0}\)}%
\end{pgfscope}%
\begin{pgfscope}%
\pgfpathrectangle{\pgfqpoint{0.800000in}{0.528000in}}{\pgfqpoint{4.960000in}{3.696000in}}%
\pgfusepath{clip}%
\pgfsetrectcap%
\pgfsetroundjoin%
\pgfsetlinewidth{0.803000pt}%
\definecolor{currentstroke}{rgb}{0.690196,0.690196,0.690196}%
\pgfsetstrokecolor{currentstroke}%
\pgfsetdash{}{0pt}%
\pgfpathmoveto{\pgfqpoint{1.856985in}{0.528000in}}%
\pgfpathlineto{\pgfqpoint{1.856985in}{4.224000in}}%
\pgfusepath{stroke}%
\end{pgfscope}%
\begin{pgfscope}%
\pgfsetbuttcap%
\pgfsetroundjoin%
\definecolor{currentfill}{rgb}{0.000000,0.000000,0.000000}%
\pgfsetfillcolor{currentfill}%
\pgfsetlinewidth{0.803000pt}%
\definecolor{currentstroke}{rgb}{0.000000,0.000000,0.000000}%
\pgfsetstrokecolor{currentstroke}%
\pgfsetdash{}{0pt}%
\pgfsys@defobject{currentmarker}{\pgfqpoint{0.000000in}{-0.048611in}}{\pgfqpoint{0.000000in}{0.000000in}}{%
\pgfpathmoveto{\pgfqpoint{0.000000in}{0.000000in}}%
\pgfpathlineto{\pgfqpoint{0.000000in}{-0.048611in}}%
\pgfusepath{stroke,fill}%
}%
\begin{pgfscope}%
\pgfsys@transformshift{1.856985in}{0.528000in}%
\pgfsys@useobject{currentmarker}{}%
\end{pgfscope}%
\end{pgfscope}%
\begin{pgfscope}%
\definecolor{textcolor}{rgb}{0.000000,0.000000,0.000000}%
\pgfsetstrokecolor{textcolor}%
\pgfsetfillcolor{textcolor}%
\pgftext[x=1.856985in,y=0.430778in,,top]{\color{textcolor}\rmfamily\fontsize{10.000000}{12.000000}\selectfont \(\displaystyle {\ensuremath{-}0.5}\)}%
\end{pgfscope}%
\begin{pgfscope}%
\pgfpathrectangle{\pgfqpoint{0.800000in}{0.528000in}}{\pgfqpoint{4.960000in}{3.696000in}}%
\pgfusepath{clip}%
\pgfsetrectcap%
\pgfsetroundjoin%
\pgfsetlinewidth{0.803000pt}%
\definecolor{currentstroke}{rgb}{0.690196,0.690196,0.690196}%
\pgfsetstrokecolor{currentstroke}%
\pgfsetdash{}{0pt}%
\pgfpathmoveto{\pgfqpoint{2.670113in}{0.528000in}}%
\pgfpathlineto{\pgfqpoint{2.670113in}{4.224000in}}%
\pgfusepath{stroke}%
\end{pgfscope}%
\begin{pgfscope}%
\pgfsetbuttcap%
\pgfsetroundjoin%
\definecolor{currentfill}{rgb}{0.000000,0.000000,0.000000}%
\pgfsetfillcolor{currentfill}%
\pgfsetlinewidth{0.803000pt}%
\definecolor{currentstroke}{rgb}{0.000000,0.000000,0.000000}%
\pgfsetstrokecolor{currentstroke}%
\pgfsetdash{}{0pt}%
\pgfsys@defobject{currentmarker}{\pgfqpoint{0.000000in}{-0.048611in}}{\pgfqpoint{0.000000in}{0.000000in}}{%
\pgfpathmoveto{\pgfqpoint{0.000000in}{0.000000in}}%
\pgfpathlineto{\pgfqpoint{0.000000in}{-0.048611in}}%
\pgfusepath{stroke,fill}%
}%
\begin{pgfscope}%
\pgfsys@transformshift{2.670113in}{0.528000in}%
\pgfsys@useobject{currentmarker}{}%
\end{pgfscope}%
\end{pgfscope}%
\begin{pgfscope}%
\definecolor{textcolor}{rgb}{0.000000,0.000000,0.000000}%
\pgfsetstrokecolor{textcolor}%
\pgfsetfillcolor{textcolor}%
\pgftext[x=2.670113in,y=0.430778in,,top]{\color{textcolor}\rmfamily\fontsize{10.000000}{12.000000}\selectfont \(\displaystyle {0.0}\)}%
\end{pgfscope}%
\begin{pgfscope}%
\pgfpathrectangle{\pgfqpoint{0.800000in}{0.528000in}}{\pgfqpoint{4.960000in}{3.696000in}}%
\pgfusepath{clip}%
\pgfsetrectcap%
\pgfsetroundjoin%
\pgfsetlinewidth{0.803000pt}%
\definecolor{currentstroke}{rgb}{0.690196,0.690196,0.690196}%
\pgfsetstrokecolor{currentstroke}%
\pgfsetdash{}{0pt}%
\pgfpathmoveto{\pgfqpoint{3.483241in}{0.528000in}}%
\pgfpathlineto{\pgfqpoint{3.483241in}{4.224000in}}%
\pgfusepath{stroke}%
\end{pgfscope}%
\begin{pgfscope}%
\pgfsetbuttcap%
\pgfsetroundjoin%
\definecolor{currentfill}{rgb}{0.000000,0.000000,0.000000}%
\pgfsetfillcolor{currentfill}%
\pgfsetlinewidth{0.803000pt}%
\definecolor{currentstroke}{rgb}{0.000000,0.000000,0.000000}%
\pgfsetstrokecolor{currentstroke}%
\pgfsetdash{}{0pt}%
\pgfsys@defobject{currentmarker}{\pgfqpoint{0.000000in}{-0.048611in}}{\pgfqpoint{0.000000in}{0.000000in}}{%
\pgfpathmoveto{\pgfqpoint{0.000000in}{0.000000in}}%
\pgfpathlineto{\pgfqpoint{0.000000in}{-0.048611in}}%
\pgfusepath{stroke,fill}%
}%
\begin{pgfscope}%
\pgfsys@transformshift{3.483241in}{0.528000in}%
\pgfsys@useobject{currentmarker}{}%
\end{pgfscope}%
\end{pgfscope}%
\begin{pgfscope}%
\definecolor{textcolor}{rgb}{0.000000,0.000000,0.000000}%
\pgfsetstrokecolor{textcolor}%
\pgfsetfillcolor{textcolor}%
\pgftext[x=3.483241in,y=0.430778in,,top]{\color{textcolor}\rmfamily\fontsize{10.000000}{12.000000}\selectfont \(\displaystyle {0.5}\)}%
\end{pgfscope}%
\begin{pgfscope}%
\pgfpathrectangle{\pgfqpoint{0.800000in}{0.528000in}}{\pgfqpoint{4.960000in}{3.696000in}}%
\pgfusepath{clip}%
\pgfsetrectcap%
\pgfsetroundjoin%
\pgfsetlinewidth{0.803000pt}%
\definecolor{currentstroke}{rgb}{0.690196,0.690196,0.690196}%
\pgfsetstrokecolor{currentstroke}%
\pgfsetdash{}{0pt}%
\pgfpathmoveto{\pgfqpoint{4.296369in}{0.528000in}}%
\pgfpathlineto{\pgfqpoint{4.296369in}{4.224000in}}%
\pgfusepath{stroke}%
\end{pgfscope}%
\begin{pgfscope}%
\pgfsetbuttcap%
\pgfsetroundjoin%
\definecolor{currentfill}{rgb}{0.000000,0.000000,0.000000}%
\pgfsetfillcolor{currentfill}%
\pgfsetlinewidth{0.803000pt}%
\definecolor{currentstroke}{rgb}{0.000000,0.000000,0.000000}%
\pgfsetstrokecolor{currentstroke}%
\pgfsetdash{}{0pt}%
\pgfsys@defobject{currentmarker}{\pgfqpoint{0.000000in}{-0.048611in}}{\pgfqpoint{0.000000in}{0.000000in}}{%
\pgfpathmoveto{\pgfqpoint{0.000000in}{0.000000in}}%
\pgfpathlineto{\pgfqpoint{0.000000in}{-0.048611in}}%
\pgfusepath{stroke,fill}%
}%
\begin{pgfscope}%
\pgfsys@transformshift{4.296369in}{0.528000in}%
\pgfsys@useobject{currentmarker}{}%
\end{pgfscope}%
\end{pgfscope}%
\begin{pgfscope}%
\definecolor{textcolor}{rgb}{0.000000,0.000000,0.000000}%
\pgfsetstrokecolor{textcolor}%
\pgfsetfillcolor{textcolor}%
\pgftext[x=4.296369in,y=0.430778in,,top]{\color{textcolor}\rmfamily\fontsize{10.000000}{12.000000}\selectfont \(\displaystyle {1.0}\)}%
\end{pgfscope}%
\begin{pgfscope}%
\pgfpathrectangle{\pgfqpoint{0.800000in}{0.528000in}}{\pgfqpoint{4.960000in}{3.696000in}}%
\pgfusepath{clip}%
\pgfsetrectcap%
\pgfsetroundjoin%
\pgfsetlinewidth{0.803000pt}%
\definecolor{currentstroke}{rgb}{0.690196,0.690196,0.690196}%
\pgfsetstrokecolor{currentstroke}%
\pgfsetdash{}{0pt}%
\pgfpathmoveto{\pgfqpoint{5.109498in}{0.528000in}}%
\pgfpathlineto{\pgfqpoint{5.109498in}{4.224000in}}%
\pgfusepath{stroke}%
\end{pgfscope}%
\begin{pgfscope}%
\pgfsetbuttcap%
\pgfsetroundjoin%
\definecolor{currentfill}{rgb}{0.000000,0.000000,0.000000}%
\pgfsetfillcolor{currentfill}%
\pgfsetlinewidth{0.803000pt}%
\definecolor{currentstroke}{rgb}{0.000000,0.000000,0.000000}%
\pgfsetstrokecolor{currentstroke}%
\pgfsetdash{}{0pt}%
\pgfsys@defobject{currentmarker}{\pgfqpoint{0.000000in}{-0.048611in}}{\pgfqpoint{0.000000in}{0.000000in}}{%
\pgfpathmoveto{\pgfqpoint{0.000000in}{0.000000in}}%
\pgfpathlineto{\pgfqpoint{0.000000in}{-0.048611in}}%
\pgfusepath{stroke,fill}%
}%
\begin{pgfscope}%
\pgfsys@transformshift{5.109498in}{0.528000in}%
\pgfsys@useobject{currentmarker}{}%
\end{pgfscope}%
\end{pgfscope}%
\begin{pgfscope}%
\definecolor{textcolor}{rgb}{0.000000,0.000000,0.000000}%
\pgfsetstrokecolor{textcolor}%
\pgfsetfillcolor{textcolor}%
\pgftext[x=5.109498in,y=0.430778in,,top]{\color{textcolor}\rmfamily\fontsize{10.000000}{12.000000}\selectfont \(\displaystyle {1.5}\)}%
\end{pgfscope}%
\begin{pgfscope}%
\pgfpathrectangle{\pgfqpoint{0.800000in}{0.528000in}}{\pgfqpoint{4.960000in}{3.696000in}}%
\pgfusepath{clip}%
\pgfsetrectcap%
\pgfsetroundjoin%
\pgfsetlinewidth{0.803000pt}%
\definecolor{currentstroke}{rgb}{0.690196,0.690196,0.690196}%
\pgfsetstrokecolor{currentstroke}%
\pgfsetdash{}{0pt}%
\pgfpathmoveto{\pgfqpoint{0.800000in}{0.975982in}}%
\pgfpathlineto{\pgfqpoint{5.760000in}{0.975982in}}%
\pgfusepath{stroke}%
\end{pgfscope}%
\begin{pgfscope}%
\pgfsetbuttcap%
\pgfsetroundjoin%
\definecolor{currentfill}{rgb}{0.000000,0.000000,0.000000}%
\pgfsetfillcolor{currentfill}%
\pgfsetlinewidth{0.803000pt}%
\definecolor{currentstroke}{rgb}{0.000000,0.000000,0.000000}%
\pgfsetstrokecolor{currentstroke}%
\pgfsetdash{}{0pt}%
\pgfsys@defobject{currentmarker}{\pgfqpoint{-0.048611in}{0.000000in}}{\pgfqpoint{-0.000000in}{0.000000in}}{%
\pgfpathmoveto{\pgfqpoint{-0.000000in}{0.000000in}}%
\pgfpathlineto{\pgfqpoint{-0.048611in}{0.000000in}}%
\pgfusepath{stroke,fill}%
}%
\begin{pgfscope}%
\pgfsys@transformshift{0.800000in}{0.975982in}%
\pgfsys@useobject{currentmarker}{}%
\end{pgfscope}%
\end{pgfscope}%
\begin{pgfscope}%
\definecolor{textcolor}{rgb}{0.000000,0.000000,0.000000}%
\pgfsetstrokecolor{textcolor}%
\pgfsetfillcolor{textcolor}%
\pgftext[x=0.417283in, y=0.924882in, left, base]{\color{textcolor}\rmfamily\fontsize{10.000000}{12.000000}\selectfont \(\displaystyle {\ensuremath{-}1.0}\)}%
\end{pgfscope}%
\begin{pgfscope}%
\pgfpathrectangle{\pgfqpoint{0.800000in}{0.528000in}}{\pgfqpoint{4.960000in}{3.696000in}}%
\pgfusepath{clip}%
\pgfsetrectcap%
\pgfsetroundjoin%
\pgfsetlinewidth{0.803000pt}%
\definecolor{currentstroke}{rgb}{0.690196,0.690196,0.690196}%
\pgfsetstrokecolor{currentstroke}%
\pgfsetdash{}{0pt}%
\pgfpathmoveto{\pgfqpoint{0.800000in}{1.675986in}}%
\pgfpathlineto{\pgfqpoint{5.760000in}{1.675986in}}%
\pgfusepath{stroke}%
\end{pgfscope}%
\begin{pgfscope}%
\pgfsetbuttcap%
\pgfsetroundjoin%
\definecolor{currentfill}{rgb}{0.000000,0.000000,0.000000}%
\pgfsetfillcolor{currentfill}%
\pgfsetlinewidth{0.803000pt}%
\definecolor{currentstroke}{rgb}{0.000000,0.000000,0.000000}%
\pgfsetstrokecolor{currentstroke}%
\pgfsetdash{}{0pt}%
\pgfsys@defobject{currentmarker}{\pgfqpoint{-0.048611in}{0.000000in}}{\pgfqpoint{-0.000000in}{0.000000in}}{%
\pgfpathmoveto{\pgfqpoint{-0.000000in}{0.000000in}}%
\pgfpathlineto{\pgfqpoint{-0.048611in}{0.000000in}}%
\pgfusepath{stroke,fill}%
}%
\begin{pgfscope}%
\pgfsys@transformshift{0.800000in}{1.675986in}%
\pgfsys@useobject{currentmarker}{}%
\end{pgfscope}%
\end{pgfscope}%
\begin{pgfscope}%
\definecolor{textcolor}{rgb}{0.000000,0.000000,0.000000}%
\pgfsetstrokecolor{textcolor}%
\pgfsetfillcolor{textcolor}%
\pgftext[x=0.417283in, y=1.624886in, left, base]{\color{textcolor}\rmfamily\fontsize{10.000000}{12.000000}\selectfont \(\displaystyle {\ensuremath{-}0.5}\)}%
\end{pgfscope}%
\begin{pgfscope}%
\pgfpathrectangle{\pgfqpoint{0.800000in}{0.528000in}}{\pgfqpoint{4.960000in}{3.696000in}}%
\pgfusepath{clip}%
\pgfsetrectcap%
\pgfsetroundjoin%
\pgfsetlinewidth{0.803000pt}%
\definecolor{currentstroke}{rgb}{0.690196,0.690196,0.690196}%
\pgfsetstrokecolor{currentstroke}%
\pgfsetdash{}{0pt}%
\pgfpathmoveto{\pgfqpoint{0.800000in}{2.375991in}}%
\pgfpathlineto{\pgfqpoint{5.760000in}{2.375991in}}%
\pgfusepath{stroke}%
\end{pgfscope}%
\begin{pgfscope}%
\pgfsetbuttcap%
\pgfsetroundjoin%
\definecolor{currentfill}{rgb}{0.000000,0.000000,0.000000}%
\pgfsetfillcolor{currentfill}%
\pgfsetlinewidth{0.803000pt}%
\definecolor{currentstroke}{rgb}{0.000000,0.000000,0.000000}%
\pgfsetstrokecolor{currentstroke}%
\pgfsetdash{}{0pt}%
\pgfsys@defobject{currentmarker}{\pgfqpoint{-0.048611in}{0.000000in}}{\pgfqpoint{-0.000000in}{0.000000in}}{%
\pgfpathmoveto{\pgfqpoint{-0.000000in}{0.000000in}}%
\pgfpathlineto{\pgfqpoint{-0.048611in}{0.000000in}}%
\pgfusepath{stroke,fill}%
}%
\begin{pgfscope}%
\pgfsys@transformshift{0.800000in}{2.375991in}%
\pgfsys@useobject{currentmarker}{}%
\end{pgfscope}%
\end{pgfscope}%
\begin{pgfscope}%
\definecolor{textcolor}{rgb}{0.000000,0.000000,0.000000}%
\pgfsetstrokecolor{textcolor}%
\pgfsetfillcolor{textcolor}%
\pgftext[x=0.525308in, y=2.324891in, left, base]{\color{textcolor}\rmfamily\fontsize{10.000000}{12.000000}\selectfont \(\displaystyle {0.0}\)}%
\end{pgfscope}%
\begin{pgfscope}%
\pgfpathrectangle{\pgfqpoint{0.800000in}{0.528000in}}{\pgfqpoint{4.960000in}{3.696000in}}%
\pgfusepath{clip}%
\pgfsetrectcap%
\pgfsetroundjoin%
\pgfsetlinewidth{0.803000pt}%
\definecolor{currentstroke}{rgb}{0.690196,0.690196,0.690196}%
\pgfsetstrokecolor{currentstroke}%
\pgfsetdash{}{0pt}%
\pgfpathmoveto{\pgfqpoint{0.800000in}{3.075995in}}%
\pgfpathlineto{\pgfqpoint{5.760000in}{3.075995in}}%
\pgfusepath{stroke}%
\end{pgfscope}%
\begin{pgfscope}%
\pgfsetbuttcap%
\pgfsetroundjoin%
\definecolor{currentfill}{rgb}{0.000000,0.000000,0.000000}%
\pgfsetfillcolor{currentfill}%
\pgfsetlinewidth{0.803000pt}%
\definecolor{currentstroke}{rgb}{0.000000,0.000000,0.000000}%
\pgfsetstrokecolor{currentstroke}%
\pgfsetdash{}{0pt}%
\pgfsys@defobject{currentmarker}{\pgfqpoint{-0.048611in}{0.000000in}}{\pgfqpoint{-0.000000in}{0.000000in}}{%
\pgfpathmoveto{\pgfqpoint{-0.000000in}{0.000000in}}%
\pgfpathlineto{\pgfqpoint{-0.048611in}{0.000000in}}%
\pgfusepath{stroke,fill}%
}%
\begin{pgfscope}%
\pgfsys@transformshift{0.800000in}{3.075995in}%
\pgfsys@useobject{currentmarker}{}%
\end{pgfscope}%
\end{pgfscope}%
\begin{pgfscope}%
\definecolor{textcolor}{rgb}{0.000000,0.000000,0.000000}%
\pgfsetstrokecolor{textcolor}%
\pgfsetfillcolor{textcolor}%
\pgftext[x=0.525308in, y=3.024895in, left, base]{\color{textcolor}\rmfamily\fontsize{10.000000}{12.000000}\selectfont \(\displaystyle {0.5}\)}%
\end{pgfscope}%
\begin{pgfscope}%
\pgfpathrectangle{\pgfqpoint{0.800000in}{0.528000in}}{\pgfqpoint{4.960000in}{3.696000in}}%
\pgfusepath{clip}%
\pgfsetrectcap%
\pgfsetroundjoin%
\pgfsetlinewidth{0.803000pt}%
\definecolor{currentstroke}{rgb}{0.690196,0.690196,0.690196}%
\pgfsetstrokecolor{currentstroke}%
\pgfsetdash{}{0pt}%
\pgfpathmoveto{\pgfqpoint{0.800000in}{3.775999in}}%
\pgfpathlineto{\pgfqpoint{5.760000in}{3.775999in}}%
\pgfusepath{stroke}%
\end{pgfscope}%
\begin{pgfscope}%
\pgfsetbuttcap%
\pgfsetroundjoin%
\definecolor{currentfill}{rgb}{0.000000,0.000000,0.000000}%
\pgfsetfillcolor{currentfill}%
\pgfsetlinewidth{0.803000pt}%
\definecolor{currentstroke}{rgb}{0.000000,0.000000,0.000000}%
\pgfsetstrokecolor{currentstroke}%
\pgfsetdash{}{0pt}%
\pgfsys@defobject{currentmarker}{\pgfqpoint{-0.048611in}{0.000000in}}{\pgfqpoint{-0.000000in}{0.000000in}}{%
\pgfpathmoveto{\pgfqpoint{-0.000000in}{0.000000in}}%
\pgfpathlineto{\pgfqpoint{-0.048611in}{0.000000in}}%
\pgfusepath{stroke,fill}%
}%
\begin{pgfscope}%
\pgfsys@transformshift{0.800000in}{3.775999in}%
\pgfsys@useobject{currentmarker}{}%
\end{pgfscope}%
\end{pgfscope}%
\begin{pgfscope}%
\definecolor{textcolor}{rgb}{0.000000,0.000000,0.000000}%
\pgfsetstrokecolor{textcolor}%
\pgfsetfillcolor{textcolor}%
\pgftext[x=0.525308in, y=3.724899in, left, base]{\color{textcolor}\rmfamily\fontsize{10.000000}{12.000000}\selectfont \(\displaystyle {1.0}\)}%
\end{pgfscope}%
\begin{pgfscope}%
\pgfpathrectangle{\pgfqpoint{0.800000in}{0.528000in}}{\pgfqpoint{4.960000in}{3.696000in}}%
\pgfusepath{clip}%
\pgfsetrectcap%
\pgfsetroundjoin%
\pgfsetlinewidth{1.505625pt}%
\definecolor{currentstroke}{rgb}{0.121569,0.466667,0.705882}%
\pgfsetstrokecolor{currentstroke}%
\pgfsetdash{}{0pt}%
\pgfpathmoveto{\pgfqpoint{1.043857in}{3.636185in}}%
\pgfpathlineto{\pgfqpoint{2.668487in}{3.636171in}}%
\pgfpathlineto{\pgfqpoint{2.670113in}{3.224123in}}%
\pgfpathlineto{\pgfqpoint{2.694507in}{3.225300in}}%
\pgfpathlineto{\pgfqpoint{2.718901in}{3.228805in}}%
\pgfpathlineto{\pgfqpoint{2.743295in}{3.234578in}}%
\pgfpathlineto{\pgfqpoint{2.769315in}{3.243120in}}%
\pgfpathlineto{\pgfqpoint{2.796961in}{3.254714in}}%
\pgfpathlineto{\pgfqpoint{2.826234in}{3.269541in}}%
\pgfpathlineto{\pgfqpoint{2.858759in}{3.288667in}}%
\pgfpathlineto{\pgfqpoint{2.896163in}{3.313398in}}%
\pgfpathlineto{\pgfqpoint{2.944951in}{3.348518in}}%
\pgfpathlineto{\pgfqpoint{3.049031in}{3.424220in}}%
\pgfpathlineto{\pgfqpoint{3.086435in}{3.448355in}}%
\pgfpathlineto{\pgfqpoint{3.118960in}{3.466807in}}%
\pgfpathlineto{\pgfqpoint{3.148233in}{3.480911in}}%
\pgfpathlineto{\pgfqpoint{3.175879in}{3.491709in}}%
\pgfpathlineto{\pgfqpoint{3.200273in}{3.498972in}}%
\pgfpathlineto{\pgfqpoint{3.224667in}{3.503927in}}%
\pgfpathlineto{\pgfqpoint{3.247434in}{3.506324in}}%
\pgfpathlineto{\pgfqpoint{3.270202in}{3.506442in}}%
\pgfpathlineto{\pgfqpoint{3.291343in}{3.504406in}}%
\pgfpathlineto{\pgfqpoint{3.309232in}{3.501014in}}%
\pgfpathlineto{\pgfqpoint{3.310858in}{4.047022in}}%
\pgfpathlineto{\pgfqpoint{3.332000in}{4.038385in}}%
\pgfpathlineto{\pgfqpoint{3.356393in}{4.025805in}}%
\pgfpathlineto{\pgfqpoint{3.385666in}{4.007940in}}%
\pgfpathlineto{\pgfqpoint{3.423070in}{3.982243in}}%
\pgfpathlineto{\pgfqpoint{3.496251in}{3.928451in}}%
\pgfpathlineto{\pgfqpoint{3.558049in}{3.884367in}}%
\pgfpathlineto{\pgfqpoint{3.605211in}{3.853459in}}%
\pgfpathlineto{\pgfqpoint{3.647493in}{3.828339in}}%
\pgfpathlineto{\pgfqpoint{3.688150in}{3.806683in}}%
\pgfpathlineto{\pgfqpoint{3.728806in}{3.787515in}}%
\pgfpathlineto{\pgfqpoint{3.769462in}{3.770791in}}%
\pgfpathlineto{\pgfqpoint{3.810119in}{3.756417in}}%
\pgfpathlineto{\pgfqpoint{3.852402in}{3.743832in}}%
\pgfpathlineto{\pgfqpoint{3.894684in}{3.733518in}}%
\pgfpathlineto{\pgfqpoint{3.938593in}{3.725065in}}%
\pgfpathlineto{\pgfqpoint{3.984128in}{3.718585in}}%
\pgfpathlineto{\pgfqpoint{4.029663in}{3.714315in}}%
\pgfpathlineto{\pgfqpoint{4.076825in}{3.712129in}}%
\pgfpathlineto{\pgfqpoint{4.123986in}{3.712166in}}%
\pgfpathlineto{\pgfqpoint{4.171148in}{3.714418in}}%
\pgfpathlineto{\pgfqpoint{4.218309in}{3.718923in}}%
\pgfpathlineto{\pgfqpoint{4.263844in}{3.725484in}}%
\pgfpathlineto{\pgfqpoint{4.309379in}{3.734326in}}%
\pgfpathlineto{\pgfqpoint{4.353288in}{3.745137in}}%
\pgfpathlineto{\pgfqpoint{4.395571in}{3.757798in}}%
\pgfpathlineto{\pgfqpoint{4.437854in}{3.772796in}}%
\pgfpathlineto{\pgfqpoint{4.478510in}{3.789530in}}%
\pgfpathlineto{\pgfqpoint{4.519167in}{3.808611in}}%
\pgfpathlineto{\pgfqpoint{4.561449in}{3.830979in}}%
\pgfpathlineto{\pgfqpoint{4.603732in}{3.855862in}}%
\pgfpathlineto{\pgfqpoint{4.649267in}{3.885238in}}%
\pgfpathlineto{\pgfqpoint{4.704560in}{3.923739in}}%
\pgfpathlineto{\pgfqpoint{4.820024in}{4.005181in}}%
\pgfpathlineto{\pgfqpoint{4.852549in}{4.024810in}}%
\pgfpathlineto{\pgfqpoint{4.878569in}{4.038020in}}%
\pgfpathlineto{\pgfqpoint{4.901337in}{4.047111in}}%
\pgfpathlineto{\pgfqpoint{4.920852in}{4.052614in}}%
\pgfpathlineto{\pgfqpoint{4.938741in}{4.055448in}}%
\pgfpathlineto{\pgfqpoint{4.955003in}{4.055913in}}%
\pgfpathlineto{\pgfqpoint{4.971266in}{4.054109in}}%
\pgfpathlineto{\pgfqpoint{4.985902in}{4.050336in}}%
\pgfpathlineto{\pgfqpoint{5.000538in}{4.044334in}}%
\pgfpathlineto{\pgfqpoint{5.015175in}{4.035917in}}%
\pgfpathlineto{\pgfqpoint{5.029811in}{4.024906in}}%
\pgfpathlineto{\pgfqpoint{5.044447in}{4.011132in}}%
\pgfpathlineto{\pgfqpoint{5.059084in}{3.994437in}}%
\pgfpathlineto{\pgfqpoint{5.075346in}{3.972291in}}%
\pgfpathlineto{\pgfqpoint{5.091609in}{3.946203in}}%
\pgfpathlineto{\pgfqpoint{5.107871in}{3.916054in}}%
\pgfpathlineto{\pgfqpoint{5.125760in}{3.878113in}}%
\pgfpathlineto{\pgfqpoint{5.145275in}{3.830994in}}%
\pgfpathlineto{\pgfqpoint{5.164790in}{3.777995in}}%
\pgfpathlineto{\pgfqpoint{5.185932in}{3.714217in}}%
\pgfpathlineto{\pgfqpoint{5.210325in}{3.633048in}}%
\pgfpathlineto{\pgfqpoint{5.237972in}{3.532589in}}%
\pgfpathlineto{\pgfqpoint{5.272123in}{3.399159in}}%
\pgfpathlineto{\pgfqpoint{5.385961in}{2.945549in}}%
\pgfpathlineto{\pgfqpoint{5.411981in}{2.855000in}}%
\pgfpathlineto{\pgfqpoint{5.434749in}{2.783696in}}%
\pgfpathlineto{\pgfqpoint{5.454264in}{2.729445in}}%
\pgfpathlineto{\pgfqpoint{5.472153in}{2.685873in}}%
\pgfpathlineto{\pgfqpoint{5.488415in}{2.651739in}}%
\pgfpathlineto{\pgfqpoint{5.503052in}{2.625702in}}%
\pgfpathlineto{\pgfqpoint{5.517688in}{2.604263in}}%
\pgfpathlineto{\pgfqpoint{5.530698in}{2.589166in}}%
\pgfpathlineto{\pgfqpoint{5.542082in}{2.579068in}}%
\pgfpathlineto{\pgfqpoint{5.553465in}{2.571908in}}%
\pgfpathlineto{\pgfqpoint{5.563223in}{2.568126in}}%
\pgfpathlineto{\pgfqpoint{5.572981in}{2.566528in}}%
\pgfpathlineto{\pgfqpoint{5.582738in}{2.567114in}}%
\pgfpathlineto{\pgfqpoint{5.592496in}{2.569882in}}%
\pgfpathlineto{\pgfqpoint{5.602253in}{2.574822in}}%
\pgfpathlineto{\pgfqpoint{5.613637in}{2.583314in}}%
\pgfpathlineto{\pgfqpoint{5.625021in}{2.594711in}}%
\pgfpathlineto{\pgfqpoint{5.638031in}{2.611241in}}%
\pgfpathlineto{\pgfqpoint{5.651041in}{2.631429in}}%
\pgfpathlineto{\pgfqpoint{5.665677in}{2.658389in}}%
\pgfpathlineto{\pgfqpoint{5.681940in}{2.693419in}}%
\pgfpathlineto{\pgfqpoint{5.699829in}{2.737791in}}%
\pgfpathlineto{\pgfqpoint{5.719344in}{2.792664in}}%
\pgfpathlineto{\pgfqpoint{5.740485in}{2.858956in}}%
\pgfpathlineto{\pgfqpoint{5.760000in}{2.925631in}}%
\pgfpathlineto{\pgfqpoint{5.760000in}{2.925631in}}%
\pgfusepath{stroke}%
\end{pgfscope}%
\begin{pgfscope}%
\pgfpathrectangle{\pgfqpoint{0.800000in}{0.528000in}}{\pgfqpoint{4.960000in}{3.696000in}}%
\pgfusepath{clip}%
\pgfsetrectcap%
\pgfsetroundjoin%
\pgfsetlinewidth{1.505625pt}%
\definecolor{currentstroke}{rgb}{1.000000,0.498039,0.054902}%
\pgfsetstrokecolor{currentstroke}%
\pgfsetdash{}{0pt}%
\pgfpathmoveto{\pgfqpoint{1.043857in}{3.636185in}}%
\pgfpathlineto{\pgfqpoint{2.668487in}{3.636171in}}%
\pgfpathlineto{\pgfqpoint{2.670113in}{3.224123in}}%
\pgfpathlineto{\pgfqpoint{2.694507in}{3.225300in}}%
\pgfpathlineto{\pgfqpoint{2.718901in}{3.228805in}}%
\pgfpathlineto{\pgfqpoint{2.743295in}{3.234578in}}%
\pgfpathlineto{\pgfqpoint{2.769315in}{3.243120in}}%
\pgfpathlineto{\pgfqpoint{2.796961in}{3.254714in}}%
\pgfpathlineto{\pgfqpoint{2.826234in}{3.269541in}}%
\pgfpathlineto{\pgfqpoint{2.858759in}{3.288667in}}%
\pgfpathlineto{\pgfqpoint{2.896163in}{3.313398in}}%
\pgfpathlineto{\pgfqpoint{2.944951in}{3.348518in}}%
\pgfpathlineto{\pgfqpoint{3.049031in}{3.424220in}}%
\pgfpathlineto{\pgfqpoint{3.086435in}{3.448355in}}%
\pgfpathlineto{\pgfqpoint{3.118960in}{3.466807in}}%
\pgfpathlineto{\pgfqpoint{3.148233in}{3.480911in}}%
\pgfpathlineto{\pgfqpoint{3.175879in}{3.491709in}}%
\pgfpathlineto{\pgfqpoint{3.200273in}{3.498972in}}%
\pgfpathlineto{\pgfqpoint{3.224667in}{3.503927in}}%
\pgfpathlineto{\pgfqpoint{3.247434in}{3.506324in}}%
\pgfpathlineto{\pgfqpoint{3.270202in}{3.506442in}}%
\pgfpathlineto{\pgfqpoint{3.291343in}{3.504406in}}%
\pgfpathlineto{\pgfqpoint{3.309232in}{3.500997in}}%
\pgfpathlineto{\pgfqpoint{3.310858in}{4.046982in}}%
\pgfpathlineto{\pgfqpoint{3.330373in}{4.038712in}}%
\pgfpathlineto{\pgfqpoint{3.353141in}{4.026511in}}%
\pgfpathlineto{\pgfqpoint{3.379161in}{4.009936in}}%
\pgfpathlineto{\pgfqpoint{3.411686in}{3.986397in}}%
\pgfpathlineto{\pgfqpoint{3.457221in}{3.950365in}}%
\pgfpathlineto{\pgfqpoint{3.590574in}{3.843184in}}%
\pgfpathlineto{\pgfqpoint{3.642614in}{3.805022in}}%
\pgfpathlineto{\pgfqpoint{3.693028in}{3.770770in}}%
\pgfpathlineto{\pgfqpoint{3.746695in}{3.737066in}}%
\pgfpathlineto{\pgfqpoint{3.806866in}{3.702021in}}%
\pgfpathlineto{\pgfqpoint{3.891432in}{3.655672in}}%
\pgfpathlineto{\pgfqpoint{3.992260in}{3.599986in}}%
\pgfpathlineto{\pgfqpoint{4.044300in}{3.568614in}}%
\pgfpathlineto{\pgfqpoint{4.086582in}{3.540562in}}%
\pgfpathlineto{\pgfqpoint{4.123986in}{3.513064in}}%
\pgfpathlineto{\pgfqpoint{4.158138in}{3.485113in}}%
\pgfpathlineto{\pgfqpoint{4.189037in}{3.456926in}}%
\pgfpathlineto{\pgfqpoint{4.218309in}{3.427155in}}%
\pgfpathlineto{\pgfqpoint{4.245956in}{3.395787in}}%
\pgfpathlineto{\pgfqpoint{4.271976in}{3.362887in}}%
\pgfpathlineto{\pgfqpoint{4.296369in}{3.328598in}}%
\pgfpathlineto{\pgfqpoint{4.320763in}{3.290473in}}%
\pgfpathlineto{\pgfqpoint{4.343531in}{3.250936in}}%
\pgfpathlineto{\pgfqpoint{4.366298in}{3.207057in}}%
\pgfpathlineto{\pgfqpoint{4.389066in}{3.158251in}}%
\pgfpathlineto{\pgfqpoint{4.411834in}{3.103866in}}%
\pgfpathlineto{\pgfqpoint{4.432975in}{3.047728in}}%
\pgfpathlineto{\pgfqpoint{4.454116in}{2.985496in}}%
\pgfpathlineto{\pgfqpoint{4.475258in}{2.916454in}}%
\pgfpathlineto{\pgfqpoint{4.496399in}{2.839830in}}%
\pgfpathlineto{\pgfqpoint{4.517540in}{2.754799in}}%
\pgfpathlineto{\pgfqpoint{4.538682in}{2.660504in}}%
\pgfpathlineto{\pgfqpoint{4.559823in}{2.556083in}}%
\pgfpathlineto{\pgfqpoint{4.580964in}{2.440722in}}%
\pgfpathlineto{\pgfqpoint{4.603732in}{2.303456in}}%
\pgfpathlineto{\pgfqpoint{4.626499in}{2.152137in}}%
\pgfpathlineto{\pgfqpoint{4.650893in}{1.974411in}}%
\pgfpathlineto{\pgfqpoint{4.676913in}{1.768338in}}%
\pgfpathlineto{\pgfqpoint{4.711065in}{1.478526in}}%
\pgfpathlineto{\pgfqpoint{4.759852in}{1.064024in}}%
\pgfpathlineto{\pgfqpoint{4.777741in}{0.929867in}}%
\pgfpathlineto{\pgfqpoint{4.792378in}{0.835762in}}%
\pgfpathlineto{\pgfqpoint{4.803761in}{0.775953in}}%
\pgfpathlineto{\pgfqpoint{4.813519in}{0.736286in}}%
\pgfpathlineto{\pgfqpoint{4.821650in}{0.712791in}}%
\pgfpathlineto{\pgfqpoint{4.828155in}{0.701024in}}%
\pgfpathlineto{\pgfqpoint{4.833034in}{0.696659in}}%
\pgfpathlineto{\pgfqpoint{4.836286in}{0.696000in}}%
\pgfpathlineto{\pgfqpoint{4.839539in}{0.697218in}}%
\pgfpathlineto{\pgfqpoint{4.842792in}{0.700379in}}%
\pgfpathlineto{\pgfqpoint{4.847670in}{0.708901in}}%
\pgfpathlineto{\pgfqpoint{4.852549in}{0.722138in}}%
\pgfpathlineto{\pgfqpoint{4.859054in}{0.747452in}}%
\pgfpathlineto{\pgfqpoint{4.865559in}{0.781895in}}%
\pgfpathlineto{\pgfqpoint{4.873690in}{0.838305in}}%
\pgfpathlineto{\pgfqpoint{4.881822in}{0.910014in}}%
\pgfpathlineto{\pgfqpoint{4.891579in}{1.016636in}}%
\pgfpathlineto{\pgfqpoint{4.902963in}{1.169307in}}%
\pgfpathlineto{\pgfqpoint{4.915973in}{1.379627in}}%
\pgfpathlineto{\pgfqpoint{4.930609in}{1.657365in}}%
\pgfpathlineto{\pgfqpoint{4.948498in}{2.043406in}}%
\pgfpathlineto{\pgfqpoint{5.010296in}{3.427329in}}%
\pgfpathlineto{\pgfqpoint{5.023306in}{3.652883in}}%
\pgfpathlineto{\pgfqpoint{5.034690in}{3.813715in}}%
\pgfpathlineto{\pgfqpoint{5.044447in}{3.920897in}}%
\pgfpathlineto{\pgfqpoint{5.052579in}{3.987048in}}%
\pgfpathlineto{\pgfqpoint{5.059084in}{4.024248in}}%
\pgfpathlineto{\pgfqpoint{5.063962in}{4.042824in}}%
\pgfpathlineto{\pgfqpoint{5.068841in}{4.053360in}}%
\pgfpathlineto{\pgfqpoint{5.072094in}{4.055917in}}%
\pgfpathlineto{\pgfqpoint{5.075346in}{4.054910in}}%
\pgfpathlineto{\pgfqpoint{5.078599in}{4.050359in}}%
\pgfpathlineto{\pgfqpoint{5.083477in}{4.036943in}}%
\pgfpathlineto{\pgfqpoint{5.088356in}{4.015724in}}%
\pgfpathlineto{\pgfqpoint{5.094861in}{3.975553in}}%
\pgfpathlineto{\pgfqpoint{5.102993in}{3.906872in}}%
\pgfpathlineto{\pgfqpoint{5.111124in}{3.818642in}}%
\pgfpathlineto{\pgfqpoint{5.120881in}{3.688678in}}%
\pgfpathlineto{\pgfqpoint{5.132265in}{3.506887in}}%
\pgfpathlineto{\pgfqpoint{5.146901in}{3.232046in}}%
\pgfpathlineto{\pgfqpoint{5.164790in}{2.847205in}}%
\pgfpathlineto{\pgfqpoint{5.192437in}{2.193301in}}%
\pgfpathlineto{\pgfqpoint{5.221709in}{1.512252in}}%
\pgfpathlineto{\pgfqpoint{5.237972in}{1.184191in}}%
\pgfpathlineto{\pgfqpoint{5.249356in}{0.992640in}}%
\pgfpathlineto{\pgfqpoint{5.259113in}{0.860900in}}%
\pgfpathlineto{\pgfqpoint{5.267244in}{0.778028in}}%
\pgfpathlineto{\pgfqpoint{5.273749in}{0.731413in}}%
\pgfpathlineto{\pgfqpoint{5.278628in}{0.708815in}}%
\pgfpathlineto{\pgfqpoint{5.281881in}{0.699921in}}%
\pgfpathlineto{\pgfqpoint{5.285133in}{0.696121in}}%
\pgfpathlineto{\pgfqpoint{5.286759in}{0.696170in}}%
\pgfpathlineto{\pgfqpoint{5.290012in}{0.700230in}}%
\pgfpathlineto{\pgfqpoint{5.293264in}{0.709654in}}%
\pgfpathlineto{\pgfqpoint{5.298143in}{0.734013in}}%
\pgfpathlineto{\pgfqpoint{5.303022in}{0.770796in}}%
\pgfpathlineto{\pgfqpoint{5.309527in}{0.839285in}}%
\pgfpathlineto{\pgfqpoint{5.317658in}{0.955879in}}%
\pgfpathlineto{\pgfqpoint{5.325790in}{1.105771in}}%
\pgfpathlineto{\pgfqpoint{5.335547in}{1.326402in}}%
\pgfpathlineto{\pgfqpoint{5.348557in}{1.679660in}}%
\pgfpathlineto{\pgfqpoint{5.366446in}{2.240420in}}%
\pgfpathlineto{\pgfqpoint{5.397345in}{3.223668in}}%
\pgfpathlineto{\pgfqpoint{5.410355in}{3.566256in}}%
\pgfpathlineto{\pgfqpoint{5.420112in}{3.772339in}}%
\pgfpathlineto{\pgfqpoint{5.428244in}{3.904241in}}%
\pgfpathlineto{\pgfqpoint{5.434749in}{3.981212in}}%
\pgfpathlineto{\pgfqpoint{5.441254in}{4.031514in}}%
\pgfpathlineto{\pgfqpoint{5.446133in}{4.051315in}}%
\pgfpathlineto{\pgfqpoint{5.449385in}{4.055905in}}%
\pgfpathlineto{\pgfqpoint{5.451011in}{4.055617in}}%
\pgfpathlineto{\pgfqpoint{5.454264in}{4.049887in}}%
\pgfpathlineto{\pgfqpoint{5.457516in}{4.037331in}}%
\pgfpathlineto{\pgfqpoint{5.462395in}{4.005862in}}%
\pgfpathlineto{\pgfqpoint{5.467274in}{3.959561in}}%
\pgfpathlineto{\pgfqpoint{5.473779in}{3.875600in}}%
\pgfpathlineto{\pgfqpoint{5.481910in}{3.737031in}}%
\pgfpathlineto{\pgfqpoint{5.491668in}{3.526223in}}%
\pgfpathlineto{\pgfqpoint{5.503052in}{3.228412in}}%
\pgfpathlineto{\pgfqpoint{5.517688in}{2.785036in}}%
\pgfpathlineto{\pgfqpoint{5.564849in}{1.297972in}}%
\pgfpathlineto{\pgfqpoint{5.576233in}{1.027085in}}%
\pgfpathlineto{\pgfqpoint{5.584364in}{0.876560in}}%
\pgfpathlineto{\pgfqpoint{5.590869in}{0.786484in}}%
\pgfpathlineto{\pgfqpoint{5.597374in}{0.726237in}}%
\pgfpathlineto{\pgfqpoint{5.602253in}{0.701952in}}%
\pgfpathlineto{\pgfqpoint{5.605506in}{0.696131in}}%
\pgfpathlineto{\pgfqpoint{5.607132in}{0.696397in}}%
\pgfpathlineto{\pgfqpoint{5.608758in}{0.698804in}}%
\pgfpathlineto{\pgfqpoint{5.612011in}{0.710094in}}%
\pgfpathlineto{\pgfqpoint{5.615263in}{0.730080in}}%
\pgfpathlineto{\pgfqpoint{5.620142in}{0.776418in}}%
\pgfpathlineto{\pgfqpoint{5.626647in}{0.868499in}}%
\pgfpathlineto{\pgfqpoint{5.633152in}{0.994225in}}%
\pgfpathlineto{\pgfqpoint{5.641283in}{1.195698in}}%
\pgfpathlineto{\pgfqpoint{5.651041in}{1.494602in}}%
\pgfpathlineto{\pgfqpoint{5.664051in}{1.966391in}}%
\pgfpathlineto{\pgfqpoint{5.703081in}{3.441750in}}%
\pgfpathlineto{\pgfqpoint{5.712839in}{3.714460in}}%
\pgfpathlineto{\pgfqpoint{5.720970in}{3.886076in}}%
\pgfpathlineto{\pgfqpoint{5.727475in}{3.982145in}}%
\pgfpathlineto{\pgfqpoint{5.732354in}{4.028629in}}%
\pgfpathlineto{\pgfqpoint{5.735606in}{4.047106in}}%
\pgfpathlineto{\pgfqpoint{5.738859in}{4.055467in}}%
\pgfpathlineto{\pgfqpoint{5.740485in}{4.055844in}}%
\pgfpathlineto{\pgfqpoint{5.742111in}{4.053689in}}%
\pgfpathlineto{\pgfqpoint{5.745364in}{4.041815in}}%
\pgfpathlineto{\pgfqpoint{5.748616in}{4.019942in}}%
\pgfpathlineto{\pgfqpoint{5.753495in}{3.968750in}}%
\pgfpathlineto{\pgfqpoint{5.760000in}{3.867444in}}%
\pgfpathlineto{\pgfqpoint{5.760000in}{3.867444in}}%
\pgfusepath{stroke}%
\end{pgfscope}%
\begin{pgfscope}%
\pgfsetrectcap%
\pgfsetmiterjoin%
\pgfsetlinewidth{0.803000pt}%
\definecolor{currentstroke}{rgb}{0.000000,0.000000,0.000000}%
\pgfsetstrokecolor{currentstroke}%
\pgfsetdash{}{0pt}%
\pgfpathmoveto{\pgfqpoint{0.800000in}{0.528000in}}%
\pgfpathlineto{\pgfqpoint{0.800000in}{4.224000in}}%
\pgfusepath{stroke}%
\end{pgfscope}%
\begin{pgfscope}%
\pgfsetrectcap%
\pgfsetmiterjoin%
\pgfsetlinewidth{0.803000pt}%
\definecolor{currentstroke}{rgb}{0.000000,0.000000,0.000000}%
\pgfsetstrokecolor{currentstroke}%
\pgfsetdash{}{0pt}%
\pgfpathmoveto{\pgfqpoint{5.760000in}{0.528000in}}%
\pgfpathlineto{\pgfqpoint{5.760000in}{4.224000in}}%
\pgfusepath{stroke}%
\end{pgfscope}%
\begin{pgfscope}%
\pgfsetrectcap%
\pgfsetmiterjoin%
\pgfsetlinewidth{0.803000pt}%
\definecolor{currentstroke}{rgb}{0.000000,0.000000,0.000000}%
\pgfsetstrokecolor{currentstroke}%
\pgfsetdash{}{0pt}%
\pgfpathmoveto{\pgfqpoint{0.800000in}{0.528000in}}%
\pgfpathlineto{\pgfqpoint{5.760000in}{0.528000in}}%
\pgfusepath{stroke}%
\end{pgfscope}%
\begin{pgfscope}%
\pgfsetrectcap%
\pgfsetmiterjoin%
\pgfsetlinewidth{0.803000pt}%
\definecolor{currentstroke}{rgb}{0.000000,0.000000,0.000000}%
\pgfsetstrokecolor{currentstroke}%
\pgfsetdash{}{0pt}%
\pgfpathmoveto{\pgfqpoint{0.800000in}{4.224000in}}%
\pgfpathlineto{\pgfqpoint{5.760000in}{4.224000in}}%
\pgfusepath{stroke}%
\end{pgfscope}%
\begin{pgfscope}%
\pgfsetbuttcap%
\pgfsetmiterjoin%
\definecolor{currentfill}{rgb}{1.000000,1.000000,1.000000}%
\pgfsetfillcolor{currentfill}%
\pgfsetfillopacity{0.800000}%
\pgfsetlinewidth{1.003750pt}%
\definecolor{currentstroke}{rgb}{0.800000,0.800000,0.800000}%
\pgfsetstrokecolor{currentstroke}%
\pgfsetstrokeopacity{0.800000}%
\pgfsetdash{}{0pt}%
\pgfpathmoveto{\pgfqpoint{0.897222in}{3.709108in}}%
\pgfpathlineto{\pgfqpoint{2.740115in}{3.709108in}}%
\pgfpathquadraticcurveto{\pgfqpoint{2.767893in}{3.709108in}}{\pgfqpoint{2.767893in}{3.736886in}}%
\pgfpathlineto{\pgfqpoint{2.767893in}{4.126778in}}%
\pgfpathquadraticcurveto{\pgfqpoint{2.767893in}{4.154556in}}{\pgfqpoint{2.740115in}{4.154556in}}%
\pgfpathlineto{\pgfqpoint{0.897222in}{4.154556in}}%
\pgfpathquadraticcurveto{\pgfqpoint{0.869444in}{4.154556in}}{\pgfqpoint{0.869444in}{4.126778in}}%
\pgfpathlineto{\pgfqpoint{0.869444in}{3.736886in}}%
\pgfpathquadraticcurveto{\pgfqpoint{0.869444in}{3.709108in}}{\pgfqpoint{0.897222in}{3.709108in}}%
\pgfpathlineto{\pgfqpoint{0.897222in}{3.709108in}}%
\pgfpathclose%
\pgfusepath{stroke,fill}%
\end{pgfscope}%
\begin{pgfscope}%
\pgfsetrectcap%
\pgfsetroundjoin%
\pgfsetlinewidth{1.505625pt}%
\definecolor{currentstroke}{rgb}{0.121569,0.466667,0.705882}%
\pgfsetstrokecolor{currentstroke}%
\pgfsetdash{}{0pt}%
\pgfpathmoveto{\pgfqpoint{0.925000in}{4.045411in}}%
\pgfpathlineto{\pgfqpoint{1.063889in}{4.045411in}}%
\pgfpathlineto{\pgfqpoint{1.202778in}{4.045411in}}%
\pgfusepath{stroke}%
\end{pgfscope}%
\begin{pgfscope}%
\definecolor{textcolor}{rgb}{0.000000,0.000000,0.000000}%
\pgfsetstrokecolor{textcolor}%
\pgfsetfillcolor{textcolor}%
\pgftext[x=1.313889in,y=3.996800in,left,base]{\color{textcolor}\rmfamily\fontsize{10.000000}{12.000000}\selectfont \(\displaystyle \Delta P\) - stable scenario}%
\end{pgfscope}%
\begin{pgfscope}%
\pgfsetrectcap%
\pgfsetroundjoin%
\pgfsetlinewidth{1.505625pt}%
\definecolor{currentstroke}{rgb}{1.000000,0.498039,0.054902}%
\pgfsetstrokecolor{currentstroke}%
\pgfsetdash{}{0pt}%
\pgfpathmoveto{\pgfqpoint{0.925000in}{3.843521in}}%
\pgfpathlineto{\pgfqpoint{1.063889in}{3.843521in}}%
\pgfpathlineto{\pgfqpoint{1.202778in}{3.843521in}}%
\pgfusepath{stroke}%
\end{pgfscope}%
\begin{pgfscope}%
\definecolor{textcolor}{rgb}{0.000000,0.000000,0.000000}%
\pgfsetstrokecolor{textcolor}%
\pgfsetfillcolor{textcolor}%
\pgftext[x=1.313889in,y=3.794909in,left,base]{\color{textcolor}\rmfamily\fontsize{10.000000}{12.000000}\selectfont \(\displaystyle \Delta P\) - unstable scenario}%
\end{pgfscope}%
\end{pgfpicture}%
\makeatother%
\endgroup%


\section{Fault 3}
\label{app:fault3}

%% Creator: Matplotlib, PGF backend
%%
%% To include the figure in your LaTeX document, write
%%   \input{<filename>.pgf}
%%
%% Make sure the required packages are loaded in your preamble
%%   \usepackage{pgf}
%%
%% Also ensure that all the required font packages are loaded; for instance,
%% the lmodern package is sometimes necessary when using math font.
%%   \usepackage{lmodern}
%%
%% Figures using additional raster images can only be included by \input if
%% they are in the same directory as the main LaTeX file. For loading figures
%% from other directories you can use the `import` package
%%   \usepackage{import}
%%
%% and then include the figures with
%%   \import{<path to file>}{<filename>.pgf}
%%
%% Matplotlib used the following preamble
%%   
%%   \usepackage{fontspec}
%%   \setmainfont{Charter.ttc}[Path=\detokenize{/System/Library/Fonts/Supplemental/}]
%%   \setsansfont{DejaVuSans.ttf}[Path=\detokenize{/opt/homebrew/lib/python3.10/site-packages/matplotlib/mpl-data/fonts/ttf/}]
%%   \setmonofont{DejaVuSansMono.ttf}[Path=\detokenize{/opt/homebrew/lib/python3.10/site-packages/matplotlib/mpl-data/fonts/ttf/}]
%%   \makeatletter\@ifpackageloaded{underscore}{}{\usepackage[strings]{underscore}}\makeatother
%%
\begingroup%
\makeatletter%
\begin{pgfpicture}%
\pgfpathrectangle{\pgfpointorigin}{\pgfqpoint{5.000000in}{6.000000in}}%
\pgfusepath{use as bounding box, clip}%
\begin{pgfscope}%
\pgfsetbuttcap%
\pgfsetmiterjoin%
\definecolor{currentfill}{rgb}{1.000000,1.000000,1.000000}%
\pgfsetfillcolor{currentfill}%
\pgfsetlinewidth{0.000000pt}%
\definecolor{currentstroke}{rgb}{1.000000,1.000000,1.000000}%
\pgfsetstrokecolor{currentstroke}%
\pgfsetdash{}{0pt}%
\pgfpathmoveto{\pgfqpoint{0.000000in}{0.000000in}}%
\pgfpathlineto{\pgfqpoint{5.000000in}{0.000000in}}%
\pgfpathlineto{\pgfqpoint{5.000000in}{6.000000in}}%
\pgfpathlineto{\pgfqpoint{0.000000in}{6.000000in}}%
\pgfpathlineto{\pgfqpoint{0.000000in}{0.000000in}}%
\pgfpathclose%
\pgfusepath{fill}%
\end{pgfscope}%
\begin{pgfscope}%
\pgfsetbuttcap%
\pgfsetmiterjoin%
\definecolor{currentfill}{rgb}{1.000000,1.000000,1.000000}%
\pgfsetfillcolor{currentfill}%
\pgfsetlinewidth{0.000000pt}%
\definecolor{currentstroke}{rgb}{0.000000,0.000000,0.000000}%
\pgfsetstrokecolor{currentstroke}%
\pgfsetstrokeopacity{0.000000}%
\pgfsetdash{}{0pt}%
\pgfpathmoveto{\pgfqpoint{0.625000in}{2.970000in}}%
\pgfpathlineto{\pgfqpoint{4.500000in}{2.970000in}}%
\pgfpathlineto{\pgfqpoint{4.500000in}{5.280000in}}%
\pgfpathlineto{\pgfqpoint{0.625000in}{5.280000in}}%
\pgfpathlineto{\pgfqpoint{0.625000in}{2.970000in}}%
\pgfpathclose%
\pgfusepath{fill}%
\end{pgfscope}%
\begin{pgfscope}%
\pgfpathrectangle{\pgfqpoint{0.625000in}{2.970000in}}{\pgfqpoint{3.875000in}{2.310000in}}%
\pgfusepath{clip}%
\pgfsetbuttcap%
\pgfsetroundjoin%
\definecolor{currentfill}{rgb}{0.900000,0.900000,0.900000}%
\pgfsetfillcolor{currentfill}%
\pgfsetlinewidth{1.003750pt}%
\definecolor{currentstroke}{rgb}{0.500000,0.500000,0.500000}%
\pgfsetstrokecolor{currentstroke}%
\pgfsetdash{}{0pt}%
\pgfsys@defobject{currentmarker}{\pgfqpoint{1.270833in}{3.710708in}}{\pgfqpoint{1.672006in}{4.081826in}}{%
\pgfpathmoveto{\pgfqpoint{1.270833in}{4.070565in}}%
\pgfpathlineto{\pgfqpoint{1.270833in}{3.710708in}}%
\pgfpathlineto{\pgfqpoint{1.279021in}{3.719207in}}%
\pgfpathlineto{\pgfqpoint{1.287208in}{3.727673in}}%
\pgfpathlineto{\pgfqpoint{1.295395in}{3.736106in}}%
\pgfpathlineto{\pgfqpoint{1.303582in}{3.744505in}}%
\pgfpathlineto{\pgfqpoint{1.311769in}{3.752870in}}%
\pgfpathlineto{\pgfqpoint{1.319956in}{3.761201in}}%
\pgfpathlineto{\pgfqpoint{1.328144in}{3.769497in}}%
\pgfpathlineto{\pgfqpoint{1.336331in}{3.777757in}}%
\pgfpathlineto{\pgfqpoint{1.344518in}{3.785982in}}%
\pgfpathlineto{\pgfqpoint{1.352705in}{3.794171in}}%
\pgfpathlineto{\pgfqpoint{1.360892in}{3.802324in}}%
\pgfpathlineto{\pgfqpoint{1.369080in}{3.810440in}}%
\pgfpathlineto{\pgfqpoint{1.377267in}{3.818518in}}%
\pgfpathlineto{\pgfqpoint{1.385454in}{3.826560in}}%
\pgfpathlineto{\pgfqpoint{1.393641in}{3.834564in}}%
\pgfpathlineto{\pgfqpoint{1.401828in}{3.842530in}}%
\pgfpathlineto{\pgfqpoint{1.410016in}{3.850457in}}%
\pgfpathlineto{\pgfqpoint{1.418203in}{3.858345in}}%
\pgfpathlineto{\pgfqpoint{1.426390in}{3.866195in}}%
\pgfpathlineto{\pgfqpoint{1.434577in}{3.874004in}}%
\pgfpathlineto{\pgfqpoint{1.442764in}{3.881775in}}%
\pgfpathlineto{\pgfqpoint{1.450951in}{3.889504in}}%
\pgfpathlineto{\pgfqpoint{1.459139in}{3.897194in}}%
\pgfpathlineto{\pgfqpoint{1.467326in}{3.904842in}}%
\pgfpathlineto{\pgfqpoint{1.475513in}{3.912450in}}%
\pgfpathlineto{\pgfqpoint{1.483700in}{3.920015in}}%
\pgfpathlineto{\pgfqpoint{1.491887in}{3.927539in}}%
\pgfpathlineto{\pgfqpoint{1.500075in}{3.935021in}}%
\pgfpathlineto{\pgfqpoint{1.508262in}{3.942460in}}%
\pgfpathlineto{\pgfqpoint{1.516449in}{3.949857in}}%
\pgfpathlineto{\pgfqpoint{1.524636in}{3.957210in}}%
\pgfpathlineto{\pgfqpoint{1.532823in}{3.964520in}}%
\pgfpathlineto{\pgfqpoint{1.541011in}{3.971785in}}%
\pgfpathlineto{\pgfqpoint{1.549198in}{3.979007in}}%
\pgfpathlineto{\pgfqpoint{1.557385in}{3.986185in}}%
\pgfpathlineto{\pgfqpoint{1.565572in}{3.993317in}}%
\pgfpathlineto{\pgfqpoint{1.573759in}{4.000405in}}%
\pgfpathlineto{\pgfqpoint{1.581946in}{4.007447in}}%
\pgfpathlineto{\pgfqpoint{1.590134in}{4.014443in}}%
\pgfpathlineto{\pgfqpoint{1.598321in}{4.021393in}}%
\pgfpathlineto{\pgfqpoint{1.606508in}{4.028297in}}%
\pgfpathlineto{\pgfqpoint{1.614695in}{4.035155in}}%
\pgfpathlineto{\pgfqpoint{1.622882in}{4.041965in}}%
\pgfpathlineto{\pgfqpoint{1.631070in}{4.048728in}}%
\pgfpathlineto{\pgfqpoint{1.639257in}{4.055444in}}%
\pgfpathlineto{\pgfqpoint{1.647444in}{4.062112in}}%
\pgfpathlineto{\pgfqpoint{1.655631in}{4.068732in}}%
\pgfpathlineto{\pgfqpoint{1.663818in}{4.075303in}}%
\pgfpathlineto{\pgfqpoint{1.672006in}{4.081826in}}%
\pgfpathlineto{\pgfqpoint{1.672006in}{4.070565in}}%
\pgfpathlineto{\pgfqpoint{1.672006in}{4.070565in}}%
\pgfpathlineto{\pgfqpoint{1.663818in}{4.070565in}}%
\pgfpathlineto{\pgfqpoint{1.655631in}{4.070565in}}%
\pgfpathlineto{\pgfqpoint{1.647444in}{4.070565in}}%
\pgfpathlineto{\pgfqpoint{1.639257in}{4.070565in}}%
\pgfpathlineto{\pgfqpoint{1.631070in}{4.070565in}}%
\pgfpathlineto{\pgfqpoint{1.622882in}{4.070565in}}%
\pgfpathlineto{\pgfqpoint{1.614695in}{4.070565in}}%
\pgfpathlineto{\pgfqpoint{1.606508in}{4.070565in}}%
\pgfpathlineto{\pgfqpoint{1.598321in}{4.070565in}}%
\pgfpathlineto{\pgfqpoint{1.590134in}{4.070565in}}%
\pgfpathlineto{\pgfqpoint{1.581946in}{4.070565in}}%
\pgfpathlineto{\pgfqpoint{1.573759in}{4.070565in}}%
\pgfpathlineto{\pgfqpoint{1.565572in}{4.070565in}}%
\pgfpathlineto{\pgfqpoint{1.557385in}{4.070565in}}%
\pgfpathlineto{\pgfqpoint{1.549198in}{4.070565in}}%
\pgfpathlineto{\pgfqpoint{1.541011in}{4.070565in}}%
\pgfpathlineto{\pgfqpoint{1.532823in}{4.070565in}}%
\pgfpathlineto{\pgfqpoint{1.524636in}{4.070565in}}%
\pgfpathlineto{\pgfqpoint{1.516449in}{4.070565in}}%
\pgfpathlineto{\pgfqpoint{1.508262in}{4.070565in}}%
\pgfpathlineto{\pgfqpoint{1.500075in}{4.070565in}}%
\pgfpathlineto{\pgfqpoint{1.491887in}{4.070565in}}%
\pgfpathlineto{\pgfqpoint{1.483700in}{4.070565in}}%
\pgfpathlineto{\pgfqpoint{1.475513in}{4.070565in}}%
\pgfpathlineto{\pgfqpoint{1.467326in}{4.070565in}}%
\pgfpathlineto{\pgfqpoint{1.459139in}{4.070565in}}%
\pgfpathlineto{\pgfqpoint{1.450951in}{4.070565in}}%
\pgfpathlineto{\pgfqpoint{1.442764in}{4.070565in}}%
\pgfpathlineto{\pgfqpoint{1.434577in}{4.070565in}}%
\pgfpathlineto{\pgfqpoint{1.426390in}{4.070565in}}%
\pgfpathlineto{\pgfqpoint{1.418203in}{4.070565in}}%
\pgfpathlineto{\pgfqpoint{1.410016in}{4.070565in}}%
\pgfpathlineto{\pgfqpoint{1.401828in}{4.070565in}}%
\pgfpathlineto{\pgfqpoint{1.393641in}{4.070565in}}%
\pgfpathlineto{\pgfqpoint{1.385454in}{4.070565in}}%
\pgfpathlineto{\pgfqpoint{1.377267in}{4.070565in}}%
\pgfpathlineto{\pgfqpoint{1.369080in}{4.070565in}}%
\pgfpathlineto{\pgfqpoint{1.360892in}{4.070565in}}%
\pgfpathlineto{\pgfqpoint{1.352705in}{4.070565in}}%
\pgfpathlineto{\pgfqpoint{1.344518in}{4.070565in}}%
\pgfpathlineto{\pgfqpoint{1.336331in}{4.070565in}}%
\pgfpathlineto{\pgfqpoint{1.328144in}{4.070565in}}%
\pgfpathlineto{\pgfqpoint{1.319956in}{4.070565in}}%
\pgfpathlineto{\pgfqpoint{1.311769in}{4.070565in}}%
\pgfpathlineto{\pgfqpoint{1.303582in}{4.070565in}}%
\pgfpathlineto{\pgfqpoint{1.295395in}{4.070565in}}%
\pgfpathlineto{\pgfqpoint{1.287208in}{4.070565in}}%
\pgfpathlineto{\pgfqpoint{1.279021in}{4.070565in}}%
\pgfpathlineto{\pgfqpoint{1.270833in}{4.070565in}}%
\pgfpathlineto{\pgfqpoint{1.270833in}{4.070565in}}%
\pgfpathclose%
\pgfusepath{stroke,fill}%
}%
\begin{pgfscope}%
\pgfsys@transformshift{0.000000in}{0.000000in}%
\pgfsys@useobject{currentmarker}{}%
\end{pgfscope}%
\end{pgfscope}%
\begin{pgfscope}%
\pgfpathrectangle{\pgfqpoint{0.625000in}{2.970000in}}{\pgfqpoint{3.875000in}{2.310000in}}%
\pgfusepath{clip}%
\pgfsetbuttcap%
\pgfsetroundjoin%
\definecolor{currentfill}{rgb}{0.900000,0.900000,0.900000}%
\pgfsetfillcolor{currentfill}%
\pgfsetlinewidth{1.003750pt}%
\definecolor{currentstroke}{rgb}{0.500000,0.500000,0.500000}%
\pgfsetstrokecolor{currentstroke}%
\pgfsetdash{}{0pt}%
\pgfsys@defobject{currentmarker}{\pgfqpoint{1.672006in}{4.070565in}}{\pgfqpoint{2.116048in}{4.355429in}}{%
\pgfpathmoveto{\pgfqpoint{1.672006in}{4.070565in}}%
\pgfpathlineto{\pgfqpoint{1.672006in}{4.081826in}}%
\pgfpathlineto{\pgfqpoint{1.681068in}{4.088988in}}%
\pgfpathlineto{\pgfqpoint{1.690130in}{4.096090in}}%
\pgfpathlineto{\pgfqpoint{1.699192in}{4.103132in}}%
\pgfpathlineto{\pgfqpoint{1.708254in}{4.110112in}}%
\pgfpathlineto{\pgfqpoint{1.717316in}{4.117031in}}%
\pgfpathlineto{\pgfqpoint{1.726378in}{4.123887in}}%
\pgfpathlineto{\pgfqpoint{1.735440in}{4.130682in}}%
\pgfpathlineto{\pgfqpoint{1.744502in}{4.137414in}}%
\pgfpathlineto{\pgfqpoint{1.753564in}{4.144083in}}%
\pgfpathlineto{\pgfqpoint{1.762626in}{4.150688in}}%
\pgfpathlineto{\pgfqpoint{1.771689in}{4.157230in}}%
\pgfpathlineto{\pgfqpoint{1.780751in}{4.163707in}}%
\pgfpathlineto{\pgfqpoint{1.789813in}{4.170121in}}%
\pgfpathlineto{\pgfqpoint{1.798875in}{4.176469in}}%
\pgfpathlineto{\pgfqpoint{1.807937in}{4.182752in}}%
\pgfpathlineto{\pgfqpoint{1.816999in}{4.188970in}}%
\pgfpathlineto{\pgfqpoint{1.826061in}{4.195122in}}%
\pgfpathlineto{\pgfqpoint{1.835123in}{4.201208in}}%
\pgfpathlineto{\pgfqpoint{1.844185in}{4.207227in}}%
\pgfpathlineto{\pgfqpoint{1.853247in}{4.213180in}}%
\pgfpathlineto{\pgfqpoint{1.862309in}{4.219066in}}%
\pgfpathlineto{\pgfqpoint{1.871371in}{4.224884in}}%
\pgfpathlineto{\pgfqpoint{1.880434in}{4.230634in}}%
\pgfpathlineto{\pgfqpoint{1.889496in}{4.236317in}}%
\pgfpathlineto{\pgfqpoint{1.898558in}{4.241931in}}%
\pgfpathlineto{\pgfqpoint{1.907620in}{4.247476in}}%
\pgfpathlineto{\pgfqpoint{1.916682in}{4.252952in}}%
\pgfpathlineto{\pgfqpoint{1.925744in}{4.258359in}}%
\pgfpathlineto{\pgfqpoint{1.934806in}{4.263697in}}%
\pgfpathlineto{\pgfqpoint{1.943868in}{4.268965in}}%
\pgfpathlineto{\pgfqpoint{1.952930in}{4.274162in}}%
\pgfpathlineto{\pgfqpoint{1.961992in}{4.279290in}}%
\pgfpathlineto{\pgfqpoint{1.971054in}{4.284346in}}%
\pgfpathlineto{\pgfqpoint{1.980117in}{4.289332in}}%
\pgfpathlineto{\pgfqpoint{1.989179in}{4.294246in}}%
\pgfpathlineto{\pgfqpoint{1.998241in}{4.299089in}}%
\pgfpathlineto{\pgfqpoint{2.007303in}{4.303861in}}%
\pgfpathlineto{\pgfqpoint{2.016365in}{4.308560in}}%
\pgfpathlineto{\pgfqpoint{2.025427in}{4.313187in}}%
\pgfpathlineto{\pgfqpoint{2.034489in}{4.317741in}}%
\pgfpathlineto{\pgfqpoint{2.043551in}{4.322223in}}%
\pgfpathlineto{\pgfqpoint{2.052613in}{4.326632in}}%
\pgfpathlineto{\pgfqpoint{2.061675in}{4.330967in}}%
\pgfpathlineto{\pgfqpoint{2.070737in}{4.335229in}}%
\pgfpathlineto{\pgfqpoint{2.079800in}{4.339417in}}%
\pgfpathlineto{\pgfqpoint{2.088862in}{4.343532in}}%
\pgfpathlineto{\pgfqpoint{2.097924in}{4.347572in}}%
\pgfpathlineto{\pgfqpoint{2.106986in}{4.351538in}}%
\pgfpathlineto{\pgfqpoint{2.116048in}{4.355429in}}%
\pgfpathlineto{\pgfqpoint{2.116048in}{4.070565in}}%
\pgfpathlineto{\pgfqpoint{2.116048in}{4.070565in}}%
\pgfpathlineto{\pgfqpoint{2.106986in}{4.070565in}}%
\pgfpathlineto{\pgfqpoint{2.097924in}{4.070565in}}%
\pgfpathlineto{\pgfqpoint{2.088862in}{4.070565in}}%
\pgfpathlineto{\pgfqpoint{2.079800in}{4.070565in}}%
\pgfpathlineto{\pgfqpoint{2.070737in}{4.070565in}}%
\pgfpathlineto{\pgfqpoint{2.061675in}{4.070565in}}%
\pgfpathlineto{\pgfqpoint{2.052613in}{4.070565in}}%
\pgfpathlineto{\pgfqpoint{2.043551in}{4.070565in}}%
\pgfpathlineto{\pgfqpoint{2.034489in}{4.070565in}}%
\pgfpathlineto{\pgfqpoint{2.025427in}{4.070565in}}%
\pgfpathlineto{\pgfqpoint{2.016365in}{4.070565in}}%
\pgfpathlineto{\pgfqpoint{2.007303in}{4.070565in}}%
\pgfpathlineto{\pgfqpoint{1.998241in}{4.070565in}}%
\pgfpathlineto{\pgfqpoint{1.989179in}{4.070565in}}%
\pgfpathlineto{\pgfqpoint{1.980117in}{4.070565in}}%
\pgfpathlineto{\pgfqpoint{1.971054in}{4.070565in}}%
\pgfpathlineto{\pgfqpoint{1.961992in}{4.070565in}}%
\pgfpathlineto{\pgfqpoint{1.952930in}{4.070565in}}%
\pgfpathlineto{\pgfqpoint{1.943868in}{4.070565in}}%
\pgfpathlineto{\pgfqpoint{1.934806in}{4.070565in}}%
\pgfpathlineto{\pgfqpoint{1.925744in}{4.070565in}}%
\pgfpathlineto{\pgfqpoint{1.916682in}{4.070565in}}%
\pgfpathlineto{\pgfqpoint{1.907620in}{4.070565in}}%
\pgfpathlineto{\pgfqpoint{1.898558in}{4.070565in}}%
\pgfpathlineto{\pgfqpoint{1.889496in}{4.070565in}}%
\pgfpathlineto{\pgfqpoint{1.880434in}{4.070565in}}%
\pgfpathlineto{\pgfqpoint{1.871371in}{4.070565in}}%
\pgfpathlineto{\pgfqpoint{1.862309in}{4.070565in}}%
\pgfpathlineto{\pgfqpoint{1.853247in}{4.070565in}}%
\pgfpathlineto{\pgfqpoint{1.844185in}{4.070565in}}%
\pgfpathlineto{\pgfqpoint{1.835123in}{4.070565in}}%
\pgfpathlineto{\pgfqpoint{1.826061in}{4.070565in}}%
\pgfpathlineto{\pgfqpoint{1.816999in}{4.070565in}}%
\pgfpathlineto{\pgfqpoint{1.807937in}{4.070565in}}%
\pgfpathlineto{\pgfqpoint{1.798875in}{4.070565in}}%
\pgfpathlineto{\pgfqpoint{1.789813in}{4.070565in}}%
\pgfpathlineto{\pgfqpoint{1.780751in}{4.070565in}}%
\pgfpathlineto{\pgfqpoint{1.771689in}{4.070565in}}%
\pgfpathlineto{\pgfqpoint{1.762626in}{4.070565in}}%
\pgfpathlineto{\pgfqpoint{1.753564in}{4.070565in}}%
\pgfpathlineto{\pgfqpoint{1.744502in}{4.070565in}}%
\pgfpathlineto{\pgfqpoint{1.735440in}{4.070565in}}%
\pgfpathlineto{\pgfqpoint{1.726378in}{4.070565in}}%
\pgfpathlineto{\pgfqpoint{1.717316in}{4.070565in}}%
\pgfpathlineto{\pgfqpoint{1.708254in}{4.070565in}}%
\pgfpathlineto{\pgfqpoint{1.699192in}{4.070565in}}%
\pgfpathlineto{\pgfqpoint{1.690130in}{4.070565in}}%
\pgfpathlineto{\pgfqpoint{1.681068in}{4.070565in}}%
\pgfpathlineto{\pgfqpoint{1.672006in}{4.070565in}}%
\pgfpathlineto{\pgfqpoint{1.672006in}{4.070565in}}%
\pgfpathclose%
\pgfusepath{stroke,fill}%
}%
\begin{pgfscope}%
\pgfsys@transformshift{0.000000in}{0.000000in}%
\pgfsys@useobject{currentmarker}{}%
\end{pgfscope}%
\end{pgfscope}%
\begin{pgfscope}%
\pgfpathrectangle{\pgfqpoint{0.625000in}{2.970000in}}{\pgfqpoint{3.875000in}{2.310000in}}%
\pgfusepath{clip}%
\pgfsetrectcap%
\pgfsetroundjoin%
\pgfsetlinewidth{0.803000pt}%
\definecolor{currentstroke}{rgb}{0.690196,0.690196,0.690196}%
\pgfsetstrokecolor{currentstroke}%
\pgfsetdash{}{0pt}%
\pgfpathmoveto{\pgfqpoint{0.625000in}{2.970000in}}%
\pgfpathlineto{\pgfqpoint{0.625000in}{5.280000in}}%
\pgfusepath{stroke}%
\end{pgfscope}%
\begin{pgfscope}%
\pgfsetbuttcap%
\pgfsetroundjoin%
\definecolor{currentfill}{rgb}{0.000000,0.000000,0.000000}%
\pgfsetfillcolor{currentfill}%
\pgfsetlinewidth{0.803000pt}%
\definecolor{currentstroke}{rgb}{0.000000,0.000000,0.000000}%
\pgfsetstrokecolor{currentstroke}%
\pgfsetdash{}{0pt}%
\pgfsys@defobject{currentmarker}{\pgfqpoint{0.000000in}{-0.048611in}}{\pgfqpoint{0.000000in}{0.000000in}}{%
\pgfpathmoveto{\pgfqpoint{0.000000in}{0.000000in}}%
\pgfpathlineto{\pgfqpoint{0.000000in}{-0.048611in}}%
\pgfusepath{stroke,fill}%
}%
\begin{pgfscope}%
\pgfsys@transformshift{0.625000in}{2.970000in}%
\pgfsys@useobject{currentmarker}{}%
\end{pgfscope}%
\end{pgfscope}%
\begin{pgfscope}%
\pgfpathrectangle{\pgfqpoint{0.625000in}{2.970000in}}{\pgfqpoint{3.875000in}{2.310000in}}%
\pgfusepath{clip}%
\pgfsetrectcap%
\pgfsetroundjoin%
\pgfsetlinewidth{0.803000pt}%
\definecolor{currentstroke}{rgb}{0.690196,0.690196,0.690196}%
\pgfsetstrokecolor{currentstroke}%
\pgfsetdash{}{0pt}%
\pgfpathmoveto{\pgfqpoint{1.055556in}{2.970000in}}%
\pgfpathlineto{\pgfqpoint{1.055556in}{5.280000in}}%
\pgfusepath{stroke}%
\end{pgfscope}%
\begin{pgfscope}%
\pgfsetbuttcap%
\pgfsetroundjoin%
\definecolor{currentfill}{rgb}{0.000000,0.000000,0.000000}%
\pgfsetfillcolor{currentfill}%
\pgfsetlinewidth{0.803000pt}%
\definecolor{currentstroke}{rgb}{0.000000,0.000000,0.000000}%
\pgfsetstrokecolor{currentstroke}%
\pgfsetdash{}{0pt}%
\pgfsys@defobject{currentmarker}{\pgfqpoint{0.000000in}{-0.048611in}}{\pgfqpoint{0.000000in}{0.000000in}}{%
\pgfpathmoveto{\pgfqpoint{0.000000in}{0.000000in}}%
\pgfpathlineto{\pgfqpoint{0.000000in}{-0.048611in}}%
\pgfusepath{stroke,fill}%
}%
\begin{pgfscope}%
\pgfsys@transformshift{1.055556in}{2.970000in}%
\pgfsys@useobject{currentmarker}{}%
\end{pgfscope}%
\end{pgfscope}%
\begin{pgfscope}%
\pgfpathrectangle{\pgfqpoint{0.625000in}{2.970000in}}{\pgfqpoint{3.875000in}{2.310000in}}%
\pgfusepath{clip}%
\pgfsetrectcap%
\pgfsetroundjoin%
\pgfsetlinewidth{0.803000pt}%
\definecolor{currentstroke}{rgb}{0.690196,0.690196,0.690196}%
\pgfsetstrokecolor{currentstroke}%
\pgfsetdash{}{0pt}%
\pgfpathmoveto{\pgfqpoint{1.486111in}{2.970000in}}%
\pgfpathlineto{\pgfqpoint{1.486111in}{5.280000in}}%
\pgfusepath{stroke}%
\end{pgfscope}%
\begin{pgfscope}%
\pgfsetbuttcap%
\pgfsetroundjoin%
\definecolor{currentfill}{rgb}{0.000000,0.000000,0.000000}%
\pgfsetfillcolor{currentfill}%
\pgfsetlinewidth{0.803000pt}%
\definecolor{currentstroke}{rgb}{0.000000,0.000000,0.000000}%
\pgfsetstrokecolor{currentstroke}%
\pgfsetdash{}{0pt}%
\pgfsys@defobject{currentmarker}{\pgfqpoint{0.000000in}{-0.048611in}}{\pgfqpoint{0.000000in}{0.000000in}}{%
\pgfpathmoveto{\pgfqpoint{0.000000in}{0.000000in}}%
\pgfpathlineto{\pgfqpoint{0.000000in}{-0.048611in}}%
\pgfusepath{stroke,fill}%
}%
\begin{pgfscope}%
\pgfsys@transformshift{1.486111in}{2.970000in}%
\pgfsys@useobject{currentmarker}{}%
\end{pgfscope}%
\end{pgfscope}%
\begin{pgfscope}%
\pgfpathrectangle{\pgfqpoint{0.625000in}{2.970000in}}{\pgfqpoint{3.875000in}{2.310000in}}%
\pgfusepath{clip}%
\pgfsetrectcap%
\pgfsetroundjoin%
\pgfsetlinewidth{0.803000pt}%
\definecolor{currentstroke}{rgb}{0.690196,0.690196,0.690196}%
\pgfsetstrokecolor{currentstroke}%
\pgfsetdash{}{0pt}%
\pgfpathmoveto{\pgfqpoint{1.916667in}{2.970000in}}%
\pgfpathlineto{\pgfqpoint{1.916667in}{5.280000in}}%
\pgfusepath{stroke}%
\end{pgfscope}%
\begin{pgfscope}%
\pgfsetbuttcap%
\pgfsetroundjoin%
\definecolor{currentfill}{rgb}{0.000000,0.000000,0.000000}%
\pgfsetfillcolor{currentfill}%
\pgfsetlinewidth{0.803000pt}%
\definecolor{currentstroke}{rgb}{0.000000,0.000000,0.000000}%
\pgfsetstrokecolor{currentstroke}%
\pgfsetdash{}{0pt}%
\pgfsys@defobject{currentmarker}{\pgfqpoint{0.000000in}{-0.048611in}}{\pgfqpoint{0.000000in}{0.000000in}}{%
\pgfpathmoveto{\pgfqpoint{0.000000in}{0.000000in}}%
\pgfpathlineto{\pgfqpoint{0.000000in}{-0.048611in}}%
\pgfusepath{stroke,fill}%
}%
\begin{pgfscope}%
\pgfsys@transformshift{1.916667in}{2.970000in}%
\pgfsys@useobject{currentmarker}{}%
\end{pgfscope}%
\end{pgfscope}%
\begin{pgfscope}%
\pgfpathrectangle{\pgfqpoint{0.625000in}{2.970000in}}{\pgfqpoint{3.875000in}{2.310000in}}%
\pgfusepath{clip}%
\pgfsetrectcap%
\pgfsetroundjoin%
\pgfsetlinewidth{0.803000pt}%
\definecolor{currentstroke}{rgb}{0.690196,0.690196,0.690196}%
\pgfsetstrokecolor{currentstroke}%
\pgfsetdash{}{0pt}%
\pgfpathmoveto{\pgfqpoint{2.347222in}{2.970000in}}%
\pgfpathlineto{\pgfqpoint{2.347222in}{5.280000in}}%
\pgfusepath{stroke}%
\end{pgfscope}%
\begin{pgfscope}%
\pgfsetbuttcap%
\pgfsetroundjoin%
\definecolor{currentfill}{rgb}{0.000000,0.000000,0.000000}%
\pgfsetfillcolor{currentfill}%
\pgfsetlinewidth{0.803000pt}%
\definecolor{currentstroke}{rgb}{0.000000,0.000000,0.000000}%
\pgfsetstrokecolor{currentstroke}%
\pgfsetdash{}{0pt}%
\pgfsys@defobject{currentmarker}{\pgfqpoint{0.000000in}{-0.048611in}}{\pgfqpoint{0.000000in}{0.000000in}}{%
\pgfpathmoveto{\pgfqpoint{0.000000in}{0.000000in}}%
\pgfpathlineto{\pgfqpoint{0.000000in}{-0.048611in}}%
\pgfusepath{stroke,fill}%
}%
\begin{pgfscope}%
\pgfsys@transformshift{2.347222in}{2.970000in}%
\pgfsys@useobject{currentmarker}{}%
\end{pgfscope}%
\end{pgfscope}%
\begin{pgfscope}%
\pgfpathrectangle{\pgfqpoint{0.625000in}{2.970000in}}{\pgfqpoint{3.875000in}{2.310000in}}%
\pgfusepath{clip}%
\pgfsetrectcap%
\pgfsetroundjoin%
\pgfsetlinewidth{0.803000pt}%
\definecolor{currentstroke}{rgb}{0.690196,0.690196,0.690196}%
\pgfsetstrokecolor{currentstroke}%
\pgfsetdash{}{0pt}%
\pgfpathmoveto{\pgfqpoint{2.777778in}{2.970000in}}%
\pgfpathlineto{\pgfqpoint{2.777778in}{5.280000in}}%
\pgfusepath{stroke}%
\end{pgfscope}%
\begin{pgfscope}%
\pgfsetbuttcap%
\pgfsetroundjoin%
\definecolor{currentfill}{rgb}{0.000000,0.000000,0.000000}%
\pgfsetfillcolor{currentfill}%
\pgfsetlinewidth{0.803000pt}%
\definecolor{currentstroke}{rgb}{0.000000,0.000000,0.000000}%
\pgfsetstrokecolor{currentstroke}%
\pgfsetdash{}{0pt}%
\pgfsys@defobject{currentmarker}{\pgfqpoint{0.000000in}{-0.048611in}}{\pgfqpoint{0.000000in}{0.000000in}}{%
\pgfpathmoveto{\pgfqpoint{0.000000in}{0.000000in}}%
\pgfpathlineto{\pgfqpoint{0.000000in}{-0.048611in}}%
\pgfusepath{stroke,fill}%
}%
\begin{pgfscope}%
\pgfsys@transformshift{2.777778in}{2.970000in}%
\pgfsys@useobject{currentmarker}{}%
\end{pgfscope}%
\end{pgfscope}%
\begin{pgfscope}%
\pgfpathrectangle{\pgfqpoint{0.625000in}{2.970000in}}{\pgfqpoint{3.875000in}{2.310000in}}%
\pgfusepath{clip}%
\pgfsetrectcap%
\pgfsetroundjoin%
\pgfsetlinewidth{0.803000pt}%
\definecolor{currentstroke}{rgb}{0.690196,0.690196,0.690196}%
\pgfsetstrokecolor{currentstroke}%
\pgfsetdash{}{0pt}%
\pgfpathmoveto{\pgfqpoint{3.208333in}{2.970000in}}%
\pgfpathlineto{\pgfqpoint{3.208333in}{5.280000in}}%
\pgfusepath{stroke}%
\end{pgfscope}%
\begin{pgfscope}%
\pgfsetbuttcap%
\pgfsetroundjoin%
\definecolor{currentfill}{rgb}{0.000000,0.000000,0.000000}%
\pgfsetfillcolor{currentfill}%
\pgfsetlinewidth{0.803000pt}%
\definecolor{currentstroke}{rgb}{0.000000,0.000000,0.000000}%
\pgfsetstrokecolor{currentstroke}%
\pgfsetdash{}{0pt}%
\pgfsys@defobject{currentmarker}{\pgfqpoint{0.000000in}{-0.048611in}}{\pgfqpoint{0.000000in}{0.000000in}}{%
\pgfpathmoveto{\pgfqpoint{0.000000in}{0.000000in}}%
\pgfpathlineto{\pgfqpoint{0.000000in}{-0.048611in}}%
\pgfusepath{stroke,fill}%
}%
\begin{pgfscope}%
\pgfsys@transformshift{3.208333in}{2.970000in}%
\pgfsys@useobject{currentmarker}{}%
\end{pgfscope}%
\end{pgfscope}%
\begin{pgfscope}%
\pgfpathrectangle{\pgfqpoint{0.625000in}{2.970000in}}{\pgfqpoint{3.875000in}{2.310000in}}%
\pgfusepath{clip}%
\pgfsetrectcap%
\pgfsetroundjoin%
\pgfsetlinewidth{0.803000pt}%
\definecolor{currentstroke}{rgb}{0.690196,0.690196,0.690196}%
\pgfsetstrokecolor{currentstroke}%
\pgfsetdash{}{0pt}%
\pgfpathmoveto{\pgfqpoint{3.638889in}{2.970000in}}%
\pgfpathlineto{\pgfqpoint{3.638889in}{5.280000in}}%
\pgfusepath{stroke}%
\end{pgfscope}%
\begin{pgfscope}%
\pgfsetbuttcap%
\pgfsetroundjoin%
\definecolor{currentfill}{rgb}{0.000000,0.000000,0.000000}%
\pgfsetfillcolor{currentfill}%
\pgfsetlinewidth{0.803000pt}%
\definecolor{currentstroke}{rgb}{0.000000,0.000000,0.000000}%
\pgfsetstrokecolor{currentstroke}%
\pgfsetdash{}{0pt}%
\pgfsys@defobject{currentmarker}{\pgfqpoint{0.000000in}{-0.048611in}}{\pgfqpoint{0.000000in}{0.000000in}}{%
\pgfpathmoveto{\pgfqpoint{0.000000in}{0.000000in}}%
\pgfpathlineto{\pgfqpoint{0.000000in}{-0.048611in}}%
\pgfusepath{stroke,fill}%
}%
\begin{pgfscope}%
\pgfsys@transformshift{3.638889in}{2.970000in}%
\pgfsys@useobject{currentmarker}{}%
\end{pgfscope}%
\end{pgfscope}%
\begin{pgfscope}%
\pgfpathrectangle{\pgfqpoint{0.625000in}{2.970000in}}{\pgfqpoint{3.875000in}{2.310000in}}%
\pgfusepath{clip}%
\pgfsetrectcap%
\pgfsetroundjoin%
\pgfsetlinewidth{0.803000pt}%
\definecolor{currentstroke}{rgb}{0.690196,0.690196,0.690196}%
\pgfsetstrokecolor{currentstroke}%
\pgfsetdash{}{0pt}%
\pgfpathmoveto{\pgfqpoint{4.069444in}{2.970000in}}%
\pgfpathlineto{\pgfqpoint{4.069444in}{5.280000in}}%
\pgfusepath{stroke}%
\end{pgfscope}%
\begin{pgfscope}%
\pgfsetbuttcap%
\pgfsetroundjoin%
\definecolor{currentfill}{rgb}{0.000000,0.000000,0.000000}%
\pgfsetfillcolor{currentfill}%
\pgfsetlinewidth{0.803000pt}%
\definecolor{currentstroke}{rgb}{0.000000,0.000000,0.000000}%
\pgfsetstrokecolor{currentstroke}%
\pgfsetdash{}{0pt}%
\pgfsys@defobject{currentmarker}{\pgfqpoint{0.000000in}{-0.048611in}}{\pgfqpoint{0.000000in}{0.000000in}}{%
\pgfpathmoveto{\pgfqpoint{0.000000in}{0.000000in}}%
\pgfpathlineto{\pgfqpoint{0.000000in}{-0.048611in}}%
\pgfusepath{stroke,fill}%
}%
\begin{pgfscope}%
\pgfsys@transformshift{4.069444in}{2.970000in}%
\pgfsys@useobject{currentmarker}{}%
\end{pgfscope}%
\end{pgfscope}%
\begin{pgfscope}%
\pgfpathrectangle{\pgfqpoint{0.625000in}{2.970000in}}{\pgfqpoint{3.875000in}{2.310000in}}%
\pgfusepath{clip}%
\pgfsetrectcap%
\pgfsetroundjoin%
\pgfsetlinewidth{0.803000pt}%
\definecolor{currentstroke}{rgb}{0.690196,0.690196,0.690196}%
\pgfsetstrokecolor{currentstroke}%
\pgfsetdash{}{0pt}%
\pgfpathmoveto{\pgfqpoint{4.500000in}{2.970000in}}%
\pgfpathlineto{\pgfqpoint{4.500000in}{5.280000in}}%
\pgfusepath{stroke}%
\end{pgfscope}%
\begin{pgfscope}%
\pgfsetbuttcap%
\pgfsetroundjoin%
\definecolor{currentfill}{rgb}{0.000000,0.000000,0.000000}%
\pgfsetfillcolor{currentfill}%
\pgfsetlinewidth{0.803000pt}%
\definecolor{currentstroke}{rgb}{0.000000,0.000000,0.000000}%
\pgfsetstrokecolor{currentstroke}%
\pgfsetdash{}{0pt}%
\pgfsys@defobject{currentmarker}{\pgfqpoint{0.000000in}{-0.048611in}}{\pgfqpoint{0.000000in}{0.000000in}}{%
\pgfpathmoveto{\pgfqpoint{0.000000in}{0.000000in}}%
\pgfpathlineto{\pgfqpoint{0.000000in}{-0.048611in}}%
\pgfusepath{stroke,fill}%
}%
\begin{pgfscope}%
\pgfsys@transformshift{4.500000in}{2.970000in}%
\pgfsys@useobject{currentmarker}{}%
\end{pgfscope}%
\end{pgfscope}%
\begin{pgfscope}%
\pgfpathrectangle{\pgfqpoint{0.625000in}{2.970000in}}{\pgfqpoint{3.875000in}{2.310000in}}%
\pgfusepath{clip}%
\pgfsetrectcap%
\pgfsetroundjoin%
\pgfsetlinewidth{0.803000pt}%
\definecolor{currentstroke}{rgb}{0.690196,0.690196,0.690196}%
\pgfsetstrokecolor{currentstroke}%
\pgfsetdash{}{0pt}%
\pgfpathmoveto{\pgfqpoint{0.625000in}{2.970000in}}%
\pgfpathlineto{\pgfqpoint{4.500000in}{2.970000in}}%
\pgfusepath{stroke}%
\end{pgfscope}%
\begin{pgfscope}%
\pgfsetbuttcap%
\pgfsetroundjoin%
\definecolor{currentfill}{rgb}{0.000000,0.000000,0.000000}%
\pgfsetfillcolor{currentfill}%
\pgfsetlinewidth{0.803000pt}%
\definecolor{currentstroke}{rgb}{0.000000,0.000000,0.000000}%
\pgfsetstrokecolor{currentstroke}%
\pgfsetdash{}{0pt}%
\pgfsys@defobject{currentmarker}{\pgfqpoint{-0.048611in}{0.000000in}}{\pgfqpoint{-0.000000in}{0.000000in}}{%
\pgfpathmoveto{\pgfqpoint{-0.000000in}{0.000000in}}%
\pgfpathlineto{\pgfqpoint{-0.048611in}{0.000000in}}%
\pgfusepath{stroke,fill}%
}%
\begin{pgfscope}%
\pgfsys@transformshift{0.625000in}{2.970000in}%
\pgfsys@useobject{currentmarker}{}%
\end{pgfscope}%
\end{pgfscope}%
\begin{pgfscope}%
\definecolor{textcolor}{rgb}{0.000000,0.000000,0.000000}%
\pgfsetstrokecolor{textcolor}%
\pgfsetfillcolor{textcolor}%
\pgftext[x=0.350308in, y=2.918900in, left, base]{\color{textcolor}\rmfamily\fontsize{10.000000}{12.000000}\selectfont \(\displaystyle {0.0}\)}%
\end{pgfscope}%
\begin{pgfscope}%
\pgfpathrectangle{\pgfqpoint{0.625000in}{2.970000in}}{\pgfqpoint{3.875000in}{2.310000in}}%
\pgfusepath{clip}%
\pgfsetrectcap%
\pgfsetroundjoin%
\pgfsetlinewidth{0.803000pt}%
\definecolor{currentstroke}{rgb}{0.690196,0.690196,0.690196}%
\pgfsetstrokecolor{currentstroke}%
\pgfsetdash{}{0pt}%
\pgfpathmoveto{\pgfqpoint{0.625000in}{3.336855in}}%
\pgfpathlineto{\pgfqpoint{4.500000in}{3.336855in}}%
\pgfusepath{stroke}%
\end{pgfscope}%
\begin{pgfscope}%
\pgfsetbuttcap%
\pgfsetroundjoin%
\definecolor{currentfill}{rgb}{0.000000,0.000000,0.000000}%
\pgfsetfillcolor{currentfill}%
\pgfsetlinewidth{0.803000pt}%
\definecolor{currentstroke}{rgb}{0.000000,0.000000,0.000000}%
\pgfsetstrokecolor{currentstroke}%
\pgfsetdash{}{0pt}%
\pgfsys@defobject{currentmarker}{\pgfqpoint{-0.048611in}{0.000000in}}{\pgfqpoint{-0.000000in}{0.000000in}}{%
\pgfpathmoveto{\pgfqpoint{-0.000000in}{0.000000in}}%
\pgfpathlineto{\pgfqpoint{-0.048611in}{0.000000in}}%
\pgfusepath{stroke,fill}%
}%
\begin{pgfscope}%
\pgfsys@transformshift{0.625000in}{3.336855in}%
\pgfsys@useobject{currentmarker}{}%
\end{pgfscope}%
\end{pgfscope}%
\begin{pgfscope}%
\definecolor{textcolor}{rgb}{0.000000,0.000000,0.000000}%
\pgfsetstrokecolor{textcolor}%
\pgfsetfillcolor{textcolor}%
\pgftext[x=0.350308in, y=3.285755in, left, base]{\color{textcolor}\rmfamily\fontsize{10.000000}{12.000000}\selectfont \(\displaystyle {0.2}\)}%
\end{pgfscope}%
\begin{pgfscope}%
\pgfpathrectangle{\pgfqpoint{0.625000in}{2.970000in}}{\pgfqpoint{3.875000in}{2.310000in}}%
\pgfusepath{clip}%
\pgfsetrectcap%
\pgfsetroundjoin%
\pgfsetlinewidth{0.803000pt}%
\definecolor{currentstroke}{rgb}{0.690196,0.690196,0.690196}%
\pgfsetstrokecolor{currentstroke}%
\pgfsetdash{}{0pt}%
\pgfpathmoveto{\pgfqpoint{0.625000in}{3.703710in}}%
\pgfpathlineto{\pgfqpoint{4.500000in}{3.703710in}}%
\pgfusepath{stroke}%
\end{pgfscope}%
\begin{pgfscope}%
\pgfsetbuttcap%
\pgfsetroundjoin%
\definecolor{currentfill}{rgb}{0.000000,0.000000,0.000000}%
\pgfsetfillcolor{currentfill}%
\pgfsetlinewidth{0.803000pt}%
\definecolor{currentstroke}{rgb}{0.000000,0.000000,0.000000}%
\pgfsetstrokecolor{currentstroke}%
\pgfsetdash{}{0pt}%
\pgfsys@defobject{currentmarker}{\pgfqpoint{-0.048611in}{0.000000in}}{\pgfqpoint{-0.000000in}{0.000000in}}{%
\pgfpathmoveto{\pgfqpoint{-0.000000in}{0.000000in}}%
\pgfpathlineto{\pgfqpoint{-0.048611in}{0.000000in}}%
\pgfusepath{stroke,fill}%
}%
\begin{pgfscope}%
\pgfsys@transformshift{0.625000in}{3.703710in}%
\pgfsys@useobject{currentmarker}{}%
\end{pgfscope}%
\end{pgfscope}%
\begin{pgfscope}%
\definecolor{textcolor}{rgb}{0.000000,0.000000,0.000000}%
\pgfsetstrokecolor{textcolor}%
\pgfsetfillcolor{textcolor}%
\pgftext[x=0.350308in, y=3.652610in, left, base]{\color{textcolor}\rmfamily\fontsize{10.000000}{12.000000}\selectfont \(\displaystyle {0.4}\)}%
\end{pgfscope}%
\begin{pgfscope}%
\pgfpathrectangle{\pgfqpoint{0.625000in}{2.970000in}}{\pgfqpoint{3.875000in}{2.310000in}}%
\pgfusepath{clip}%
\pgfsetrectcap%
\pgfsetroundjoin%
\pgfsetlinewidth{0.803000pt}%
\definecolor{currentstroke}{rgb}{0.690196,0.690196,0.690196}%
\pgfsetstrokecolor{currentstroke}%
\pgfsetdash{}{0pt}%
\pgfpathmoveto{\pgfqpoint{0.625000in}{4.070565in}}%
\pgfpathlineto{\pgfqpoint{4.500000in}{4.070565in}}%
\pgfusepath{stroke}%
\end{pgfscope}%
\begin{pgfscope}%
\pgfsetbuttcap%
\pgfsetroundjoin%
\definecolor{currentfill}{rgb}{0.000000,0.000000,0.000000}%
\pgfsetfillcolor{currentfill}%
\pgfsetlinewidth{0.803000pt}%
\definecolor{currentstroke}{rgb}{0.000000,0.000000,0.000000}%
\pgfsetstrokecolor{currentstroke}%
\pgfsetdash{}{0pt}%
\pgfsys@defobject{currentmarker}{\pgfqpoint{-0.048611in}{0.000000in}}{\pgfqpoint{-0.000000in}{0.000000in}}{%
\pgfpathmoveto{\pgfqpoint{-0.000000in}{0.000000in}}%
\pgfpathlineto{\pgfqpoint{-0.048611in}{0.000000in}}%
\pgfusepath{stroke,fill}%
}%
\begin{pgfscope}%
\pgfsys@transformshift{0.625000in}{4.070565in}%
\pgfsys@useobject{currentmarker}{}%
\end{pgfscope}%
\end{pgfscope}%
\begin{pgfscope}%
\definecolor{textcolor}{rgb}{0.000000,0.000000,0.000000}%
\pgfsetstrokecolor{textcolor}%
\pgfsetfillcolor{textcolor}%
\pgftext[x=0.350308in, y=4.019465in, left, base]{\color{textcolor}\rmfamily\fontsize{10.000000}{12.000000}\selectfont \(\displaystyle {0.6}\)}%
\end{pgfscope}%
\begin{pgfscope}%
\pgfpathrectangle{\pgfqpoint{0.625000in}{2.970000in}}{\pgfqpoint{3.875000in}{2.310000in}}%
\pgfusepath{clip}%
\pgfsetrectcap%
\pgfsetroundjoin%
\pgfsetlinewidth{0.803000pt}%
\definecolor{currentstroke}{rgb}{0.690196,0.690196,0.690196}%
\pgfsetstrokecolor{currentstroke}%
\pgfsetdash{}{0pt}%
\pgfpathmoveto{\pgfqpoint{0.625000in}{4.437421in}}%
\pgfpathlineto{\pgfqpoint{4.500000in}{4.437421in}}%
\pgfusepath{stroke}%
\end{pgfscope}%
\begin{pgfscope}%
\pgfsetbuttcap%
\pgfsetroundjoin%
\definecolor{currentfill}{rgb}{0.000000,0.000000,0.000000}%
\pgfsetfillcolor{currentfill}%
\pgfsetlinewidth{0.803000pt}%
\definecolor{currentstroke}{rgb}{0.000000,0.000000,0.000000}%
\pgfsetstrokecolor{currentstroke}%
\pgfsetdash{}{0pt}%
\pgfsys@defobject{currentmarker}{\pgfqpoint{-0.048611in}{0.000000in}}{\pgfqpoint{-0.000000in}{0.000000in}}{%
\pgfpathmoveto{\pgfqpoint{-0.000000in}{0.000000in}}%
\pgfpathlineto{\pgfqpoint{-0.048611in}{0.000000in}}%
\pgfusepath{stroke,fill}%
}%
\begin{pgfscope}%
\pgfsys@transformshift{0.625000in}{4.437421in}%
\pgfsys@useobject{currentmarker}{}%
\end{pgfscope}%
\end{pgfscope}%
\begin{pgfscope}%
\definecolor{textcolor}{rgb}{0.000000,0.000000,0.000000}%
\pgfsetstrokecolor{textcolor}%
\pgfsetfillcolor{textcolor}%
\pgftext[x=0.350308in, y=4.386321in, left, base]{\color{textcolor}\rmfamily\fontsize{10.000000}{12.000000}\selectfont \(\displaystyle {0.8}\)}%
\end{pgfscope}%
\begin{pgfscope}%
\pgfpathrectangle{\pgfqpoint{0.625000in}{2.970000in}}{\pgfqpoint{3.875000in}{2.310000in}}%
\pgfusepath{clip}%
\pgfsetrectcap%
\pgfsetroundjoin%
\pgfsetlinewidth{0.803000pt}%
\definecolor{currentstroke}{rgb}{0.690196,0.690196,0.690196}%
\pgfsetstrokecolor{currentstroke}%
\pgfsetdash{}{0pt}%
\pgfpathmoveto{\pgfqpoint{0.625000in}{4.804276in}}%
\pgfpathlineto{\pgfqpoint{4.500000in}{4.804276in}}%
\pgfusepath{stroke}%
\end{pgfscope}%
\begin{pgfscope}%
\pgfsetbuttcap%
\pgfsetroundjoin%
\definecolor{currentfill}{rgb}{0.000000,0.000000,0.000000}%
\pgfsetfillcolor{currentfill}%
\pgfsetlinewidth{0.803000pt}%
\definecolor{currentstroke}{rgb}{0.000000,0.000000,0.000000}%
\pgfsetstrokecolor{currentstroke}%
\pgfsetdash{}{0pt}%
\pgfsys@defobject{currentmarker}{\pgfqpoint{-0.048611in}{0.000000in}}{\pgfqpoint{-0.000000in}{0.000000in}}{%
\pgfpathmoveto{\pgfqpoint{-0.000000in}{0.000000in}}%
\pgfpathlineto{\pgfqpoint{-0.048611in}{0.000000in}}%
\pgfusepath{stroke,fill}%
}%
\begin{pgfscope}%
\pgfsys@transformshift{0.625000in}{4.804276in}%
\pgfsys@useobject{currentmarker}{}%
\end{pgfscope}%
\end{pgfscope}%
\begin{pgfscope}%
\definecolor{textcolor}{rgb}{0.000000,0.000000,0.000000}%
\pgfsetstrokecolor{textcolor}%
\pgfsetfillcolor{textcolor}%
\pgftext[x=0.350308in, y=4.753176in, left, base]{\color{textcolor}\rmfamily\fontsize{10.000000}{12.000000}\selectfont \(\displaystyle {1.0}\)}%
\end{pgfscope}%
\begin{pgfscope}%
\pgfpathrectangle{\pgfqpoint{0.625000in}{2.970000in}}{\pgfqpoint{3.875000in}{2.310000in}}%
\pgfusepath{clip}%
\pgfsetrectcap%
\pgfsetroundjoin%
\pgfsetlinewidth{0.803000pt}%
\definecolor{currentstroke}{rgb}{0.690196,0.690196,0.690196}%
\pgfsetstrokecolor{currentstroke}%
\pgfsetdash{}{0pt}%
\pgfpathmoveto{\pgfqpoint{0.625000in}{5.171131in}}%
\pgfpathlineto{\pgfqpoint{4.500000in}{5.171131in}}%
\pgfusepath{stroke}%
\end{pgfscope}%
\begin{pgfscope}%
\pgfsetbuttcap%
\pgfsetroundjoin%
\definecolor{currentfill}{rgb}{0.000000,0.000000,0.000000}%
\pgfsetfillcolor{currentfill}%
\pgfsetlinewidth{0.803000pt}%
\definecolor{currentstroke}{rgb}{0.000000,0.000000,0.000000}%
\pgfsetstrokecolor{currentstroke}%
\pgfsetdash{}{0pt}%
\pgfsys@defobject{currentmarker}{\pgfqpoint{-0.048611in}{0.000000in}}{\pgfqpoint{-0.000000in}{0.000000in}}{%
\pgfpathmoveto{\pgfqpoint{-0.000000in}{0.000000in}}%
\pgfpathlineto{\pgfqpoint{-0.048611in}{0.000000in}}%
\pgfusepath{stroke,fill}%
}%
\begin{pgfscope}%
\pgfsys@transformshift{0.625000in}{5.171131in}%
\pgfsys@useobject{currentmarker}{}%
\end{pgfscope}%
\end{pgfscope}%
\begin{pgfscope}%
\definecolor{textcolor}{rgb}{0.000000,0.000000,0.000000}%
\pgfsetstrokecolor{textcolor}%
\pgfsetfillcolor{textcolor}%
\pgftext[x=0.350308in, y=5.120031in, left, base]{\color{textcolor}\rmfamily\fontsize{10.000000}{12.000000}\selectfont \(\displaystyle {1.2}\)}%
\end{pgfscope}%
\begin{pgfscope}%
\definecolor{textcolor}{rgb}{0.000000,0.000000,0.000000}%
\pgfsetstrokecolor{textcolor}%
\pgfsetfillcolor{textcolor}%
\pgftext[x=0.294753in,y=4.125000in,,bottom,rotate=90.000000]{\color{textcolor}\rmfamily\fontsize{10.000000}{12.000000}\selectfont power in p.u.}%
\end{pgfscope}%
\begin{pgfscope}%
\pgfpathrectangle{\pgfqpoint{0.625000in}{2.970000in}}{\pgfqpoint{3.875000in}{2.310000in}}%
\pgfusepath{clip}%
\pgfsetrectcap%
\pgfsetroundjoin%
\pgfsetlinewidth{2.007500pt}%
\definecolor{currentstroke}{rgb}{0.121569,0.466667,0.705882}%
\pgfsetstrokecolor{currentstroke}%
\pgfsetdash{}{0pt}%
\pgfpathmoveto{\pgfqpoint{0.625000in}{2.970000in}}%
\pgfpathlineto{\pgfqpoint{0.704082in}{3.111027in}}%
\pgfpathlineto{\pgfqpoint{0.783163in}{3.251474in}}%
\pgfpathlineto{\pgfqpoint{0.862245in}{3.390765in}}%
\pgfpathlineto{\pgfqpoint{0.941327in}{3.528327in}}%
\pgfpathlineto{\pgfqpoint{1.020408in}{3.663594in}}%
\pgfpathlineto{\pgfqpoint{1.099490in}{3.796012in}}%
\pgfpathlineto{\pgfqpoint{1.178571in}{3.925035in}}%
\pgfpathlineto{\pgfqpoint{1.257653in}{4.050134in}}%
\pgfpathlineto{\pgfqpoint{1.336735in}{4.170794in}}%
\pgfpathlineto{\pgfqpoint{1.415816in}{4.286520in}}%
\pgfpathlineto{\pgfqpoint{1.494898in}{4.396836in}}%
\pgfpathlineto{\pgfqpoint{1.573980in}{4.501288in}}%
\pgfpathlineto{\pgfqpoint{1.653061in}{4.599449in}}%
\pgfpathlineto{\pgfqpoint{1.732143in}{4.690913in}}%
\pgfpathlineto{\pgfqpoint{1.811224in}{4.775306in}}%
\pgfpathlineto{\pgfqpoint{1.890306in}{4.852281in}}%
\pgfpathlineto{\pgfqpoint{1.969388in}{4.921521in}}%
\pgfpathlineto{\pgfqpoint{2.048469in}{4.982742in}}%
\pgfpathlineto{\pgfqpoint{2.127551in}{5.035692in}}%
\pgfpathlineto{\pgfqpoint{2.206633in}{5.080153in}}%
\pgfpathlineto{\pgfqpoint{2.285714in}{5.115944in}}%
\pgfpathlineto{\pgfqpoint{2.364796in}{5.142916in}}%
\pgfpathlineto{\pgfqpoint{2.443878in}{5.160960in}}%
\pgfpathlineto{\pgfqpoint{2.522959in}{5.170000in}}%
\pgfpathlineto{\pgfqpoint{2.602041in}{5.170000in}}%
\pgfpathlineto{\pgfqpoint{2.681122in}{5.160960in}}%
\pgfpathlineto{\pgfqpoint{2.760204in}{5.142916in}}%
\pgfpathlineto{\pgfqpoint{2.839286in}{5.115944in}}%
\pgfpathlineto{\pgfqpoint{2.918367in}{5.080153in}}%
\pgfpathlineto{\pgfqpoint{2.997449in}{5.035692in}}%
\pgfpathlineto{\pgfqpoint{3.076531in}{4.982742in}}%
\pgfpathlineto{\pgfqpoint{3.155612in}{4.921521in}}%
\pgfpathlineto{\pgfqpoint{3.234694in}{4.852281in}}%
\pgfpathlineto{\pgfqpoint{3.313776in}{4.775306in}}%
\pgfpathlineto{\pgfqpoint{3.392857in}{4.690913in}}%
\pgfpathlineto{\pgfqpoint{3.471939in}{4.599449in}}%
\pgfpathlineto{\pgfqpoint{3.551020in}{4.501288in}}%
\pgfpathlineto{\pgfqpoint{3.630102in}{4.396836in}}%
\pgfpathlineto{\pgfqpoint{3.709184in}{4.286520in}}%
\pgfpathlineto{\pgfqpoint{3.788265in}{4.170794in}}%
\pgfpathlineto{\pgfqpoint{3.867347in}{4.050134in}}%
\pgfpathlineto{\pgfqpoint{3.946429in}{3.925035in}}%
\pgfpathlineto{\pgfqpoint{4.025510in}{3.796012in}}%
\pgfpathlineto{\pgfqpoint{4.104592in}{3.663594in}}%
\pgfpathlineto{\pgfqpoint{4.183673in}{3.528327in}}%
\pgfpathlineto{\pgfqpoint{4.262755in}{3.390765in}}%
\pgfpathlineto{\pgfqpoint{4.341837in}{3.251474in}}%
\pgfpathlineto{\pgfqpoint{4.420918in}{3.111027in}}%
\pgfpathlineto{\pgfqpoint{4.500000in}{2.970000in}}%
\pgfusepath{stroke}%
\end{pgfscope}%
\begin{pgfscope}%
\pgfpathrectangle{\pgfqpoint{0.625000in}{2.970000in}}{\pgfqpoint{3.875000in}{2.310000in}}%
\pgfusepath{clip}%
\pgfsetrectcap%
\pgfsetroundjoin%
\pgfsetlinewidth{2.007500pt}%
\definecolor{currentstroke}{rgb}{1.000000,0.498039,0.054902}%
\pgfsetstrokecolor{currentstroke}%
\pgfsetdash{}{0pt}%
\pgfpathmoveto{\pgfqpoint{0.625000in}{2.970000in}}%
\pgfpathlineto{\pgfqpoint{0.704082in}{3.064915in}}%
\pgfpathlineto{\pgfqpoint{0.783163in}{3.159439in}}%
\pgfpathlineto{\pgfqpoint{0.862245in}{3.253185in}}%
\pgfpathlineto{\pgfqpoint{0.941327in}{3.345768in}}%
\pgfpathlineto{\pgfqpoint{1.020408in}{3.436806in}}%
\pgfpathlineto{\pgfqpoint{1.099490in}{3.525926in}}%
\pgfpathlineto{\pgfqpoint{1.178571in}{3.612762in}}%
\pgfpathlineto{\pgfqpoint{1.257653in}{3.696956in}}%
\pgfpathlineto{\pgfqpoint{1.336735in}{3.778164in}}%
\pgfpathlineto{\pgfqpoint{1.415816in}{3.856050in}}%
\pgfpathlineto{\pgfqpoint{1.494898in}{3.930295in}}%
\pgfpathlineto{\pgfqpoint{1.573980in}{4.000595in}}%
\pgfpathlineto{\pgfqpoint{1.653061in}{4.066659in}}%
\pgfpathlineto{\pgfqpoint{1.732143in}{4.128217in}}%
\pgfpathlineto{\pgfqpoint{1.811224in}{4.185016in}}%
\pgfpathlineto{\pgfqpoint{1.890306in}{4.236821in}}%
\pgfpathlineto{\pgfqpoint{1.969388in}{4.283422in}}%
\pgfpathlineto{\pgfqpoint{2.048469in}{4.324625in}}%
\pgfpathlineto{\pgfqpoint{2.127551in}{4.360261in}}%
\pgfpathlineto{\pgfqpoint{2.206633in}{4.390185in}}%
\pgfpathlineto{\pgfqpoint{2.285714in}{4.414273in}}%
\pgfpathlineto{\pgfqpoint{2.364796in}{4.432426in}}%
\pgfpathlineto{\pgfqpoint{2.443878in}{4.444570in}}%
\pgfpathlineto{\pgfqpoint{2.522959in}{4.450654in}}%
\pgfpathlineto{\pgfqpoint{2.602041in}{4.450654in}}%
\pgfpathlineto{\pgfqpoint{2.681122in}{4.444570in}}%
\pgfpathlineto{\pgfqpoint{2.760204in}{4.432426in}}%
\pgfpathlineto{\pgfqpoint{2.839286in}{4.414273in}}%
\pgfpathlineto{\pgfqpoint{2.918367in}{4.390185in}}%
\pgfpathlineto{\pgfqpoint{2.997449in}{4.360261in}}%
\pgfpathlineto{\pgfqpoint{3.076531in}{4.324625in}}%
\pgfpathlineto{\pgfqpoint{3.155612in}{4.283422in}}%
\pgfpathlineto{\pgfqpoint{3.234694in}{4.236821in}}%
\pgfpathlineto{\pgfqpoint{3.313776in}{4.185016in}}%
\pgfpathlineto{\pgfqpoint{3.392857in}{4.128217in}}%
\pgfpathlineto{\pgfqpoint{3.471939in}{4.066659in}}%
\pgfpathlineto{\pgfqpoint{3.551020in}{4.000595in}}%
\pgfpathlineto{\pgfqpoint{3.630102in}{3.930295in}}%
\pgfpathlineto{\pgfqpoint{3.709184in}{3.856050in}}%
\pgfpathlineto{\pgfqpoint{3.788265in}{3.778164in}}%
\pgfpathlineto{\pgfqpoint{3.867347in}{3.696956in}}%
\pgfpathlineto{\pgfqpoint{3.946429in}{3.612762in}}%
\pgfpathlineto{\pgfqpoint{4.025510in}{3.525926in}}%
\pgfpathlineto{\pgfqpoint{4.104592in}{3.436806in}}%
\pgfpathlineto{\pgfqpoint{4.183673in}{3.345768in}}%
\pgfpathlineto{\pgfqpoint{4.262755in}{3.253185in}}%
\pgfpathlineto{\pgfqpoint{4.341837in}{3.159439in}}%
\pgfpathlineto{\pgfqpoint{4.420918in}{3.064915in}}%
\pgfpathlineto{\pgfqpoint{4.500000in}{2.970000in}}%
\pgfusepath{stroke}%
\end{pgfscope}%
\begin{pgfscope}%
\pgfpathrectangle{\pgfqpoint{0.625000in}{2.970000in}}{\pgfqpoint{3.875000in}{2.310000in}}%
\pgfusepath{clip}%
\pgfsetrectcap%
\pgfsetroundjoin%
\pgfsetlinewidth{2.007500pt}%
\definecolor{currentstroke}{rgb}{0.172549,0.627451,0.172549}%
\pgfsetstrokecolor{currentstroke}%
\pgfsetdash{}{0pt}%
\pgfpathmoveto{\pgfqpoint{0.625000in}{4.070565in}}%
\pgfpathlineto{\pgfqpoint{0.704082in}{4.070565in}}%
\pgfpathlineto{\pgfqpoint{0.783163in}{4.070565in}}%
\pgfpathlineto{\pgfqpoint{0.862245in}{4.070565in}}%
\pgfpathlineto{\pgfqpoint{0.941327in}{4.070565in}}%
\pgfpathlineto{\pgfqpoint{1.020408in}{4.070565in}}%
\pgfpathlineto{\pgfqpoint{1.099490in}{4.070565in}}%
\pgfpathlineto{\pgfqpoint{1.178571in}{4.070565in}}%
\pgfpathlineto{\pgfqpoint{1.257653in}{4.070565in}}%
\pgfpathlineto{\pgfqpoint{1.336735in}{4.070565in}}%
\pgfpathlineto{\pgfqpoint{1.415816in}{4.070565in}}%
\pgfpathlineto{\pgfqpoint{1.494898in}{4.070565in}}%
\pgfpathlineto{\pgfqpoint{1.573980in}{4.070565in}}%
\pgfpathlineto{\pgfqpoint{1.653061in}{4.070565in}}%
\pgfpathlineto{\pgfqpoint{1.732143in}{4.070565in}}%
\pgfpathlineto{\pgfqpoint{1.811224in}{4.070565in}}%
\pgfpathlineto{\pgfqpoint{1.890306in}{4.070565in}}%
\pgfpathlineto{\pgfqpoint{1.969388in}{4.070565in}}%
\pgfpathlineto{\pgfqpoint{2.048469in}{4.070565in}}%
\pgfpathlineto{\pgfqpoint{2.127551in}{4.070565in}}%
\pgfpathlineto{\pgfqpoint{2.206633in}{4.070565in}}%
\pgfpathlineto{\pgfqpoint{2.285714in}{4.070565in}}%
\pgfpathlineto{\pgfqpoint{2.364796in}{4.070565in}}%
\pgfpathlineto{\pgfqpoint{2.443878in}{4.070565in}}%
\pgfpathlineto{\pgfqpoint{2.522959in}{4.070565in}}%
\pgfpathlineto{\pgfqpoint{2.602041in}{4.070565in}}%
\pgfpathlineto{\pgfqpoint{2.681122in}{4.070565in}}%
\pgfpathlineto{\pgfqpoint{2.760204in}{4.070565in}}%
\pgfpathlineto{\pgfqpoint{2.839286in}{4.070565in}}%
\pgfpathlineto{\pgfqpoint{2.918367in}{4.070565in}}%
\pgfpathlineto{\pgfqpoint{2.997449in}{4.070565in}}%
\pgfpathlineto{\pgfqpoint{3.076531in}{4.070565in}}%
\pgfpathlineto{\pgfqpoint{3.155612in}{4.070565in}}%
\pgfpathlineto{\pgfqpoint{3.234694in}{4.070565in}}%
\pgfpathlineto{\pgfqpoint{3.313776in}{4.070565in}}%
\pgfpathlineto{\pgfqpoint{3.392857in}{4.070565in}}%
\pgfpathlineto{\pgfqpoint{3.471939in}{4.070565in}}%
\pgfpathlineto{\pgfqpoint{3.551020in}{4.070565in}}%
\pgfpathlineto{\pgfqpoint{3.630102in}{4.070565in}}%
\pgfpathlineto{\pgfqpoint{3.709184in}{4.070565in}}%
\pgfpathlineto{\pgfqpoint{3.788265in}{4.070565in}}%
\pgfpathlineto{\pgfqpoint{3.867347in}{4.070565in}}%
\pgfpathlineto{\pgfqpoint{3.946429in}{4.070565in}}%
\pgfpathlineto{\pgfqpoint{4.025510in}{4.070565in}}%
\pgfpathlineto{\pgfqpoint{4.104592in}{4.070565in}}%
\pgfpathlineto{\pgfqpoint{4.183673in}{4.070565in}}%
\pgfpathlineto{\pgfqpoint{4.262755in}{4.070565in}}%
\pgfpathlineto{\pgfqpoint{4.341837in}{4.070565in}}%
\pgfpathlineto{\pgfqpoint{4.420918in}{4.070565in}}%
\pgfpathlineto{\pgfqpoint{4.500000in}{4.070565in}}%
\pgfusepath{stroke}%
\end{pgfscope}%
\begin{pgfscope}%
\pgfsetrectcap%
\pgfsetmiterjoin%
\pgfsetlinewidth{0.803000pt}%
\definecolor{currentstroke}{rgb}{0.000000,0.000000,0.000000}%
\pgfsetstrokecolor{currentstroke}%
\pgfsetdash{}{0pt}%
\pgfpathmoveto{\pgfqpoint{0.625000in}{2.970000in}}%
\pgfpathlineto{\pgfqpoint{0.625000in}{5.280000in}}%
\pgfusepath{stroke}%
\end{pgfscope}%
\begin{pgfscope}%
\pgfsetrectcap%
\pgfsetmiterjoin%
\pgfsetlinewidth{0.803000pt}%
\definecolor{currentstroke}{rgb}{0.000000,0.000000,0.000000}%
\pgfsetstrokecolor{currentstroke}%
\pgfsetdash{}{0pt}%
\pgfpathmoveto{\pgfqpoint{4.500000in}{2.970000in}}%
\pgfpathlineto{\pgfqpoint{4.500000in}{5.280000in}}%
\pgfusepath{stroke}%
\end{pgfscope}%
\begin{pgfscope}%
\pgfsetrectcap%
\pgfsetmiterjoin%
\pgfsetlinewidth{0.803000pt}%
\definecolor{currentstroke}{rgb}{0.000000,0.000000,0.000000}%
\pgfsetstrokecolor{currentstroke}%
\pgfsetdash{}{0pt}%
\pgfpathmoveto{\pgfqpoint{0.625000in}{2.970000in}}%
\pgfpathlineto{\pgfqpoint{4.500000in}{2.970000in}}%
\pgfusepath{stroke}%
\end{pgfscope}%
\begin{pgfscope}%
\pgfsetrectcap%
\pgfsetmiterjoin%
\pgfsetlinewidth{0.803000pt}%
\definecolor{currentstroke}{rgb}{0.000000,0.000000,0.000000}%
\pgfsetstrokecolor{currentstroke}%
\pgfsetdash{}{0pt}%
\pgfpathmoveto{\pgfqpoint{0.625000in}{5.280000in}}%
\pgfpathlineto{\pgfqpoint{4.500000in}{5.280000in}}%
\pgfusepath{stroke}%
\end{pgfscope}%
\begin{pgfscope}%
\pgfsetbuttcap%
\pgfsetmiterjoin%
\definecolor{currentfill}{rgb}{1.000000,1.000000,1.000000}%
\pgfsetfillcolor{currentfill}%
\pgfsetfillopacity{0.800000}%
\pgfsetlinewidth{1.003750pt}%
\definecolor{currentstroke}{rgb}{0.800000,0.800000,0.800000}%
\pgfsetstrokecolor{currentstroke}%
\pgfsetstrokeopacity{0.800000}%
\pgfsetdash{}{0pt}%
\pgfpathmoveto{\pgfqpoint{1.829504in}{3.039444in}}%
\pgfpathlineto{\pgfqpoint{3.295496in}{3.039444in}}%
\pgfpathquadraticcurveto{\pgfqpoint{3.323274in}{3.039444in}}{\pgfqpoint{3.323274in}{3.067222in}}%
\pgfpathlineto{\pgfqpoint{3.323274in}{3.661311in}}%
\pgfpathquadraticcurveto{\pgfqpoint{3.323274in}{3.689088in}}{\pgfqpoint{3.295496in}{3.689088in}}%
\pgfpathlineto{\pgfqpoint{1.829504in}{3.689088in}}%
\pgfpathquadraticcurveto{\pgfqpoint{1.801726in}{3.689088in}}{\pgfqpoint{1.801726in}{3.661311in}}%
\pgfpathlineto{\pgfqpoint{1.801726in}{3.067222in}}%
\pgfpathquadraticcurveto{\pgfqpoint{1.801726in}{3.039444in}}{\pgfqpoint{1.829504in}{3.039444in}}%
\pgfpathlineto{\pgfqpoint{1.829504in}{3.039444in}}%
\pgfpathclose%
\pgfusepath{stroke,fill}%
\end{pgfscope}%
\begin{pgfscope}%
\pgfsetrectcap%
\pgfsetroundjoin%
\pgfsetlinewidth{2.007500pt}%
\definecolor{currentstroke}{rgb}{0.121569,0.466667,0.705882}%
\pgfsetstrokecolor{currentstroke}%
\pgfsetdash{}{0pt}%
\pgfpathmoveto{\pgfqpoint{1.857281in}{3.578791in}}%
\pgfpathlineto{\pgfqpoint{1.996170in}{3.578791in}}%
\pgfpathlineto{\pgfqpoint{2.135059in}{3.578791in}}%
\pgfusepath{stroke}%
\end{pgfscope}%
\begin{pgfscope}%
\definecolor{textcolor}{rgb}{0.000000,0.000000,0.000000}%
\pgfsetstrokecolor{textcolor}%
\pgfsetfillcolor{textcolor}%
\pgftext[x=2.246170in,y=3.530180in,left,base]{\color{textcolor}\rmfamily\fontsize{10.000000}{12.000000}\selectfont \(\displaystyle P_\mathrm{e}\) pre-fault}%
\end{pgfscope}%
\begin{pgfscope}%
\pgfsetrectcap%
\pgfsetroundjoin%
\pgfsetlinewidth{2.007500pt}%
\definecolor{currentstroke}{rgb}{1.000000,0.498039,0.054902}%
\pgfsetstrokecolor{currentstroke}%
\pgfsetdash{}{0pt}%
\pgfpathmoveto{\pgfqpoint{1.857281in}{3.375748in}}%
\pgfpathlineto{\pgfqpoint{1.996170in}{3.375748in}}%
\pgfpathlineto{\pgfqpoint{2.135059in}{3.375748in}}%
\pgfusepath{stroke}%
\end{pgfscope}%
\begin{pgfscope}%
\definecolor{textcolor}{rgb}{0.000000,0.000000,0.000000}%
\pgfsetstrokecolor{textcolor}%
\pgfsetfillcolor{textcolor}%
\pgftext[x=2.246170in,y=3.327137in,left,base]{\color{textcolor}\rmfamily\fontsize{10.000000}{12.000000}\selectfont \(\displaystyle P_\mathrm{e}\) post-fault}%
\end{pgfscope}%
\begin{pgfscope}%
\pgfsetrectcap%
\pgfsetroundjoin%
\pgfsetlinewidth{2.007500pt}%
\definecolor{currentstroke}{rgb}{0.172549,0.627451,0.172549}%
\pgfsetstrokecolor{currentstroke}%
\pgfsetdash{}{0pt}%
\pgfpathmoveto{\pgfqpoint{1.857281in}{3.173857in}}%
\pgfpathlineto{\pgfqpoint{1.996170in}{3.173857in}}%
\pgfpathlineto{\pgfqpoint{2.135059in}{3.173857in}}%
\pgfusepath{stroke}%
\end{pgfscope}%
\begin{pgfscope}%
\definecolor{textcolor}{rgb}{0.000000,0.000000,0.000000}%
\pgfsetstrokecolor{textcolor}%
\pgfsetfillcolor{textcolor}%
\pgftext[x=2.246170in,y=3.125246in,left,base]{\color{textcolor}\rmfamily\fontsize{10.000000}{12.000000}\selectfont \(\displaystyle P_\mathrm{T}\) of the turbine}%
\end{pgfscope}%
\begin{pgfscope}%
\pgfsetbuttcap%
\pgfsetmiterjoin%
\definecolor{currentfill}{rgb}{1.000000,1.000000,1.000000}%
\pgfsetfillcolor{currentfill}%
\pgfsetlinewidth{0.000000pt}%
\definecolor{currentstroke}{rgb}{0.000000,0.000000,0.000000}%
\pgfsetstrokecolor{currentstroke}%
\pgfsetstrokeopacity{0.000000}%
\pgfsetdash{}{0pt}%
\pgfpathmoveto{\pgfqpoint{0.625000in}{0.660000in}}%
\pgfpathlineto{\pgfqpoint{4.500000in}{0.660000in}}%
\pgfpathlineto{\pgfqpoint{4.500000in}{2.970000in}}%
\pgfpathlineto{\pgfqpoint{0.625000in}{2.970000in}}%
\pgfpathlineto{\pgfqpoint{0.625000in}{0.660000in}}%
\pgfpathclose%
\pgfusepath{fill}%
\end{pgfscope}%
\begin{pgfscope}%
\pgfpathrectangle{\pgfqpoint{0.625000in}{0.660000in}}{\pgfqpoint{3.875000in}{2.310000in}}%
\pgfusepath{clip}%
\pgfsetrectcap%
\pgfsetroundjoin%
\pgfsetlinewidth{0.803000pt}%
\definecolor{currentstroke}{rgb}{0.690196,0.690196,0.690196}%
\pgfsetstrokecolor{currentstroke}%
\pgfsetdash{}{0pt}%
\pgfpathmoveto{\pgfqpoint{0.625000in}{0.660000in}}%
\pgfpathlineto{\pgfqpoint{0.625000in}{2.970000in}}%
\pgfusepath{stroke}%
\end{pgfscope}%
\begin{pgfscope}%
\pgfsetbuttcap%
\pgfsetroundjoin%
\definecolor{currentfill}{rgb}{0.000000,0.000000,0.000000}%
\pgfsetfillcolor{currentfill}%
\pgfsetlinewidth{0.803000pt}%
\definecolor{currentstroke}{rgb}{0.000000,0.000000,0.000000}%
\pgfsetstrokecolor{currentstroke}%
\pgfsetdash{}{0pt}%
\pgfsys@defobject{currentmarker}{\pgfqpoint{0.000000in}{-0.048611in}}{\pgfqpoint{0.000000in}{0.000000in}}{%
\pgfpathmoveto{\pgfqpoint{0.000000in}{0.000000in}}%
\pgfpathlineto{\pgfqpoint{0.000000in}{-0.048611in}}%
\pgfusepath{stroke,fill}%
}%
\begin{pgfscope}%
\pgfsys@transformshift{0.625000in}{0.660000in}%
\pgfsys@useobject{currentmarker}{}%
\end{pgfscope}%
\end{pgfscope}%
\begin{pgfscope}%
\definecolor{textcolor}{rgb}{0.000000,0.000000,0.000000}%
\pgfsetstrokecolor{textcolor}%
\pgfsetfillcolor{textcolor}%
\pgftext[x=0.625000in,y=0.562778in,,top]{\color{textcolor}\rmfamily\fontsize{10.000000}{12.000000}\selectfont \(\displaystyle {0}\)}%
\end{pgfscope}%
\begin{pgfscope}%
\pgfpathrectangle{\pgfqpoint{0.625000in}{0.660000in}}{\pgfqpoint{3.875000in}{2.310000in}}%
\pgfusepath{clip}%
\pgfsetrectcap%
\pgfsetroundjoin%
\pgfsetlinewidth{0.803000pt}%
\definecolor{currentstroke}{rgb}{0.690196,0.690196,0.690196}%
\pgfsetstrokecolor{currentstroke}%
\pgfsetdash{}{0pt}%
\pgfpathmoveto{\pgfqpoint{1.055556in}{0.660000in}}%
\pgfpathlineto{\pgfqpoint{1.055556in}{2.970000in}}%
\pgfusepath{stroke}%
\end{pgfscope}%
\begin{pgfscope}%
\pgfsetbuttcap%
\pgfsetroundjoin%
\definecolor{currentfill}{rgb}{0.000000,0.000000,0.000000}%
\pgfsetfillcolor{currentfill}%
\pgfsetlinewidth{0.803000pt}%
\definecolor{currentstroke}{rgb}{0.000000,0.000000,0.000000}%
\pgfsetstrokecolor{currentstroke}%
\pgfsetdash{}{0pt}%
\pgfsys@defobject{currentmarker}{\pgfqpoint{0.000000in}{-0.048611in}}{\pgfqpoint{0.000000in}{0.000000in}}{%
\pgfpathmoveto{\pgfqpoint{0.000000in}{0.000000in}}%
\pgfpathlineto{\pgfqpoint{0.000000in}{-0.048611in}}%
\pgfusepath{stroke,fill}%
}%
\begin{pgfscope}%
\pgfsys@transformshift{1.055556in}{0.660000in}%
\pgfsys@useobject{currentmarker}{}%
\end{pgfscope}%
\end{pgfscope}%
\begin{pgfscope}%
\definecolor{textcolor}{rgb}{0.000000,0.000000,0.000000}%
\pgfsetstrokecolor{textcolor}%
\pgfsetfillcolor{textcolor}%
\pgftext[x=1.055556in,y=0.562778in,,top]{\color{textcolor}\rmfamily\fontsize{10.000000}{12.000000}\selectfont \(\displaystyle {20}\)}%
\end{pgfscope}%
\begin{pgfscope}%
\pgfpathrectangle{\pgfqpoint{0.625000in}{0.660000in}}{\pgfqpoint{3.875000in}{2.310000in}}%
\pgfusepath{clip}%
\pgfsetrectcap%
\pgfsetroundjoin%
\pgfsetlinewidth{0.803000pt}%
\definecolor{currentstroke}{rgb}{0.690196,0.690196,0.690196}%
\pgfsetstrokecolor{currentstroke}%
\pgfsetdash{}{0pt}%
\pgfpathmoveto{\pgfqpoint{1.486111in}{0.660000in}}%
\pgfpathlineto{\pgfqpoint{1.486111in}{2.970000in}}%
\pgfusepath{stroke}%
\end{pgfscope}%
\begin{pgfscope}%
\pgfsetbuttcap%
\pgfsetroundjoin%
\definecolor{currentfill}{rgb}{0.000000,0.000000,0.000000}%
\pgfsetfillcolor{currentfill}%
\pgfsetlinewidth{0.803000pt}%
\definecolor{currentstroke}{rgb}{0.000000,0.000000,0.000000}%
\pgfsetstrokecolor{currentstroke}%
\pgfsetdash{}{0pt}%
\pgfsys@defobject{currentmarker}{\pgfqpoint{0.000000in}{-0.048611in}}{\pgfqpoint{0.000000in}{0.000000in}}{%
\pgfpathmoveto{\pgfqpoint{0.000000in}{0.000000in}}%
\pgfpathlineto{\pgfqpoint{0.000000in}{-0.048611in}}%
\pgfusepath{stroke,fill}%
}%
\begin{pgfscope}%
\pgfsys@transformshift{1.486111in}{0.660000in}%
\pgfsys@useobject{currentmarker}{}%
\end{pgfscope}%
\end{pgfscope}%
\begin{pgfscope}%
\definecolor{textcolor}{rgb}{0.000000,0.000000,0.000000}%
\pgfsetstrokecolor{textcolor}%
\pgfsetfillcolor{textcolor}%
\pgftext[x=1.486111in,y=0.562778in,,top]{\color{textcolor}\rmfamily\fontsize{10.000000}{12.000000}\selectfont \(\displaystyle {40}\)}%
\end{pgfscope}%
\begin{pgfscope}%
\pgfpathrectangle{\pgfqpoint{0.625000in}{0.660000in}}{\pgfqpoint{3.875000in}{2.310000in}}%
\pgfusepath{clip}%
\pgfsetrectcap%
\pgfsetroundjoin%
\pgfsetlinewidth{0.803000pt}%
\definecolor{currentstroke}{rgb}{0.690196,0.690196,0.690196}%
\pgfsetstrokecolor{currentstroke}%
\pgfsetdash{}{0pt}%
\pgfpathmoveto{\pgfqpoint{1.916667in}{0.660000in}}%
\pgfpathlineto{\pgfqpoint{1.916667in}{2.970000in}}%
\pgfusepath{stroke}%
\end{pgfscope}%
\begin{pgfscope}%
\pgfsetbuttcap%
\pgfsetroundjoin%
\definecolor{currentfill}{rgb}{0.000000,0.000000,0.000000}%
\pgfsetfillcolor{currentfill}%
\pgfsetlinewidth{0.803000pt}%
\definecolor{currentstroke}{rgb}{0.000000,0.000000,0.000000}%
\pgfsetstrokecolor{currentstroke}%
\pgfsetdash{}{0pt}%
\pgfsys@defobject{currentmarker}{\pgfqpoint{0.000000in}{-0.048611in}}{\pgfqpoint{0.000000in}{0.000000in}}{%
\pgfpathmoveto{\pgfqpoint{0.000000in}{0.000000in}}%
\pgfpathlineto{\pgfqpoint{0.000000in}{-0.048611in}}%
\pgfusepath{stroke,fill}%
}%
\begin{pgfscope}%
\pgfsys@transformshift{1.916667in}{0.660000in}%
\pgfsys@useobject{currentmarker}{}%
\end{pgfscope}%
\end{pgfscope}%
\begin{pgfscope}%
\definecolor{textcolor}{rgb}{0.000000,0.000000,0.000000}%
\pgfsetstrokecolor{textcolor}%
\pgfsetfillcolor{textcolor}%
\pgftext[x=1.916667in,y=0.562778in,,top]{\color{textcolor}\rmfamily\fontsize{10.000000}{12.000000}\selectfont \(\displaystyle {60}\)}%
\end{pgfscope}%
\begin{pgfscope}%
\pgfpathrectangle{\pgfqpoint{0.625000in}{0.660000in}}{\pgfqpoint{3.875000in}{2.310000in}}%
\pgfusepath{clip}%
\pgfsetrectcap%
\pgfsetroundjoin%
\pgfsetlinewidth{0.803000pt}%
\definecolor{currentstroke}{rgb}{0.690196,0.690196,0.690196}%
\pgfsetstrokecolor{currentstroke}%
\pgfsetdash{}{0pt}%
\pgfpathmoveto{\pgfqpoint{2.347222in}{0.660000in}}%
\pgfpathlineto{\pgfqpoint{2.347222in}{2.970000in}}%
\pgfusepath{stroke}%
\end{pgfscope}%
\begin{pgfscope}%
\pgfsetbuttcap%
\pgfsetroundjoin%
\definecolor{currentfill}{rgb}{0.000000,0.000000,0.000000}%
\pgfsetfillcolor{currentfill}%
\pgfsetlinewidth{0.803000pt}%
\definecolor{currentstroke}{rgb}{0.000000,0.000000,0.000000}%
\pgfsetstrokecolor{currentstroke}%
\pgfsetdash{}{0pt}%
\pgfsys@defobject{currentmarker}{\pgfqpoint{0.000000in}{-0.048611in}}{\pgfqpoint{0.000000in}{0.000000in}}{%
\pgfpathmoveto{\pgfqpoint{0.000000in}{0.000000in}}%
\pgfpathlineto{\pgfqpoint{0.000000in}{-0.048611in}}%
\pgfusepath{stroke,fill}%
}%
\begin{pgfscope}%
\pgfsys@transformshift{2.347222in}{0.660000in}%
\pgfsys@useobject{currentmarker}{}%
\end{pgfscope}%
\end{pgfscope}%
\begin{pgfscope}%
\definecolor{textcolor}{rgb}{0.000000,0.000000,0.000000}%
\pgfsetstrokecolor{textcolor}%
\pgfsetfillcolor{textcolor}%
\pgftext[x=2.347222in,y=0.562778in,,top]{\color{textcolor}\rmfamily\fontsize{10.000000}{12.000000}\selectfont \(\displaystyle {80}\)}%
\end{pgfscope}%
\begin{pgfscope}%
\pgfpathrectangle{\pgfqpoint{0.625000in}{0.660000in}}{\pgfqpoint{3.875000in}{2.310000in}}%
\pgfusepath{clip}%
\pgfsetrectcap%
\pgfsetroundjoin%
\pgfsetlinewidth{0.803000pt}%
\definecolor{currentstroke}{rgb}{0.690196,0.690196,0.690196}%
\pgfsetstrokecolor{currentstroke}%
\pgfsetdash{}{0pt}%
\pgfpathmoveto{\pgfqpoint{2.777778in}{0.660000in}}%
\pgfpathlineto{\pgfqpoint{2.777778in}{2.970000in}}%
\pgfusepath{stroke}%
\end{pgfscope}%
\begin{pgfscope}%
\pgfsetbuttcap%
\pgfsetroundjoin%
\definecolor{currentfill}{rgb}{0.000000,0.000000,0.000000}%
\pgfsetfillcolor{currentfill}%
\pgfsetlinewidth{0.803000pt}%
\definecolor{currentstroke}{rgb}{0.000000,0.000000,0.000000}%
\pgfsetstrokecolor{currentstroke}%
\pgfsetdash{}{0pt}%
\pgfsys@defobject{currentmarker}{\pgfqpoint{0.000000in}{-0.048611in}}{\pgfqpoint{0.000000in}{0.000000in}}{%
\pgfpathmoveto{\pgfqpoint{0.000000in}{0.000000in}}%
\pgfpathlineto{\pgfqpoint{0.000000in}{-0.048611in}}%
\pgfusepath{stroke,fill}%
}%
\begin{pgfscope}%
\pgfsys@transformshift{2.777778in}{0.660000in}%
\pgfsys@useobject{currentmarker}{}%
\end{pgfscope}%
\end{pgfscope}%
\begin{pgfscope}%
\definecolor{textcolor}{rgb}{0.000000,0.000000,0.000000}%
\pgfsetstrokecolor{textcolor}%
\pgfsetfillcolor{textcolor}%
\pgftext[x=2.777778in,y=0.562778in,,top]{\color{textcolor}\rmfamily\fontsize{10.000000}{12.000000}\selectfont \(\displaystyle {100}\)}%
\end{pgfscope}%
\begin{pgfscope}%
\pgfpathrectangle{\pgfqpoint{0.625000in}{0.660000in}}{\pgfqpoint{3.875000in}{2.310000in}}%
\pgfusepath{clip}%
\pgfsetrectcap%
\pgfsetroundjoin%
\pgfsetlinewidth{0.803000pt}%
\definecolor{currentstroke}{rgb}{0.690196,0.690196,0.690196}%
\pgfsetstrokecolor{currentstroke}%
\pgfsetdash{}{0pt}%
\pgfpathmoveto{\pgfqpoint{3.208333in}{0.660000in}}%
\pgfpathlineto{\pgfqpoint{3.208333in}{2.970000in}}%
\pgfusepath{stroke}%
\end{pgfscope}%
\begin{pgfscope}%
\pgfsetbuttcap%
\pgfsetroundjoin%
\definecolor{currentfill}{rgb}{0.000000,0.000000,0.000000}%
\pgfsetfillcolor{currentfill}%
\pgfsetlinewidth{0.803000pt}%
\definecolor{currentstroke}{rgb}{0.000000,0.000000,0.000000}%
\pgfsetstrokecolor{currentstroke}%
\pgfsetdash{}{0pt}%
\pgfsys@defobject{currentmarker}{\pgfqpoint{0.000000in}{-0.048611in}}{\pgfqpoint{0.000000in}{0.000000in}}{%
\pgfpathmoveto{\pgfqpoint{0.000000in}{0.000000in}}%
\pgfpathlineto{\pgfqpoint{0.000000in}{-0.048611in}}%
\pgfusepath{stroke,fill}%
}%
\begin{pgfscope}%
\pgfsys@transformshift{3.208333in}{0.660000in}%
\pgfsys@useobject{currentmarker}{}%
\end{pgfscope}%
\end{pgfscope}%
\begin{pgfscope}%
\definecolor{textcolor}{rgb}{0.000000,0.000000,0.000000}%
\pgfsetstrokecolor{textcolor}%
\pgfsetfillcolor{textcolor}%
\pgftext[x=3.208333in,y=0.562778in,,top]{\color{textcolor}\rmfamily\fontsize{10.000000}{12.000000}\selectfont \(\displaystyle {120}\)}%
\end{pgfscope}%
\begin{pgfscope}%
\pgfpathrectangle{\pgfqpoint{0.625000in}{0.660000in}}{\pgfqpoint{3.875000in}{2.310000in}}%
\pgfusepath{clip}%
\pgfsetrectcap%
\pgfsetroundjoin%
\pgfsetlinewidth{0.803000pt}%
\definecolor{currentstroke}{rgb}{0.690196,0.690196,0.690196}%
\pgfsetstrokecolor{currentstroke}%
\pgfsetdash{}{0pt}%
\pgfpathmoveto{\pgfqpoint{3.638889in}{0.660000in}}%
\pgfpathlineto{\pgfqpoint{3.638889in}{2.970000in}}%
\pgfusepath{stroke}%
\end{pgfscope}%
\begin{pgfscope}%
\pgfsetbuttcap%
\pgfsetroundjoin%
\definecolor{currentfill}{rgb}{0.000000,0.000000,0.000000}%
\pgfsetfillcolor{currentfill}%
\pgfsetlinewidth{0.803000pt}%
\definecolor{currentstroke}{rgb}{0.000000,0.000000,0.000000}%
\pgfsetstrokecolor{currentstroke}%
\pgfsetdash{}{0pt}%
\pgfsys@defobject{currentmarker}{\pgfqpoint{0.000000in}{-0.048611in}}{\pgfqpoint{0.000000in}{0.000000in}}{%
\pgfpathmoveto{\pgfqpoint{0.000000in}{0.000000in}}%
\pgfpathlineto{\pgfqpoint{0.000000in}{-0.048611in}}%
\pgfusepath{stroke,fill}%
}%
\begin{pgfscope}%
\pgfsys@transformshift{3.638889in}{0.660000in}%
\pgfsys@useobject{currentmarker}{}%
\end{pgfscope}%
\end{pgfscope}%
\begin{pgfscope}%
\definecolor{textcolor}{rgb}{0.000000,0.000000,0.000000}%
\pgfsetstrokecolor{textcolor}%
\pgfsetfillcolor{textcolor}%
\pgftext[x=3.638889in,y=0.562778in,,top]{\color{textcolor}\rmfamily\fontsize{10.000000}{12.000000}\selectfont \(\displaystyle {140}\)}%
\end{pgfscope}%
\begin{pgfscope}%
\pgfpathrectangle{\pgfqpoint{0.625000in}{0.660000in}}{\pgfqpoint{3.875000in}{2.310000in}}%
\pgfusepath{clip}%
\pgfsetrectcap%
\pgfsetroundjoin%
\pgfsetlinewidth{0.803000pt}%
\definecolor{currentstroke}{rgb}{0.690196,0.690196,0.690196}%
\pgfsetstrokecolor{currentstroke}%
\pgfsetdash{}{0pt}%
\pgfpathmoveto{\pgfqpoint{4.069444in}{0.660000in}}%
\pgfpathlineto{\pgfqpoint{4.069444in}{2.970000in}}%
\pgfusepath{stroke}%
\end{pgfscope}%
\begin{pgfscope}%
\pgfsetbuttcap%
\pgfsetroundjoin%
\definecolor{currentfill}{rgb}{0.000000,0.000000,0.000000}%
\pgfsetfillcolor{currentfill}%
\pgfsetlinewidth{0.803000pt}%
\definecolor{currentstroke}{rgb}{0.000000,0.000000,0.000000}%
\pgfsetstrokecolor{currentstroke}%
\pgfsetdash{}{0pt}%
\pgfsys@defobject{currentmarker}{\pgfqpoint{0.000000in}{-0.048611in}}{\pgfqpoint{0.000000in}{0.000000in}}{%
\pgfpathmoveto{\pgfqpoint{0.000000in}{0.000000in}}%
\pgfpathlineto{\pgfqpoint{0.000000in}{-0.048611in}}%
\pgfusepath{stroke,fill}%
}%
\begin{pgfscope}%
\pgfsys@transformshift{4.069444in}{0.660000in}%
\pgfsys@useobject{currentmarker}{}%
\end{pgfscope}%
\end{pgfscope}%
\begin{pgfscope}%
\definecolor{textcolor}{rgb}{0.000000,0.000000,0.000000}%
\pgfsetstrokecolor{textcolor}%
\pgfsetfillcolor{textcolor}%
\pgftext[x=4.069444in,y=0.562778in,,top]{\color{textcolor}\rmfamily\fontsize{10.000000}{12.000000}\selectfont \(\displaystyle {160}\)}%
\end{pgfscope}%
\begin{pgfscope}%
\pgfpathrectangle{\pgfqpoint{0.625000in}{0.660000in}}{\pgfqpoint{3.875000in}{2.310000in}}%
\pgfusepath{clip}%
\pgfsetrectcap%
\pgfsetroundjoin%
\pgfsetlinewidth{0.803000pt}%
\definecolor{currentstroke}{rgb}{0.690196,0.690196,0.690196}%
\pgfsetstrokecolor{currentstroke}%
\pgfsetdash{}{0pt}%
\pgfpathmoveto{\pgfqpoint{4.500000in}{0.660000in}}%
\pgfpathlineto{\pgfqpoint{4.500000in}{2.970000in}}%
\pgfusepath{stroke}%
\end{pgfscope}%
\begin{pgfscope}%
\pgfsetbuttcap%
\pgfsetroundjoin%
\definecolor{currentfill}{rgb}{0.000000,0.000000,0.000000}%
\pgfsetfillcolor{currentfill}%
\pgfsetlinewidth{0.803000pt}%
\definecolor{currentstroke}{rgb}{0.000000,0.000000,0.000000}%
\pgfsetstrokecolor{currentstroke}%
\pgfsetdash{}{0pt}%
\pgfsys@defobject{currentmarker}{\pgfqpoint{0.000000in}{-0.048611in}}{\pgfqpoint{0.000000in}{0.000000in}}{%
\pgfpathmoveto{\pgfqpoint{0.000000in}{0.000000in}}%
\pgfpathlineto{\pgfqpoint{0.000000in}{-0.048611in}}%
\pgfusepath{stroke,fill}%
}%
\begin{pgfscope}%
\pgfsys@transformshift{4.500000in}{0.660000in}%
\pgfsys@useobject{currentmarker}{}%
\end{pgfscope}%
\end{pgfscope}%
\begin{pgfscope}%
\definecolor{textcolor}{rgb}{0.000000,0.000000,0.000000}%
\pgfsetstrokecolor{textcolor}%
\pgfsetfillcolor{textcolor}%
\pgftext[x=4.500000in,y=0.562778in,,top]{\color{textcolor}\rmfamily\fontsize{10.000000}{12.000000}\selectfont \(\displaystyle {180}\)}%
\end{pgfscope}%
\begin{pgfscope}%
\definecolor{textcolor}{rgb}{0.000000,0.000000,0.000000}%
\pgfsetstrokecolor{textcolor}%
\pgfsetfillcolor{textcolor}%
\pgftext[x=2.562500in,y=0.374776in,,top]{\color{textcolor}\rmfamily\fontsize{10.000000}{12.000000}\selectfont power angle \(\displaystyle \delta\) in deg}%
\end{pgfscope}%
\begin{pgfscope}%
\pgfpathrectangle{\pgfqpoint{0.625000in}{0.660000in}}{\pgfqpoint{3.875000in}{2.310000in}}%
\pgfusepath{clip}%
\pgfsetrectcap%
\pgfsetroundjoin%
\pgfsetlinewidth{0.803000pt}%
\definecolor{currentstroke}{rgb}{0.690196,0.690196,0.690196}%
\pgfsetstrokecolor{currentstroke}%
\pgfsetdash{}{0pt}%
\pgfpathmoveto{\pgfqpoint{0.625000in}{2.770826in}}%
\pgfpathlineto{\pgfqpoint{4.500000in}{2.770826in}}%
\pgfusepath{stroke}%
\end{pgfscope}%
\begin{pgfscope}%
\pgfsetbuttcap%
\pgfsetroundjoin%
\definecolor{currentfill}{rgb}{0.000000,0.000000,0.000000}%
\pgfsetfillcolor{currentfill}%
\pgfsetlinewidth{0.803000pt}%
\definecolor{currentstroke}{rgb}{0.000000,0.000000,0.000000}%
\pgfsetstrokecolor{currentstroke}%
\pgfsetdash{}{0pt}%
\pgfsys@defobject{currentmarker}{\pgfqpoint{-0.048611in}{0.000000in}}{\pgfqpoint{-0.000000in}{0.000000in}}{%
\pgfpathmoveto{\pgfqpoint{-0.000000in}{0.000000in}}%
\pgfpathlineto{\pgfqpoint{-0.048611in}{0.000000in}}%
\pgfusepath{stroke,fill}%
}%
\begin{pgfscope}%
\pgfsys@transformshift{0.625000in}{2.770826in}%
\pgfsys@useobject{currentmarker}{}%
\end{pgfscope}%
\end{pgfscope}%
\begin{pgfscope}%
\definecolor{textcolor}{rgb}{0.000000,0.000000,0.000000}%
\pgfsetstrokecolor{textcolor}%
\pgfsetfillcolor{textcolor}%
\pgftext[x=0.458333in, y=2.719726in, left, base]{\color{textcolor}\rmfamily\fontsize{10.000000}{12.000000}\selectfont \(\displaystyle {0}\)}%
\end{pgfscope}%
\begin{pgfscope}%
\pgfpathrectangle{\pgfqpoint{0.625000in}{0.660000in}}{\pgfqpoint{3.875000in}{2.310000in}}%
\pgfusepath{clip}%
\pgfsetrectcap%
\pgfsetroundjoin%
\pgfsetlinewidth{0.803000pt}%
\definecolor{currentstroke}{rgb}{0.690196,0.690196,0.690196}%
\pgfsetstrokecolor{currentstroke}%
\pgfsetdash{}{0pt}%
\pgfpathmoveto{\pgfqpoint{0.625000in}{2.372478in}}%
\pgfpathlineto{\pgfqpoint{4.500000in}{2.372478in}}%
\pgfusepath{stroke}%
\end{pgfscope}%
\begin{pgfscope}%
\pgfsetbuttcap%
\pgfsetroundjoin%
\definecolor{currentfill}{rgb}{0.000000,0.000000,0.000000}%
\pgfsetfillcolor{currentfill}%
\pgfsetlinewidth{0.803000pt}%
\definecolor{currentstroke}{rgb}{0.000000,0.000000,0.000000}%
\pgfsetstrokecolor{currentstroke}%
\pgfsetdash{}{0pt}%
\pgfsys@defobject{currentmarker}{\pgfqpoint{-0.048611in}{0.000000in}}{\pgfqpoint{-0.000000in}{0.000000in}}{%
\pgfpathmoveto{\pgfqpoint{-0.000000in}{0.000000in}}%
\pgfpathlineto{\pgfqpoint{-0.048611in}{0.000000in}}%
\pgfusepath{stroke,fill}%
}%
\begin{pgfscope}%
\pgfsys@transformshift{0.625000in}{2.372478in}%
\pgfsys@useobject{currentmarker}{}%
\end{pgfscope}%
\end{pgfscope}%
\begin{pgfscope}%
\definecolor{textcolor}{rgb}{0.000000,0.000000,0.000000}%
\pgfsetstrokecolor{textcolor}%
\pgfsetfillcolor{textcolor}%
\pgftext[x=0.458333in, y=2.321378in, left, base]{\color{textcolor}\rmfamily\fontsize{10.000000}{12.000000}\selectfont \(\displaystyle {1}\)}%
\end{pgfscope}%
\begin{pgfscope}%
\pgfpathrectangle{\pgfqpoint{0.625000in}{0.660000in}}{\pgfqpoint{3.875000in}{2.310000in}}%
\pgfusepath{clip}%
\pgfsetrectcap%
\pgfsetroundjoin%
\pgfsetlinewidth{0.803000pt}%
\definecolor{currentstroke}{rgb}{0.690196,0.690196,0.690196}%
\pgfsetstrokecolor{currentstroke}%
\pgfsetdash{}{0pt}%
\pgfpathmoveto{\pgfqpoint{0.625000in}{1.974130in}}%
\pgfpathlineto{\pgfqpoint{4.500000in}{1.974130in}}%
\pgfusepath{stroke}%
\end{pgfscope}%
\begin{pgfscope}%
\pgfsetbuttcap%
\pgfsetroundjoin%
\definecolor{currentfill}{rgb}{0.000000,0.000000,0.000000}%
\pgfsetfillcolor{currentfill}%
\pgfsetlinewidth{0.803000pt}%
\definecolor{currentstroke}{rgb}{0.000000,0.000000,0.000000}%
\pgfsetstrokecolor{currentstroke}%
\pgfsetdash{}{0pt}%
\pgfsys@defobject{currentmarker}{\pgfqpoint{-0.048611in}{0.000000in}}{\pgfqpoint{-0.000000in}{0.000000in}}{%
\pgfpathmoveto{\pgfqpoint{-0.000000in}{0.000000in}}%
\pgfpathlineto{\pgfqpoint{-0.048611in}{0.000000in}}%
\pgfusepath{stroke,fill}%
}%
\begin{pgfscope}%
\pgfsys@transformshift{0.625000in}{1.974130in}%
\pgfsys@useobject{currentmarker}{}%
\end{pgfscope}%
\end{pgfscope}%
\begin{pgfscope}%
\definecolor{textcolor}{rgb}{0.000000,0.000000,0.000000}%
\pgfsetstrokecolor{textcolor}%
\pgfsetfillcolor{textcolor}%
\pgftext[x=0.458333in, y=1.923030in, left, base]{\color{textcolor}\rmfamily\fontsize{10.000000}{12.000000}\selectfont \(\displaystyle {2}\)}%
\end{pgfscope}%
\begin{pgfscope}%
\pgfpathrectangle{\pgfqpoint{0.625000in}{0.660000in}}{\pgfqpoint{3.875000in}{2.310000in}}%
\pgfusepath{clip}%
\pgfsetrectcap%
\pgfsetroundjoin%
\pgfsetlinewidth{0.803000pt}%
\definecolor{currentstroke}{rgb}{0.690196,0.690196,0.690196}%
\pgfsetstrokecolor{currentstroke}%
\pgfsetdash{}{0pt}%
\pgfpathmoveto{\pgfqpoint{0.625000in}{1.575782in}}%
\pgfpathlineto{\pgfqpoint{4.500000in}{1.575782in}}%
\pgfusepath{stroke}%
\end{pgfscope}%
\begin{pgfscope}%
\pgfsetbuttcap%
\pgfsetroundjoin%
\definecolor{currentfill}{rgb}{0.000000,0.000000,0.000000}%
\pgfsetfillcolor{currentfill}%
\pgfsetlinewidth{0.803000pt}%
\definecolor{currentstroke}{rgb}{0.000000,0.000000,0.000000}%
\pgfsetstrokecolor{currentstroke}%
\pgfsetdash{}{0pt}%
\pgfsys@defobject{currentmarker}{\pgfqpoint{-0.048611in}{0.000000in}}{\pgfqpoint{-0.000000in}{0.000000in}}{%
\pgfpathmoveto{\pgfqpoint{-0.000000in}{0.000000in}}%
\pgfpathlineto{\pgfqpoint{-0.048611in}{0.000000in}}%
\pgfusepath{stroke,fill}%
}%
\begin{pgfscope}%
\pgfsys@transformshift{0.625000in}{1.575782in}%
\pgfsys@useobject{currentmarker}{}%
\end{pgfscope}%
\end{pgfscope}%
\begin{pgfscope}%
\definecolor{textcolor}{rgb}{0.000000,0.000000,0.000000}%
\pgfsetstrokecolor{textcolor}%
\pgfsetfillcolor{textcolor}%
\pgftext[x=0.458333in, y=1.524682in, left, base]{\color{textcolor}\rmfamily\fontsize{10.000000}{12.000000}\selectfont \(\displaystyle {3}\)}%
\end{pgfscope}%
\begin{pgfscope}%
\pgfpathrectangle{\pgfqpoint{0.625000in}{0.660000in}}{\pgfqpoint{3.875000in}{2.310000in}}%
\pgfusepath{clip}%
\pgfsetrectcap%
\pgfsetroundjoin%
\pgfsetlinewidth{0.803000pt}%
\definecolor{currentstroke}{rgb}{0.690196,0.690196,0.690196}%
\pgfsetstrokecolor{currentstroke}%
\pgfsetdash{}{0pt}%
\pgfpathmoveto{\pgfqpoint{0.625000in}{1.177434in}}%
\pgfpathlineto{\pgfqpoint{4.500000in}{1.177434in}}%
\pgfusepath{stroke}%
\end{pgfscope}%
\begin{pgfscope}%
\pgfsetbuttcap%
\pgfsetroundjoin%
\definecolor{currentfill}{rgb}{0.000000,0.000000,0.000000}%
\pgfsetfillcolor{currentfill}%
\pgfsetlinewidth{0.803000pt}%
\definecolor{currentstroke}{rgb}{0.000000,0.000000,0.000000}%
\pgfsetstrokecolor{currentstroke}%
\pgfsetdash{}{0pt}%
\pgfsys@defobject{currentmarker}{\pgfqpoint{-0.048611in}{0.000000in}}{\pgfqpoint{-0.000000in}{0.000000in}}{%
\pgfpathmoveto{\pgfqpoint{-0.000000in}{0.000000in}}%
\pgfpathlineto{\pgfqpoint{-0.048611in}{0.000000in}}%
\pgfusepath{stroke,fill}%
}%
\begin{pgfscope}%
\pgfsys@transformshift{0.625000in}{1.177434in}%
\pgfsys@useobject{currentmarker}{}%
\end{pgfscope}%
\end{pgfscope}%
\begin{pgfscope}%
\definecolor{textcolor}{rgb}{0.000000,0.000000,0.000000}%
\pgfsetstrokecolor{textcolor}%
\pgfsetfillcolor{textcolor}%
\pgftext[x=0.458333in, y=1.126334in, left, base]{\color{textcolor}\rmfamily\fontsize{10.000000}{12.000000}\selectfont \(\displaystyle {4}\)}%
\end{pgfscope}%
\begin{pgfscope}%
\pgfpathrectangle{\pgfqpoint{0.625000in}{0.660000in}}{\pgfqpoint{3.875000in}{2.310000in}}%
\pgfusepath{clip}%
\pgfsetrectcap%
\pgfsetroundjoin%
\pgfsetlinewidth{0.803000pt}%
\definecolor{currentstroke}{rgb}{0.690196,0.690196,0.690196}%
\pgfsetstrokecolor{currentstroke}%
\pgfsetdash{}{0pt}%
\pgfpathmoveto{\pgfqpoint{0.625000in}{0.779086in}}%
\pgfpathlineto{\pgfqpoint{4.500000in}{0.779086in}}%
\pgfusepath{stroke}%
\end{pgfscope}%
\begin{pgfscope}%
\pgfsetbuttcap%
\pgfsetroundjoin%
\definecolor{currentfill}{rgb}{0.000000,0.000000,0.000000}%
\pgfsetfillcolor{currentfill}%
\pgfsetlinewidth{0.803000pt}%
\definecolor{currentstroke}{rgb}{0.000000,0.000000,0.000000}%
\pgfsetstrokecolor{currentstroke}%
\pgfsetdash{}{0pt}%
\pgfsys@defobject{currentmarker}{\pgfqpoint{-0.048611in}{0.000000in}}{\pgfqpoint{-0.000000in}{0.000000in}}{%
\pgfpathmoveto{\pgfqpoint{-0.000000in}{0.000000in}}%
\pgfpathlineto{\pgfqpoint{-0.048611in}{0.000000in}}%
\pgfusepath{stroke,fill}%
}%
\begin{pgfscope}%
\pgfsys@transformshift{0.625000in}{0.779086in}%
\pgfsys@useobject{currentmarker}{}%
\end{pgfscope}%
\end{pgfscope}%
\begin{pgfscope}%
\definecolor{textcolor}{rgb}{0.000000,0.000000,0.000000}%
\pgfsetstrokecolor{textcolor}%
\pgfsetfillcolor{textcolor}%
\pgftext[x=0.458333in, y=0.727986in, left, base]{\color{textcolor}\rmfamily\fontsize{10.000000}{12.000000}\selectfont \(\displaystyle {5}\)}%
\end{pgfscope}%
\begin{pgfscope}%
\definecolor{textcolor}{rgb}{0.000000,0.000000,0.000000}%
\pgfsetstrokecolor{textcolor}%
\pgfsetfillcolor{textcolor}%
\pgftext[x=0.402777in,y=1.815000in,,bottom,rotate=90.000000]{\color{textcolor}\rmfamily\fontsize{10.000000}{12.000000}\selectfont time in s}%
\end{pgfscope}%
\begin{pgfscope}%
\pgfpathrectangle{\pgfqpoint{0.625000in}{0.660000in}}{\pgfqpoint{3.875000in}{2.310000in}}%
\pgfusepath{clip}%
\pgfsetrectcap%
\pgfsetroundjoin%
\pgfsetlinewidth{1.505625pt}%
\definecolor{currentstroke}{rgb}{0.121569,0.466667,0.705882}%
\pgfsetstrokecolor{currentstroke}%
\pgfsetdash{}{0pt}%
\pgfpathmoveto{\pgfqpoint{1.270833in}{2.980000in}}%
\pgfpathlineto{\pgfqpoint{1.271806in}{2.765647in}}%
\pgfpathlineto{\pgfqpoint{1.275019in}{2.760071in}}%
\pgfpathlineto{\pgfqpoint{1.281413in}{2.753697in}}%
\pgfpathlineto{\pgfqpoint{1.292003in}{2.746527in}}%
\pgfpathlineto{\pgfqpoint{1.308748in}{2.738161in}}%
\pgfpathlineto{\pgfqpoint{1.333362in}{2.728601in}}%
\pgfpathlineto{\pgfqpoint{1.367464in}{2.717846in}}%
\pgfpathlineto{\pgfqpoint{1.415585in}{2.705099in}}%
\pgfpathlineto{\pgfqpoint{1.483362in}{2.689563in}}%
\pgfpathlineto{\pgfqpoint{1.588199in}{2.668052in}}%
\pgfpathlineto{\pgfqpoint{1.848509in}{2.615470in}}%
\pgfpathlineto{\pgfqpoint{1.922349in}{2.597943in}}%
\pgfpathlineto{\pgfqpoint{1.975459in}{2.583204in}}%
\pgfpathlineto{\pgfqpoint{2.013682in}{2.570457in}}%
\pgfpathlineto{\pgfqpoint{2.041563in}{2.558905in}}%
\pgfpathlineto{\pgfqpoint{2.060148in}{2.548946in}}%
\pgfpathlineto{\pgfqpoint{2.072183in}{2.540183in}}%
\pgfpathlineto{\pgfqpoint{2.079544in}{2.532216in}}%
\pgfpathlineto{\pgfqpoint{2.083220in}{2.525045in}}%
\pgfpathlineto{\pgfqpoint{2.084132in}{2.518672in}}%
\pgfpathlineto{\pgfqpoint{2.082825in}{2.512298in}}%
\pgfpathlineto{\pgfqpoint{2.079015in}{2.505526in}}%
\pgfpathlineto{\pgfqpoint{2.072284in}{2.498356in}}%
\pgfpathlineto{\pgfqpoint{2.061597in}{2.490389in}}%
\pgfpathlineto{\pgfqpoint{2.046021in}{2.481625in}}%
\pgfpathlineto{\pgfqpoint{2.023621in}{2.471667in}}%
\pgfpathlineto{\pgfqpoint{1.992879in}{2.460513in}}%
\pgfpathlineto{\pgfqpoint{1.950954in}{2.447766in}}%
\pgfpathlineto{\pgfqpoint{1.894449in}{2.433027in}}%
\pgfpathlineto{\pgfqpoint{1.816239in}{2.415101in}}%
\pgfpathlineto{\pgfqpoint{1.697414in}{2.390404in}}%
\pgfpathlineto{\pgfqpoint{1.493913in}{2.348179in}}%
\pgfpathlineto{\pgfqpoint{1.422806in}{2.331050in}}%
\pgfpathlineto{\pgfqpoint{1.374000in}{2.317108in}}%
\pgfpathlineto{\pgfqpoint{1.341159in}{2.305556in}}%
\pgfpathlineto{\pgfqpoint{1.319090in}{2.295597in}}%
\pgfpathlineto{\pgfqpoint{1.304820in}{2.286833in}}%
\pgfpathlineto{\pgfqpoint{1.296553in}{2.279265in}}%
\pgfpathlineto{\pgfqpoint{1.292581in}{2.272891in}}%
\pgfpathlineto{\pgfqpoint{1.291370in}{2.266916in}}%
\pgfpathlineto{\pgfqpoint{1.292438in}{2.261339in}}%
\pgfpathlineto{\pgfqpoint{1.295927in}{2.255364in}}%
\pgfpathlineto{\pgfqpoint{1.302290in}{2.248990in}}%
\pgfpathlineto{\pgfqpoint{1.312635in}{2.241820in}}%
\pgfpathlineto{\pgfqpoint{1.328819in}{2.233455in}}%
\pgfpathlineto{\pgfqpoint{1.352442in}{2.223894in}}%
\pgfpathlineto{\pgfqpoint{1.386331in}{2.212740in}}%
\pgfpathlineto{\pgfqpoint{1.433898in}{2.199595in}}%
\pgfpathlineto{\pgfqpoint{1.500381in}{2.183661in}}%
\pgfpathlineto{\pgfqpoint{1.603902in}{2.161354in}}%
\pgfpathlineto{\pgfqpoint{1.829799in}{2.113154in}}%
\pgfpathlineto{\pgfqpoint{1.902043in}{2.095228in}}%
\pgfpathlineto{\pgfqpoint{1.954223in}{2.080091in}}%
\pgfpathlineto{\pgfqpoint{1.991807in}{2.066945in}}%
\pgfpathlineto{\pgfqpoint{2.018307in}{2.055393in}}%
\pgfpathlineto{\pgfqpoint{2.035956in}{2.045434in}}%
\pgfpathlineto{\pgfqpoint{2.047364in}{2.036671in}}%
\pgfpathlineto{\pgfqpoint{2.054314in}{2.028704in}}%
\pgfpathlineto{\pgfqpoint{2.057751in}{2.021533in}}%
\pgfpathlineto{\pgfqpoint{2.058534in}{2.014762in}}%
\pgfpathlineto{\pgfqpoint{2.057083in}{2.008388in}}%
\pgfpathlineto{\pgfqpoint{2.053229in}{2.001616in}}%
\pgfpathlineto{\pgfqpoint{2.046126in}{1.994047in}}%
\pgfpathlineto{\pgfqpoint{2.034897in}{1.985682in}}%
\pgfpathlineto{\pgfqpoint{2.018627in}{1.976520in}}%
\pgfpathlineto{\pgfqpoint{1.995418in}{1.966163in}}%
\pgfpathlineto{\pgfqpoint{1.963829in}{1.954611in}}%
\pgfpathlineto{\pgfqpoint{1.921168in}{1.941466in}}%
\pgfpathlineto{\pgfqpoint{1.862713in}{1.925930in}}%
\pgfpathlineto{\pgfqpoint{1.781394in}{1.906809in}}%
\pgfpathlineto{\pgfqpoint{1.639439in}{1.876136in}}%
\pgfpathlineto{\pgfqpoint{1.504846in}{1.846260in}}%
\pgfpathlineto{\pgfqpoint{1.435923in}{1.828733in}}%
\pgfpathlineto{\pgfqpoint{1.389714in}{1.814791in}}%
\pgfpathlineto{\pgfqpoint{1.358565in}{1.803239in}}%
\pgfpathlineto{\pgfqpoint{1.337580in}{1.793280in}}%
\pgfpathlineto{\pgfqpoint{1.323952in}{1.784516in}}%
\pgfpathlineto{\pgfqpoint{1.315998in}{1.776948in}}%
\pgfpathlineto{\pgfqpoint{1.311954in}{1.770176in}}%
\pgfpathlineto{\pgfqpoint{1.310826in}{1.764201in}}%
\pgfpathlineto{\pgfqpoint{1.311992in}{1.758225in}}%
\pgfpathlineto{\pgfqpoint{1.315439in}{1.752250in}}%
\pgfpathlineto{\pgfqpoint{1.322068in}{1.745478in}}%
\pgfpathlineto{\pgfqpoint{1.332819in}{1.737910in}}%
\pgfpathlineto{\pgfqpoint{1.349504in}{1.729146in}}%
\pgfpathlineto{\pgfqpoint{1.373652in}{1.719187in}}%
\pgfpathlineto{\pgfqpoint{1.407946in}{1.707635in}}%
\pgfpathlineto{\pgfqpoint{1.455554in}{1.694091in}}%
\pgfpathlineto{\pgfqpoint{1.522959in}{1.677361in}}%
\pgfpathlineto{\pgfqpoint{1.634567in}{1.652265in}}%
\pgfpathlineto{\pgfqpoint{1.807676in}{1.613227in}}%
\pgfpathlineto{\pgfqpoint{1.880358in}{1.594504in}}%
\pgfpathlineto{\pgfqpoint{1.930820in}{1.579367in}}%
\pgfpathlineto{\pgfqpoint{1.967422in}{1.566222in}}%
\pgfpathlineto{\pgfqpoint{1.993444in}{1.554670in}}%
\pgfpathlineto{\pgfqpoint{2.011567in}{1.544313in}}%
\pgfpathlineto{\pgfqpoint{2.023330in}{1.535151in}}%
\pgfpathlineto{\pgfqpoint{2.030222in}{1.527184in}}%
\pgfpathlineto{\pgfqpoint{2.033736in}{1.520013in}}%
\pgfpathlineto{\pgfqpoint{2.034703in}{1.513241in}}%
\pgfpathlineto{\pgfqpoint{2.033383in}{1.506470in}}%
\pgfpathlineto{\pgfqpoint{2.029787in}{1.499698in}}%
\pgfpathlineto{\pgfqpoint{2.023098in}{1.492129in}}%
\pgfpathlineto{\pgfqpoint{2.012480in}{1.483764in}}%
\pgfpathlineto{\pgfqpoint{1.997062in}{1.474602in}}%
\pgfpathlineto{\pgfqpoint{1.975043in}{1.464245in}}%
\pgfpathlineto{\pgfqpoint{1.945059in}{1.452693in}}%
\pgfpathlineto{\pgfqpoint{1.904566in}{1.439547in}}%
\pgfpathlineto{\pgfqpoint{1.849112in}{1.424012in}}%
\pgfpathlineto{\pgfqpoint{1.772050in}{1.404891in}}%
\pgfpathlineto{\pgfqpoint{1.632526in}{1.373023in}}%
\pgfpathlineto{\pgfqpoint{1.510900in}{1.344342in}}%
\pgfpathlineto{\pgfqpoint{1.447439in}{1.327213in}}%
\pgfpathlineto{\pgfqpoint{1.403837in}{1.313271in}}%
\pgfpathlineto{\pgfqpoint{1.374440in}{1.301719in}}%
\pgfpathlineto{\pgfqpoint{1.354624in}{1.291760in}}%
\pgfpathlineto{\pgfqpoint{1.341743in}{1.282996in}}%
\pgfpathlineto{\pgfqpoint{1.334208in}{1.275428in}}%
\pgfpathlineto{\pgfqpoint{1.330359in}{1.268656in}}%
\pgfpathlineto{\pgfqpoint{1.329264in}{1.262282in}}%
\pgfpathlineto{\pgfqpoint{1.330467in}{1.256307in}}%
\pgfpathlineto{\pgfqpoint{1.334117in}{1.249933in}}%
\pgfpathlineto{\pgfqpoint{1.340632in}{1.243162in}}%
\pgfpathlineto{\pgfqpoint{1.351052in}{1.235593in}}%
\pgfpathlineto{\pgfqpoint{1.367091in}{1.226829in}}%
\pgfpathlineto{\pgfqpoint{1.390179in}{1.216871in}}%
\pgfpathlineto{\pgfqpoint{1.422845in}{1.205318in}}%
\pgfpathlineto{\pgfqpoint{1.468062in}{1.191775in}}%
\pgfpathlineto{\pgfqpoint{1.533545in}{1.174646in}}%
\pgfpathlineto{\pgfqpoint{1.642657in}{1.148753in}}%
\pgfpathlineto{\pgfqpoint{1.799335in}{1.111308in}}%
\pgfpathlineto{\pgfqpoint{1.867972in}{1.092586in}}%
\pgfpathlineto{\pgfqpoint{1.915573in}{1.077449in}}%
\pgfpathlineto{\pgfqpoint{1.950037in}{1.064303in}}%
\pgfpathlineto{\pgfqpoint{1.974466in}{1.052751in}}%
\pgfpathlineto{\pgfqpoint{1.991399in}{1.042394in}}%
\pgfpathlineto{\pgfqpoint{2.002303in}{1.033232in}}%
\pgfpathlineto{\pgfqpoint{2.008599in}{1.025265in}}%
\pgfpathlineto{\pgfqpoint{2.011799in}{1.017697in}}%
\pgfpathlineto{\pgfqpoint{2.012353in}{1.010925in}}%
\pgfpathlineto{\pgfqpoint{2.010727in}{1.004153in}}%
\pgfpathlineto{\pgfqpoint{2.006640in}{0.996982in}}%
\pgfpathlineto{\pgfqpoint{1.999717in}{0.989414in}}%
\pgfpathlineto{\pgfqpoint{1.989004in}{0.981049in}}%
\pgfpathlineto{\pgfqpoint{1.972932in}{0.971488in}}%
\pgfpathlineto{\pgfqpoint{1.950180in}{0.960733in}}%
\pgfpathlineto{\pgfqpoint{1.919466in}{0.948782in}}%
\pgfpathlineto{\pgfqpoint{1.877103in}{0.934840in}}%
\pgfpathlineto{\pgfqpoint{1.819897in}{0.918508in}}%
\pgfpathlineto{\pgfqpoint{1.737073in}{0.897395in}}%
\pgfpathlineto{\pgfqpoint{1.487444in}{0.835253in}}%
\pgfpathlineto{\pgfqpoint{1.437037in}{0.819718in}}%
\pgfpathlineto{\pgfqpoint{1.402748in}{0.806970in}}%
\pgfpathlineto{\pgfqpoint{1.378880in}{0.795817in}}%
\pgfpathlineto{\pgfqpoint{1.363438in}{0.786256in}}%
\pgfpathlineto{\pgfqpoint{1.355465in}{0.779484in}}%
\pgfpathlineto{\pgfqpoint{1.355465in}{0.779484in}}%
\pgfusepath{stroke}%
\end{pgfscope}%
\begin{pgfscope}%
\pgfpathrectangle{\pgfqpoint{0.625000in}{0.660000in}}{\pgfqpoint{3.875000in}{2.310000in}}%
\pgfusepath{clip}%
\pgfsetbuttcap%
\pgfsetroundjoin%
\pgfsetlinewidth{1.505625pt}%
\definecolor{currentstroke}{rgb}{0.121569,0.466667,0.705882}%
\pgfsetstrokecolor{currentstroke}%
\pgfsetdash{{5.550000pt}{2.400000pt}}{0.000000pt}%
\pgfpathmoveto{\pgfqpoint{1.672006in}{0.660000in}}%
\pgfpathlineto{\pgfqpoint{1.672006in}{2.970000in}}%
\pgfusepath{stroke}%
\end{pgfscope}%
\begin{pgfscope}%
\pgfsetrectcap%
\pgfsetmiterjoin%
\pgfsetlinewidth{0.803000pt}%
\definecolor{currentstroke}{rgb}{0.000000,0.000000,0.000000}%
\pgfsetstrokecolor{currentstroke}%
\pgfsetdash{}{0pt}%
\pgfpathmoveto{\pgfqpoint{0.625000in}{0.660000in}}%
\pgfpathlineto{\pgfqpoint{0.625000in}{2.970000in}}%
\pgfusepath{stroke}%
\end{pgfscope}%
\begin{pgfscope}%
\pgfsetrectcap%
\pgfsetmiterjoin%
\pgfsetlinewidth{0.803000pt}%
\definecolor{currentstroke}{rgb}{0.000000,0.000000,0.000000}%
\pgfsetstrokecolor{currentstroke}%
\pgfsetdash{}{0pt}%
\pgfpathmoveto{\pgfqpoint{4.500000in}{0.660000in}}%
\pgfpathlineto{\pgfqpoint{4.500000in}{2.970000in}}%
\pgfusepath{stroke}%
\end{pgfscope}%
\begin{pgfscope}%
\pgfsetrectcap%
\pgfsetmiterjoin%
\pgfsetlinewidth{0.803000pt}%
\definecolor{currentstroke}{rgb}{0.000000,0.000000,0.000000}%
\pgfsetstrokecolor{currentstroke}%
\pgfsetdash{}{0pt}%
\pgfpathmoveto{\pgfqpoint{0.625000in}{0.660000in}}%
\pgfpathlineto{\pgfqpoint{4.500000in}{0.660000in}}%
\pgfusepath{stroke}%
\end{pgfscope}%
\begin{pgfscope}%
\pgfsetrectcap%
\pgfsetmiterjoin%
\pgfsetlinewidth{0.803000pt}%
\definecolor{currentstroke}{rgb}{0.000000,0.000000,0.000000}%
\pgfsetstrokecolor{currentstroke}%
\pgfsetdash{}{0pt}%
\pgfpathmoveto{\pgfqpoint{0.625000in}{2.970000in}}%
\pgfpathlineto{\pgfqpoint{4.500000in}{2.970000in}}%
\pgfusepath{stroke}%
\end{pgfscope}%
\begin{pgfscope}%
\pgfsetbuttcap%
\pgfsetmiterjoin%
\definecolor{currentfill}{rgb}{1.000000,1.000000,1.000000}%
\pgfsetfillcolor{currentfill}%
\pgfsetfillopacity{0.800000}%
\pgfsetlinewidth{1.003750pt}%
\definecolor{currentstroke}{rgb}{0.800000,0.800000,0.800000}%
\pgfsetstrokecolor{currentstroke}%
\pgfsetstrokeopacity{0.800000}%
\pgfsetdash{}{0pt}%
\pgfpathmoveto{\pgfqpoint{2.600708in}{2.455108in}}%
\pgfpathlineto{\pgfqpoint{4.402778in}{2.455108in}}%
\pgfpathquadraticcurveto{\pgfqpoint{4.430556in}{2.455108in}}{\pgfqpoint{4.430556in}{2.482886in}}%
\pgfpathlineto{\pgfqpoint{4.430556in}{2.872778in}}%
\pgfpathquadraticcurveto{\pgfqpoint{4.430556in}{2.900556in}}{\pgfqpoint{4.402778in}{2.900556in}}%
\pgfpathlineto{\pgfqpoint{2.600708in}{2.900556in}}%
\pgfpathquadraticcurveto{\pgfqpoint{2.572930in}{2.900556in}}{\pgfqpoint{2.572930in}{2.872778in}}%
\pgfpathlineto{\pgfqpoint{2.572930in}{2.482886in}}%
\pgfpathquadraticcurveto{\pgfqpoint{2.572930in}{2.455108in}}{\pgfqpoint{2.600708in}{2.455108in}}%
\pgfpathlineto{\pgfqpoint{2.600708in}{2.455108in}}%
\pgfpathclose%
\pgfusepath{stroke,fill}%
\end{pgfscope}%
\begin{pgfscope}%
\pgfsetrectcap%
\pgfsetroundjoin%
\pgfsetlinewidth{1.505625pt}%
\definecolor{currentstroke}{rgb}{0.121569,0.466667,0.705882}%
\pgfsetstrokecolor{currentstroke}%
\pgfsetdash{}{0pt}%
\pgfpathmoveto{\pgfqpoint{2.628486in}{2.791411in}}%
\pgfpathlineto{\pgfqpoint{2.767375in}{2.791411in}}%
\pgfpathlineto{\pgfqpoint{2.906264in}{2.791411in}}%
\pgfusepath{stroke}%
\end{pgfscope}%
\begin{pgfscope}%
\definecolor{textcolor}{rgb}{0.000000,0.000000,0.000000}%
\pgfsetstrokecolor{textcolor}%
\pgfsetfillcolor{textcolor}%
\pgftext[x=3.017375in,y=2.742800in,left,base]{\color{textcolor}\rmfamily\fontsize{10.000000}{12.000000}\selectfont delta}%
\end{pgfscope}%
\begin{pgfscope}%
\pgfsetbuttcap%
\pgfsetroundjoin%
\pgfsetlinewidth{1.505625pt}%
\definecolor{currentstroke}{rgb}{0.121569,0.466667,0.705882}%
\pgfsetstrokecolor{currentstroke}%
\pgfsetdash{{5.550000pt}{2.400000pt}}{0.000000pt}%
\pgfpathmoveto{\pgfqpoint{2.628486in}{2.589521in}}%
\pgfpathlineto{\pgfqpoint{2.767375in}{2.589521in}}%
\pgfpathlineto{\pgfqpoint{2.906264in}{2.589521in}}%
\pgfusepath{stroke}%
\end{pgfscope}%
\begin{pgfscope}%
\definecolor{textcolor}{rgb}{0.000000,0.000000,0.000000}%
\pgfsetstrokecolor{textcolor}%
\pgfsetfillcolor{textcolor}%
\pgftext[x=3.017375in,y=2.540909in,left,base]{\color{textcolor}\rmfamily\fontsize{10.000000}{12.000000}\selectfont new stable convergent}%
\end{pgfscope}%
\end{pgfpicture}%
\makeatother%
\endgroup%


%% Creator: Matplotlib, PGF backend
%%
%% To include the figure in your LaTeX document, write
%%   \input{<filename>.pgf}
%%
%% Make sure the required packages are loaded in your preamble
%%   \usepackage{pgf}
%%
%% Also ensure that all the required font packages are loaded; for instance,
%% the lmodern package is sometimes necessary when using math font.
%%   \usepackage{lmodern}
%%
%% Figures using additional raster images can only be included by \input if
%% they are in the same directory as the main LaTeX file. For loading figures
%% from other directories you can use the `import` package
%%   \usepackage{import}
%%
%% and then include the figures with
%%   \import{<path to file>}{<filename>.pgf}
%%
%% Matplotlib used the following preamble
%%   
%%   \usepackage{fontspec}
%%   \setmainfont{Charter.ttc}[Path=\detokenize{/System/Library/Fonts/Supplemental/}]
%%   \setsansfont{DejaVuSans.ttf}[Path=\detokenize{/opt/homebrew/lib/python3.10/site-packages/matplotlib/mpl-data/fonts/ttf/}]
%%   \setmonofont{DejaVuSansMono.ttf}[Path=\detokenize{/opt/homebrew/lib/python3.10/site-packages/matplotlib/mpl-data/fonts/ttf/}]
%%   \makeatletter\@ifpackageloaded{underscore}{}{\usepackage[strings]{underscore}}\makeatother
%%
\begingroup%
\makeatletter%
\begin{pgfpicture}%
\pgfpathrectangle{\pgfpointorigin}{\pgfqpoint{6.400000in}{4.800000in}}%
\pgfusepath{use as bounding box, clip}%
\begin{pgfscope}%
\pgfsetbuttcap%
\pgfsetmiterjoin%
\definecolor{currentfill}{rgb}{1.000000,1.000000,1.000000}%
\pgfsetfillcolor{currentfill}%
\pgfsetlinewidth{0.000000pt}%
\definecolor{currentstroke}{rgb}{1.000000,1.000000,1.000000}%
\pgfsetstrokecolor{currentstroke}%
\pgfsetdash{}{0pt}%
\pgfpathmoveto{\pgfqpoint{0.000000in}{0.000000in}}%
\pgfpathlineto{\pgfqpoint{6.400000in}{0.000000in}}%
\pgfpathlineto{\pgfqpoint{6.400000in}{4.800000in}}%
\pgfpathlineto{\pgfqpoint{0.000000in}{4.800000in}}%
\pgfpathlineto{\pgfqpoint{0.000000in}{0.000000in}}%
\pgfpathclose%
\pgfusepath{fill}%
\end{pgfscope}%
\begin{pgfscope}%
\pgfsetbuttcap%
\pgfsetmiterjoin%
\definecolor{currentfill}{rgb}{1.000000,1.000000,1.000000}%
\pgfsetfillcolor{currentfill}%
\pgfsetlinewidth{0.000000pt}%
\definecolor{currentstroke}{rgb}{0.000000,0.000000,0.000000}%
\pgfsetstrokecolor{currentstroke}%
\pgfsetstrokeopacity{0.000000}%
\pgfsetdash{}{0pt}%
\pgfpathmoveto{\pgfqpoint{0.800000in}{0.528000in}}%
\pgfpathlineto{\pgfqpoint{5.760000in}{0.528000in}}%
\pgfpathlineto{\pgfqpoint{5.760000in}{4.224000in}}%
\pgfpathlineto{\pgfqpoint{0.800000in}{4.224000in}}%
\pgfpathlineto{\pgfqpoint{0.800000in}{0.528000in}}%
\pgfpathclose%
\pgfusepath{fill}%
\end{pgfscope}%
\begin{pgfscope}%
\pgfpathrectangle{\pgfqpoint{0.800000in}{0.528000in}}{\pgfqpoint{4.960000in}{3.696000in}}%
\pgfusepath{clip}%
\pgfsetrectcap%
\pgfsetroundjoin%
\pgfsetlinewidth{0.803000pt}%
\definecolor{currentstroke}{rgb}{0.690196,0.690196,0.690196}%
\pgfsetstrokecolor{currentstroke}%
\pgfsetdash{}{0pt}%
\pgfpathmoveto{\pgfqpoint{1.056511in}{0.528000in}}%
\pgfpathlineto{\pgfqpoint{1.056511in}{4.224000in}}%
\pgfusepath{stroke}%
\end{pgfscope}%
\begin{pgfscope}%
\pgfsetbuttcap%
\pgfsetroundjoin%
\definecolor{currentfill}{rgb}{0.000000,0.000000,0.000000}%
\pgfsetfillcolor{currentfill}%
\pgfsetlinewidth{0.803000pt}%
\definecolor{currentstroke}{rgb}{0.000000,0.000000,0.000000}%
\pgfsetstrokecolor{currentstroke}%
\pgfsetdash{}{0pt}%
\pgfsys@defobject{currentmarker}{\pgfqpoint{0.000000in}{-0.048611in}}{\pgfqpoint{0.000000in}{0.000000in}}{%
\pgfpathmoveto{\pgfqpoint{0.000000in}{0.000000in}}%
\pgfpathlineto{\pgfqpoint{0.000000in}{-0.048611in}}%
\pgfusepath{stroke,fill}%
}%
\begin{pgfscope}%
\pgfsys@transformshift{1.056511in}{0.528000in}%
\pgfsys@useobject{currentmarker}{}%
\end{pgfscope}%
\end{pgfscope}%
\begin{pgfscope}%
\definecolor{textcolor}{rgb}{0.000000,0.000000,0.000000}%
\pgfsetstrokecolor{textcolor}%
\pgfsetfillcolor{textcolor}%
\pgftext[x=1.056511in,y=0.430778in,,top]{\color{textcolor}\rmfamily\fontsize{10.000000}{12.000000}\selectfont \(\displaystyle {\ensuremath{-}1}\)}%
\end{pgfscope}%
\begin{pgfscope}%
\pgfpathrectangle{\pgfqpoint{0.800000in}{0.528000in}}{\pgfqpoint{4.960000in}{3.696000in}}%
\pgfusepath{clip}%
\pgfsetrectcap%
\pgfsetroundjoin%
\pgfsetlinewidth{0.803000pt}%
\definecolor{currentstroke}{rgb}{0.690196,0.690196,0.690196}%
\pgfsetstrokecolor{currentstroke}%
\pgfsetdash{}{0pt}%
\pgfpathmoveto{\pgfqpoint{1.911691in}{0.528000in}}%
\pgfpathlineto{\pgfqpoint{1.911691in}{4.224000in}}%
\pgfusepath{stroke}%
\end{pgfscope}%
\begin{pgfscope}%
\pgfsetbuttcap%
\pgfsetroundjoin%
\definecolor{currentfill}{rgb}{0.000000,0.000000,0.000000}%
\pgfsetfillcolor{currentfill}%
\pgfsetlinewidth{0.803000pt}%
\definecolor{currentstroke}{rgb}{0.000000,0.000000,0.000000}%
\pgfsetstrokecolor{currentstroke}%
\pgfsetdash{}{0pt}%
\pgfsys@defobject{currentmarker}{\pgfqpoint{0.000000in}{-0.048611in}}{\pgfqpoint{0.000000in}{0.000000in}}{%
\pgfpathmoveto{\pgfqpoint{0.000000in}{0.000000in}}%
\pgfpathlineto{\pgfqpoint{0.000000in}{-0.048611in}}%
\pgfusepath{stroke,fill}%
}%
\begin{pgfscope}%
\pgfsys@transformshift{1.911691in}{0.528000in}%
\pgfsys@useobject{currentmarker}{}%
\end{pgfscope}%
\end{pgfscope}%
\begin{pgfscope}%
\definecolor{textcolor}{rgb}{0.000000,0.000000,0.000000}%
\pgfsetstrokecolor{textcolor}%
\pgfsetfillcolor{textcolor}%
\pgftext[x=1.911691in,y=0.430778in,,top]{\color{textcolor}\rmfamily\fontsize{10.000000}{12.000000}\selectfont \(\displaystyle {0}\)}%
\end{pgfscope}%
\begin{pgfscope}%
\pgfpathrectangle{\pgfqpoint{0.800000in}{0.528000in}}{\pgfqpoint{4.960000in}{3.696000in}}%
\pgfusepath{clip}%
\pgfsetrectcap%
\pgfsetroundjoin%
\pgfsetlinewidth{0.803000pt}%
\definecolor{currentstroke}{rgb}{0.690196,0.690196,0.690196}%
\pgfsetstrokecolor{currentstroke}%
\pgfsetdash{}{0pt}%
\pgfpathmoveto{\pgfqpoint{2.766871in}{0.528000in}}%
\pgfpathlineto{\pgfqpoint{2.766871in}{4.224000in}}%
\pgfusepath{stroke}%
\end{pgfscope}%
\begin{pgfscope}%
\pgfsetbuttcap%
\pgfsetroundjoin%
\definecolor{currentfill}{rgb}{0.000000,0.000000,0.000000}%
\pgfsetfillcolor{currentfill}%
\pgfsetlinewidth{0.803000pt}%
\definecolor{currentstroke}{rgb}{0.000000,0.000000,0.000000}%
\pgfsetstrokecolor{currentstroke}%
\pgfsetdash{}{0pt}%
\pgfsys@defobject{currentmarker}{\pgfqpoint{0.000000in}{-0.048611in}}{\pgfqpoint{0.000000in}{0.000000in}}{%
\pgfpathmoveto{\pgfqpoint{0.000000in}{0.000000in}}%
\pgfpathlineto{\pgfqpoint{0.000000in}{-0.048611in}}%
\pgfusepath{stroke,fill}%
}%
\begin{pgfscope}%
\pgfsys@transformshift{2.766871in}{0.528000in}%
\pgfsys@useobject{currentmarker}{}%
\end{pgfscope}%
\end{pgfscope}%
\begin{pgfscope}%
\definecolor{textcolor}{rgb}{0.000000,0.000000,0.000000}%
\pgfsetstrokecolor{textcolor}%
\pgfsetfillcolor{textcolor}%
\pgftext[x=2.766871in,y=0.430778in,,top]{\color{textcolor}\rmfamily\fontsize{10.000000}{12.000000}\selectfont \(\displaystyle {1}\)}%
\end{pgfscope}%
\begin{pgfscope}%
\pgfpathrectangle{\pgfqpoint{0.800000in}{0.528000in}}{\pgfqpoint{4.960000in}{3.696000in}}%
\pgfusepath{clip}%
\pgfsetrectcap%
\pgfsetroundjoin%
\pgfsetlinewidth{0.803000pt}%
\definecolor{currentstroke}{rgb}{0.690196,0.690196,0.690196}%
\pgfsetstrokecolor{currentstroke}%
\pgfsetdash{}{0pt}%
\pgfpathmoveto{\pgfqpoint{3.622051in}{0.528000in}}%
\pgfpathlineto{\pgfqpoint{3.622051in}{4.224000in}}%
\pgfusepath{stroke}%
\end{pgfscope}%
\begin{pgfscope}%
\pgfsetbuttcap%
\pgfsetroundjoin%
\definecolor{currentfill}{rgb}{0.000000,0.000000,0.000000}%
\pgfsetfillcolor{currentfill}%
\pgfsetlinewidth{0.803000pt}%
\definecolor{currentstroke}{rgb}{0.000000,0.000000,0.000000}%
\pgfsetstrokecolor{currentstroke}%
\pgfsetdash{}{0pt}%
\pgfsys@defobject{currentmarker}{\pgfqpoint{0.000000in}{-0.048611in}}{\pgfqpoint{0.000000in}{0.000000in}}{%
\pgfpathmoveto{\pgfqpoint{0.000000in}{0.000000in}}%
\pgfpathlineto{\pgfqpoint{0.000000in}{-0.048611in}}%
\pgfusepath{stroke,fill}%
}%
\begin{pgfscope}%
\pgfsys@transformshift{3.622051in}{0.528000in}%
\pgfsys@useobject{currentmarker}{}%
\end{pgfscope}%
\end{pgfscope}%
\begin{pgfscope}%
\definecolor{textcolor}{rgb}{0.000000,0.000000,0.000000}%
\pgfsetstrokecolor{textcolor}%
\pgfsetfillcolor{textcolor}%
\pgftext[x=3.622051in,y=0.430778in,,top]{\color{textcolor}\rmfamily\fontsize{10.000000}{12.000000}\selectfont \(\displaystyle {2}\)}%
\end{pgfscope}%
\begin{pgfscope}%
\pgfpathrectangle{\pgfqpoint{0.800000in}{0.528000in}}{\pgfqpoint{4.960000in}{3.696000in}}%
\pgfusepath{clip}%
\pgfsetrectcap%
\pgfsetroundjoin%
\pgfsetlinewidth{0.803000pt}%
\definecolor{currentstroke}{rgb}{0.690196,0.690196,0.690196}%
\pgfsetstrokecolor{currentstroke}%
\pgfsetdash{}{0pt}%
\pgfpathmoveto{\pgfqpoint{4.477230in}{0.528000in}}%
\pgfpathlineto{\pgfqpoint{4.477230in}{4.224000in}}%
\pgfusepath{stroke}%
\end{pgfscope}%
\begin{pgfscope}%
\pgfsetbuttcap%
\pgfsetroundjoin%
\definecolor{currentfill}{rgb}{0.000000,0.000000,0.000000}%
\pgfsetfillcolor{currentfill}%
\pgfsetlinewidth{0.803000pt}%
\definecolor{currentstroke}{rgb}{0.000000,0.000000,0.000000}%
\pgfsetstrokecolor{currentstroke}%
\pgfsetdash{}{0pt}%
\pgfsys@defobject{currentmarker}{\pgfqpoint{0.000000in}{-0.048611in}}{\pgfqpoint{0.000000in}{0.000000in}}{%
\pgfpathmoveto{\pgfqpoint{0.000000in}{0.000000in}}%
\pgfpathlineto{\pgfqpoint{0.000000in}{-0.048611in}}%
\pgfusepath{stroke,fill}%
}%
\begin{pgfscope}%
\pgfsys@transformshift{4.477230in}{0.528000in}%
\pgfsys@useobject{currentmarker}{}%
\end{pgfscope}%
\end{pgfscope}%
\begin{pgfscope}%
\definecolor{textcolor}{rgb}{0.000000,0.000000,0.000000}%
\pgfsetstrokecolor{textcolor}%
\pgfsetfillcolor{textcolor}%
\pgftext[x=4.477230in,y=0.430778in,,top]{\color{textcolor}\rmfamily\fontsize{10.000000}{12.000000}\selectfont \(\displaystyle {3}\)}%
\end{pgfscope}%
\begin{pgfscope}%
\pgfpathrectangle{\pgfqpoint{0.800000in}{0.528000in}}{\pgfqpoint{4.960000in}{3.696000in}}%
\pgfusepath{clip}%
\pgfsetrectcap%
\pgfsetroundjoin%
\pgfsetlinewidth{0.803000pt}%
\definecolor{currentstroke}{rgb}{0.690196,0.690196,0.690196}%
\pgfsetstrokecolor{currentstroke}%
\pgfsetdash{}{0pt}%
\pgfpathmoveto{\pgfqpoint{5.332410in}{0.528000in}}%
\pgfpathlineto{\pgfqpoint{5.332410in}{4.224000in}}%
\pgfusepath{stroke}%
\end{pgfscope}%
\begin{pgfscope}%
\pgfsetbuttcap%
\pgfsetroundjoin%
\definecolor{currentfill}{rgb}{0.000000,0.000000,0.000000}%
\pgfsetfillcolor{currentfill}%
\pgfsetlinewidth{0.803000pt}%
\definecolor{currentstroke}{rgb}{0.000000,0.000000,0.000000}%
\pgfsetstrokecolor{currentstroke}%
\pgfsetdash{}{0pt}%
\pgfsys@defobject{currentmarker}{\pgfqpoint{0.000000in}{-0.048611in}}{\pgfqpoint{0.000000in}{0.000000in}}{%
\pgfpathmoveto{\pgfqpoint{0.000000in}{0.000000in}}%
\pgfpathlineto{\pgfqpoint{0.000000in}{-0.048611in}}%
\pgfusepath{stroke,fill}%
}%
\begin{pgfscope}%
\pgfsys@transformshift{5.332410in}{0.528000in}%
\pgfsys@useobject{currentmarker}{}%
\end{pgfscope}%
\end{pgfscope}%
\begin{pgfscope}%
\definecolor{textcolor}{rgb}{0.000000,0.000000,0.000000}%
\pgfsetstrokecolor{textcolor}%
\pgfsetfillcolor{textcolor}%
\pgftext[x=5.332410in,y=0.430778in,,top]{\color{textcolor}\rmfamily\fontsize{10.000000}{12.000000}\selectfont \(\displaystyle {4}\)}%
\end{pgfscope}%
\begin{pgfscope}%
\pgfpathrectangle{\pgfqpoint{0.800000in}{0.528000in}}{\pgfqpoint{4.960000in}{3.696000in}}%
\pgfusepath{clip}%
\pgfsetrectcap%
\pgfsetroundjoin%
\pgfsetlinewidth{0.803000pt}%
\definecolor{currentstroke}{rgb}{0.690196,0.690196,0.690196}%
\pgfsetstrokecolor{currentstroke}%
\pgfsetdash{}{0pt}%
\pgfpathmoveto{\pgfqpoint{0.800000in}{0.528000in}}%
\pgfpathlineto{\pgfqpoint{5.760000in}{0.528000in}}%
\pgfusepath{stroke}%
\end{pgfscope}%
\begin{pgfscope}%
\pgfsetbuttcap%
\pgfsetroundjoin%
\definecolor{currentfill}{rgb}{0.000000,0.000000,0.000000}%
\pgfsetfillcolor{currentfill}%
\pgfsetlinewidth{0.803000pt}%
\definecolor{currentstroke}{rgb}{0.000000,0.000000,0.000000}%
\pgfsetstrokecolor{currentstroke}%
\pgfsetdash{}{0pt}%
\pgfsys@defobject{currentmarker}{\pgfqpoint{-0.048611in}{0.000000in}}{\pgfqpoint{-0.000000in}{0.000000in}}{%
\pgfpathmoveto{\pgfqpoint{-0.000000in}{0.000000in}}%
\pgfpathlineto{\pgfqpoint{-0.048611in}{0.000000in}}%
\pgfusepath{stroke,fill}%
}%
\begin{pgfscope}%
\pgfsys@transformshift{0.800000in}{0.528000in}%
\pgfsys@useobject{currentmarker}{}%
\end{pgfscope}%
\end{pgfscope}%
\begin{pgfscope}%
\definecolor{textcolor}{rgb}{0.000000,0.000000,0.000000}%
\pgfsetstrokecolor{textcolor}%
\pgfsetfillcolor{textcolor}%
\pgftext[x=0.525308in, y=0.476900in, left, base]{\color{textcolor}\rmfamily\fontsize{10.000000}{12.000000}\selectfont \(\displaystyle {0.0}\)}%
\end{pgfscope}%
\begin{pgfscope}%
\pgfpathrectangle{\pgfqpoint{0.800000in}{0.528000in}}{\pgfqpoint{4.960000in}{3.696000in}}%
\pgfusepath{clip}%
\pgfsetrectcap%
\pgfsetroundjoin%
\pgfsetlinewidth{0.803000pt}%
\definecolor{currentstroke}{rgb}{0.690196,0.690196,0.690196}%
\pgfsetstrokecolor{currentstroke}%
\pgfsetdash{}{0pt}%
\pgfpathmoveto{\pgfqpoint{0.800000in}{0.998808in}}%
\pgfpathlineto{\pgfqpoint{5.760000in}{0.998808in}}%
\pgfusepath{stroke}%
\end{pgfscope}%
\begin{pgfscope}%
\pgfsetbuttcap%
\pgfsetroundjoin%
\definecolor{currentfill}{rgb}{0.000000,0.000000,0.000000}%
\pgfsetfillcolor{currentfill}%
\pgfsetlinewidth{0.803000pt}%
\definecolor{currentstroke}{rgb}{0.000000,0.000000,0.000000}%
\pgfsetstrokecolor{currentstroke}%
\pgfsetdash{}{0pt}%
\pgfsys@defobject{currentmarker}{\pgfqpoint{-0.048611in}{0.000000in}}{\pgfqpoint{-0.000000in}{0.000000in}}{%
\pgfpathmoveto{\pgfqpoint{-0.000000in}{0.000000in}}%
\pgfpathlineto{\pgfqpoint{-0.048611in}{0.000000in}}%
\pgfusepath{stroke,fill}%
}%
\begin{pgfscope}%
\pgfsys@transformshift{0.800000in}{0.998808in}%
\pgfsys@useobject{currentmarker}{}%
\end{pgfscope}%
\end{pgfscope}%
\begin{pgfscope}%
\definecolor{textcolor}{rgb}{0.000000,0.000000,0.000000}%
\pgfsetstrokecolor{textcolor}%
\pgfsetfillcolor{textcolor}%
\pgftext[x=0.525308in, y=0.947708in, left, base]{\color{textcolor}\rmfamily\fontsize{10.000000}{12.000000}\selectfont \(\displaystyle {0.1}\)}%
\end{pgfscope}%
\begin{pgfscope}%
\pgfpathrectangle{\pgfqpoint{0.800000in}{0.528000in}}{\pgfqpoint{4.960000in}{3.696000in}}%
\pgfusepath{clip}%
\pgfsetrectcap%
\pgfsetroundjoin%
\pgfsetlinewidth{0.803000pt}%
\definecolor{currentstroke}{rgb}{0.690196,0.690196,0.690196}%
\pgfsetstrokecolor{currentstroke}%
\pgfsetdash{}{0pt}%
\pgfpathmoveto{\pgfqpoint{0.800000in}{1.469616in}}%
\pgfpathlineto{\pgfqpoint{5.760000in}{1.469616in}}%
\pgfusepath{stroke}%
\end{pgfscope}%
\begin{pgfscope}%
\pgfsetbuttcap%
\pgfsetroundjoin%
\definecolor{currentfill}{rgb}{0.000000,0.000000,0.000000}%
\pgfsetfillcolor{currentfill}%
\pgfsetlinewidth{0.803000pt}%
\definecolor{currentstroke}{rgb}{0.000000,0.000000,0.000000}%
\pgfsetstrokecolor{currentstroke}%
\pgfsetdash{}{0pt}%
\pgfsys@defobject{currentmarker}{\pgfqpoint{-0.048611in}{0.000000in}}{\pgfqpoint{-0.000000in}{0.000000in}}{%
\pgfpathmoveto{\pgfqpoint{-0.000000in}{0.000000in}}%
\pgfpathlineto{\pgfqpoint{-0.048611in}{0.000000in}}%
\pgfusepath{stroke,fill}%
}%
\begin{pgfscope}%
\pgfsys@transformshift{0.800000in}{1.469616in}%
\pgfsys@useobject{currentmarker}{}%
\end{pgfscope}%
\end{pgfscope}%
\begin{pgfscope}%
\definecolor{textcolor}{rgb}{0.000000,0.000000,0.000000}%
\pgfsetstrokecolor{textcolor}%
\pgfsetfillcolor{textcolor}%
\pgftext[x=0.525308in, y=1.418516in, left, base]{\color{textcolor}\rmfamily\fontsize{10.000000}{12.000000}\selectfont \(\displaystyle {0.2}\)}%
\end{pgfscope}%
\begin{pgfscope}%
\pgfpathrectangle{\pgfqpoint{0.800000in}{0.528000in}}{\pgfqpoint{4.960000in}{3.696000in}}%
\pgfusepath{clip}%
\pgfsetrectcap%
\pgfsetroundjoin%
\pgfsetlinewidth{0.803000pt}%
\definecolor{currentstroke}{rgb}{0.690196,0.690196,0.690196}%
\pgfsetstrokecolor{currentstroke}%
\pgfsetdash{}{0pt}%
\pgfpathmoveto{\pgfqpoint{0.800000in}{1.940425in}}%
\pgfpathlineto{\pgfqpoint{5.760000in}{1.940425in}}%
\pgfusepath{stroke}%
\end{pgfscope}%
\begin{pgfscope}%
\pgfsetbuttcap%
\pgfsetroundjoin%
\definecolor{currentfill}{rgb}{0.000000,0.000000,0.000000}%
\pgfsetfillcolor{currentfill}%
\pgfsetlinewidth{0.803000pt}%
\definecolor{currentstroke}{rgb}{0.000000,0.000000,0.000000}%
\pgfsetstrokecolor{currentstroke}%
\pgfsetdash{}{0pt}%
\pgfsys@defobject{currentmarker}{\pgfqpoint{-0.048611in}{0.000000in}}{\pgfqpoint{-0.000000in}{0.000000in}}{%
\pgfpathmoveto{\pgfqpoint{-0.000000in}{0.000000in}}%
\pgfpathlineto{\pgfqpoint{-0.048611in}{0.000000in}}%
\pgfusepath{stroke,fill}%
}%
\begin{pgfscope}%
\pgfsys@transformshift{0.800000in}{1.940425in}%
\pgfsys@useobject{currentmarker}{}%
\end{pgfscope}%
\end{pgfscope}%
\begin{pgfscope}%
\definecolor{textcolor}{rgb}{0.000000,0.000000,0.000000}%
\pgfsetstrokecolor{textcolor}%
\pgfsetfillcolor{textcolor}%
\pgftext[x=0.525308in, y=1.889325in, left, base]{\color{textcolor}\rmfamily\fontsize{10.000000}{12.000000}\selectfont \(\displaystyle {0.3}\)}%
\end{pgfscope}%
\begin{pgfscope}%
\pgfpathrectangle{\pgfqpoint{0.800000in}{0.528000in}}{\pgfqpoint{4.960000in}{3.696000in}}%
\pgfusepath{clip}%
\pgfsetrectcap%
\pgfsetroundjoin%
\pgfsetlinewidth{0.803000pt}%
\definecolor{currentstroke}{rgb}{0.690196,0.690196,0.690196}%
\pgfsetstrokecolor{currentstroke}%
\pgfsetdash{}{0pt}%
\pgfpathmoveto{\pgfqpoint{0.800000in}{2.411233in}}%
\pgfpathlineto{\pgfqpoint{5.760000in}{2.411233in}}%
\pgfusepath{stroke}%
\end{pgfscope}%
\begin{pgfscope}%
\pgfsetbuttcap%
\pgfsetroundjoin%
\definecolor{currentfill}{rgb}{0.000000,0.000000,0.000000}%
\pgfsetfillcolor{currentfill}%
\pgfsetlinewidth{0.803000pt}%
\definecolor{currentstroke}{rgb}{0.000000,0.000000,0.000000}%
\pgfsetstrokecolor{currentstroke}%
\pgfsetdash{}{0pt}%
\pgfsys@defobject{currentmarker}{\pgfqpoint{-0.048611in}{0.000000in}}{\pgfqpoint{-0.000000in}{0.000000in}}{%
\pgfpathmoveto{\pgfqpoint{-0.000000in}{0.000000in}}%
\pgfpathlineto{\pgfqpoint{-0.048611in}{0.000000in}}%
\pgfusepath{stroke,fill}%
}%
\begin{pgfscope}%
\pgfsys@transformshift{0.800000in}{2.411233in}%
\pgfsys@useobject{currentmarker}{}%
\end{pgfscope}%
\end{pgfscope}%
\begin{pgfscope}%
\definecolor{textcolor}{rgb}{0.000000,0.000000,0.000000}%
\pgfsetstrokecolor{textcolor}%
\pgfsetfillcolor{textcolor}%
\pgftext[x=0.525308in, y=2.360133in, left, base]{\color{textcolor}\rmfamily\fontsize{10.000000}{12.000000}\selectfont \(\displaystyle {0.4}\)}%
\end{pgfscope}%
\begin{pgfscope}%
\pgfpathrectangle{\pgfqpoint{0.800000in}{0.528000in}}{\pgfqpoint{4.960000in}{3.696000in}}%
\pgfusepath{clip}%
\pgfsetrectcap%
\pgfsetroundjoin%
\pgfsetlinewidth{0.803000pt}%
\definecolor{currentstroke}{rgb}{0.690196,0.690196,0.690196}%
\pgfsetstrokecolor{currentstroke}%
\pgfsetdash{}{0pt}%
\pgfpathmoveto{\pgfqpoint{0.800000in}{2.882041in}}%
\pgfpathlineto{\pgfqpoint{5.760000in}{2.882041in}}%
\pgfusepath{stroke}%
\end{pgfscope}%
\begin{pgfscope}%
\pgfsetbuttcap%
\pgfsetroundjoin%
\definecolor{currentfill}{rgb}{0.000000,0.000000,0.000000}%
\pgfsetfillcolor{currentfill}%
\pgfsetlinewidth{0.803000pt}%
\definecolor{currentstroke}{rgb}{0.000000,0.000000,0.000000}%
\pgfsetstrokecolor{currentstroke}%
\pgfsetdash{}{0pt}%
\pgfsys@defobject{currentmarker}{\pgfqpoint{-0.048611in}{0.000000in}}{\pgfqpoint{-0.000000in}{0.000000in}}{%
\pgfpathmoveto{\pgfqpoint{-0.000000in}{0.000000in}}%
\pgfpathlineto{\pgfqpoint{-0.048611in}{0.000000in}}%
\pgfusepath{stroke,fill}%
}%
\begin{pgfscope}%
\pgfsys@transformshift{0.800000in}{2.882041in}%
\pgfsys@useobject{currentmarker}{}%
\end{pgfscope}%
\end{pgfscope}%
\begin{pgfscope}%
\definecolor{textcolor}{rgb}{0.000000,0.000000,0.000000}%
\pgfsetstrokecolor{textcolor}%
\pgfsetfillcolor{textcolor}%
\pgftext[x=0.525308in, y=2.830941in, left, base]{\color{textcolor}\rmfamily\fontsize{10.000000}{12.000000}\selectfont \(\displaystyle {0.5}\)}%
\end{pgfscope}%
\begin{pgfscope}%
\pgfpathrectangle{\pgfqpoint{0.800000in}{0.528000in}}{\pgfqpoint{4.960000in}{3.696000in}}%
\pgfusepath{clip}%
\pgfsetrectcap%
\pgfsetroundjoin%
\pgfsetlinewidth{0.803000pt}%
\definecolor{currentstroke}{rgb}{0.690196,0.690196,0.690196}%
\pgfsetstrokecolor{currentstroke}%
\pgfsetdash{}{0pt}%
\pgfpathmoveto{\pgfqpoint{0.800000in}{3.352849in}}%
\pgfpathlineto{\pgfqpoint{5.760000in}{3.352849in}}%
\pgfusepath{stroke}%
\end{pgfscope}%
\begin{pgfscope}%
\pgfsetbuttcap%
\pgfsetroundjoin%
\definecolor{currentfill}{rgb}{0.000000,0.000000,0.000000}%
\pgfsetfillcolor{currentfill}%
\pgfsetlinewidth{0.803000pt}%
\definecolor{currentstroke}{rgb}{0.000000,0.000000,0.000000}%
\pgfsetstrokecolor{currentstroke}%
\pgfsetdash{}{0pt}%
\pgfsys@defobject{currentmarker}{\pgfqpoint{-0.048611in}{0.000000in}}{\pgfqpoint{-0.000000in}{0.000000in}}{%
\pgfpathmoveto{\pgfqpoint{-0.000000in}{0.000000in}}%
\pgfpathlineto{\pgfqpoint{-0.048611in}{0.000000in}}%
\pgfusepath{stroke,fill}%
}%
\begin{pgfscope}%
\pgfsys@transformshift{0.800000in}{3.352849in}%
\pgfsys@useobject{currentmarker}{}%
\end{pgfscope}%
\end{pgfscope}%
\begin{pgfscope}%
\definecolor{textcolor}{rgb}{0.000000,0.000000,0.000000}%
\pgfsetstrokecolor{textcolor}%
\pgfsetfillcolor{textcolor}%
\pgftext[x=0.525308in, y=3.301749in, left, base]{\color{textcolor}\rmfamily\fontsize{10.000000}{12.000000}\selectfont \(\displaystyle {0.6}\)}%
\end{pgfscope}%
\begin{pgfscope}%
\pgfpathrectangle{\pgfqpoint{0.800000in}{0.528000in}}{\pgfqpoint{4.960000in}{3.696000in}}%
\pgfusepath{clip}%
\pgfsetrectcap%
\pgfsetroundjoin%
\pgfsetlinewidth{0.803000pt}%
\definecolor{currentstroke}{rgb}{0.690196,0.690196,0.690196}%
\pgfsetstrokecolor{currentstroke}%
\pgfsetdash{}{0pt}%
\pgfpathmoveto{\pgfqpoint{0.800000in}{3.823657in}}%
\pgfpathlineto{\pgfqpoint{5.760000in}{3.823657in}}%
\pgfusepath{stroke}%
\end{pgfscope}%
\begin{pgfscope}%
\pgfsetbuttcap%
\pgfsetroundjoin%
\definecolor{currentfill}{rgb}{0.000000,0.000000,0.000000}%
\pgfsetfillcolor{currentfill}%
\pgfsetlinewidth{0.803000pt}%
\definecolor{currentstroke}{rgb}{0.000000,0.000000,0.000000}%
\pgfsetstrokecolor{currentstroke}%
\pgfsetdash{}{0pt}%
\pgfsys@defobject{currentmarker}{\pgfqpoint{-0.048611in}{0.000000in}}{\pgfqpoint{-0.000000in}{0.000000in}}{%
\pgfpathmoveto{\pgfqpoint{-0.000000in}{0.000000in}}%
\pgfpathlineto{\pgfqpoint{-0.048611in}{0.000000in}}%
\pgfusepath{stroke,fill}%
}%
\begin{pgfscope}%
\pgfsys@transformshift{0.800000in}{3.823657in}%
\pgfsys@useobject{currentmarker}{}%
\end{pgfscope}%
\end{pgfscope}%
\begin{pgfscope}%
\definecolor{textcolor}{rgb}{0.000000,0.000000,0.000000}%
\pgfsetstrokecolor{textcolor}%
\pgfsetfillcolor{textcolor}%
\pgftext[x=0.525308in, y=3.772557in, left, base]{\color{textcolor}\rmfamily\fontsize{10.000000}{12.000000}\selectfont \(\displaystyle {0.7}\)}%
\end{pgfscope}%
\begin{pgfscope}%
\pgfpathrectangle{\pgfqpoint{0.800000in}{0.528000in}}{\pgfqpoint{4.960000in}{3.696000in}}%
\pgfusepath{clip}%
\pgfsetrectcap%
\pgfsetroundjoin%
\pgfsetlinewidth{1.505625pt}%
\definecolor{currentstroke}{rgb}{0.121569,0.466667,0.705882}%
\pgfsetstrokecolor{currentstroke}%
\pgfsetdash{}{0pt}%
\pgfpathmoveto{\pgfqpoint{1.056511in}{3.352849in}}%
\pgfpathlineto{\pgfqpoint{1.910836in}{3.352849in}}%
\pgfpathlineto{\pgfqpoint{1.911691in}{2.429193in}}%
\pgfpathlineto{\pgfqpoint{1.912546in}{2.429208in}}%
\pgfpathlineto{\pgfqpoint{1.920243in}{2.430729in}}%
\pgfpathlineto{\pgfqpoint{1.927939in}{2.434731in}}%
\pgfpathlineto{\pgfqpoint{1.935636in}{2.441196in}}%
\pgfpathlineto{\pgfqpoint{1.944188in}{2.451236in}}%
\pgfpathlineto{\pgfqpoint{1.953595in}{2.465689in}}%
\pgfpathlineto{\pgfqpoint{1.964712in}{2.487255in}}%
\pgfpathlineto{\pgfqpoint{1.976685in}{2.515711in}}%
\pgfpathlineto{\pgfqpoint{1.990368in}{2.554521in}}%
\pgfpathlineto{\pgfqpoint{2.005761in}{2.605607in}}%
\pgfpathlineto{\pgfqpoint{2.022864in}{2.670640in}}%
\pgfpathlineto{\pgfqpoint{2.042533in}{2.754614in}}%
\pgfpathlineto{\pgfqpoint{2.066479in}{2.867216in}}%
\pgfpathlineto{\pgfqpoint{2.099831in}{3.035966in}}%
\pgfpathlineto{\pgfqpoint{2.169955in}{3.393180in}}%
\pgfpathlineto{\pgfqpoint{2.198176in}{3.523646in}}%
\pgfpathlineto{\pgfqpoint{2.222121in}{3.624152in}}%
\pgfpathlineto{\pgfqpoint{2.243501in}{3.704864in}}%
\pgfpathlineto{\pgfqpoint{2.264025in}{3.773861in}}%
\pgfpathlineto{\pgfqpoint{2.282839in}{3.829685in}}%
\pgfpathlineto{\pgfqpoint{2.300798in}{3.876410in}}%
\pgfpathlineto{\pgfqpoint{2.317901in}{3.915104in}}%
\pgfpathlineto{\pgfqpoint{2.335005in}{3.948342in}}%
\pgfpathlineto{\pgfqpoint{2.351253in}{3.975085in}}%
\pgfpathlineto{\pgfqpoint{2.366647in}{3.996277in}}%
\pgfpathlineto{\pgfqpoint{2.381185in}{4.012756in}}%
\pgfpathlineto{\pgfqpoint{2.395723in}{4.025948in}}%
\pgfpathlineto{\pgfqpoint{2.409406in}{4.035472in}}%
\pgfpathlineto{\pgfqpoint{2.423088in}{4.042281in}}%
\pgfpathlineto{\pgfqpoint{2.435916in}{4.046256in}}%
\pgfpathlineto{\pgfqpoint{2.448744in}{4.047937in}}%
\pgfpathlineto{\pgfqpoint{2.461572in}{4.047341in}}%
\pgfpathlineto{\pgfqpoint{2.474399in}{4.044466in}}%
\pgfpathlineto{\pgfqpoint{2.487227in}{4.039293in}}%
\pgfpathlineto{\pgfqpoint{2.500055in}{4.031782in}}%
\pgfpathlineto{\pgfqpoint{2.513738in}{4.021132in}}%
\pgfpathlineto{\pgfqpoint{2.527420in}{4.007677in}}%
\pgfpathlineto{\pgfqpoint{2.541958in}{3.990194in}}%
\pgfpathlineto{\pgfqpoint{2.557352in}{3.967959in}}%
\pgfpathlineto{\pgfqpoint{2.572745in}{3.941732in}}%
\pgfpathlineto{\pgfqpoint{2.588993in}{3.909536in}}%
\pgfpathlineto{\pgfqpoint{2.606097in}{3.870449in}}%
\pgfpathlineto{\pgfqpoint{2.624056in}{3.823497in}}%
\pgfpathlineto{\pgfqpoint{2.642870in}{3.767692in}}%
\pgfpathlineto{\pgfqpoint{2.662539in}{3.702109in}}%
\pgfpathlineto{\pgfqpoint{2.683918in}{3.622650in}}%
\pgfpathlineto{\pgfqpoint{2.707008in}{3.527903in}}%
\pgfpathlineto{\pgfqpoint{2.733519in}{3.409208in}}%
\pgfpathlineto{\pgfqpoint{2.766871in}{3.248872in}}%
\pgfpathlineto{\pgfqpoint{2.848113in}{2.853698in}}%
\pgfpathlineto{\pgfqpoint{2.872058in}{2.750893in}}%
\pgfpathlineto{\pgfqpoint{2.891727in}{2.675828in}}%
\pgfpathlineto{\pgfqpoint{2.908831in}{2.618962in}}%
\pgfpathlineto{\pgfqpoint{2.924224in}{2.575395in}}%
\pgfpathlineto{\pgfqpoint{2.937907in}{2.543284in}}%
\pgfpathlineto{\pgfqpoint{2.949879in}{2.520614in}}%
\pgfpathlineto{\pgfqpoint{2.960997in}{2.504293in}}%
\pgfpathlineto{\pgfqpoint{2.970404in}{2.494139in}}%
\pgfpathlineto{\pgfqpoint{2.978955in}{2.487868in}}%
\pgfpathlineto{\pgfqpoint{2.986652in}{2.484659in}}%
\pgfpathlineto{\pgfqpoint{2.994349in}{2.483769in}}%
\pgfpathlineto{\pgfqpoint{3.002045in}{2.485198in}}%
\pgfpathlineto{\pgfqpoint{3.009742in}{2.488939in}}%
\pgfpathlineto{\pgfqpoint{3.018294in}{2.495788in}}%
\pgfpathlineto{\pgfqpoint{3.027701in}{2.506551in}}%
\pgfpathlineto{\pgfqpoint{3.037963in}{2.522075in}}%
\pgfpathlineto{\pgfqpoint{3.049080in}{2.543218in}}%
\pgfpathlineto{\pgfqpoint{3.061053in}{2.570818in}}%
\pgfpathlineto{\pgfqpoint{3.074735in}{2.608159in}}%
\pgfpathlineto{\pgfqpoint{3.090129in}{2.656999in}}%
\pgfpathlineto{\pgfqpoint{3.107232in}{2.718859in}}%
\pgfpathlineto{\pgfqpoint{3.127757in}{2.802047in}}%
\pgfpathlineto{\pgfqpoint{3.152557in}{2.912635in}}%
\pgfpathlineto{\pgfqpoint{3.187619in}{3.080306in}}%
\pgfpathlineto{\pgfqpoint{3.251758in}{3.387899in}}%
\pgfpathlineto{\pgfqpoint{3.279979in}{3.511753in}}%
\pgfpathlineto{\pgfqpoint{3.304779in}{3.610845in}}%
\pgfpathlineto{\pgfqpoint{3.327014in}{3.690725in}}%
\pgfpathlineto{\pgfqpoint{3.347538in}{3.756425in}}%
\pgfpathlineto{\pgfqpoint{3.367207in}{3.811985in}}%
\pgfpathlineto{\pgfqpoint{3.385166in}{3.856408in}}%
\pgfpathlineto{\pgfqpoint{3.403125in}{3.894939in}}%
\pgfpathlineto{\pgfqpoint{3.420228in}{3.926328in}}%
\pgfpathlineto{\pgfqpoint{3.436477in}{3.951530in}}%
\pgfpathlineto{\pgfqpoint{3.451870in}{3.971421in}}%
\pgfpathlineto{\pgfqpoint{3.466408in}{3.986781in}}%
\pgfpathlineto{\pgfqpoint{3.480946in}{3.998932in}}%
\pgfpathlineto{\pgfqpoint{3.494629in}{4.007526in}}%
\pgfpathlineto{\pgfqpoint{3.508312in}{4.013431in}}%
\pgfpathlineto{\pgfqpoint{3.521139in}{4.016571in}}%
\pgfpathlineto{\pgfqpoint{3.533967in}{4.017417in}}%
\pgfpathlineto{\pgfqpoint{3.546795in}{4.015978in}}%
\pgfpathlineto{\pgfqpoint{3.559622in}{4.012247in}}%
\pgfpathlineto{\pgfqpoint{3.572450in}{4.006202in}}%
\pgfpathlineto{\pgfqpoint{3.586133in}{3.997161in}}%
\pgfpathlineto{\pgfqpoint{3.599816in}{3.985383in}}%
\pgfpathlineto{\pgfqpoint{3.614354in}{3.969787in}}%
\pgfpathlineto{\pgfqpoint{3.628892in}{3.950912in}}%
\pgfpathlineto{\pgfqpoint{3.644285in}{3.927231in}}%
\pgfpathlineto{\pgfqpoint{3.660534in}{3.897965in}}%
\pgfpathlineto{\pgfqpoint{3.677637in}{3.862268in}}%
\pgfpathlineto{\pgfqpoint{3.695596in}{3.819239in}}%
\pgfpathlineto{\pgfqpoint{3.714410in}{3.767964in}}%
\pgfpathlineto{\pgfqpoint{3.734079in}{3.707573in}}%
\pgfpathlineto{\pgfqpoint{3.754603in}{3.637328in}}%
\pgfpathlineto{\pgfqpoint{3.777693in}{3.549991in}}%
\pgfpathlineto{\pgfqpoint{3.803349in}{3.443820in}}%
\pgfpathlineto{\pgfqpoint{3.834990in}{3.302659in}}%
\pgfpathlineto{\pgfqpoint{3.935046in}{2.848108in}}%
\pgfpathlineto{\pgfqpoint{3.957281in}{2.761725in}}%
\pgfpathlineto{\pgfqpoint{3.976095in}{2.697162in}}%
\pgfpathlineto{\pgfqpoint{3.992343in}{2.648857in}}%
\pgfpathlineto{\pgfqpoint{4.006881in}{2.612188in}}%
\pgfpathlineto{\pgfqpoint{4.019709in}{2.585380in}}%
\pgfpathlineto{\pgfqpoint{4.031682in}{2.565311in}}%
\pgfpathlineto{\pgfqpoint{4.041944in}{2.552058in}}%
\pgfpathlineto{\pgfqpoint{4.051351in}{2.543191in}}%
\pgfpathlineto{\pgfqpoint{4.059903in}{2.537896in}}%
\pgfpathlineto{\pgfqpoint{4.068454in}{2.535261in}}%
\pgfpathlineto{\pgfqpoint{4.076151in}{2.535171in}}%
\pgfpathlineto{\pgfqpoint{4.083848in}{2.537242in}}%
\pgfpathlineto{\pgfqpoint{4.092399in}{2.542065in}}%
\pgfpathlineto{\pgfqpoint{4.100951in}{2.549519in}}%
\pgfpathlineto{\pgfqpoint{4.110358in}{2.560708in}}%
\pgfpathlineto{\pgfqpoint{4.120620in}{2.576412in}}%
\pgfpathlineto{\pgfqpoint{4.132593in}{2.599196in}}%
\pgfpathlineto{\pgfqpoint{4.145421in}{2.628704in}}%
\pgfpathlineto{\pgfqpoint{4.159959in}{2.668122in}}%
\pgfpathlineto{\pgfqpoint{4.176207in}{2.719044in}}%
\pgfpathlineto{\pgfqpoint{4.195021in}{2.785965in}}%
\pgfpathlineto{\pgfqpoint{4.216400in}{2.870580in}}%
\pgfpathlineto{\pgfqpoint{4.243766in}{2.988595in}}%
\pgfpathlineto{\pgfqpoint{4.292511in}{3.211090in}}%
\pgfpathlineto{\pgfqpoint{4.335270in}{3.401687in}}%
\pgfpathlineto{\pgfqpoint{4.363491in}{3.517738in}}%
\pgfpathlineto{\pgfqpoint{4.388292in}{3.610573in}}%
\pgfpathlineto{\pgfqpoint{4.410526in}{3.685424in}}%
\pgfpathlineto{\pgfqpoint{4.431051in}{3.747007in}}%
\pgfpathlineto{\pgfqpoint{4.450720in}{3.799089in}}%
\pgfpathlineto{\pgfqpoint{4.469534in}{3.842560in}}%
\pgfpathlineto{\pgfqpoint{4.486637in}{3.876783in}}%
\pgfpathlineto{\pgfqpoint{4.503741in}{3.906094in}}%
\pgfpathlineto{\pgfqpoint{4.519989in}{3.929532in}}%
\pgfpathlineto{\pgfqpoint{4.535383in}{3.947906in}}%
\pgfpathlineto{\pgfqpoint{4.549921in}{3.961946in}}%
\pgfpathlineto{\pgfqpoint{4.564459in}{3.972859in}}%
\pgfpathlineto{\pgfqpoint{4.578142in}{3.980344in}}%
\pgfpathlineto{\pgfqpoint{4.590969in}{3.984954in}}%
\pgfpathlineto{\pgfqpoint{4.603797in}{3.987264in}}%
\pgfpathlineto{\pgfqpoint{4.616625in}{3.987292in}}%
\pgfpathlineto{\pgfqpoint{4.629452in}{3.985041in}}%
\pgfpathlineto{\pgfqpoint{4.642280in}{3.980500in}}%
\pgfpathlineto{\pgfqpoint{4.655108in}{3.973647in}}%
\pgfpathlineto{\pgfqpoint{4.668791in}{3.963750in}}%
\pgfpathlineto{\pgfqpoint{4.682473in}{3.951129in}}%
\pgfpathlineto{\pgfqpoint{4.697012in}{3.934662in}}%
\pgfpathlineto{\pgfqpoint{4.712405in}{3.913700in}}%
\pgfpathlineto{\pgfqpoint{4.728653in}{3.887525in}}%
\pgfpathlineto{\pgfqpoint{4.744902in}{3.857078in}}%
\pgfpathlineto{\pgfqpoint{4.762005in}{3.820304in}}%
\pgfpathlineto{\pgfqpoint{4.779964in}{3.776395in}}%
\pgfpathlineto{\pgfqpoint{4.798778in}{3.724558in}}%
\pgfpathlineto{\pgfqpoint{4.819302in}{3.661299in}}%
\pgfpathlineto{\pgfqpoint{4.841537in}{3.585216in}}%
\pgfpathlineto{\pgfqpoint{4.866337in}{3.491924in}}%
\pgfpathlineto{\pgfqpoint{4.895413in}{3.373190in}}%
\pgfpathlineto{\pgfqpoint{4.936462in}{3.194835in}}%
\pgfpathlineto{\pgfqpoint{4.991193in}{2.958165in}}%
\pgfpathlineto{\pgfqpoint{5.017704in}{2.853663in}}%
\pgfpathlineto{\pgfqpoint{5.039083in}{2.778174in}}%
\pgfpathlineto{\pgfqpoint{5.057042in}{2.722469in}}%
\pgfpathlineto{\pgfqpoint{5.073291in}{2.679112in}}%
\pgfpathlineto{\pgfqpoint{5.087829in}{2.646592in}}%
\pgfpathlineto{\pgfqpoint{5.100656in}{2.623177in}}%
\pgfpathlineto{\pgfqpoint{5.111774in}{2.607083in}}%
\pgfpathlineto{\pgfqpoint{5.122036in}{2.595801in}}%
\pgfpathlineto{\pgfqpoint{5.131443in}{2.588539in}}%
\pgfpathlineto{\pgfqpoint{5.139995in}{2.584528in}}%
\pgfpathlineto{\pgfqpoint{5.148546in}{2.583001in}}%
\pgfpathlineto{\pgfqpoint{5.156243in}{2.583755in}}%
\pgfpathlineto{\pgfqpoint{5.164795in}{2.586952in}}%
\pgfpathlineto{\pgfqpoint{5.173347in}{2.592615in}}%
\pgfpathlineto{\pgfqpoint{5.182754in}{2.601659in}}%
\pgfpathlineto{\pgfqpoint{5.193016in}{2.614823in}}%
\pgfpathlineto{\pgfqpoint{5.204133in}{2.632864in}}%
\pgfpathlineto{\pgfqpoint{5.216106in}{2.656521in}}%
\pgfpathlineto{\pgfqpoint{5.229789in}{2.688649in}}%
\pgfpathlineto{\pgfqpoint{5.245182in}{2.730816in}}%
\pgfpathlineto{\pgfqpoint{5.262285in}{2.784404in}}%
\pgfpathlineto{\pgfqpoint{5.281954in}{2.853565in}}%
\pgfpathlineto{\pgfqpoint{5.305900in}{2.946391in}}%
\pgfpathlineto{\pgfqpoint{5.338396in}{3.082306in}}%
\pgfpathlineto{\pgfqpoint{5.420494in}{3.429748in}}%
\pgfpathlineto{\pgfqpoint{5.448715in}{3.537160in}}%
\pgfpathlineto{\pgfqpoint{5.472660in}{3.619912in}}%
\pgfpathlineto{\pgfqpoint{5.494894in}{3.688942in}}%
\pgfpathlineto{\pgfqpoint{5.515419in}{3.745581in}}%
\pgfpathlineto{\pgfqpoint{5.535088in}{3.793337in}}%
\pgfpathlineto{\pgfqpoint{5.553046in}{3.831364in}}%
\pgfpathlineto{\pgfqpoint{5.570150in}{3.862699in}}%
\pgfpathlineto{\pgfqpoint{5.586399in}{3.888152in}}%
\pgfpathlineto{\pgfqpoint{5.602647in}{3.909508in}}%
\pgfpathlineto{\pgfqpoint{5.618040in}{3.926063in}}%
\pgfpathlineto{\pgfqpoint{5.632578in}{3.938496in}}%
\pgfpathlineto{\pgfqpoint{5.646261in}{3.947422in}}%
\pgfpathlineto{\pgfqpoint{5.659944in}{3.953705in}}%
\pgfpathlineto{\pgfqpoint{5.672772in}{3.957229in}}%
\pgfpathlineto{\pgfqpoint{5.685599in}{3.958483in}}%
\pgfpathlineto{\pgfqpoint{5.698427in}{3.957476in}}%
\pgfpathlineto{\pgfqpoint{5.711255in}{3.954207in}}%
\pgfpathlineto{\pgfqpoint{5.724082in}{3.948662in}}%
\pgfpathlineto{\pgfqpoint{5.737765in}{3.940215in}}%
\pgfpathlineto{\pgfqpoint{5.751448in}{3.929115in}}%
\pgfpathlineto{\pgfqpoint{5.760000in}{3.920809in}}%
\pgfpathlineto{\pgfqpoint{5.760000in}{3.920809in}}%
\pgfusepath{stroke}%
\end{pgfscope}%
\begin{pgfscope}%
\pgfsetrectcap%
\pgfsetmiterjoin%
\pgfsetlinewidth{0.803000pt}%
\definecolor{currentstroke}{rgb}{0.000000,0.000000,0.000000}%
\pgfsetstrokecolor{currentstroke}%
\pgfsetdash{}{0pt}%
\pgfpathmoveto{\pgfqpoint{0.800000in}{0.528000in}}%
\pgfpathlineto{\pgfqpoint{0.800000in}{4.224000in}}%
\pgfusepath{stroke}%
\end{pgfscope}%
\begin{pgfscope}%
\pgfsetrectcap%
\pgfsetmiterjoin%
\pgfsetlinewidth{0.803000pt}%
\definecolor{currentstroke}{rgb}{0.000000,0.000000,0.000000}%
\pgfsetstrokecolor{currentstroke}%
\pgfsetdash{}{0pt}%
\pgfpathmoveto{\pgfqpoint{5.760000in}{0.528000in}}%
\pgfpathlineto{\pgfqpoint{5.760000in}{4.224000in}}%
\pgfusepath{stroke}%
\end{pgfscope}%
\begin{pgfscope}%
\pgfsetrectcap%
\pgfsetmiterjoin%
\pgfsetlinewidth{0.803000pt}%
\definecolor{currentstroke}{rgb}{0.000000,0.000000,0.000000}%
\pgfsetstrokecolor{currentstroke}%
\pgfsetdash{}{0pt}%
\pgfpathmoveto{\pgfqpoint{0.800000in}{0.528000in}}%
\pgfpathlineto{\pgfqpoint{5.760000in}{0.528000in}}%
\pgfusepath{stroke}%
\end{pgfscope}%
\begin{pgfscope}%
\pgfsetrectcap%
\pgfsetmiterjoin%
\pgfsetlinewidth{0.803000pt}%
\definecolor{currentstroke}{rgb}{0.000000,0.000000,0.000000}%
\pgfsetstrokecolor{currentstroke}%
\pgfsetdash{}{0pt}%
\pgfpathmoveto{\pgfqpoint{0.800000in}{4.224000in}}%
\pgfpathlineto{\pgfqpoint{5.760000in}{4.224000in}}%
\pgfusepath{stroke}%
\end{pgfscope}%
\begin{pgfscope}%
\pgfsetbuttcap%
\pgfsetmiterjoin%
\definecolor{currentfill}{rgb}{1.000000,1.000000,1.000000}%
\pgfsetfillcolor{currentfill}%
\pgfsetfillopacity{0.800000}%
\pgfsetlinewidth{1.003750pt}%
\definecolor{currentstroke}{rgb}{0.800000,0.800000,0.800000}%
\pgfsetstrokecolor{currentstroke}%
\pgfsetstrokeopacity{0.800000}%
\pgfsetdash{}{0pt}%
\pgfpathmoveto{\pgfqpoint{0.897222in}{0.597444in}}%
\pgfpathlineto{\pgfqpoint{2.654872in}{0.597444in}}%
\pgfpathquadraticcurveto{\pgfqpoint{2.682650in}{0.597444in}}{\pgfqpoint{2.682650in}{0.625222in}}%
\pgfpathlineto{\pgfqpoint{2.682650in}{0.813224in}}%
\pgfpathquadraticcurveto{\pgfqpoint{2.682650in}{0.841002in}}{\pgfqpoint{2.654872in}{0.841002in}}%
\pgfpathlineto{\pgfqpoint{0.897222in}{0.841002in}}%
\pgfpathquadraticcurveto{\pgfqpoint{0.869444in}{0.841002in}}{\pgfqpoint{0.869444in}{0.813224in}}%
\pgfpathlineto{\pgfqpoint{0.869444in}{0.625222in}}%
\pgfpathquadraticcurveto{\pgfqpoint{0.869444in}{0.597444in}}{\pgfqpoint{0.897222in}{0.597444in}}%
\pgfpathlineto{\pgfqpoint{0.897222in}{0.597444in}}%
\pgfpathclose%
\pgfusepath{stroke,fill}%
\end{pgfscope}%
\begin{pgfscope}%
\pgfsetrectcap%
\pgfsetroundjoin%
\pgfsetlinewidth{1.505625pt}%
\definecolor{currentstroke}{rgb}{0.121569,0.466667,0.705882}%
\pgfsetstrokecolor{currentstroke}%
\pgfsetdash{}{0pt}%
\pgfpathmoveto{\pgfqpoint{0.925000in}{0.731857in}}%
\pgfpathlineto{\pgfqpoint{1.063889in}{0.731857in}}%
\pgfpathlineto{\pgfqpoint{1.202778in}{0.731857in}}%
\pgfusepath{stroke}%
\end{pgfscope}%
\begin{pgfscope}%
\definecolor{textcolor}{rgb}{0.000000,0.000000,0.000000}%
\pgfsetstrokecolor{textcolor}%
\pgfsetfillcolor{textcolor}%
\pgftext[x=1.313889in,y=0.683246in,left,base]{\color{textcolor}\rmfamily\fontsize{10.000000}{12.000000}\selectfont power stable scenario}%
\end{pgfscope}%
\end{pgfpicture}%
\makeatother%
\endgroup%


\section{Comparison algebraic vs. non-algebraic}
\label{app:alg-non-alg-comparison}

%% Creator: Matplotlib, PGF backend
%%
%% To include the figure in your LaTeX document, write
%%   \input{<filename>.pgf}
%%
%% Make sure the required packages are loaded in your preamble
%%   \usepackage{pgf}
%%
%% Also ensure that all the required font packages are loaded; for instance,
%% the lmodern package is sometimes necessary when using math font.
%%   \usepackage{lmodern}
%%
%% Figures using additional raster images can only be included by \input if
%% they are in the same directory as the main LaTeX file. For loading figures
%% from other directories you can use the `import` package
%%   \usepackage{import}
%%
%% and then include the figures with
%%   \import{<path to file>}{<filename>.pgf}
%%
%% Matplotlib used the following preamble
%%   
%%   \usepackage{fontspec}
%%   \setmainfont{Charter.ttc}[Path=\detokenize{/System/Library/Fonts/Supplemental/}]
%%   \setsansfont{DejaVuSans.ttf}[Path=\detokenize{/opt/homebrew/lib/python3.10/site-packages/matplotlib/mpl-data/fonts/ttf/}]
%%   \setmonofont{DejaVuSansMono.ttf}[Path=\detokenize{/opt/homebrew/lib/python3.10/site-packages/matplotlib/mpl-data/fonts/ttf/}]
%%   \makeatletter\@ifpackageloaded{underscore}{}{\usepackage[strings]{underscore}}\makeatother
%%
\begingroup%
\makeatletter%
\begin{pgfpicture}%
\pgfpathrectangle{\pgfpointorigin}{\pgfqpoint{6.400000in}{4.800000in}}%
\pgfusepath{use as bounding box, clip}%
\begin{pgfscope}%
\pgfsetbuttcap%
\pgfsetmiterjoin%
\definecolor{currentfill}{rgb}{1.000000,1.000000,1.000000}%
\pgfsetfillcolor{currentfill}%
\pgfsetlinewidth{0.000000pt}%
\definecolor{currentstroke}{rgb}{1.000000,1.000000,1.000000}%
\pgfsetstrokecolor{currentstroke}%
\pgfsetdash{}{0pt}%
\pgfpathmoveto{\pgfqpoint{0.000000in}{0.000000in}}%
\pgfpathlineto{\pgfqpoint{6.400000in}{0.000000in}}%
\pgfpathlineto{\pgfqpoint{6.400000in}{4.800000in}}%
\pgfpathlineto{\pgfqpoint{0.000000in}{4.800000in}}%
\pgfpathlineto{\pgfqpoint{0.000000in}{0.000000in}}%
\pgfpathclose%
\pgfusepath{fill}%
\end{pgfscope}%
\begin{pgfscope}%
\pgfsetbuttcap%
\pgfsetmiterjoin%
\definecolor{currentfill}{rgb}{1.000000,1.000000,1.000000}%
\pgfsetfillcolor{currentfill}%
\pgfsetlinewidth{0.000000pt}%
\definecolor{currentstroke}{rgb}{0.000000,0.000000,0.000000}%
\pgfsetstrokecolor{currentstroke}%
\pgfsetstrokeopacity{0.000000}%
\pgfsetdash{}{0pt}%
\pgfpathmoveto{\pgfqpoint{0.800000in}{0.528000in}}%
\pgfpathlineto{\pgfqpoint{5.760000in}{0.528000in}}%
\pgfpathlineto{\pgfqpoint{5.760000in}{4.224000in}}%
\pgfpathlineto{\pgfqpoint{0.800000in}{4.224000in}}%
\pgfpathlineto{\pgfqpoint{0.800000in}{0.528000in}}%
\pgfpathclose%
\pgfusepath{fill}%
\end{pgfscope}%
\begin{pgfscope}%
\pgfpathrectangle{\pgfqpoint{0.800000in}{0.528000in}}{\pgfqpoint{4.960000in}{3.696000in}}%
\pgfusepath{clip}%
\pgfsetrectcap%
\pgfsetroundjoin%
\pgfsetlinewidth{0.803000pt}%
\definecolor{currentstroke}{rgb}{0.690196,0.690196,0.690196}%
\pgfsetstrokecolor{currentstroke}%
\pgfsetdash{}{0pt}%
\pgfpathmoveto{\pgfqpoint{1.025455in}{0.528000in}}%
\pgfpathlineto{\pgfqpoint{1.025455in}{4.224000in}}%
\pgfusepath{stroke}%
\end{pgfscope}%
\begin{pgfscope}%
\pgfsetbuttcap%
\pgfsetroundjoin%
\definecolor{currentfill}{rgb}{0.000000,0.000000,0.000000}%
\pgfsetfillcolor{currentfill}%
\pgfsetlinewidth{0.803000pt}%
\definecolor{currentstroke}{rgb}{0.000000,0.000000,0.000000}%
\pgfsetstrokecolor{currentstroke}%
\pgfsetdash{}{0pt}%
\pgfsys@defobject{currentmarker}{\pgfqpoint{0.000000in}{-0.048611in}}{\pgfqpoint{0.000000in}{0.000000in}}{%
\pgfpathmoveto{\pgfqpoint{0.000000in}{0.000000in}}%
\pgfpathlineto{\pgfqpoint{0.000000in}{-0.048611in}}%
\pgfusepath{stroke,fill}%
}%
\begin{pgfscope}%
\pgfsys@transformshift{1.025455in}{0.528000in}%
\pgfsys@useobject{currentmarker}{}%
\end{pgfscope}%
\end{pgfscope}%
\begin{pgfscope}%
\definecolor{textcolor}{rgb}{0.000000,0.000000,0.000000}%
\pgfsetstrokecolor{textcolor}%
\pgfsetfillcolor{textcolor}%
\pgftext[x=1.025455in,y=0.430778in,,top]{\color{textcolor}\rmfamily\fontsize{10.000000}{12.000000}\selectfont \(\displaystyle {\ensuremath{-}1}\)}%
\end{pgfscope}%
\begin{pgfscope}%
\pgfpathrectangle{\pgfqpoint{0.800000in}{0.528000in}}{\pgfqpoint{4.960000in}{3.696000in}}%
\pgfusepath{clip}%
\pgfsetrectcap%
\pgfsetroundjoin%
\pgfsetlinewidth{0.803000pt}%
\definecolor{currentstroke}{rgb}{0.690196,0.690196,0.690196}%
\pgfsetstrokecolor{currentstroke}%
\pgfsetdash{}{0pt}%
\pgfpathmoveto{\pgfqpoint{1.777095in}{0.528000in}}%
\pgfpathlineto{\pgfqpoint{1.777095in}{4.224000in}}%
\pgfusepath{stroke}%
\end{pgfscope}%
\begin{pgfscope}%
\pgfsetbuttcap%
\pgfsetroundjoin%
\definecolor{currentfill}{rgb}{0.000000,0.000000,0.000000}%
\pgfsetfillcolor{currentfill}%
\pgfsetlinewidth{0.803000pt}%
\definecolor{currentstroke}{rgb}{0.000000,0.000000,0.000000}%
\pgfsetstrokecolor{currentstroke}%
\pgfsetdash{}{0pt}%
\pgfsys@defobject{currentmarker}{\pgfqpoint{0.000000in}{-0.048611in}}{\pgfqpoint{0.000000in}{0.000000in}}{%
\pgfpathmoveto{\pgfqpoint{0.000000in}{0.000000in}}%
\pgfpathlineto{\pgfqpoint{0.000000in}{-0.048611in}}%
\pgfusepath{stroke,fill}%
}%
\begin{pgfscope}%
\pgfsys@transformshift{1.777095in}{0.528000in}%
\pgfsys@useobject{currentmarker}{}%
\end{pgfscope}%
\end{pgfscope}%
\begin{pgfscope}%
\definecolor{textcolor}{rgb}{0.000000,0.000000,0.000000}%
\pgfsetstrokecolor{textcolor}%
\pgfsetfillcolor{textcolor}%
\pgftext[x=1.777095in,y=0.430778in,,top]{\color{textcolor}\rmfamily\fontsize{10.000000}{12.000000}\selectfont \(\displaystyle {0}\)}%
\end{pgfscope}%
\begin{pgfscope}%
\pgfpathrectangle{\pgfqpoint{0.800000in}{0.528000in}}{\pgfqpoint{4.960000in}{3.696000in}}%
\pgfusepath{clip}%
\pgfsetrectcap%
\pgfsetroundjoin%
\pgfsetlinewidth{0.803000pt}%
\definecolor{currentstroke}{rgb}{0.690196,0.690196,0.690196}%
\pgfsetstrokecolor{currentstroke}%
\pgfsetdash{}{0pt}%
\pgfpathmoveto{\pgfqpoint{2.528735in}{0.528000in}}%
\pgfpathlineto{\pgfqpoint{2.528735in}{4.224000in}}%
\pgfusepath{stroke}%
\end{pgfscope}%
\begin{pgfscope}%
\pgfsetbuttcap%
\pgfsetroundjoin%
\definecolor{currentfill}{rgb}{0.000000,0.000000,0.000000}%
\pgfsetfillcolor{currentfill}%
\pgfsetlinewidth{0.803000pt}%
\definecolor{currentstroke}{rgb}{0.000000,0.000000,0.000000}%
\pgfsetstrokecolor{currentstroke}%
\pgfsetdash{}{0pt}%
\pgfsys@defobject{currentmarker}{\pgfqpoint{0.000000in}{-0.048611in}}{\pgfqpoint{0.000000in}{0.000000in}}{%
\pgfpathmoveto{\pgfqpoint{0.000000in}{0.000000in}}%
\pgfpathlineto{\pgfqpoint{0.000000in}{-0.048611in}}%
\pgfusepath{stroke,fill}%
}%
\begin{pgfscope}%
\pgfsys@transformshift{2.528735in}{0.528000in}%
\pgfsys@useobject{currentmarker}{}%
\end{pgfscope}%
\end{pgfscope}%
\begin{pgfscope}%
\definecolor{textcolor}{rgb}{0.000000,0.000000,0.000000}%
\pgfsetstrokecolor{textcolor}%
\pgfsetfillcolor{textcolor}%
\pgftext[x=2.528735in,y=0.430778in,,top]{\color{textcolor}\rmfamily\fontsize{10.000000}{12.000000}\selectfont \(\displaystyle {1}\)}%
\end{pgfscope}%
\begin{pgfscope}%
\pgfpathrectangle{\pgfqpoint{0.800000in}{0.528000in}}{\pgfqpoint{4.960000in}{3.696000in}}%
\pgfusepath{clip}%
\pgfsetrectcap%
\pgfsetroundjoin%
\pgfsetlinewidth{0.803000pt}%
\definecolor{currentstroke}{rgb}{0.690196,0.690196,0.690196}%
\pgfsetstrokecolor{currentstroke}%
\pgfsetdash{}{0pt}%
\pgfpathmoveto{\pgfqpoint{3.280376in}{0.528000in}}%
\pgfpathlineto{\pgfqpoint{3.280376in}{4.224000in}}%
\pgfusepath{stroke}%
\end{pgfscope}%
\begin{pgfscope}%
\pgfsetbuttcap%
\pgfsetroundjoin%
\definecolor{currentfill}{rgb}{0.000000,0.000000,0.000000}%
\pgfsetfillcolor{currentfill}%
\pgfsetlinewidth{0.803000pt}%
\definecolor{currentstroke}{rgb}{0.000000,0.000000,0.000000}%
\pgfsetstrokecolor{currentstroke}%
\pgfsetdash{}{0pt}%
\pgfsys@defobject{currentmarker}{\pgfqpoint{0.000000in}{-0.048611in}}{\pgfqpoint{0.000000in}{0.000000in}}{%
\pgfpathmoveto{\pgfqpoint{0.000000in}{0.000000in}}%
\pgfpathlineto{\pgfqpoint{0.000000in}{-0.048611in}}%
\pgfusepath{stroke,fill}%
}%
\begin{pgfscope}%
\pgfsys@transformshift{3.280376in}{0.528000in}%
\pgfsys@useobject{currentmarker}{}%
\end{pgfscope}%
\end{pgfscope}%
\begin{pgfscope}%
\definecolor{textcolor}{rgb}{0.000000,0.000000,0.000000}%
\pgfsetstrokecolor{textcolor}%
\pgfsetfillcolor{textcolor}%
\pgftext[x=3.280376in,y=0.430778in,,top]{\color{textcolor}\rmfamily\fontsize{10.000000}{12.000000}\selectfont \(\displaystyle {2}\)}%
\end{pgfscope}%
\begin{pgfscope}%
\pgfpathrectangle{\pgfqpoint{0.800000in}{0.528000in}}{\pgfqpoint{4.960000in}{3.696000in}}%
\pgfusepath{clip}%
\pgfsetrectcap%
\pgfsetroundjoin%
\pgfsetlinewidth{0.803000pt}%
\definecolor{currentstroke}{rgb}{0.690196,0.690196,0.690196}%
\pgfsetstrokecolor{currentstroke}%
\pgfsetdash{}{0pt}%
\pgfpathmoveto{\pgfqpoint{4.032016in}{0.528000in}}%
\pgfpathlineto{\pgfqpoint{4.032016in}{4.224000in}}%
\pgfusepath{stroke}%
\end{pgfscope}%
\begin{pgfscope}%
\pgfsetbuttcap%
\pgfsetroundjoin%
\definecolor{currentfill}{rgb}{0.000000,0.000000,0.000000}%
\pgfsetfillcolor{currentfill}%
\pgfsetlinewidth{0.803000pt}%
\definecolor{currentstroke}{rgb}{0.000000,0.000000,0.000000}%
\pgfsetstrokecolor{currentstroke}%
\pgfsetdash{}{0pt}%
\pgfsys@defobject{currentmarker}{\pgfqpoint{0.000000in}{-0.048611in}}{\pgfqpoint{0.000000in}{0.000000in}}{%
\pgfpathmoveto{\pgfqpoint{0.000000in}{0.000000in}}%
\pgfpathlineto{\pgfqpoint{0.000000in}{-0.048611in}}%
\pgfusepath{stroke,fill}%
}%
\begin{pgfscope}%
\pgfsys@transformshift{4.032016in}{0.528000in}%
\pgfsys@useobject{currentmarker}{}%
\end{pgfscope}%
\end{pgfscope}%
\begin{pgfscope}%
\definecolor{textcolor}{rgb}{0.000000,0.000000,0.000000}%
\pgfsetstrokecolor{textcolor}%
\pgfsetfillcolor{textcolor}%
\pgftext[x=4.032016in,y=0.430778in,,top]{\color{textcolor}\rmfamily\fontsize{10.000000}{12.000000}\selectfont \(\displaystyle {3}\)}%
\end{pgfscope}%
\begin{pgfscope}%
\pgfpathrectangle{\pgfqpoint{0.800000in}{0.528000in}}{\pgfqpoint{4.960000in}{3.696000in}}%
\pgfusepath{clip}%
\pgfsetrectcap%
\pgfsetroundjoin%
\pgfsetlinewidth{0.803000pt}%
\definecolor{currentstroke}{rgb}{0.690196,0.690196,0.690196}%
\pgfsetstrokecolor{currentstroke}%
\pgfsetdash{}{0pt}%
\pgfpathmoveto{\pgfqpoint{4.783657in}{0.528000in}}%
\pgfpathlineto{\pgfqpoint{4.783657in}{4.224000in}}%
\pgfusepath{stroke}%
\end{pgfscope}%
\begin{pgfscope}%
\pgfsetbuttcap%
\pgfsetroundjoin%
\definecolor{currentfill}{rgb}{0.000000,0.000000,0.000000}%
\pgfsetfillcolor{currentfill}%
\pgfsetlinewidth{0.803000pt}%
\definecolor{currentstroke}{rgb}{0.000000,0.000000,0.000000}%
\pgfsetstrokecolor{currentstroke}%
\pgfsetdash{}{0pt}%
\pgfsys@defobject{currentmarker}{\pgfqpoint{0.000000in}{-0.048611in}}{\pgfqpoint{0.000000in}{0.000000in}}{%
\pgfpathmoveto{\pgfqpoint{0.000000in}{0.000000in}}%
\pgfpathlineto{\pgfqpoint{0.000000in}{-0.048611in}}%
\pgfusepath{stroke,fill}%
}%
\begin{pgfscope}%
\pgfsys@transformshift{4.783657in}{0.528000in}%
\pgfsys@useobject{currentmarker}{}%
\end{pgfscope}%
\end{pgfscope}%
\begin{pgfscope}%
\definecolor{textcolor}{rgb}{0.000000,0.000000,0.000000}%
\pgfsetstrokecolor{textcolor}%
\pgfsetfillcolor{textcolor}%
\pgftext[x=4.783657in,y=0.430778in,,top]{\color{textcolor}\rmfamily\fontsize{10.000000}{12.000000}\selectfont \(\displaystyle {4}\)}%
\end{pgfscope}%
\begin{pgfscope}%
\pgfpathrectangle{\pgfqpoint{0.800000in}{0.528000in}}{\pgfqpoint{4.960000in}{3.696000in}}%
\pgfusepath{clip}%
\pgfsetrectcap%
\pgfsetroundjoin%
\pgfsetlinewidth{0.803000pt}%
\definecolor{currentstroke}{rgb}{0.690196,0.690196,0.690196}%
\pgfsetstrokecolor{currentstroke}%
\pgfsetdash{}{0pt}%
\pgfpathmoveto{\pgfqpoint{5.535297in}{0.528000in}}%
\pgfpathlineto{\pgfqpoint{5.535297in}{4.224000in}}%
\pgfusepath{stroke}%
\end{pgfscope}%
\begin{pgfscope}%
\pgfsetbuttcap%
\pgfsetroundjoin%
\definecolor{currentfill}{rgb}{0.000000,0.000000,0.000000}%
\pgfsetfillcolor{currentfill}%
\pgfsetlinewidth{0.803000pt}%
\definecolor{currentstroke}{rgb}{0.000000,0.000000,0.000000}%
\pgfsetstrokecolor{currentstroke}%
\pgfsetdash{}{0pt}%
\pgfsys@defobject{currentmarker}{\pgfqpoint{0.000000in}{-0.048611in}}{\pgfqpoint{0.000000in}{0.000000in}}{%
\pgfpathmoveto{\pgfqpoint{0.000000in}{0.000000in}}%
\pgfpathlineto{\pgfqpoint{0.000000in}{-0.048611in}}%
\pgfusepath{stroke,fill}%
}%
\begin{pgfscope}%
\pgfsys@transformshift{5.535297in}{0.528000in}%
\pgfsys@useobject{currentmarker}{}%
\end{pgfscope}%
\end{pgfscope}%
\begin{pgfscope}%
\definecolor{textcolor}{rgb}{0.000000,0.000000,0.000000}%
\pgfsetstrokecolor{textcolor}%
\pgfsetfillcolor{textcolor}%
\pgftext[x=5.535297in,y=0.430778in,,top]{\color{textcolor}\rmfamily\fontsize{10.000000}{12.000000}\selectfont \(\displaystyle {5}\)}%
\end{pgfscope}%
\begin{pgfscope}%
\definecolor{textcolor}{rgb}{0.000000,0.000000,0.000000}%
\pgfsetstrokecolor{textcolor}%
\pgfsetfillcolor{textcolor}%
\pgftext[x=3.280000in,y=0.242776in,,top]{\color{textcolor}\rmfamily\fontsize{10.000000}{12.000000}\selectfont time in s}%
\end{pgfscope}%
\begin{pgfscope}%
\pgfpathrectangle{\pgfqpoint{0.800000in}{0.528000in}}{\pgfqpoint{4.960000in}{3.696000in}}%
\pgfusepath{clip}%
\pgfsetrectcap%
\pgfsetroundjoin%
\pgfsetlinewidth{0.803000pt}%
\definecolor{currentstroke}{rgb}{0.690196,0.690196,0.690196}%
\pgfsetstrokecolor{currentstroke}%
\pgfsetdash{}{0pt}%
\pgfpathmoveto{\pgfqpoint{0.800000in}{1.070775in}}%
\pgfpathlineto{\pgfqpoint{5.760000in}{1.070775in}}%
\pgfusepath{stroke}%
\end{pgfscope}%
\begin{pgfscope}%
\pgfsetbuttcap%
\pgfsetroundjoin%
\definecolor{currentfill}{rgb}{0.000000,0.000000,0.000000}%
\pgfsetfillcolor{currentfill}%
\pgfsetlinewidth{0.803000pt}%
\definecolor{currentstroke}{rgb}{0.000000,0.000000,0.000000}%
\pgfsetstrokecolor{currentstroke}%
\pgfsetdash{}{0pt}%
\pgfsys@defobject{currentmarker}{\pgfqpoint{-0.048611in}{0.000000in}}{\pgfqpoint{-0.000000in}{0.000000in}}{%
\pgfpathmoveto{\pgfqpoint{-0.000000in}{0.000000in}}%
\pgfpathlineto{\pgfqpoint{-0.048611in}{0.000000in}}%
\pgfusepath{stroke,fill}%
}%
\begin{pgfscope}%
\pgfsys@transformshift{0.800000in}{1.070775in}%
\pgfsys@useobject{currentmarker}{}%
\end{pgfscope}%
\end{pgfscope}%
\begin{pgfscope}%
\definecolor{textcolor}{rgb}{0.000000,0.000000,0.000000}%
\pgfsetstrokecolor{textcolor}%
\pgfsetfillcolor{textcolor}%
\pgftext[x=0.563888in, y=1.019675in, left, base]{\color{textcolor}\rmfamily\fontsize{10.000000}{12.000000}\selectfont \(\displaystyle {20}\)}%
\end{pgfscope}%
\begin{pgfscope}%
\pgfpathrectangle{\pgfqpoint{0.800000in}{0.528000in}}{\pgfqpoint{4.960000in}{3.696000in}}%
\pgfusepath{clip}%
\pgfsetrectcap%
\pgfsetroundjoin%
\pgfsetlinewidth{0.803000pt}%
\definecolor{currentstroke}{rgb}{0.690196,0.690196,0.690196}%
\pgfsetstrokecolor{currentstroke}%
\pgfsetdash{}{0pt}%
\pgfpathmoveto{\pgfqpoint{0.800000in}{1.677284in}}%
\pgfpathlineto{\pgfqpoint{5.760000in}{1.677284in}}%
\pgfusepath{stroke}%
\end{pgfscope}%
\begin{pgfscope}%
\pgfsetbuttcap%
\pgfsetroundjoin%
\definecolor{currentfill}{rgb}{0.000000,0.000000,0.000000}%
\pgfsetfillcolor{currentfill}%
\pgfsetlinewidth{0.803000pt}%
\definecolor{currentstroke}{rgb}{0.000000,0.000000,0.000000}%
\pgfsetstrokecolor{currentstroke}%
\pgfsetdash{}{0pt}%
\pgfsys@defobject{currentmarker}{\pgfqpoint{-0.048611in}{0.000000in}}{\pgfqpoint{-0.000000in}{0.000000in}}{%
\pgfpathmoveto{\pgfqpoint{-0.000000in}{0.000000in}}%
\pgfpathlineto{\pgfqpoint{-0.048611in}{0.000000in}}%
\pgfusepath{stroke,fill}%
}%
\begin{pgfscope}%
\pgfsys@transformshift{0.800000in}{1.677284in}%
\pgfsys@useobject{currentmarker}{}%
\end{pgfscope}%
\end{pgfscope}%
\begin{pgfscope}%
\definecolor{textcolor}{rgb}{0.000000,0.000000,0.000000}%
\pgfsetstrokecolor{textcolor}%
\pgfsetfillcolor{textcolor}%
\pgftext[x=0.563888in, y=1.626184in, left, base]{\color{textcolor}\rmfamily\fontsize{10.000000}{12.000000}\selectfont \(\displaystyle {40}\)}%
\end{pgfscope}%
\begin{pgfscope}%
\pgfpathrectangle{\pgfqpoint{0.800000in}{0.528000in}}{\pgfqpoint{4.960000in}{3.696000in}}%
\pgfusepath{clip}%
\pgfsetrectcap%
\pgfsetroundjoin%
\pgfsetlinewidth{0.803000pt}%
\definecolor{currentstroke}{rgb}{0.690196,0.690196,0.690196}%
\pgfsetstrokecolor{currentstroke}%
\pgfsetdash{}{0pt}%
\pgfpathmoveto{\pgfqpoint{0.800000in}{2.283792in}}%
\pgfpathlineto{\pgfqpoint{5.760000in}{2.283792in}}%
\pgfusepath{stroke}%
\end{pgfscope}%
\begin{pgfscope}%
\pgfsetbuttcap%
\pgfsetroundjoin%
\definecolor{currentfill}{rgb}{0.000000,0.000000,0.000000}%
\pgfsetfillcolor{currentfill}%
\pgfsetlinewidth{0.803000pt}%
\definecolor{currentstroke}{rgb}{0.000000,0.000000,0.000000}%
\pgfsetstrokecolor{currentstroke}%
\pgfsetdash{}{0pt}%
\pgfsys@defobject{currentmarker}{\pgfqpoint{-0.048611in}{0.000000in}}{\pgfqpoint{-0.000000in}{0.000000in}}{%
\pgfpathmoveto{\pgfqpoint{-0.000000in}{0.000000in}}%
\pgfpathlineto{\pgfqpoint{-0.048611in}{0.000000in}}%
\pgfusepath{stroke,fill}%
}%
\begin{pgfscope}%
\pgfsys@transformshift{0.800000in}{2.283792in}%
\pgfsys@useobject{currentmarker}{}%
\end{pgfscope}%
\end{pgfscope}%
\begin{pgfscope}%
\definecolor{textcolor}{rgb}{0.000000,0.000000,0.000000}%
\pgfsetstrokecolor{textcolor}%
\pgfsetfillcolor{textcolor}%
\pgftext[x=0.563888in, y=2.232692in, left, base]{\color{textcolor}\rmfamily\fontsize{10.000000}{12.000000}\selectfont \(\displaystyle {60}\)}%
\end{pgfscope}%
\begin{pgfscope}%
\pgfpathrectangle{\pgfqpoint{0.800000in}{0.528000in}}{\pgfqpoint{4.960000in}{3.696000in}}%
\pgfusepath{clip}%
\pgfsetrectcap%
\pgfsetroundjoin%
\pgfsetlinewidth{0.803000pt}%
\definecolor{currentstroke}{rgb}{0.690196,0.690196,0.690196}%
\pgfsetstrokecolor{currentstroke}%
\pgfsetdash{}{0pt}%
\pgfpathmoveto{\pgfqpoint{0.800000in}{2.890301in}}%
\pgfpathlineto{\pgfqpoint{5.760000in}{2.890301in}}%
\pgfusepath{stroke}%
\end{pgfscope}%
\begin{pgfscope}%
\pgfsetbuttcap%
\pgfsetroundjoin%
\definecolor{currentfill}{rgb}{0.000000,0.000000,0.000000}%
\pgfsetfillcolor{currentfill}%
\pgfsetlinewidth{0.803000pt}%
\definecolor{currentstroke}{rgb}{0.000000,0.000000,0.000000}%
\pgfsetstrokecolor{currentstroke}%
\pgfsetdash{}{0pt}%
\pgfsys@defobject{currentmarker}{\pgfqpoint{-0.048611in}{0.000000in}}{\pgfqpoint{-0.000000in}{0.000000in}}{%
\pgfpathmoveto{\pgfqpoint{-0.000000in}{0.000000in}}%
\pgfpathlineto{\pgfqpoint{-0.048611in}{0.000000in}}%
\pgfusepath{stroke,fill}%
}%
\begin{pgfscope}%
\pgfsys@transformshift{0.800000in}{2.890301in}%
\pgfsys@useobject{currentmarker}{}%
\end{pgfscope}%
\end{pgfscope}%
\begin{pgfscope}%
\definecolor{textcolor}{rgb}{0.000000,0.000000,0.000000}%
\pgfsetstrokecolor{textcolor}%
\pgfsetfillcolor{textcolor}%
\pgftext[x=0.563888in, y=2.839201in, left, base]{\color{textcolor}\rmfamily\fontsize{10.000000}{12.000000}\selectfont \(\displaystyle {80}\)}%
\end{pgfscope}%
\begin{pgfscope}%
\pgfpathrectangle{\pgfqpoint{0.800000in}{0.528000in}}{\pgfqpoint{4.960000in}{3.696000in}}%
\pgfusepath{clip}%
\pgfsetrectcap%
\pgfsetroundjoin%
\pgfsetlinewidth{0.803000pt}%
\definecolor{currentstroke}{rgb}{0.690196,0.690196,0.690196}%
\pgfsetstrokecolor{currentstroke}%
\pgfsetdash{}{0pt}%
\pgfpathmoveto{\pgfqpoint{0.800000in}{3.496810in}}%
\pgfpathlineto{\pgfqpoint{5.760000in}{3.496810in}}%
\pgfusepath{stroke}%
\end{pgfscope}%
\begin{pgfscope}%
\pgfsetbuttcap%
\pgfsetroundjoin%
\definecolor{currentfill}{rgb}{0.000000,0.000000,0.000000}%
\pgfsetfillcolor{currentfill}%
\pgfsetlinewidth{0.803000pt}%
\definecolor{currentstroke}{rgb}{0.000000,0.000000,0.000000}%
\pgfsetstrokecolor{currentstroke}%
\pgfsetdash{}{0pt}%
\pgfsys@defobject{currentmarker}{\pgfqpoint{-0.048611in}{0.000000in}}{\pgfqpoint{-0.000000in}{0.000000in}}{%
\pgfpathmoveto{\pgfqpoint{-0.000000in}{0.000000in}}%
\pgfpathlineto{\pgfqpoint{-0.048611in}{0.000000in}}%
\pgfusepath{stroke,fill}%
}%
\begin{pgfscope}%
\pgfsys@transformshift{0.800000in}{3.496810in}%
\pgfsys@useobject{currentmarker}{}%
\end{pgfscope}%
\end{pgfscope}%
\begin{pgfscope}%
\definecolor{textcolor}{rgb}{0.000000,0.000000,0.000000}%
\pgfsetstrokecolor{textcolor}%
\pgfsetfillcolor{textcolor}%
\pgftext[x=0.494444in, y=3.445710in, left, base]{\color{textcolor}\rmfamily\fontsize{10.000000}{12.000000}\selectfont \(\displaystyle {100}\)}%
\end{pgfscope}%
\begin{pgfscope}%
\pgfpathrectangle{\pgfqpoint{0.800000in}{0.528000in}}{\pgfqpoint{4.960000in}{3.696000in}}%
\pgfusepath{clip}%
\pgfsetrectcap%
\pgfsetroundjoin%
\pgfsetlinewidth{0.803000pt}%
\definecolor{currentstroke}{rgb}{0.690196,0.690196,0.690196}%
\pgfsetstrokecolor{currentstroke}%
\pgfsetdash{}{0pt}%
\pgfpathmoveto{\pgfqpoint{0.800000in}{4.103319in}}%
\pgfpathlineto{\pgfqpoint{5.760000in}{4.103319in}}%
\pgfusepath{stroke}%
\end{pgfscope}%
\begin{pgfscope}%
\pgfsetbuttcap%
\pgfsetroundjoin%
\definecolor{currentfill}{rgb}{0.000000,0.000000,0.000000}%
\pgfsetfillcolor{currentfill}%
\pgfsetlinewidth{0.803000pt}%
\definecolor{currentstroke}{rgb}{0.000000,0.000000,0.000000}%
\pgfsetstrokecolor{currentstroke}%
\pgfsetdash{}{0pt}%
\pgfsys@defobject{currentmarker}{\pgfqpoint{-0.048611in}{0.000000in}}{\pgfqpoint{-0.000000in}{0.000000in}}{%
\pgfpathmoveto{\pgfqpoint{-0.000000in}{0.000000in}}%
\pgfpathlineto{\pgfqpoint{-0.048611in}{0.000000in}}%
\pgfusepath{stroke,fill}%
}%
\begin{pgfscope}%
\pgfsys@transformshift{0.800000in}{4.103319in}%
\pgfsys@useobject{currentmarker}{}%
\end{pgfscope}%
\end{pgfscope}%
\begin{pgfscope}%
\definecolor{textcolor}{rgb}{0.000000,0.000000,0.000000}%
\pgfsetstrokecolor{textcolor}%
\pgfsetfillcolor{textcolor}%
\pgftext[x=0.494444in, y=4.052219in, left, base]{\color{textcolor}\rmfamily\fontsize{10.000000}{12.000000}\selectfont \(\displaystyle {120}\)}%
\end{pgfscope}%
\begin{pgfscope}%
\definecolor{textcolor}{rgb}{0.000000,0.000000,0.000000}%
\pgfsetstrokecolor{textcolor}%
\pgfsetfillcolor{textcolor}%
\pgftext[x=0.438888in,y=2.376000in,,bottom,rotate=90.000000]{\color{textcolor}\rmfamily\fontsize{10.000000}{12.000000}\selectfont power angle \(\displaystyle \delta\) in deg}%
\end{pgfscope}%
\begin{pgfscope}%
\pgfpathrectangle{\pgfqpoint{0.800000in}{0.528000in}}{\pgfqpoint{4.960000in}{3.696000in}}%
\pgfusepath{clip}%
\pgfsetrectcap%
\pgfsetroundjoin%
\pgfsetlinewidth{1.505625pt}%
\definecolor{currentstroke}{rgb}{0.121569,0.466667,0.705882}%
\pgfsetstrokecolor{currentstroke}%
\pgfsetdash{}{0pt}%
\pgfpathmoveto{\pgfqpoint{1.025455in}{1.938082in}}%
\pgfpathlineto{\pgfqpoint{1.780853in}{1.938992in}}%
\pgfpathlineto{\pgfqpoint{1.785363in}{1.942564in}}%
\pgfpathlineto{\pgfqpoint{1.790624in}{1.950113in}}%
\pgfpathlineto{\pgfqpoint{1.796638in}{1.963195in}}%
\pgfpathlineto{\pgfqpoint{1.803402in}{1.983586in}}%
\pgfpathlineto{\pgfqpoint{1.810919in}{2.013280in}}%
\pgfpathlineto{\pgfqpoint{1.819938in}{2.058677in}}%
\pgfpathlineto{\pgfqpoint{1.829710in}{2.119859in}}%
\pgfpathlineto{\pgfqpoint{1.840984in}{2.205936in}}%
\pgfpathlineto{\pgfqpoint{1.853011in}{2.316000in}}%
\pgfpathlineto{\pgfqpoint{1.867292in}{2.469493in}}%
\pgfpathlineto{\pgfqpoint{1.894351in}{2.756389in}}%
\pgfpathlineto{\pgfqpoint{1.917652in}{2.980255in}}%
\pgfpathlineto{\pgfqpoint{1.938698in}{3.161958in}}%
\pgfpathlineto{\pgfqpoint{1.958992in}{3.318208in}}%
\pgfpathlineto{\pgfqpoint{1.977783in}{3.446465in}}%
\pgfpathlineto{\pgfqpoint{1.995822in}{3.555245in}}%
\pgfpathlineto{\pgfqpoint{2.013862in}{3.650645in}}%
\pgfpathlineto{\pgfqpoint{2.031149in}{3.730225in}}%
\pgfpathlineto{\pgfqpoint{2.047686in}{3.796162in}}%
\pgfpathlineto{\pgfqpoint{2.063470in}{3.850385in}}%
\pgfpathlineto{\pgfqpoint{2.078503in}{3.894584in}}%
\pgfpathlineto{\pgfqpoint{2.093536in}{3.931932in}}%
\pgfpathlineto{\pgfqpoint{2.107817in}{3.961395in}}%
\pgfpathlineto{\pgfqpoint{2.121346in}{3.984150in}}%
\pgfpathlineto{\pgfqpoint{2.133373in}{4.000325in}}%
\pgfpathlineto{\pgfqpoint{2.145399in}{4.012805in}}%
\pgfpathlineto{\pgfqpoint{2.156673in}{4.021230in}}%
\pgfpathlineto{\pgfqpoint{2.167196in}{4.026284in}}%
\pgfpathlineto{\pgfqpoint{2.176968in}{4.028578in}}%
\pgfpathlineto{\pgfqpoint{2.186739in}{4.028575in}}%
\pgfpathlineto{\pgfqpoint{2.196510in}{4.026281in}}%
\pgfpathlineto{\pgfqpoint{2.206282in}{4.021688in}}%
\pgfpathlineto{\pgfqpoint{2.216805in}{4.014152in}}%
\pgfpathlineto{\pgfqpoint{2.227328in}{4.003899in}}%
\pgfpathlineto{\pgfqpoint{2.238602in}{3.989844in}}%
\pgfpathlineto{\pgfqpoint{2.250628in}{3.971271in}}%
\pgfpathlineto{\pgfqpoint{2.263406in}{3.947363in}}%
\pgfpathlineto{\pgfqpoint{2.276936in}{3.917190in}}%
\pgfpathlineto{\pgfqpoint{2.291217in}{3.879682in}}%
\pgfpathlineto{\pgfqpoint{2.306250in}{3.833607in}}%
\pgfpathlineto{\pgfqpoint{2.322034in}{3.777549in}}%
\pgfpathlineto{\pgfqpoint{2.337819in}{3.713171in}}%
\pgfpathlineto{\pgfqpoint{2.354355in}{3.636278in}}%
\pgfpathlineto{\pgfqpoint{2.371643in}{3.544968in}}%
\pgfpathlineto{\pgfqpoint{2.389682in}{3.437162in}}%
\pgfpathlineto{\pgfqpoint{2.408473in}{3.310675in}}%
\pgfpathlineto{\pgfqpoint{2.428016in}{3.163344in}}%
\pgfpathlineto{\pgfqpoint{2.448310in}{2.993247in}}%
\pgfpathlineto{\pgfqpoint{2.470859in}{2.784513in}}%
\pgfpathlineto{\pgfqpoint{2.495663in}{2.533282in}}%
\pgfpathlineto{\pgfqpoint{2.526480in}{2.196998in}}%
\pgfpathlineto{\pgfqpoint{2.600893in}{1.374416in}}%
\pgfpathlineto{\pgfqpoint{2.621939in}{1.172321in}}%
\pgfpathlineto{\pgfqpoint{2.639227in}{1.027066in}}%
\pgfpathlineto{\pgfqpoint{2.653508in}{0.924295in}}%
\pgfpathlineto{\pgfqpoint{2.666286in}{0.847305in}}%
\pgfpathlineto{\pgfqpoint{2.677560in}{0.792114in}}%
\pgfpathlineto{\pgfqpoint{2.687332in}{0.754491in}}%
\pgfpathlineto{\pgfqpoint{2.695600in}{0.730333in}}%
\pgfpathlineto{\pgfqpoint{2.703116in}{0.714617in}}%
\pgfpathlineto{\pgfqpoint{2.709129in}{0.706385in}}%
\pgfpathlineto{\pgfqpoint{2.714391in}{0.702370in}}%
\pgfpathlineto{\pgfqpoint{2.718900in}{0.701305in}}%
\pgfpathlineto{\pgfqpoint{2.723410in}{0.702433in}}%
\pgfpathlineto{\pgfqpoint{2.727920in}{0.705750in}}%
\pgfpathlineto{\pgfqpoint{2.733182in}{0.712375in}}%
\pgfpathlineto{\pgfqpoint{2.739195in}{0.723554in}}%
\pgfpathlineto{\pgfqpoint{2.746711in}{0.742870in}}%
\pgfpathlineto{\pgfqpoint{2.754979in}{0.770841in}}%
\pgfpathlineto{\pgfqpoint{2.764750in}{0.812709in}}%
\pgfpathlineto{\pgfqpoint{2.775273in}{0.868015in}}%
\pgfpathlineto{\pgfqpoint{2.787300in}{0.943411in}}%
\pgfpathlineto{\pgfqpoint{2.801581in}{1.048363in}}%
\pgfpathlineto{\pgfqpoint{2.818117in}{1.188157in}}%
\pgfpathlineto{\pgfqpoint{2.837660in}{1.373857in}}%
\pgfpathlineto{\pgfqpoint{2.863215in}{1.639959in}}%
\pgfpathlineto{\pgfqpoint{2.953412in}{2.599543in}}%
\pgfpathlineto{\pgfqpoint{2.977465in}{2.821418in}}%
\pgfpathlineto{\pgfqpoint{2.998511in}{2.996073in}}%
\pgfpathlineto{\pgfqpoint{3.018053in}{3.140798in}}%
\pgfpathlineto{\pgfqpoint{3.036844in}{3.263752in}}%
\pgfpathlineto{\pgfqpoint{3.054132in}{3.362886in}}%
\pgfpathlineto{\pgfqpoint{3.070668in}{3.445397in}}%
\pgfpathlineto{\pgfqpoint{3.086453in}{3.513210in}}%
\pgfpathlineto{\pgfqpoint{3.101485in}{3.568138in}}%
\pgfpathlineto{\pgfqpoint{3.115015in}{3.609747in}}%
\pgfpathlineto{\pgfqpoint{3.127793in}{3.642418in}}%
\pgfpathlineto{\pgfqpoint{3.139819in}{3.667419in}}%
\pgfpathlineto{\pgfqpoint{3.151094in}{3.685895in}}%
\pgfpathlineto{\pgfqpoint{3.161617in}{3.698873in}}%
\pgfpathlineto{\pgfqpoint{3.170636in}{3.706758in}}%
\pgfpathlineto{\pgfqpoint{3.178904in}{3.711381in}}%
\pgfpathlineto{\pgfqpoint{3.187172in}{3.713526in}}%
\pgfpathlineto{\pgfqpoint{3.194689in}{3.713331in}}%
\pgfpathlineto{\pgfqpoint{3.202957in}{3.710763in}}%
\pgfpathlineto{\pgfqpoint{3.211225in}{3.705725in}}%
\pgfpathlineto{\pgfqpoint{3.220245in}{3.697406in}}%
\pgfpathlineto{\pgfqpoint{3.230016in}{3.685053in}}%
\pgfpathlineto{\pgfqpoint{3.240539in}{3.667831in}}%
\pgfpathlineto{\pgfqpoint{3.251813in}{3.644820in}}%
\pgfpathlineto{\pgfqpoint{3.263840in}{3.615003in}}%
\pgfpathlineto{\pgfqpoint{3.276618in}{3.577251in}}%
\pgfpathlineto{\pgfqpoint{3.290147in}{3.530323in}}%
\pgfpathlineto{\pgfqpoint{3.304428in}{3.472851in}}%
\pgfpathlineto{\pgfqpoint{3.320213in}{3.399628in}}%
\pgfpathlineto{\pgfqpoint{3.336749in}{3.311757in}}%
\pgfpathlineto{\pgfqpoint{3.354037in}{3.207481in}}%
\pgfpathlineto{\pgfqpoint{3.372076in}{3.085073in}}%
\pgfpathlineto{\pgfqpoint{3.391619in}{2.936990in}}%
\pgfpathlineto{\pgfqpoint{3.412665in}{2.760290in}}%
\pgfpathlineto{\pgfqpoint{3.436717in}{2.538597in}}%
\pgfpathlineto{\pgfqpoint{3.465279in}{2.253609in}}%
\pgfpathlineto{\pgfqpoint{3.554725in}{1.343829in}}%
\pgfpathlineto{\pgfqpoint{3.574267in}{1.176818in}}%
\pgfpathlineto{\pgfqpoint{3.590803in}{1.054093in}}%
\pgfpathlineto{\pgfqpoint{3.605084in}{0.964366in}}%
\pgfpathlineto{\pgfqpoint{3.617111in}{0.901843in}}%
\pgfpathlineto{\pgfqpoint{3.627634in}{0.857638in}}%
\pgfpathlineto{\pgfqpoint{3.637405in}{0.825795in}}%
\pgfpathlineto{\pgfqpoint{3.645673in}{0.805993in}}%
\pgfpathlineto{\pgfqpoint{3.652438in}{0.794757in}}%
\pgfpathlineto{\pgfqpoint{3.658451in}{0.788562in}}%
\pgfpathlineto{\pgfqpoint{3.663712in}{0.786085in}}%
\pgfpathlineto{\pgfqpoint{3.668222in}{0.786153in}}%
\pgfpathlineto{\pgfqpoint{3.672732in}{0.788241in}}%
\pgfpathlineto{\pgfqpoint{3.677994in}{0.793222in}}%
\pgfpathlineto{\pgfqpoint{3.684007in}{0.802250in}}%
\pgfpathlineto{\pgfqpoint{3.690771in}{0.816617in}}%
\pgfpathlineto{\pgfqpoint{3.699040in}{0.840126in}}%
\pgfpathlineto{\pgfqpoint{3.708059in}{0.873049in}}%
\pgfpathlineto{\pgfqpoint{3.718582in}{0.920692in}}%
\pgfpathlineto{\pgfqpoint{3.730608in}{0.986652in}}%
\pgfpathlineto{\pgfqpoint{3.744138in}{1.074388in}}%
\pgfpathlineto{\pgfqpoint{3.759922in}{1.192818in}}%
\pgfpathlineto{\pgfqpoint{3.777962in}{1.345906in}}%
\pgfpathlineto{\pgfqpoint{3.800511in}{1.557192in}}%
\pgfpathlineto{\pgfqpoint{3.836590in}{1.919772in}}%
\pgfpathlineto{\pgfqpoint{3.883191in}{2.383654in}}%
\pgfpathlineto{\pgfqpoint{3.909499in}{2.623297in}}%
\pgfpathlineto{\pgfqpoint{3.932048in}{2.809165in}}%
\pgfpathlineto{\pgfqpoint{3.952342in}{2.958734in}}%
\pgfpathlineto{\pgfqpoint{3.971133in}{3.081283in}}%
\pgfpathlineto{\pgfqpoint{3.988421in}{3.180131in}}%
\pgfpathlineto{\pgfqpoint{4.004957in}{3.262143in}}%
\pgfpathlineto{\pgfqpoint{4.019990in}{3.326116in}}%
\pgfpathlineto{\pgfqpoint{4.034271in}{3.377657in}}%
\pgfpathlineto{\pgfqpoint{4.047801in}{3.418295in}}%
\pgfpathlineto{\pgfqpoint{4.059827in}{3.447816in}}%
\pgfpathlineto{\pgfqpoint{4.071102in}{3.469915in}}%
\pgfpathlineto{\pgfqpoint{4.081625in}{3.485721in}}%
\pgfpathlineto{\pgfqpoint{4.090644in}{3.495595in}}%
\pgfpathlineto{\pgfqpoint{4.098912in}{3.501684in}}%
\pgfpathlineto{\pgfqpoint{4.106429in}{3.504771in}}%
\pgfpathlineto{\pgfqpoint{4.113945in}{3.505532in}}%
\pgfpathlineto{\pgfqpoint{4.121461in}{3.503973in}}%
\pgfpathlineto{\pgfqpoint{4.128978in}{3.500094in}}%
\pgfpathlineto{\pgfqpoint{4.137246in}{3.493146in}}%
\pgfpathlineto{\pgfqpoint{4.146266in}{3.482360in}}%
\pgfpathlineto{\pgfqpoint{4.156037in}{3.466888in}}%
\pgfpathlineto{\pgfqpoint{4.166560in}{3.445802in}}%
\pgfpathlineto{\pgfqpoint{4.177834in}{3.418088in}}%
\pgfpathlineto{\pgfqpoint{4.189861in}{3.382647in}}%
\pgfpathlineto{\pgfqpoint{4.203390in}{3.335461in}}%
\pgfpathlineto{\pgfqpoint{4.217671in}{3.277186in}}%
\pgfpathlineto{\pgfqpoint{4.232704in}{3.206391in}}%
\pgfpathlineto{\pgfqpoint{4.249240in}{3.117304in}}%
\pgfpathlineto{\pgfqpoint{4.266528in}{3.011701in}}%
\pgfpathlineto{\pgfqpoint{4.285319in}{2.882795in}}%
\pgfpathlineto{\pgfqpoint{4.305613in}{2.727871in}}%
\pgfpathlineto{\pgfqpoint{4.328914in}{2.531827in}}%
\pgfpathlineto{\pgfqpoint{4.355973in}{2.284393in}}%
\pgfpathlineto{\pgfqpoint{4.401072in}{1.847410in}}%
\pgfpathlineto{\pgfqpoint{4.437150in}{1.507246in}}%
\pgfpathlineto{\pgfqpoint{4.459700in}{1.315028in}}%
\pgfpathlineto{\pgfqpoint{4.477739in}{1.179357in}}%
\pgfpathlineto{\pgfqpoint{4.492772in}{1.081712in}}%
\pgfpathlineto{\pgfqpoint{4.506301in}{1.007617in}}%
\pgfpathlineto{\pgfqpoint{4.518328in}{0.953784in}}%
\pgfpathlineto{\pgfqpoint{4.528851in}{0.916572in}}%
\pgfpathlineto{\pgfqpoint{4.537870in}{0.892333in}}%
\pgfpathlineto{\pgfqpoint{4.545387in}{0.877676in}}%
\pgfpathlineto{\pgfqpoint{4.552151in}{0.868859in}}%
\pgfpathlineto{\pgfqpoint{4.558165in}{0.864531in}}%
\pgfpathlineto{\pgfqpoint{4.563426in}{0.863464in}}%
\pgfpathlineto{\pgfqpoint{4.568688in}{0.864936in}}%
\pgfpathlineto{\pgfqpoint{4.573949in}{0.868940in}}%
\pgfpathlineto{\pgfqpoint{4.579962in}{0.876600in}}%
\pgfpathlineto{\pgfqpoint{4.586727in}{0.889112in}}%
\pgfpathlineto{\pgfqpoint{4.594243in}{0.907777in}}%
\pgfpathlineto{\pgfqpoint{4.603263in}{0.936641in}}%
\pgfpathlineto{\pgfqpoint{4.613034in}{0.975601in}}%
\pgfpathlineto{\pgfqpoint{4.624309in}{1.030013in}}%
\pgfpathlineto{\pgfqpoint{4.637087in}{1.103094in}}%
\pgfpathlineto{\pgfqpoint{4.652120in}{1.203070in}}%
\pgfpathlineto{\pgfqpoint{4.669407in}{1.334085in}}%
\pgfpathlineto{\pgfqpoint{4.690453in}{1.511748in}}%
\pgfpathlineto{\pgfqpoint{4.719767in}{1.780285in}}%
\pgfpathlineto{\pgfqpoint{4.785912in}{2.392218in}}%
\pgfpathlineto{\pgfqpoint{4.811467in}{2.604685in}}%
\pgfpathlineto{\pgfqpoint{4.833265in}{2.767950in}}%
\pgfpathlineto{\pgfqpoint{4.852808in}{2.898297in}}%
\pgfpathlineto{\pgfqpoint{4.870847in}{3.004254in}}%
\pgfpathlineto{\pgfqpoint{4.887383in}{3.088865in}}%
\pgfpathlineto{\pgfqpoint{4.903167in}{3.158302in}}%
\pgfpathlineto{\pgfqpoint{4.917449in}{3.211552in}}%
\pgfpathlineto{\pgfqpoint{4.930978in}{3.253624in}}%
\pgfpathlineto{\pgfqpoint{4.943004in}{3.284211in}}%
\pgfpathlineto{\pgfqpoint{4.954279in}{3.307093in}}%
\pgfpathlineto{\pgfqpoint{4.964802in}{3.323419in}}%
\pgfpathlineto{\pgfqpoint{4.973822in}{3.333563in}}%
\pgfpathlineto{\pgfqpoint{4.982090in}{3.339752in}}%
\pgfpathlineto{\pgfqpoint{4.989606in}{3.342803in}}%
\pgfpathlineto{\pgfqpoint{4.997123in}{3.343407in}}%
\pgfpathlineto{\pgfqpoint{5.004639in}{3.341567in}}%
\pgfpathlineto{\pgfqpoint{5.012155in}{3.337286in}}%
\pgfpathlineto{\pgfqpoint{5.020423in}{3.329759in}}%
\pgfpathlineto{\pgfqpoint{5.029443in}{3.318182in}}%
\pgfpathlineto{\pgfqpoint{5.039214in}{3.301673in}}%
\pgfpathlineto{\pgfqpoint{5.049737in}{3.279278in}}%
\pgfpathlineto{\pgfqpoint{5.061012in}{3.249966in}}%
\pgfpathlineto{\pgfqpoint{5.073790in}{3.210092in}}%
\pgfpathlineto{\pgfqpoint{5.087319in}{3.160169in}}%
\pgfpathlineto{\pgfqpoint{5.101601in}{3.098907in}}%
\pgfpathlineto{\pgfqpoint{5.117385in}{3.021046in}}%
\pgfpathlineto{\pgfqpoint{5.134673in}{2.923760in}}%
\pgfpathlineto{\pgfqpoint{5.152712in}{2.809303in}}%
\pgfpathlineto{\pgfqpoint{5.173006in}{2.665683in}}%
\pgfpathlineto{\pgfqpoint{5.195556in}{2.489597in}}%
\pgfpathlineto{\pgfqpoint{5.222615in}{2.259863in}}%
\pgfpathlineto{\pgfqpoint{5.267713in}{1.853847in}}%
\pgfpathlineto{\pgfqpoint{5.303792in}{1.537725in}}%
\pgfpathlineto{\pgfqpoint{5.326341in}{1.358961in}}%
\pgfpathlineto{\pgfqpoint{5.344380in}{1.232624in}}%
\pgfpathlineto{\pgfqpoint{5.360165in}{1.137356in}}%
\pgfpathlineto{\pgfqpoint{5.373694in}{1.068770in}}%
\pgfpathlineto{\pgfqpoint{5.385721in}{1.018929in}}%
\pgfpathlineto{\pgfqpoint{5.396244in}{0.984460in}}%
\pgfpathlineto{\pgfqpoint{5.405263in}{0.961989in}}%
\pgfpathlineto{\pgfqpoint{5.412780in}{0.948381in}}%
\pgfpathlineto{\pgfqpoint{5.419544in}{0.940175in}}%
\pgfpathlineto{\pgfqpoint{5.425558in}{0.936121in}}%
\pgfpathlineto{\pgfqpoint{5.430819in}{0.935086in}}%
\pgfpathlineto{\pgfqpoint{5.436081in}{0.936395in}}%
\pgfpathlineto{\pgfqpoint{5.441342in}{0.940044in}}%
\pgfpathlineto{\pgfqpoint{5.447355in}{0.947062in}}%
\pgfpathlineto{\pgfqpoint{5.454120in}{0.958556in}}%
\pgfpathlineto{\pgfqpoint{5.461636in}{0.975728in}}%
\pgfpathlineto{\pgfqpoint{5.470656in}{1.002311in}}%
\pgfpathlineto{\pgfqpoint{5.480427in}{1.038223in}}%
\pgfpathlineto{\pgfqpoint{5.491702in}{1.088413in}}%
\pgfpathlineto{\pgfqpoint{5.504480in}{1.155870in}}%
\pgfpathlineto{\pgfqpoint{5.519513in}{1.248222in}}%
\pgfpathlineto{\pgfqpoint{5.534545in}{1.352755in}}%
\pgfpathlineto{\pgfqpoint{5.534545in}{1.352755in}}%
\pgfusepath{stroke}%
\end{pgfscope}%
\begin{pgfscope}%
\pgfpathrectangle{\pgfqpoint{0.800000in}{0.528000in}}{\pgfqpoint{4.960000in}{3.696000in}}%
\pgfusepath{clip}%
\pgfsetrectcap%
\pgfsetroundjoin%
\pgfsetlinewidth{1.505625pt}%
\definecolor{currentstroke}{rgb}{1.000000,0.498039,0.054902}%
\pgfsetstrokecolor{currentstroke}%
\pgfsetdash{}{0pt}%
\pgfpathmoveto{\pgfqpoint{1.025455in}{1.954352in}}%
\pgfpathlineto{\pgfqpoint{1.070553in}{1.953247in}}%
\pgfpathlineto{\pgfqpoint{1.118658in}{1.949818in}}%
\pgfpathlineto{\pgfqpoint{1.177286in}{1.943291in}}%
\pgfpathlineto{\pgfqpoint{1.329117in}{1.925221in}}%
\pgfpathlineto{\pgfqpoint{1.377974in}{1.922390in}}%
\pgfpathlineto{\pgfqpoint{1.423824in}{1.921960in}}%
\pgfpathlineto{\pgfqpoint{1.470426in}{1.923782in}}%
\pgfpathlineto{\pgfqpoint{1.521537in}{1.928055in}}%
\pgfpathlineto{\pgfqpoint{1.591440in}{1.936345in}}%
\pgfpathlineto{\pgfqpoint{1.693663in}{1.948327in}}%
\pgfpathlineto{\pgfqpoint{1.746278in}{1.952083in}}%
\pgfpathlineto{\pgfqpoint{1.781605in}{1.954507in}}%
\pgfpathlineto{\pgfqpoint{1.786115in}{1.958602in}}%
\pgfpathlineto{\pgfqpoint{1.791376in}{1.966760in}}%
\pgfpathlineto{\pgfqpoint{1.797389in}{1.980537in}}%
\pgfpathlineto{\pgfqpoint{1.804154in}{2.001710in}}%
\pgfpathlineto{\pgfqpoint{1.812422in}{2.035735in}}%
\pgfpathlineto{\pgfqpoint{1.821442in}{2.083059in}}%
\pgfpathlineto{\pgfqpoint{1.831213in}{2.146326in}}%
\pgfpathlineto{\pgfqpoint{1.842488in}{2.234804in}}%
\pgfpathlineto{\pgfqpoint{1.854514in}{2.347424in}}%
\pgfpathlineto{\pgfqpoint{1.870298in}{2.519224in}}%
\pgfpathlineto{\pgfqpoint{1.896606in}{2.795765in}}%
\pgfpathlineto{\pgfqpoint{1.919155in}{3.010489in}}%
\pgfpathlineto{\pgfqpoint{1.940201in}{3.190571in}}%
\pgfpathlineto{\pgfqpoint{1.960495in}{3.345248in}}%
\pgfpathlineto{\pgfqpoint{1.979286in}{3.472130in}}%
\pgfpathlineto{\pgfqpoint{1.997326in}{3.579722in}}%
\pgfpathlineto{\pgfqpoint{2.015365in}{3.674110in}}%
\pgfpathlineto{\pgfqpoint{2.032653in}{3.752915in}}%
\pgfpathlineto{\pgfqpoint{2.049189in}{3.818311in}}%
\pgfpathlineto{\pgfqpoint{2.064973in}{3.872217in}}%
\pgfpathlineto{\pgfqpoint{2.080758in}{3.918332in}}%
\pgfpathlineto{\pgfqpoint{2.095791in}{3.955443in}}%
\pgfpathlineto{\pgfqpoint{2.110072in}{3.984892in}}%
\pgfpathlineto{\pgfqpoint{2.123601in}{4.007825in}}%
\pgfpathlineto{\pgfqpoint{2.136379in}{4.025235in}}%
\pgfpathlineto{\pgfqpoint{2.148405in}{4.037978in}}%
\pgfpathlineto{\pgfqpoint{2.159680in}{4.046803in}}%
\pgfpathlineto{\pgfqpoint{2.170203in}{4.052368in}}%
\pgfpathlineto{\pgfqpoint{2.180726in}{4.055388in}}%
\pgfpathlineto{\pgfqpoint{2.190497in}{4.055933in}}%
\pgfpathlineto{\pgfqpoint{2.200269in}{4.054309in}}%
\pgfpathlineto{\pgfqpoint{2.210040in}{4.050512in}}%
\pgfpathlineto{\pgfqpoint{2.220563in}{4.043977in}}%
\pgfpathlineto{\pgfqpoint{2.231837in}{4.034128in}}%
\pgfpathlineto{\pgfqpoint{2.243112in}{4.021281in}}%
\pgfpathlineto{\pgfqpoint{2.255138in}{4.004193in}}%
\pgfpathlineto{\pgfqpoint{2.267916in}{3.982093in}}%
\pgfpathlineto{\pgfqpoint{2.281446in}{3.954097in}}%
\pgfpathlineto{\pgfqpoint{2.295727in}{3.919188in}}%
\pgfpathlineto{\pgfqpoint{2.310760in}{3.876189in}}%
\pgfpathlineto{\pgfqpoint{2.325792in}{3.826413in}}%
\pgfpathlineto{\pgfqpoint{2.341577in}{3.766419in}}%
\pgfpathlineto{\pgfqpoint{2.358113in}{3.694548in}}%
\pgfpathlineto{\pgfqpoint{2.375401in}{3.608930in}}%
\pgfpathlineto{\pgfqpoint{2.392688in}{3.511974in}}%
\pgfpathlineto{\pgfqpoint{2.410728in}{3.398095in}}%
\pgfpathlineto{\pgfqpoint{2.429519in}{3.265132in}}%
\pgfpathlineto{\pgfqpoint{2.449062in}{3.111003in}}%
\pgfpathlineto{\pgfqpoint{2.470107in}{2.927068in}}%
\pgfpathlineto{\pgfqpoint{2.493408in}{2.702846in}}%
\pgfpathlineto{\pgfqpoint{2.519716in}{2.427196in}}%
\pgfpathlineto{\pgfqpoint{2.555794in}{2.023341in}}%
\pgfpathlineto{\pgfqpoint{2.606154in}{1.461554in}}%
\pgfpathlineto{\pgfqpoint{2.628704in}{1.234901in}}%
\pgfpathlineto{\pgfqpoint{2.646743in}{1.074343in}}%
\pgfpathlineto{\pgfqpoint{2.662527in}{0.953111in}}%
\pgfpathlineto{\pgfqpoint{2.676057in}{0.865782in}}%
\pgfpathlineto{\pgfqpoint{2.688083in}{0.802323in}}%
\pgfpathlineto{\pgfqpoint{2.698606in}{0.758461in}}%
\pgfpathlineto{\pgfqpoint{2.707626in}{0.729904in}}%
\pgfpathlineto{\pgfqpoint{2.715142in}{0.712651in}}%
\pgfpathlineto{\pgfqpoint{2.721907in}{0.702291in}}%
\pgfpathlineto{\pgfqpoint{2.727168in}{0.697646in}}%
\pgfpathlineto{\pgfqpoint{2.731678in}{0.696051in}}%
\pgfpathlineto{\pgfqpoint{2.736188in}{0.696661in}}%
\pgfpathlineto{\pgfqpoint{2.740698in}{0.699472in}}%
\pgfpathlineto{\pgfqpoint{2.745959in}{0.705524in}}%
\pgfpathlineto{\pgfqpoint{2.751973in}{0.716071in}}%
\pgfpathlineto{\pgfqpoint{2.758737in}{0.732515in}}%
\pgfpathlineto{\pgfqpoint{2.767005in}{0.759077in}}%
\pgfpathlineto{\pgfqpoint{2.776025in}{0.795951in}}%
\pgfpathlineto{\pgfqpoint{2.786548in}{0.848973in}}%
\pgfpathlineto{\pgfqpoint{2.798574in}{0.922019in}}%
\pgfpathlineto{\pgfqpoint{2.812104in}{1.018784in}}%
\pgfpathlineto{\pgfqpoint{2.827888in}{1.148931in}}%
\pgfpathlineto{\pgfqpoint{2.846679in}{1.323946in}}%
\pgfpathlineto{\pgfqpoint{2.869980in}{1.563140in}}%
\pgfpathlineto{\pgfqpoint{2.910569in}{2.007358in}}%
\pgfpathlineto{\pgfqpoint{2.950406in}{2.434467in}}%
\pgfpathlineto{\pgfqpoint{2.976713in}{2.693385in}}%
\pgfpathlineto{\pgfqpoint{2.999262in}{2.894532in}}%
\pgfpathlineto{\pgfqpoint{3.020308in}{3.062739in}}%
\pgfpathlineto{\pgfqpoint{3.039851in}{3.201234in}}%
\pgfpathlineto{\pgfqpoint{3.057890in}{3.313808in}}%
\pgfpathlineto{\pgfqpoint{3.075178in}{3.408098in}}%
\pgfpathlineto{\pgfqpoint{3.091714in}{3.486127in}}%
\pgfpathlineto{\pgfqpoint{3.107499in}{3.549836in}}%
\pgfpathlineto{\pgfqpoint{3.121780in}{3.598686in}}%
\pgfpathlineto{\pgfqpoint{3.135309in}{3.637486in}}%
\pgfpathlineto{\pgfqpoint{3.148087in}{3.667624in}}%
\pgfpathlineto{\pgfqpoint{3.160113in}{3.690346in}}%
\pgfpathlineto{\pgfqpoint{3.170636in}{3.705824in}}%
\pgfpathlineto{\pgfqpoint{3.180408in}{3.716569in}}%
\pgfpathlineto{\pgfqpoint{3.189427in}{3.723419in}}%
\pgfpathlineto{\pgfqpoint{3.197695in}{3.727130in}}%
\pgfpathlineto{\pgfqpoint{3.205963in}{3.728396in}}%
\pgfpathlineto{\pgfqpoint{3.214231in}{3.727225in}}%
\pgfpathlineto{\pgfqpoint{3.222500in}{3.723616in}}%
\pgfpathlineto{\pgfqpoint{3.230768in}{3.717566in}}%
\pgfpathlineto{\pgfqpoint{3.239787in}{3.708172in}}%
\pgfpathlineto{\pgfqpoint{3.249559in}{3.694685in}}%
\pgfpathlineto{\pgfqpoint{3.260082in}{3.676273in}}%
\pgfpathlineto{\pgfqpoint{3.271356in}{3.652019in}}%
\pgfpathlineto{\pgfqpoint{3.283382in}{3.620903in}}%
\pgfpathlineto{\pgfqpoint{3.296912in}{3.579302in}}%
\pgfpathlineto{\pgfqpoint{3.311193in}{3.527641in}}%
\pgfpathlineto{\pgfqpoint{3.326226in}{3.464451in}}%
\pgfpathlineto{\pgfqpoint{3.342010in}{3.388138in}}%
\pgfpathlineto{\pgfqpoint{3.358546in}{3.297004in}}%
\pgfpathlineto{\pgfqpoint{3.375834in}{3.189302in}}%
\pgfpathlineto{\pgfqpoint{3.394625in}{3.057783in}}%
\pgfpathlineto{\pgfqpoint{3.414919in}{2.899132in}}%
\pgfpathlineto{\pgfqpoint{3.436717in}{2.710541in}}%
\pgfpathlineto{\pgfqpoint{3.461521in}{2.475711in}}%
\pgfpathlineto{\pgfqpoint{3.493090in}{2.153713in}}%
\pgfpathlineto{\pgfqpoint{3.562993in}{1.432696in}}%
\pgfpathlineto{\pgfqpoint{3.584039in}{1.242071in}}%
\pgfpathlineto{\pgfqpoint{3.601326in}{1.104112in}}%
\pgfpathlineto{\pgfqpoint{3.616359in}{1.000967in}}%
\pgfpathlineto{\pgfqpoint{3.629137in}{0.927320in}}%
\pgfpathlineto{\pgfqpoint{3.640412in}{0.874011in}}%
\pgfpathlineto{\pgfqpoint{3.650935in}{0.834727in}}%
\pgfpathlineto{\pgfqpoint{3.659954in}{0.809422in}}%
\pgfpathlineto{\pgfqpoint{3.667471in}{0.794385in}}%
\pgfpathlineto{\pgfqpoint{3.674235in}{0.785624in}}%
\pgfpathlineto{\pgfqpoint{3.679497in}{0.781959in}}%
\pgfpathlineto{\pgfqpoint{3.684758in}{0.781059in}}%
\pgfpathlineto{\pgfqpoint{3.689268in}{0.782488in}}%
\pgfpathlineto{\pgfqpoint{3.694530in}{0.786716in}}%
\pgfpathlineto{\pgfqpoint{3.700543in}{0.794906in}}%
\pgfpathlineto{\pgfqpoint{3.707308in}{0.808363in}}%
\pgfpathlineto{\pgfqpoint{3.714824in}{0.828503in}}%
\pgfpathlineto{\pgfqpoint{3.723844in}{0.859713in}}%
\pgfpathlineto{\pgfqpoint{3.733615in}{0.901902in}}%
\pgfpathlineto{\pgfqpoint{3.744890in}{0.960880in}}%
\pgfpathlineto{\pgfqpoint{3.757667in}{1.040141in}}%
\pgfpathlineto{\pgfqpoint{3.772700in}{1.148608in}}%
\pgfpathlineto{\pgfqpoint{3.789988in}{1.290751in}}%
\pgfpathlineto{\pgfqpoint{3.811034in}{1.483437in}}%
\pgfpathlineto{\pgfqpoint{3.840348in}{1.774416in}}%
\pgfpathlineto{\pgfqpoint{3.905741in}{2.429117in}}%
\pgfpathlineto{\pgfqpoint{3.931296in}{2.659486in}}%
\pgfpathlineto{\pgfqpoint{3.953094in}{2.837063in}}%
\pgfpathlineto{\pgfqpoint{3.973388in}{2.984720in}}%
\pgfpathlineto{\pgfqpoint{3.992179in}{3.105378in}}%
\pgfpathlineto{\pgfqpoint{4.009467in}{3.202444in}}%
\pgfpathlineto{\pgfqpoint{4.026003in}{3.282756in}}%
\pgfpathlineto{\pgfqpoint{4.041036in}{3.345210in}}%
\pgfpathlineto{\pgfqpoint{4.055317in}{3.395343in}}%
\pgfpathlineto{\pgfqpoint{4.068095in}{3.432708in}}%
\pgfpathlineto{\pgfqpoint{4.080121in}{3.461499in}}%
\pgfpathlineto{\pgfqpoint{4.091396in}{3.482943in}}%
\pgfpathlineto{\pgfqpoint{4.101919in}{3.498163in}}%
\pgfpathlineto{\pgfqpoint{4.110938in}{3.507554in}}%
\pgfpathlineto{\pgfqpoint{4.119207in}{3.513216in}}%
\pgfpathlineto{\pgfqpoint{4.126723in}{3.515928in}}%
\pgfpathlineto{\pgfqpoint{4.134239in}{3.516328in}}%
\pgfpathlineto{\pgfqpoint{4.141756in}{3.514417in}}%
\pgfpathlineto{\pgfqpoint{4.149272in}{3.510198in}}%
\pgfpathlineto{\pgfqpoint{4.157540in}{3.502890in}}%
\pgfpathlineto{\pgfqpoint{4.166560in}{3.491723in}}%
\pgfpathlineto{\pgfqpoint{4.176331in}{3.475855in}}%
\pgfpathlineto{\pgfqpoint{4.186854in}{3.454357in}}%
\pgfpathlineto{\pgfqpoint{4.198129in}{3.426220in}}%
\pgfpathlineto{\pgfqpoint{4.210907in}{3.387897in}}%
\pgfpathlineto{\pgfqpoint{4.224436in}{3.339806in}}%
\pgfpathlineto{\pgfqpoint{4.238717in}{3.280584in}}%
\pgfpathlineto{\pgfqpoint{4.253750in}{3.208799in}}%
\pgfpathlineto{\pgfqpoint{4.270286in}{3.118627in}}%
\pgfpathlineto{\pgfqpoint{4.287574in}{3.011891in}}%
\pgfpathlineto{\pgfqpoint{4.306365in}{2.881758in}}%
\pgfpathlineto{\pgfqpoint{4.326659in}{2.725521in}}%
\pgfpathlineto{\pgfqpoint{4.349960in}{2.528011in}}%
\pgfpathlineto{\pgfqpoint{4.377771in}{2.271835in}}%
\pgfpathlineto{\pgfqpoint{4.427379in}{1.788634in}}%
\pgfpathlineto{\pgfqpoint{4.459700in}{1.485296in}}%
\pgfpathlineto{\pgfqpoint{4.481497in}{1.300407in}}%
\pgfpathlineto{\pgfqpoint{4.499537in}{1.165692in}}%
\pgfpathlineto{\pgfqpoint{4.514569in}{1.069157in}}%
\pgfpathlineto{\pgfqpoint{4.528099in}{0.996288in}}%
\pgfpathlineto{\pgfqpoint{4.539374in}{0.946645in}}%
\pgfpathlineto{\pgfqpoint{4.549896in}{0.909963in}}%
\pgfpathlineto{\pgfqpoint{4.558916in}{0.886235in}}%
\pgfpathlineto{\pgfqpoint{4.566433in}{0.872040in}}%
\pgfpathlineto{\pgfqpoint{4.573197in}{0.863666in}}%
\pgfpathlineto{\pgfqpoint{4.579210in}{0.859750in}}%
\pgfpathlineto{\pgfqpoint{4.584472in}{0.859058in}}%
\pgfpathlineto{\pgfqpoint{4.589733in}{0.860916in}}%
\pgfpathlineto{\pgfqpoint{4.594995in}{0.865317in}}%
\pgfpathlineto{\pgfqpoint{4.601008in}{0.873442in}}%
\pgfpathlineto{\pgfqpoint{4.607773in}{0.886490in}}%
\pgfpathlineto{\pgfqpoint{4.616041in}{0.907963in}}%
\pgfpathlineto{\pgfqpoint{4.625061in}{0.938149in}}%
\pgfpathlineto{\pgfqpoint{4.635584in}{0.981953in}}%
\pgfpathlineto{\pgfqpoint{4.647610in}{1.042733in}}%
\pgfpathlineto{\pgfqpoint{4.661139in}{1.123729in}}%
\pgfpathlineto{\pgfqpoint{4.676172in}{1.227703in}}%
\pgfpathlineto{\pgfqpoint{4.694211in}{1.368860in}}%
\pgfpathlineto{\pgfqpoint{4.716761in}{1.564407in}}%
\pgfpathlineto{\pgfqpoint{4.750584in}{1.880127in}}%
\pgfpathlineto{\pgfqpoint{4.801696in}{2.355306in}}%
\pgfpathlineto{\pgfqpoint{4.828003in}{2.578295in}}%
\pgfpathlineto{\pgfqpoint{4.850553in}{2.750770in}}%
\pgfpathlineto{\pgfqpoint{4.870847in}{2.888932in}}%
\pgfpathlineto{\pgfqpoint{4.888886in}{2.997153in}}%
\pgfpathlineto{\pgfqpoint{4.906174in}{3.087511in}}%
\pgfpathlineto{\pgfqpoint{4.921959in}{3.158419in}}%
\pgfpathlineto{\pgfqpoint{4.936240in}{3.212997in}}%
\pgfpathlineto{\pgfqpoint{4.949769in}{3.256325in}}%
\pgfpathlineto{\pgfqpoint{4.961795in}{3.288027in}}%
\pgfpathlineto{\pgfqpoint{4.973070in}{3.311955in}}%
\pgfpathlineto{\pgfqpoint{4.983593in}{3.329259in}}%
\pgfpathlineto{\pgfqpoint{4.992613in}{3.340244in}}%
\pgfpathlineto{\pgfqpoint{5.000881in}{3.347206in}}%
\pgfpathlineto{\pgfqpoint{5.008397in}{3.350963in}}%
\pgfpathlineto{\pgfqpoint{5.015914in}{3.352276in}}%
\pgfpathlineto{\pgfqpoint{5.023430in}{3.351148in}}%
\pgfpathlineto{\pgfqpoint{5.030946in}{3.347584in}}%
\pgfpathlineto{\pgfqpoint{5.038463in}{3.341584in}}%
\pgfpathlineto{\pgfqpoint{5.046731in}{3.332171in}}%
\pgfpathlineto{\pgfqpoint{5.055751in}{3.318540in}}%
\pgfpathlineto{\pgfqpoint{5.065522in}{3.299811in}}%
\pgfpathlineto{\pgfqpoint{5.076796in}{3.273075in}}%
\pgfpathlineto{\pgfqpoint{5.088823in}{3.238497in}}%
\pgfpathlineto{\pgfqpoint{5.101601in}{3.194898in}}%
\pgfpathlineto{\pgfqpoint{5.115882in}{3.137817in}}%
\pgfpathlineto{\pgfqpoint{5.130915in}{3.068254in}}%
\pgfpathlineto{\pgfqpoint{5.147451in}{2.980643in}}%
\pgfpathlineto{\pgfqpoint{5.164738in}{2.876918in}}%
\pgfpathlineto{\pgfqpoint{5.184281in}{2.745389in}}%
\pgfpathlineto{\pgfqpoint{5.205327in}{2.588118in}}%
\pgfpathlineto{\pgfqpoint{5.230131in}{2.384918in}}%
\pgfpathlineto{\pgfqpoint{5.261700in}{2.106317in}}%
\pgfpathlineto{\pgfqpoint{5.328596in}{1.510665in}}%
\pgfpathlineto{\pgfqpoint{5.350394in}{1.339426in}}%
\pgfpathlineto{\pgfqpoint{5.368433in}{1.214749in}}%
\pgfpathlineto{\pgfqpoint{5.383466in}{1.125444in}}%
\pgfpathlineto{\pgfqpoint{5.396995in}{1.058045in}}%
\pgfpathlineto{\pgfqpoint{5.408270in}{1.012129in}}%
\pgfpathlineto{\pgfqpoint{5.418793in}{0.978195in}}%
\pgfpathlineto{\pgfqpoint{5.427813in}{0.956235in}}%
\pgfpathlineto{\pgfqpoint{5.435329in}{0.943088in}}%
\pgfpathlineto{\pgfqpoint{5.442094in}{0.935320in}}%
\pgfpathlineto{\pgfqpoint{5.448107in}{0.931674in}}%
\pgfpathlineto{\pgfqpoint{5.453368in}{0.931008in}}%
\pgfpathlineto{\pgfqpoint{5.458630in}{0.932697in}}%
\pgfpathlineto{\pgfqpoint{5.463891in}{0.936735in}}%
\pgfpathlineto{\pgfqpoint{5.469904in}{0.944209in}}%
\pgfpathlineto{\pgfqpoint{5.476669in}{0.956227in}}%
\pgfpathlineto{\pgfqpoint{5.484937in}{0.976020in}}%
\pgfpathlineto{\pgfqpoint{5.493957in}{1.003862in}}%
\pgfpathlineto{\pgfqpoint{5.504480in}{1.044284in}}%
\pgfpathlineto{\pgfqpoint{5.516506in}{1.100398in}}%
\pgfpathlineto{\pgfqpoint{5.530036in}{1.175218in}}%
\pgfpathlineto{\pgfqpoint{5.534545in}{1.202697in}}%
\pgfpathlineto{\pgfqpoint{5.534545in}{1.202697in}}%
\pgfusepath{stroke}%
\end{pgfscope}%
\begin{pgfscope}%
\pgfsetrectcap%
\pgfsetmiterjoin%
\pgfsetlinewidth{0.803000pt}%
\definecolor{currentstroke}{rgb}{0.000000,0.000000,0.000000}%
\pgfsetstrokecolor{currentstroke}%
\pgfsetdash{}{0pt}%
\pgfpathmoveto{\pgfqpoint{0.800000in}{0.528000in}}%
\pgfpathlineto{\pgfqpoint{0.800000in}{4.224000in}}%
\pgfusepath{stroke}%
\end{pgfscope}%
\begin{pgfscope}%
\pgfsetrectcap%
\pgfsetmiterjoin%
\pgfsetlinewidth{0.803000pt}%
\definecolor{currentstroke}{rgb}{0.000000,0.000000,0.000000}%
\pgfsetstrokecolor{currentstroke}%
\pgfsetdash{}{0pt}%
\pgfpathmoveto{\pgfqpoint{5.760000in}{0.528000in}}%
\pgfpathlineto{\pgfqpoint{5.760000in}{4.224000in}}%
\pgfusepath{stroke}%
\end{pgfscope}%
\begin{pgfscope}%
\pgfsetrectcap%
\pgfsetmiterjoin%
\pgfsetlinewidth{0.803000pt}%
\definecolor{currentstroke}{rgb}{0.000000,0.000000,0.000000}%
\pgfsetstrokecolor{currentstroke}%
\pgfsetdash{}{0pt}%
\pgfpathmoveto{\pgfqpoint{0.800000in}{0.528000in}}%
\pgfpathlineto{\pgfqpoint{5.760000in}{0.528000in}}%
\pgfusepath{stroke}%
\end{pgfscope}%
\begin{pgfscope}%
\pgfsetrectcap%
\pgfsetmiterjoin%
\pgfsetlinewidth{0.803000pt}%
\definecolor{currentstroke}{rgb}{0.000000,0.000000,0.000000}%
\pgfsetstrokecolor{currentstroke}%
\pgfsetdash{}{0pt}%
\pgfpathmoveto{\pgfqpoint{0.800000in}{4.224000in}}%
\pgfpathlineto{\pgfqpoint{5.760000in}{4.224000in}}%
\pgfusepath{stroke}%
\end{pgfscope}%
\begin{pgfscope}%
\definecolor{textcolor}{rgb}{0.000000,0.000000,0.000000}%
\pgfsetstrokecolor{textcolor}%
\pgfsetfillcolor{textcolor}%
\pgftext[x=3.280000in,y=4.307333in,,base]{\color{textcolor}\rmfamily\fontsize{12.000000}{14.400000}\selectfont Power angle - comparison algebraic vs. non-algebraic}%
\end{pgfscope}%
\begin{pgfscope}%
\pgfsetbuttcap%
\pgfsetmiterjoin%
\definecolor{currentfill}{rgb}{1.000000,1.000000,1.000000}%
\pgfsetfillcolor{currentfill}%
\pgfsetfillopacity{0.800000}%
\pgfsetlinewidth{1.003750pt}%
\definecolor{currentstroke}{rgb}{0.800000,0.800000,0.800000}%
\pgfsetstrokecolor{currentstroke}%
\pgfsetstrokeopacity{0.800000}%
\pgfsetdash{}{0pt}%
\pgfpathmoveto{\pgfqpoint{4.391917in}{3.709108in}}%
\pgfpathlineto{\pgfqpoint{5.662778in}{3.709108in}}%
\pgfpathquadraticcurveto{\pgfqpoint{5.690556in}{3.709108in}}{\pgfqpoint{5.690556in}{3.736886in}}%
\pgfpathlineto{\pgfqpoint{5.690556in}{4.126778in}}%
\pgfpathquadraticcurveto{\pgfqpoint{5.690556in}{4.154556in}}{\pgfqpoint{5.662778in}{4.154556in}}%
\pgfpathlineto{\pgfqpoint{4.391917in}{4.154556in}}%
\pgfpathquadraticcurveto{\pgfqpoint{4.364140in}{4.154556in}}{\pgfqpoint{4.364140in}{4.126778in}}%
\pgfpathlineto{\pgfqpoint{4.364140in}{3.736886in}}%
\pgfpathquadraticcurveto{\pgfqpoint{4.364140in}{3.709108in}}{\pgfqpoint{4.391917in}{3.709108in}}%
\pgfpathlineto{\pgfqpoint{4.391917in}{3.709108in}}%
\pgfpathclose%
\pgfusepath{stroke,fill}%
\end{pgfscope}%
\begin{pgfscope}%
\pgfsetrectcap%
\pgfsetroundjoin%
\pgfsetlinewidth{1.505625pt}%
\definecolor{currentstroke}{rgb}{0.121569,0.466667,0.705882}%
\pgfsetstrokecolor{currentstroke}%
\pgfsetdash{}{0pt}%
\pgfpathmoveto{\pgfqpoint{4.419695in}{4.045411in}}%
\pgfpathlineto{\pgfqpoint{4.558584in}{4.045411in}}%
\pgfpathlineto{\pgfqpoint{4.697473in}{4.045411in}}%
\pgfusepath{stroke}%
\end{pgfscope}%
\begin{pgfscope}%
\definecolor{textcolor}{rgb}{0.000000,0.000000,0.000000}%
\pgfsetstrokecolor{textcolor}%
\pgfsetfillcolor{textcolor}%
\pgftext[x=4.808584in,y=3.996800in,left,base]{\color{textcolor}\rmfamily\fontsize{10.000000}{12.000000}\selectfont algebraic}%
\end{pgfscope}%
\begin{pgfscope}%
\pgfsetrectcap%
\pgfsetroundjoin%
\pgfsetlinewidth{1.505625pt}%
\definecolor{currentstroke}{rgb}{1.000000,0.498039,0.054902}%
\pgfsetstrokecolor{currentstroke}%
\pgfsetdash{}{0pt}%
\pgfpathmoveto{\pgfqpoint{4.419695in}{3.843521in}}%
\pgfpathlineto{\pgfqpoint{4.558584in}{3.843521in}}%
\pgfpathlineto{\pgfqpoint{4.697473in}{3.843521in}}%
\pgfusepath{stroke}%
\end{pgfscope}%
\begin{pgfscope}%
\definecolor{textcolor}{rgb}{0.000000,0.000000,0.000000}%
\pgfsetstrokecolor{textcolor}%
\pgfsetfillcolor{textcolor}%
\pgftext[x=4.808584in,y=3.794909in,left,base]{\color{textcolor}\rmfamily\fontsize{10.000000}{12.000000}\selectfont non-algebraic}%
\end{pgfscope}%
\end{pgfpicture}%
\makeatother%
\endgroup%


%% Creator: Matplotlib, PGF backend
%%
%% To include the figure in your LaTeX document, write
%%   \input{<filename>.pgf}
%%
%% Make sure the required packages are loaded in your preamble
%%   \usepackage{pgf}
%%
%% Also ensure that all the required font packages are loaded; for instance,
%% the lmodern package is sometimes necessary when using math font.
%%   \usepackage{lmodern}
%%
%% Figures using additional raster images can only be included by \input if
%% they are in the same directory as the main LaTeX file. For loading figures
%% from other directories you can use the `import` package
%%   \usepackage{import}
%%
%% and then include the figures with
%%   \import{<path to file>}{<filename>.pgf}
%%
%% Matplotlib used the following preamble
%%   
%%   \usepackage{fontspec}
%%   \setmainfont{Charter.ttc}[Path=\detokenize{/System/Library/Fonts/Supplemental/}]
%%   \setsansfont{DejaVuSans.ttf}[Path=\detokenize{/opt/homebrew/lib/python3.10/site-packages/matplotlib/mpl-data/fonts/ttf/}]
%%   \setmonofont{DejaVuSansMono.ttf}[Path=\detokenize{/opt/homebrew/lib/python3.10/site-packages/matplotlib/mpl-data/fonts/ttf/}]
%%   \makeatletter\@ifpackageloaded{underscore}{}{\usepackage[strings]{underscore}}\makeatother
%%
\begingroup%
\makeatletter%
\begin{pgfpicture}%
\pgfpathrectangle{\pgfpointorigin}{\pgfqpoint{6.400000in}{4.800000in}}%
\pgfusepath{use as bounding box, clip}%
\begin{pgfscope}%
\pgfsetbuttcap%
\pgfsetmiterjoin%
\definecolor{currentfill}{rgb}{1.000000,1.000000,1.000000}%
\pgfsetfillcolor{currentfill}%
\pgfsetlinewidth{0.000000pt}%
\definecolor{currentstroke}{rgb}{1.000000,1.000000,1.000000}%
\pgfsetstrokecolor{currentstroke}%
\pgfsetdash{}{0pt}%
\pgfpathmoveto{\pgfqpoint{0.000000in}{0.000000in}}%
\pgfpathlineto{\pgfqpoint{6.400000in}{0.000000in}}%
\pgfpathlineto{\pgfqpoint{6.400000in}{4.800000in}}%
\pgfpathlineto{\pgfqpoint{0.000000in}{4.800000in}}%
\pgfpathlineto{\pgfqpoint{0.000000in}{0.000000in}}%
\pgfpathclose%
\pgfusepath{fill}%
\end{pgfscope}%
\begin{pgfscope}%
\pgfsetbuttcap%
\pgfsetmiterjoin%
\definecolor{currentfill}{rgb}{1.000000,1.000000,1.000000}%
\pgfsetfillcolor{currentfill}%
\pgfsetlinewidth{0.000000pt}%
\definecolor{currentstroke}{rgb}{0.000000,0.000000,0.000000}%
\pgfsetstrokecolor{currentstroke}%
\pgfsetstrokeopacity{0.000000}%
\pgfsetdash{}{0pt}%
\pgfpathmoveto{\pgfqpoint{0.800000in}{0.528000in}}%
\pgfpathlineto{\pgfqpoint{5.760000in}{0.528000in}}%
\pgfpathlineto{\pgfqpoint{5.760000in}{4.224000in}}%
\pgfpathlineto{\pgfqpoint{0.800000in}{4.224000in}}%
\pgfpathlineto{\pgfqpoint{0.800000in}{0.528000in}}%
\pgfpathclose%
\pgfusepath{fill}%
\end{pgfscope}%
\begin{pgfscope}%
\pgfpathrectangle{\pgfqpoint{0.800000in}{0.528000in}}{\pgfqpoint{4.960000in}{3.696000in}}%
\pgfusepath{clip}%
\pgfsetrectcap%
\pgfsetroundjoin%
\pgfsetlinewidth{0.803000pt}%
\definecolor{currentstroke}{rgb}{0.690196,0.690196,0.690196}%
\pgfsetstrokecolor{currentstroke}%
\pgfsetdash{}{0pt}%
\pgfpathmoveto{\pgfqpoint{1.025455in}{0.528000in}}%
\pgfpathlineto{\pgfqpoint{1.025455in}{4.224000in}}%
\pgfusepath{stroke}%
\end{pgfscope}%
\begin{pgfscope}%
\pgfsetbuttcap%
\pgfsetroundjoin%
\definecolor{currentfill}{rgb}{0.000000,0.000000,0.000000}%
\pgfsetfillcolor{currentfill}%
\pgfsetlinewidth{0.803000pt}%
\definecolor{currentstroke}{rgb}{0.000000,0.000000,0.000000}%
\pgfsetstrokecolor{currentstroke}%
\pgfsetdash{}{0pt}%
\pgfsys@defobject{currentmarker}{\pgfqpoint{0.000000in}{-0.048611in}}{\pgfqpoint{0.000000in}{0.000000in}}{%
\pgfpathmoveto{\pgfqpoint{0.000000in}{0.000000in}}%
\pgfpathlineto{\pgfqpoint{0.000000in}{-0.048611in}}%
\pgfusepath{stroke,fill}%
}%
\begin{pgfscope}%
\pgfsys@transformshift{1.025455in}{0.528000in}%
\pgfsys@useobject{currentmarker}{}%
\end{pgfscope}%
\end{pgfscope}%
\begin{pgfscope}%
\definecolor{textcolor}{rgb}{0.000000,0.000000,0.000000}%
\pgfsetstrokecolor{textcolor}%
\pgfsetfillcolor{textcolor}%
\pgftext[x=1.025455in,y=0.430778in,,top]{\color{textcolor}\rmfamily\fontsize{10.000000}{12.000000}\selectfont \(\displaystyle {\ensuremath{-}1}\)}%
\end{pgfscope}%
\begin{pgfscope}%
\pgfpathrectangle{\pgfqpoint{0.800000in}{0.528000in}}{\pgfqpoint{4.960000in}{3.696000in}}%
\pgfusepath{clip}%
\pgfsetrectcap%
\pgfsetroundjoin%
\pgfsetlinewidth{0.803000pt}%
\definecolor{currentstroke}{rgb}{0.690196,0.690196,0.690196}%
\pgfsetstrokecolor{currentstroke}%
\pgfsetdash{}{0pt}%
\pgfpathmoveto{\pgfqpoint{1.777095in}{0.528000in}}%
\pgfpathlineto{\pgfqpoint{1.777095in}{4.224000in}}%
\pgfusepath{stroke}%
\end{pgfscope}%
\begin{pgfscope}%
\pgfsetbuttcap%
\pgfsetroundjoin%
\definecolor{currentfill}{rgb}{0.000000,0.000000,0.000000}%
\pgfsetfillcolor{currentfill}%
\pgfsetlinewidth{0.803000pt}%
\definecolor{currentstroke}{rgb}{0.000000,0.000000,0.000000}%
\pgfsetstrokecolor{currentstroke}%
\pgfsetdash{}{0pt}%
\pgfsys@defobject{currentmarker}{\pgfqpoint{0.000000in}{-0.048611in}}{\pgfqpoint{0.000000in}{0.000000in}}{%
\pgfpathmoveto{\pgfqpoint{0.000000in}{0.000000in}}%
\pgfpathlineto{\pgfqpoint{0.000000in}{-0.048611in}}%
\pgfusepath{stroke,fill}%
}%
\begin{pgfscope}%
\pgfsys@transformshift{1.777095in}{0.528000in}%
\pgfsys@useobject{currentmarker}{}%
\end{pgfscope}%
\end{pgfscope}%
\begin{pgfscope}%
\definecolor{textcolor}{rgb}{0.000000,0.000000,0.000000}%
\pgfsetstrokecolor{textcolor}%
\pgfsetfillcolor{textcolor}%
\pgftext[x=1.777095in,y=0.430778in,,top]{\color{textcolor}\rmfamily\fontsize{10.000000}{12.000000}\selectfont \(\displaystyle {0}\)}%
\end{pgfscope}%
\begin{pgfscope}%
\pgfpathrectangle{\pgfqpoint{0.800000in}{0.528000in}}{\pgfqpoint{4.960000in}{3.696000in}}%
\pgfusepath{clip}%
\pgfsetrectcap%
\pgfsetroundjoin%
\pgfsetlinewidth{0.803000pt}%
\definecolor{currentstroke}{rgb}{0.690196,0.690196,0.690196}%
\pgfsetstrokecolor{currentstroke}%
\pgfsetdash{}{0pt}%
\pgfpathmoveto{\pgfqpoint{2.528735in}{0.528000in}}%
\pgfpathlineto{\pgfqpoint{2.528735in}{4.224000in}}%
\pgfusepath{stroke}%
\end{pgfscope}%
\begin{pgfscope}%
\pgfsetbuttcap%
\pgfsetroundjoin%
\definecolor{currentfill}{rgb}{0.000000,0.000000,0.000000}%
\pgfsetfillcolor{currentfill}%
\pgfsetlinewidth{0.803000pt}%
\definecolor{currentstroke}{rgb}{0.000000,0.000000,0.000000}%
\pgfsetstrokecolor{currentstroke}%
\pgfsetdash{}{0pt}%
\pgfsys@defobject{currentmarker}{\pgfqpoint{0.000000in}{-0.048611in}}{\pgfqpoint{0.000000in}{0.000000in}}{%
\pgfpathmoveto{\pgfqpoint{0.000000in}{0.000000in}}%
\pgfpathlineto{\pgfqpoint{0.000000in}{-0.048611in}}%
\pgfusepath{stroke,fill}%
}%
\begin{pgfscope}%
\pgfsys@transformshift{2.528735in}{0.528000in}%
\pgfsys@useobject{currentmarker}{}%
\end{pgfscope}%
\end{pgfscope}%
\begin{pgfscope}%
\definecolor{textcolor}{rgb}{0.000000,0.000000,0.000000}%
\pgfsetstrokecolor{textcolor}%
\pgfsetfillcolor{textcolor}%
\pgftext[x=2.528735in,y=0.430778in,,top]{\color{textcolor}\rmfamily\fontsize{10.000000}{12.000000}\selectfont \(\displaystyle {1}\)}%
\end{pgfscope}%
\begin{pgfscope}%
\pgfpathrectangle{\pgfqpoint{0.800000in}{0.528000in}}{\pgfqpoint{4.960000in}{3.696000in}}%
\pgfusepath{clip}%
\pgfsetrectcap%
\pgfsetroundjoin%
\pgfsetlinewidth{0.803000pt}%
\definecolor{currentstroke}{rgb}{0.690196,0.690196,0.690196}%
\pgfsetstrokecolor{currentstroke}%
\pgfsetdash{}{0pt}%
\pgfpathmoveto{\pgfqpoint{3.280376in}{0.528000in}}%
\pgfpathlineto{\pgfqpoint{3.280376in}{4.224000in}}%
\pgfusepath{stroke}%
\end{pgfscope}%
\begin{pgfscope}%
\pgfsetbuttcap%
\pgfsetroundjoin%
\definecolor{currentfill}{rgb}{0.000000,0.000000,0.000000}%
\pgfsetfillcolor{currentfill}%
\pgfsetlinewidth{0.803000pt}%
\definecolor{currentstroke}{rgb}{0.000000,0.000000,0.000000}%
\pgfsetstrokecolor{currentstroke}%
\pgfsetdash{}{0pt}%
\pgfsys@defobject{currentmarker}{\pgfqpoint{0.000000in}{-0.048611in}}{\pgfqpoint{0.000000in}{0.000000in}}{%
\pgfpathmoveto{\pgfqpoint{0.000000in}{0.000000in}}%
\pgfpathlineto{\pgfqpoint{0.000000in}{-0.048611in}}%
\pgfusepath{stroke,fill}%
}%
\begin{pgfscope}%
\pgfsys@transformshift{3.280376in}{0.528000in}%
\pgfsys@useobject{currentmarker}{}%
\end{pgfscope}%
\end{pgfscope}%
\begin{pgfscope}%
\definecolor{textcolor}{rgb}{0.000000,0.000000,0.000000}%
\pgfsetstrokecolor{textcolor}%
\pgfsetfillcolor{textcolor}%
\pgftext[x=3.280376in,y=0.430778in,,top]{\color{textcolor}\rmfamily\fontsize{10.000000}{12.000000}\selectfont \(\displaystyle {2}\)}%
\end{pgfscope}%
\begin{pgfscope}%
\pgfpathrectangle{\pgfqpoint{0.800000in}{0.528000in}}{\pgfqpoint{4.960000in}{3.696000in}}%
\pgfusepath{clip}%
\pgfsetrectcap%
\pgfsetroundjoin%
\pgfsetlinewidth{0.803000pt}%
\definecolor{currentstroke}{rgb}{0.690196,0.690196,0.690196}%
\pgfsetstrokecolor{currentstroke}%
\pgfsetdash{}{0pt}%
\pgfpathmoveto{\pgfqpoint{4.032016in}{0.528000in}}%
\pgfpathlineto{\pgfqpoint{4.032016in}{4.224000in}}%
\pgfusepath{stroke}%
\end{pgfscope}%
\begin{pgfscope}%
\pgfsetbuttcap%
\pgfsetroundjoin%
\definecolor{currentfill}{rgb}{0.000000,0.000000,0.000000}%
\pgfsetfillcolor{currentfill}%
\pgfsetlinewidth{0.803000pt}%
\definecolor{currentstroke}{rgb}{0.000000,0.000000,0.000000}%
\pgfsetstrokecolor{currentstroke}%
\pgfsetdash{}{0pt}%
\pgfsys@defobject{currentmarker}{\pgfqpoint{0.000000in}{-0.048611in}}{\pgfqpoint{0.000000in}{0.000000in}}{%
\pgfpathmoveto{\pgfqpoint{0.000000in}{0.000000in}}%
\pgfpathlineto{\pgfqpoint{0.000000in}{-0.048611in}}%
\pgfusepath{stroke,fill}%
}%
\begin{pgfscope}%
\pgfsys@transformshift{4.032016in}{0.528000in}%
\pgfsys@useobject{currentmarker}{}%
\end{pgfscope}%
\end{pgfscope}%
\begin{pgfscope}%
\definecolor{textcolor}{rgb}{0.000000,0.000000,0.000000}%
\pgfsetstrokecolor{textcolor}%
\pgfsetfillcolor{textcolor}%
\pgftext[x=4.032016in,y=0.430778in,,top]{\color{textcolor}\rmfamily\fontsize{10.000000}{12.000000}\selectfont \(\displaystyle {3}\)}%
\end{pgfscope}%
\begin{pgfscope}%
\pgfpathrectangle{\pgfqpoint{0.800000in}{0.528000in}}{\pgfqpoint{4.960000in}{3.696000in}}%
\pgfusepath{clip}%
\pgfsetrectcap%
\pgfsetroundjoin%
\pgfsetlinewidth{0.803000pt}%
\definecolor{currentstroke}{rgb}{0.690196,0.690196,0.690196}%
\pgfsetstrokecolor{currentstroke}%
\pgfsetdash{}{0pt}%
\pgfpathmoveto{\pgfqpoint{4.783657in}{0.528000in}}%
\pgfpathlineto{\pgfqpoint{4.783657in}{4.224000in}}%
\pgfusepath{stroke}%
\end{pgfscope}%
\begin{pgfscope}%
\pgfsetbuttcap%
\pgfsetroundjoin%
\definecolor{currentfill}{rgb}{0.000000,0.000000,0.000000}%
\pgfsetfillcolor{currentfill}%
\pgfsetlinewidth{0.803000pt}%
\definecolor{currentstroke}{rgb}{0.000000,0.000000,0.000000}%
\pgfsetstrokecolor{currentstroke}%
\pgfsetdash{}{0pt}%
\pgfsys@defobject{currentmarker}{\pgfqpoint{0.000000in}{-0.048611in}}{\pgfqpoint{0.000000in}{0.000000in}}{%
\pgfpathmoveto{\pgfqpoint{0.000000in}{0.000000in}}%
\pgfpathlineto{\pgfqpoint{0.000000in}{-0.048611in}}%
\pgfusepath{stroke,fill}%
}%
\begin{pgfscope}%
\pgfsys@transformshift{4.783657in}{0.528000in}%
\pgfsys@useobject{currentmarker}{}%
\end{pgfscope}%
\end{pgfscope}%
\begin{pgfscope}%
\definecolor{textcolor}{rgb}{0.000000,0.000000,0.000000}%
\pgfsetstrokecolor{textcolor}%
\pgfsetfillcolor{textcolor}%
\pgftext[x=4.783657in,y=0.430778in,,top]{\color{textcolor}\rmfamily\fontsize{10.000000}{12.000000}\selectfont \(\displaystyle {4}\)}%
\end{pgfscope}%
\begin{pgfscope}%
\pgfpathrectangle{\pgfqpoint{0.800000in}{0.528000in}}{\pgfqpoint{4.960000in}{3.696000in}}%
\pgfusepath{clip}%
\pgfsetrectcap%
\pgfsetroundjoin%
\pgfsetlinewidth{0.803000pt}%
\definecolor{currentstroke}{rgb}{0.690196,0.690196,0.690196}%
\pgfsetstrokecolor{currentstroke}%
\pgfsetdash{}{0pt}%
\pgfpathmoveto{\pgfqpoint{5.535297in}{0.528000in}}%
\pgfpathlineto{\pgfqpoint{5.535297in}{4.224000in}}%
\pgfusepath{stroke}%
\end{pgfscope}%
\begin{pgfscope}%
\pgfsetbuttcap%
\pgfsetroundjoin%
\definecolor{currentfill}{rgb}{0.000000,0.000000,0.000000}%
\pgfsetfillcolor{currentfill}%
\pgfsetlinewidth{0.803000pt}%
\definecolor{currentstroke}{rgb}{0.000000,0.000000,0.000000}%
\pgfsetstrokecolor{currentstroke}%
\pgfsetdash{}{0pt}%
\pgfsys@defobject{currentmarker}{\pgfqpoint{0.000000in}{-0.048611in}}{\pgfqpoint{0.000000in}{0.000000in}}{%
\pgfpathmoveto{\pgfqpoint{0.000000in}{0.000000in}}%
\pgfpathlineto{\pgfqpoint{0.000000in}{-0.048611in}}%
\pgfusepath{stroke,fill}%
}%
\begin{pgfscope}%
\pgfsys@transformshift{5.535297in}{0.528000in}%
\pgfsys@useobject{currentmarker}{}%
\end{pgfscope}%
\end{pgfscope}%
\begin{pgfscope}%
\definecolor{textcolor}{rgb}{0.000000,0.000000,0.000000}%
\pgfsetstrokecolor{textcolor}%
\pgfsetfillcolor{textcolor}%
\pgftext[x=5.535297in,y=0.430778in,,top]{\color{textcolor}\rmfamily\fontsize{10.000000}{12.000000}\selectfont \(\displaystyle {5}\)}%
\end{pgfscope}%
\begin{pgfscope}%
\pgfpathrectangle{\pgfqpoint{0.800000in}{0.528000in}}{\pgfqpoint{4.960000in}{3.696000in}}%
\pgfusepath{clip}%
\pgfsetrectcap%
\pgfsetroundjoin%
\pgfsetlinewidth{0.803000pt}%
\definecolor{currentstroke}{rgb}{0.690196,0.690196,0.690196}%
\pgfsetstrokecolor{currentstroke}%
\pgfsetdash{}{0pt}%
\pgfpathmoveto{\pgfqpoint{0.800000in}{0.696000in}}%
\pgfpathlineto{\pgfqpoint{5.760000in}{0.696000in}}%
\pgfusepath{stroke}%
\end{pgfscope}%
\begin{pgfscope}%
\pgfsetbuttcap%
\pgfsetroundjoin%
\definecolor{currentfill}{rgb}{0.000000,0.000000,0.000000}%
\pgfsetfillcolor{currentfill}%
\pgfsetlinewidth{0.803000pt}%
\definecolor{currentstroke}{rgb}{0.000000,0.000000,0.000000}%
\pgfsetstrokecolor{currentstroke}%
\pgfsetdash{}{0pt}%
\pgfsys@defobject{currentmarker}{\pgfqpoint{-0.048611in}{0.000000in}}{\pgfqpoint{-0.000000in}{0.000000in}}{%
\pgfpathmoveto{\pgfqpoint{-0.000000in}{0.000000in}}%
\pgfpathlineto{\pgfqpoint{-0.048611in}{0.000000in}}%
\pgfusepath{stroke,fill}%
}%
\begin{pgfscope}%
\pgfsys@transformshift{0.800000in}{0.696000in}%
\pgfsys@useobject{currentmarker}{}%
\end{pgfscope}%
\end{pgfscope}%
\begin{pgfscope}%
\definecolor{textcolor}{rgb}{0.000000,0.000000,0.000000}%
\pgfsetstrokecolor{textcolor}%
\pgfsetfillcolor{textcolor}%
\pgftext[x=0.525308in, y=0.644900in, left, base]{\color{textcolor}\rmfamily\fontsize{10.000000}{12.000000}\selectfont \(\displaystyle {0.0}\)}%
\end{pgfscope}%
\begin{pgfscope}%
\pgfpathrectangle{\pgfqpoint{0.800000in}{0.528000in}}{\pgfqpoint{4.960000in}{3.696000in}}%
\pgfusepath{clip}%
\pgfsetrectcap%
\pgfsetroundjoin%
\pgfsetlinewidth{0.803000pt}%
\definecolor{currentstroke}{rgb}{0.690196,0.690196,0.690196}%
\pgfsetstrokecolor{currentstroke}%
\pgfsetdash{}{0pt}%
\pgfpathmoveto{\pgfqpoint{0.800000in}{1.256000in}}%
\pgfpathlineto{\pgfqpoint{5.760000in}{1.256000in}}%
\pgfusepath{stroke}%
\end{pgfscope}%
\begin{pgfscope}%
\pgfsetbuttcap%
\pgfsetroundjoin%
\definecolor{currentfill}{rgb}{0.000000,0.000000,0.000000}%
\pgfsetfillcolor{currentfill}%
\pgfsetlinewidth{0.803000pt}%
\definecolor{currentstroke}{rgb}{0.000000,0.000000,0.000000}%
\pgfsetstrokecolor{currentstroke}%
\pgfsetdash{}{0pt}%
\pgfsys@defobject{currentmarker}{\pgfqpoint{-0.048611in}{0.000000in}}{\pgfqpoint{-0.000000in}{0.000000in}}{%
\pgfpathmoveto{\pgfqpoint{-0.000000in}{0.000000in}}%
\pgfpathlineto{\pgfqpoint{-0.048611in}{0.000000in}}%
\pgfusepath{stroke,fill}%
}%
\begin{pgfscope}%
\pgfsys@transformshift{0.800000in}{1.256000in}%
\pgfsys@useobject{currentmarker}{}%
\end{pgfscope}%
\end{pgfscope}%
\begin{pgfscope}%
\definecolor{textcolor}{rgb}{0.000000,0.000000,0.000000}%
\pgfsetstrokecolor{textcolor}%
\pgfsetfillcolor{textcolor}%
\pgftext[x=0.525308in, y=1.204900in, left, base]{\color{textcolor}\rmfamily\fontsize{10.000000}{12.000000}\selectfont \(\displaystyle {0.2}\)}%
\end{pgfscope}%
\begin{pgfscope}%
\pgfpathrectangle{\pgfqpoint{0.800000in}{0.528000in}}{\pgfqpoint{4.960000in}{3.696000in}}%
\pgfusepath{clip}%
\pgfsetrectcap%
\pgfsetroundjoin%
\pgfsetlinewidth{0.803000pt}%
\definecolor{currentstroke}{rgb}{0.690196,0.690196,0.690196}%
\pgfsetstrokecolor{currentstroke}%
\pgfsetdash{}{0pt}%
\pgfpathmoveto{\pgfqpoint{0.800000in}{1.816000in}}%
\pgfpathlineto{\pgfqpoint{5.760000in}{1.816000in}}%
\pgfusepath{stroke}%
\end{pgfscope}%
\begin{pgfscope}%
\pgfsetbuttcap%
\pgfsetroundjoin%
\definecolor{currentfill}{rgb}{0.000000,0.000000,0.000000}%
\pgfsetfillcolor{currentfill}%
\pgfsetlinewidth{0.803000pt}%
\definecolor{currentstroke}{rgb}{0.000000,0.000000,0.000000}%
\pgfsetstrokecolor{currentstroke}%
\pgfsetdash{}{0pt}%
\pgfsys@defobject{currentmarker}{\pgfqpoint{-0.048611in}{0.000000in}}{\pgfqpoint{-0.000000in}{0.000000in}}{%
\pgfpathmoveto{\pgfqpoint{-0.000000in}{0.000000in}}%
\pgfpathlineto{\pgfqpoint{-0.048611in}{0.000000in}}%
\pgfusepath{stroke,fill}%
}%
\begin{pgfscope}%
\pgfsys@transformshift{0.800000in}{1.816000in}%
\pgfsys@useobject{currentmarker}{}%
\end{pgfscope}%
\end{pgfscope}%
\begin{pgfscope}%
\definecolor{textcolor}{rgb}{0.000000,0.000000,0.000000}%
\pgfsetstrokecolor{textcolor}%
\pgfsetfillcolor{textcolor}%
\pgftext[x=0.525308in, y=1.764900in, left, base]{\color{textcolor}\rmfamily\fontsize{10.000000}{12.000000}\selectfont \(\displaystyle {0.4}\)}%
\end{pgfscope}%
\begin{pgfscope}%
\pgfpathrectangle{\pgfqpoint{0.800000in}{0.528000in}}{\pgfqpoint{4.960000in}{3.696000in}}%
\pgfusepath{clip}%
\pgfsetrectcap%
\pgfsetroundjoin%
\pgfsetlinewidth{0.803000pt}%
\definecolor{currentstroke}{rgb}{0.690196,0.690196,0.690196}%
\pgfsetstrokecolor{currentstroke}%
\pgfsetdash{}{0pt}%
\pgfpathmoveto{\pgfqpoint{0.800000in}{2.376000in}}%
\pgfpathlineto{\pgfqpoint{5.760000in}{2.376000in}}%
\pgfusepath{stroke}%
\end{pgfscope}%
\begin{pgfscope}%
\pgfsetbuttcap%
\pgfsetroundjoin%
\definecolor{currentfill}{rgb}{0.000000,0.000000,0.000000}%
\pgfsetfillcolor{currentfill}%
\pgfsetlinewidth{0.803000pt}%
\definecolor{currentstroke}{rgb}{0.000000,0.000000,0.000000}%
\pgfsetstrokecolor{currentstroke}%
\pgfsetdash{}{0pt}%
\pgfsys@defobject{currentmarker}{\pgfqpoint{-0.048611in}{0.000000in}}{\pgfqpoint{-0.000000in}{0.000000in}}{%
\pgfpathmoveto{\pgfqpoint{-0.000000in}{0.000000in}}%
\pgfpathlineto{\pgfqpoint{-0.048611in}{0.000000in}}%
\pgfusepath{stroke,fill}%
}%
\begin{pgfscope}%
\pgfsys@transformshift{0.800000in}{2.376000in}%
\pgfsys@useobject{currentmarker}{}%
\end{pgfscope}%
\end{pgfscope}%
\begin{pgfscope}%
\definecolor{textcolor}{rgb}{0.000000,0.000000,0.000000}%
\pgfsetstrokecolor{textcolor}%
\pgfsetfillcolor{textcolor}%
\pgftext[x=0.525308in, y=2.324900in, left, base]{\color{textcolor}\rmfamily\fontsize{10.000000}{12.000000}\selectfont \(\displaystyle {0.6}\)}%
\end{pgfscope}%
\begin{pgfscope}%
\pgfpathrectangle{\pgfqpoint{0.800000in}{0.528000in}}{\pgfqpoint{4.960000in}{3.696000in}}%
\pgfusepath{clip}%
\pgfsetrectcap%
\pgfsetroundjoin%
\pgfsetlinewidth{0.803000pt}%
\definecolor{currentstroke}{rgb}{0.690196,0.690196,0.690196}%
\pgfsetstrokecolor{currentstroke}%
\pgfsetdash{}{0pt}%
\pgfpathmoveto{\pgfqpoint{0.800000in}{2.936000in}}%
\pgfpathlineto{\pgfqpoint{5.760000in}{2.936000in}}%
\pgfusepath{stroke}%
\end{pgfscope}%
\begin{pgfscope}%
\pgfsetbuttcap%
\pgfsetroundjoin%
\definecolor{currentfill}{rgb}{0.000000,0.000000,0.000000}%
\pgfsetfillcolor{currentfill}%
\pgfsetlinewidth{0.803000pt}%
\definecolor{currentstroke}{rgb}{0.000000,0.000000,0.000000}%
\pgfsetstrokecolor{currentstroke}%
\pgfsetdash{}{0pt}%
\pgfsys@defobject{currentmarker}{\pgfqpoint{-0.048611in}{0.000000in}}{\pgfqpoint{-0.000000in}{0.000000in}}{%
\pgfpathmoveto{\pgfqpoint{-0.000000in}{0.000000in}}%
\pgfpathlineto{\pgfqpoint{-0.048611in}{0.000000in}}%
\pgfusepath{stroke,fill}%
}%
\begin{pgfscope}%
\pgfsys@transformshift{0.800000in}{2.936000in}%
\pgfsys@useobject{currentmarker}{}%
\end{pgfscope}%
\end{pgfscope}%
\begin{pgfscope}%
\definecolor{textcolor}{rgb}{0.000000,0.000000,0.000000}%
\pgfsetstrokecolor{textcolor}%
\pgfsetfillcolor{textcolor}%
\pgftext[x=0.525308in, y=2.884900in, left, base]{\color{textcolor}\rmfamily\fontsize{10.000000}{12.000000}\selectfont \(\displaystyle {0.8}\)}%
\end{pgfscope}%
\begin{pgfscope}%
\pgfpathrectangle{\pgfqpoint{0.800000in}{0.528000in}}{\pgfqpoint{4.960000in}{3.696000in}}%
\pgfusepath{clip}%
\pgfsetrectcap%
\pgfsetroundjoin%
\pgfsetlinewidth{0.803000pt}%
\definecolor{currentstroke}{rgb}{0.690196,0.690196,0.690196}%
\pgfsetstrokecolor{currentstroke}%
\pgfsetdash{}{0pt}%
\pgfpathmoveto{\pgfqpoint{0.800000in}{3.496000in}}%
\pgfpathlineto{\pgfqpoint{5.760000in}{3.496000in}}%
\pgfusepath{stroke}%
\end{pgfscope}%
\begin{pgfscope}%
\pgfsetbuttcap%
\pgfsetroundjoin%
\definecolor{currentfill}{rgb}{0.000000,0.000000,0.000000}%
\pgfsetfillcolor{currentfill}%
\pgfsetlinewidth{0.803000pt}%
\definecolor{currentstroke}{rgb}{0.000000,0.000000,0.000000}%
\pgfsetstrokecolor{currentstroke}%
\pgfsetdash{}{0pt}%
\pgfsys@defobject{currentmarker}{\pgfqpoint{-0.048611in}{0.000000in}}{\pgfqpoint{-0.000000in}{0.000000in}}{%
\pgfpathmoveto{\pgfqpoint{-0.000000in}{0.000000in}}%
\pgfpathlineto{\pgfqpoint{-0.048611in}{0.000000in}}%
\pgfusepath{stroke,fill}%
}%
\begin{pgfscope}%
\pgfsys@transformshift{0.800000in}{3.496000in}%
\pgfsys@useobject{currentmarker}{}%
\end{pgfscope}%
\end{pgfscope}%
\begin{pgfscope}%
\definecolor{textcolor}{rgb}{0.000000,0.000000,0.000000}%
\pgfsetstrokecolor{textcolor}%
\pgfsetfillcolor{textcolor}%
\pgftext[x=0.525308in, y=3.444900in, left, base]{\color{textcolor}\rmfamily\fontsize{10.000000}{12.000000}\selectfont \(\displaystyle {1.0}\)}%
\end{pgfscope}%
\begin{pgfscope}%
\pgfpathrectangle{\pgfqpoint{0.800000in}{0.528000in}}{\pgfqpoint{4.960000in}{3.696000in}}%
\pgfusepath{clip}%
\pgfsetrectcap%
\pgfsetroundjoin%
\pgfsetlinewidth{0.803000pt}%
\definecolor{currentstroke}{rgb}{0.690196,0.690196,0.690196}%
\pgfsetstrokecolor{currentstroke}%
\pgfsetdash{}{0pt}%
\pgfpathmoveto{\pgfqpoint{0.800000in}{4.056000in}}%
\pgfpathlineto{\pgfqpoint{5.760000in}{4.056000in}}%
\pgfusepath{stroke}%
\end{pgfscope}%
\begin{pgfscope}%
\pgfsetbuttcap%
\pgfsetroundjoin%
\definecolor{currentfill}{rgb}{0.000000,0.000000,0.000000}%
\pgfsetfillcolor{currentfill}%
\pgfsetlinewidth{0.803000pt}%
\definecolor{currentstroke}{rgb}{0.000000,0.000000,0.000000}%
\pgfsetstrokecolor{currentstroke}%
\pgfsetdash{}{0pt}%
\pgfsys@defobject{currentmarker}{\pgfqpoint{-0.048611in}{0.000000in}}{\pgfqpoint{-0.000000in}{0.000000in}}{%
\pgfpathmoveto{\pgfqpoint{-0.000000in}{0.000000in}}%
\pgfpathlineto{\pgfqpoint{-0.048611in}{0.000000in}}%
\pgfusepath{stroke,fill}%
}%
\begin{pgfscope}%
\pgfsys@transformshift{0.800000in}{4.056000in}%
\pgfsys@useobject{currentmarker}{}%
\end{pgfscope}%
\end{pgfscope}%
\begin{pgfscope}%
\definecolor{textcolor}{rgb}{0.000000,0.000000,0.000000}%
\pgfsetstrokecolor{textcolor}%
\pgfsetfillcolor{textcolor}%
\pgftext[x=0.525308in, y=4.004900in, left, base]{\color{textcolor}\rmfamily\fontsize{10.000000}{12.000000}\selectfont \(\displaystyle {1.2}\)}%
\end{pgfscope}%
\begin{pgfscope}%
\pgfpathrectangle{\pgfqpoint{0.800000in}{0.528000in}}{\pgfqpoint{4.960000in}{3.696000in}}%
\pgfusepath{clip}%
\pgfsetrectcap%
\pgfsetroundjoin%
\pgfsetlinewidth{1.505625pt}%
\definecolor{currentstroke}{rgb}{0.121569,0.466667,0.705882}%
\pgfsetstrokecolor{currentstroke}%
\pgfsetdash{}{0pt}%
\pgfpathmoveto{\pgfqpoint{1.025455in}{3.237068in}}%
\pgfpathlineto{\pgfqpoint{1.065291in}{3.235975in}}%
\pgfpathlineto{\pgfqpoint{1.107383in}{3.232579in}}%
\pgfpathlineto{\pgfqpoint{1.155488in}{3.226399in}}%
\pgfpathlineto{\pgfqpoint{1.229149in}{3.214340in}}%
\pgfpathlineto{\pgfqpoint{1.303562in}{3.202857in}}%
\pgfpathlineto{\pgfqpoint{1.350163in}{3.197920in}}%
\pgfpathlineto{\pgfqpoint{1.391503in}{3.195738in}}%
\pgfpathlineto{\pgfqpoint{1.431340in}{3.195832in}}%
\pgfpathlineto{\pgfqpoint{1.472681in}{3.198184in}}%
\pgfpathlineto{\pgfqpoint{1.518531in}{3.203114in}}%
\pgfpathlineto{\pgfqpoint{1.577910in}{3.211945in}}%
\pgfpathlineto{\pgfqpoint{1.690656in}{3.229072in}}%
\pgfpathlineto{\pgfqpoint{1.738010in}{3.233652in}}%
\pgfpathlineto{\pgfqpoint{1.776343in}{3.235455in}}%
\pgfpathlineto{\pgfqpoint{1.777847in}{0.696000in}}%
\pgfpathlineto{\pgfqpoint{1.865789in}{0.696000in}}%
\pgfpathlineto{\pgfqpoint{1.867292in}{3.781272in}}%
\pgfpathlineto{\pgfqpoint{1.878566in}{3.867567in}}%
\pgfpathlineto{\pgfqpoint{1.889089in}{3.931686in}}%
\pgfpathlineto{\pgfqpoint{1.898861in}{3.977907in}}%
\pgfpathlineto{\pgfqpoint{1.907880in}{4.010049in}}%
\pgfpathlineto{\pgfqpoint{1.916148in}{4.031392in}}%
\pgfpathlineto{\pgfqpoint{1.923665in}{4.044659in}}%
\pgfpathlineto{\pgfqpoint{1.930430in}{4.052064in}}%
\pgfpathlineto{\pgfqpoint{1.936443in}{4.055369in}}%
\pgfpathlineto{\pgfqpoint{1.941704in}{4.055954in}}%
\pgfpathlineto{\pgfqpoint{1.947717in}{4.054226in}}%
\pgfpathlineto{\pgfqpoint{1.954482in}{4.049536in}}%
\pgfpathlineto{\pgfqpoint{1.961999in}{4.041332in}}%
\pgfpathlineto{\pgfqpoint{1.971018in}{4.027955in}}%
\pgfpathlineto{\pgfqpoint{1.982293in}{4.006883in}}%
\pgfpathlineto{\pgfqpoint{1.996574in}{3.975143in}}%
\pgfpathlineto{\pgfqpoint{2.018372in}{3.920537in}}%
\pgfpathlineto{\pgfqpoint{2.064222in}{3.804692in}}%
\pgfpathlineto{\pgfqpoint{2.084516in}{3.760101in}}%
\pgfpathlineto{\pgfqpoint{2.101804in}{3.727303in}}%
\pgfpathlineto{\pgfqpoint{2.117588in}{3.702086in}}%
\pgfpathlineto{\pgfqpoint{2.131869in}{3.683426in}}%
\pgfpathlineto{\pgfqpoint{2.144647in}{3.670204in}}%
\pgfpathlineto{\pgfqpoint{2.156673in}{3.660819in}}%
\pgfpathlineto{\pgfqpoint{2.167948in}{3.654748in}}%
\pgfpathlineto{\pgfqpoint{2.178471in}{3.651476in}}%
\pgfpathlineto{\pgfqpoint{2.188994in}{3.650522in}}%
\pgfpathlineto{\pgfqpoint{2.199517in}{3.651883in}}%
\pgfpathlineto{\pgfqpoint{2.210040in}{3.655552in}}%
\pgfpathlineto{\pgfqpoint{2.221314in}{3.662027in}}%
\pgfpathlineto{\pgfqpoint{2.232589in}{3.671106in}}%
\pgfpathlineto{\pgfqpoint{2.244615in}{3.683616in}}%
\pgfpathlineto{\pgfqpoint{2.258145in}{3.701093in}}%
\pgfpathlineto{\pgfqpoint{2.272426in}{3.723303in}}%
\pgfpathlineto{\pgfqpoint{2.288210in}{3.752077in}}%
\pgfpathlineto{\pgfqpoint{2.306250in}{3.789846in}}%
\pgfpathlineto{\pgfqpoint{2.328047in}{3.841107in}}%
\pgfpathlineto{\pgfqpoint{2.397198in}{4.008926in}}%
\pgfpathlineto{\pgfqpoint{2.409225in}{4.030420in}}%
\pgfpathlineto{\pgfqpoint{2.418996in}{4.043841in}}%
\pgfpathlineto{\pgfqpoint{2.427264in}{4.051660in}}%
\pgfpathlineto{\pgfqpoint{2.434029in}{4.055220in}}%
\pgfpathlineto{\pgfqpoint{2.440042in}{4.055951in}}%
\pgfpathlineto{\pgfqpoint{2.445303in}{4.054515in}}%
\pgfpathlineto{\pgfqpoint{2.450565in}{4.050971in}}%
\pgfpathlineto{\pgfqpoint{2.456578in}{4.044123in}}%
\pgfpathlineto{\pgfqpoint{2.463343in}{4.032556in}}%
\pgfpathlineto{\pgfqpoint{2.470859in}{4.014493in}}%
\pgfpathlineto{\pgfqpoint{2.478375in}{3.990487in}}%
\pgfpathlineto{\pgfqpoint{2.486644in}{3.956683in}}%
\pgfpathlineto{\pgfqpoint{2.495663in}{3.910347in}}%
\pgfpathlineto{\pgfqpoint{2.505435in}{3.848330in}}%
\pgfpathlineto{\pgfqpoint{2.515958in}{3.767142in}}%
\pgfpathlineto{\pgfqpoint{2.527984in}{3.655566in}}%
\pgfpathlineto{\pgfqpoint{2.540762in}{3.515100in}}%
\pgfpathlineto{\pgfqpoint{2.555043in}{3.332630in}}%
\pgfpathlineto{\pgfqpoint{2.571579in}{3.091690in}}%
\pgfpathlineto{\pgfqpoint{2.593376in}{2.738091in}}%
\pgfpathlineto{\pgfqpoint{2.648998in}{1.818014in}}%
\pgfpathlineto{\pgfqpoint{2.665534in}{1.590203in}}%
\pgfpathlineto{\pgfqpoint{2.679063in}{1.433225in}}%
\pgfpathlineto{\pgfqpoint{2.690338in}{1.325995in}}%
\pgfpathlineto{\pgfqpoint{2.700109in}{1.252004in}}%
\pgfpathlineto{\pgfqpoint{2.709129in}{1.200177in}}%
\pgfpathlineto{\pgfqpoint{2.716646in}{1.169469in}}%
\pgfpathlineto{\pgfqpoint{2.722659in}{1.153229in}}%
\pgfpathlineto{\pgfqpoint{2.727920in}{1.145150in}}%
\pgfpathlineto{\pgfqpoint{2.731678in}{1.142896in}}%
\pgfpathlineto{\pgfqpoint{2.734685in}{1.143206in}}%
\pgfpathlineto{\pgfqpoint{2.738443in}{1.146232in}}%
\pgfpathlineto{\pgfqpoint{2.742953in}{1.153723in}}%
\pgfpathlineto{\pgfqpoint{2.748214in}{1.167753in}}%
\pgfpathlineto{\pgfqpoint{2.754228in}{1.190698in}}%
\pgfpathlineto{\pgfqpoint{2.761744in}{1.229571in}}%
\pgfpathlineto{\pgfqpoint{2.770012in}{1.285067in}}%
\pgfpathlineto{\pgfqpoint{2.779783in}{1.367142in}}%
\pgfpathlineto{\pgfqpoint{2.791058in}{1.482631in}}%
\pgfpathlineto{\pgfqpoint{2.803836in}{1.637792in}}%
\pgfpathlineto{\pgfqpoint{2.818869in}{1.848219in}}%
\pgfpathlineto{\pgfqpoint{2.838411in}{2.154735in}}%
\pgfpathlineto{\pgfqpoint{2.904556in}{3.220103in}}%
\pgfpathlineto{\pgfqpoint{2.921092in}{3.437953in}}%
\pgfpathlineto{\pgfqpoint{2.935373in}{3.599264in}}%
\pgfpathlineto{\pgfqpoint{2.948902in}{3.727633in}}%
\pgfpathlineto{\pgfqpoint{2.960929in}{3.821705in}}%
\pgfpathlineto{\pgfqpoint{2.972203in}{3.893358in}}%
\pgfpathlineto{\pgfqpoint{2.982726in}{3.946641in}}%
\pgfpathlineto{\pgfqpoint{2.992498in}{3.985260in}}%
\pgfpathlineto{\pgfqpoint{3.001517in}{4.012443in}}%
\pgfpathlineto{\pgfqpoint{3.009785in}{4.030907in}}%
\pgfpathlineto{\pgfqpoint{3.017302in}{4.042871in}}%
\pgfpathlineto{\pgfqpoint{3.024818in}{4.050730in}}%
\pgfpathlineto{\pgfqpoint{3.031583in}{4.054674in}}%
\pgfpathlineto{\pgfqpoint{3.038348in}{4.055997in}}%
\pgfpathlineto{\pgfqpoint{3.045112in}{4.055017in}}%
\pgfpathlineto{\pgfqpoint{3.052629in}{4.051600in}}%
\pgfpathlineto{\pgfqpoint{3.061648in}{4.044805in}}%
\pgfpathlineto{\pgfqpoint{3.072171in}{4.033968in}}%
\pgfpathlineto{\pgfqpoint{3.086453in}{4.015809in}}%
\pgfpathlineto{\pgfqpoint{3.114263in}{3.975777in}}%
\pgfpathlineto{\pgfqpoint{3.137564in}{3.944088in}}%
\pgfpathlineto{\pgfqpoint{3.153349in}{3.926122in}}%
\pgfpathlineto{\pgfqpoint{3.166878in}{3.913853in}}%
\pgfpathlineto{\pgfqpoint{3.178904in}{3.905742in}}%
\pgfpathlineto{\pgfqpoint{3.190179in}{3.900718in}}%
\pgfpathlineto{\pgfqpoint{3.200702in}{3.898376in}}%
\pgfpathlineto{\pgfqpoint{3.211225in}{3.898341in}}%
\pgfpathlineto{\pgfqpoint{3.221748in}{3.900609in}}%
\pgfpathlineto{\pgfqpoint{3.232271in}{3.905134in}}%
\pgfpathlineto{\pgfqpoint{3.243545in}{3.912386in}}%
\pgfpathlineto{\pgfqpoint{3.256323in}{3.923402in}}%
\pgfpathlineto{\pgfqpoint{3.270604in}{3.938820in}}%
\pgfpathlineto{\pgfqpoint{3.287892in}{3.960952in}}%
\pgfpathlineto{\pgfqpoint{3.323219in}{4.011132in}}%
\pgfpathlineto{\pgfqpoint{3.341259in}{4.034108in}}%
\pgfpathlineto{\pgfqpoint{3.353285in}{4.046058in}}%
\pgfpathlineto{\pgfqpoint{3.363056in}{4.052737in}}%
\pgfpathlineto{\pgfqpoint{3.371324in}{4.055650in}}%
\pgfpathlineto{\pgfqpoint{3.378089in}{4.055794in}}%
\pgfpathlineto{\pgfqpoint{3.384854in}{4.053612in}}%
\pgfpathlineto{\pgfqpoint{3.391619in}{4.048806in}}%
\pgfpathlineto{\pgfqpoint{3.398383in}{4.041068in}}%
\pgfpathlineto{\pgfqpoint{3.405900in}{4.028655in}}%
\pgfpathlineto{\pgfqpoint{3.413416in}{4.011814in}}%
\pgfpathlineto{\pgfqpoint{3.421684in}{3.987671in}}%
\pgfpathlineto{\pgfqpoint{3.430704in}{3.954008in}}%
\pgfpathlineto{\pgfqpoint{3.440475in}{3.908183in}}%
\pgfpathlineto{\pgfqpoint{3.450998in}{3.847146in}}%
\pgfpathlineto{\pgfqpoint{3.462273in}{3.767523in}}%
\pgfpathlineto{\pgfqpoint{3.474299in}{3.665785in}}%
\pgfpathlineto{\pgfqpoint{3.487077in}{3.538569in}}%
\pgfpathlineto{\pgfqpoint{3.501358in}{3.373911in}}%
\pgfpathlineto{\pgfqpoint{3.517894in}{3.156618in}}%
\pgfpathlineto{\pgfqpoint{3.538940in}{2.848093in}}%
\pgfpathlineto{\pgfqpoint{3.603581in}{1.875668in}}%
\pgfpathlineto{\pgfqpoint{3.619366in}{1.683228in}}%
\pgfpathlineto{\pgfqpoint{3.632144in}{1.552278in}}%
\pgfpathlineto{\pgfqpoint{3.643418in}{1.457935in}}%
\pgfpathlineto{\pgfqpoint{3.653189in}{1.393629in}}%
\pgfpathlineto{\pgfqpoint{3.661458in}{1.352535in}}%
\pgfpathlineto{\pgfqpoint{3.668974in}{1.326104in}}%
\pgfpathlineto{\pgfqpoint{3.674987in}{1.312585in}}%
\pgfpathlineto{\pgfqpoint{3.679497in}{1.306934in}}%
\pgfpathlineto{\pgfqpoint{3.683255in}{1.305175in}}%
\pgfpathlineto{\pgfqpoint{3.687013in}{1.306099in}}%
\pgfpathlineto{\pgfqpoint{3.690771in}{1.309702in}}%
\pgfpathlineto{\pgfqpoint{3.695281in}{1.317551in}}%
\pgfpathlineto{\pgfqpoint{3.700543in}{1.331538in}}%
\pgfpathlineto{\pgfqpoint{3.706556in}{1.353818in}}%
\pgfpathlineto{\pgfqpoint{3.714072in}{1.390942in}}%
\pgfpathlineto{\pgfqpoint{3.722340in}{1.443337in}}%
\pgfpathlineto{\pgfqpoint{3.732112in}{1.520181in}}%
\pgfpathlineto{\pgfqpoint{3.743386in}{1.627586in}}%
\pgfpathlineto{\pgfqpoint{3.756164in}{1.771093in}}%
\pgfpathlineto{\pgfqpoint{3.771949in}{1.975092in}}%
\pgfpathlineto{\pgfqpoint{3.792243in}{2.268463in}}%
\pgfpathlineto{\pgfqpoint{3.855381in}{3.203576in}}%
\pgfpathlineto{\pgfqpoint{3.872668in}{3.417212in}}%
\pgfpathlineto{\pgfqpoint{3.887701in}{3.577463in}}%
\pgfpathlineto{\pgfqpoint{3.901231in}{3.699869in}}%
\pgfpathlineto{\pgfqpoint{3.914009in}{3.796350in}}%
\pgfpathlineto{\pgfqpoint{3.926035in}{3.870740in}}%
\pgfpathlineto{\pgfqpoint{3.937310in}{3.926898in}}%
\pgfpathlineto{\pgfqpoint{3.947833in}{3.968405in}}%
\pgfpathlineto{\pgfqpoint{3.957604in}{3.998410in}}%
\pgfpathlineto{\pgfqpoint{3.966624in}{4.019572in}}%
\pgfpathlineto{\pgfqpoint{3.975643in}{4.035179in}}%
\pgfpathlineto{\pgfqpoint{3.983911in}{4.045202in}}%
\pgfpathlineto{\pgfqpoint{3.992179in}{4.051674in}}%
\pgfpathlineto{\pgfqpoint{3.999696in}{4.054907in}}%
\pgfpathlineto{\pgfqpoint{4.007964in}{4.056000in}}%
\pgfpathlineto{\pgfqpoint{4.016983in}{4.054776in}}%
\pgfpathlineto{\pgfqpoint{4.027506in}{4.050884in}}%
\pgfpathlineto{\pgfqpoint{4.041036in}{4.043210in}}%
\pgfpathlineto{\pgfqpoint{4.065088in}{4.026351in}}%
\pgfpathlineto{\pgfqpoint{4.087638in}{4.011583in}}%
\pgfpathlineto{\pgfqpoint{4.103422in}{4.003890in}}%
\pgfpathlineto{\pgfqpoint{4.116952in}{3.999744in}}%
\pgfpathlineto{\pgfqpoint{4.129730in}{3.998178in}}%
\pgfpathlineto{\pgfqpoint{4.141756in}{3.998872in}}%
\pgfpathlineto{\pgfqpoint{4.154534in}{4.001866in}}%
\pgfpathlineto{\pgfqpoint{4.168063in}{4.007367in}}%
\pgfpathlineto{\pgfqpoint{4.183848in}{4.016285in}}%
\pgfpathlineto{\pgfqpoint{4.207900in}{4.032873in}}%
\pgfpathlineto{\pgfqpoint{4.231201in}{4.048085in}}%
\pgfpathlineto{\pgfqpoint{4.243979in}{4.053791in}}%
\pgfpathlineto{\pgfqpoint{4.253750in}{4.055871in}}%
\pgfpathlineto{\pgfqpoint{4.262018in}{4.055552in}}%
\pgfpathlineto{\pgfqpoint{4.269535in}{4.053229in}}%
\pgfpathlineto{\pgfqpoint{4.277051in}{4.048619in}}%
\pgfpathlineto{\pgfqpoint{4.284567in}{4.041371in}}%
\pgfpathlineto{\pgfqpoint{4.292084in}{4.031121in}}%
\pgfpathlineto{\pgfqpoint{4.300352in}{4.015935in}}%
\pgfpathlineto{\pgfqpoint{4.309372in}{3.994133in}}%
\pgfpathlineto{\pgfqpoint{4.318391in}{3.966252in}}%
\pgfpathlineto{\pgfqpoint{4.328163in}{3.928484in}}%
\pgfpathlineto{\pgfqpoint{4.338686in}{3.878225in}}%
\pgfpathlineto{\pgfqpoint{4.349960in}{3.812506in}}%
\pgfpathlineto{\pgfqpoint{4.361986in}{3.728104in}}%
\pgfpathlineto{\pgfqpoint{4.374764in}{3.621747in}}%
\pgfpathlineto{\pgfqpoint{4.389045in}{3.482627in}}%
\pgfpathlineto{\pgfqpoint{4.404830in}{3.305382in}}%
\pgfpathlineto{\pgfqpoint{4.423621in}{3.066804in}}%
\pgfpathlineto{\pgfqpoint{4.449177in}{2.710119in}}%
\pgfpathlineto{\pgfqpoint{4.492020in}{2.110931in}}%
\pgfpathlineto{\pgfqpoint{4.510060in}{1.891763in}}%
\pgfpathlineto{\pgfqpoint{4.524341in}{1.743475in}}%
\pgfpathlineto{\pgfqpoint{4.536367in}{1.639573in}}%
\pgfpathlineto{\pgfqpoint{4.546890in}{1.566229in}}%
\pgfpathlineto{\pgfqpoint{4.555910in}{1.517331in}}%
\pgfpathlineto{\pgfqpoint{4.564178in}{1.484327in}}%
\pgfpathlineto{\pgfqpoint{4.570942in}{1.465950in}}%
\pgfpathlineto{\pgfqpoint{4.576204in}{1.457101in}}%
\pgfpathlineto{\pgfqpoint{4.580714in}{1.453334in}}%
\pgfpathlineto{\pgfqpoint{4.584472in}{1.452892in}}%
\pgfpathlineto{\pgfqpoint{4.588230in}{1.454903in}}%
\pgfpathlineto{\pgfqpoint{4.592740in}{1.460544in}}%
\pgfpathlineto{\pgfqpoint{4.598001in}{1.471553in}}%
\pgfpathlineto{\pgfqpoint{4.604015in}{1.489914in}}%
\pgfpathlineto{\pgfqpoint{4.610779in}{1.517822in}}%
\pgfpathlineto{\pgfqpoint{4.619047in}{1.562091in}}%
\pgfpathlineto{\pgfqpoint{4.628819in}{1.628237in}}%
\pgfpathlineto{\pgfqpoint{4.639342in}{1.715205in}}%
\pgfpathlineto{\pgfqpoint{4.652120in}{1.840704in}}%
\pgfpathlineto{\pgfqpoint{4.667152in}{2.012260in}}%
\pgfpathlineto{\pgfqpoint{4.685943in}{2.254263in}}%
\pgfpathlineto{\pgfqpoint{4.719016in}{2.715823in}}%
\pgfpathlineto{\pgfqpoint{4.748330in}{3.112269in}}%
\pgfpathlineto{\pgfqpoint{4.767872in}{3.347287in}}%
\pgfpathlineto{\pgfqpoint{4.784408in}{3.520221in}}%
\pgfpathlineto{\pgfqpoint{4.799441in}{3.654341in}}%
\pgfpathlineto{\pgfqpoint{4.812971in}{3.755746in}}%
\pgfpathlineto{\pgfqpoint{4.825749in}{3.835122in}}%
\pgfpathlineto{\pgfqpoint{4.837775in}{3.896093in}}%
\pgfpathlineto{\pgfqpoint{4.849049in}{3.942109in}}%
\pgfpathlineto{\pgfqpoint{4.859572in}{3.976261in}}%
\pgfpathlineto{\pgfqpoint{4.870095in}{4.002857in}}%
\pgfpathlineto{\pgfqpoint{4.879867in}{4.021624in}}%
\pgfpathlineto{\pgfqpoint{4.888886in}{4.034557in}}%
\pgfpathlineto{\pgfqpoint{4.897906in}{4.043882in}}%
\pgfpathlineto{\pgfqpoint{4.906926in}{4.050165in}}%
\pgfpathlineto{\pgfqpoint{4.916697in}{4.054158in}}%
\pgfpathlineto{\pgfqpoint{4.927220in}{4.055891in}}%
\pgfpathlineto{\pgfqpoint{4.939246in}{4.055507in}}%
\pgfpathlineto{\pgfqpoint{4.955782in}{4.052491in}}%
\pgfpathlineto{\pgfqpoint{5.002384in}{4.042709in}}%
\pgfpathlineto{\pgfqpoint{5.019672in}{4.042036in}}%
\pgfpathlineto{\pgfqpoint{5.036959in}{4.043583in}}%
\pgfpathlineto{\pgfqpoint{5.058005in}{4.047852in}}%
\pgfpathlineto{\pgfqpoint{5.094084in}{4.055569in}}%
\pgfpathlineto{\pgfqpoint{5.106862in}{4.055806in}}%
\pgfpathlineto{\pgfqpoint{5.117385in}{4.053843in}}%
\pgfpathlineto{\pgfqpoint{5.126405in}{4.050050in}}%
\pgfpathlineto{\pgfqpoint{5.135424in}{4.043810in}}%
\pgfpathlineto{\pgfqpoint{5.144444in}{4.034617in}}%
\pgfpathlineto{\pgfqpoint{5.153464in}{4.021941in}}%
\pgfpathlineto{\pgfqpoint{5.162483in}{4.005234in}}%
\pgfpathlineto{\pgfqpoint{5.172255in}{3.981941in}}%
\pgfpathlineto{\pgfqpoint{5.182026in}{3.952563in}}%
\pgfpathlineto{\pgfqpoint{5.192549in}{3.913347in}}%
\pgfpathlineto{\pgfqpoint{5.203072in}{3.865514in}}%
\pgfpathlineto{\pgfqpoint{5.214347in}{3.803938in}}%
\pgfpathlineto{\pgfqpoint{5.226373in}{3.725785in}}%
\pgfpathlineto{\pgfqpoint{5.239151in}{3.628149in}}%
\pgfpathlineto{\pgfqpoint{5.253432in}{3.501197in}}%
\pgfpathlineto{\pgfqpoint{5.269216in}{3.340007in}}%
\pgfpathlineto{\pgfqpoint{5.287256in}{3.132287in}}%
\pgfpathlineto{\pgfqpoint{5.311308in}{2.827394in}}%
\pgfpathlineto{\pgfqpoint{5.362420in}{2.171719in}}%
\pgfpathlineto{\pgfqpoint{5.379708in}{1.981496in}}%
\pgfpathlineto{\pgfqpoint{5.393989in}{1.847223in}}%
\pgfpathlineto{\pgfqpoint{5.406015in}{1.753428in}}%
\pgfpathlineto{\pgfqpoint{5.416538in}{1.687469in}}%
\pgfpathlineto{\pgfqpoint{5.425558in}{1.643729in}}%
\pgfpathlineto{\pgfqpoint{5.433074in}{1.616679in}}%
\pgfpathlineto{\pgfqpoint{5.439839in}{1.599825in}}%
\pgfpathlineto{\pgfqpoint{5.445100in}{1.591689in}}%
\pgfpathlineto{\pgfqpoint{5.449610in}{1.588204in}}%
\pgfpathlineto{\pgfqpoint{5.453368in}{1.587765in}}%
\pgfpathlineto{\pgfqpoint{5.457126in}{1.589567in}}%
\pgfpathlineto{\pgfqpoint{5.461636in}{1.594680in}}%
\pgfpathlineto{\pgfqpoint{5.466898in}{1.604691in}}%
\pgfpathlineto{\pgfqpoint{5.472911in}{1.621414in}}%
\pgfpathlineto{\pgfqpoint{5.479676in}{1.646851in}}%
\pgfpathlineto{\pgfqpoint{5.487944in}{1.687219in}}%
\pgfpathlineto{\pgfqpoint{5.497715in}{1.747552in}}%
\pgfpathlineto{\pgfqpoint{5.508238in}{1.826887in}}%
\pgfpathlineto{\pgfqpoint{5.521016in}{1.941387in}}%
\pgfpathlineto{\pgfqpoint{5.533794in}{2.073151in}}%
\pgfpathlineto{\pgfqpoint{5.534545in}{0.696000in}}%
\pgfpathlineto{\pgfqpoint{5.534545in}{0.696000in}}%
\pgfusepath{stroke}%
\end{pgfscope}%
\begin{pgfscope}%
\pgfpathrectangle{\pgfqpoint{0.800000in}{0.528000in}}{\pgfqpoint{4.960000in}{3.696000in}}%
\pgfusepath{clip}%
\pgfsetrectcap%
\pgfsetroundjoin%
\pgfsetlinewidth{1.505625pt}%
\definecolor{currentstroke}{rgb}{1.000000,0.498039,0.054902}%
\pgfsetstrokecolor{currentstroke}%
\pgfsetdash{}{0pt}%
\pgfpathmoveto{\pgfqpoint{1.025455in}{3.237068in}}%
\pgfpathlineto{\pgfqpoint{1.065291in}{3.235975in}}%
\pgfpathlineto{\pgfqpoint{1.107383in}{3.232579in}}%
\pgfpathlineto{\pgfqpoint{1.155488in}{3.226399in}}%
\pgfpathlineto{\pgfqpoint{1.229149in}{3.214340in}}%
\pgfpathlineto{\pgfqpoint{1.303562in}{3.202857in}}%
\pgfpathlineto{\pgfqpoint{1.350163in}{3.197920in}}%
\pgfpathlineto{\pgfqpoint{1.391503in}{3.195738in}}%
\pgfpathlineto{\pgfqpoint{1.431340in}{3.195832in}}%
\pgfpathlineto{\pgfqpoint{1.472681in}{3.198184in}}%
\pgfpathlineto{\pgfqpoint{1.518531in}{3.203114in}}%
\pgfpathlineto{\pgfqpoint{1.577910in}{3.211945in}}%
\pgfpathlineto{\pgfqpoint{1.690656in}{3.229072in}}%
\pgfpathlineto{\pgfqpoint{1.738010in}{3.233652in}}%
\pgfpathlineto{\pgfqpoint{1.776343in}{3.235455in}}%
\pgfpathlineto{\pgfqpoint{1.777847in}{0.696105in}}%
\pgfpathlineto{\pgfqpoint{1.865789in}{0.696100in}}%
\pgfpathlineto{\pgfqpoint{1.867292in}{3.781272in}}%
\pgfpathlineto{\pgfqpoint{1.878566in}{3.867567in}}%
\pgfpathlineto{\pgfqpoint{1.889089in}{3.931686in}}%
\pgfpathlineto{\pgfqpoint{1.898861in}{3.977907in}}%
\pgfpathlineto{\pgfqpoint{1.907880in}{4.010049in}}%
\pgfpathlineto{\pgfqpoint{1.916148in}{4.031392in}}%
\pgfpathlineto{\pgfqpoint{1.923665in}{4.044659in}}%
\pgfpathlineto{\pgfqpoint{1.930430in}{4.052064in}}%
\pgfpathlineto{\pgfqpoint{1.936443in}{4.055369in}}%
\pgfpathlineto{\pgfqpoint{1.941704in}{4.055954in}}%
\pgfpathlineto{\pgfqpoint{1.947717in}{4.054226in}}%
\pgfpathlineto{\pgfqpoint{1.954482in}{4.049536in}}%
\pgfpathlineto{\pgfqpoint{1.961999in}{4.041332in}}%
\pgfpathlineto{\pgfqpoint{1.971018in}{4.027955in}}%
\pgfpathlineto{\pgfqpoint{1.982293in}{4.006883in}}%
\pgfpathlineto{\pgfqpoint{1.996574in}{3.975143in}}%
\pgfpathlineto{\pgfqpoint{2.018372in}{3.920537in}}%
\pgfpathlineto{\pgfqpoint{2.064222in}{3.804692in}}%
\pgfpathlineto{\pgfqpoint{2.084516in}{3.760101in}}%
\pgfpathlineto{\pgfqpoint{2.101804in}{3.727303in}}%
\pgfpathlineto{\pgfqpoint{2.117588in}{3.702086in}}%
\pgfpathlineto{\pgfqpoint{2.131869in}{3.683426in}}%
\pgfpathlineto{\pgfqpoint{2.144647in}{3.670204in}}%
\pgfpathlineto{\pgfqpoint{2.156673in}{3.660819in}}%
\pgfpathlineto{\pgfqpoint{2.167948in}{3.654748in}}%
\pgfpathlineto{\pgfqpoint{2.178471in}{3.651476in}}%
\pgfpathlineto{\pgfqpoint{2.188994in}{3.650522in}}%
\pgfpathlineto{\pgfqpoint{2.199517in}{3.651883in}}%
\pgfpathlineto{\pgfqpoint{2.210040in}{3.655552in}}%
\pgfpathlineto{\pgfqpoint{2.221314in}{3.662027in}}%
\pgfpathlineto{\pgfqpoint{2.232589in}{3.671106in}}%
\pgfpathlineto{\pgfqpoint{2.244615in}{3.683616in}}%
\pgfpathlineto{\pgfqpoint{2.258145in}{3.701093in}}%
\pgfpathlineto{\pgfqpoint{2.272426in}{3.723303in}}%
\pgfpathlineto{\pgfqpoint{2.288210in}{3.752077in}}%
\pgfpathlineto{\pgfqpoint{2.306250in}{3.789846in}}%
\pgfpathlineto{\pgfqpoint{2.328047in}{3.841107in}}%
\pgfpathlineto{\pgfqpoint{2.397198in}{4.008926in}}%
\pgfpathlineto{\pgfqpoint{2.409225in}{4.030420in}}%
\pgfpathlineto{\pgfqpoint{2.418996in}{4.043841in}}%
\pgfpathlineto{\pgfqpoint{2.427264in}{4.051660in}}%
\pgfpathlineto{\pgfqpoint{2.434029in}{4.055220in}}%
\pgfpathlineto{\pgfqpoint{2.440042in}{4.055951in}}%
\pgfpathlineto{\pgfqpoint{2.445303in}{4.054515in}}%
\pgfpathlineto{\pgfqpoint{2.450565in}{4.050971in}}%
\pgfpathlineto{\pgfqpoint{2.456578in}{4.044123in}}%
\pgfpathlineto{\pgfqpoint{2.463343in}{4.032556in}}%
\pgfpathlineto{\pgfqpoint{2.470859in}{4.014493in}}%
\pgfpathlineto{\pgfqpoint{2.478375in}{3.990487in}}%
\pgfpathlineto{\pgfqpoint{2.486644in}{3.956683in}}%
\pgfpathlineto{\pgfqpoint{2.495663in}{3.910347in}}%
\pgfpathlineto{\pgfqpoint{2.505435in}{3.848330in}}%
\pgfpathlineto{\pgfqpoint{2.515958in}{3.767142in}}%
\pgfpathlineto{\pgfqpoint{2.527984in}{3.655566in}}%
\pgfpathlineto{\pgfqpoint{2.540762in}{3.515100in}}%
\pgfpathlineto{\pgfqpoint{2.555043in}{3.332630in}}%
\pgfpathlineto{\pgfqpoint{2.571579in}{3.091690in}}%
\pgfpathlineto{\pgfqpoint{2.593376in}{2.738091in}}%
\pgfpathlineto{\pgfqpoint{2.648998in}{1.818014in}}%
\pgfpathlineto{\pgfqpoint{2.665534in}{1.590203in}}%
\pgfpathlineto{\pgfqpoint{2.679063in}{1.433225in}}%
\pgfpathlineto{\pgfqpoint{2.690338in}{1.325995in}}%
\pgfpathlineto{\pgfqpoint{2.700109in}{1.252004in}}%
\pgfpathlineto{\pgfqpoint{2.709129in}{1.200177in}}%
\pgfpathlineto{\pgfqpoint{2.716646in}{1.169469in}}%
\pgfpathlineto{\pgfqpoint{2.722659in}{1.153229in}}%
\pgfpathlineto{\pgfqpoint{2.727920in}{1.145150in}}%
\pgfpathlineto{\pgfqpoint{2.731678in}{1.142896in}}%
\pgfpathlineto{\pgfqpoint{2.734685in}{1.143206in}}%
\pgfpathlineto{\pgfqpoint{2.738443in}{1.146232in}}%
\pgfpathlineto{\pgfqpoint{2.742953in}{1.153723in}}%
\pgfpathlineto{\pgfqpoint{2.748214in}{1.167753in}}%
\pgfpathlineto{\pgfqpoint{2.754228in}{1.190698in}}%
\pgfpathlineto{\pgfqpoint{2.761744in}{1.229571in}}%
\pgfpathlineto{\pgfqpoint{2.770012in}{1.285067in}}%
\pgfpathlineto{\pgfqpoint{2.779783in}{1.367142in}}%
\pgfpathlineto{\pgfqpoint{2.791058in}{1.482631in}}%
\pgfpathlineto{\pgfqpoint{2.803836in}{1.637792in}}%
\pgfpathlineto{\pgfqpoint{2.818869in}{1.848219in}}%
\pgfpathlineto{\pgfqpoint{2.838411in}{2.154735in}}%
\pgfpathlineto{\pgfqpoint{2.904556in}{3.220103in}}%
\pgfpathlineto{\pgfqpoint{2.921092in}{3.437953in}}%
\pgfpathlineto{\pgfqpoint{2.935373in}{3.599264in}}%
\pgfpathlineto{\pgfqpoint{2.948902in}{3.727633in}}%
\pgfpathlineto{\pgfqpoint{2.960929in}{3.821705in}}%
\pgfpathlineto{\pgfqpoint{2.972203in}{3.893358in}}%
\pgfpathlineto{\pgfqpoint{2.982726in}{3.946641in}}%
\pgfpathlineto{\pgfqpoint{2.992498in}{3.985260in}}%
\pgfpathlineto{\pgfqpoint{3.001517in}{4.012443in}}%
\pgfpathlineto{\pgfqpoint{3.009785in}{4.030907in}}%
\pgfpathlineto{\pgfqpoint{3.017302in}{4.042871in}}%
\pgfpathlineto{\pgfqpoint{3.024818in}{4.050730in}}%
\pgfpathlineto{\pgfqpoint{3.031583in}{4.054674in}}%
\pgfpathlineto{\pgfqpoint{3.038348in}{4.055997in}}%
\pgfpathlineto{\pgfqpoint{3.045112in}{4.055017in}}%
\pgfpathlineto{\pgfqpoint{3.052629in}{4.051600in}}%
\pgfpathlineto{\pgfqpoint{3.061648in}{4.044805in}}%
\pgfpathlineto{\pgfqpoint{3.072171in}{4.033968in}}%
\pgfpathlineto{\pgfqpoint{3.086453in}{4.015809in}}%
\pgfpathlineto{\pgfqpoint{3.114263in}{3.975777in}}%
\pgfpathlineto{\pgfqpoint{3.137564in}{3.944088in}}%
\pgfpathlineto{\pgfqpoint{3.153349in}{3.926122in}}%
\pgfpathlineto{\pgfqpoint{3.166878in}{3.913853in}}%
\pgfpathlineto{\pgfqpoint{3.178904in}{3.905742in}}%
\pgfpathlineto{\pgfqpoint{3.190179in}{3.900718in}}%
\pgfpathlineto{\pgfqpoint{3.200702in}{3.898376in}}%
\pgfpathlineto{\pgfqpoint{3.211225in}{3.898341in}}%
\pgfpathlineto{\pgfqpoint{3.221748in}{3.900609in}}%
\pgfpathlineto{\pgfqpoint{3.232271in}{3.905134in}}%
\pgfpathlineto{\pgfqpoint{3.243545in}{3.912386in}}%
\pgfpathlineto{\pgfqpoint{3.256323in}{3.923402in}}%
\pgfpathlineto{\pgfqpoint{3.270604in}{3.938820in}}%
\pgfpathlineto{\pgfqpoint{3.287892in}{3.960952in}}%
\pgfpathlineto{\pgfqpoint{3.323219in}{4.011132in}}%
\pgfpathlineto{\pgfqpoint{3.341259in}{4.034108in}}%
\pgfpathlineto{\pgfqpoint{3.353285in}{4.046058in}}%
\pgfpathlineto{\pgfqpoint{3.363056in}{4.052737in}}%
\pgfpathlineto{\pgfqpoint{3.371324in}{4.055650in}}%
\pgfpathlineto{\pgfqpoint{3.378089in}{4.055794in}}%
\pgfpathlineto{\pgfqpoint{3.384854in}{4.053612in}}%
\pgfpathlineto{\pgfqpoint{3.391619in}{4.048806in}}%
\pgfpathlineto{\pgfqpoint{3.398383in}{4.041068in}}%
\pgfpathlineto{\pgfqpoint{3.405900in}{4.028655in}}%
\pgfpathlineto{\pgfqpoint{3.413416in}{4.011814in}}%
\pgfpathlineto{\pgfqpoint{3.421684in}{3.987671in}}%
\pgfpathlineto{\pgfqpoint{3.430704in}{3.954008in}}%
\pgfpathlineto{\pgfqpoint{3.440475in}{3.908183in}}%
\pgfpathlineto{\pgfqpoint{3.450998in}{3.847146in}}%
\pgfpathlineto{\pgfqpoint{3.462273in}{3.767523in}}%
\pgfpathlineto{\pgfqpoint{3.474299in}{3.665785in}}%
\pgfpathlineto{\pgfqpoint{3.487077in}{3.538569in}}%
\pgfpathlineto{\pgfqpoint{3.501358in}{3.373911in}}%
\pgfpathlineto{\pgfqpoint{3.517894in}{3.156618in}}%
\pgfpathlineto{\pgfqpoint{3.538940in}{2.848093in}}%
\pgfpathlineto{\pgfqpoint{3.603581in}{1.875668in}}%
\pgfpathlineto{\pgfqpoint{3.619366in}{1.683228in}}%
\pgfpathlineto{\pgfqpoint{3.632144in}{1.552278in}}%
\pgfpathlineto{\pgfqpoint{3.643418in}{1.457935in}}%
\pgfpathlineto{\pgfqpoint{3.653189in}{1.393629in}}%
\pgfpathlineto{\pgfqpoint{3.661458in}{1.352535in}}%
\pgfpathlineto{\pgfqpoint{3.668974in}{1.326104in}}%
\pgfpathlineto{\pgfqpoint{3.674987in}{1.312585in}}%
\pgfpathlineto{\pgfqpoint{3.679497in}{1.306934in}}%
\pgfpathlineto{\pgfqpoint{3.683255in}{1.305175in}}%
\pgfpathlineto{\pgfqpoint{3.687013in}{1.306099in}}%
\pgfpathlineto{\pgfqpoint{3.690771in}{1.309702in}}%
\pgfpathlineto{\pgfqpoint{3.695281in}{1.317551in}}%
\pgfpathlineto{\pgfqpoint{3.700543in}{1.331538in}}%
\pgfpathlineto{\pgfqpoint{3.706556in}{1.353818in}}%
\pgfpathlineto{\pgfqpoint{3.714072in}{1.390942in}}%
\pgfpathlineto{\pgfqpoint{3.722340in}{1.443337in}}%
\pgfpathlineto{\pgfqpoint{3.732112in}{1.520181in}}%
\pgfpathlineto{\pgfqpoint{3.743386in}{1.627586in}}%
\pgfpathlineto{\pgfqpoint{3.756164in}{1.771093in}}%
\pgfpathlineto{\pgfqpoint{3.771949in}{1.975092in}}%
\pgfpathlineto{\pgfqpoint{3.792243in}{2.268463in}}%
\pgfpathlineto{\pgfqpoint{3.855381in}{3.203576in}}%
\pgfpathlineto{\pgfqpoint{3.872668in}{3.417212in}}%
\pgfpathlineto{\pgfqpoint{3.887701in}{3.577463in}}%
\pgfpathlineto{\pgfqpoint{3.901231in}{3.699869in}}%
\pgfpathlineto{\pgfqpoint{3.914009in}{3.796350in}}%
\pgfpathlineto{\pgfqpoint{3.926035in}{3.870740in}}%
\pgfpathlineto{\pgfqpoint{3.937310in}{3.926898in}}%
\pgfpathlineto{\pgfqpoint{3.947833in}{3.968405in}}%
\pgfpathlineto{\pgfqpoint{3.957604in}{3.998410in}}%
\pgfpathlineto{\pgfqpoint{3.966624in}{4.019572in}}%
\pgfpathlineto{\pgfqpoint{3.975643in}{4.035179in}}%
\pgfpathlineto{\pgfqpoint{3.983911in}{4.045202in}}%
\pgfpathlineto{\pgfqpoint{3.992179in}{4.051674in}}%
\pgfpathlineto{\pgfqpoint{3.999696in}{4.054907in}}%
\pgfpathlineto{\pgfqpoint{4.007964in}{4.056000in}}%
\pgfpathlineto{\pgfqpoint{4.016983in}{4.054776in}}%
\pgfpathlineto{\pgfqpoint{4.027506in}{4.050884in}}%
\pgfpathlineto{\pgfqpoint{4.041036in}{4.043210in}}%
\pgfpathlineto{\pgfqpoint{4.065088in}{4.026351in}}%
\pgfpathlineto{\pgfqpoint{4.087638in}{4.011583in}}%
\pgfpathlineto{\pgfqpoint{4.103422in}{4.003890in}}%
\pgfpathlineto{\pgfqpoint{4.116952in}{3.999744in}}%
\pgfpathlineto{\pgfqpoint{4.129730in}{3.998178in}}%
\pgfpathlineto{\pgfqpoint{4.141756in}{3.998872in}}%
\pgfpathlineto{\pgfqpoint{4.154534in}{4.001866in}}%
\pgfpathlineto{\pgfqpoint{4.168063in}{4.007367in}}%
\pgfpathlineto{\pgfqpoint{4.183848in}{4.016285in}}%
\pgfpathlineto{\pgfqpoint{4.207900in}{4.032873in}}%
\pgfpathlineto{\pgfqpoint{4.231201in}{4.048085in}}%
\pgfpathlineto{\pgfqpoint{4.243979in}{4.053791in}}%
\pgfpathlineto{\pgfqpoint{4.253750in}{4.055871in}}%
\pgfpathlineto{\pgfqpoint{4.262018in}{4.055552in}}%
\pgfpathlineto{\pgfqpoint{4.269535in}{4.053229in}}%
\pgfpathlineto{\pgfqpoint{4.277051in}{4.048619in}}%
\pgfpathlineto{\pgfqpoint{4.284567in}{4.041371in}}%
\pgfpathlineto{\pgfqpoint{4.292084in}{4.031121in}}%
\pgfpathlineto{\pgfqpoint{4.300352in}{4.015935in}}%
\pgfpathlineto{\pgfqpoint{4.309372in}{3.994133in}}%
\pgfpathlineto{\pgfqpoint{4.318391in}{3.966252in}}%
\pgfpathlineto{\pgfqpoint{4.328163in}{3.928484in}}%
\pgfpathlineto{\pgfqpoint{4.338686in}{3.878225in}}%
\pgfpathlineto{\pgfqpoint{4.349960in}{3.812506in}}%
\pgfpathlineto{\pgfqpoint{4.361986in}{3.728104in}}%
\pgfpathlineto{\pgfqpoint{4.374764in}{3.621747in}}%
\pgfpathlineto{\pgfqpoint{4.389045in}{3.482627in}}%
\pgfpathlineto{\pgfqpoint{4.404830in}{3.305382in}}%
\pgfpathlineto{\pgfqpoint{4.423621in}{3.066804in}}%
\pgfpathlineto{\pgfqpoint{4.449177in}{2.710119in}}%
\pgfpathlineto{\pgfqpoint{4.492020in}{2.110931in}}%
\pgfpathlineto{\pgfqpoint{4.510060in}{1.891763in}}%
\pgfpathlineto{\pgfqpoint{4.524341in}{1.743475in}}%
\pgfpathlineto{\pgfqpoint{4.536367in}{1.639573in}}%
\pgfpathlineto{\pgfqpoint{4.546890in}{1.566229in}}%
\pgfpathlineto{\pgfqpoint{4.555910in}{1.517331in}}%
\pgfpathlineto{\pgfqpoint{4.564178in}{1.484327in}}%
\pgfpathlineto{\pgfqpoint{4.570942in}{1.465950in}}%
\pgfpathlineto{\pgfqpoint{4.576204in}{1.457101in}}%
\pgfpathlineto{\pgfqpoint{4.580714in}{1.453334in}}%
\pgfpathlineto{\pgfqpoint{4.584472in}{1.452892in}}%
\pgfpathlineto{\pgfqpoint{4.588230in}{1.454903in}}%
\pgfpathlineto{\pgfqpoint{4.592740in}{1.460544in}}%
\pgfpathlineto{\pgfqpoint{4.598001in}{1.471553in}}%
\pgfpathlineto{\pgfqpoint{4.604015in}{1.489914in}}%
\pgfpathlineto{\pgfqpoint{4.610779in}{1.517822in}}%
\pgfpathlineto{\pgfqpoint{4.619047in}{1.562091in}}%
\pgfpathlineto{\pgfqpoint{4.628819in}{1.628237in}}%
\pgfpathlineto{\pgfqpoint{4.639342in}{1.715205in}}%
\pgfpathlineto{\pgfqpoint{4.652120in}{1.840704in}}%
\pgfpathlineto{\pgfqpoint{4.667152in}{2.012260in}}%
\pgfpathlineto{\pgfqpoint{4.685943in}{2.254263in}}%
\pgfpathlineto{\pgfqpoint{4.719016in}{2.715823in}}%
\pgfpathlineto{\pgfqpoint{4.748330in}{3.112269in}}%
\pgfpathlineto{\pgfqpoint{4.767872in}{3.347287in}}%
\pgfpathlineto{\pgfqpoint{4.784408in}{3.520221in}}%
\pgfpathlineto{\pgfqpoint{4.799441in}{3.654341in}}%
\pgfpathlineto{\pgfqpoint{4.812971in}{3.755746in}}%
\pgfpathlineto{\pgfqpoint{4.825749in}{3.835122in}}%
\pgfpathlineto{\pgfqpoint{4.837775in}{3.896093in}}%
\pgfpathlineto{\pgfqpoint{4.849049in}{3.942109in}}%
\pgfpathlineto{\pgfqpoint{4.859572in}{3.976261in}}%
\pgfpathlineto{\pgfqpoint{4.870095in}{4.002857in}}%
\pgfpathlineto{\pgfqpoint{4.879867in}{4.021624in}}%
\pgfpathlineto{\pgfqpoint{4.888886in}{4.034557in}}%
\pgfpathlineto{\pgfqpoint{4.897906in}{4.043882in}}%
\pgfpathlineto{\pgfqpoint{4.906926in}{4.050165in}}%
\pgfpathlineto{\pgfqpoint{4.916697in}{4.054158in}}%
\pgfpathlineto{\pgfqpoint{4.927220in}{4.055891in}}%
\pgfpathlineto{\pgfqpoint{4.939246in}{4.055507in}}%
\pgfpathlineto{\pgfqpoint{4.955782in}{4.052491in}}%
\pgfpathlineto{\pgfqpoint{5.002384in}{4.042709in}}%
\pgfpathlineto{\pgfqpoint{5.019672in}{4.042036in}}%
\pgfpathlineto{\pgfqpoint{5.036959in}{4.043583in}}%
\pgfpathlineto{\pgfqpoint{5.058005in}{4.047852in}}%
\pgfpathlineto{\pgfqpoint{5.094084in}{4.055569in}}%
\pgfpathlineto{\pgfqpoint{5.106862in}{4.055806in}}%
\pgfpathlineto{\pgfqpoint{5.117385in}{4.053843in}}%
\pgfpathlineto{\pgfqpoint{5.126405in}{4.050050in}}%
\pgfpathlineto{\pgfqpoint{5.135424in}{4.043810in}}%
\pgfpathlineto{\pgfqpoint{5.144444in}{4.034617in}}%
\pgfpathlineto{\pgfqpoint{5.153464in}{4.021941in}}%
\pgfpathlineto{\pgfqpoint{5.162483in}{4.005234in}}%
\pgfpathlineto{\pgfqpoint{5.172255in}{3.981941in}}%
\pgfpathlineto{\pgfqpoint{5.182026in}{3.952563in}}%
\pgfpathlineto{\pgfqpoint{5.192549in}{3.913347in}}%
\pgfpathlineto{\pgfqpoint{5.203072in}{3.865514in}}%
\pgfpathlineto{\pgfqpoint{5.214347in}{3.803938in}}%
\pgfpathlineto{\pgfqpoint{5.226373in}{3.725785in}}%
\pgfpathlineto{\pgfqpoint{5.239151in}{3.628149in}}%
\pgfpathlineto{\pgfqpoint{5.253432in}{3.501197in}}%
\pgfpathlineto{\pgfqpoint{5.269216in}{3.340007in}}%
\pgfpathlineto{\pgfqpoint{5.287256in}{3.132287in}}%
\pgfpathlineto{\pgfqpoint{5.311308in}{2.827394in}}%
\pgfpathlineto{\pgfqpoint{5.362420in}{2.171719in}}%
\pgfpathlineto{\pgfqpoint{5.379708in}{1.981496in}}%
\pgfpathlineto{\pgfqpoint{5.393989in}{1.847223in}}%
\pgfpathlineto{\pgfqpoint{5.406015in}{1.753428in}}%
\pgfpathlineto{\pgfqpoint{5.416538in}{1.687469in}}%
\pgfpathlineto{\pgfqpoint{5.425558in}{1.643729in}}%
\pgfpathlineto{\pgfqpoint{5.433074in}{1.616679in}}%
\pgfpathlineto{\pgfqpoint{5.439839in}{1.599825in}}%
\pgfpathlineto{\pgfqpoint{5.445100in}{1.591689in}}%
\pgfpathlineto{\pgfqpoint{5.449610in}{1.588204in}}%
\pgfpathlineto{\pgfqpoint{5.453368in}{1.587765in}}%
\pgfpathlineto{\pgfqpoint{5.457126in}{1.589567in}}%
\pgfpathlineto{\pgfqpoint{5.461636in}{1.594680in}}%
\pgfpathlineto{\pgfqpoint{5.466898in}{1.604691in}}%
\pgfpathlineto{\pgfqpoint{5.472911in}{1.621414in}}%
\pgfpathlineto{\pgfqpoint{5.479676in}{1.646851in}}%
\pgfpathlineto{\pgfqpoint{5.487944in}{1.687219in}}%
\pgfpathlineto{\pgfqpoint{5.497715in}{1.747552in}}%
\pgfpathlineto{\pgfqpoint{5.508238in}{1.826887in}}%
\pgfpathlineto{\pgfqpoint{5.521016in}{1.941387in}}%
\pgfpathlineto{\pgfqpoint{5.533794in}{2.073151in}}%
\pgfpathlineto{\pgfqpoint{5.534545in}{0.696000in}}%
\pgfpathlineto{\pgfqpoint{5.534545in}{0.696000in}}%
\pgfusepath{stroke}%
\end{pgfscope}%
\begin{pgfscope}%
\pgfsetrectcap%
\pgfsetmiterjoin%
\pgfsetlinewidth{0.803000pt}%
\definecolor{currentstroke}{rgb}{0.000000,0.000000,0.000000}%
\pgfsetstrokecolor{currentstroke}%
\pgfsetdash{}{0pt}%
\pgfpathmoveto{\pgfqpoint{0.800000in}{0.528000in}}%
\pgfpathlineto{\pgfqpoint{0.800000in}{4.224000in}}%
\pgfusepath{stroke}%
\end{pgfscope}%
\begin{pgfscope}%
\pgfsetrectcap%
\pgfsetmiterjoin%
\pgfsetlinewidth{0.803000pt}%
\definecolor{currentstroke}{rgb}{0.000000,0.000000,0.000000}%
\pgfsetstrokecolor{currentstroke}%
\pgfsetdash{}{0pt}%
\pgfpathmoveto{\pgfqpoint{5.760000in}{0.528000in}}%
\pgfpathlineto{\pgfqpoint{5.760000in}{4.224000in}}%
\pgfusepath{stroke}%
\end{pgfscope}%
\begin{pgfscope}%
\pgfsetrectcap%
\pgfsetmiterjoin%
\pgfsetlinewidth{0.803000pt}%
\definecolor{currentstroke}{rgb}{0.000000,0.000000,0.000000}%
\pgfsetstrokecolor{currentstroke}%
\pgfsetdash{}{0pt}%
\pgfpathmoveto{\pgfqpoint{0.800000in}{0.528000in}}%
\pgfpathlineto{\pgfqpoint{5.760000in}{0.528000in}}%
\pgfusepath{stroke}%
\end{pgfscope}%
\begin{pgfscope}%
\pgfsetrectcap%
\pgfsetmiterjoin%
\pgfsetlinewidth{0.803000pt}%
\definecolor{currentstroke}{rgb}{0.000000,0.000000,0.000000}%
\pgfsetstrokecolor{currentstroke}%
\pgfsetdash{}{0pt}%
\pgfpathmoveto{\pgfqpoint{0.800000in}{4.224000in}}%
\pgfpathlineto{\pgfqpoint{5.760000in}{4.224000in}}%
\pgfusepath{stroke}%
\end{pgfscope}%
\begin{pgfscope}%
\pgfsetbuttcap%
\pgfsetmiterjoin%
\definecolor{currentfill}{rgb}{1.000000,1.000000,1.000000}%
\pgfsetfillcolor{currentfill}%
\pgfsetfillopacity{0.800000}%
\pgfsetlinewidth{1.003750pt}%
\definecolor{currentstroke}{rgb}{0.800000,0.800000,0.800000}%
\pgfsetstrokecolor{currentstroke}%
\pgfsetstrokeopacity{0.800000}%
\pgfsetdash{}{0pt}%
\pgfpathmoveto{\pgfqpoint{2.435423in}{0.597444in}}%
\pgfpathlineto{\pgfqpoint{4.124577in}{0.597444in}}%
\pgfpathquadraticcurveto{\pgfqpoint{4.152355in}{0.597444in}}{\pgfqpoint{4.152355in}{0.625222in}}%
\pgfpathlineto{\pgfqpoint{4.152355in}{1.015114in}}%
\pgfpathquadraticcurveto{\pgfqpoint{4.152355in}{1.042892in}}{\pgfqpoint{4.124577in}{1.042892in}}%
\pgfpathlineto{\pgfqpoint{2.435423in}{1.042892in}}%
\pgfpathquadraticcurveto{\pgfqpoint{2.407645in}{1.042892in}}{\pgfqpoint{2.407645in}{1.015114in}}%
\pgfpathlineto{\pgfqpoint{2.407645in}{0.625222in}}%
\pgfpathquadraticcurveto{\pgfqpoint{2.407645in}{0.597444in}}{\pgfqpoint{2.435423in}{0.597444in}}%
\pgfpathlineto{\pgfqpoint{2.435423in}{0.597444in}}%
\pgfpathclose%
\pgfusepath{stroke,fill}%
\end{pgfscope}%
\begin{pgfscope}%
\pgfsetrectcap%
\pgfsetroundjoin%
\pgfsetlinewidth{1.505625pt}%
\definecolor{currentstroke}{rgb}{0.121569,0.466667,0.705882}%
\pgfsetstrokecolor{currentstroke}%
\pgfsetdash{}{0pt}%
\pgfpathmoveto{\pgfqpoint{2.463200in}{0.933748in}}%
\pgfpathlineto{\pgfqpoint{2.602089in}{0.933748in}}%
\pgfpathlineto{\pgfqpoint{2.740978in}{0.933748in}}%
\pgfusepath{stroke}%
\end{pgfscope}%
\begin{pgfscope}%
\definecolor{textcolor}{rgb}{0.000000,0.000000,0.000000}%
\pgfsetstrokecolor{textcolor}%
\pgfsetfillcolor{textcolor}%
\pgftext[x=2.852089in,y=0.885137in,left,base]{\color{textcolor}\rmfamily\fontsize{10.000000}{12.000000}\selectfont power non-algebraic}%
\end{pgfscope}%
\begin{pgfscope}%
\pgfsetrectcap%
\pgfsetroundjoin%
\pgfsetlinewidth{1.505625pt}%
\definecolor{currentstroke}{rgb}{1.000000,0.498039,0.054902}%
\pgfsetstrokecolor{currentstroke}%
\pgfsetdash{}{0pt}%
\pgfpathmoveto{\pgfqpoint{2.463200in}{0.731857in}}%
\pgfpathlineto{\pgfqpoint{2.602089in}{0.731857in}}%
\pgfpathlineto{\pgfqpoint{2.740978in}{0.731857in}}%
\pgfusepath{stroke}%
\end{pgfscope}%
\begin{pgfscope}%
\definecolor{textcolor}{rgb}{0.000000,0.000000,0.000000}%
\pgfsetstrokecolor{textcolor}%
\pgfsetfillcolor{textcolor}%
\pgftext[x=2.852089in,y=0.683246in,left,base]{\color{textcolor}\rmfamily\fontsize{10.000000}{12.000000}\selectfont power algebraic}%
\end{pgfscope}%
\end{pgfpicture}%
\makeatother%
\endgroup%


% \chapter{Code}
% \label{app:code}

% \section{Fault scenario parameters}
% \label{app:fault-scenarios}

% \lstinputlisting[caption={Module containing all relevant functions of the SMIB model in Python},captionpos=b,style=style-python,label=lst:model-func]{python/smib_model.py}

% \section{Model of GK}

% \lstinputlisting[caption={GK's SMIB model with Heun's integration method},captionpos=b,style=style-python,label=lst:model-func-heun]{python/simulation_example_gk/simulation_even_better.py}

% \section{Main model}

% \lstinputlisting[caption={Main \acs{SMIB} model code\\}, captionpos=t, style=style-python]{python/smib_model.py}

% \section{Fault models}

% \subsection*{Fault 1}
% \lstinputlisting{python/fault_1.py}

% \subsection*{Fault 2}
% \lstinputlisting{python/fault_2.py}

% \subsection*{Fault 3}
% \lstinputlisting{python/fault_3.py}

% \section{Additional comparison}

% \subsection*{Generator parameter}
% \lstinputlisting{python/parameter_comparison.py}

% \subsection*{Algebraic vs. non-algebraic}
% \lstinputlisting{python/comparison_alg-vs-nonalg.py}

% \chapter{Additional}
% \label{app:additional}

