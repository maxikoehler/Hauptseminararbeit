% !TeX root = ../main.tex

\appendixtoc
% \appendix
\addcontentsline{maintoc}{chapter}{Appendix}
\ohead[]{\textsc{Appendix}}
\label{app:appendix}

% \renewcommand\thechapter{\roman{chapter}}
% \setcounter{chapter}{0}

% \pagebreak
% \includepdf[pages=-,scale=.9,pagecommand={}]{Aufgabenstellung.pdf} 
% PDF um 10% verkleinert einbinden --> Kopf- und Fußzeile  werden so korrekt dargestellt. Die Option `pages' ermöglicht es, eine bestimmte Sequenz von Seiten (z.B. 2-10 oder `-' für alle Seiten) auszuwählen.
% \pagebreak
%\includepdf[pages=-,scale=.8,pagecommand=\section*{A. eventGenerator.py}]{../appendix/eventGenerator.py.pdf}
%\includepdf[pages=-,scale=.8,pagecommand=\section*{B. sendEvents.py}]{../appendix/sendEvents.py.pdf}

%%%%%%%%%%%%%%%%%%%%%%%%%%%%%%%%%%%%%%%%%%
%%%%%%%%%%%%%%%%%%%%%%%%%%%%%%%%%%%%%%%%%%
%%%%%%%%%%%%%%%%%%%%%%%%%%%%%%%%%%%%%%%%%%
\centering

\chapter{Graphics and tables}

\section{Initial values}
\label{app:initials}

\begin{centering}
    \small
    % \caption[short]{title}
    % \vspace{12pt}
    \begin{tabularx}{\textwidth}{l|X|l|l}
        % \hline % \rowcolor{lightgray} 
        & fault 1 & fault 2 & fault 3 \\ \hline \hline
        $P_\mathrm{e,~max}$     & \multicolumn{3}{c}{1.2 p.u.} \\ \hline
        $P_\mathrm{mech}$       & 0.9 p.u. & 1.0 p.u. & 0.7 p.u. \\ \hline
        $E_\mathrm{ibb}$        & \multicolumn{3}{c}{1 p.u.} \\ \hline
        $E_\mathrm{gen}$        & \multicolumn{3}{c}{1.14 p.u.} \\ \hline
        $\delta_\mathrm{ibb}$   & 48.6 deg & 56.4 deg & 35.7 deg \\ \hline
        $\delta_\mathrm{gen}$   & \multicolumn{3}{c}{0 deg} \\ \hline

        $Y_\mathrm{bus,~init}$ & \multicolumn{3}{c}{$\begin{bmatrix} -\frac{1j}{X_\mathrm{gen}}-\frac{3j}{X_\mathrm{line}} & \frac{3j}{X_\mathrm{line}} \\ \frac{3j}{X_\mathrm{line}} & -\frac{3j}{X_\mathrm{line}}-\frac{1j}{X_\mathrm{ibb}}\end{bmatrix}$} \\ \hline

        $Y_\mathrm{bus,~fault}$ & $\begin{bmatrix} -\frac{1j}{X_\mathrm{gen}}-\frac{3j}{X_\mathrm{line}} + 10^6 & \frac{3j}{X_\mathrm{line}} \\ \frac{3j}{X_\mathrm{line}} & -\frac{3j}{X_\mathrm{line}}-\frac{1j}{X_\mathrm{ibb}}\end{bmatrix}$ 
        & \multicolumn{2}{c}{$\begin{bmatrix} -\frac{1j}{X_\mathrm{gen}}-\frac{2j}{X_\mathrm{line}} & \frac{2j}{X_\mathrm{line}} \\ \frac{2j}{X_\mathrm{line}} & -\frac{2j}{X_\mathrm{line}}-\frac{1j}{X_\mathrm{ibb}}\end{bmatrix}$} \\ \hline \hline

        $t_\mathrm{start}$      & \multicolumn{3}{c}{-1 s} \\ \hline
        $t_\mathrm{end}$        & \multicolumn{2}{c|}{2 s} & 5 s \\ \hline
        $t_\mathrm{step}$       & \multicolumn{3}{c}{0.001 s} \\ \hline
        $t_\mathrm{fault,~start}$   & \multicolumn{3}{c}{0 s} \\
    \end{tabularx}
\end{centering}

\flushleft
For the modules {\itshape parameter\_comparison.py} and {\itshape comparison\_alg-vs-nonalg.py} the initial values from fault one are used, besides the described variations. \\[24pt]

\begin{minipage}[l]{.49\textwidth}
    \begin{tabular}{l|r}
        $X_\mathrm{line}$       & 1.95 p.u. \\ \hline
        $X_\mathrm{gen}$        & 0.2 p.u. \\ \hline
        $X_\mathrm{ibb}$        & 0.1 p.u. \\ 
    \end{tabular}
\end{minipage}
\begin{minipage}[r]{.49\textwidth}
    \begin{tabular}{l|r}
        $H_\mathrm{gen}$        & 3.3 s \\ \hline
        $f_\mathrm{n}$          & 50 $\mathrm{\frac{1}{s}}$ \\ \hline
        $\omega$                & 0 \\ 
    \end{tabular}
\end{minipage}

\section{Fault 1}
\label{app:fault1}

%% Creator: Matplotlib, PGF backend
%%
%% To include the figure in your LaTeX document, write
%%   \input{<filename>.pgf}
%%
%% Make sure the required packages are loaded in your preamble
%%   \usepackage{pgf}
%%
%% Also ensure that all the required font packages are loaded; for instance,
%% the lmodern package is sometimes necessary when using math font.
%%   \usepackage{lmodern}
%%
%% Figures using additional raster images can only be included by \input if
%% they are in the same directory as the main LaTeX file. For loading figures
%% from other directories you can use the `import` package
%%   \usepackage{import}
%%
%% and then include the figures with
%%   \import{<path to file>}{<filename>.pgf}
%%
%% Matplotlib used the following preamble
%%   
%%   \usepackage{fontspec}
%%   \setmainfont{Charter.ttc}[Path=\detokenize{/System/Library/Fonts/Supplemental/}]
%%   \setsansfont{DejaVuSans.ttf}[Path=\detokenize{/opt/homebrew/lib/python3.10/site-packages/matplotlib/mpl-data/fonts/ttf/}]
%%   \setmonofont{DejaVuSansMono.ttf}[Path=\detokenize{/opt/homebrew/lib/python3.10/site-packages/matplotlib/mpl-data/fonts/ttf/}]
%%   \makeatletter\@ifpackageloaded{underscore}{}{\usepackage[strings]{underscore}}\makeatother
%%
\begingroup%
\makeatletter%
\begin{pgfpicture}%
\pgfpathrectangle{\pgfpointorigin}{\pgfqpoint{6.000000in}{8.000000in}}%
\pgfusepath{use as bounding box, clip}%
\begin{pgfscope}%
\pgfsetbuttcap%
\pgfsetmiterjoin%
\definecolor{currentfill}{rgb}{1.000000,1.000000,1.000000}%
\pgfsetfillcolor{currentfill}%
\pgfsetlinewidth{0.000000pt}%
\definecolor{currentstroke}{rgb}{1.000000,1.000000,1.000000}%
\pgfsetstrokecolor{currentstroke}%
\pgfsetdash{}{0pt}%
\pgfpathmoveto{\pgfqpoint{0.000000in}{0.000000in}}%
\pgfpathlineto{\pgfqpoint{6.000000in}{0.000000in}}%
\pgfpathlineto{\pgfqpoint{6.000000in}{8.000000in}}%
\pgfpathlineto{\pgfqpoint{0.000000in}{8.000000in}}%
\pgfpathlineto{\pgfqpoint{0.000000in}{0.000000in}}%
\pgfpathclose%
\pgfusepath{fill}%
\end{pgfscope}%
\begin{pgfscope}%
\pgfsetbuttcap%
\pgfsetmiterjoin%
\definecolor{currentfill}{rgb}{1.000000,1.000000,1.000000}%
\pgfsetfillcolor{currentfill}%
\pgfsetlinewidth{0.000000pt}%
\definecolor{currentstroke}{rgb}{0.000000,0.000000,0.000000}%
\pgfsetstrokecolor{currentstroke}%
\pgfsetstrokeopacity{0.000000}%
\pgfsetdash{}{0pt}%
\pgfpathmoveto{\pgfqpoint{0.750000in}{3.960000in}}%
\pgfpathlineto{\pgfqpoint{5.400000in}{3.960000in}}%
\pgfpathlineto{\pgfqpoint{5.400000in}{7.040000in}}%
\pgfpathlineto{\pgfqpoint{0.750000in}{7.040000in}}%
\pgfpathlineto{\pgfqpoint{0.750000in}{3.960000in}}%
\pgfpathclose%
\pgfusepath{fill}%
\end{pgfscope}%
\begin{pgfscope}%
\pgfpathrectangle{\pgfqpoint{0.750000in}{3.960000in}}{\pgfqpoint{4.650000in}{3.080000in}}%
\pgfusepath{clip}%
\pgfsetbuttcap%
\pgfsetroundjoin%
\definecolor{currentfill}{rgb}{0.900000,0.900000,0.900000}%
\pgfsetfillcolor{currentfill}%
\pgfsetlinewidth{1.003750pt}%
\definecolor{currentstroke}{rgb}{0.500000,0.500000,0.500000}%
\pgfsetstrokecolor{currentstroke}%
\pgfsetdash{}{0pt}%
\pgfsys@defobject{currentmarker}{\pgfqpoint{2.064917in}{3.960080in}}{\pgfqpoint{2.440004in}{6.240196in}}{%
\pgfpathmoveto{\pgfqpoint{2.064917in}{6.240196in}}%
\pgfpathlineto{\pgfqpoint{2.064917in}{3.960084in}}%
\pgfpathlineto{\pgfqpoint{2.072572in}{3.960084in}}%
\pgfpathlineto{\pgfqpoint{2.080226in}{3.960084in}}%
\pgfpathlineto{\pgfqpoint{2.087881in}{3.960084in}}%
\pgfpathlineto{\pgfqpoint{2.095536in}{3.960084in}}%
\pgfpathlineto{\pgfqpoint{2.103191in}{3.960084in}}%
\pgfpathlineto{\pgfqpoint{2.110846in}{3.960084in}}%
\pgfpathlineto{\pgfqpoint{2.118501in}{3.960084in}}%
\pgfpathlineto{\pgfqpoint{2.126155in}{3.960084in}}%
\pgfpathlineto{\pgfqpoint{2.133810in}{3.960084in}}%
\pgfpathlineto{\pgfqpoint{2.141465in}{3.960083in}}%
\pgfpathlineto{\pgfqpoint{2.149120in}{3.960083in}}%
\pgfpathlineto{\pgfqpoint{2.156775in}{3.960083in}}%
\pgfpathlineto{\pgfqpoint{2.164430in}{3.960083in}}%
\pgfpathlineto{\pgfqpoint{2.172084in}{3.960083in}}%
\pgfpathlineto{\pgfqpoint{2.179739in}{3.960083in}}%
\pgfpathlineto{\pgfqpoint{2.187394in}{3.960083in}}%
\pgfpathlineto{\pgfqpoint{2.195049in}{3.960083in}}%
\pgfpathlineto{\pgfqpoint{2.202704in}{3.960083in}}%
\pgfpathlineto{\pgfqpoint{2.210359in}{3.960083in}}%
\pgfpathlineto{\pgfqpoint{2.218013in}{3.960083in}}%
\pgfpathlineto{\pgfqpoint{2.225668in}{3.960083in}}%
\pgfpathlineto{\pgfqpoint{2.233323in}{3.960083in}}%
\pgfpathlineto{\pgfqpoint{2.240978in}{3.960082in}}%
\pgfpathlineto{\pgfqpoint{2.248633in}{3.960082in}}%
\pgfpathlineto{\pgfqpoint{2.256288in}{3.960082in}}%
\pgfpathlineto{\pgfqpoint{2.263942in}{3.960082in}}%
\pgfpathlineto{\pgfqpoint{2.271597in}{3.960082in}}%
\pgfpathlineto{\pgfqpoint{2.279252in}{3.960082in}}%
\pgfpathlineto{\pgfqpoint{2.286907in}{3.960082in}}%
\pgfpathlineto{\pgfqpoint{2.294562in}{3.960082in}}%
\pgfpathlineto{\pgfqpoint{2.302217in}{3.960082in}}%
\pgfpathlineto{\pgfqpoint{2.309871in}{3.960082in}}%
\pgfpathlineto{\pgfqpoint{2.317526in}{3.960082in}}%
\pgfpathlineto{\pgfqpoint{2.325181in}{3.960082in}}%
\pgfpathlineto{\pgfqpoint{2.332836in}{3.960081in}}%
\pgfpathlineto{\pgfqpoint{2.340491in}{3.960081in}}%
\pgfpathlineto{\pgfqpoint{2.348146in}{3.960081in}}%
\pgfpathlineto{\pgfqpoint{2.355800in}{3.960081in}}%
\pgfpathlineto{\pgfqpoint{2.363455in}{3.960081in}}%
\pgfpathlineto{\pgfqpoint{2.371110in}{3.960081in}}%
\pgfpathlineto{\pgfqpoint{2.378765in}{3.960081in}}%
\pgfpathlineto{\pgfqpoint{2.386420in}{3.960081in}}%
\pgfpathlineto{\pgfqpoint{2.394075in}{3.960081in}}%
\pgfpathlineto{\pgfqpoint{2.401729in}{3.960081in}}%
\pgfpathlineto{\pgfqpoint{2.409384in}{3.960081in}}%
\pgfpathlineto{\pgfqpoint{2.417039in}{3.960081in}}%
\pgfpathlineto{\pgfqpoint{2.424694in}{3.960080in}}%
\pgfpathlineto{\pgfqpoint{2.432349in}{3.960080in}}%
\pgfpathlineto{\pgfqpoint{2.440004in}{3.960080in}}%
\pgfpathlineto{\pgfqpoint{2.440004in}{6.240196in}}%
\pgfpathlineto{\pgfqpoint{2.440004in}{6.240196in}}%
\pgfpathlineto{\pgfqpoint{2.432349in}{6.240196in}}%
\pgfpathlineto{\pgfqpoint{2.424694in}{6.240196in}}%
\pgfpathlineto{\pgfqpoint{2.417039in}{6.240196in}}%
\pgfpathlineto{\pgfqpoint{2.409384in}{6.240196in}}%
\pgfpathlineto{\pgfqpoint{2.401729in}{6.240196in}}%
\pgfpathlineto{\pgfqpoint{2.394075in}{6.240196in}}%
\pgfpathlineto{\pgfqpoint{2.386420in}{6.240196in}}%
\pgfpathlineto{\pgfqpoint{2.378765in}{6.240196in}}%
\pgfpathlineto{\pgfqpoint{2.371110in}{6.240196in}}%
\pgfpathlineto{\pgfqpoint{2.363455in}{6.240196in}}%
\pgfpathlineto{\pgfqpoint{2.355800in}{6.240196in}}%
\pgfpathlineto{\pgfqpoint{2.348146in}{6.240196in}}%
\pgfpathlineto{\pgfqpoint{2.340491in}{6.240196in}}%
\pgfpathlineto{\pgfqpoint{2.332836in}{6.240196in}}%
\pgfpathlineto{\pgfqpoint{2.325181in}{6.240196in}}%
\pgfpathlineto{\pgfqpoint{2.317526in}{6.240196in}}%
\pgfpathlineto{\pgfqpoint{2.309871in}{6.240196in}}%
\pgfpathlineto{\pgfqpoint{2.302217in}{6.240196in}}%
\pgfpathlineto{\pgfqpoint{2.294562in}{6.240196in}}%
\pgfpathlineto{\pgfqpoint{2.286907in}{6.240196in}}%
\pgfpathlineto{\pgfqpoint{2.279252in}{6.240196in}}%
\pgfpathlineto{\pgfqpoint{2.271597in}{6.240196in}}%
\pgfpathlineto{\pgfqpoint{2.263942in}{6.240196in}}%
\pgfpathlineto{\pgfqpoint{2.256288in}{6.240196in}}%
\pgfpathlineto{\pgfqpoint{2.248633in}{6.240196in}}%
\pgfpathlineto{\pgfqpoint{2.240978in}{6.240196in}}%
\pgfpathlineto{\pgfqpoint{2.233323in}{6.240196in}}%
\pgfpathlineto{\pgfqpoint{2.225668in}{6.240196in}}%
\pgfpathlineto{\pgfqpoint{2.218013in}{6.240196in}}%
\pgfpathlineto{\pgfqpoint{2.210359in}{6.240196in}}%
\pgfpathlineto{\pgfqpoint{2.202704in}{6.240196in}}%
\pgfpathlineto{\pgfqpoint{2.195049in}{6.240196in}}%
\pgfpathlineto{\pgfqpoint{2.187394in}{6.240196in}}%
\pgfpathlineto{\pgfqpoint{2.179739in}{6.240196in}}%
\pgfpathlineto{\pgfqpoint{2.172084in}{6.240196in}}%
\pgfpathlineto{\pgfqpoint{2.164430in}{6.240196in}}%
\pgfpathlineto{\pgfqpoint{2.156775in}{6.240196in}}%
\pgfpathlineto{\pgfqpoint{2.149120in}{6.240196in}}%
\pgfpathlineto{\pgfqpoint{2.141465in}{6.240196in}}%
\pgfpathlineto{\pgfqpoint{2.133810in}{6.240196in}}%
\pgfpathlineto{\pgfqpoint{2.126155in}{6.240196in}}%
\pgfpathlineto{\pgfqpoint{2.118501in}{6.240196in}}%
\pgfpathlineto{\pgfqpoint{2.110846in}{6.240196in}}%
\pgfpathlineto{\pgfqpoint{2.103191in}{6.240196in}}%
\pgfpathlineto{\pgfqpoint{2.095536in}{6.240196in}}%
\pgfpathlineto{\pgfqpoint{2.087881in}{6.240196in}}%
\pgfpathlineto{\pgfqpoint{2.080226in}{6.240196in}}%
\pgfpathlineto{\pgfqpoint{2.072572in}{6.240196in}}%
\pgfpathlineto{\pgfqpoint{2.064917in}{6.240196in}}%
\pgfpathlineto{\pgfqpoint{2.064917in}{6.240196in}}%
\pgfpathclose%
\pgfusepath{stroke,fill}%
}%
\begin{pgfscope}%
\pgfsys@transformshift{0.000000in}{0.000000in}%
\pgfsys@useobject{currentmarker}{}%
\end{pgfscope}%
\end{pgfscope}%
\begin{pgfscope}%
\pgfpathrectangle{\pgfqpoint{0.750000in}{3.960000in}}{\pgfqpoint{4.650000in}{3.080000in}}%
\pgfusepath{clip}%
\pgfsetbuttcap%
\pgfsetroundjoin%
\definecolor{currentfill}{rgb}{0.900000,0.900000,0.900000}%
\pgfsetfillcolor{currentfill}%
\pgfsetlinewidth{1.003750pt}%
\definecolor{currentstroke}{rgb}{0.500000,0.500000,0.500000}%
\pgfsetstrokecolor{currentstroke}%
\pgfsetdash{}{0pt}%
\pgfsys@defobject{currentmarker}{\pgfqpoint{2.440004in}{6.237573in}}{\pgfqpoint{4.085083in}{6.894836in}}{%
\pgfpathmoveto{\pgfqpoint{2.440004in}{6.240196in}}%
\pgfpathlineto{\pgfqpoint{2.440004in}{6.628879in}}%
\pgfpathlineto{\pgfqpoint{2.473577in}{6.655881in}}%
\pgfpathlineto{\pgfqpoint{2.507150in}{6.681496in}}%
\pgfpathlineto{\pgfqpoint{2.540723in}{6.705711in}}%
\pgfpathlineto{\pgfqpoint{2.574296in}{6.728513in}}%
\pgfpathlineto{\pgfqpoint{2.607869in}{6.749891in}}%
\pgfpathlineto{\pgfqpoint{2.641442in}{6.769834in}}%
\pgfpathlineto{\pgfqpoint{2.675015in}{6.788331in}}%
\pgfpathlineto{\pgfqpoint{2.708588in}{6.805373in}}%
\pgfpathlineto{\pgfqpoint{2.742161in}{6.820951in}}%
\pgfpathlineto{\pgfqpoint{2.775734in}{6.835058in}}%
\pgfpathlineto{\pgfqpoint{2.809307in}{6.847685in}}%
\pgfpathlineto{\pgfqpoint{2.842880in}{6.858826in}}%
\pgfpathlineto{\pgfqpoint{2.876453in}{6.868477in}}%
\pgfpathlineto{\pgfqpoint{2.910026in}{6.876630in}}%
\pgfpathlineto{\pgfqpoint{2.943599in}{6.883284in}}%
\pgfpathlineto{\pgfqpoint{2.977173in}{6.888433in}}%
\pgfpathlineto{\pgfqpoint{3.010746in}{6.892076in}}%
\pgfpathlineto{\pgfqpoint{3.044319in}{6.894211in}}%
\pgfpathlineto{\pgfqpoint{3.077892in}{6.894836in}}%
\pgfpathlineto{\pgfqpoint{3.111465in}{6.893951in}}%
\pgfpathlineto{\pgfqpoint{3.145038in}{6.891556in}}%
\pgfpathlineto{\pgfqpoint{3.178611in}{6.887654in}}%
\pgfpathlineto{\pgfqpoint{3.212184in}{6.882245in}}%
\pgfpathlineto{\pgfqpoint{3.245757in}{6.875333in}}%
\pgfpathlineto{\pgfqpoint{3.279330in}{6.866921in}}%
\pgfpathlineto{\pgfqpoint{3.312903in}{6.857013in}}%
\pgfpathlineto{\pgfqpoint{3.346476in}{6.845615in}}%
\pgfpathlineto{\pgfqpoint{3.380049in}{6.832733in}}%
\pgfpathlineto{\pgfqpoint{3.413622in}{6.818372in}}%
\pgfpathlineto{\pgfqpoint{3.447195in}{6.802541in}}%
\pgfpathlineto{\pgfqpoint{3.480768in}{6.785248in}}%
\pgfpathlineto{\pgfqpoint{3.514341in}{6.766501in}}%
\pgfpathlineto{\pgfqpoint{3.547914in}{6.746311in}}%
\pgfpathlineto{\pgfqpoint{3.581488in}{6.724686in}}%
\pgfpathlineto{\pgfqpoint{3.615061in}{6.701640in}}%
\pgfpathlineto{\pgfqpoint{3.648634in}{6.677183in}}%
\pgfpathlineto{\pgfqpoint{3.682207in}{6.651328in}}%
\pgfpathlineto{\pgfqpoint{3.715780in}{6.624089in}}%
\pgfpathlineto{\pgfqpoint{3.749353in}{6.595479in}}%
\pgfpathlineto{\pgfqpoint{3.782926in}{6.565513in}}%
\pgfpathlineto{\pgfqpoint{3.816499in}{6.534206in}}%
\pgfpathlineto{\pgfqpoint{3.850072in}{6.501576in}}%
\pgfpathlineto{\pgfqpoint{3.883645in}{6.467637in}}%
\pgfpathlineto{\pgfqpoint{3.917218in}{6.432409in}}%
\pgfpathlineto{\pgfqpoint{3.950791in}{6.395909in}}%
\pgfpathlineto{\pgfqpoint{3.984364in}{6.358155in}}%
\pgfpathlineto{\pgfqpoint{4.017937in}{6.319168in}}%
\pgfpathlineto{\pgfqpoint{4.051510in}{6.278967in}}%
\pgfpathlineto{\pgfqpoint{4.085083in}{6.237573in}}%
\pgfpathlineto{\pgfqpoint{4.085083in}{6.240196in}}%
\pgfpathlineto{\pgfqpoint{4.085083in}{6.240196in}}%
\pgfpathlineto{\pgfqpoint{4.051510in}{6.240196in}}%
\pgfpathlineto{\pgfqpoint{4.017937in}{6.240196in}}%
\pgfpathlineto{\pgfqpoint{3.984364in}{6.240196in}}%
\pgfpathlineto{\pgfqpoint{3.950791in}{6.240196in}}%
\pgfpathlineto{\pgfqpoint{3.917218in}{6.240196in}}%
\pgfpathlineto{\pgfqpoint{3.883645in}{6.240196in}}%
\pgfpathlineto{\pgfqpoint{3.850072in}{6.240196in}}%
\pgfpathlineto{\pgfqpoint{3.816499in}{6.240196in}}%
\pgfpathlineto{\pgfqpoint{3.782926in}{6.240196in}}%
\pgfpathlineto{\pgfqpoint{3.749353in}{6.240196in}}%
\pgfpathlineto{\pgfqpoint{3.715780in}{6.240196in}}%
\pgfpathlineto{\pgfqpoint{3.682207in}{6.240196in}}%
\pgfpathlineto{\pgfqpoint{3.648634in}{6.240196in}}%
\pgfpathlineto{\pgfqpoint{3.615061in}{6.240196in}}%
\pgfpathlineto{\pgfqpoint{3.581488in}{6.240196in}}%
\pgfpathlineto{\pgfqpoint{3.547914in}{6.240196in}}%
\pgfpathlineto{\pgfqpoint{3.514341in}{6.240196in}}%
\pgfpathlineto{\pgfqpoint{3.480768in}{6.240196in}}%
\pgfpathlineto{\pgfqpoint{3.447195in}{6.240196in}}%
\pgfpathlineto{\pgfqpoint{3.413622in}{6.240196in}}%
\pgfpathlineto{\pgfqpoint{3.380049in}{6.240196in}}%
\pgfpathlineto{\pgfqpoint{3.346476in}{6.240196in}}%
\pgfpathlineto{\pgfqpoint{3.312903in}{6.240196in}}%
\pgfpathlineto{\pgfqpoint{3.279330in}{6.240196in}}%
\pgfpathlineto{\pgfqpoint{3.245757in}{6.240196in}}%
\pgfpathlineto{\pgfqpoint{3.212184in}{6.240196in}}%
\pgfpathlineto{\pgfqpoint{3.178611in}{6.240196in}}%
\pgfpathlineto{\pgfqpoint{3.145038in}{6.240196in}}%
\pgfpathlineto{\pgfqpoint{3.111465in}{6.240196in}}%
\pgfpathlineto{\pgfqpoint{3.077892in}{6.240196in}}%
\pgfpathlineto{\pgfqpoint{3.044319in}{6.240196in}}%
\pgfpathlineto{\pgfqpoint{3.010746in}{6.240196in}}%
\pgfpathlineto{\pgfqpoint{2.977173in}{6.240196in}}%
\pgfpathlineto{\pgfqpoint{2.943599in}{6.240196in}}%
\pgfpathlineto{\pgfqpoint{2.910026in}{6.240196in}}%
\pgfpathlineto{\pgfqpoint{2.876453in}{6.240196in}}%
\pgfpathlineto{\pgfqpoint{2.842880in}{6.240196in}}%
\pgfpathlineto{\pgfqpoint{2.809307in}{6.240196in}}%
\pgfpathlineto{\pgfqpoint{2.775734in}{6.240196in}}%
\pgfpathlineto{\pgfqpoint{2.742161in}{6.240196in}}%
\pgfpathlineto{\pgfqpoint{2.708588in}{6.240196in}}%
\pgfpathlineto{\pgfqpoint{2.675015in}{6.240196in}}%
\pgfpathlineto{\pgfqpoint{2.641442in}{6.240196in}}%
\pgfpathlineto{\pgfqpoint{2.607869in}{6.240196in}}%
\pgfpathlineto{\pgfqpoint{2.574296in}{6.240196in}}%
\pgfpathlineto{\pgfqpoint{2.540723in}{6.240196in}}%
\pgfpathlineto{\pgfqpoint{2.507150in}{6.240196in}}%
\pgfpathlineto{\pgfqpoint{2.473577in}{6.240196in}}%
\pgfpathlineto{\pgfqpoint{2.440004in}{6.240196in}}%
\pgfpathlineto{\pgfqpoint{2.440004in}{6.240196in}}%
\pgfpathclose%
\pgfusepath{stroke,fill}%
}%
\begin{pgfscope}%
\pgfsys@transformshift{0.000000in}{0.000000in}%
\pgfsys@useobject{currentmarker}{}%
\end{pgfscope}%
\end{pgfscope}%
\begin{pgfscope}%
\pgfpathrectangle{\pgfqpoint{0.750000in}{3.960000in}}{\pgfqpoint{4.650000in}{3.080000in}}%
\pgfusepath{clip}%
\pgfsetrectcap%
\pgfsetroundjoin%
\pgfsetlinewidth{0.803000pt}%
\definecolor{currentstroke}{rgb}{0.690196,0.690196,0.690196}%
\pgfsetstrokecolor{currentstroke}%
\pgfsetdash{}{0pt}%
\pgfpathmoveto{\pgfqpoint{0.750000in}{3.960000in}}%
\pgfpathlineto{\pgfqpoint{0.750000in}{7.040000in}}%
\pgfusepath{stroke}%
\end{pgfscope}%
\begin{pgfscope}%
\pgfsetbuttcap%
\pgfsetroundjoin%
\definecolor{currentfill}{rgb}{0.000000,0.000000,0.000000}%
\pgfsetfillcolor{currentfill}%
\pgfsetlinewidth{0.803000pt}%
\definecolor{currentstroke}{rgb}{0.000000,0.000000,0.000000}%
\pgfsetstrokecolor{currentstroke}%
\pgfsetdash{}{0pt}%
\pgfsys@defobject{currentmarker}{\pgfqpoint{0.000000in}{-0.048611in}}{\pgfqpoint{0.000000in}{0.000000in}}{%
\pgfpathmoveto{\pgfqpoint{0.000000in}{0.000000in}}%
\pgfpathlineto{\pgfqpoint{0.000000in}{-0.048611in}}%
\pgfusepath{stroke,fill}%
}%
\begin{pgfscope}%
\pgfsys@transformshift{0.750000in}{3.960000in}%
\pgfsys@useobject{currentmarker}{}%
\end{pgfscope}%
\end{pgfscope}%
\begin{pgfscope}%
\pgfpathrectangle{\pgfqpoint{0.750000in}{3.960000in}}{\pgfqpoint{4.650000in}{3.080000in}}%
\pgfusepath{clip}%
\pgfsetrectcap%
\pgfsetroundjoin%
\pgfsetlinewidth{0.803000pt}%
\definecolor{currentstroke}{rgb}{0.690196,0.690196,0.690196}%
\pgfsetstrokecolor{currentstroke}%
\pgfsetdash{}{0pt}%
\pgfpathmoveto{\pgfqpoint{1.266667in}{3.960000in}}%
\pgfpathlineto{\pgfqpoint{1.266667in}{7.040000in}}%
\pgfusepath{stroke}%
\end{pgfscope}%
\begin{pgfscope}%
\pgfsetbuttcap%
\pgfsetroundjoin%
\definecolor{currentfill}{rgb}{0.000000,0.000000,0.000000}%
\pgfsetfillcolor{currentfill}%
\pgfsetlinewidth{0.803000pt}%
\definecolor{currentstroke}{rgb}{0.000000,0.000000,0.000000}%
\pgfsetstrokecolor{currentstroke}%
\pgfsetdash{}{0pt}%
\pgfsys@defobject{currentmarker}{\pgfqpoint{0.000000in}{-0.048611in}}{\pgfqpoint{0.000000in}{0.000000in}}{%
\pgfpathmoveto{\pgfqpoint{0.000000in}{0.000000in}}%
\pgfpathlineto{\pgfqpoint{0.000000in}{-0.048611in}}%
\pgfusepath{stroke,fill}%
}%
\begin{pgfscope}%
\pgfsys@transformshift{1.266667in}{3.960000in}%
\pgfsys@useobject{currentmarker}{}%
\end{pgfscope}%
\end{pgfscope}%
\begin{pgfscope}%
\pgfpathrectangle{\pgfqpoint{0.750000in}{3.960000in}}{\pgfqpoint{4.650000in}{3.080000in}}%
\pgfusepath{clip}%
\pgfsetrectcap%
\pgfsetroundjoin%
\pgfsetlinewidth{0.803000pt}%
\definecolor{currentstroke}{rgb}{0.690196,0.690196,0.690196}%
\pgfsetstrokecolor{currentstroke}%
\pgfsetdash{}{0pt}%
\pgfpathmoveto{\pgfqpoint{1.783333in}{3.960000in}}%
\pgfpathlineto{\pgfqpoint{1.783333in}{7.040000in}}%
\pgfusepath{stroke}%
\end{pgfscope}%
\begin{pgfscope}%
\pgfsetbuttcap%
\pgfsetroundjoin%
\definecolor{currentfill}{rgb}{0.000000,0.000000,0.000000}%
\pgfsetfillcolor{currentfill}%
\pgfsetlinewidth{0.803000pt}%
\definecolor{currentstroke}{rgb}{0.000000,0.000000,0.000000}%
\pgfsetstrokecolor{currentstroke}%
\pgfsetdash{}{0pt}%
\pgfsys@defobject{currentmarker}{\pgfqpoint{0.000000in}{-0.048611in}}{\pgfqpoint{0.000000in}{0.000000in}}{%
\pgfpathmoveto{\pgfqpoint{0.000000in}{0.000000in}}%
\pgfpathlineto{\pgfqpoint{0.000000in}{-0.048611in}}%
\pgfusepath{stroke,fill}%
}%
\begin{pgfscope}%
\pgfsys@transformshift{1.783333in}{3.960000in}%
\pgfsys@useobject{currentmarker}{}%
\end{pgfscope}%
\end{pgfscope}%
\begin{pgfscope}%
\pgfpathrectangle{\pgfqpoint{0.750000in}{3.960000in}}{\pgfqpoint{4.650000in}{3.080000in}}%
\pgfusepath{clip}%
\pgfsetrectcap%
\pgfsetroundjoin%
\pgfsetlinewidth{0.803000pt}%
\definecolor{currentstroke}{rgb}{0.690196,0.690196,0.690196}%
\pgfsetstrokecolor{currentstroke}%
\pgfsetdash{}{0pt}%
\pgfpathmoveto{\pgfqpoint{2.300000in}{3.960000in}}%
\pgfpathlineto{\pgfqpoint{2.300000in}{7.040000in}}%
\pgfusepath{stroke}%
\end{pgfscope}%
\begin{pgfscope}%
\pgfsetbuttcap%
\pgfsetroundjoin%
\definecolor{currentfill}{rgb}{0.000000,0.000000,0.000000}%
\pgfsetfillcolor{currentfill}%
\pgfsetlinewidth{0.803000pt}%
\definecolor{currentstroke}{rgb}{0.000000,0.000000,0.000000}%
\pgfsetstrokecolor{currentstroke}%
\pgfsetdash{}{0pt}%
\pgfsys@defobject{currentmarker}{\pgfqpoint{0.000000in}{-0.048611in}}{\pgfqpoint{0.000000in}{0.000000in}}{%
\pgfpathmoveto{\pgfqpoint{0.000000in}{0.000000in}}%
\pgfpathlineto{\pgfqpoint{0.000000in}{-0.048611in}}%
\pgfusepath{stroke,fill}%
}%
\begin{pgfscope}%
\pgfsys@transformshift{2.300000in}{3.960000in}%
\pgfsys@useobject{currentmarker}{}%
\end{pgfscope}%
\end{pgfscope}%
\begin{pgfscope}%
\pgfpathrectangle{\pgfqpoint{0.750000in}{3.960000in}}{\pgfqpoint{4.650000in}{3.080000in}}%
\pgfusepath{clip}%
\pgfsetrectcap%
\pgfsetroundjoin%
\pgfsetlinewidth{0.803000pt}%
\definecolor{currentstroke}{rgb}{0.690196,0.690196,0.690196}%
\pgfsetstrokecolor{currentstroke}%
\pgfsetdash{}{0pt}%
\pgfpathmoveto{\pgfqpoint{2.816667in}{3.960000in}}%
\pgfpathlineto{\pgfqpoint{2.816667in}{7.040000in}}%
\pgfusepath{stroke}%
\end{pgfscope}%
\begin{pgfscope}%
\pgfsetbuttcap%
\pgfsetroundjoin%
\definecolor{currentfill}{rgb}{0.000000,0.000000,0.000000}%
\pgfsetfillcolor{currentfill}%
\pgfsetlinewidth{0.803000pt}%
\definecolor{currentstroke}{rgb}{0.000000,0.000000,0.000000}%
\pgfsetstrokecolor{currentstroke}%
\pgfsetdash{}{0pt}%
\pgfsys@defobject{currentmarker}{\pgfqpoint{0.000000in}{-0.048611in}}{\pgfqpoint{0.000000in}{0.000000in}}{%
\pgfpathmoveto{\pgfqpoint{0.000000in}{0.000000in}}%
\pgfpathlineto{\pgfqpoint{0.000000in}{-0.048611in}}%
\pgfusepath{stroke,fill}%
}%
\begin{pgfscope}%
\pgfsys@transformshift{2.816667in}{3.960000in}%
\pgfsys@useobject{currentmarker}{}%
\end{pgfscope}%
\end{pgfscope}%
\begin{pgfscope}%
\pgfpathrectangle{\pgfqpoint{0.750000in}{3.960000in}}{\pgfqpoint{4.650000in}{3.080000in}}%
\pgfusepath{clip}%
\pgfsetrectcap%
\pgfsetroundjoin%
\pgfsetlinewidth{0.803000pt}%
\definecolor{currentstroke}{rgb}{0.690196,0.690196,0.690196}%
\pgfsetstrokecolor{currentstroke}%
\pgfsetdash{}{0pt}%
\pgfpathmoveto{\pgfqpoint{3.333333in}{3.960000in}}%
\pgfpathlineto{\pgfqpoint{3.333333in}{7.040000in}}%
\pgfusepath{stroke}%
\end{pgfscope}%
\begin{pgfscope}%
\pgfsetbuttcap%
\pgfsetroundjoin%
\definecolor{currentfill}{rgb}{0.000000,0.000000,0.000000}%
\pgfsetfillcolor{currentfill}%
\pgfsetlinewidth{0.803000pt}%
\definecolor{currentstroke}{rgb}{0.000000,0.000000,0.000000}%
\pgfsetstrokecolor{currentstroke}%
\pgfsetdash{}{0pt}%
\pgfsys@defobject{currentmarker}{\pgfqpoint{0.000000in}{-0.048611in}}{\pgfqpoint{0.000000in}{0.000000in}}{%
\pgfpathmoveto{\pgfqpoint{0.000000in}{0.000000in}}%
\pgfpathlineto{\pgfqpoint{0.000000in}{-0.048611in}}%
\pgfusepath{stroke,fill}%
}%
\begin{pgfscope}%
\pgfsys@transformshift{3.333333in}{3.960000in}%
\pgfsys@useobject{currentmarker}{}%
\end{pgfscope}%
\end{pgfscope}%
\begin{pgfscope}%
\pgfpathrectangle{\pgfqpoint{0.750000in}{3.960000in}}{\pgfqpoint{4.650000in}{3.080000in}}%
\pgfusepath{clip}%
\pgfsetrectcap%
\pgfsetroundjoin%
\pgfsetlinewidth{0.803000pt}%
\definecolor{currentstroke}{rgb}{0.690196,0.690196,0.690196}%
\pgfsetstrokecolor{currentstroke}%
\pgfsetdash{}{0pt}%
\pgfpathmoveto{\pgfqpoint{3.850000in}{3.960000in}}%
\pgfpathlineto{\pgfqpoint{3.850000in}{7.040000in}}%
\pgfusepath{stroke}%
\end{pgfscope}%
\begin{pgfscope}%
\pgfsetbuttcap%
\pgfsetroundjoin%
\definecolor{currentfill}{rgb}{0.000000,0.000000,0.000000}%
\pgfsetfillcolor{currentfill}%
\pgfsetlinewidth{0.803000pt}%
\definecolor{currentstroke}{rgb}{0.000000,0.000000,0.000000}%
\pgfsetstrokecolor{currentstroke}%
\pgfsetdash{}{0pt}%
\pgfsys@defobject{currentmarker}{\pgfqpoint{0.000000in}{-0.048611in}}{\pgfqpoint{0.000000in}{0.000000in}}{%
\pgfpathmoveto{\pgfqpoint{0.000000in}{0.000000in}}%
\pgfpathlineto{\pgfqpoint{0.000000in}{-0.048611in}}%
\pgfusepath{stroke,fill}%
}%
\begin{pgfscope}%
\pgfsys@transformshift{3.850000in}{3.960000in}%
\pgfsys@useobject{currentmarker}{}%
\end{pgfscope}%
\end{pgfscope}%
\begin{pgfscope}%
\pgfpathrectangle{\pgfqpoint{0.750000in}{3.960000in}}{\pgfqpoint{4.650000in}{3.080000in}}%
\pgfusepath{clip}%
\pgfsetrectcap%
\pgfsetroundjoin%
\pgfsetlinewidth{0.803000pt}%
\definecolor{currentstroke}{rgb}{0.690196,0.690196,0.690196}%
\pgfsetstrokecolor{currentstroke}%
\pgfsetdash{}{0pt}%
\pgfpathmoveto{\pgfqpoint{4.366667in}{3.960000in}}%
\pgfpathlineto{\pgfqpoint{4.366667in}{7.040000in}}%
\pgfusepath{stroke}%
\end{pgfscope}%
\begin{pgfscope}%
\pgfsetbuttcap%
\pgfsetroundjoin%
\definecolor{currentfill}{rgb}{0.000000,0.000000,0.000000}%
\pgfsetfillcolor{currentfill}%
\pgfsetlinewidth{0.803000pt}%
\definecolor{currentstroke}{rgb}{0.000000,0.000000,0.000000}%
\pgfsetstrokecolor{currentstroke}%
\pgfsetdash{}{0pt}%
\pgfsys@defobject{currentmarker}{\pgfqpoint{0.000000in}{-0.048611in}}{\pgfqpoint{0.000000in}{0.000000in}}{%
\pgfpathmoveto{\pgfqpoint{0.000000in}{0.000000in}}%
\pgfpathlineto{\pgfqpoint{0.000000in}{-0.048611in}}%
\pgfusepath{stroke,fill}%
}%
\begin{pgfscope}%
\pgfsys@transformshift{4.366667in}{3.960000in}%
\pgfsys@useobject{currentmarker}{}%
\end{pgfscope}%
\end{pgfscope}%
\begin{pgfscope}%
\pgfpathrectangle{\pgfqpoint{0.750000in}{3.960000in}}{\pgfqpoint{4.650000in}{3.080000in}}%
\pgfusepath{clip}%
\pgfsetrectcap%
\pgfsetroundjoin%
\pgfsetlinewidth{0.803000pt}%
\definecolor{currentstroke}{rgb}{0.690196,0.690196,0.690196}%
\pgfsetstrokecolor{currentstroke}%
\pgfsetdash{}{0pt}%
\pgfpathmoveto{\pgfqpoint{4.883333in}{3.960000in}}%
\pgfpathlineto{\pgfqpoint{4.883333in}{7.040000in}}%
\pgfusepath{stroke}%
\end{pgfscope}%
\begin{pgfscope}%
\pgfsetbuttcap%
\pgfsetroundjoin%
\definecolor{currentfill}{rgb}{0.000000,0.000000,0.000000}%
\pgfsetfillcolor{currentfill}%
\pgfsetlinewidth{0.803000pt}%
\definecolor{currentstroke}{rgb}{0.000000,0.000000,0.000000}%
\pgfsetstrokecolor{currentstroke}%
\pgfsetdash{}{0pt}%
\pgfsys@defobject{currentmarker}{\pgfqpoint{0.000000in}{-0.048611in}}{\pgfqpoint{0.000000in}{0.000000in}}{%
\pgfpathmoveto{\pgfqpoint{0.000000in}{0.000000in}}%
\pgfpathlineto{\pgfqpoint{0.000000in}{-0.048611in}}%
\pgfusepath{stroke,fill}%
}%
\begin{pgfscope}%
\pgfsys@transformshift{4.883333in}{3.960000in}%
\pgfsys@useobject{currentmarker}{}%
\end{pgfscope}%
\end{pgfscope}%
\begin{pgfscope}%
\pgfpathrectangle{\pgfqpoint{0.750000in}{3.960000in}}{\pgfqpoint{4.650000in}{3.080000in}}%
\pgfusepath{clip}%
\pgfsetrectcap%
\pgfsetroundjoin%
\pgfsetlinewidth{0.803000pt}%
\definecolor{currentstroke}{rgb}{0.690196,0.690196,0.690196}%
\pgfsetstrokecolor{currentstroke}%
\pgfsetdash{}{0pt}%
\pgfpathmoveto{\pgfqpoint{5.400000in}{3.960000in}}%
\pgfpathlineto{\pgfqpoint{5.400000in}{7.040000in}}%
\pgfusepath{stroke}%
\end{pgfscope}%
\begin{pgfscope}%
\pgfsetbuttcap%
\pgfsetroundjoin%
\definecolor{currentfill}{rgb}{0.000000,0.000000,0.000000}%
\pgfsetfillcolor{currentfill}%
\pgfsetlinewidth{0.803000pt}%
\definecolor{currentstroke}{rgb}{0.000000,0.000000,0.000000}%
\pgfsetstrokecolor{currentstroke}%
\pgfsetdash{}{0pt}%
\pgfsys@defobject{currentmarker}{\pgfqpoint{0.000000in}{-0.048611in}}{\pgfqpoint{0.000000in}{0.000000in}}{%
\pgfpathmoveto{\pgfqpoint{0.000000in}{0.000000in}}%
\pgfpathlineto{\pgfqpoint{0.000000in}{-0.048611in}}%
\pgfusepath{stroke,fill}%
}%
\begin{pgfscope}%
\pgfsys@transformshift{5.400000in}{3.960000in}%
\pgfsys@useobject{currentmarker}{}%
\end{pgfscope}%
\end{pgfscope}%
\begin{pgfscope}%
\pgfpathrectangle{\pgfqpoint{0.750000in}{3.960000in}}{\pgfqpoint{4.650000in}{3.080000in}}%
\pgfusepath{clip}%
\pgfsetrectcap%
\pgfsetroundjoin%
\pgfsetlinewidth{0.803000pt}%
\definecolor{currentstroke}{rgb}{0.690196,0.690196,0.690196}%
\pgfsetstrokecolor{currentstroke}%
\pgfsetdash{}{0pt}%
\pgfpathmoveto{\pgfqpoint{0.750000in}{3.960000in}}%
\pgfpathlineto{\pgfqpoint{5.400000in}{3.960000in}}%
\pgfusepath{stroke}%
\end{pgfscope}%
\begin{pgfscope}%
\pgfsetbuttcap%
\pgfsetroundjoin%
\definecolor{currentfill}{rgb}{0.000000,0.000000,0.000000}%
\pgfsetfillcolor{currentfill}%
\pgfsetlinewidth{0.803000pt}%
\definecolor{currentstroke}{rgb}{0.000000,0.000000,0.000000}%
\pgfsetstrokecolor{currentstroke}%
\pgfsetdash{}{0pt}%
\pgfsys@defobject{currentmarker}{\pgfqpoint{-0.048611in}{0.000000in}}{\pgfqpoint{-0.000000in}{0.000000in}}{%
\pgfpathmoveto{\pgfqpoint{-0.000000in}{0.000000in}}%
\pgfpathlineto{\pgfqpoint{-0.048611in}{0.000000in}}%
\pgfusepath{stroke,fill}%
}%
\begin{pgfscope}%
\pgfsys@transformshift{0.750000in}{3.960000in}%
\pgfsys@useobject{currentmarker}{}%
\end{pgfscope}%
\end{pgfscope}%
\begin{pgfscope}%
\definecolor{textcolor}{rgb}{0.000000,0.000000,0.000000}%
\pgfsetstrokecolor{textcolor}%
\pgfsetfillcolor{textcolor}%
\pgftext[x=0.444254in, y=3.898680in, left, base]{\color{textcolor}\rmfamily\fontsize{12.000000}{14.400000}\selectfont \(\displaystyle {0.0}\)}%
\end{pgfscope}%
\begin{pgfscope}%
\pgfpathrectangle{\pgfqpoint{0.750000in}{3.960000in}}{\pgfqpoint{4.650000in}{3.080000in}}%
\pgfusepath{clip}%
\pgfsetrectcap%
\pgfsetroundjoin%
\pgfsetlinewidth{0.803000pt}%
\definecolor{currentstroke}{rgb}{0.690196,0.690196,0.690196}%
\pgfsetstrokecolor{currentstroke}%
\pgfsetdash{}{0pt}%
\pgfpathmoveto{\pgfqpoint{0.750000in}{4.462145in}}%
\pgfpathlineto{\pgfqpoint{5.400000in}{4.462145in}}%
\pgfusepath{stroke}%
\end{pgfscope}%
\begin{pgfscope}%
\pgfsetbuttcap%
\pgfsetroundjoin%
\definecolor{currentfill}{rgb}{0.000000,0.000000,0.000000}%
\pgfsetfillcolor{currentfill}%
\pgfsetlinewidth{0.803000pt}%
\definecolor{currentstroke}{rgb}{0.000000,0.000000,0.000000}%
\pgfsetstrokecolor{currentstroke}%
\pgfsetdash{}{0pt}%
\pgfsys@defobject{currentmarker}{\pgfqpoint{-0.048611in}{0.000000in}}{\pgfqpoint{-0.000000in}{0.000000in}}{%
\pgfpathmoveto{\pgfqpoint{-0.000000in}{0.000000in}}%
\pgfpathlineto{\pgfqpoint{-0.048611in}{0.000000in}}%
\pgfusepath{stroke,fill}%
}%
\begin{pgfscope}%
\pgfsys@transformshift{0.750000in}{4.462145in}%
\pgfsys@useobject{currentmarker}{}%
\end{pgfscope}%
\end{pgfscope}%
\begin{pgfscope}%
\definecolor{textcolor}{rgb}{0.000000,0.000000,0.000000}%
\pgfsetstrokecolor{textcolor}%
\pgfsetfillcolor{textcolor}%
\pgftext[x=0.444254in, y=4.400825in, left, base]{\color{textcolor}\rmfamily\fontsize{12.000000}{14.400000}\selectfont \(\displaystyle {0.2}\)}%
\end{pgfscope}%
\begin{pgfscope}%
\pgfpathrectangle{\pgfqpoint{0.750000in}{3.960000in}}{\pgfqpoint{4.650000in}{3.080000in}}%
\pgfusepath{clip}%
\pgfsetrectcap%
\pgfsetroundjoin%
\pgfsetlinewidth{0.803000pt}%
\definecolor{currentstroke}{rgb}{0.690196,0.690196,0.690196}%
\pgfsetstrokecolor{currentstroke}%
\pgfsetdash{}{0pt}%
\pgfpathmoveto{\pgfqpoint{0.750000in}{4.964291in}}%
\pgfpathlineto{\pgfqpoint{5.400000in}{4.964291in}}%
\pgfusepath{stroke}%
\end{pgfscope}%
\begin{pgfscope}%
\pgfsetbuttcap%
\pgfsetroundjoin%
\definecolor{currentfill}{rgb}{0.000000,0.000000,0.000000}%
\pgfsetfillcolor{currentfill}%
\pgfsetlinewidth{0.803000pt}%
\definecolor{currentstroke}{rgb}{0.000000,0.000000,0.000000}%
\pgfsetstrokecolor{currentstroke}%
\pgfsetdash{}{0pt}%
\pgfsys@defobject{currentmarker}{\pgfqpoint{-0.048611in}{0.000000in}}{\pgfqpoint{-0.000000in}{0.000000in}}{%
\pgfpathmoveto{\pgfqpoint{-0.000000in}{0.000000in}}%
\pgfpathlineto{\pgfqpoint{-0.048611in}{0.000000in}}%
\pgfusepath{stroke,fill}%
}%
\begin{pgfscope}%
\pgfsys@transformshift{0.750000in}{4.964291in}%
\pgfsys@useobject{currentmarker}{}%
\end{pgfscope}%
\end{pgfscope}%
\begin{pgfscope}%
\definecolor{textcolor}{rgb}{0.000000,0.000000,0.000000}%
\pgfsetstrokecolor{textcolor}%
\pgfsetfillcolor{textcolor}%
\pgftext[x=0.444254in, y=4.902971in, left, base]{\color{textcolor}\rmfamily\fontsize{12.000000}{14.400000}\selectfont \(\displaystyle {0.4}\)}%
\end{pgfscope}%
\begin{pgfscope}%
\pgfpathrectangle{\pgfqpoint{0.750000in}{3.960000in}}{\pgfqpoint{4.650000in}{3.080000in}}%
\pgfusepath{clip}%
\pgfsetrectcap%
\pgfsetroundjoin%
\pgfsetlinewidth{0.803000pt}%
\definecolor{currentstroke}{rgb}{0.690196,0.690196,0.690196}%
\pgfsetstrokecolor{currentstroke}%
\pgfsetdash{}{0pt}%
\pgfpathmoveto{\pgfqpoint{0.750000in}{5.466436in}}%
\pgfpathlineto{\pgfqpoint{5.400000in}{5.466436in}}%
\pgfusepath{stroke}%
\end{pgfscope}%
\begin{pgfscope}%
\pgfsetbuttcap%
\pgfsetroundjoin%
\definecolor{currentfill}{rgb}{0.000000,0.000000,0.000000}%
\pgfsetfillcolor{currentfill}%
\pgfsetlinewidth{0.803000pt}%
\definecolor{currentstroke}{rgb}{0.000000,0.000000,0.000000}%
\pgfsetstrokecolor{currentstroke}%
\pgfsetdash{}{0pt}%
\pgfsys@defobject{currentmarker}{\pgfqpoint{-0.048611in}{0.000000in}}{\pgfqpoint{-0.000000in}{0.000000in}}{%
\pgfpathmoveto{\pgfqpoint{-0.000000in}{0.000000in}}%
\pgfpathlineto{\pgfqpoint{-0.048611in}{0.000000in}}%
\pgfusepath{stroke,fill}%
}%
\begin{pgfscope}%
\pgfsys@transformshift{0.750000in}{5.466436in}%
\pgfsys@useobject{currentmarker}{}%
\end{pgfscope}%
\end{pgfscope}%
\begin{pgfscope}%
\definecolor{textcolor}{rgb}{0.000000,0.000000,0.000000}%
\pgfsetstrokecolor{textcolor}%
\pgfsetfillcolor{textcolor}%
\pgftext[x=0.444254in, y=5.405116in, left, base]{\color{textcolor}\rmfamily\fontsize{12.000000}{14.400000}\selectfont \(\displaystyle {0.6}\)}%
\end{pgfscope}%
\begin{pgfscope}%
\pgfpathrectangle{\pgfqpoint{0.750000in}{3.960000in}}{\pgfqpoint{4.650000in}{3.080000in}}%
\pgfusepath{clip}%
\pgfsetrectcap%
\pgfsetroundjoin%
\pgfsetlinewidth{0.803000pt}%
\definecolor{currentstroke}{rgb}{0.690196,0.690196,0.690196}%
\pgfsetstrokecolor{currentstroke}%
\pgfsetdash{}{0pt}%
\pgfpathmoveto{\pgfqpoint{0.750000in}{5.968581in}}%
\pgfpathlineto{\pgfqpoint{5.400000in}{5.968581in}}%
\pgfusepath{stroke}%
\end{pgfscope}%
\begin{pgfscope}%
\pgfsetbuttcap%
\pgfsetroundjoin%
\definecolor{currentfill}{rgb}{0.000000,0.000000,0.000000}%
\pgfsetfillcolor{currentfill}%
\pgfsetlinewidth{0.803000pt}%
\definecolor{currentstroke}{rgb}{0.000000,0.000000,0.000000}%
\pgfsetstrokecolor{currentstroke}%
\pgfsetdash{}{0pt}%
\pgfsys@defobject{currentmarker}{\pgfqpoint{-0.048611in}{0.000000in}}{\pgfqpoint{-0.000000in}{0.000000in}}{%
\pgfpathmoveto{\pgfqpoint{-0.000000in}{0.000000in}}%
\pgfpathlineto{\pgfqpoint{-0.048611in}{0.000000in}}%
\pgfusepath{stroke,fill}%
}%
\begin{pgfscope}%
\pgfsys@transformshift{0.750000in}{5.968581in}%
\pgfsys@useobject{currentmarker}{}%
\end{pgfscope}%
\end{pgfscope}%
\begin{pgfscope}%
\definecolor{textcolor}{rgb}{0.000000,0.000000,0.000000}%
\pgfsetstrokecolor{textcolor}%
\pgfsetfillcolor{textcolor}%
\pgftext[x=0.444254in, y=5.907261in, left, base]{\color{textcolor}\rmfamily\fontsize{12.000000}{14.400000}\selectfont \(\displaystyle {0.8}\)}%
\end{pgfscope}%
\begin{pgfscope}%
\pgfpathrectangle{\pgfqpoint{0.750000in}{3.960000in}}{\pgfqpoint{4.650000in}{3.080000in}}%
\pgfusepath{clip}%
\pgfsetrectcap%
\pgfsetroundjoin%
\pgfsetlinewidth{0.803000pt}%
\definecolor{currentstroke}{rgb}{0.690196,0.690196,0.690196}%
\pgfsetstrokecolor{currentstroke}%
\pgfsetdash{}{0pt}%
\pgfpathmoveto{\pgfqpoint{0.750000in}{6.470727in}}%
\pgfpathlineto{\pgfqpoint{5.400000in}{6.470727in}}%
\pgfusepath{stroke}%
\end{pgfscope}%
\begin{pgfscope}%
\pgfsetbuttcap%
\pgfsetroundjoin%
\definecolor{currentfill}{rgb}{0.000000,0.000000,0.000000}%
\pgfsetfillcolor{currentfill}%
\pgfsetlinewidth{0.803000pt}%
\definecolor{currentstroke}{rgb}{0.000000,0.000000,0.000000}%
\pgfsetstrokecolor{currentstroke}%
\pgfsetdash{}{0pt}%
\pgfsys@defobject{currentmarker}{\pgfqpoint{-0.048611in}{0.000000in}}{\pgfqpoint{-0.000000in}{0.000000in}}{%
\pgfpathmoveto{\pgfqpoint{-0.000000in}{0.000000in}}%
\pgfpathlineto{\pgfqpoint{-0.048611in}{0.000000in}}%
\pgfusepath{stroke,fill}%
}%
\begin{pgfscope}%
\pgfsys@transformshift{0.750000in}{6.470727in}%
\pgfsys@useobject{currentmarker}{}%
\end{pgfscope}%
\end{pgfscope}%
\begin{pgfscope}%
\definecolor{textcolor}{rgb}{0.000000,0.000000,0.000000}%
\pgfsetstrokecolor{textcolor}%
\pgfsetfillcolor{textcolor}%
\pgftext[x=0.444254in, y=6.409407in, left, base]{\color{textcolor}\rmfamily\fontsize{12.000000}{14.400000}\selectfont \(\displaystyle {1.0}\)}%
\end{pgfscope}%
\begin{pgfscope}%
\pgfpathrectangle{\pgfqpoint{0.750000in}{3.960000in}}{\pgfqpoint{4.650000in}{3.080000in}}%
\pgfusepath{clip}%
\pgfsetrectcap%
\pgfsetroundjoin%
\pgfsetlinewidth{0.803000pt}%
\definecolor{currentstroke}{rgb}{0.690196,0.690196,0.690196}%
\pgfsetstrokecolor{currentstroke}%
\pgfsetdash{}{0pt}%
\pgfpathmoveto{\pgfqpoint{0.750000in}{6.972872in}}%
\pgfpathlineto{\pgfqpoint{5.400000in}{6.972872in}}%
\pgfusepath{stroke}%
\end{pgfscope}%
\begin{pgfscope}%
\pgfsetbuttcap%
\pgfsetroundjoin%
\definecolor{currentfill}{rgb}{0.000000,0.000000,0.000000}%
\pgfsetfillcolor{currentfill}%
\pgfsetlinewidth{0.803000pt}%
\definecolor{currentstroke}{rgb}{0.000000,0.000000,0.000000}%
\pgfsetstrokecolor{currentstroke}%
\pgfsetdash{}{0pt}%
\pgfsys@defobject{currentmarker}{\pgfqpoint{-0.048611in}{0.000000in}}{\pgfqpoint{-0.000000in}{0.000000in}}{%
\pgfpathmoveto{\pgfqpoint{-0.000000in}{0.000000in}}%
\pgfpathlineto{\pgfqpoint{-0.048611in}{0.000000in}}%
\pgfusepath{stroke,fill}%
}%
\begin{pgfscope}%
\pgfsys@transformshift{0.750000in}{6.972872in}%
\pgfsys@useobject{currentmarker}{}%
\end{pgfscope}%
\end{pgfscope}%
\begin{pgfscope}%
\definecolor{textcolor}{rgb}{0.000000,0.000000,0.000000}%
\pgfsetstrokecolor{textcolor}%
\pgfsetfillcolor{textcolor}%
\pgftext[x=0.444254in, y=6.911552in, left, base]{\color{textcolor}\rmfamily\fontsize{12.000000}{14.400000}\selectfont \(\displaystyle {1.2}\)}%
\end{pgfscope}%
\begin{pgfscope}%
\definecolor{textcolor}{rgb}{0.000000,0.000000,0.000000}%
\pgfsetstrokecolor{textcolor}%
\pgfsetfillcolor{textcolor}%
\pgftext[x=0.388698in,y=5.500000in,,bottom,rotate=90.000000]{\color{textcolor}\rmfamily\fontsize{12.000000}{14.400000}\selectfont power in pu}%
\end{pgfscope}%
\begin{pgfscope}%
\pgfpathrectangle{\pgfqpoint{0.750000in}{3.960000in}}{\pgfqpoint{4.650000in}{3.080000in}}%
\pgfusepath{clip}%
\pgfsetrectcap%
\pgfsetroundjoin%
\pgfsetlinewidth{2.007500pt}%
\definecolor{currentstroke}{rgb}{0.121569,0.466667,0.705882}%
\pgfsetstrokecolor{currentstroke}%
\pgfsetdash{}{0pt}%
\pgfpathmoveto{\pgfqpoint{0.750000in}{3.960000in}}%
\pgfpathlineto{\pgfqpoint{0.844898in}{4.148036in}}%
\pgfpathlineto{\pgfqpoint{0.939796in}{4.335299in}}%
\pgfpathlineto{\pgfqpoint{1.034694in}{4.521020in}}%
\pgfpathlineto{\pgfqpoint{1.129592in}{4.704436in}}%
\pgfpathlineto{\pgfqpoint{1.224490in}{4.884793in}}%
\pgfpathlineto{\pgfqpoint{1.319388in}{5.061349in}}%
\pgfpathlineto{\pgfqpoint{1.414286in}{5.233380in}}%
\pgfpathlineto{\pgfqpoint{1.509184in}{5.400178in}}%
\pgfpathlineto{\pgfqpoint{1.604082in}{5.561058in}}%
\pgfpathlineto{\pgfqpoint{1.698980in}{5.715359in}}%
\pgfpathlineto{\pgfqpoint{1.793878in}{5.862447in}}%
\pgfpathlineto{\pgfqpoint{1.888776in}{6.001718in}}%
\pgfpathlineto{\pgfqpoint{1.983673in}{6.132598in}}%
\pgfpathlineto{\pgfqpoint{2.078571in}{6.254551in}}%
\pgfpathlineto{\pgfqpoint{2.173469in}{6.367075in}}%
\pgfpathlineto{\pgfqpoint{2.268367in}{6.469708in}}%
\pgfpathlineto{\pgfqpoint{2.363265in}{6.562028in}}%
\pgfpathlineto{\pgfqpoint{2.458163in}{6.643656in}}%
\pgfpathlineto{\pgfqpoint{2.553061in}{6.714256in}}%
\pgfpathlineto{\pgfqpoint{2.647959in}{6.773538in}}%
\pgfpathlineto{\pgfqpoint{2.742857in}{6.821259in}}%
\pgfpathlineto{\pgfqpoint{2.837755in}{6.857222in}}%
\pgfpathlineto{\pgfqpoint{2.932653in}{6.881280in}}%
\pgfpathlineto{\pgfqpoint{3.027551in}{6.893333in}}%
\pgfpathlineto{\pgfqpoint{3.122449in}{6.893333in}}%
\pgfpathlineto{\pgfqpoint{3.217347in}{6.881280in}}%
\pgfpathlineto{\pgfqpoint{3.312245in}{6.857222in}}%
\pgfpathlineto{\pgfqpoint{3.407143in}{6.821259in}}%
\pgfpathlineto{\pgfqpoint{3.502041in}{6.773538in}}%
\pgfpathlineto{\pgfqpoint{3.596939in}{6.714256in}}%
\pgfpathlineto{\pgfqpoint{3.691837in}{6.643656in}}%
\pgfpathlineto{\pgfqpoint{3.786735in}{6.562028in}}%
\pgfpathlineto{\pgfqpoint{3.881633in}{6.469708in}}%
\pgfpathlineto{\pgfqpoint{3.976531in}{6.367075in}}%
\pgfpathlineto{\pgfqpoint{4.071429in}{6.254551in}}%
\pgfpathlineto{\pgfqpoint{4.166327in}{6.132598in}}%
\pgfpathlineto{\pgfqpoint{4.261224in}{6.001718in}}%
\pgfpathlineto{\pgfqpoint{4.356122in}{5.862447in}}%
\pgfpathlineto{\pgfqpoint{4.451020in}{5.715359in}}%
\pgfpathlineto{\pgfqpoint{4.545918in}{5.561058in}}%
\pgfpathlineto{\pgfqpoint{4.640816in}{5.400178in}}%
\pgfpathlineto{\pgfqpoint{4.735714in}{5.233380in}}%
\pgfpathlineto{\pgfqpoint{4.830612in}{5.061349in}}%
\pgfpathlineto{\pgfqpoint{4.925510in}{4.884793in}}%
\pgfpathlineto{\pgfqpoint{5.020408in}{4.704436in}}%
\pgfpathlineto{\pgfqpoint{5.115306in}{4.521020in}}%
\pgfpathlineto{\pgfqpoint{5.210204in}{4.335299in}}%
\pgfpathlineto{\pgfqpoint{5.305102in}{4.148036in}}%
\pgfpathlineto{\pgfqpoint{5.400000in}{3.960000in}}%
\pgfusepath{stroke}%
\end{pgfscope}%
\begin{pgfscope}%
\pgfpathrectangle{\pgfqpoint{0.750000in}{3.960000in}}{\pgfqpoint{4.650000in}{3.080000in}}%
\pgfusepath{clip}%
\pgfsetrectcap%
\pgfsetroundjoin%
\pgfsetlinewidth{2.007500pt}%
\definecolor{currentstroke}{rgb}{1.000000,0.498039,0.054902}%
\pgfsetstrokecolor{currentstroke}%
\pgfsetdash{}{0pt}%
\pgfpathmoveto{\pgfqpoint{0.750000in}{6.240196in}}%
\pgfpathlineto{\pgfqpoint{0.844898in}{6.240196in}}%
\pgfpathlineto{\pgfqpoint{0.939796in}{6.240196in}}%
\pgfpathlineto{\pgfqpoint{1.034694in}{6.240196in}}%
\pgfpathlineto{\pgfqpoint{1.129592in}{6.240196in}}%
\pgfpathlineto{\pgfqpoint{1.224490in}{6.240196in}}%
\pgfpathlineto{\pgfqpoint{1.319388in}{6.240196in}}%
\pgfpathlineto{\pgfqpoint{1.414286in}{6.240196in}}%
\pgfpathlineto{\pgfqpoint{1.509184in}{6.240196in}}%
\pgfpathlineto{\pgfqpoint{1.604082in}{6.240196in}}%
\pgfpathlineto{\pgfqpoint{1.698980in}{6.240196in}}%
\pgfpathlineto{\pgfqpoint{1.793878in}{6.240196in}}%
\pgfpathlineto{\pgfqpoint{1.888776in}{6.240196in}}%
\pgfpathlineto{\pgfqpoint{1.983673in}{6.240196in}}%
\pgfpathlineto{\pgfqpoint{2.078571in}{6.240196in}}%
\pgfpathlineto{\pgfqpoint{2.173469in}{6.240196in}}%
\pgfpathlineto{\pgfqpoint{2.268367in}{6.240196in}}%
\pgfpathlineto{\pgfqpoint{2.363265in}{6.240196in}}%
\pgfpathlineto{\pgfqpoint{2.458163in}{6.240196in}}%
\pgfpathlineto{\pgfqpoint{2.553061in}{6.240196in}}%
\pgfpathlineto{\pgfqpoint{2.647959in}{6.240196in}}%
\pgfpathlineto{\pgfqpoint{2.742857in}{6.240196in}}%
\pgfpathlineto{\pgfqpoint{2.837755in}{6.240196in}}%
\pgfpathlineto{\pgfqpoint{2.932653in}{6.240196in}}%
\pgfpathlineto{\pgfqpoint{3.027551in}{6.240196in}}%
\pgfpathlineto{\pgfqpoint{3.122449in}{6.240196in}}%
\pgfpathlineto{\pgfqpoint{3.217347in}{6.240196in}}%
\pgfpathlineto{\pgfqpoint{3.312245in}{6.240196in}}%
\pgfpathlineto{\pgfqpoint{3.407143in}{6.240196in}}%
\pgfpathlineto{\pgfqpoint{3.502041in}{6.240196in}}%
\pgfpathlineto{\pgfqpoint{3.596939in}{6.240196in}}%
\pgfpathlineto{\pgfqpoint{3.691837in}{6.240196in}}%
\pgfpathlineto{\pgfqpoint{3.786735in}{6.240196in}}%
\pgfpathlineto{\pgfqpoint{3.881633in}{6.240196in}}%
\pgfpathlineto{\pgfqpoint{3.976531in}{6.240196in}}%
\pgfpathlineto{\pgfqpoint{4.071429in}{6.240196in}}%
\pgfpathlineto{\pgfqpoint{4.166327in}{6.240196in}}%
\pgfpathlineto{\pgfqpoint{4.261224in}{6.240196in}}%
\pgfpathlineto{\pgfqpoint{4.356122in}{6.240196in}}%
\pgfpathlineto{\pgfqpoint{4.451020in}{6.240196in}}%
\pgfpathlineto{\pgfqpoint{4.545918in}{6.240196in}}%
\pgfpathlineto{\pgfqpoint{4.640816in}{6.240196in}}%
\pgfpathlineto{\pgfqpoint{4.735714in}{6.240196in}}%
\pgfpathlineto{\pgfqpoint{4.830612in}{6.240196in}}%
\pgfpathlineto{\pgfqpoint{4.925510in}{6.240196in}}%
\pgfpathlineto{\pgfqpoint{5.020408in}{6.240196in}}%
\pgfpathlineto{\pgfqpoint{5.115306in}{6.240196in}}%
\pgfpathlineto{\pgfqpoint{5.210204in}{6.240196in}}%
\pgfpathlineto{\pgfqpoint{5.305102in}{6.240196in}}%
\pgfpathlineto{\pgfqpoint{5.400000in}{6.240196in}}%
\pgfusepath{stroke}%
\end{pgfscope}%
\begin{pgfscope}%
\pgfsetrectcap%
\pgfsetmiterjoin%
\pgfsetlinewidth{0.803000pt}%
\definecolor{currentstroke}{rgb}{0.000000,0.000000,0.000000}%
\pgfsetstrokecolor{currentstroke}%
\pgfsetdash{}{0pt}%
\pgfpathmoveto{\pgfqpoint{0.750000in}{3.960000in}}%
\pgfpathlineto{\pgfqpoint{0.750000in}{7.040000in}}%
\pgfusepath{stroke}%
\end{pgfscope}%
\begin{pgfscope}%
\pgfsetrectcap%
\pgfsetmiterjoin%
\pgfsetlinewidth{0.803000pt}%
\definecolor{currentstroke}{rgb}{0.000000,0.000000,0.000000}%
\pgfsetstrokecolor{currentstroke}%
\pgfsetdash{}{0pt}%
\pgfpathmoveto{\pgfqpoint{5.400000in}{3.960000in}}%
\pgfpathlineto{\pgfqpoint{5.400000in}{7.040000in}}%
\pgfusepath{stroke}%
\end{pgfscope}%
\begin{pgfscope}%
\pgfsetrectcap%
\pgfsetmiterjoin%
\pgfsetlinewidth{0.803000pt}%
\definecolor{currentstroke}{rgb}{0.000000,0.000000,0.000000}%
\pgfsetstrokecolor{currentstroke}%
\pgfsetdash{}{0pt}%
\pgfpathmoveto{\pgfqpoint{0.750000in}{3.960000in}}%
\pgfpathlineto{\pgfqpoint{5.400000in}{3.960000in}}%
\pgfusepath{stroke}%
\end{pgfscope}%
\begin{pgfscope}%
\pgfsetrectcap%
\pgfsetmiterjoin%
\pgfsetlinewidth{0.803000pt}%
\definecolor{currentstroke}{rgb}{0.000000,0.000000,0.000000}%
\pgfsetstrokecolor{currentstroke}%
\pgfsetdash{}{0pt}%
\pgfpathmoveto{\pgfqpoint{0.750000in}{7.040000in}}%
\pgfpathlineto{\pgfqpoint{5.400000in}{7.040000in}}%
\pgfusepath{stroke}%
\end{pgfscope}%
\begin{pgfscope}%
\pgfsetbuttcap%
\pgfsetmiterjoin%
\definecolor{currentfill}{rgb}{1.000000,1.000000,1.000000}%
\pgfsetfillcolor{currentfill}%
\pgfsetfillopacity{0.800000}%
\pgfsetlinewidth{1.003750pt}%
\definecolor{currentstroke}{rgb}{0.800000,0.800000,0.800000}%
\pgfsetstrokecolor{currentstroke}%
\pgfsetstrokeopacity{0.800000}%
\pgfsetdash{}{0pt}%
\pgfpathmoveto{\pgfqpoint{2.198665in}{4.043333in}}%
\pgfpathlineto{\pgfqpoint{3.951335in}{4.043333in}}%
\pgfpathquadraticcurveto{\pgfqpoint{3.984669in}{4.043333in}}{\pgfqpoint{3.984669in}{4.076667in}}%
\pgfpathlineto{\pgfqpoint{3.984669in}{4.545921in}}%
\pgfpathquadraticcurveto{\pgfqpoint{3.984669in}{4.579254in}}{\pgfqpoint{3.951335in}{4.579254in}}%
\pgfpathlineto{\pgfqpoint{2.198665in}{4.579254in}}%
\pgfpathquadraticcurveto{\pgfqpoint{2.165331in}{4.579254in}}{\pgfqpoint{2.165331in}{4.545921in}}%
\pgfpathlineto{\pgfqpoint{2.165331in}{4.076667in}}%
\pgfpathquadraticcurveto{\pgfqpoint{2.165331in}{4.043333in}}{\pgfqpoint{2.198665in}{4.043333in}}%
\pgfpathlineto{\pgfqpoint{2.198665in}{4.043333in}}%
\pgfpathclose%
\pgfusepath{stroke,fill}%
\end{pgfscope}%
\begin{pgfscope}%
\pgfsetrectcap%
\pgfsetroundjoin%
\pgfsetlinewidth{2.007500pt}%
\definecolor{currentstroke}{rgb}{0.121569,0.466667,0.705882}%
\pgfsetstrokecolor{currentstroke}%
\pgfsetdash{}{0pt}%
\pgfpathmoveto{\pgfqpoint{2.231998in}{4.446897in}}%
\pgfpathlineto{\pgfqpoint{2.398665in}{4.446897in}}%
\pgfpathlineto{\pgfqpoint{2.565331in}{4.446897in}}%
\pgfusepath{stroke}%
\end{pgfscope}%
\begin{pgfscope}%
\definecolor{textcolor}{rgb}{0.000000,0.000000,0.000000}%
\pgfsetstrokecolor{textcolor}%
\pgfsetfillcolor{textcolor}%
\pgftext[x=2.698665in,y=4.388564in,left,base]{\color{textcolor}\rmfamily\fontsize{12.000000}{14.400000}\selectfont \(\displaystyle P_\mathrm{e}\) pre-fault}%
\end{pgfscope}%
\begin{pgfscope}%
\pgfsetrectcap%
\pgfsetroundjoin%
\pgfsetlinewidth{2.007500pt}%
\definecolor{currentstroke}{rgb}{1.000000,0.498039,0.054902}%
\pgfsetstrokecolor{currentstroke}%
\pgfsetdash{}{0pt}%
\pgfpathmoveto{\pgfqpoint{2.231998in}{4.204629in}}%
\pgfpathlineto{\pgfqpoint{2.398665in}{4.204629in}}%
\pgfpathlineto{\pgfqpoint{2.565331in}{4.204629in}}%
\pgfusepath{stroke}%
\end{pgfscope}%
\begin{pgfscope}%
\definecolor{textcolor}{rgb}{0.000000,0.000000,0.000000}%
\pgfsetstrokecolor{textcolor}%
\pgfsetfillcolor{textcolor}%
\pgftext[x=2.698665in,y=4.146295in,left,base]{\color{textcolor}\rmfamily\fontsize{12.000000}{14.400000}\selectfont \(\displaystyle P_\mathrm{T}\) of the turbine}%
\end{pgfscope}%
\begin{pgfscope}%
\pgfsetbuttcap%
\pgfsetmiterjoin%
\definecolor{currentfill}{rgb}{1.000000,1.000000,1.000000}%
\pgfsetfillcolor{currentfill}%
\pgfsetlinewidth{0.000000pt}%
\definecolor{currentstroke}{rgb}{0.000000,0.000000,0.000000}%
\pgfsetstrokecolor{currentstroke}%
\pgfsetstrokeopacity{0.000000}%
\pgfsetdash{}{0pt}%
\pgfpathmoveto{\pgfqpoint{0.750000in}{0.880000in}}%
\pgfpathlineto{\pgfqpoint{5.400000in}{0.880000in}}%
\pgfpathlineto{\pgfqpoint{5.400000in}{3.960000in}}%
\pgfpathlineto{\pgfqpoint{0.750000in}{3.960000in}}%
\pgfpathlineto{\pgfqpoint{0.750000in}{0.880000in}}%
\pgfpathclose%
\pgfusepath{fill}%
\end{pgfscope}%
\begin{pgfscope}%
\pgfpathrectangle{\pgfqpoint{0.750000in}{0.880000in}}{\pgfqpoint{4.650000in}{3.080000in}}%
\pgfusepath{clip}%
\pgfsetrectcap%
\pgfsetroundjoin%
\pgfsetlinewidth{0.803000pt}%
\definecolor{currentstroke}{rgb}{0.690196,0.690196,0.690196}%
\pgfsetstrokecolor{currentstroke}%
\pgfsetdash{}{0pt}%
\pgfpathmoveto{\pgfqpoint{0.750000in}{0.880000in}}%
\pgfpathlineto{\pgfqpoint{0.750000in}{3.960000in}}%
\pgfusepath{stroke}%
\end{pgfscope}%
\begin{pgfscope}%
\pgfsetbuttcap%
\pgfsetroundjoin%
\definecolor{currentfill}{rgb}{0.000000,0.000000,0.000000}%
\pgfsetfillcolor{currentfill}%
\pgfsetlinewidth{0.803000pt}%
\definecolor{currentstroke}{rgb}{0.000000,0.000000,0.000000}%
\pgfsetstrokecolor{currentstroke}%
\pgfsetdash{}{0pt}%
\pgfsys@defobject{currentmarker}{\pgfqpoint{0.000000in}{-0.048611in}}{\pgfqpoint{0.000000in}{0.000000in}}{%
\pgfpathmoveto{\pgfqpoint{0.000000in}{0.000000in}}%
\pgfpathlineto{\pgfqpoint{0.000000in}{-0.048611in}}%
\pgfusepath{stroke,fill}%
}%
\begin{pgfscope}%
\pgfsys@transformshift{0.750000in}{0.880000in}%
\pgfsys@useobject{currentmarker}{}%
\end{pgfscope}%
\end{pgfscope}%
\begin{pgfscope}%
\definecolor{textcolor}{rgb}{0.000000,0.000000,0.000000}%
\pgfsetstrokecolor{textcolor}%
\pgfsetfillcolor{textcolor}%
\pgftext[x=0.750000in,y=0.782778in,,top]{\color{textcolor}\rmfamily\fontsize{12.000000}{14.400000}\selectfont \(\displaystyle {0}\)}%
\end{pgfscope}%
\begin{pgfscope}%
\pgfpathrectangle{\pgfqpoint{0.750000in}{0.880000in}}{\pgfqpoint{4.650000in}{3.080000in}}%
\pgfusepath{clip}%
\pgfsetrectcap%
\pgfsetroundjoin%
\pgfsetlinewidth{0.803000pt}%
\definecolor{currentstroke}{rgb}{0.690196,0.690196,0.690196}%
\pgfsetstrokecolor{currentstroke}%
\pgfsetdash{}{0pt}%
\pgfpathmoveto{\pgfqpoint{1.266667in}{0.880000in}}%
\pgfpathlineto{\pgfqpoint{1.266667in}{3.960000in}}%
\pgfusepath{stroke}%
\end{pgfscope}%
\begin{pgfscope}%
\pgfsetbuttcap%
\pgfsetroundjoin%
\definecolor{currentfill}{rgb}{0.000000,0.000000,0.000000}%
\pgfsetfillcolor{currentfill}%
\pgfsetlinewidth{0.803000pt}%
\definecolor{currentstroke}{rgb}{0.000000,0.000000,0.000000}%
\pgfsetstrokecolor{currentstroke}%
\pgfsetdash{}{0pt}%
\pgfsys@defobject{currentmarker}{\pgfqpoint{0.000000in}{-0.048611in}}{\pgfqpoint{0.000000in}{0.000000in}}{%
\pgfpathmoveto{\pgfqpoint{0.000000in}{0.000000in}}%
\pgfpathlineto{\pgfqpoint{0.000000in}{-0.048611in}}%
\pgfusepath{stroke,fill}%
}%
\begin{pgfscope}%
\pgfsys@transformshift{1.266667in}{0.880000in}%
\pgfsys@useobject{currentmarker}{}%
\end{pgfscope}%
\end{pgfscope}%
\begin{pgfscope}%
\definecolor{textcolor}{rgb}{0.000000,0.000000,0.000000}%
\pgfsetstrokecolor{textcolor}%
\pgfsetfillcolor{textcolor}%
\pgftext[x=1.266667in,y=0.782778in,,top]{\color{textcolor}\rmfamily\fontsize{12.000000}{14.400000}\selectfont \(\displaystyle {20}\)}%
\end{pgfscope}%
\begin{pgfscope}%
\pgfpathrectangle{\pgfqpoint{0.750000in}{0.880000in}}{\pgfqpoint{4.650000in}{3.080000in}}%
\pgfusepath{clip}%
\pgfsetrectcap%
\pgfsetroundjoin%
\pgfsetlinewidth{0.803000pt}%
\definecolor{currentstroke}{rgb}{0.690196,0.690196,0.690196}%
\pgfsetstrokecolor{currentstroke}%
\pgfsetdash{}{0pt}%
\pgfpathmoveto{\pgfqpoint{1.783333in}{0.880000in}}%
\pgfpathlineto{\pgfqpoint{1.783333in}{3.960000in}}%
\pgfusepath{stroke}%
\end{pgfscope}%
\begin{pgfscope}%
\pgfsetbuttcap%
\pgfsetroundjoin%
\definecolor{currentfill}{rgb}{0.000000,0.000000,0.000000}%
\pgfsetfillcolor{currentfill}%
\pgfsetlinewidth{0.803000pt}%
\definecolor{currentstroke}{rgb}{0.000000,0.000000,0.000000}%
\pgfsetstrokecolor{currentstroke}%
\pgfsetdash{}{0pt}%
\pgfsys@defobject{currentmarker}{\pgfqpoint{0.000000in}{-0.048611in}}{\pgfqpoint{0.000000in}{0.000000in}}{%
\pgfpathmoveto{\pgfqpoint{0.000000in}{0.000000in}}%
\pgfpathlineto{\pgfqpoint{0.000000in}{-0.048611in}}%
\pgfusepath{stroke,fill}%
}%
\begin{pgfscope}%
\pgfsys@transformshift{1.783333in}{0.880000in}%
\pgfsys@useobject{currentmarker}{}%
\end{pgfscope}%
\end{pgfscope}%
\begin{pgfscope}%
\definecolor{textcolor}{rgb}{0.000000,0.000000,0.000000}%
\pgfsetstrokecolor{textcolor}%
\pgfsetfillcolor{textcolor}%
\pgftext[x=1.783333in,y=0.782778in,,top]{\color{textcolor}\rmfamily\fontsize{12.000000}{14.400000}\selectfont \(\displaystyle {40}\)}%
\end{pgfscope}%
\begin{pgfscope}%
\pgfpathrectangle{\pgfqpoint{0.750000in}{0.880000in}}{\pgfqpoint{4.650000in}{3.080000in}}%
\pgfusepath{clip}%
\pgfsetrectcap%
\pgfsetroundjoin%
\pgfsetlinewidth{0.803000pt}%
\definecolor{currentstroke}{rgb}{0.690196,0.690196,0.690196}%
\pgfsetstrokecolor{currentstroke}%
\pgfsetdash{}{0pt}%
\pgfpathmoveto{\pgfqpoint{2.300000in}{0.880000in}}%
\pgfpathlineto{\pgfqpoint{2.300000in}{3.960000in}}%
\pgfusepath{stroke}%
\end{pgfscope}%
\begin{pgfscope}%
\pgfsetbuttcap%
\pgfsetroundjoin%
\definecolor{currentfill}{rgb}{0.000000,0.000000,0.000000}%
\pgfsetfillcolor{currentfill}%
\pgfsetlinewidth{0.803000pt}%
\definecolor{currentstroke}{rgb}{0.000000,0.000000,0.000000}%
\pgfsetstrokecolor{currentstroke}%
\pgfsetdash{}{0pt}%
\pgfsys@defobject{currentmarker}{\pgfqpoint{0.000000in}{-0.048611in}}{\pgfqpoint{0.000000in}{0.000000in}}{%
\pgfpathmoveto{\pgfqpoint{0.000000in}{0.000000in}}%
\pgfpathlineto{\pgfqpoint{0.000000in}{-0.048611in}}%
\pgfusepath{stroke,fill}%
}%
\begin{pgfscope}%
\pgfsys@transformshift{2.300000in}{0.880000in}%
\pgfsys@useobject{currentmarker}{}%
\end{pgfscope}%
\end{pgfscope}%
\begin{pgfscope}%
\definecolor{textcolor}{rgb}{0.000000,0.000000,0.000000}%
\pgfsetstrokecolor{textcolor}%
\pgfsetfillcolor{textcolor}%
\pgftext[x=2.300000in,y=0.782778in,,top]{\color{textcolor}\rmfamily\fontsize{12.000000}{14.400000}\selectfont \(\displaystyle {60}\)}%
\end{pgfscope}%
\begin{pgfscope}%
\pgfpathrectangle{\pgfqpoint{0.750000in}{0.880000in}}{\pgfqpoint{4.650000in}{3.080000in}}%
\pgfusepath{clip}%
\pgfsetrectcap%
\pgfsetroundjoin%
\pgfsetlinewidth{0.803000pt}%
\definecolor{currentstroke}{rgb}{0.690196,0.690196,0.690196}%
\pgfsetstrokecolor{currentstroke}%
\pgfsetdash{}{0pt}%
\pgfpathmoveto{\pgfqpoint{2.816667in}{0.880000in}}%
\pgfpathlineto{\pgfqpoint{2.816667in}{3.960000in}}%
\pgfusepath{stroke}%
\end{pgfscope}%
\begin{pgfscope}%
\pgfsetbuttcap%
\pgfsetroundjoin%
\definecolor{currentfill}{rgb}{0.000000,0.000000,0.000000}%
\pgfsetfillcolor{currentfill}%
\pgfsetlinewidth{0.803000pt}%
\definecolor{currentstroke}{rgb}{0.000000,0.000000,0.000000}%
\pgfsetstrokecolor{currentstroke}%
\pgfsetdash{}{0pt}%
\pgfsys@defobject{currentmarker}{\pgfqpoint{0.000000in}{-0.048611in}}{\pgfqpoint{0.000000in}{0.000000in}}{%
\pgfpathmoveto{\pgfqpoint{0.000000in}{0.000000in}}%
\pgfpathlineto{\pgfqpoint{0.000000in}{-0.048611in}}%
\pgfusepath{stroke,fill}%
}%
\begin{pgfscope}%
\pgfsys@transformshift{2.816667in}{0.880000in}%
\pgfsys@useobject{currentmarker}{}%
\end{pgfscope}%
\end{pgfscope}%
\begin{pgfscope}%
\definecolor{textcolor}{rgb}{0.000000,0.000000,0.000000}%
\pgfsetstrokecolor{textcolor}%
\pgfsetfillcolor{textcolor}%
\pgftext[x=2.816667in,y=0.782778in,,top]{\color{textcolor}\rmfamily\fontsize{12.000000}{14.400000}\selectfont \(\displaystyle {80}\)}%
\end{pgfscope}%
\begin{pgfscope}%
\pgfpathrectangle{\pgfqpoint{0.750000in}{0.880000in}}{\pgfqpoint{4.650000in}{3.080000in}}%
\pgfusepath{clip}%
\pgfsetrectcap%
\pgfsetroundjoin%
\pgfsetlinewidth{0.803000pt}%
\definecolor{currentstroke}{rgb}{0.690196,0.690196,0.690196}%
\pgfsetstrokecolor{currentstroke}%
\pgfsetdash{}{0pt}%
\pgfpathmoveto{\pgfqpoint{3.333333in}{0.880000in}}%
\pgfpathlineto{\pgfqpoint{3.333333in}{3.960000in}}%
\pgfusepath{stroke}%
\end{pgfscope}%
\begin{pgfscope}%
\pgfsetbuttcap%
\pgfsetroundjoin%
\definecolor{currentfill}{rgb}{0.000000,0.000000,0.000000}%
\pgfsetfillcolor{currentfill}%
\pgfsetlinewidth{0.803000pt}%
\definecolor{currentstroke}{rgb}{0.000000,0.000000,0.000000}%
\pgfsetstrokecolor{currentstroke}%
\pgfsetdash{}{0pt}%
\pgfsys@defobject{currentmarker}{\pgfqpoint{0.000000in}{-0.048611in}}{\pgfqpoint{0.000000in}{0.000000in}}{%
\pgfpathmoveto{\pgfqpoint{0.000000in}{0.000000in}}%
\pgfpathlineto{\pgfqpoint{0.000000in}{-0.048611in}}%
\pgfusepath{stroke,fill}%
}%
\begin{pgfscope}%
\pgfsys@transformshift{3.333333in}{0.880000in}%
\pgfsys@useobject{currentmarker}{}%
\end{pgfscope}%
\end{pgfscope}%
\begin{pgfscope}%
\definecolor{textcolor}{rgb}{0.000000,0.000000,0.000000}%
\pgfsetstrokecolor{textcolor}%
\pgfsetfillcolor{textcolor}%
\pgftext[x=3.333333in,y=0.782778in,,top]{\color{textcolor}\rmfamily\fontsize{12.000000}{14.400000}\selectfont \(\displaystyle {100}\)}%
\end{pgfscope}%
\begin{pgfscope}%
\pgfpathrectangle{\pgfqpoint{0.750000in}{0.880000in}}{\pgfqpoint{4.650000in}{3.080000in}}%
\pgfusepath{clip}%
\pgfsetrectcap%
\pgfsetroundjoin%
\pgfsetlinewidth{0.803000pt}%
\definecolor{currentstroke}{rgb}{0.690196,0.690196,0.690196}%
\pgfsetstrokecolor{currentstroke}%
\pgfsetdash{}{0pt}%
\pgfpathmoveto{\pgfqpoint{3.850000in}{0.880000in}}%
\pgfpathlineto{\pgfqpoint{3.850000in}{3.960000in}}%
\pgfusepath{stroke}%
\end{pgfscope}%
\begin{pgfscope}%
\pgfsetbuttcap%
\pgfsetroundjoin%
\definecolor{currentfill}{rgb}{0.000000,0.000000,0.000000}%
\pgfsetfillcolor{currentfill}%
\pgfsetlinewidth{0.803000pt}%
\definecolor{currentstroke}{rgb}{0.000000,0.000000,0.000000}%
\pgfsetstrokecolor{currentstroke}%
\pgfsetdash{}{0pt}%
\pgfsys@defobject{currentmarker}{\pgfqpoint{0.000000in}{-0.048611in}}{\pgfqpoint{0.000000in}{0.000000in}}{%
\pgfpathmoveto{\pgfqpoint{0.000000in}{0.000000in}}%
\pgfpathlineto{\pgfqpoint{0.000000in}{-0.048611in}}%
\pgfusepath{stroke,fill}%
}%
\begin{pgfscope}%
\pgfsys@transformshift{3.850000in}{0.880000in}%
\pgfsys@useobject{currentmarker}{}%
\end{pgfscope}%
\end{pgfscope}%
\begin{pgfscope}%
\definecolor{textcolor}{rgb}{0.000000,0.000000,0.000000}%
\pgfsetstrokecolor{textcolor}%
\pgfsetfillcolor{textcolor}%
\pgftext[x=3.850000in,y=0.782778in,,top]{\color{textcolor}\rmfamily\fontsize{12.000000}{14.400000}\selectfont \(\displaystyle {120}\)}%
\end{pgfscope}%
\begin{pgfscope}%
\pgfpathrectangle{\pgfqpoint{0.750000in}{0.880000in}}{\pgfqpoint{4.650000in}{3.080000in}}%
\pgfusepath{clip}%
\pgfsetrectcap%
\pgfsetroundjoin%
\pgfsetlinewidth{0.803000pt}%
\definecolor{currentstroke}{rgb}{0.690196,0.690196,0.690196}%
\pgfsetstrokecolor{currentstroke}%
\pgfsetdash{}{0pt}%
\pgfpathmoveto{\pgfqpoint{4.366667in}{0.880000in}}%
\pgfpathlineto{\pgfqpoint{4.366667in}{3.960000in}}%
\pgfusepath{stroke}%
\end{pgfscope}%
\begin{pgfscope}%
\pgfsetbuttcap%
\pgfsetroundjoin%
\definecolor{currentfill}{rgb}{0.000000,0.000000,0.000000}%
\pgfsetfillcolor{currentfill}%
\pgfsetlinewidth{0.803000pt}%
\definecolor{currentstroke}{rgb}{0.000000,0.000000,0.000000}%
\pgfsetstrokecolor{currentstroke}%
\pgfsetdash{}{0pt}%
\pgfsys@defobject{currentmarker}{\pgfqpoint{0.000000in}{-0.048611in}}{\pgfqpoint{0.000000in}{0.000000in}}{%
\pgfpathmoveto{\pgfqpoint{0.000000in}{0.000000in}}%
\pgfpathlineto{\pgfqpoint{0.000000in}{-0.048611in}}%
\pgfusepath{stroke,fill}%
}%
\begin{pgfscope}%
\pgfsys@transformshift{4.366667in}{0.880000in}%
\pgfsys@useobject{currentmarker}{}%
\end{pgfscope}%
\end{pgfscope}%
\begin{pgfscope}%
\definecolor{textcolor}{rgb}{0.000000,0.000000,0.000000}%
\pgfsetstrokecolor{textcolor}%
\pgfsetfillcolor{textcolor}%
\pgftext[x=4.366667in,y=0.782778in,,top]{\color{textcolor}\rmfamily\fontsize{12.000000}{14.400000}\selectfont \(\displaystyle {140}\)}%
\end{pgfscope}%
\begin{pgfscope}%
\pgfpathrectangle{\pgfqpoint{0.750000in}{0.880000in}}{\pgfqpoint{4.650000in}{3.080000in}}%
\pgfusepath{clip}%
\pgfsetrectcap%
\pgfsetroundjoin%
\pgfsetlinewidth{0.803000pt}%
\definecolor{currentstroke}{rgb}{0.690196,0.690196,0.690196}%
\pgfsetstrokecolor{currentstroke}%
\pgfsetdash{}{0pt}%
\pgfpathmoveto{\pgfqpoint{4.883333in}{0.880000in}}%
\pgfpathlineto{\pgfqpoint{4.883333in}{3.960000in}}%
\pgfusepath{stroke}%
\end{pgfscope}%
\begin{pgfscope}%
\pgfsetbuttcap%
\pgfsetroundjoin%
\definecolor{currentfill}{rgb}{0.000000,0.000000,0.000000}%
\pgfsetfillcolor{currentfill}%
\pgfsetlinewidth{0.803000pt}%
\definecolor{currentstroke}{rgb}{0.000000,0.000000,0.000000}%
\pgfsetstrokecolor{currentstroke}%
\pgfsetdash{}{0pt}%
\pgfsys@defobject{currentmarker}{\pgfqpoint{0.000000in}{-0.048611in}}{\pgfqpoint{0.000000in}{0.000000in}}{%
\pgfpathmoveto{\pgfqpoint{0.000000in}{0.000000in}}%
\pgfpathlineto{\pgfqpoint{0.000000in}{-0.048611in}}%
\pgfusepath{stroke,fill}%
}%
\begin{pgfscope}%
\pgfsys@transformshift{4.883333in}{0.880000in}%
\pgfsys@useobject{currentmarker}{}%
\end{pgfscope}%
\end{pgfscope}%
\begin{pgfscope}%
\definecolor{textcolor}{rgb}{0.000000,0.000000,0.000000}%
\pgfsetstrokecolor{textcolor}%
\pgfsetfillcolor{textcolor}%
\pgftext[x=4.883333in,y=0.782778in,,top]{\color{textcolor}\rmfamily\fontsize{12.000000}{14.400000}\selectfont \(\displaystyle {160}\)}%
\end{pgfscope}%
\begin{pgfscope}%
\pgfpathrectangle{\pgfqpoint{0.750000in}{0.880000in}}{\pgfqpoint{4.650000in}{3.080000in}}%
\pgfusepath{clip}%
\pgfsetrectcap%
\pgfsetroundjoin%
\pgfsetlinewidth{0.803000pt}%
\definecolor{currentstroke}{rgb}{0.690196,0.690196,0.690196}%
\pgfsetstrokecolor{currentstroke}%
\pgfsetdash{}{0pt}%
\pgfpathmoveto{\pgfqpoint{5.400000in}{0.880000in}}%
\pgfpathlineto{\pgfqpoint{5.400000in}{3.960000in}}%
\pgfusepath{stroke}%
\end{pgfscope}%
\begin{pgfscope}%
\pgfsetbuttcap%
\pgfsetroundjoin%
\definecolor{currentfill}{rgb}{0.000000,0.000000,0.000000}%
\pgfsetfillcolor{currentfill}%
\pgfsetlinewidth{0.803000pt}%
\definecolor{currentstroke}{rgb}{0.000000,0.000000,0.000000}%
\pgfsetstrokecolor{currentstroke}%
\pgfsetdash{}{0pt}%
\pgfsys@defobject{currentmarker}{\pgfqpoint{0.000000in}{-0.048611in}}{\pgfqpoint{0.000000in}{0.000000in}}{%
\pgfpathmoveto{\pgfqpoint{0.000000in}{0.000000in}}%
\pgfpathlineto{\pgfqpoint{0.000000in}{-0.048611in}}%
\pgfusepath{stroke,fill}%
}%
\begin{pgfscope}%
\pgfsys@transformshift{5.400000in}{0.880000in}%
\pgfsys@useobject{currentmarker}{}%
\end{pgfscope}%
\end{pgfscope}%
\begin{pgfscope}%
\definecolor{textcolor}{rgb}{0.000000,0.000000,0.000000}%
\pgfsetstrokecolor{textcolor}%
\pgfsetfillcolor{textcolor}%
\pgftext[x=5.400000in,y=0.782778in,,top]{\color{textcolor}\rmfamily\fontsize{12.000000}{14.400000}\selectfont \(\displaystyle {180}\)}%
\end{pgfscope}%
\begin{pgfscope}%
\definecolor{textcolor}{rgb}{0.000000,0.000000,0.000000}%
\pgfsetstrokecolor{textcolor}%
\pgfsetfillcolor{textcolor}%
\pgftext[x=3.075000in,y=0.568287in,,top]{\color{textcolor}\rmfamily\fontsize{12.000000}{14.400000}\selectfont power angle \(\displaystyle \delta\) in deg}%
\end{pgfscope}%
\begin{pgfscope}%
\pgfpathrectangle{\pgfqpoint{0.750000in}{0.880000in}}{\pgfqpoint{4.650000in}{3.080000in}}%
\pgfusepath{clip}%
\pgfsetrectcap%
\pgfsetroundjoin%
\pgfsetlinewidth{0.803000pt}%
\definecolor{currentstroke}{rgb}{0.690196,0.690196,0.690196}%
\pgfsetstrokecolor{currentstroke}%
\pgfsetdash{}{0pt}%
\pgfpathmoveto{\pgfqpoint{0.750000in}{3.703109in}}%
\pgfpathlineto{\pgfqpoint{5.400000in}{3.703109in}}%
\pgfusepath{stroke}%
\end{pgfscope}%
\begin{pgfscope}%
\pgfsetbuttcap%
\pgfsetroundjoin%
\definecolor{currentfill}{rgb}{0.000000,0.000000,0.000000}%
\pgfsetfillcolor{currentfill}%
\pgfsetlinewidth{0.803000pt}%
\definecolor{currentstroke}{rgb}{0.000000,0.000000,0.000000}%
\pgfsetstrokecolor{currentstroke}%
\pgfsetdash{}{0pt}%
\pgfsys@defobject{currentmarker}{\pgfqpoint{-0.048611in}{0.000000in}}{\pgfqpoint{-0.000000in}{0.000000in}}{%
\pgfpathmoveto{\pgfqpoint{-0.000000in}{0.000000in}}%
\pgfpathlineto{\pgfqpoint{-0.048611in}{0.000000in}}%
\pgfusepath{stroke,fill}%
}%
\begin{pgfscope}%
\pgfsys@transformshift{0.750000in}{3.703109in}%
\pgfsys@useobject{currentmarker}{}%
\end{pgfscope}%
\end{pgfscope}%
\begin{pgfscope}%
\definecolor{textcolor}{rgb}{0.000000,0.000000,0.000000}%
\pgfsetstrokecolor{textcolor}%
\pgfsetfillcolor{textcolor}%
\pgftext[x=0.444254in, y=3.641789in, left, base]{\color{textcolor}\rmfamily\fontsize{12.000000}{14.400000}\selectfont \(\displaystyle {0.0}\)}%
\end{pgfscope}%
\begin{pgfscope}%
\pgfpathrectangle{\pgfqpoint{0.750000in}{0.880000in}}{\pgfqpoint{4.650000in}{3.080000in}}%
\pgfusepath{clip}%
\pgfsetrectcap%
\pgfsetroundjoin%
\pgfsetlinewidth{0.803000pt}%
\definecolor{currentstroke}{rgb}{0.690196,0.690196,0.690196}%
\pgfsetstrokecolor{currentstroke}%
\pgfsetdash{}{0pt}%
\pgfpathmoveto{\pgfqpoint{0.750000in}{3.189326in}}%
\pgfpathlineto{\pgfqpoint{5.400000in}{3.189326in}}%
\pgfusepath{stroke}%
\end{pgfscope}%
\begin{pgfscope}%
\pgfsetbuttcap%
\pgfsetroundjoin%
\definecolor{currentfill}{rgb}{0.000000,0.000000,0.000000}%
\pgfsetfillcolor{currentfill}%
\pgfsetlinewidth{0.803000pt}%
\definecolor{currentstroke}{rgb}{0.000000,0.000000,0.000000}%
\pgfsetstrokecolor{currentstroke}%
\pgfsetdash{}{0pt}%
\pgfsys@defobject{currentmarker}{\pgfqpoint{-0.048611in}{0.000000in}}{\pgfqpoint{-0.000000in}{0.000000in}}{%
\pgfpathmoveto{\pgfqpoint{-0.000000in}{0.000000in}}%
\pgfpathlineto{\pgfqpoint{-0.048611in}{0.000000in}}%
\pgfusepath{stroke,fill}%
}%
\begin{pgfscope}%
\pgfsys@transformshift{0.750000in}{3.189326in}%
\pgfsys@useobject{currentmarker}{}%
\end{pgfscope}%
\end{pgfscope}%
\begin{pgfscope}%
\definecolor{textcolor}{rgb}{0.000000,0.000000,0.000000}%
\pgfsetstrokecolor{textcolor}%
\pgfsetfillcolor{textcolor}%
\pgftext[x=0.444254in, y=3.128006in, left, base]{\color{textcolor}\rmfamily\fontsize{12.000000}{14.400000}\selectfont \(\displaystyle {0.2}\)}%
\end{pgfscope}%
\begin{pgfscope}%
\pgfpathrectangle{\pgfqpoint{0.750000in}{0.880000in}}{\pgfqpoint{4.650000in}{3.080000in}}%
\pgfusepath{clip}%
\pgfsetrectcap%
\pgfsetroundjoin%
\pgfsetlinewidth{0.803000pt}%
\definecolor{currentstroke}{rgb}{0.690196,0.690196,0.690196}%
\pgfsetstrokecolor{currentstroke}%
\pgfsetdash{}{0pt}%
\pgfpathmoveto{\pgfqpoint{0.750000in}{2.675543in}}%
\pgfpathlineto{\pgfqpoint{5.400000in}{2.675543in}}%
\pgfusepath{stroke}%
\end{pgfscope}%
\begin{pgfscope}%
\pgfsetbuttcap%
\pgfsetroundjoin%
\definecolor{currentfill}{rgb}{0.000000,0.000000,0.000000}%
\pgfsetfillcolor{currentfill}%
\pgfsetlinewidth{0.803000pt}%
\definecolor{currentstroke}{rgb}{0.000000,0.000000,0.000000}%
\pgfsetstrokecolor{currentstroke}%
\pgfsetdash{}{0pt}%
\pgfsys@defobject{currentmarker}{\pgfqpoint{-0.048611in}{0.000000in}}{\pgfqpoint{-0.000000in}{0.000000in}}{%
\pgfpathmoveto{\pgfqpoint{-0.000000in}{0.000000in}}%
\pgfpathlineto{\pgfqpoint{-0.048611in}{0.000000in}}%
\pgfusepath{stroke,fill}%
}%
\begin{pgfscope}%
\pgfsys@transformshift{0.750000in}{2.675543in}%
\pgfsys@useobject{currentmarker}{}%
\end{pgfscope}%
\end{pgfscope}%
\begin{pgfscope}%
\definecolor{textcolor}{rgb}{0.000000,0.000000,0.000000}%
\pgfsetstrokecolor{textcolor}%
\pgfsetfillcolor{textcolor}%
\pgftext[x=0.444254in, y=2.614223in, left, base]{\color{textcolor}\rmfamily\fontsize{12.000000}{14.400000}\selectfont \(\displaystyle {0.4}\)}%
\end{pgfscope}%
\begin{pgfscope}%
\pgfpathrectangle{\pgfqpoint{0.750000in}{0.880000in}}{\pgfqpoint{4.650000in}{3.080000in}}%
\pgfusepath{clip}%
\pgfsetrectcap%
\pgfsetroundjoin%
\pgfsetlinewidth{0.803000pt}%
\definecolor{currentstroke}{rgb}{0.690196,0.690196,0.690196}%
\pgfsetstrokecolor{currentstroke}%
\pgfsetdash{}{0pt}%
\pgfpathmoveto{\pgfqpoint{0.750000in}{2.161760in}}%
\pgfpathlineto{\pgfqpoint{5.400000in}{2.161760in}}%
\pgfusepath{stroke}%
\end{pgfscope}%
\begin{pgfscope}%
\pgfsetbuttcap%
\pgfsetroundjoin%
\definecolor{currentfill}{rgb}{0.000000,0.000000,0.000000}%
\pgfsetfillcolor{currentfill}%
\pgfsetlinewidth{0.803000pt}%
\definecolor{currentstroke}{rgb}{0.000000,0.000000,0.000000}%
\pgfsetstrokecolor{currentstroke}%
\pgfsetdash{}{0pt}%
\pgfsys@defobject{currentmarker}{\pgfqpoint{-0.048611in}{0.000000in}}{\pgfqpoint{-0.000000in}{0.000000in}}{%
\pgfpathmoveto{\pgfqpoint{-0.000000in}{0.000000in}}%
\pgfpathlineto{\pgfqpoint{-0.048611in}{0.000000in}}%
\pgfusepath{stroke,fill}%
}%
\begin{pgfscope}%
\pgfsys@transformshift{0.750000in}{2.161760in}%
\pgfsys@useobject{currentmarker}{}%
\end{pgfscope}%
\end{pgfscope}%
\begin{pgfscope}%
\definecolor{textcolor}{rgb}{0.000000,0.000000,0.000000}%
\pgfsetstrokecolor{textcolor}%
\pgfsetfillcolor{textcolor}%
\pgftext[x=0.444254in, y=2.100440in, left, base]{\color{textcolor}\rmfamily\fontsize{12.000000}{14.400000}\selectfont \(\displaystyle {0.6}\)}%
\end{pgfscope}%
\begin{pgfscope}%
\pgfpathrectangle{\pgfqpoint{0.750000in}{0.880000in}}{\pgfqpoint{4.650000in}{3.080000in}}%
\pgfusepath{clip}%
\pgfsetrectcap%
\pgfsetroundjoin%
\pgfsetlinewidth{0.803000pt}%
\definecolor{currentstroke}{rgb}{0.690196,0.690196,0.690196}%
\pgfsetstrokecolor{currentstroke}%
\pgfsetdash{}{0pt}%
\pgfpathmoveto{\pgfqpoint{0.750000in}{1.647977in}}%
\pgfpathlineto{\pgfqpoint{5.400000in}{1.647977in}}%
\pgfusepath{stroke}%
\end{pgfscope}%
\begin{pgfscope}%
\pgfsetbuttcap%
\pgfsetroundjoin%
\definecolor{currentfill}{rgb}{0.000000,0.000000,0.000000}%
\pgfsetfillcolor{currentfill}%
\pgfsetlinewidth{0.803000pt}%
\definecolor{currentstroke}{rgb}{0.000000,0.000000,0.000000}%
\pgfsetstrokecolor{currentstroke}%
\pgfsetdash{}{0pt}%
\pgfsys@defobject{currentmarker}{\pgfqpoint{-0.048611in}{0.000000in}}{\pgfqpoint{-0.000000in}{0.000000in}}{%
\pgfpathmoveto{\pgfqpoint{-0.000000in}{0.000000in}}%
\pgfpathlineto{\pgfqpoint{-0.048611in}{0.000000in}}%
\pgfusepath{stroke,fill}%
}%
\begin{pgfscope}%
\pgfsys@transformshift{0.750000in}{1.647977in}%
\pgfsys@useobject{currentmarker}{}%
\end{pgfscope}%
\end{pgfscope}%
\begin{pgfscope}%
\definecolor{textcolor}{rgb}{0.000000,0.000000,0.000000}%
\pgfsetstrokecolor{textcolor}%
\pgfsetfillcolor{textcolor}%
\pgftext[x=0.444254in, y=1.586657in, left, base]{\color{textcolor}\rmfamily\fontsize{12.000000}{14.400000}\selectfont \(\displaystyle {0.8}\)}%
\end{pgfscope}%
\begin{pgfscope}%
\pgfpathrectangle{\pgfqpoint{0.750000in}{0.880000in}}{\pgfqpoint{4.650000in}{3.080000in}}%
\pgfusepath{clip}%
\pgfsetrectcap%
\pgfsetroundjoin%
\pgfsetlinewidth{0.803000pt}%
\definecolor{currentstroke}{rgb}{0.690196,0.690196,0.690196}%
\pgfsetstrokecolor{currentstroke}%
\pgfsetdash{}{0pt}%
\pgfpathmoveto{\pgfqpoint{0.750000in}{1.134194in}}%
\pgfpathlineto{\pgfqpoint{5.400000in}{1.134194in}}%
\pgfusepath{stroke}%
\end{pgfscope}%
\begin{pgfscope}%
\pgfsetbuttcap%
\pgfsetroundjoin%
\definecolor{currentfill}{rgb}{0.000000,0.000000,0.000000}%
\pgfsetfillcolor{currentfill}%
\pgfsetlinewidth{0.803000pt}%
\definecolor{currentstroke}{rgb}{0.000000,0.000000,0.000000}%
\pgfsetstrokecolor{currentstroke}%
\pgfsetdash{}{0pt}%
\pgfsys@defobject{currentmarker}{\pgfqpoint{-0.048611in}{0.000000in}}{\pgfqpoint{-0.000000in}{0.000000in}}{%
\pgfpathmoveto{\pgfqpoint{-0.000000in}{0.000000in}}%
\pgfpathlineto{\pgfqpoint{-0.048611in}{0.000000in}}%
\pgfusepath{stroke,fill}%
}%
\begin{pgfscope}%
\pgfsys@transformshift{0.750000in}{1.134194in}%
\pgfsys@useobject{currentmarker}{}%
\end{pgfscope}%
\end{pgfscope}%
\begin{pgfscope}%
\definecolor{textcolor}{rgb}{0.000000,0.000000,0.000000}%
\pgfsetstrokecolor{textcolor}%
\pgfsetfillcolor{textcolor}%
\pgftext[x=0.444254in, y=1.072874in, left, base]{\color{textcolor}\rmfamily\fontsize{12.000000}{14.400000}\selectfont \(\displaystyle {1.0}\)}%
\end{pgfscope}%
\begin{pgfscope}%
\definecolor{textcolor}{rgb}{0.000000,0.000000,0.000000}%
\pgfsetstrokecolor{textcolor}%
\pgfsetfillcolor{textcolor}%
\pgftext[x=0.388698in,y=2.420000in,,bottom,rotate=90.000000]{\color{textcolor}\rmfamily\fontsize{12.000000}{14.400000}\selectfont time in s}%
\end{pgfscope}%
\begin{pgfscope}%
\pgfpathrectangle{\pgfqpoint{0.750000in}{0.880000in}}{\pgfqpoint{4.650000in}{3.080000in}}%
\pgfusepath{clip}%
\pgfsetrectcap%
\pgfsetroundjoin%
\pgfsetlinewidth{1.505625pt}%
\definecolor{currentstroke}{rgb}{0.121569,0.466667,0.705882}%
\pgfsetstrokecolor{currentstroke}%
\pgfsetdash{}{0pt}%
\pgfpathmoveto{\pgfqpoint{2.065441in}{3.970000in}}%
\pgfpathlineto{\pgfqpoint{2.065627in}{3.692833in}}%
\pgfpathlineto{\pgfqpoint{2.068667in}{3.677419in}}%
\pgfpathlineto{\pgfqpoint{2.074310in}{3.662006in}}%
\pgfpathlineto{\pgfqpoint{2.082553in}{3.646592in}}%
\pgfpathlineto{\pgfqpoint{2.093395in}{3.631179in}}%
\pgfpathlineto{\pgfqpoint{2.106834in}{3.615765in}}%
\pgfpathlineto{\pgfqpoint{2.125792in}{3.597783in}}%
\pgfpathlineto{\pgfqpoint{2.148279in}{3.579801in}}%
\pgfpathlineto{\pgfqpoint{2.174290in}{3.561818in}}%
\pgfpathlineto{\pgfqpoint{2.208331in}{3.541267in}}%
\pgfpathlineto{\pgfqpoint{2.246967in}{3.520716in}}%
\pgfpathlineto{\pgfqpoint{2.290194in}{3.500164in}}%
\pgfpathlineto{\pgfqpoint{2.344307in}{3.477044in}}%
\pgfpathlineto{\pgfqpoint{2.404217in}{3.453924in}}%
\pgfpathlineto{\pgfqpoint{2.576432in}{3.389701in}}%
\pgfpathlineto{\pgfqpoint{2.700643in}{3.340892in}}%
\pgfpathlineto{\pgfqpoint{2.812139in}{3.294651in}}%
\pgfpathlineto{\pgfqpoint{2.911484in}{3.250980in}}%
\pgfpathlineto{\pgfqpoint{2.999468in}{3.209877in}}%
\pgfpathlineto{\pgfqpoint{3.076997in}{3.171343in}}%
\pgfpathlineto{\pgfqpoint{3.149709in}{3.132810in}}%
\pgfpathlineto{\pgfqpoint{3.217638in}{3.094276in}}%
\pgfpathlineto{\pgfqpoint{3.280865in}{3.055742in}}%
\pgfpathlineto{\pgfqpoint{3.335733in}{3.019777in}}%
\pgfpathlineto{\pgfqpoint{3.386716in}{2.983813in}}%
\pgfpathlineto{\pgfqpoint{3.433941in}{2.947848in}}%
\pgfpathlineto{\pgfqpoint{3.477545in}{2.911883in}}%
\pgfpathlineto{\pgfqpoint{3.517671in}{2.875918in}}%
\pgfpathlineto{\pgfqpoint{3.554464in}{2.839953in}}%
\pgfpathlineto{\pgfqpoint{3.588070in}{2.803988in}}%
\pgfpathlineto{\pgfqpoint{3.618629in}{2.768024in}}%
\pgfpathlineto{\pgfqpoint{3.646280in}{2.732059in}}%
\pgfpathlineto{\pgfqpoint{3.671152in}{2.696094in}}%
\pgfpathlineto{\pgfqpoint{3.693369in}{2.660129in}}%
\pgfpathlineto{\pgfqpoint{3.713043in}{2.624164in}}%
\pgfpathlineto{\pgfqpoint{3.730281in}{2.588200in}}%
\pgfpathlineto{\pgfqpoint{3.745175in}{2.552235in}}%
\pgfpathlineto{\pgfqpoint{3.757810in}{2.516270in}}%
\pgfpathlineto{\pgfqpoint{3.768923in}{2.477736in}}%
\pgfpathlineto{\pgfqpoint{3.777601in}{2.439203in}}%
\pgfpathlineto{\pgfqpoint{3.783905in}{2.400669in}}%
\pgfpathlineto{\pgfqpoint{3.787880in}{2.362135in}}%
\pgfpathlineto{\pgfqpoint{3.789557in}{2.323601in}}%
\pgfpathlineto{\pgfqpoint{3.788950in}{2.285068in}}%
\pgfpathlineto{\pgfqpoint{3.786062in}{2.246534in}}%
\pgfpathlineto{\pgfqpoint{3.780877in}{2.208000in}}%
\pgfpathlineto{\pgfqpoint{3.773365in}{2.169467in}}%
\pgfpathlineto{\pgfqpoint{3.763483in}{2.130933in}}%
\pgfpathlineto{\pgfqpoint{3.751173in}{2.092399in}}%
\pgfpathlineto{\pgfqpoint{3.737428in}{2.056434in}}%
\pgfpathlineto{\pgfqpoint{3.721434in}{2.020470in}}%
\pgfpathlineto{\pgfqpoint{3.703112in}{1.984505in}}%
\pgfpathlineto{\pgfqpoint{3.682370in}{1.948540in}}%
\pgfpathlineto{\pgfqpoint{3.659109in}{1.912575in}}%
\pgfpathlineto{\pgfqpoint{3.633223in}{1.876610in}}%
\pgfpathlineto{\pgfqpoint{3.604596in}{1.840646in}}%
\pgfpathlineto{\pgfqpoint{3.573106in}{1.804681in}}%
\pgfpathlineto{\pgfqpoint{3.538626in}{1.768716in}}%
\pgfpathlineto{\pgfqpoint{3.501027in}{1.732751in}}%
\pgfpathlineto{\pgfqpoint{3.460177in}{1.696786in}}%
\pgfpathlineto{\pgfqpoint{3.415946in}{1.660822in}}%
\pgfpathlineto{\pgfqpoint{3.368208in}{1.624857in}}%
\pgfpathlineto{\pgfqpoint{3.316845in}{1.588892in}}%
\pgfpathlineto{\pgfqpoint{3.261750in}{1.552927in}}%
\pgfpathlineto{\pgfqpoint{3.198476in}{1.514393in}}%
\pgfpathlineto{\pgfqpoint{3.130730in}{1.475860in}}%
\pgfpathlineto{\pgfqpoint{3.058463in}{1.437326in}}%
\pgfpathlineto{\pgfqpoint{2.976393in}{1.396223in}}%
\pgfpathlineto{\pgfqpoint{2.889238in}{1.355121in}}%
\pgfpathlineto{\pgfqpoint{2.791232in}{1.311449in}}%
\pgfpathlineto{\pgfqpoint{2.681717in}{1.265209in}}%
\pgfpathlineto{\pgfqpoint{2.560281in}{1.216399in}}%
\pgfpathlineto{\pgfqpoint{2.420103in}{1.162452in}}%
\pgfpathlineto{\pgfqpoint{2.351563in}{1.136763in}}%
\pgfpathlineto{\pgfqpoint{2.351563in}{1.136763in}}%
\pgfusepath{stroke}%
\end{pgfscope}%
\begin{pgfscope}%
\pgfpathrectangle{\pgfqpoint{0.750000in}{0.880000in}}{\pgfqpoint{4.650000in}{3.080000in}}%
\pgfusepath{clip}%
\pgfsetbuttcap%
\pgfsetroundjoin%
\pgfsetlinewidth{1.505625pt}%
\definecolor{currentstroke}{rgb}{0.121569,0.466667,0.705882}%
\pgfsetstrokecolor{currentstroke}%
\pgfsetdash{{5.550000pt}{2.400000pt}}{0.000000pt}%
\pgfpathmoveto{\pgfqpoint{0.750000in}{3.441079in}}%
\pgfpathlineto{\pgfqpoint{5.400000in}{3.441079in}}%
\pgfusepath{stroke}%
\end{pgfscope}%
\begin{pgfscope}%
\pgfsetrectcap%
\pgfsetmiterjoin%
\pgfsetlinewidth{0.803000pt}%
\definecolor{currentstroke}{rgb}{0.000000,0.000000,0.000000}%
\pgfsetstrokecolor{currentstroke}%
\pgfsetdash{}{0pt}%
\pgfpathmoveto{\pgfqpoint{0.750000in}{0.880000in}}%
\pgfpathlineto{\pgfqpoint{0.750000in}{3.960000in}}%
\pgfusepath{stroke}%
\end{pgfscope}%
\begin{pgfscope}%
\pgfsetrectcap%
\pgfsetmiterjoin%
\pgfsetlinewidth{0.803000pt}%
\definecolor{currentstroke}{rgb}{0.000000,0.000000,0.000000}%
\pgfsetstrokecolor{currentstroke}%
\pgfsetdash{}{0pt}%
\pgfpathmoveto{\pgfqpoint{5.400000in}{0.880000in}}%
\pgfpathlineto{\pgfqpoint{5.400000in}{3.960000in}}%
\pgfusepath{stroke}%
\end{pgfscope}%
\begin{pgfscope}%
\pgfsetrectcap%
\pgfsetmiterjoin%
\pgfsetlinewidth{0.803000pt}%
\definecolor{currentstroke}{rgb}{0.000000,0.000000,0.000000}%
\pgfsetstrokecolor{currentstroke}%
\pgfsetdash{}{0pt}%
\pgfpathmoveto{\pgfqpoint{0.750000in}{0.880000in}}%
\pgfpathlineto{\pgfqpoint{5.400000in}{0.880000in}}%
\pgfusepath{stroke}%
\end{pgfscope}%
\begin{pgfscope}%
\pgfsetrectcap%
\pgfsetmiterjoin%
\pgfsetlinewidth{0.803000pt}%
\definecolor{currentstroke}{rgb}{0.000000,0.000000,0.000000}%
\pgfsetstrokecolor{currentstroke}%
\pgfsetdash{}{0pt}%
\pgfpathmoveto{\pgfqpoint{0.750000in}{3.960000in}}%
\pgfpathlineto{\pgfqpoint{5.400000in}{3.960000in}}%
\pgfusepath{stroke}%
\end{pgfscope}%
\begin{pgfscope}%
\pgfsetbuttcap%
\pgfsetmiterjoin%
\definecolor{currentfill}{rgb}{1.000000,1.000000,1.000000}%
\pgfsetfillcolor{currentfill}%
\pgfsetfillopacity{0.800000}%
\pgfsetlinewidth{1.003750pt}%
\definecolor{currentstroke}{rgb}{0.800000,0.800000,0.800000}%
\pgfsetstrokecolor{currentstroke}%
\pgfsetstrokeopacity{0.800000}%
\pgfsetdash{}{0pt}%
\pgfpathmoveto{\pgfqpoint{3.598877in}{0.963333in}}%
\pgfpathlineto{\pgfqpoint{5.283333in}{0.963333in}}%
\pgfpathquadraticcurveto{\pgfqpoint{5.316667in}{0.963333in}}{\pgfqpoint{5.316667in}{0.996667in}}%
\pgfpathlineto{\pgfqpoint{5.316667in}{1.465921in}}%
\pgfpathquadraticcurveto{\pgfqpoint{5.316667in}{1.499254in}}{\pgfqpoint{5.283333in}{1.499254in}}%
\pgfpathlineto{\pgfqpoint{3.598877in}{1.499254in}}%
\pgfpathquadraticcurveto{\pgfqpoint{3.565544in}{1.499254in}}{\pgfqpoint{3.565544in}{1.465921in}}%
\pgfpathlineto{\pgfqpoint{3.565544in}{0.996667in}}%
\pgfpathquadraticcurveto{\pgfqpoint{3.565544in}{0.963333in}}{\pgfqpoint{3.598877in}{0.963333in}}%
\pgfpathlineto{\pgfqpoint{3.598877in}{0.963333in}}%
\pgfpathclose%
\pgfusepath{stroke,fill}%
\end{pgfscope}%
\begin{pgfscope}%
\pgfsetrectcap%
\pgfsetroundjoin%
\pgfsetlinewidth{1.505625pt}%
\definecolor{currentstroke}{rgb}{0.121569,0.466667,0.705882}%
\pgfsetstrokecolor{currentstroke}%
\pgfsetdash{}{0pt}%
\pgfpathmoveto{\pgfqpoint{3.632210in}{1.368281in}}%
\pgfpathlineto{\pgfqpoint{3.798877in}{1.368281in}}%
\pgfpathlineto{\pgfqpoint{3.965544in}{1.368281in}}%
\pgfusepath{stroke}%
\end{pgfscope}%
\begin{pgfscope}%
\definecolor{textcolor}{rgb}{0.000000,0.000000,0.000000}%
\pgfsetstrokecolor{textcolor}%
\pgfsetfillcolor{textcolor}%
\pgftext[x=4.098877in,y=1.309948in,left,base]{\color{textcolor}\rmfamily\fontsize{12.000000}{14.400000}\selectfont delta}%
\end{pgfscope}%
\begin{pgfscope}%
\pgfsetbuttcap%
\pgfsetroundjoin%
\pgfsetlinewidth{1.505625pt}%
\definecolor{currentstroke}{rgb}{0.121569,0.466667,0.705882}%
\pgfsetstrokecolor{currentstroke}%
\pgfsetdash{{5.550000pt}{2.400000pt}}{0.000000pt}%
\pgfpathmoveto{\pgfqpoint{3.632210in}{1.124629in}}%
\pgfpathlineto{\pgfqpoint{3.798877in}{1.124629in}}%
\pgfpathlineto{\pgfqpoint{3.965544in}{1.124629in}}%
\pgfusepath{stroke}%
\end{pgfscope}%
\begin{pgfscope}%
\definecolor{textcolor}{rgb}{0.000000,0.000000,0.000000}%
\pgfsetstrokecolor{textcolor}%
\pgfsetfillcolor{textcolor}%
\pgftext[x=4.098877in,y=1.066295in,left,base]{\color{textcolor}\rmfamily\fontsize{12.000000}{14.400000}\selectfont clearing of fault}%
\end{pgfscope}%
\begin{pgfscope}%
\definecolor{textcolor}{rgb}{0.000000,0.000000,0.000000}%
\pgfsetstrokecolor{textcolor}%
\pgfsetfillcolor{textcolor}%
\pgftext[x=3.000000in,y=7.840000in,,top]{\color{textcolor}\rmfamily\fontsize{14.400000}{17.280000}\selectfont Stable scenario}%
\end{pgfscope}%
\end{pgfpicture}%
\makeatother%
\endgroup%


%% Creator: Matplotlib, PGF backend
%%
%% To include the figure in your LaTeX document, write
%%   \input{<filename>.pgf}
%%
%% Make sure the required packages are loaded in your preamble
%%   \usepackage{pgf}
%%
%% Also ensure that all the required font packages are loaded; for instance,
%% the lmodern package is sometimes necessary when using math font.
%%   \usepackage{lmodern}
%%
%% Figures using additional raster images can only be included by \input if
%% they are in the same directory as the main LaTeX file. For loading figures
%% from other directories you can use the `import` package
%%   \usepackage{import}
%%
%% and then include the figures with
%%   \import{<path to file>}{<filename>.pgf}
%%
%% Matplotlib used the following preamble
%%   
%%   \usepackage{fontspec}
%%   \setmainfont{Charter.ttc}[Path=\detokenize{/System/Library/Fonts/Supplemental/}]
%%   \setsansfont{DejaVuSans.ttf}[Path=\detokenize{/opt/homebrew/lib/python3.10/site-packages/matplotlib/mpl-data/fonts/ttf/}]
%%   \setmonofont{DejaVuSansMono.ttf}[Path=\detokenize{/opt/homebrew/lib/python3.10/site-packages/matplotlib/mpl-data/fonts/ttf/}]
%%   \makeatletter\@ifpackageloaded{underscore}{}{\usepackage[strings]{underscore}}\makeatother
%%
\begingroup%
\makeatletter%
\begin{pgfpicture}%
\pgfpathrectangle{\pgfpointorigin}{\pgfqpoint{6.000000in}{8.000000in}}%
\pgfusepath{use as bounding box, clip}%
\begin{pgfscope}%
\pgfsetbuttcap%
\pgfsetmiterjoin%
\definecolor{currentfill}{rgb}{1.000000,1.000000,1.000000}%
\pgfsetfillcolor{currentfill}%
\pgfsetlinewidth{0.000000pt}%
\definecolor{currentstroke}{rgb}{1.000000,1.000000,1.000000}%
\pgfsetstrokecolor{currentstroke}%
\pgfsetdash{}{0pt}%
\pgfpathmoveto{\pgfqpoint{0.000000in}{0.000000in}}%
\pgfpathlineto{\pgfqpoint{6.000000in}{0.000000in}}%
\pgfpathlineto{\pgfqpoint{6.000000in}{8.000000in}}%
\pgfpathlineto{\pgfqpoint{0.000000in}{8.000000in}}%
\pgfpathlineto{\pgfqpoint{0.000000in}{0.000000in}}%
\pgfpathclose%
\pgfusepath{fill}%
\end{pgfscope}%
\begin{pgfscope}%
\pgfsetbuttcap%
\pgfsetmiterjoin%
\definecolor{currentfill}{rgb}{1.000000,1.000000,1.000000}%
\pgfsetfillcolor{currentfill}%
\pgfsetlinewidth{0.000000pt}%
\definecolor{currentstroke}{rgb}{0.000000,0.000000,0.000000}%
\pgfsetstrokecolor{currentstroke}%
\pgfsetstrokeopacity{0.000000}%
\pgfsetdash{}{0pt}%
\pgfpathmoveto{\pgfqpoint{0.750000in}{3.960000in}}%
\pgfpathlineto{\pgfqpoint{5.400000in}{3.960000in}}%
\pgfpathlineto{\pgfqpoint{5.400000in}{7.040000in}}%
\pgfpathlineto{\pgfqpoint{0.750000in}{7.040000in}}%
\pgfpathlineto{\pgfqpoint{0.750000in}{3.960000in}}%
\pgfpathclose%
\pgfusepath{fill}%
\end{pgfscope}%
\begin{pgfscope}%
\pgfpathrectangle{\pgfqpoint{0.750000in}{3.960000in}}{\pgfqpoint{4.650000in}{3.080000in}}%
\pgfusepath{clip}%
\pgfsetbuttcap%
\pgfsetroundjoin%
\definecolor{currentfill}{rgb}{0.900000,0.900000,0.900000}%
\pgfsetfillcolor{currentfill}%
\pgfsetlinewidth{1.003750pt}%
\definecolor{currentstroke}{rgb}{0.500000,0.500000,0.500000}%
\pgfsetstrokecolor{currentstroke}%
\pgfsetdash{}{0pt}%
\pgfsys@defobject{currentmarker}{\pgfqpoint{2.005500in}{3.960087in}}{\pgfqpoint{2.452055in}{6.161131in}}{%
\pgfpathmoveto{\pgfqpoint{2.005500in}{6.161131in}}%
\pgfpathlineto{\pgfqpoint{2.005500in}{3.960092in}}%
\pgfpathlineto{\pgfqpoint{2.014613in}{3.960092in}}%
\pgfpathlineto{\pgfqpoint{2.023727in}{3.960092in}}%
\pgfpathlineto{\pgfqpoint{2.032840in}{3.960091in}}%
\pgfpathlineto{\pgfqpoint{2.041953in}{3.960091in}}%
\pgfpathlineto{\pgfqpoint{2.051067in}{3.960091in}}%
\pgfpathlineto{\pgfqpoint{2.060180in}{3.960091in}}%
\pgfpathlineto{\pgfqpoint{2.069294in}{3.960091in}}%
\pgfpathlineto{\pgfqpoint{2.078407in}{3.960091in}}%
\pgfpathlineto{\pgfqpoint{2.087520in}{3.960091in}}%
\pgfpathlineto{\pgfqpoint{2.096634in}{3.960091in}}%
\pgfpathlineto{\pgfqpoint{2.105747in}{3.960091in}}%
\pgfpathlineto{\pgfqpoint{2.114860in}{3.960091in}}%
\pgfpathlineto{\pgfqpoint{2.123974in}{3.960091in}}%
\pgfpathlineto{\pgfqpoint{2.133087in}{3.960091in}}%
\pgfpathlineto{\pgfqpoint{2.142201in}{3.960090in}}%
\pgfpathlineto{\pgfqpoint{2.151314in}{3.960090in}}%
\pgfpathlineto{\pgfqpoint{2.160427in}{3.960090in}}%
\pgfpathlineto{\pgfqpoint{2.169541in}{3.960090in}}%
\pgfpathlineto{\pgfqpoint{2.178654in}{3.960090in}}%
\pgfpathlineto{\pgfqpoint{2.187767in}{3.960090in}}%
\pgfpathlineto{\pgfqpoint{2.196881in}{3.960090in}}%
\pgfpathlineto{\pgfqpoint{2.205994in}{3.960090in}}%
\pgfpathlineto{\pgfqpoint{2.215108in}{3.960090in}}%
\pgfpathlineto{\pgfqpoint{2.224221in}{3.960090in}}%
\pgfpathlineto{\pgfqpoint{2.233334in}{3.960089in}}%
\pgfpathlineto{\pgfqpoint{2.242448in}{3.960089in}}%
\pgfpathlineto{\pgfqpoint{2.251561in}{3.960089in}}%
\pgfpathlineto{\pgfqpoint{2.260674in}{3.960089in}}%
\pgfpathlineto{\pgfqpoint{2.269788in}{3.960089in}}%
\pgfpathlineto{\pgfqpoint{2.278901in}{3.960089in}}%
\pgfpathlineto{\pgfqpoint{2.288015in}{3.960089in}}%
\pgfpathlineto{\pgfqpoint{2.297128in}{3.960089in}}%
\pgfpathlineto{\pgfqpoint{2.306241in}{3.960089in}}%
\pgfpathlineto{\pgfqpoint{2.315355in}{3.960089in}}%
\pgfpathlineto{\pgfqpoint{2.324468in}{3.960088in}}%
\pgfpathlineto{\pgfqpoint{2.333581in}{3.960088in}}%
\pgfpathlineto{\pgfqpoint{2.342695in}{3.960088in}}%
\pgfpathlineto{\pgfqpoint{2.351808in}{3.960088in}}%
\pgfpathlineto{\pgfqpoint{2.360922in}{3.960088in}}%
\pgfpathlineto{\pgfqpoint{2.370035in}{3.960088in}}%
\pgfpathlineto{\pgfqpoint{2.379148in}{3.960088in}}%
\pgfpathlineto{\pgfqpoint{2.388262in}{3.960088in}}%
\pgfpathlineto{\pgfqpoint{2.397375in}{3.960088in}}%
\pgfpathlineto{\pgfqpoint{2.406488in}{3.960088in}}%
\pgfpathlineto{\pgfqpoint{2.415602in}{3.960087in}}%
\pgfpathlineto{\pgfqpoint{2.424715in}{3.960087in}}%
\pgfpathlineto{\pgfqpoint{2.433828in}{3.960087in}}%
\pgfpathlineto{\pgfqpoint{2.442942in}{3.960087in}}%
\pgfpathlineto{\pgfqpoint{2.452055in}{3.960087in}}%
\pgfpathlineto{\pgfqpoint{2.452055in}{6.161131in}}%
\pgfpathlineto{\pgfqpoint{2.452055in}{6.161131in}}%
\pgfpathlineto{\pgfqpoint{2.442942in}{6.161131in}}%
\pgfpathlineto{\pgfqpoint{2.433828in}{6.161131in}}%
\pgfpathlineto{\pgfqpoint{2.424715in}{6.161131in}}%
\pgfpathlineto{\pgfqpoint{2.415602in}{6.161131in}}%
\pgfpathlineto{\pgfqpoint{2.406488in}{6.161131in}}%
\pgfpathlineto{\pgfqpoint{2.397375in}{6.161131in}}%
\pgfpathlineto{\pgfqpoint{2.388262in}{6.161131in}}%
\pgfpathlineto{\pgfqpoint{2.379148in}{6.161131in}}%
\pgfpathlineto{\pgfqpoint{2.370035in}{6.161131in}}%
\pgfpathlineto{\pgfqpoint{2.360922in}{6.161131in}}%
\pgfpathlineto{\pgfqpoint{2.351808in}{6.161131in}}%
\pgfpathlineto{\pgfqpoint{2.342695in}{6.161131in}}%
\pgfpathlineto{\pgfqpoint{2.333581in}{6.161131in}}%
\pgfpathlineto{\pgfqpoint{2.324468in}{6.161131in}}%
\pgfpathlineto{\pgfqpoint{2.315355in}{6.161131in}}%
\pgfpathlineto{\pgfqpoint{2.306241in}{6.161131in}}%
\pgfpathlineto{\pgfqpoint{2.297128in}{6.161131in}}%
\pgfpathlineto{\pgfqpoint{2.288015in}{6.161131in}}%
\pgfpathlineto{\pgfqpoint{2.278901in}{6.161131in}}%
\pgfpathlineto{\pgfqpoint{2.269788in}{6.161131in}}%
\pgfpathlineto{\pgfqpoint{2.260674in}{6.161131in}}%
\pgfpathlineto{\pgfqpoint{2.251561in}{6.161131in}}%
\pgfpathlineto{\pgfqpoint{2.242448in}{6.161131in}}%
\pgfpathlineto{\pgfqpoint{2.233334in}{6.161131in}}%
\pgfpathlineto{\pgfqpoint{2.224221in}{6.161131in}}%
\pgfpathlineto{\pgfqpoint{2.215108in}{6.161131in}}%
\pgfpathlineto{\pgfqpoint{2.205994in}{6.161131in}}%
\pgfpathlineto{\pgfqpoint{2.196881in}{6.161131in}}%
\pgfpathlineto{\pgfqpoint{2.187767in}{6.161131in}}%
\pgfpathlineto{\pgfqpoint{2.178654in}{6.161131in}}%
\pgfpathlineto{\pgfqpoint{2.169541in}{6.161131in}}%
\pgfpathlineto{\pgfqpoint{2.160427in}{6.161131in}}%
\pgfpathlineto{\pgfqpoint{2.151314in}{6.161131in}}%
\pgfpathlineto{\pgfqpoint{2.142201in}{6.161131in}}%
\pgfpathlineto{\pgfqpoint{2.133087in}{6.161131in}}%
\pgfpathlineto{\pgfqpoint{2.123974in}{6.161131in}}%
\pgfpathlineto{\pgfqpoint{2.114860in}{6.161131in}}%
\pgfpathlineto{\pgfqpoint{2.105747in}{6.161131in}}%
\pgfpathlineto{\pgfqpoint{2.096634in}{6.161131in}}%
\pgfpathlineto{\pgfqpoint{2.087520in}{6.161131in}}%
\pgfpathlineto{\pgfqpoint{2.078407in}{6.161131in}}%
\pgfpathlineto{\pgfqpoint{2.069294in}{6.161131in}}%
\pgfpathlineto{\pgfqpoint{2.060180in}{6.161131in}}%
\pgfpathlineto{\pgfqpoint{2.051067in}{6.161131in}}%
\pgfpathlineto{\pgfqpoint{2.041953in}{6.161131in}}%
\pgfpathlineto{\pgfqpoint{2.032840in}{6.161131in}}%
\pgfpathlineto{\pgfqpoint{2.023727in}{6.161131in}}%
\pgfpathlineto{\pgfqpoint{2.014613in}{6.161131in}}%
\pgfpathlineto{\pgfqpoint{2.005500in}{6.161131in}}%
\pgfpathlineto{\pgfqpoint{2.005500in}{6.161131in}}%
\pgfpathclose%
\pgfusepath{stroke,fill}%
}%
\begin{pgfscope}%
\pgfsys@transformshift{0.000000in}{0.000000in}%
\pgfsys@useobject{currentmarker}{}%
\end{pgfscope}%
\end{pgfscope}%
\begin{pgfscope}%
\pgfpathrectangle{\pgfqpoint{0.750000in}{3.960000in}}{\pgfqpoint{4.650000in}{3.080000in}}%
\pgfusepath{clip}%
\pgfsetbuttcap%
\pgfsetroundjoin%
\definecolor{currentfill}{rgb}{0.900000,0.900000,0.900000}%
\pgfsetfillcolor{currentfill}%
\pgfsetlinewidth{1.003750pt}%
\definecolor{currentstroke}{rgb}{0.500000,0.500000,0.500000}%
\pgfsetstrokecolor{currentstroke}%
\pgfsetdash{}{0pt}%
\pgfsys@defobject{currentmarker}{\pgfqpoint{2.452055in}{6.161131in}}{\pgfqpoint{4.144500in}{6.894840in}}{%
\pgfpathmoveto{\pgfqpoint{2.452055in}{6.161131in}}%
\pgfpathlineto{\pgfqpoint{2.452055in}{6.638731in}}%
\pgfpathlineto{\pgfqpoint{2.486595in}{6.665979in}}%
\pgfpathlineto{\pgfqpoint{2.521135in}{6.691754in}}%
\pgfpathlineto{\pgfqpoint{2.555674in}{6.716041in}}%
\pgfpathlineto{\pgfqpoint{2.590214in}{6.738827in}}%
\pgfpathlineto{\pgfqpoint{2.624754in}{6.760101in}}%
\pgfpathlineto{\pgfqpoint{2.659293in}{6.779850in}}%
\pgfpathlineto{\pgfqpoint{2.693833in}{6.798063in}}%
\pgfpathlineto{\pgfqpoint{2.728373in}{6.814731in}}%
\pgfpathlineto{\pgfqpoint{2.762912in}{6.829844in}}%
\pgfpathlineto{\pgfqpoint{2.797452in}{6.843395in}}%
\pgfpathlineto{\pgfqpoint{2.831992in}{6.855376in}}%
\pgfpathlineto{\pgfqpoint{2.866531in}{6.865780in}}%
\pgfpathlineto{\pgfqpoint{2.901071in}{6.874602in}}%
\pgfpathlineto{\pgfqpoint{2.935611in}{6.881837in}}%
\pgfpathlineto{\pgfqpoint{2.970151in}{6.887481in}}%
\pgfpathlineto{\pgfqpoint{3.004690in}{6.891531in}}%
\pgfpathlineto{\pgfqpoint{3.039230in}{6.893984in}}%
\pgfpathlineto{\pgfqpoint{3.073770in}{6.894840in}}%
\pgfpathlineto{\pgfqpoint{3.108309in}{6.894098in}}%
\pgfpathlineto{\pgfqpoint{3.142849in}{6.891758in}}%
\pgfpathlineto{\pgfqpoint{3.177389in}{6.887822in}}%
\pgfpathlineto{\pgfqpoint{3.211928in}{6.882292in}}%
\pgfpathlineto{\pgfqpoint{3.246468in}{6.875170in}}%
\pgfpathlineto{\pgfqpoint{3.281008in}{6.866461in}}%
\pgfpathlineto{\pgfqpoint{3.315547in}{6.856169in}}%
\pgfpathlineto{\pgfqpoint{3.350087in}{6.844301in}}%
\pgfpathlineto{\pgfqpoint{3.384627in}{6.830862in}}%
\pgfpathlineto{\pgfqpoint{3.419167in}{6.815859in}}%
\pgfpathlineto{\pgfqpoint{3.453706in}{6.799302in}}%
\pgfpathlineto{\pgfqpoint{3.488246in}{6.781198in}}%
\pgfpathlineto{\pgfqpoint{3.522786in}{6.761559in}}%
\pgfpathlineto{\pgfqpoint{3.557325in}{6.740393in}}%
\pgfpathlineto{\pgfqpoint{3.591865in}{6.717714in}}%
\pgfpathlineto{\pgfqpoint{3.626405in}{6.693533in}}%
\pgfpathlineto{\pgfqpoint{3.660944in}{6.667864in}}%
\pgfpathlineto{\pgfqpoint{3.695484in}{6.640720in}}%
\pgfpathlineto{\pgfqpoint{3.730024in}{6.612117in}}%
\pgfpathlineto{\pgfqpoint{3.764563in}{6.582069in}}%
\pgfpathlineto{\pgfqpoint{3.799103in}{6.550594in}}%
\pgfpathlineto{\pgfqpoint{3.833643in}{6.517708in}}%
\pgfpathlineto{\pgfqpoint{3.868182in}{6.483429in}}%
\pgfpathlineto{\pgfqpoint{3.902722in}{6.447777in}}%
\pgfpathlineto{\pgfqpoint{3.937262in}{6.410769in}}%
\pgfpathlineto{\pgfqpoint{3.971802in}{6.372428in}}%
\pgfpathlineto{\pgfqpoint{4.006341in}{6.332772in}}%
\pgfpathlineto{\pgfqpoint{4.040881in}{6.291825in}}%
\pgfpathlineto{\pgfqpoint{4.075421in}{6.249608in}}%
\pgfpathlineto{\pgfqpoint{4.109960in}{6.206144in}}%
\pgfpathlineto{\pgfqpoint{4.144500in}{6.161457in}}%
\pgfpathlineto{\pgfqpoint{4.144500in}{6.161131in}}%
\pgfpathlineto{\pgfqpoint{4.144500in}{6.161131in}}%
\pgfpathlineto{\pgfqpoint{4.109960in}{6.161131in}}%
\pgfpathlineto{\pgfqpoint{4.075421in}{6.161131in}}%
\pgfpathlineto{\pgfqpoint{4.040881in}{6.161131in}}%
\pgfpathlineto{\pgfqpoint{4.006341in}{6.161131in}}%
\pgfpathlineto{\pgfqpoint{3.971802in}{6.161131in}}%
\pgfpathlineto{\pgfqpoint{3.937262in}{6.161131in}}%
\pgfpathlineto{\pgfqpoint{3.902722in}{6.161131in}}%
\pgfpathlineto{\pgfqpoint{3.868182in}{6.161131in}}%
\pgfpathlineto{\pgfqpoint{3.833643in}{6.161131in}}%
\pgfpathlineto{\pgfqpoint{3.799103in}{6.161131in}}%
\pgfpathlineto{\pgfqpoint{3.764563in}{6.161131in}}%
\pgfpathlineto{\pgfqpoint{3.730024in}{6.161131in}}%
\pgfpathlineto{\pgfqpoint{3.695484in}{6.161131in}}%
\pgfpathlineto{\pgfqpoint{3.660944in}{6.161131in}}%
\pgfpathlineto{\pgfqpoint{3.626405in}{6.161131in}}%
\pgfpathlineto{\pgfqpoint{3.591865in}{6.161131in}}%
\pgfpathlineto{\pgfqpoint{3.557325in}{6.161131in}}%
\pgfpathlineto{\pgfqpoint{3.522786in}{6.161131in}}%
\pgfpathlineto{\pgfqpoint{3.488246in}{6.161131in}}%
\pgfpathlineto{\pgfqpoint{3.453706in}{6.161131in}}%
\pgfpathlineto{\pgfqpoint{3.419167in}{6.161131in}}%
\pgfpathlineto{\pgfqpoint{3.384627in}{6.161131in}}%
\pgfpathlineto{\pgfqpoint{3.350087in}{6.161131in}}%
\pgfpathlineto{\pgfqpoint{3.315547in}{6.161131in}}%
\pgfpathlineto{\pgfqpoint{3.281008in}{6.161131in}}%
\pgfpathlineto{\pgfqpoint{3.246468in}{6.161131in}}%
\pgfpathlineto{\pgfqpoint{3.211928in}{6.161131in}}%
\pgfpathlineto{\pgfqpoint{3.177389in}{6.161131in}}%
\pgfpathlineto{\pgfqpoint{3.142849in}{6.161131in}}%
\pgfpathlineto{\pgfqpoint{3.108309in}{6.161131in}}%
\pgfpathlineto{\pgfqpoint{3.073770in}{6.161131in}}%
\pgfpathlineto{\pgfqpoint{3.039230in}{6.161131in}}%
\pgfpathlineto{\pgfqpoint{3.004690in}{6.161131in}}%
\pgfpathlineto{\pgfqpoint{2.970151in}{6.161131in}}%
\pgfpathlineto{\pgfqpoint{2.935611in}{6.161131in}}%
\pgfpathlineto{\pgfqpoint{2.901071in}{6.161131in}}%
\pgfpathlineto{\pgfqpoint{2.866531in}{6.161131in}}%
\pgfpathlineto{\pgfqpoint{2.831992in}{6.161131in}}%
\pgfpathlineto{\pgfqpoint{2.797452in}{6.161131in}}%
\pgfpathlineto{\pgfqpoint{2.762912in}{6.161131in}}%
\pgfpathlineto{\pgfqpoint{2.728373in}{6.161131in}}%
\pgfpathlineto{\pgfqpoint{2.693833in}{6.161131in}}%
\pgfpathlineto{\pgfqpoint{2.659293in}{6.161131in}}%
\pgfpathlineto{\pgfqpoint{2.624754in}{6.161131in}}%
\pgfpathlineto{\pgfqpoint{2.590214in}{6.161131in}}%
\pgfpathlineto{\pgfqpoint{2.555674in}{6.161131in}}%
\pgfpathlineto{\pgfqpoint{2.521135in}{6.161131in}}%
\pgfpathlineto{\pgfqpoint{2.486595in}{6.161131in}}%
\pgfpathlineto{\pgfqpoint{2.452055in}{6.161131in}}%
\pgfpathlineto{\pgfqpoint{2.452055in}{6.161131in}}%
\pgfpathclose%
\pgfusepath{stroke,fill}%
}%
\begin{pgfscope}%
\pgfsys@transformshift{0.000000in}{0.000000in}%
\pgfsys@useobject{currentmarker}{}%
\end{pgfscope}%
\end{pgfscope}%
\begin{pgfscope}%
\pgfpathrectangle{\pgfqpoint{0.750000in}{3.960000in}}{\pgfqpoint{4.650000in}{3.080000in}}%
\pgfusepath{clip}%
\pgfsetrectcap%
\pgfsetroundjoin%
\pgfsetlinewidth{0.803000pt}%
\definecolor{currentstroke}{rgb}{0.690196,0.690196,0.690196}%
\pgfsetstrokecolor{currentstroke}%
\pgfsetdash{}{0pt}%
\pgfpathmoveto{\pgfqpoint{0.750000in}{3.960000in}}%
\pgfpathlineto{\pgfqpoint{0.750000in}{7.040000in}}%
\pgfusepath{stroke}%
\end{pgfscope}%
\begin{pgfscope}%
\pgfsetbuttcap%
\pgfsetroundjoin%
\definecolor{currentfill}{rgb}{0.000000,0.000000,0.000000}%
\pgfsetfillcolor{currentfill}%
\pgfsetlinewidth{0.803000pt}%
\definecolor{currentstroke}{rgb}{0.000000,0.000000,0.000000}%
\pgfsetstrokecolor{currentstroke}%
\pgfsetdash{}{0pt}%
\pgfsys@defobject{currentmarker}{\pgfqpoint{0.000000in}{-0.048611in}}{\pgfqpoint{0.000000in}{0.000000in}}{%
\pgfpathmoveto{\pgfqpoint{0.000000in}{0.000000in}}%
\pgfpathlineto{\pgfqpoint{0.000000in}{-0.048611in}}%
\pgfusepath{stroke,fill}%
}%
\begin{pgfscope}%
\pgfsys@transformshift{0.750000in}{3.960000in}%
\pgfsys@useobject{currentmarker}{}%
\end{pgfscope}%
\end{pgfscope}%
\begin{pgfscope}%
\pgfpathrectangle{\pgfqpoint{0.750000in}{3.960000in}}{\pgfqpoint{4.650000in}{3.080000in}}%
\pgfusepath{clip}%
\pgfsetrectcap%
\pgfsetroundjoin%
\pgfsetlinewidth{0.803000pt}%
\definecolor{currentstroke}{rgb}{0.690196,0.690196,0.690196}%
\pgfsetstrokecolor{currentstroke}%
\pgfsetdash{}{0pt}%
\pgfpathmoveto{\pgfqpoint{1.266667in}{3.960000in}}%
\pgfpathlineto{\pgfqpoint{1.266667in}{7.040000in}}%
\pgfusepath{stroke}%
\end{pgfscope}%
\begin{pgfscope}%
\pgfsetbuttcap%
\pgfsetroundjoin%
\definecolor{currentfill}{rgb}{0.000000,0.000000,0.000000}%
\pgfsetfillcolor{currentfill}%
\pgfsetlinewidth{0.803000pt}%
\definecolor{currentstroke}{rgb}{0.000000,0.000000,0.000000}%
\pgfsetstrokecolor{currentstroke}%
\pgfsetdash{}{0pt}%
\pgfsys@defobject{currentmarker}{\pgfqpoint{0.000000in}{-0.048611in}}{\pgfqpoint{0.000000in}{0.000000in}}{%
\pgfpathmoveto{\pgfqpoint{0.000000in}{0.000000in}}%
\pgfpathlineto{\pgfqpoint{0.000000in}{-0.048611in}}%
\pgfusepath{stroke,fill}%
}%
\begin{pgfscope}%
\pgfsys@transformshift{1.266667in}{3.960000in}%
\pgfsys@useobject{currentmarker}{}%
\end{pgfscope}%
\end{pgfscope}%
\begin{pgfscope}%
\pgfpathrectangle{\pgfqpoint{0.750000in}{3.960000in}}{\pgfqpoint{4.650000in}{3.080000in}}%
\pgfusepath{clip}%
\pgfsetrectcap%
\pgfsetroundjoin%
\pgfsetlinewidth{0.803000pt}%
\definecolor{currentstroke}{rgb}{0.690196,0.690196,0.690196}%
\pgfsetstrokecolor{currentstroke}%
\pgfsetdash{}{0pt}%
\pgfpathmoveto{\pgfqpoint{1.783333in}{3.960000in}}%
\pgfpathlineto{\pgfqpoint{1.783333in}{7.040000in}}%
\pgfusepath{stroke}%
\end{pgfscope}%
\begin{pgfscope}%
\pgfsetbuttcap%
\pgfsetroundjoin%
\definecolor{currentfill}{rgb}{0.000000,0.000000,0.000000}%
\pgfsetfillcolor{currentfill}%
\pgfsetlinewidth{0.803000pt}%
\definecolor{currentstroke}{rgb}{0.000000,0.000000,0.000000}%
\pgfsetstrokecolor{currentstroke}%
\pgfsetdash{}{0pt}%
\pgfsys@defobject{currentmarker}{\pgfqpoint{0.000000in}{-0.048611in}}{\pgfqpoint{0.000000in}{0.000000in}}{%
\pgfpathmoveto{\pgfqpoint{0.000000in}{0.000000in}}%
\pgfpathlineto{\pgfqpoint{0.000000in}{-0.048611in}}%
\pgfusepath{stroke,fill}%
}%
\begin{pgfscope}%
\pgfsys@transformshift{1.783333in}{3.960000in}%
\pgfsys@useobject{currentmarker}{}%
\end{pgfscope}%
\end{pgfscope}%
\begin{pgfscope}%
\pgfpathrectangle{\pgfqpoint{0.750000in}{3.960000in}}{\pgfqpoint{4.650000in}{3.080000in}}%
\pgfusepath{clip}%
\pgfsetrectcap%
\pgfsetroundjoin%
\pgfsetlinewidth{0.803000pt}%
\definecolor{currentstroke}{rgb}{0.690196,0.690196,0.690196}%
\pgfsetstrokecolor{currentstroke}%
\pgfsetdash{}{0pt}%
\pgfpathmoveto{\pgfqpoint{2.300000in}{3.960000in}}%
\pgfpathlineto{\pgfqpoint{2.300000in}{7.040000in}}%
\pgfusepath{stroke}%
\end{pgfscope}%
\begin{pgfscope}%
\pgfsetbuttcap%
\pgfsetroundjoin%
\definecolor{currentfill}{rgb}{0.000000,0.000000,0.000000}%
\pgfsetfillcolor{currentfill}%
\pgfsetlinewidth{0.803000pt}%
\definecolor{currentstroke}{rgb}{0.000000,0.000000,0.000000}%
\pgfsetstrokecolor{currentstroke}%
\pgfsetdash{}{0pt}%
\pgfsys@defobject{currentmarker}{\pgfqpoint{0.000000in}{-0.048611in}}{\pgfqpoint{0.000000in}{0.000000in}}{%
\pgfpathmoveto{\pgfqpoint{0.000000in}{0.000000in}}%
\pgfpathlineto{\pgfqpoint{0.000000in}{-0.048611in}}%
\pgfusepath{stroke,fill}%
}%
\begin{pgfscope}%
\pgfsys@transformshift{2.300000in}{3.960000in}%
\pgfsys@useobject{currentmarker}{}%
\end{pgfscope}%
\end{pgfscope}%
\begin{pgfscope}%
\pgfpathrectangle{\pgfqpoint{0.750000in}{3.960000in}}{\pgfqpoint{4.650000in}{3.080000in}}%
\pgfusepath{clip}%
\pgfsetrectcap%
\pgfsetroundjoin%
\pgfsetlinewidth{0.803000pt}%
\definecolor{currentstroke}{rgb}{0.690196,0.690196,0.690196}%
\pgfsetstrokecolor{currentstroke}%
\pgfsetdash{}{0pt}%
\pgfpathmoveto{\pgfqpoint{2.816667in}{3.960000in}}%
\pgfpathlineto{\pgfqpoint{2.816667in}{7.040000in}}%
\pgfusepath{stroke}%
\end{pgfscope}%
\begin{pgfscope}%
\pgfsetbuttcap%
\pgfsetroundjoin%
\definecolor{currentfill}{rgb}{0.000000,0.000000,0.000000}%
\pgfsetfillcolor{currentfill}%
\pgfsetlinewidth{0.803000pt}%
\definecolor{currentstroke}{rgb}{0.000000,0.000000,0.000000}%
\pgfsetstrokecolor{currentstroke}%
\pgfsetdash{}{0pt}%
\pgfsys@defobject{currentmarker}{\pgfqpoint{0.000000in}{-0.048611in}}{\pgfqpoint{0.000000in}{0.000000in}}{%
\pgfpathmoveto{\pgfqpoint{0.000000in}{0.000000in}}%
\pgfpathlineto{\pgfqpoint{0.000000in}{-0.048611in}}%
\pgfusepath{stroke,fill}%
}%
\begin{pgfscope}%
\pgfsys@transformshift{2.816667in}{3.960000in}%
\pgfsys@useobject{currentmarker}{}%
\end{pgfscope}%
\end{pgfscope}%
\begin{pgfscope}%
\pgfpathrectangle{\pgfqpoint{0.750000in}{3.960000in}}{\pgfqpoint{4.650000in}{3.080000in}}%
\pgfusepath{clip}%
\pgfsetrectcap%
\pgfsetroundjoin%
\pgfsetlinewidth{0.803000pt}%
\definecolor{currentstroke}{rgb}{0.690196,0.690196,0.690196}%
\pgfsetstrokecolor{currentstroke}%
\pgfsetdash{}{0pt}%
\pgfpathmoveto{\pgfqpoint{3.333333in}{3.960000in}}%
\pgfpathlineto{\pgfqpoint{3.333333in}{7.040000in}}%
\pgfusepath{stroke}%
\end{pgfscope}%
\begin{pgfscope}%
\pgfsetbuttcap%
\pgfsetroundjoin%
\definecolor{currentfill}{rgb}{0.000000,0.000000,0.000000}%
\pgfsetfillcolor{currentfill}%
\pgfsetlinewidth{0.803000pt}%
\definecolor{currentstroke}{rgb}{0.000000,0.000000,0.000000}%
\pgfsetstrokecolor{currentstroke}%
\pgfsetdash{}{0pt}%
\pgfsys@defobject{currentmarker}{\pgfqpoint{0.000000in}{-0.048611in}}{\pgfqpoint{0.000000in}{0.000000in}}{%
\pgfpathmoveto{\pgfqpoint{0.000000in}{0.000000in}}%
\pgfpathlineto{\pgfqpoint{0.000000in}{-0.048611in}}%
\pgfusepath{stroke,fill}%
}%
\begin{pgfscope}%
\pgfsys@transformshift{3.333333in}{3.960000in}%
\pgfsys@useobject{currentmarker}{}%
\end{pgfscope}%
\end{pgfscope}%
\begin{pgfscope}%
\pgfpathrectangle{\pgfqpoint{0.750000in}{3.960000in}}{\pgfqpoint{4.650000in}{3.080000in}}%
\pgfusepath{clip}%
\pgfsetrectcap%
\pgfsetroundjoin%
\pgfsetlinewidth{0.803000pt}%
\definecolor{currentstroke}{rgb}{0.690196,0.690196,0.690196}%
\pgfsetstrokecolor{currentstroke}%
\pgfsetdash{}{0pt}%
\pgfpathmoveto{\pgfqpoint{3.850000in}{3.960000in}}%
\pgfpathlineto{\pgfqpoint{3.850000in}{7.040000in}}%
\pgfusepath{stroke}%
\end{pgfscope}%
\begin{pgfscope}%
\pgfsetbuttcap%
\pgfsetroundjoin%
\definecolor{currentfill}{rgb}{0.000000,0.000000,0.000000}%
\pgfsetfillcolor{currentfill}%
\pgfsetlinewidth{0.803000pt}%
\definecolor{currentstroke}{rgb}{0.000000,0.000000,0.000000}%
\pgfsetstrokecolor{currentstroke}%
\pgfsetdash{}{0pt}%
\pgfsys@defobject{currentmarker}{\pgfqpoint{0.000000in}{-0.048611in}}{\pgfqpoint{0.000000in}{0.000000in}}{%
\pgfpathmoveto{\pgfqpoint{0.000000in}{0.000000in}}%
\pgfpathlineto{\pgfqpoint{0.000000in}{-0.048611in}}%
\pgfusepath{stroke,fill}%
}%
\begin{pgfscope}%
\pgfsys@transformshift{3.850000in}{3.960000in}%
\pgfsys@useobject{currentmarker}{}%
\end{pgfscope}%
\end{pgfscope}%
\begin{pgfscope}%
\pgfpathrectangle{\pgfqpoint{0.750000in}{3.960000in}}{\pgfqpoint{4.650000in}{3.080000in}}%
\pgfusepath{clip}%
\pgfsetrectcap%
\pgfsetroundjoin%
\pgfsetlinewidth{0.803000pt}%
\definecolor{currentstroke}{rgb}{0.690196,0.690196,0.690196}%
\pgfsetstrokecolor{currentstroke}%
\pgfsetdash{}{0pt}%
\pgfpathmoveto{\pgfqpoint{4.366667in}{3.960000in}}%
\pgfpathlineto{\pgfqpoint{4.366667in}{7.040000in}}%
\pgfusepath{stroke}%
\end{pgfscope}%
\begin{pgfscope}%
\pgfsetbuttcap%
\pgfsetroundjoin%
\definecolor{currentfill}{rgb}{0.000000,0.000000,0.000000}%
\pgfsetfillcolor{currentfill}%
\pgfsetlinewidth{0.803000pt}%
\definecolor{currentstroke}{rgb}{0.000000,0.000000,0.000000}%
\pgfsetstrokecolor{currentstroke}%
\pgfsetdash{}{0pt}%
\pgfsys@defobject{currentmarker}{\pgfqpoint{0.000000in}{-0.048611in}}{\pgfqpoint{0.000000in}{0.000000in}}{%
\pgfpathmoveto{\pgfqpoint{0.000000in}{0.000000in}}%
\pgfpathlineto{\pgfqpoint{0.000000in}{-0.048611in}}%
\pgfusepath{stroke,fill}%
}%
\begin{pgfscope}%
\pgfsys@transformshift{4.366667in}{3.960000in}%
\pgfsys@useobject{currentmarker}{}%
\end{pgfscope}%
\end{pgfscope}%
\begin{pgfscope}%
\pgfpathrectangle{\pgfqpoint{0.750000in}{3.960000in}}{\pgfqpoint{4.650000in}{3.080000in}}%
\pgfusepath{clip}%
\pgfsetrectcap%
\pgfsetroundjoin%
\pgfsetlinewidth{0.803000pt}%
\definecolor{currentstroke}{rgb}{0.690196,0.690196,0.690196}%
\pgfsetstrokecolor{currentstroke}%
\pgfsetdash{}{0pt}%
\pgfpathmoveto{\pgfqpoint{4.883333in}{3.960000in}}%
\pgfpathlineto{\pgfqpoint{4.883333in}{7.040000in}}%
\pgfusepath{stroke}%
\end{pgfscope}%
\begin{pgfscope}%
\pgfsetbuttcap%
\pgfsetroundjoin%
\definecolor{currentfill}{rgb}{0.000000,0.000000,0.000000}%
\pgfsetfillcolor{currentfill}%
\pgfsetlinewidth{0.803000pt}%
\definecolor{currentstroke}{rgb}{0.000000,0.000000,0.000000}%
\pgfsetstrokecolor{currentstroke}%
\pgfsetdash{}{0pt}%
\pgfsys@defobject{currentmarker}{\pgfqpoint{0.000000in}{-0.048611in}}{\pgfqpoint{0.000000in}{0.000000in}}{%
\pgfpathmoveto{\pgfqpoint{0.000000in}{0.000000in}}%
\pgfpathlineto{\pgfqpoint{0.000000in}{-0.048611in}}%
\pgfusepath{stroke,fill}%
}%
\begin{pgfscope}%
\pgfsys@transformshift{4.883333in}{3.960000in}%
\pgfsys@useobject{currentmarker}{}%
\end{pgfscope}%
\end{pgfscope}%
\begin{pgfscope}%
\pgfpathrectangle{\pgfqpoint{0.750000in}{3.960000in}}{\pgfqpoint{4.650000in}{3.080000in}}%
\pgfusepath{clip}%
\pgfsetrectcap%
\pgfsetroundjoin%
\pgfsetlinewidth{0.803000pt}%
\definecolor{currentstroke}{rgb}{0.690196,0.690196,0.690196}%
\pgfsetstrokecolor{currentstroke}%
\pgfsetdash{}{0pt}%
\pgfpathmoveto{\pgfqpoint{5.400000in}{3.960000in}}%
\pgfpathlineto{\pgfqpoint{5.400000in}{7.040000in}}%
\pgfusepath{stroke}%
\end{pgfscope}%
\begin{pgfscope}%
\pgfsetbuttcap%
\pgfsetroundjoin%
\definecolor{currentfill}{rgb}{0.000000,0.000000,0.000000}%
\pgfsetfillcolor{currentfill}%
\pgfsetlinewidth{0.803000pt}%
\definecolor{currentstroke}{rgb}{0.000000,0.000000,0.000000}%
\pgfsetstrokecolor{currentstroke}%
\pgfsetdash{}{0pt}%
\pgfsys@defobject{currentmarker}{\pgfqpoint{0.000000in}{-0.048611in}}{\pgfqpoint{0.000000in}{0.000000in}}{%
\pgfpathmoveto{\pgfqpoint{0.000000in}{0.000000in}}%
\pgfpathlineto{\pgfqpoint{0.000000in}{-0.048611in}}%
\pgfusepath{stroke,fill}%
}%
\begin{pgfscope}%
\pgfsys@transformshift{5.400000in}{3.960000in}%
\pgfsys@useobject{currentmarker}{}%
\end{pgfscope}%
\end{pgfscope}%
\begin{pgfscope}%
\pgfpathrectangle{\pgfqpoint{0.750000in}{3.960000in}}{\pgfqpoint{4.650000in}{3.080000in}}%
\pgfusepath{clip}%
\pgfsetrectcap%
\pgfsetroundjoin%
\pgfsetlinewidth{0.803000pt}%
\definecolor{currentstroke}{rgb}{0.690196,0.690196,0.690196}%
\pgfsetstrokecolor{currentstroke}%
\pgfsetdash{}{0pt}%
\pgfpathmoveto{\pgfqpoint{0.750000in}{3.960000in}}%
\pgfpathlineto{\pgfqpoint{5.400000in}{3.960000in}}%
\pgfusepath{stroke}%
\end{pgfscope}%
\begin{pgfscope}%
\pgfsetbuttcap%
\pgfsetroundjoin%
\definecolor{currentfill}{rgb}{0.000000,0.000000,0.000000}%
\pgfsetfillcolor{currentfill}%
\pgfsetlinewidth{0.803000pt}%
\definecolor{currentstroke}{rgb}{0.000000,0.000000,0.000000}%
\pgfsetstrokecolor{currentstroke}%
\pgfsetdash{}{0pt}%
\pgfsys@defobject{currentmarker}{\pgfqpoint{-0.048611in}{0.000000in}}{\pgfqpoint{-0.000000in}{0.000000in}}{%
\pgfpathmoveto{\pgfqpoint{-0.000000in}{0.000000in}}%
\pgfpathlineto{\pgfqpoint{-0.048611in}{0.000000in}}%
\pgfusepath{stroke,fill}%
}%
\begin{pgfscope}%
\pgfsys@transformshift{0.750000in}{3.960000in}%
\pgfsys@useobject{currentmarker}{}%
\end{pgfscope}%
\end{pgfscope}%
\begin{pgfscope}%
\definecolor{textcolor}{rgb}{0.000000,0.000000,0.000000}%
\pgfsetstrokecolor{textcolor}%
\pgfsetfillcolor{textcolor}%
\pgftext[x=0.475308in, y=3.908900in, left, base]{\color{textcolor}\rmfamily\fontsize{10.000000}{12.000000}\selectfont \(\displaystyle {0.0}\)}%
\end{pgfscope}%
\begin{pgfscope}%
\pgfpathrectangle{\pgfqpoint{0.750000in}{3.960000in}}{\pgfqpoint{4.650000in}{3.080000in}}%
\pgfusepath{clip}%
\pgfsetrectcap%
\pgfsetroundjoin%
\pgfsetlinewidth{0.803000pt}%
\definecolor{currentstroke}{rgb}{0.690196,0.690196,0.690196}%
\pgfsetstrokecolor{currentstroke}%
\pgfsetdash{}{0pt}%
\pgfpathmoveto{\pgfqpoint{0.750000in}{4.449140in}}%
\pgfpathlineto{\pgfqpoint{5.400000in}{4.449140in}}%
\pgfusepath{stroke}%
\end{pgfscope}%
\begin{pgfscope}%
\pgfsetbuttcap%
\pgfsetroundjoin%
\definecolor{currentfill}{rgb}{0.000000,0.000000,0.000000}%
\pgfsetfillcolor{currentfill}%
\pgfsetlinewidth{0.803000pt}%
\definecolor{currentstroke}{rgb}{0.000000,0.000000,0.000000}%
\pgfsetstrokecolor{currentstroke}%
\pgfsetdash{}{0pt}%
\pgfsys@defobject{currentmarker}{\pgfqpoint{-0.048611in}{0.000000in}}{\pgfqpoint{-0.000000in}{0.000000in}}{%
\pgfpathmoveto{\pgfqpoint{-0.000000in}{0.000000in}}%
\pgfpathlineto{\pgfqpoint{-0.048611in}{0.000000in}}%
\pgfusepath{stroke,fill}%
}%
\begin{pgfscope}%
\pgfsys@transformshift{0.750000in}{4.449140in}%
\pgfsys@useobject{currentmarker}{}%
\end{pgfscope}%
\end{pgfscope}%
\begin{pgfscope}%
\definecolor{textcolor}{rgb}{0.000000,0.000000,0.000000}%
\pgfsetstrokecolor{textcolor}%
\pgfsetfillcolor{textcolor}%
\pgftext[x=0.475308in, y=4.398040in, left, base]{\color{textcolor}\rmfamily\fontsize{10.000000}{12.000000}\selectfont \(\displaystyle {0.2}\)}%
\end{pgfscope}%
\begin{pgfscope}%
\pgfpathrectangle{\pgfqpoint{0.750000in}{3.960000in}}{\pgfqpoint{4.650000in}{3.080000in}}%
\pgfusepath{clip}%
\pgfsetrectcap%
\pgfsetroundjoin%
\pgfsetlinewidth{0.803000pt}%
\definecolor{currentstroke}{rgb}{0.690196,0.690196,0.690196}%
\pgfsetstrokecolor{currentstroke}%
\pgfsetdash{}{0pt}%
\pgfpathmoveto{\pgfqpoint{0.750000in}{4.938280in}}%
\pgfpathlineto{\pgfqpoint{5.400000in}{4.938280in}}%
\pgfusepath{stroke}%
\end{pgfscope}%
\begin{pgfscope}%
\pgfsetbuttcap%
\pgfsetroundjoin%
\definecolor{currentfill}{rgb}{0.000000,0.000000,0.000000}%
\pgfsetfillcolor{currentfill}%
\pgfsetlinewidth{0.803000pt}%
\definecolor{currentstroke}{rgb}{0.000000,0.000000,0.000000}%
\pgfsetstrokecolor{currentstroke}%
\pgfsetdash{}{0pt}%
\pgfsys@defobject{currentmarker}{\pgfqpoint{-0.048611in}{0.000000in}}{\pgfqpoint{-0.000000in}{0.000000in}}{%
\pgfpathmoveto{\pgfqpoint{-0.000000in}{0.000000in}}%
\pgfpathlineto{\pgfqpoint{-0.048611in}{0.000000in}}%
\pgfusepath{stroke,fill}%
}%
\begin{pgfscope}%
\pgfsys@transformshift{0.750000in}{4.938280in}%
\pgfsys@useobject{currentmarker}{}%
\end{pgfscope}%
\end{pgfscope}%
\begin{pgfscope}%
\definecolor{textcolor}{rgb}{0.000000,0.000000,0.000000}%
\pgfsetstrokecolor{textcolor}%
\pgfsetfillcolor{textcolor}%
\pgftext[x=0.475308in, y=4.887180in, left, base]{\color{textcolor}\rmfamily\fontsize{10.000000}{12.000000}\selectfont \(\displaystyle {0.4}\)}%
\end{pgfscope}%
\begin{pgfscope}%
\pgfpathrectangle{\pgfqpoint{0.750000in}{3.960000in}}{\pgfqpoint{4.650000in}{3.080000in}}%
\pgfusepath{clip}%
\pgfsetrectcap%
\pgfsetroundjoin%
\pgfsetlinewidth{0.803000pt}%
\definecolor{currentstroke}{rgb}{0.690196,0.690196,0.690196}%
\pgfsetstrokecolor{currentstroke}%
\pgfsetdash{}{0pt}%
\pgfpathmoveto{\pgfqpoint{0.750000in}{5.427421in}}%
\pgfpathlineto{\pgfqpoint{5.400000in}{5.427421in}}%
\pgfusepath{stroke}%
\end{pgfscope}%
\begin{pgfscope}%
\pgfsetbuttcap%
\pgfsetroundjoin%
\definecolor{currentfill}{rgb}{0.000000,0.000000,0.000000}%
\pgfsetfillcolor{currentfill}%
\pgfsetlinewidth{0.803000pt}%
\definecolor{currentstroke}{rgb}{0.000000,0.000000,0.000000}%
\pgfsetstrokecolor{currentstroke}%
\pgfsetdash{}{0pt}%
\pgfsys@defobject{currentmarker}{\pgfqpoint{-0.048611in}{0.000000in}}{\pgfqpoint{-0.000000in}{0.000000in}}{%
\pgfpathmoveto{\pgfqpoint{-0.000000in}{0.000000in}}%
\pgfpathlineto{\pgfqpoint{-0.048611in}{0.000000in}}%
\pgfusepath{stroke,fill}%
}%
\begin{pgfscope}%
\pgfsys@transformshift{0.750000in}{5.427421in}%
\pgfsys@useobject{currentmarker}{}%
\end{pgfscope}%
\end{pgfscope}%
\begin{pgfscope}%
\definecolor{textcolor}{rgb}{0.000000,0.000000,0.000000}%
\pgfsetstrokecolor{textcolor}%
\pgfsetfillcolor{textcolor}%
\pgftext[x=0.475308in, y=5.376321in, left, base]{\color{textcolor}\rmfamily\fontsize{10.000000}{12.000000}\selectfont \(\displaystyle {0.6}\)}%
\end{pgfscope}%
\begin{pgfscope}%
\pgfpathrectangle{\pgfqpoint{0.750000in}{3.960000in}}{\pgfqpoint{4.650000in}{3.080000in}}%
\pgfusepath{clip}%
\pgfsetrectcap%
\pgfsetroundjoin%
\pgfsetlinewidth{0.803000pt}%
\definecolor{currentstroke}{rgb}{0.690196,0.690196,0.690196}%
\pgfsetstrokecolor{currentstroke}%
\pgfsetdash{}{0pt}%
\pgfpathmoveto{\pgfqpoint{0.750000in}{5.916561in}}%
\pgfpathlineto{\pgfqpoint{5.400000in}{5.916561in}}%
\pgfusepath{stroke}%
\end{pgfscope}%
\begin{pgfscope}%
\pgfsetbuttcap%
\pgfsetroundjoin%
\definecolor{currentfill}{rgb}{0.000000,0.000000,0.000000}%
\pgfsetfillcolor{currentfill}%
\pgfsetlinewidth{0.803000pt}%
\definecolor{currentstroke}{rgb}{0.000000,0.000000,0.000000}%
\pgfsetstrokecolor{currentstroke}%
\pgfsetdash{}{0pt}%
\pgfsys@defobject{currentmarker}{\pgfqpoint{-0.048611in}{0.000000in}}{\pgfqpoint{-0.000000in}{0.000000in}}{%
\pgfpathmoveto{\pgfqpoint{-0.000000in}{0.000000in}}%
\pgfpathlineto{\pgfqpoint{-0.048611in}{0.000000in}}%
\pgfusepath{stroke,fill}%
}%
\begin{pgfscope}%
\pgfsys@transformshift{0.750000in}{5.916561in}%
\pgfsys@useobject{currentmarker}{}%
\end{pgfscope}%
\end{pgfscope}%
\begin{pgfscope}%
\definecolor{textcolor}{rgb}{0.000000,0.000000,0.000000}%
\pgfsetstrokecolor{textcolor}%
\pgfsetfillcolor{textcolor}%
\pgftext[x=0.475308in, y=5.865461in, left, base]{\color{textcolor}\rmfamily\fontsize{10.000000}{12.000000}\selectfont \(\displaystyle {0.8}\)}%
\end{pgfscope}%
\begin{pgfscope}%
\pgfpathrectangle{\pgfqpoint{0.750000in}{3.960000in}}{\pgfqpoint{4.650000in}{3.080000in}}%
\pgfusepath{clip}%
\pgfsetrectcap%
\pgfsetroundjoin%
\pgfsetlinewidth{0.803000pt}%
\definecolor{currentstroke}{rgb}{0.690196,0.690196,0.690196}%
\pgfsetstrokecolor{currentstroke}%
\pgfsetdash{}{0pt}%
\pgfpathmoveto{\pgfqpoint{0.750000in}{6.405701in}}%
\pgfpathlineto{\pgfqpoint{5.400000in}{6.405701in}}%
\pgfusepath{stroke}%
\end{pgfscope}%
\begin{pgfscope}%
\pgfsetbuttcap%
\pgfsetroundjoin%
\definecolor{currentfill}{rgb}{0.000000,0.000000,0.000000}%
\pgfsetfillcolor{currentfill}%
\pgfsetlinewidth{0.803000pt}%
\definecolor{currentstroke}{rgb}{0.000000,0.000000,0.000000}%
\pgfsetstrokecolor{currentstroke}%
\pgfsetdash{}{0pt}%
\pgfsys@defobject{currentmarker}{\pgfqpoint{-0.048611in}{0.000000in}}{\pgfqpoint{-0.000000in}{0.000000in}}{%
\pgfpathmoveto{\pgfqpoint{-0.000000in}{0.000000in}}%
\pgfpathlineto{\pgfqpoint{-0.048611in}{0.000000in}}%
\pgfusepath{stroke,fill}%
}%
\begin{pgfscope}%
\pgfsys@transformshift{0.750000in}{6.405701in}%
\pgfsys@useobject{currentmarker}{}%
\end{pgfscope}%
\end{pgfscope}%
\begin{pgfscope}%
\definecolor{textcolor}{rgb}{0.000000,0.000000,0.000000}%
\pgfsetstrokecolor{textcolor}%
\pgfsetfillcolor{textcolor}%
\pgftext[x=0.475308in, y=6.354601in, left, base]{\color{textcolor}\rmfamily\fontsize{10.000000}{12.000000}\selectfont \(\displaystyle {1.0}\)}%
\end{pgfscope}%
\begin{pgfscope}%
\pgfpathrectangle{\pgfqpoint{0.750000in}{3.960000in}}{\pgfqpoint{4.650000in}{3.080000in}}%
\pgfusepath{clip}%
\pgfsetrectcap%
\pgfsetroundjoin%
\pgfsetlinewidth{0.803000pt}%
\definecolor{currentstroke}{rgb}{0.690196,0.690196,0.690196}%
\pgfsetstrokecolor{currentstroke}%
\pgfsetdash{}{0pt}%
\pgfpathmoveto{\pgfqpoint{0.750000in}{6.894841in}}%
\pgfpathlineto{\pgfqpoint{5.400000in}{6.894841in}}%
\pgfusepath{stroke}%
\end{pgfscope}%
\begin{pgfscope}%
\pgfsetbuttcap%
\pgfsetroundjoin%
\definecolor{currentfill}{rgb}{0.000000,0.000000,0.000000}%
\pgfsetfillcolor{currentfill}%
\pgfsetlinewidth{0.803000pt}%
\definecolor{currentstroke}{rgb}{0.000000,0.000000,0.000000}%
\pgfsetstrokecolor{currentstroke}%
\pgfsetdash{}{0pt}%
\pgfsys@defobject{currentmarker}{\pgfqpoint{-0.048611in}{0.000000in}}{\pgfqpoint{-0.000000in}{0.000000in}}{%
\pgfpathmoveto{\pgfqpoint{-0.000000in}{0.000000in}}%
\pgfpathlineto{\pgfqpoint{-0.048611in}{0.000000in}}%
\pgfusepath{stroke,fill}%
}%
\begin{pgfscope}%
\pgfsys@transformshift{0.750000in}{6.894841in}%
\pgfsys@useobject{currentmarker}{}%
\end{pgfscope}%
\end{pgfscope}%
\begin{pgfscope}%
\definecolor{textcolor}{rgb}{0.000000,0.000000,0.000000}%
\pgfsetstrokecolor{textcolor}%
\pgfsetfillcolor{textcolor}%
\pgftext[x=0.475308in, y=6.843741in, left, base]{\color{textcolor}\rmfamily\fontsize{10.000000}{12.000000}\selectfont \(\displaystyle {1.2}\)}%
\end{pgfscope}%
\begin{pgfscope}%
\definecolor{textcolor}{rgb}{0.000000,0.000000,0.000000}%
\pgfsetstrokecolor{textcolor}%
\pgfsetfillcolor{textcolor}%
\pgftext[x=0.419752in,y=5.500000in,,bottom,rotate=90.000000]{\color{textcolor}\rmfamily\fontsize{10.000000}{12.000000}\selectfont power in pu}%
\end{pgfscope}%
\begin{pgfscope}%
\pgfpathrectangle{\pgfqpoint{0.750000in}{3.960000in}}{\pgfqpoint{4.650000in}{3.080000in}}%
\pgfusepath{clip}%
\pgfsetrectcap%
\pgfsetroundjoin%
\pgfsetlinewidth{2.007500pt}%
\definecolor{currentstroke}{rgb}{0.121569,0.466667,0.705882}%
\pgfsetstrokecolor{currentstroke}%
\pgfsetdash{}{0pt}%
\pgfpathmoveto{\pgfqpoint{0.750000in}{3.960000in}}%
\pgfpathlineto{\pgfqpoint{0.844898in}{4.148036in}}%
\pgfpathlineto{\pgfqpoint{0.939796in}{4.335299in}}%
\pgfpathlineto{\pgfqpoint{1.034694in}{4.521020in}}%
\pgfpathlineto{\pgfqpoint{1.129592in}{4.704436in}}%
\pgfpathlineto{\pgfqpoint{1.224490in}{4.884793in}}%
\pgfpathlineto{\pgfqpoint{1.319388in}{5.061349in}}%
\pgfpathlineto{\pgfqpoint{1.414286in}{5.233380in}}%
\pgfpathlineto{\pgfqpoint{1.509184in}{5.400178in}}%
\pgfpathlineto{\pgfqpoint{1.604082in}{5.561058in}}%
\pgfpathlineto{\pgfqpoint{1.698980in}{5.715359in}}%
\pgfpathlineto{\pgfqpoint{1.793878in}{5.862447in}}%
\pgfpathlineto{\pgfqpoint{1.888776in}{6.001718in}}%
\pgfpathlineto{\pgfqpoint{1.983673in}{6.132598in}}%
\pgfpathlineto{\pgfqpoint{2.078571in}{6.254551in}}%
\pgfpathlineto{\pgfqpoint{2.173469in}{6.367075in}}%
\pgfpathlineto{\pgfqpoint{2.268367in}{6.469708in}}%
\pgfpathlineto{\pgfqpoint{2.363265in}{6.562028in}}%
\pgfpathlineto{\pgfqpoint{2.458163in}{6.643656in}}%
\pgfpathlineto{\pgfqpoint{2.553061in}{6.714256in}}%
\pgfpathlineto{\pgfqpoint{2.647959in}{6.773538in}}%
\pgfpathlineto{\pgfqpoint{2.742857in}{6.821259in}}%
\pgfpathlineto{\pgfqpoint{2.837755in}{6.857222in}}%
\pgfpathlineto{\pgfqpoint{2.932653in}{6.881280in}}%
\pgfpathlineto{\pgfqpoint{3.027551in}{6.893333in}}%
\pgfpathlineto{\pgfqpoint{3.122449in}{6.893333in}}%
\pgfpathlineto{\pgfqpoint{3.217347in}{6.881280in}}%
\pgfpathlineto{\pgfqpoint{3.312245in}{6.857222in}}%
\pgfpathlineto{\pgfqpoint{3.407143in}{6.821259in}}%
\pgfpathlineto{\pgfqpoint{3.502041in}{6.773538in}}%
\pgfpathlineto{\pgfqpoint{3.596939in}{6.714256in}}%
\pgfpathlineto{\pgfqpoint{3.691837in}{6.643656in}}%
\pgfpathlineto{\pgfqpoint{3.786735in}{6.562028in}}%
\pgfpathlineto{\pgfqpoint{3.881633in}{6.469708in}}%
\pgfpathlineto{\pgfqpoint{3.976531in}{6.367075in}}%
\pgfpathlineto{\pgfqpoint{4.071429in}{6.254551in}}%
\pgfpathlineto{\pgfqpoint{4.166327in}{6.132598in}}%
\pgfpathlineto{\pgfqpoint{4.261224in}{6.001718in}}%
\pgfpathlineto{\pgfqpoint{4.356122in}{5.862447in}}%
\pgfpathlineto{\pgfqpoint{4.451020in}{5.715359in}}%
\pgfpathlineto{\pgfqpoint{4.545918in}{5.561058in}}%
\pgfpathlineto{\pgfqpoint{4.640816in}{5.400178in}}%
\pgfpathlineto{\pgfqpoint{4.735714in}{5.233380in}}%
\pgfpathlineto{\pgfqpoint{4.830612in}{5.061349in}}%
\pgfpathlineto{\pgfqpoint{4.925510in}{4.884793in}}%
\pgfpathlineto{\pgfqpoint{5.020408in}{4.704436in}}%
\pgfpathlineto{\pgfqpoint{5.115306in}{4.521020in}}%
\pgfpathlineto{\pgfqpoint{5.210204in}{4.335299in}}%
\pgfpathlineto{\pgfqpoint{5.305102in}{4.148036in}}%
\pgfpathlineto{\pgfqpoint{5.400000in}{3.960000in}}%
\pgfusepath{stroke}%
\end{pgfscope}%
\begin{pgfscope}%
\pgfpathrectangle{\pgfqpoint{0.750000in}{3.960000in}}{\pgfqpoint{4.650000in}{3.080000in}}%
\pgfusepath{clip}%
\pgfsetrectcap%
\pgfsetroundjoin%
\pgfsetlinewidth{2.007500pt}%
\definecolor{currentstroke}{rgb}{1.000000,0.498039,0.054902}%
\pgfsetstrokecolor{currentstroke}%
\pgfsetdash{}{0pt}%
\pgfpathmoveto{\pgfqpoint{0.750000in}{6.161131in}}%
\pgfpathlineto{\pgfqpoint{0.844898in}{6.161131in}}%
\pgfpathlineto{\pgfqpoint{0.939796in}{6.161131in}}%
\pgfpathlineto{\pgfqpoint{1.034694in}{6.161131in}}%
\pgfpathlineto{\pgfqpoint{1.129592in}{6.161131in}}%
\pgfpathlineto{\pgfqpoint{1.224490in}{6.161131in}}%
\pgfpathlineto{\pgfqpoint{1.319388in}{6.161131in}}%
\pgfpathlineto{\pgfqpoint{1.414286in}{6.161131in}}%
\pgfpathlineto{\pgfqpoint{1.509184in}{6.161131in}}%
\pgfpathlineto{\pgfqpoint{1.604082in}{6.161131in}}%
\pgfpathlineto{\pgfqpoint{1.698980in}{6.161131in}}%
\pgfpathlineto{\pgfqpoint{1.793878in}{6.161131in}}%
\pgfpathlineto{\pgfqpoint{1.888776in}{6.161131in}}%
\pgfpathlineto{\pgfqpoint{1.983673in}{6.161131in}}%
\pgfpathlineto{\pgfqpoint{2.078571in}{6.161131in}}%
\pgfpathlineto{\pgfqpoint{2.173469in}{6.161131in}}%
\pgfpathlineto{\pgfqpoint{2.268367in}{6.161131in}}%
\pgfpathlineto{\pgfqpoint{2.363265in}{6.161131in}}%
\pgfpathlineto{\pgfqpoint{2.458163in}{6.161131in}}%
\pgfpathlineto{\pgfqpoint{2.553061in}{6.161131in}}%
\pgfpathlineto{\pgfqpoint{2.647959in}{6.161131in}}%
\pgfpathlineto{\pgfqpoint{2.742857in}{6.161131in}}%
\pgfpathlineto{\pgfqpoint{2.837755in}{6.161131in}}%
\pgfpathlineto{\pgfqpoint{2.932653in}{6.161131in}}%
\pgfpathlineto{\pgfqpoint{3.027551in}{6.161131in}}%
\pgfpathlineto{\pgfqpoint{3.122449in}{6.161131in}}%
\pgfpathlineto{\pgfqpoint{3.217347in}{6.161131in}}%
\pgfpathlineto{\pgfqpoint{3.312245in}{6.161131in}}%
\pgfpathlineto{\pgfqpoint{3.407143in}{6.161131in}}%
\pgfpathlineto{\pgfqpoint{3.502041in}{6.161131in}}%
\pgfpathlineto{\pgfqpoint{3.596939in}{6.161131in}}%
\pgfpathlineto{\pgfqpoint{3.691837in}{6.161131in}}%
\pgfpathlineto{\pgfqpoint{3.786735in}{6.161131in}}%
\pgfpathlineto{\pgfqpoint{3.881633in}{6.161131in}}%
\pgfpathlineto{\pgfqpoint{3.976531in}{6.161131in}}%
\pgfpathlineto{\pgfqpoint{4.071429in}{6.161131in}}%
\pgfpathlineto{\pgfqpoint{4.166327in}{6.161131in}}%
\pgfpathlineto{\pgfqpoint{4.261224in}{6.161131in}}%
\pgfpathlineto{\pgfqpoint{4.356122in}{6.161131in}}%
\pgfpathlineto{\pgfqpoint{4.451020in}{6.161131in}}%
\pgfpathlineto{\pgfqpoint{4.545918in}{6.161131in}}%
\pgfpathlineto{\pgfqpoint{4.640816in}{6.161131in}}%
\pgfpathlineto{\pgfqpoint{4.735714in}{6.161131in}}%
\pgfpathlineto{\pgfqpoint{4.830612in}{6.161131in}}%
\pgfpathlineto{\pgfqpoint{4.925510in}{6.161131in}}%
\pgfpathlineto{\pgfqpoint{5.020408in}{6.161131in}}%
\pgfpathlineto{\pgfqpoint{5.115306in}{6.161131in}}%
\pgfpathlineto{\pgfqpoint{5.210204in}{6.161131in}}%
\pgfpathlineto{\pgfqpoint{5.305102in}{6.161131in}}%
\pgfpathlineto{\pgfqpoint{5.400000in}{6.161131in}}%
\pgfusepath{stroke}%
\end{pgfscope}%
\begin{pgfscope}%
\pgfsetrectcap%
\pgfsetmiterjoin%
\pgfsetlinewidth{0.803000pt}%
\definecolor{currentstroke}{rgb}{0.000000,0.000000,0.000000}%
\pgfsetstrokecolor{currentstroke}%
\pgfsetdash{}{0pt}%
\pgfpathmoveto{\pgfqpoint{0.750000in}{3.960000in}}%
\pgfpathlineto{\pgfqpoint{0.750000in}{7.040000in}}%
\pgfusepath{stroke}%
\end{pgfscope}%
\begin{pgfscope}%
\pgfsetrectcap%
\pgfsetmiterjoin%
\pgfsetlinewidth{0.803000pt}%
\definecolor{currentstroke}{rgb}{0.000000,0.000000,0.000000}%
\pgfsetstrokecolor{currentstroke}%
\pgfsetdash{}{0pt}%
\pgfpathmoveto{\pgfqpoint{5.400000in}{3.960000in}}%
\pgfpathlineto{\pgfqpoint{5.400000in}{7.040000in}}%
\pgfusepath{stroke}%
\end{pgfscope}%
\begin{pgfscope}%
\pgfsetrectcap%
\pgfsetmiterjoin%
\pgfsetlinewidth{0.803000pt}%
\definecolor{currentstroke}{rgb}{0.000000,0.000000,0.000000}%
\pgfsetstrokecolor{currentstroke}%
\pgfsetdash{}{0pt}%
\pgfpathmoveto{\pgfqpoint{0.750000in}{3.960000in}}%
\pgfpathlineto{\pgfqpoint{5.400000in}{3.960000in}}%
\pgfusepath{stroke}%
\end{pgfscope}%
\begin{pgfscope}%
\pgfsetrectcap%
\pgfsetmiterjoin%
\pgfsetlinewidth{0.803000pt}%
\definecolor{currentstroke}{rgb}{0.000000,0.000000,0.000000}%
\pgfsetstrokecolor{currentstroke}%
\pgfsetdash{}{0pt}%
\pgfpathmoveto{\pgfqpoint{0.750000in}{7.040000in}}%
\pgfpathlineto{\pgfqpoint{5.400000in}{7.040000in}}%
\pgfusepath{stroke}%
\end{pgfscope}%
\begin{pgfscope}%
\pgfsetbuttcap%
\pgfsetmiterjoin%
\definecolor{currentfill}{rgb}{1.000000,1.000000,1.000000}%
\pgfsetfillcolor{currentfill}%
\pgfsetfillopacity{0.800000}%
\pgfsetlinewidth{1.003750pt}%
\definecolor{currentstroke}{rgb}{0.800000,0.800000,0.800000}%
\pgfsetstrokecolor{currentstroke}%
\pgfsetstrokeopacity{0.800000}%
\pgfsetdash{}{0pt}%
\pgfpathmoveto{\pgfqpoint{2.342004in}{4.029444in}}%
\pgfpathlineto{\pgfqpoint{3.807996in}{4.029444in}}%
\pgfpathquadraticcurveto{\pgfqpoint{3.835774in}{4.029444in}}{\pgfqpoint{3.835774in}{4.057222in}}%
\pgfpathlineto{\pgfqpoint{3.835774in}{4.448267in}}%
\pgfpathquadraticcurveto{\pgfqpoint{3.835774in}{4.476045in}}{\pgfqpoint{3.807996in}{4.476045in}}%
\pgfpathlineto{\pgfqpoint{2.342004in}{4.476045in}}%
\pgfpathquadraticcurveto{\pgfqpoint{2.314226in}{4.476045in}}{\pgfqpoint{2.314226in}{4.448267in}}%
\pgfpathlineto{\pgfqpoint{2.314226in}{4.057222in}}%
\pgfpathquadraticcurveto{\pgfqpoint{2.314226in}{4.029444in}}{\pgfqpoint{2.342004in}{4.029444in}}%
\pgfpathlineto{\pgfqpoint{2.342004in}{4.029444in}}%
\pgfpathclose%
\pgfusepath{stroke,fill}%
\end{pgfscope}%
\begin{pgfscope}%
\pgfsetrectcap%
\pgfsetroundjoin%
\pgfsetlinewidth{2.007500pt}%
\definecolor{currentstroke}{rgb}{0.121569,0.466667,0.705882}%
\pgfsetstrokecolor{currentstroke}%
\pgfsetdash{}{0pt}%
\pgfpathmoveto{\pgfqpoint{2.369781in}{4.365748in}}%
\pgfpathlineto{\pgfqpoint{2.508670in}{4.365748in}}%
\pgfpathlineto{\pgfqpoint{2.647559in}{4.365748in}}%
\pgfusepath{stroke}%
\end{pgfscope}%
\begin{pgfscope}%
\definecolor{textcolor}{rgb}{0.000000,0.000000,0.000000}%
\pgfsetstrokecolor{textcolor}%
\pgfsetfillcolor{textcolor}%
\pgftext[x=2.758670in,y=4.317137in,left,base]{\color{textcolor}\rmfamily\fontsize{10.000000}{12.000000}\selectfont \(\displaystyle P_\mathrm{e}\) pre-fault}%
\end{pgfscope}%
\begin{pgfscope}%
\pgfsetrectcap%
\pgfsetroundjoin%
\pgfsetlinewidth{2.007500pt}%
\definecolor{currentstroke}{rgb}{1.000000,0.498039,0.054902}%
\pgfsetstrokecolor{currentstroke}%
\pgfsetdash{}{0pt}%
\pgfpathmoveto{\pgfqpoint{2.369781in}{4.163857in}}%
\pgfpathlineto{\pgfqpoint{2.508670in}{4.163857in}}%
\pgfpathlineto{\pgfqpoint{2.647559in}{4.163857in}}%
\pgfusepath{stroke}%
\end{pgfscope}%
\begin{pgfscope}%
\definecolor{textcolor}{rgb}{0.000000,0.000000,0.000000}%
\pgfsetstrokecolor{textcolor}%
\pgfsetfillcolor{textcolor}%
\pgftext[x=2.758670in,y=4.115246in,left,base]{\color{textcolor}\rmfamily\fontsize{10.000000}{12.000000}\selectfont \(\displaystyle P_\mathrm{T}\) of the turbine}%
\end{pgfscope}%
\begin{pgfscope}%
\pgfsetbuttcap%
\pgfsetmiterjoin%
\definecolor{currentfill}{rgb}{1.000000,1.000000,1.000000}%
\pgfsetfillcolor{currentfill}%
\pgfsetlinewidth{0.000000pt}%
\definecolor{currentstroke}{rgb}{0.000000,0.000000,0.000000}%
\pgfsetstrokecolor{currentstroke}%
\pgfsetstrokeopacity{0.000000}%
\pgfsetdash{}{0pt}%
\pgfpathmoveto{\pgfqpoint{0.750000in}{0.880000in}}%
\pgfpathlineto{\pgfqpoint{5.400000in}{0.880000in}}%
\pgfpathlineto{\pgfqpoint{5.400000in}{3.960000in}}%
\pgfpathlineto{\pgfqpoint{0.750000in}{3.960000in}}%
\pgfpathlineto{\pgfqpoint{0.750000in}{0.880000in}}%
\pgfpathclose%
\pgfusepath{fill}%
\end{pgfscope}%
\begin{pgfscope}%
\pgfpathrectangle{\pgfqpoint{0.750000in}{0.880000in}}{\pgfqpoint{4.650000in}{3.080000in}}%
\pgfusepath{clip}%
\pgfsetrectcap%
\pgfsetroundjoin%
\pgfsetlinewidth{0.803000pt}%
\definecolor{currentstroke}{rgb}{0.690196,0.690196,0.690196}%
\pgfsetstrokecolor{currentstroke}%
\pgfsetdash{}{0pt}%
\pgfpathmoveto{\pgfqpoint{0.750000in}{0.880000in}}%
\pgfpathlineto{\pgfqpoint{0.750000in}{3.960000in}}%
\pgfusepath{stroke}%
\end{pgfscope}%
\begin{pgfscope}%
\pgfsetbuttcap%
\pgfsetroundjoin%
\definecolor{currentfill}{rgb}{0.000000,0.000000,0.000000}%
\pgfsetfillcolor{currentfill}%
\pgfsetlinewidth{0.803000pt}%
\definecolor{currentstroke}{rgb}{0.000000,0.000000,0.000000}%
\pgfsetstrokecolor{currentstroke}%
\pgfsetdash{}{0pt}%
\pgfsys@defobject{currentmarker}{\pgfqpoint{0.000000in}{-0.048611in}}{\pgfqpoint{0.000000in}{0.000000in}}{%
\pgfpathmoveto{\pgfqpoint{0.000000in}{0.000000in}}%
\pgfpathlineto{\pgfqpoint{0.000000in}{-0.048611in}}%
\pgfusepath{stroke,fill}%
}%
\begin{pgfscope}%
\pgfsys@transformshift{0.750000in}{0.880000in}%
\pgfsys@useobject{currentmarker}{}%
\end{pgfscope}%
\end{pgfscope}%
\begin{pgfscope}%
\definecolor{textcolor}{rgb}{0.000000,0.000000,0.000000}%
\pgfsetstrokecolor{textcolor}%
\pgfsetfillcolor{textcolor}%
\pgftext[x=0.750000in,y=0.782778in,,top]{\color{textcolor}\rmfamily\fontsize{10.000000}{12.000000}\selectfont \(\displaystyle {0}\)}%
\end{pgfscope}%
\begin{pgfscope}%
\pgfpathrectangle{\pgfqpoint{0.750000in}{0.880000in}}{\pgfqpoint{4.650000in}{3.080000in}}%
\pgfusepath{clip}%
\pgfsetrectcap%
\pgfsetroundjoin%
\pgfsetlinewidth{0.803000pt}%
\definecolor{currentstroke}{rgb}{0.690196,0.690196,0.690196}%
\pgfsetstrokecolor{currentstroke}%
\pgfsetdash{}{0pt}%
\pgfpathmoveto{\pgfqpoint{1.266667in}{0.880000in}}%
\pgfpathlineto{\pgfqpoint{1.266667in}{3.960000in}}%
\pgfusepath{stroke}%
\end{pgfscope}%
\begin{pgfscope}%
\pgfsetbuttcap%
\pgfsetroundjoin%
\definecolor{currentfill}{rgb}{0.000000,0.000000,0.000000}%
\pgfsetfillcolor{currentfill}%
\pgfsetlinewidth{0.803000pt}%
\definecolor{currentstroke}{rgb}{0.000000,0.000000,0.000000}%
\pgfsetstrokecolor{currentstroke}%
\pgfsetdash{}{0pt}%
\pgfsys@defobject{currentmarker}{\pgfqpoint{0.000000in}{-0.048611in}}{\pgfqpoint{0.000000in}{0.000000in}}{%
\pgfpathmoveto{\pgfqpoint{0.000000in}{0.000000in}}%
\pgfpathlineto{\pgfqpoint{0.000000in}{-0.048611in}}%
\pgfusepath{stroke,fill}%
}%
\begin{pgfscope}%
\pgfsys@transformshift{1.266667in}{0.880000in}%
\pgfsys@useobject{currentmarker}{}%
\end{pgfscope}%
\end{pgfscope}%
\begin{pgfscope}%
\definecolor{textcolor}{rgb}{0.000000,0.000000,0.000000}%
\pgfsetstrokecolor{textcolor}%
\pgfsetfillcolor{textcolor}%
\pgftext[x=1.266667in,y=0.782778in,,top]{\color{textcolor}\rmfamily\fontsize{10.000000}{12.000000}\selectfont \(\displaystyle {20}\)}%
\end{pgfscope}%
\begin{pgfscope}%
\pgfpathrectangle{\pgfqpoint{0.750000in}{0.880000in}}{\pgfqpoint{4.650000in}{3.080000in}}%
\pgfusepath{clip}%
\pgfsetrectcap%
\pgfsetroundjoin%
\pgfsetlinewidth{0.803000pt}%
\definecolor{currentstroke}{rgb}{0.690196,0.690196,0.690196}%
\pgfsetstrokecolor{currentstroke}%
\pgfsetdash{}{0pt}%
\pgfpathmoveto{\pgfqpoint{1.783333in}{0.880000in}}%
\pgfpathlineto{\pgfqpoint{1.783333in}{3.960000in}}%
\pgfusepath{stroke}%
\end{pgfscope}%
\begin{pgfscope}%
\pgfsetbuttcap%
\pgfsetroundjoin%
\definecolor{currentfill}{rgb}{0.000000,0.000000,0.000000}%
\pgfsetfillcolor{currentfill}%
\pgfsetlinewidth{0.803000pt}%
\definecolor{currentstroke}{rgb}{0.000000,0.000000,0.000000}%
\pgfsetstrokecolor{currentstroke}%
\pgfsetdash{}{0pt}%
\pgfsys@defobject{currentmarker}{\pgfqpoint{0.000000in}{-0.048611in}}{\pgfqpoint{0.000000in}{0.000000in}}{%
\pgfpathmoveto{\pgfqpoint{0.000000in}{0.000000in}}%
\pgfpathlineto{\pgfqpoint{0.000000in}{-0.048611in}}%
\pgfusepath{stroke,fill}%
}%
\begin{pgfscope}%
\pgfsys@transformshift{1.783333in}{0.880000in}%
\pgfsys@useobject{currentmarker}{}%
\end{pgfscope}%
\end{pgfscope}%
\begin{pgfscope}%
\definecolor{textcolor}{rgb}{0.000000,0.000000,0.000000}%
\pgfsetstrokecolor{textcolor}%
\pgfsetfillcolor{textcolor}%
\pgftext[x=1.783333in,y=0.782778in,,top]{\color{textcolor}\rmfamily\fontsize{10.000000}{12.000000}\selectfont \(\displaystyle {40}\)}%
\end{pgfscope}%
\begin{pgfscope}%
\pgfpathrectangle{\pgfqpoint{0.750000in}{0.880000in}}{\pgfqpoint{4.650000in}{3.080000in}}%
\pgfusepath{clip}%
\pgfsetrectcap%
\pgfsetroundjoin%
\pgfsetlinewidth{0.803000pt}%
\definecolor{currentstroke}{rgb}{0.690196,0.690196,0.690196}%
\pgfsetstrokecolor{currentstroke}%
\pgfsetdash{}{0pt}%
\pgfpathmoveto{\pgfqpoint{2.300000in}{0.880000in}}%
\pgfpathlineto{\pgfqpoint{2.300000in}{3.960000in}}%
\pgfusepath{stroke}%
\end{pgfscope}%
\begin{pgfscope}%
\pgfsetbuttcap%
\pgfsetroundjoin%
\definecolor{currentfill}{rgb}{0.000000,0.000000,0.000000}%
\pgfsetfillcolor{currentfill}%
\pgfsetlinewidth{0.803000pt}%
\definecolor{currentstroke}{rgb}{0.000000,0.000000,0.000000}%
\pgfsetstrokecolor{currentstroke}%
\pgfsetdash{}{0pt}%
\pgfsys@defobject{currentmarker}{\pgfqpoint{0.000000in}{-0.048611in}}{\pgfqpoint{0.000000in}{0.000000in}}{%
\pgfpathmoveto{\pgfqpoint{0.000000in}{0.000000in}}%
\pgfpathlineto{\pgfqpoint{0.000000in}{-0.048611in}}%
\pgfusepath{stroke,fill}%
}%
\begin{pgfscope}%
\pgfsys@transformshift{2.300000in}{0.880000in}%
\pgfsys@useobject{currentmarker}{}%
\end{pgfscope}%
\end{pgfscope}%
\begin{pgfscope}%
\definecolor{textcolor}{rgb}{0.000000,0.000000,0.000000}%
\pgfsetstrokecolor{textcolor}%
\pgfsetfillcolor{textcolor}%
\pgftext[x=2.300000in,y=0.782778in,,top]{\color{textcolor}\rmfamily\fontsize{10.000000}{12.000000}\selectfont \(\displaystyle {60}\)}%
\end{pgfscope}%
\begin{pgfscope}%
\pgfpathrectangle{\pgfqpoint{0.750000in}{0.880000in}}{\pgfqpoint{4.650000in}{3.080000in}}%
\pgfusepath{clip}%
\pgfsetrectcap%
\pgfsetroundjoin%
\pgfsetlinewidth{0.803000pt}%
\definecolor{currentstroke}{rgb}{0.690196,0.690196,0.690196}%
\pgfsetstrokecolor{currentstroke}%
\pgfsetdash{}{0pt}%
\pgfpathmoveto{\pgfqpoint{2.816667in}{0.880000in}}%
\pgfpathlineto{\pgfqpoint{2.816667in}{3.960000in}}%
\pgfusepath{stroke}%
\end{pgfscope}%
\begin{pgfscope}%
\pgfsetbuttcap%
\pgfsetroundjoin%
\definecolor{currentfill}{rgb}{0.000000,0.000000,0.000000}%
\pgfsetfillcolor{currentfill}%
\pgfsetlinewidth{0.803000pt}%
\definecolor{currentstroke}{rgb}{0.000000,0.000000,0.000000}%
\pgfsetstrokecolor{currentstroke}%
\pgfsetdash{}{0pt}%
\pgfsys@defobject{currentmarker}{\pgfqpoint{0.000000in}{-0.048611in}}{\pgfqpoint{0.000000in}{0.000000in}}{%
\pgfpathmoveto{\pgfqpoint{0.000000in}{0.000000in}}%
\pgfpathlineto{\pgfqpoint{0.000000in}{-0.048611in}}%
\pgfusepath{stroke,fill}%
}%
\begin{pgfscope}%
\pgfsys@transformshift{2.816667in}{0.880000in}%
\pgfsys@useobject{currentmarker}{}%
\end{pgfscope}%
\end{pgfscope}%
\begin{pgfscope}%
\definecolor{textcolor}{rgb}{0.000000,0.000000,0.000000}%
\pgfsetstrokecolor{textcolor}%
\pgfsetfillcolor{textcolor}%
\pgftext[x=2.816667in,y=0.782778in,,top]{\color{textcolor}\rmfamily\fontsize{10.000000}{12.000000}\selectfont \(\displaystyle {80}\)}%
\end{pgfscope}%
\begin{pgfscope}%
\pgfpathrectangle{\pgfqpoint{0.750000in}{0.880000in}}{\pgfqpoint{4.650000in}{3.080000in}}%
\pgfusepath{clip}%
\pgfsetrectcap%
\pgfsetroundjoin%
\pgfsetlinewidth{0.803000pt}%
\definecolor{currentstroke}{rgb}{0.690196,0.690196,0.690196}%
\pgfsetstrokecolor{currentstroke}%
\pgfsetdash{}{0pt}%
\pgfpathmoveto{\pgfqpoint{3.333333in}{0.880000in}}%
\pgfpathlineto{\pgfqpoint{3.333333in}{3.960000in}}%
\pgfusepath{stroke}%
\end{pgfscope}%
\begin{pgfscope}%
\pgfsetbuttcap%
\pgfsetroundjoin%
\definecolor{currentfill}{rgb}{0.000000,0.000000,0.000000}%
\pgfsetfillcolor{currentfill}%
\pgfsetlinewidth{0.803000pt}%
\definecolor{currentstroke}{rgb}{0.000000,0.000000,0.000000}%
\pgfsetstrokecolor{currentstroke}%
\pgfsetdash{}{0pt}%
\pgfsys@defobject{currentmarker}{\pgfqpoint{0.000000in}{-0.048611in}}{\pgfqpoint{0.000000in}{0.000000in}}{%
\pgfpathmoveto{\pgfqpoint{0.000000in}{0.000000in}}%
\pgfpathlineto{\pgfqpoint{0.000000in}{-0.048611in}}%
\pgfusepath{stroke,fill}%
}%
\begin{pgfscope}%
\pgfsys@transformshift{3.333333in}{0.880000in}%
\pgfsys@useobject{currentmarker}{}%
\end{pgfscope}%
\end{pgfscope}%
\begin{pgfscope}%
\definecolor{textcolor}{rgb}{0.000000,0.000000,0.000000}%
\pgfsetstrokecolor{textcolor}%
\pgfsetfillcolor{textcolor}%
\pgftext[x=3.333333in,y=0.782778in,,top]{\color{textcolor}\rmfamily\fontsize{10.000000}{12.000000}\selectfont \(\displaystyle {100}\)}%
\end{pgfscope}%
\begin{pgfscope}%
\pgfpathrectangle{\pgfqpoint{0.750000in}{0.880000in}}{\pgfqpoint{4.650000in}{3.080000in}}%
\pgfusepath{clip}%
\pgfsetrectcap%
\pgfsetroundjoin%
\pgfsetlinewidth{0.803000pt}%
\definecolor{currentstroke}{rgb}{0.690196,0.690196,0.690196}%
\pgfsetstrokecolor{currentstroke}%
\pgfsetdash{}{0pt}%
\pgfpathmoveto{\pgfqpoint{3.850000in}{0.880000in}}%
\pgfpathlineto{\pgfqpoint{3.850000in}{3.960000in}}%
\pgfusepath{stroke}%
\end{pgfscope}%
\begin{pgfscope}%
\pgfsetbuttcap%
\pgfsetroundjoin%
\definecolor{currentfill}{rgb}{0.000000,0.000000,0.000000}%
\pgfsetfillcolor{currentfill}%
\pgfsetlinewidth{0.803000pt}%
\definecolor{currentstroke}{rgb}{0.000000,0.000000,0.000000}%
\pgfsetstrokecolor{currentstroke}%
\pgfsetdash{}{0pt}%
\pgfsys@defobject{currentmarker}{\pgfqpoint{0.000000in}{-0.048611in}}{\pgfqpoint{0.000000in}{0.000000in}}{%
\pgfpathmoveto{\pgfqpoint{0.000000in}{0.000000in}}%
\pgfpathlineto{\pgfqpoint{0.000000in}{-0.048611in}}%
\pgfusepath{stroke,fill}%
}%
\begin{pgfscope}%
\pgfsys@transformshift{3.850000in}{0.880000in}%
\pgfsys@useobject{currentmarker}{}%
\end{pgfscope}%
\end{pgfscope}%
\begin{pgfscope}%
\definecolor{textcolor}{rgb}{0.000000,0.000000,0.000000}%
\pgfsetstrokecolor{textcolor}%
\pgfsetfillcolor{textcolor}%
\pgftext[x=3.850000in,y=0.782778in,,top]{\color{textcolor}\rmfamily\fontsize{10.000000}{12.000000}\selectfont \(\displaystyle {120}\)}%
\end{pgfscope}%
\begin{pgfscope}%
\pgfpathrectangle{\pgfqpoint{0.750000in}{0.880000in}}{\pgfqpoint{4.650000in}{3.080000in}}%
\pgfusepath{clip}%
\pgfsetrectcap%
\pgfsetroundjoin%
\pgfsetlinewidth{0.803000pt}%
\definecolor{currentstroke}{rgb}{0.690196,0.690196,0.690196}%
\pgfsetstrokecolor{currentstroke}%
\pgfsetdash{}{0pt}%
\pgfpathmoveto{\pgfqpoint{4.366667in}{0.880000in}}%
\pgfpathlineto{\pgfqpoint{4.366667in}{3.960000in}}%
\pgfusepath{stroke}%
\end{pgfscope}%
\begin{pgfscope}%
\pgfsetbuttcap%
\pgfsetroundjoin%
\definecolor{currentfill}{rgb}{0.000000,0.000000,0.000000}%
\pgfsetfillcolor{currentfill}%
\pgfsetlinewidth{0.803000pt}%
\definecolor{currentstroke}{rgb}{0.000000,0.000000,0.000000}%
\pgfsetstrokecolor{currentstroke}%
\pgfsetdash{}{0pt}%
\pgfsys@defobject{currentmarker}{\pgfqpoint{0.000000in}{-0.048611in}}{\pgfqpoint{0.000000in}{0.000000in}}{%
\pgfpathmoveto{\pgfqpoint{0.000000in}{0.000000in}}%
\pgfpathlineto{\pgfqpoint{0.000000in}{-0.048611in}}%
\pgfusepath{stroke,fill}%
}%
\begin{pgfscope}%
\pgfsys@transformshift{4.366667in}{0.880000in}%
\pgfsys@useobject{currentmarker}{}%
\end{pgfscope}%
\end{pgfscope}%
\begin{pgfscope}%
\definecolor{textcolor}{rgb}{0.000000,0.000000,0.000000}%
\pgfsetstrokecolor{textcolor}%
\pgfsetfillcolor{textcolor}%
\pgftext[x=4.366667in,y=0.782778in,,top]{\color{textcolor}\rmfamily\fontsize{10.000000}{12.000000}\selectfont \(\displaystyle {140}\)}%
\end{pgfscope}%
\begin{pgfscope}%
\pgfpathrectangle{\pgfqpoint{0.750000in}{0.880000in}}{\pgfqpoint{4.650000in}{3.080000in}}%
\pgfusepath{clip}%
\pgfsetrectcap%
\pgfsetroundjoin%
\pgfsetlinewidth{0.803000pt}%
\definecolor{currentstroke}{rgb}{0.690196,0.690196,0.690196}%
\pgfsetstrokecolor{currentstroke}%
\pgfsetdash{}{0pt}%
\pgfpathmoveto{\pgfqpoint{4.883333in}{0.880000in}}%
\pgfpathlineto{\pgfqpoint{4.883333in}{3.960000in}}%
\pgfusepath{stroke}%
\end{pgfscope}%
\begin{pgfscope}%
\pgfsetbuttcap%
\pgfsetroundjoin%
\definecolor{currentfill}{rgb}{0.000000,0.000000,0.000000}%
\pgfsetfillcolor{currentfill}%
\pgfsetlinewidth{0.803000pt}%
\definecolor{currentstroke}{rgb}{0.000000,0.000000,0.000000}%
\pgfsetstrokecolor{currentstroke}%
\pgfsetdash{}{0pt}%
\pgfsys@defobject{currentmarker}{\pgfqpoint{0.000000in}{-0.048611in}}{\pgfqpoint{0.000000in}{0.000000in}}{%
\pgfpathmoveto{\pgfqpoint{0.000000in}{0.000000in}}%
\pgfpathlineto{\pgfqpoint{0.000000in}{-0.048611in}}%
\pgfusepath{stroke,fill}%
}%
\begin{pgfscope}%
\pgfsys@transformshift{4.883333in}{0.880000in}%
\pgfsys@useobject{currentmarker}{}%
\end{pgfscope}%
\end{pgfscope}%
\begin{pgfscope}%
\definecolor{textcolor}{rgb}{0.000000,0.000000,0.000000}%
\pgfsetstrokecolor{textcolor}%
\pgfsetfillcolor{textcolor}%
\pgftext[x=4.883333in,y=0.782778in,,top]{\color{textcolor}\rmfamily\fontsize{10.000000}{12.000000}\selectfont \(\displaystyle {160}\)}%
\end{pgfscope}%
\begin{pgfscope}%
\pgfpathrectangle{\pgfqpoint{0.750000in}{0.880000in}}{\pgfqpoint{4.650000in}{3.080000in}}%
\pgfusepath{clip}%
\pgfsetrectcap%
\pgfsetroundjoin%
\pgfsetlinewidth{0.803000pt}%
\definecolor{currentstroke}{rgb}{0.690196,0.690196,0.690196}%
\pgfsetstrokecolor{currentstroke}%
\pgfsetdash{}{0pt}%
\pgfpathmoveto{\pgfqpoint{5.400000in}{0.880000in}}%
\pgfpathlineto{\pgfqpoint{5.400000in}{3.960000in}}%
\pgfusepath{stroke}%
\end{pgfscope}%
\begin{pgfscope}%
\pgfsetbuttcap%
\pgfsetroundjoin%
\definecolor{currentfill}{rgb}{0.000000,0.000000,0.000000}%
\pgfsetfillcolor{currentfill}%
\pgfsetlinewidth{0.803000pt}%
\definecolor{currentstroke}{rgb}{0.000000,0.000000,0.000000}%
\pgfsetstrokecolor{currentstroke}%
\pgfsetdash{}{0pt}%
\pgfsys@defobject{currentmarker}{\pgfqpoint{0.000000in}{-0.048611in}}{\pgfqpoint{0.000000in}{0.000000in}}{%
\pgfpathmoveto{\pgfqpoint{0.000000in}{0.000000in}}%
\pgfpathlineto{\pgfqpoint{0.000000in}{-0.048611in}}%
\pgfusepath{stroke,fill}%
}%
\begin{pgfscope}%
\pgfsys@transformshift{5.400000in}{0.880000in}%
\pgfsys@useobject{currentmarker}{}%
\end{pgfscope}%
\end{pgfscope}%
\begin{pgfscope}%
\definecolor{textcolor}{rgb}{0.000000,0.000000,0.000000}%
\pgfsetstrokecolor{textcolor}%
\pgfsetfillcolor{textcolor}%
\pgftext[x=5.400000in,y=0.782778in,,top]{\color{textcolor}\rmfamily\fontsize{10.000000}{12.000000}\selectfont \(\displaystyle {180}\)}%
\end{pgfscope}%
\begin{pgfscope}%
\definecolor{textcolor}{rgb}{0.000000,0.000000,0.000000}%
\pgfsetstrokecolor{textcolor}%
\pgfsetfillcolor{textcolor}%
\pgftext[x=3.075000in,y=0.594776in,,top]{\color{textcolor}\rmfamily\fontsize{10.000000}{12.000000}\selectfont power angle \(\displaystyle \delta\) in deg}%
\end{pgfscope}%
\begin{pgfscope}%
\pgfpathrectangle{\pgfqpoint{0.750000in}{0.880000in}}{\pgfqpoint{4.650000in}{3.080000in}}%
\pgfusepath{clip}%
\pgfsetrectcap%
\pgfsetroundjoin%
\pgfsetlinewidth{0.803000pt}%
\definecolor{currentstroke}{rgb}{0.690196,0.690196,0.690196}%
\pgfsetstrokecolor{currentstroke}%
\pgfsetdash{}{0pt}%
\pgfpathmoveto{\pgfqpoint{0.750000in}{3.823047in}}%
\pgfpathlineto{\pgfqpoint{5.400000in}{3.823047in}}%
\pgfusepath{stroke}%
\end{pgfscope}%
\begin{pgfscope}%
\pgfsetbuttcap%
\pgfsetroundjoin%
\definecolor{currentfill}{rgb}{0.000000,0.000000,0.000000}%
\pgfsetfillcolor{currentfill}%
\pgfsetlinewidth{0.803000pt}%
\definecolor{currentstroke}{rgb}{0.000000,0.000000,0.000000}%
\pgfsetstrokecolor{currentstroke}%
\pgfsetdash{}{0pt}%
\pgfsys@defobject{currentmarker}{\pgfqpoint{-0.048611in}{0.000000in}}{\pgfqpoint{-0.000000in}{0.000000in}}{%
\pgfpathmoveto{\pgfqpoint{-0.000000in}{0.000000in}}%
\pgfpathlineto{\pgfqpoint{-0.048611in}{0.000000in}}%
\pgfusepath{stroke,fill}%
}%
\begin{pgfscope}%
\pgfsys@transformshift{0.750000in}{3.823047in}%
\pgfsys@useobject{currentmarker}{}%
\end{pgfscope}%
\end{pgfscope}%
\begin{pgfscope}%
\definecolor{textcolor}{rgb}{0.000000,0.000000,0.000000}%
\pgfsetstrokecolor{textcolor}%
\pgfsetfillcolor{textcolor}%
\pgftext[x=0.405863in, y=3.771947in, left, base]{\color{textcolor}\rmfamily\fontsize{10.000000}{12.000000}\selectfont \(\displaystyle {0.00}\)}%
\end{pgfscope}%
\begin{pgfscope}%
\pgfpathrectangle{\pgfqpoint{0.750000in}{0.880000in}}{\pgfqpoint{4.650000in}{3.080000in}}%
\pgfusepath{clip}%
\pgfsetrectcap%
\pgfsetroundjoin%
\pgfsetlinewidth{0.803000pt}%
\definecolor{currentstroke}{rgb}{0.690196,0.690196,0.690196}%
\pgfsetstrokecolor{currentstroke}%
\pgfsetdash{}{0pt}%
\pgfpathmoveto{\pgfqpoint{0.750000in}{3.480665in}}%
\pgfpathlineto{\pgfqpoint{5.400000in}{3.480665in}}%
\pgfusepath{stroke}%
\end{pgfscope}%
\begin{pgfscope}%
\pgfsetbuttcap%
\pgfsetroundjoin%
\definecolor{currentfill}{rgb}{0.000000,0.000000,0.000000}%
\pgfsetfillcolor{currentfill}%
\pgfsetlinewidth{0.803000pt}%
\definecolor{currentstroke}{rgb}{0.000000,0.000000,0.000000}%
\pgfsetstrokecolor{currentstroke}%
\pgfsetdash{}{0pt}%
\pgfsys@defobject{currentmarker}{\pgfqpoint{-0.048611in}{0.000000in}}{\pgfqpoint{-0.000000in}{0.000000in}}{%
\pgfpathmoveto{\pgfqpoint{-0.000000in}{0.000000in}}%
\pgfpathlineto{\pgfqpoint{-0.048611in}{0.000000in}}%
\pgfusepath{stroke,fill}%
}%
\begin{pgfscope}%
\pgfsys@transformshift{0.750000in}{3.480665in}%
\pgfsys@useobject{currentmarker}{}%
\end{pgfscope}%
\end{pgfscope}%
\begin{pgfscope}%
\definecolor{textcolor}{rgb}{0.000000,0.000000,0.000000}%
\pgfsetstrokecolor{textcolor}%
\pgfsetfillcolor{textcolor}%
\pgftext[x=0.405863in, y=3.429565in, left, base]{\color{textcolor}\rmfamily\fontsize{10.000000}{12.000000}\selectfont \(\displaystyle {0.25}\)}%
\end{pgfscope}%
\begin{pgfscope}%
\pgfpathrectangle{\pgfqpoint{0.750000in}{0.880000in}}{\pgfqpoint{4.650000in}{3.080000in}}%
\pgfusepath{clip}%
\pgfsetrectcap%
\pgfsetroundjoin%
\pgfsetlinewidth{0.803000pt}%
\definecolor{currentstroke}{rgb}{0.690196,0.690196,0.690196}%
\pgfsetstrokecolor{currentstroke}%
\pgfsetdash{}{0pt}%
\pgfpathmoveto{\pgfqpoint{0.750000in}{3.138283in}}%
\pgfpathlineto{\pgfqpoint{5.400000in}{3.138283in}}%
\pgfusepath{stroke}%
\end{pgfscope}%
\begin{pgfscope}%
\pgfsetbuttcap%
\pgfsetroundjoin%
\definecolor{currentfill}{rgb}{0.000000,0.000000,0.000000}%
\pgfsetfillcolor{currentfill}%
\pgfsetlinewidth{0.803000pt}%
\definecolor{currentstroke}{rgb}{0.000000,0.000000,0.000000}%
\pgfsetstrokecolor{currentstroke}%
\pgfsetdash{}{0pt}%
\pgfsys@defobject{currentmarker}{\pgfqpoint{-0.048611in}{0.000000in}}{\pgfqpoint{-0.000000in}{0.000000in}}{%
\pgfpathmoveto{\pgfqpoint{-0.000000in}{0.000000in}}%
\pgfpathlineto{\pgfqpoint{-0.048611in}{0.000000in}}%
\pgfusepath{stroke,fill}%
}%
\begin{pgfscope}%
\pgfsys@transformshift{0.750000in}{3.138283in}%
\pgfsys@useobject{currentmarker}{}%
\end{pgfscope}%
\end{pgfscope}%
\begin{pgfscope}%
\definecolor{textcolor}{rgb}{0.000000,0.000000,0.000000}%
\pgfsetstrokecolor{textcolor}%
\pgfsetfillcolor{textcolor}%
\pgftext[x=0.405863in, y=3.087183in, left, base]{\color{textcolor}\rmfamily\fontsize{10.000000}{12.000000}\selectfont \(\displaystyle {0.50}\)}%
\end{pgfscope}%
\begin{pgfscope}%
\pgfpathrectangle{\pgfqpoint{0.750000in}{0.880000in}}{\pgfqpoint{4.650000in}{3.080000in}}%
\pgfusepath{clip}%
\pgfsetrectcap%
\pgfsetroundjoin%
\pgfsetlinewidth{0.803000pt}%
\definecolor{currentstroke}{rgb}{0.690196,0.690196,0.690196}%
\pgfsetstrokecolor{currentstroke}%
\pgfsetdash{}{0pt}%
\pgfpathmoveto{\pgfqpoint{0.750000in}{2.795901in}}%
\pgfpathlineto{\pgfqpoint{5.400000in}{2.795901in}}%
\pgfusepath{stroke}%
\end{pgfscope}%
\begin{pgfscope}%
\pgfsetbuttcap%
\pgfsetroundjoin%
\definecolor{currentfill}{rgb}{0.000000,0.000000,0.000000}%
\pgfsetfillcolor{currentfill}%
\pgfsetlinewidth{0.803000pt}%
\definecolor{currentstroke}{rgb}{0.000000,0.000000,0.000000}%
\pgfsetstrokecolor{currentstroke}%
\pgfsetdash{}{0pt}%
\pgfsys@defobject{currentmarker}{\pgfqpoint{-0.048611in}{0.000000in}}{\pgfqpoint{-0.000000in}{0.000000in}}{%
\pgfpathmoveto{\pgfqpoint{-0.000000in}{0.000000in}}%
\pgfpathlineto{\pgfqpoint{-0.048611in}{0.000000in}}%
\pgfusepath{stroke,fill}%
}%
\begin{pgfscope}%
\pgfsys@transformshift{0.750000in}{2.795901in}%
\pgfsys@useobject{currentmarker}{}%
\end{pgfscope}%
\end{pgfscope}%
\begin{pgfscope}%
\definecolor{textcolor}{rgb}{0.000000,0.000000,0.000000}%
\pgfsetstrokecolor{textcolor}%
\pgfsetfillcolor{textcolor}%
\pgftext[x=0.405863in, y=2.744801in, left, base]{\color{textcolor}\rmfamily\fontsize{10.000000}{12.000000}\selectfont \(\displaystyle {0.75}\)}%
\end{pgfscope}%
\begin{pgfscope}%
\pgfpathrectangle{\pgfqpoint{0.750000in}{0.880000in}}{\pgfqpoint{4.650000in}{3.080000in}}%
\pgfusepath{clip}%
\pgfsetrectcap%
\pgfsetroundjoin%
\pgfsetlinewidth{0.803000pt}%
\definecolor{currentstroke}{rgb}{0.690196,0.690196,0.690196}%
\pgfsetstrokecolor{currentstroke}%
\pgfsetdash{}{0pt}%
\pgfpathmoveto{\pgfqpoint{0.750000in}{2.453519in}}%
\pgfpathlineto{\pgfqpoint{5.400000in}{2.453519in}}%
\pgfusepath{stroke}%
\end{pgfscope}%
\begin{pgfscope}%
\pgfsetbuttcap%
\pgfsetroundjoin%
\definecolor{currentfill}{rgb}{0.000000,0.000000,0.000000}%
\pgfsetfillcolor{currentfill}%
\pgfsetlinewidth{0.803000pt}%
\definecolor{currentstroke}{rgb}{0.000000,0.000000,0.000000}%
\pgfsetstrokecolor{currentstroke}%
\pgfsetdash{}{0pt}%
\pgfsys@defobject{currentmarker}{\pgfqpoint{-0.048611in}{0.000000in}}{\pgfqpoint{-0.000000in}{0.000000in}}{%
\pgfpathmoveto{\pgfqpoint{-0.000000in}{0.000000in}}%
\pgfpathlineto{\pgfqpoint{-0.048611in}{0.000000in}}%
\pgfusepath{stroke,fill}%
}%
\begin{pgfscope}%
\pgfsys@transformshift{0.750000in}{2.453519in}%
\pgfsys@useobject{currentmarker}{}%
\end{pgfscope}%
\end{pgfscope}%
\begin{pgfscope}%
\definecolor{textcolor}{rgb}{0.000000,0.000000,0.000000}%
\pgfsetstrokecolor{textcolor}%
\pgfsetfillcolor{textcolor}%
\pgftext[x=0.405863in, y=2.402419in, left, base]{\color{textcolor}\rmfamily\fontsize{10.000000}{12.000000}\selectfont \(\displaystyle {1.00}\)}%
\end{pgfscope}%
\begin{pgfscope}%
\pgfpathrectangle{\pgfqpoint{0.750000in}{0.880000in}}{\pgfqpoint{4.650000in}{3.080000in}}%
\pgfusepath{clip}%
\pgfsetrectcap%
\pgfsetroundjoin%
\pgfsetlinewidth{0.803000pt}%
\definecolor{currentstroke}{rgb}{0.690196,0.690196,0.690196}%
\pgfsetstrokecolor{currentstroke}%
\pgfsetdash{}{0pt}%
\pgfpathmoveto{\pgfqpoint{0.750000in}{2.111137in}}%
\pgfpathlineto{\pgfqpoint{5.400000in}{2.111137in}}%
\pgfusepath{stroke}%
\end{pgfscope}%
\begin{pgfscope}%
\pgfsetbuttcap%
\pgfsetroundjoin%
\definecolor{currentfill}{rgb}{0.000000,0.000000,0.000000}%
\pgfsetfillcolor{currentfill}%
\pgfsetlinewidth{0.803000pt}%
\definecolor{currentstroke}{rgb}{0.000000,0.000000,0.000000}%
\pgfsetstrokecolor{currentstroke}%
\pgfsetdash{}{0pt}%
\pgfsys@defobject{currentmarker}{\pgfqpoint{-0.048611in}{0.000000in}}{\pgfqpoint{-0.000000in}{0.000000in}}{%
\pgfpathmoveto{\pgfqpoint{-0.000000in}{0.000000in}}%
\pgfpathlineto{\pgfqpoint{-0.048611in}{0.000000in}}%
\pgfusepath{stroke,fill}%
}%
\begin{pgfscope}%
\pgfsys@transformshift{0.750000in}{2.111137in}%
\pgfsys@useobject{currentmarker}{}%
\end{pgfscope}%
\end{pgfscope}%
\begin{pgfscope}%
\definecolor{textcolor}{rgb}{0.000000,0.000000,0.000000}%
\pgfsetstrokecolor{textcolor}%
\pgfsetfillcolor{textcolor}%
\pgftext[x=0.405863in, y=2.060037in, left, base]{\color{textcolor}\rmfamily\fontsize{10.000000}{12.000000}\selectfont \(\displaystyle {1.25}\)}%
\end{pgfscope}%
\begin{pgfscope}%
\pgfpathrectangle{\pgfqpoint{0.750000in}{0.880000in}}{\pgfqpoint{4.650000in}{3.080000in}}%
\pgfusepath{clip}%
\pgfsetrectcap%
\pgfsetroundjoin%
\pgfsetlinewidth{0.803000pt}%
\definecolor{currentstroke}{rgb}{0.690196,0.690196,0.690196}%
\pgfsetstrokecolor{currentstroke}%
\pgfsetdash{}{0pt}%
\pgfpathmoveto{\pgfqpoint{0.750000in}{1.768755in}}%
\pgfpathlineto{\pgfqpoint{5.400000in}{1.768755in}}%
\pgfusepath{stroke}%
\end{pgfscope}%
\begin{pgfscope}%
\pgfsetbuttcap%
\pgfsetroundjoin%
\definecolor{currentfill}{rgb}{0.000000,0.000000,0.000000}%
\pgfsetfillcolor{currentfill}%
\pgfsetlinewidth{0.803000pt}%
\definecolor{currentstroke}{rgb}{0.000000,0.000000,0.000000}%
\pgfsetstrokecolor{currentstroke}%
\pgfsetdash{}{0pt}%
\pgfsys@defobject{currentmarker}{\pgfqpoint{-0.048611in}{0.000000in}}{\pgfqpoint{-0.000000in}{0.000000in}}{%
\pgfpathmoveto{\pgfqpoint{-0.000000in}{0.000000in}}%
\pgfpathlineto{\pgfqpoint{-0.048611in}{0.000000in}}%
\pgfusepath{stroke,fill}%
}%
\begin{pgfscope}%
\pgfsys@transformshift{0.750000in}{1.768755in}%
\pgfsys@useobject{currentmarker}{}%
\end{pgfscope}%
\end{pgfscope}%
\begin{pgfscope}%
\definecolor{textcolor}{rgb}{0.000000,0.000000,0.000000}%
\pgfsetstrokecolor{textcolor}%
\pgfsetfillcolor{textcolor}%
\pgftext[x=0.405863in, y=1.717655in, left, base]{\color{textcolor}\rmfamily\fontsize{10.000000}{12.000000}\selectfont \(\displaystyle {1.50}\)}%
\end{pgfscope}%
\begin{pgfscope}%
\pgfpathrectangle{\pgfqpoint{0.750000in}{0.880000in}}{\pgfqpoint{4.650000in}{3.080000in}}%
\pgfusepath{clip}%
\pgfsetrectcap%
\pgfsetroundjoin%
\pgfsetlinewidth{0.803000pt}%
\definecolor{currentstroke}{rgb}{0.690196,0.690196,0.690196}%
\pgfsetstrokecolor{currentstroke}%
\pgfsetdash{}{0pt}%
\pgfpathmoveto{\pgfqpoint{0.750000in}{1.426373in}}%
\pgfpathlineto{\pgfqpoint{5.400000in}{1.426373in}}%
\pgfusepath{stroke}%
\end{pgfscope}%
\begin{pgfscope}%
\pgfsetbuttcap%
\pgfsetroundjoin%
\definecolor{currentfill}{rgb}{0.000000,0.000000,0.000000}%
\pgfsetfillcolor{currentfill}%
\pgfsetlinewidth{0.803000pt}%
\definecolor{currentstroke}{rgb}{0.000000,0.000000,0.000000}%
\pgfsetstrokecolor{currentstroke}%
\pgfsetdash{}{0pt}%
\pgfsys@defobject{currentmarker}{\pgfqpoint{-0.048611in}{0.000000in}}{\pgfqpoint{-0.000000in}{0.000000in}}{%
\pgfpathmoveto{\pgfqpoint{-0.000000in}{0.000000in}}%
\pgfpathlineto{\pgfqpoint{-0.048611in}{0.000000in}}%
\pgfusepath{stroke,fill}%
}%
\begin{pgfscope}%
\pgfsys@transformshift{0.750000in}{1.426373in}%
\pgfsys@useobject{currentmarker}{}%
\end{pgfscope}%
\end{pgfscope}%
\begin{pgfscope}%
\definecolor{textcolor}{rgb}{0.000000,0.000000,0.000000}%
\pgfsetstrokecolor{textcolor}%
\pgfsetfillcolor{textcolor}%
\pgftext[x=0.405863in, y=1.375273in, left, base]{\color{textcolor}\rmfamily\fontsize{10.000000}{12.000000}\selectfont \(\displaystyle {1.75}\)}%
\end{pgfscope}%
\begin{pgfscope}%
\pgfpathrectangle{\pgfqpoint{0.750000in}{0.880000in}}{\pgfqpoint{4.650000in}{3.080000in}}%
\pgfusepath{clip}%
\pgfsetrectcap%
\pgfsetroundjoin%
\pgfsetlinewidth{0.803000pt}%
\definecolor{currentstroke}{rgb}{0.690196,0.690196,0.690196}%
\pgfsetstrokecolor{currentstroke}%
\pgfsetdash{}{0pt}%
\pgfpathmoveto{\pgfqpoint{0.750000in}{1.083991in}}%
\pgfpathlineto{\pgfqpoint{5.400000in}{1.083991in}}%
\pgfusepath{stroke}%
\end{pgfscope}%
\begin{pgfscope}%
\pgfsetbuttcap%
\pgfsetroundjoin%
\definecolor{currentfill}{rgb}{0.000000,0.000000,0.000000}%
\pgfsetfillcolor{currentfill}%
\pgfsetlinewidth{0.803000pt}%
\definecolor{currentstroke}{rgb}{0.000000,0.000000,0.000000}%
\pgfsetstrokecolor{currentstroke}%
\pgfsetdash{}{0pt}%
\pgfsys@defobject{currentmarker}{\pgfqpoint{-0.048611in}{0.000000in}}{\pgfqpoint{-0.000000in}{0.000000in}}{%
\pgfpathmoveto{\pgfqpoint{-0.000000in}{0.000000in}}%
\pgfpathlineto{\pgfqpoint{-0.048611in}{0.000000in}}%
\pgfusepath{stroke,fill}%
}%
\begin{pgfscope}%
\pgfsys@transformshift{0.750000in}{1.083991in}%
\pgfsys@useobject{currentmarker}{}%
\end{pgfscope}%
\end{pgfscope}%
\begin{pgfscope}%
\definecolor{textcolor}{rgb}{0.000000,0.000000,0.000000}%
\pgfsetstrokecolor{textcolor}%
\pgfsetfillcolor{textcolor}%
\pgftext[x=0.405863in, y=1.032891in, left, base]{\color{textcolor}\rmfamily\fontsize{10.000000}{12.000000}\selectfont \(\displaystyle {2.00}\)}%
\end{pgfscope}%
\begin{pgfscope}%
\definecolor{textcolor}{rgb}{0.000000,0.000000,0.000000}%
\pgfsetstrokecolor{textcolor}%
\pgfsetfillcolor{textcolor}%
\pgftext[x=0.350308in,y=2.420000in,,bottom,rotate=90.000000]{\color{textcolor}\rmfamily\fontsize{10.000000}{12.000000}\selectfont time in s}%
\end{pgfscope}%
\begin{pgfscope}%
\pgfpathrectangle{\pgfqpoint{0.750000in}{0.880000in}}{\pgfqpoint{4.650000in}{3.080000in}}%
\pgfusepath{clip}%
\pgfsetrectcap%
\pgfsetroundjoin%
\pgfsetlinewidth{1.505625pt}%
\definecolor{currentstroke}{rgb}{0.121569,0.466667,0.705882}%
\pgfsetstrokecolor{currentstroke}%
\pgfsetdash{}{0pt}%
\pgfpathmoveto{\pgfqpoint{2.005414in}{3.970000in}}%
\pgfpathlineto{\pgfqpoint{2.006624in}{3.814830in}}%
\pgfpathlineto{\pgfqpoint{2.010047in}{3.806613in}}%
\pgfpathlineto{\pgfqpoint{2.015749in}{3.798396in}}%
\pgfpathlineto{\pgfqpoint{2.025279in}{3.788809in}}%
\pgfpathlineto{\pgfqpoint{2.037905in}{3.779222in}}%
\pgfpathlineto{\pgfqpoint{2.056123in}{3.768266in}}%
\pgfpathlineto{\pgfqpoint{2.081442in}{3.755940in}}%
\pgfpathlineto{\pgfqpoint{2.111863in}{3.743615in}}%
\pgfpathlineto{\pgfqpoint{2.151641in}{3.729919in}}%
\pgfpathlineto{\pgfqpoint{2.202652in}{3.714854in}}%
\pgfpathlineto{\pgfqpoint{2.266958in}{3.698420in}}%
\pgfpathlineto{\pgfqpoint{2.340283in}{3.681986in}}%
\pgfpathlineto{\pgfqpoint{2.429880in}{3.664182in}}%
\pgfpathlineto{\pgfqpoint{2.812368in}{3.591597in}}%
\pgfpathlineto{\pgfqpoint{2.961555in}{3.560098in}}%
\pgfpathlineto{\pgfqpoint{3.093538in}{3.529968in}}%
\pgfpathlineto{\pgfqpoint{3.209653in}{3.501208in}}%
\pgfpathlineto{\pgfqpoint{3.311433in}{3.473818in}}%
\pgfpathlineto{\pgfqpoint{3.404934in}{3.446427in}}%
\pgfpathlineto{\pgfqpoint{3.490583in}{3.419036in}}%
\pgfpathlineto{\pgfqpoint{3.568885in}{3.391646in}}%
\pgfpathlineto{\pgfqpoint{3.640397in}{3.364255in}}%
\pgfpathlineto{\pgfqpoint{3.705709in}{3.336865in}}%
\pgfpathlineto{\pgfqpoint{3.765420in}{3.309474in}}%
\pgfpathlineto{\pgfqpoint{3.822750in}{3.280714in}}%
\pgfpathlineto{\pgfqpoint{3.875259in}{3.251954in}}%
\pgfpathlineto{\pgfqpoint{3.928033in}{3.220455in}}%
\pgfpathlineto{\pgfqpoint{3.978725in}{3.187586in}}%
\pgfpathlineto{\pgfqpoint{4.031573in}{3.150609in}}%
\pgfpathlineto{\pgfqpoint{4.090206in}{3.106784in}}%
\pgfpathlineto{\pgfqpoint{4.293461in}{2.952027in}}%
\pgfpathlineto{\pgfqpoint{4.344456in}{2.917789in}}%
\pgfpathlineto{\pgfqpoint{4.393333in}{2.887660in}}%
\pgfpathlineto{\pgfqpoint{4.441914in}{2.860269in}}%
\pgfpathlineto{\pgfqpoint{4.492537in}{2.834248in}}%
\pgfpathlineto{\pgfqpoint{4.545278in}{2.809596in}}%
\pgfpathlineto{\pgfqpoint{4.603484in}{2.784945in}}%
\pgfpathlineto{\pgfqpoint{4.664284in}{2.761663in}}%
\pgfpathlineto{\pgfqpoint{4.731621in}{2.738381in}}%
\pgfpathlineto{\pgfqpoint{4.801840in}{2.716469in}}%
\pgfpathlineto{\pgfqpoint{4.879631in}{2.694556in}}%
\pgfpathlineto{\pgfqpoint{4.966037in}{2.672644in}}%
\pgfpathlineto{\pgfqpoint{5.055913in}{2.652101in}}%
\pgfpathlineto{\pgfqpoint{5.155494in}{2.631558in}}%
\pgfpathlineto{\pgfqpoint{5.266015in}{2.611015in}}%
\pgfpathlineto{\pgfqpoint{5.388863in}{2.590472in}}%
\pgfpathlineto{\pgfqpoint{5.410000in}{2.587150in}}%
\pgfpathlineto{\pgfqpoint{5.410000in}{2.587150in}}%
\pgfusepath{stroke}%
\end{pgfscope}%
\begin{pgfscope}%
\pgfpathrectangle{\pgfqpoint{0.750000in}{0.880000in}}{\pgfqpoint{4.650000in}{3.080000in}}%
\pgfusepath{clip}%
\pgfsetbuttcap%
\pgfsetroundjoin%
\pgfsetlinewidth{1.505625pt}%
\definecolor{currentstroke}{rgb}{0.121569,0.466667,0.705882}%
\pgfsetstrokecolor{currentstroke}%
\pgfsetdash{{5.550000pt}{2.400000pt}}{0.000000pt}%
\pgfpathmoveto{\pgfqpoint{0.750000in}{3.660073in}}%
\pgfpathlineto{\pgfqpoint{5.400000in}{3.660073in}}%
\pgfusepath{stroke}%
\end{pgfscope}%
\begin{pgfscope}%
\pgfsetrectcap%
\pgfsetmiterjoin%
\pgfsetlinewidth{0.803000pt}%
\definecolor{currentstroke}{rgb}{0.000000,0.000000,0.000000}%
\pgfsetstrokecolor{currentstroke}%
\pgfsetdash{}{0pt}%
\pgfpathmoveto{\pgfqpoint{0.750000in}{0.880000in}}%
\pgfpathlineto{\pgfqpoint{0.750000in}{3.960000in}}%
\pgfusepath{stroke}%
\end{pgfscope}%
\begin{pgfscope}%
\pgfsetrectcap%
\pgfsetmiterjoin%
\pgfsetlinewidth{0.803000pt}%
\definecolor{currentstroke}{rgb}{0.000000,0.000000,0.000000}%
\pgfsetstrokecolor{currentstroke}%
\pgfsetdash{}{0pt}%
\pgfpathmoveto{\pgfqpoint{5.400000in}{0.880000in}}%
\pgfpathlineto{\pgfqpoint{5.400000in}{3.960000in}}%
\pgfusepath{stroke}%
\end{pgfscope}%
\begin{pgfscope}%
\pgfsetrectcap%
\pgfsetmiterjoin%
\pgfsetlinewidth{0.803000pt}%
\definecolor{currentstroke}{rgb}{0.000000,0.000000,0.000000}%
\pgfsetstrokecolor{currentstroke}%
\pgfsetdash{}{0pt}%
\pgfpathmoveto{\pgfqpoint{0.750000in}{0.880000in}}%
\pgfpathlineto{\pgfqpoint{5.400000in}{0.880000in}}%
\pgfusepath{stroke}%
\end{pgfscope}%
\begin{pgfscope}%
\pgfsetrectcap%
\pgfsetmiterjoin%
\pgfsetlinewidth{0.803000pt}%
\definecolor{currentstroke}{rgb}{0.000000,0.000000,0.000000}%
\pgfsetstrokecolor{currentstroke}%
\pgfsetdash{}{0pt}%
\pgfpathmoveto{\pgfqpoint{0.750000in}{3.960000in}}%
\pgfpathlineto{\pgfqpoint{5.400000in}{3.960000in}}%
\pgfusepath{stroke}%
\end{pgfscope}%
\begin{pgfscope}%
\pgfsetbuttcap%
\pgfsetmiterjoin%
\definecolor{currentfill}{rgb}{1.000000,1.000000,1.000000}%
\pgfsetfillcolor{currentfill}%
\pgfsetfillopacity{0.800000}%
\pgfsetlinewidth{1.003750pt}%
\definecolor{currentstroke}{rgb}{0.800000,0.800000,0.800000}%
\pgfsetstrokecolor{currentstroke}%
\pgfsetstrokeopacity{0.800000}%
\pgfsetdash{}{0pt}%
\pgfpathmoveto{\pgfqpoint{0.847222in}{0.949444in}}%
\pgfpathlineto{\pgfqpoint{2.250936in}{0.949444in}}%
\pgfpathquadraticcurveto{\pgfqpoint{2.278714in}{0.949444in}}{\pgfqpoint{2.278714in}{0.977222in}}%
\pgfpathlineto{\pgfqpoint{2.278714in}{1.368267in}}%
\pgfpathquadraticcurveto{\pgfqpoint{2.278714in}{1.396045in}}{\pgfqpoint{2.250936in}{1.396045in}}%
\pgfpathlineto{\pgfqpoint{0.847222in}{1.396045in}}%
\pgfpathquadraticcurveto{\pgfqpoint{0.819444in}{1.396045in}}{\pgfqpoint{0.819444in}{1.368267in}}%
\pgfpathlineto{\pgfqpoint{0.819444in}{0.977222in}}%
\pgfpathquadraticcurveto{\pgfqpoint{0.819444in}{0.949444in}}{\pgfqpoint{0.847222in}{0.949444in}}%
\pgfpathlineto{\pgfqpoint{0.847222in}{0.949444in}}%
\pgfpathclose%
\pgfusepath{stroke,fill}%
\end{pgfscope}%
\begin{pgfscope}%
\pgfsetrectcap%
\pgfsetroundjoin%
\pgfsetlinewidth{1.505625pt}%
\definecolor{currentstroke}{rgb}{0.121569,0.466667,0.705882}%
\pgfsetstrokecolor{currentstroke}%
\pgfsetdash{}{0pt}%
\pgfpathmoveto{\pgfqpoint{0.875000in}{1.286901in}}%
\pgfpathlineto{\pgfqpoint{1.013889in}{1.286901in}}%
\pgfpathlineto{\pgfqpoint{1.152778in}{1.286901in}}%
\pgfusepath{stroke}%
\end{pgfscope}%
\begin{pgfscope}%
\definecolor{textcolor}{rgb}{0.000000,0.000000,0.000000}%
\pgfsetstrokecolor{textcolor}%
\pgfsetfillcolor{textcolor}%
\pgftext[x=1.263889in,y=1.238289in,left,base]{\color{textcolor}\rmfamily\fontsize{10.000000}{12.000000}\selectfont delta}%
\end{pgfscope}%
\begin{pgfscope}%
\pgfsetbuttcap%
\pgfsetroundjoin%
\pgfsetlinewidth{1.505625pt}%
\definecolor{currentstroke}{rgb}{0.121569,0.466667,0.705882}%
\pgfsetstrokecolor{currentstroke}%
\pgfsetdash{{5.550000pt}{2.400000pt}}{0.000000pt}%
\pgfpathmoveto{\pgfqpoint{0.875000in}{1.083857in}}%
\pgfpathlineto{\pgfqpoint{1.013889in}{1.083857in}}%
\pgfpathlineto{\pgfqpoint{1.152778in}{1.083857in}}%
\pgfusepath{stroke}%
\end{pgfscope}%
\begin{pgfscope}%
\definecolor{textcolor}{rgb}{0.000000,0.000000,0.000000}%
\pgfsetstrokecolor{textcolor}%
\pgfsetfillcolor{textcolor}%
\pgftext[x=1.263889in,y=1.035246in,left,base]{\color{textcolor}\rmfamily\fontsize{10.000000}{12.000000}\selectfont clearing of fault}%
\end{pgfscope}%
\begin{pgfscope}%
\definecolor{textcolor}{rgb}{0.000000,0.000000,0.000000}%
\pgfsetstrokecolor{textcolor}%
\pgfsetfillcolor{textcolor}%
\pgftext[x=3.000000in,y=7.840000in,,top]{\color{textcolor}\rmfamily\fontsize{12.000000}{14.400000}\selectfont Unstable scenario - fault 1}%
\end{pgfscope}%
\end{pgfpicture}%
\makeatother%
\endgroup%


%% Creator: Matplotlib, PGF backend
%%
%% To include the figure in your LaTeX document, write
%%   \input{<filename>.pgf}
%%
%% Make sure the required packages are loaded in your preamble
%%   \usepackage{pgf}
%%
%% Also ensure that all the required font packages are loaded; for instance,
%% the lmodern package is sometimes necessary when using math font.
%%   \usepackage{lmodern}
%%
%% Figures using additional raster images can only be included by \input if
%% they are in the same directory as the main LaTeX file. For loading figures
%% from other directories you can use the `import` package
%%   \usepackage{import}
%%
%% and then include the figures with
%%   \import{<path to file>}{<filename>.pgf}
%%
%% Matplotlib used the following preamble
%%   
%%   \usepackage{fontspec}
%%   \setmainfont{Charter.ttc}[Path=\detokenize{/System/Library/Fonts/Supplemental/}]
%%   \setsansfont{DejaVuSans.ttf}[Path=\detokenize{/opt/homebrew/lib/python3.10/site-packages/matplotlib/mpl-data/fonts/ttf/}]
%%   \setmonofont{DejaVuSansMono.ttf}[Path=\detokenize{/opt/homebrew/lib/python3.10/site-packages/matplotlib/mpl-data/fonts/ttf/}]
%%   \makeatletter\@ifpackageloaded{underscore}{}{\usepackage[strings]{underscore}}\makeatother
%%
\begingroup%
\makeatletter%
\begin{pgfpicture}%
\pgfpathrectangle{\pgfpointorigin}{\pgfqpoint{6.400000in}{4.800000in}}%
\pgfusepath{use as bounding box, clip}%
\begin{pgfscope}%
\pgfsetbuttcap%
\pgfsetmiterjoin%
\definecolor{currentfill}{rgb}{1.000000,1.000000,1.000000}%
\pgfsetfillcolor{currentfill}%
\pgfsetlinewidth{0.000000pt}%
\definecolor{currentstroke}{rgb}{1.000000,1.000000,1.000000}%
\pgfsetstrokecolor{currentstroke}%
\pgfsetdash{}{0pt}%
\pgfpathmoveto{\pgfqpoint{0.000000in}{0.000000in}}%
\pgfpathlineto{\pgfqpoint{6.400000in}{0.000000in}}%
\pgfpathlineto{\pgfqpoint{6.400000in}{4.800000in}}%
\pgfpathlineto{\pgfqpoint{0.000000in}{4.800000in}}%
\pgfpathlineto{\pgfqpoint{0.000000in}{0.000000in}}%
\pgfpathclose%
\pgfusepath{fill}%
\end{pgfscope}%
\begin{pgfscope}%
\pgfsetbuttcap%
\pgfsetmiterjoin%
\definecolor{currentfill}{rgb}{1.000000,1.000000,1.000000}%
\pgfsetfillcolor{currentfill}%
\pgfsetlinewidth{0.000000pt}%
\definecolor{currentstroke}{rgb}{0.000000,0.000000,0.000000}%
\pgfsetstrokecolor{currentstroke}%
\pgfsetstrokeopacity{0.000000}%
\pgfsetdash{}{0pt}%
\pgfpathmoveto{\pgfqpoint{0.800000in}{0.528000in}}%
\pgfpathlineto{\pgfqpoint{5.760000in}{0.528000in}}%
\pgfpathlineto{\pgfqpoint{5.760000in}{4.224000in}}%
\pgfpathlineto{\pgfqpoint{0.800000in}{4.224000in}}%
\pgfpathlineto{\pgfqpoint{0.800000in}{0.528000in}}%
\pgfpathclose%
\pgfusepath{fill}%
\end{pgfscope}%
\begin{pgfscope}%
\pgfpathrectangle{\pgfqpoint{0.800000in}{0.528000in}}{\pgfqpoint{4.960000in}{3.696000in}}%
\pgfusepath{clip}%
\pgfsetrectcap%
\pgfsetroundjoin%
\pgfsetlinewidth{0.803000pt}%
\definecolor{currentstroke}{rgb}{0.690196,0.690196,0.690196}%
\pgfsetstrokecolor{currentstroke}%
\pgfsetdash{}{0pt}%
\pgfpathmoveto{\pgfqpoint{1.025455in}{0.528000in}}%
\pgfpathlineto{\pgfqpoint{1.025455in}{4.224000in}}%
\pgfusepath{stroke}%
\end{pgfscope}%
\begin{pgfscope}%
\pgfsetbuttcap%
\pgfsetroundjoin%
\definecolor{currentfill}{rgb}{0.000000,0.000000,0.000000}%
\pgfsetfillcolor{currentfill}%
\pgfsetlinewidth{0.803000pt}%
\definecolor{currentstroke}{rgb}{0.000000,0.000000,0.000000}%
\pgfsetstrokecolor{currentstroke}%
\pgfsetdash{}{0pt}%
\pgfsys@defobject{currentmarker}{\pgfqpoint{0.000000in}{-0.048611in}}{\pgfqpoint{0.000000in}{0.000000in}}{%
\pgfpathmoveto{\pgfqpoint{0.000000in}{0.000000in}}%
\pgfpathlineto{\pgfqpoint{0.000000in}{-0.048611in}}%
\pgfusepath{stroke,fill}%
}%
\begin{pgfscope}%
\pgfsys@transformshift{1.025455in}{0.528000in}%
\pgfsys@useobject{currentmarker}{}%
\end{pgfscope}%
\end{pgfscope}%
\begin{pgfscope}%
\definecolor{textcolor}{rgb}{0.000000,0.000000,0.000000}%
\pgfsetstrokecolor{textcolor}%
\pgfsetfillcolor{textcolor}%
\pgftext[x=1.025455in,y=0.430778in,,top]{\color{textcolor}\rmfamily\fontsize{10.000000}{12.000000}\selectfont \(\displaystyle {\ensuremath{-}1.0}\)}%
\end{pgfscope}%
\begin{pgfscope}%
\pgfpathrectangle{\pgfqpoint{0.800000in}{0.528000in}}{\pgfqpoint{4.960000in}{3.696000in}}%
\pgfusepath{clip}%
\pgfsetrectcap%
\pgfsetroundjoin%
\pgfsetlinewidth{0.803000pt}%
\definecolor{currentstroke}{rgb}{0.690196,0.690196,0.690196}%
\pgfsetstrokecolor{currentstroke}%
\pgfsetdash{}{0pt}%
\pgfpathmoveto{\pgfqpoint{1.777220in}{0.528000in}}%
\pgfpathlineto{\pgfqpoint{1.777220in}{4.224000in}}%
\pgfusepath{stroke}%
\end{pgfscope}%
\begin{pgfscope}%
\pgfsetbuttcap%
\pgfsetroundjoin%
\definecolor{currentfill}{rgb}{0.000000,0.000000,0.000000}%
\pgfsetfillcolor{currentfill}%
\pgfsetlinewidth{0.803000pt}%
\definecolor{currentstroke}{rgb}{0.000000,0.000000,0.000000}%
\pgfsetstrokecolor{currentstroke}%
\pgfsetdash{}{0pt}%
\pgfsys@defobject{currentmarker}{\pgfqpoint{0.000000in}{-0.048611in}}{\pgfqpoint{0.000000in}{0.000000in}}{%
\pgfpathmoveto{\pgfqpoint{0.000000in}{0.000000in}}%
\pgfpathlineto{\pgfqpoint{0.000000in}{-0.048611in}}%
\pgfusepath{stroke,fill}%
}%
\begin{pgfscope}%
\pgfsys@transformshift{1.777220in}{0.528000in}%
\pgfsys@useobject{currentmarker}{}%
\end{pgfscope}%
\end{pgfscope}%
\begin{pgfscope}%
\definecolor{textcolor}{rgb}{0.000000,0.000000,0.000000}%
\pgfsetstrokecolor{textcolor}%
\pgfsetfillcolor{textcolor}%
\pgftext[x=1.777220in,y=0.430778in,,top]{\color{textcolor}\rmfamily\fontsize{10.000000}{12.000000}\selectfont \(\displaystyle {\ensuremath{-}0.5}\)}%
\end{pgfscope}%
\begin{pgfscope}%
\pgfpathrectangle{\pgfqpoint{0.800000in}{0.528000in}}{\pgfqpoint{4.960000in}{3.696000in}}%
\pgfusepath{clip}%
\pgfsetrectcap%
\pgfsetroundjoin%
\pgfsetlinewidth{0.803000pt}%
\definecolor{currentstroke}{rgb}{0.690196,0.690196,0.690196}%
\pgfsetstrokecolor{currentstroke}%
\pgfsetdash{}{0pt}%
\pgfpathmoveto{\pgfqpoint{2.528986in}{0.528000in}}%
\pgfpathlineto{\pgfqpoint{2.528986in}{4.224000in}}%
\pgfusepath{stroke}%
\end{pgfscope}%
\begin{pgfscope}%
\pgfsetbuttcap%
\pgfsetroundjoin%
\definecolor{currentfill}{rgb}{0.000000,0.000000,0.000000}%
\pgfsetfillcolor{currentfill}%
\pgfsetlinewidth{0.803000pt}%
\definecolor{currentstroke}{rgb}{0.000000,0.000000,0.000000}%
\pgfsetstrokecolor{currentstroke}%
\pgfsetdash{}{0pt}%
\pgfsys@defobject{currentmarker}{\pgfqpoint{0.000000in}{-0.048611in}}{\pgfqpoint{0.000000in}{0.000000in}}{%
\pgfpathmoveto{\pgfqpoint{0.000000in}{0.000000in}}%
\pgfpathlineto{\pgfqpoint{0.000000in}{-0.048611in}}%
\pgfusepath{stroke,fill}%
}%
\begin{pgfscope}%
\pgfsys@transformshift{2.528986in}{0.528000in}%
\pgfsys@useobject{currentmarker}{}%
\end{pgfscope}%
\end{pgfscope}%
\begin{pgfscope}%
\definecolor{textcolor}{rgb}{0.000000,0.000000,0.000000}%
\pgfsetstrokecolor{textcolor}%
\pgfsetfillcolor{textcolor}%
\pgftext[x=2.528986in,y=0.430778in,,top]{\color{textcolor}\rmfamily\fontsize{10.000000}{12.000000}\selectfont \(\displaystyle {0.0}\)}%
\end{pgfscope}%
\begin{pgfscope}%
\pgfpathrectangle{\pgfqpoint{0.800000in}{0.528000in}}{\pgfqpoint{4.960000in}{3.696000in}}%
\pgfusepath{clip}%
\pgfsetrectcap%
\pgfsetroundjoin%
\pgfsetlinewidth{0.803000pt}%
\definecolor{currentstroke}{rgb}{0.690196,0.690196,0.690196}%
\pgfsetstrokecolor{currentstroke}%
\pgfsetdash{}{0pt}%
\pgfpathmoveto{\pgfqpoint{3.280752in}{0.528000in}}%
\pgfpathlineto{\pgfqpoint{3.280752in}{4.224000in}}%
\pgfusepath{stroke}%
\end{pgfscope}%
\begin{pgfscope}%
\pgfsetbuttcap%
\pgfsetroundjoin%
\definecolor{currentfill}{rgb}{0.000000,0.000000,0.000000}%
\pgfsetfillcolor{currentfill}%
\pgfsetlinewidth{0.803000pt}%
\definecolor{currentstroke}{rgb}{0.000000,0.000000,0.000000}%
\pgfsetstrokecolor{currentstroke}%
\pgfsetdash{}{0pt}%
\pgfsys@defobject{currentmarker}{\pgfqpoint{0.000000in}{-0.048611in}}{\pgfqpoint{0.000000in}{0.000000in}}{%
\pgfpathmoveto{\pgfqpoint{0.000000in}{0.000000in}}%
\pgfpathlineto{\pgfqpoint{0.000000in}{-0.048611in}}%
\pgfusepath{stroke,fill}%
}%
\begin{pgfscope}%
\pgfsys@transformshift{3.280752in}{0.528000in}%
\pgfsys@useobject{currentmarker}{}%
\end{pgfscope}%
\end{pgfscope}%
\begin{pgfscope}%
\definecolor{textcolor}{rgb}{0.000000,0.000000,0.000000}%
\pgfsetstrokecolor{textcolor}%
\pgfsetfillcolor{textcolor}%
\pgftext[x=3.280752in,y=0.430778in,,top]{\color{textcolor}\rmfamily\fontsize{10.000000}{12.000000}\selectfont \(\displaystyle {0.5}\)}%
\end{pgfscope}%
\begin{pgfscope}%
\pgfpathrectangle{\pgfqpoint{0.800000in}{0.528000in}}{\pgfqpoint{4.960000in}{3.696000in}}%
\pgfusepath{clip}%
\pgfsetrectcap%
\pgfsetroundjoin%
\pgfsetlinewidth{0.803000pt}%
\definecolor{currentstroke}{rgb}{0.690196,0.690196,0.690196}%
\pgfsetstrokecolor{currentstroke}%
\pgfsetdash{}{0pt}%
\pgfpathmoveto{\pgfqpoint{4.032518in}{0.528000in}}%
\pgfpathlineto{\pgfqpoint{4.032518in}{4.224000in}}%
\pgfusepath{stroke}%
\end{pgfscope}%
\begin{pgfscope}%
\pgfsetbuttcap%
\pgfsetroundjoin%
\definecolor{currentfill}{rgb}{0.000000,0.000000,0.000000}%
\pgfsetfillcolor{currentfill}%
\pgfsetlinewidth{0.803000pt}%
\definecolor{currentstroke}{rgb}{0.000000,0.000000,0.000000}%
\pgfsetstrokecolor{currentstroke}%
\pgfsetdash{}{0pt}%
\pgfsys@defobject{currentmarker}{\pgfqpoint{0.000000in}{-0.048611in}}{\pgfqpoint{0.000000in}{0.000000in}}{%
\pgfpathmoveto{\pgfqpoint{0.000000in}{0.000000in}}%
\pgfpathlineto{\pgfqpoint{0.000000in}{-0.048611in}}%
\pgfusepath{stroke,fill}%
}%
\begin{pgfscope}%
\pgfsys@transformshift{4.032518in}{0.528000in}%
\pgfsys@useobject{currentmarker}{}%
\end{pgfscope}%
\end{pgfscope}%
\begin{pgfscope}%
\definecolor{textcolor}{rgb}{0.000000,0.000000,0.000000}%
\pgfsetstrokecolor{textcolor}%
\pgfsetfillcolor{textcolor}%
\pgftext[x=4.032518in,y=0.430778in,,top]{\color{textcolor}\rmfamily\fontsize{10.000000}{12.000000}\selectfont \(\displaystyle {1.0}\)}%
\end{pgfscope}%
\begin{pgfscope}%
\pgfpathrectangle{\pgfqpoint{0.800000in}{0.528000in}}{\pgfqpoint{4.960000in}{3.696000in}}%
\pgfusepath{clip}%
\pgfsetrectcap%
\pgfsetroundjoin%
\pgfsetlinewidth{0.803000pt}%
\definecolor{currentstroke}{rgb}{0.690196,0.690196,0.690196}%
\pgfsetstrokecolor{currentstroke}%
\pgfsetdash{}{0pt}%
\pgfpathmoveto{\pgfqpoint{4.784283in}{0.528000in}}%
\pgfpathlineto{\pgfqpoint{4.784283in}{4.224000in}}%
\pgfusepath{stroke}%
\end{pgfscope}%
\begin{pgfscope}%
\pgfsetbuttcap%
\pgfsetroundjoin%
\definecolor{currentfill}{rgb}{0.000000,0.000000,0.000000}%
\pgfsetfillcolor{currentfill}%
\pgfsetlinewidth{0.803000pt}%
\definecolor{currentstroke}{rgb}{0.000000,0.000000,0.000000}%
\pgfsetstrokecolor{currentstroke}%
\pgfsetdash{}{0pt}%
\pgfsys@defobject{currentmarker}{\pgfqpoint{0.000000in}{-0.048611in}}{\pgfqpoint{0.000000in}{0.000000in}}{%
\pgfpathmoveto{\pgfqpoint{0.000000in}{0.000000in}}%
\pgfpathlineto{\pgfqpoint{0.000000in}{-0.048611in}}%
\pgfusepath{stroke,fill}%
}%
\begin{pgfscope}%
\pgfsys@transformshift{4.784283in}{0.528000in}%
\pgfsys@useobject{currentmarker}{}%
\end{pgfscope}%
\end{pgfscope}%
\begin{pgfscope}%
\definecolor{textcolor}{rgb}{0.000000,0.000000,0.000000}%
\pgfsetstrokecolor{textcolor}%
\pgfsetfillcolor{textcolor}%
\pgftext[x=4.784283in,y=0.430778in,,top]{\color{textcolor}\rmfamily\fontsize{10.000000}{12.000000}\selectfont \(\displaystyle {1.5}\)}%
\end{pgfscope}%
\begin{pgfscope}%
\pgfpathrectangle{\pgfqpoint{0.800000in}{0.528000in}}{\pgfqpoint{4.960000in}{3.696000in}}%
\pgfusepath{clip}%
\pgfsetrectcap%
\pgfsetroundjoin%
\pgfsetlinewidth{0.803000pt}%
\definecolor{currentstroke}{rgb}{0.690196,0.690196,0.690196}%
\pgfsetstrokecolor{currentstroke}%
\pgfsetdash{}{0pt}%
\pgfpathmoveto{\pgfqpoint{5.536049in}{0.528000in}}%
\pgfpathlineto{\pgfqpoint{5.536049in}{4.224000in}}%
\pgfusepath{stroke}%
\end{pgfscope}%
\begin{pgfscope}%
\pgfsetbuttcap%
\pgfsetroundjoin%
\definecolor{currentfill}{rgb}{0.000000,0.000000,0.000000}%
\pgfsetfillcolor{currentfill}%
\pgfsetlinewidth{0.803000pt}%
\definecolor{currentstroke}{rgb}{0.000000,0.000000,0.000000}%
\pgfsetstrokecolor{currentstroke}%
\pgfsetdash{}{0pt}%
\pgfsys@defobject{currentmarker}{\pgfqpoint{0.000000in}{-0.048611in}}{\pgfqpoint{0.000000in}{0.000000in}}{%
\pgfpathmoveto{\pgfqpoint{0.000000in}{0.000000in}}%
\pgfpathlineto{\pgfqpoint{0.000000in}{-0.048611in}}%
\pgfusepath{stroke,fill}%
}%
\begin{pgfscope}%
\pgfsys@transformshift{5.536049in}{0.528000in}%
\pgfsys@useobject{currentmarker}{}%
\end{pgfscope}%
\end{pgfscope}%
\begin{pgfscope}%
\definecolor{textcolor}{rgb}{0.000000,0.000000,0.000000}%
\pgfsetstrokecolor{textcolor}%
\pgfsetfillcolor{textcolor}%
\pgftext[x=5.536049in,y=0.430778in,,top]{\color{textcolor}\rmfamily\fontsize{10.000000}{12.000000}\selectfont \(\displaystyle {2.0}\)}%
\end{pgfscope}%
\begin{pgfscope}%
\pgfpathrectangle{\pgfqpoint{0.800000in}{0.528000in}}{\pgfqpoint{4.960000in}{3.696000in}}%
\pgfusepath{clip}%
\pgfsetrectcap%
\pgfsetroundjoin%
\pgfsetlinewidth{0.803000pt}%
\definecolor{currentstroke}{rgb}{0.690196,0.690196,0.690196}%
\pgfsetstrokecolor{currentstroke}%
\pgfsetdash{}{0pt}%
\pgfpathmoveto{\pgfqpoint{0.800000in}{0.976000in}}%
\pgfpathlineto{\pgfqpoint{5.760000in}{0.976000in}}%
\pgfusepath{stroke}%
\end{pgfscope}%
\begin{pgfscope}%
\pgfsetbuttcap%
\pgfsetroundjoin%
\definecolor{currentfill}{rgb}{0.000000,0.000000,0.000000}%
\pgfsetfillcolor{currentfill}%
\pgfsetlinewidth{0.803000pt}%
\definecolor{currentstroke}{rgb}{0.000000,0.000000,0.000000}%
\pgfsetstrokecolor{currentstroke}%
\pgfsetdash{}{0pt}%
\pgfsys@defobject{currentmarker}{\pgfqpoint{-0.048611in}{0.000000in}}{\pgfqpoint{-0.000000in}{0.000000in}}{%
\pgfpathmoveto{\pgfqpoint{-0.000000in}{0.000000in}}%
\pgfpathlineto{\pgfqpoint{-0.048611in}{0.000000in}}%
\pgfusepath{stroke,fill}%
}%
\begin{pgfscope}%
\pgfsys@transformshift{0.800000in}{0.976000in}%
\pgfsys@useobject{currentmarker}{}%
\end{pgfscope}%
\end{pgfscope}%
\begin{pgfscope}%
\definecolor{textcolor}{rgb}{0.000000,0.000000,0.000000}%
\pgfsetstrokecolor{textcolor}%
\pgfsetfillcolor{textcolor}%
\pgftext[x=0.417283in, y=0.924900in, left, base]{\color{textcolor}\rmfamily\fontsize{10.000000}{12.000000}\selectfont \(\displaystyle {\ensuremath{-}1.0}\)}%
\end{pgfscope}%
\begin{pgfscope}%
\pgfpathrectangle{\pgfqpoint{0.800000in}{0.528000in}}{\pgfqpoint{4.960000in}{3.696000in}}%
\pgfusepath{clip}%
\pgfsetrectcap%
\pgfsetroundjoin%
\pgfsetlinewidth{0.803000pt}%
\definecolor{currentstroke}{rgb}{0.690196,0.690196,0.690196}%
\pgfsetstrokecolor{currentstroke}%
\pgfsetdash{}{0pt}%
\pgfpathmoveto{\pgfqpoint{0.800000in}{1.676000in}}%
\pgfpathlineto{\pgfqpoint{5.760000in}{1.676000in}}%
\pgfusepath{stroke}%
\end{pgfscope}%
\begin{pgfscope}%
\pgfsetbuttcap%
\pgfsetroundjoin%
\definecolor{currentfill}{rgb}{0.000000,0.000000,0.000000}%
\pgfsetfillcolor{currentfill}%
\pgfsetlinewidth{0.803000pt}%
\definecolor{currentstroke}{rgb}{0.000000,0.000000,0.000000}%
\pgfsetstrokecolor{currentstroke}%
\pgfsetdash{}{0pt}%
\pgfsys@defobject{currentmarker}{\pgfqpoint{-0.048611in}{0.000000in}}{\pgfqpoint{-0.000000in}{0.000000in}}{%
\pgfpathmoveto{\pgfqpoint{-0.000000in}{0.000000in}}%
\pgfpathlineto{\pgfqpoint{-0.048611in}{0.000000in}}%
\pgfusepath{stroke,fill}%
}%
\begin{pgfscope}%
\pgfsys@transformshift{0.800000in}{1.676000in}%
\pgfsys@useobject{currentmarker}{}%
\end{pgfscope}%
\end{pgfscope}%
\begin{pgfscope}%
\definecolor{textcolor}{rgb}{0.000000,0.000000,0.000000}%
\pgfsetstrokecolor{textcolor}%
\pgfsetfillcolor{textcolor}%
\pgftext[x=0.417283in, y=1.624900in, left, base]{\color{textcolor}\rmfamily\fontsize{10.000000}{12.000000}\selectfont \(\displaystyle {\ensuremath{-}0.5}\)}%
\end{pgfscope}%
\begin{pgfscope}%
\pgfpathrectangle{\pgfqpoint{0.800000in}{0.528000in}}{\pgfqpoint{4.960000in}{3.696000in}}%
\pgfusepath{clip}%
\pgfsetrectcap%
\pgfsetroundjoin%
\pgfsetlinewidth{0.803000pt}%
\definecolor{currentstroke}{rgb}{0.690196,0.690196,0.690196}%
\pgfsetstrokecolor{currentstroke}%
\pgfsetdash{}{0pt}%
\pgfpathmoveto{\pgfqpoint{0.800000in}{2.376000in}}%
\pgfpathlineto{\pgfqpoint{5.760000in}{2.376000in}}%
\pgfusepath{stroke}%
\end{pgfscope}%
\begin{pgfscope}%
\pgfsetbuttcap%
\pgfsetroundjoin%
\definecolor{currentfill}{rgb}{0.000000,0.000000,0.000000}%
\pgfsetfillcolor{currentfill}%
\pgfsetlinewidth{0.803000pt}%
\definecolor{currentstroke}{rgb}{0.000000,0.000000,0.000000}%
\pgfsetstrokecolor{currentstroke}%
\pgfsetdash{}{0pt}%
\pgfsys@defobject{currentmarker}{\pgfqpoint{-0.048611in}{0.000000in}}{\pgfqpoint{-0.000000in}{0.000000in}}{%
\pgfpathmoveto{\pgfqpoint{-0.000000in}{0.000000in}}%
\pgfpathlineto{\pgfqpoint{-0.048611in}{0.000000in}}%
\pgfusepath{stroke,fill}%
}%
\begin{pgfscope}%
\pgfsys@transformshift{0.800000in}{2.376000in}%
\pgfsys@useobject{currentmarker}{}%
\end{pgfscope}%
\end{pgfscope}%
\begin{pgfscope}%
\definecolor{textcolor}{rgb}{0.000000,0.000000,0.000000}%
\pgfsetstrokecolor{textcolor}%
\pgfsetfillcolor{textcolor}%
\pgftext[x=0.525308in, y=2.324900in, left, base]{\color{textcolor}\rmfamily\fontsize{10.000000}{12.000000}\selectfont \(\displaystyle {0.0}\)}%
\end{pgfscope}%
\begin{pgfscope}%
\pgfpathrectangle{\pgfqpoint{0.800000in}{0.528000in}}{\pgfqpoint{4.960000in}{3.696000in}}%
\pgfusepath{clip}%
\pgfsetrectcap%
\pgfsetroundjoin%
\pgfsetlinewidth{0.803000pt}%
\definecolor{currentstroke}{rgb}{0.690196,0.690196,0.690196}%
\pgfsetstrokecolor{currentstroke}%
\pgfsetdash{}{0pt}%
\pgfpathmoveto{\pgfqpoint{0.800000in}{3.076000in}}%
\pgfpathlineto{\pgfqpoint{5.760000in}{3.076000in}}%
\pgfusepath{stroke}%
\end{pgfscope}%
\begin{pgfscope}%
\pgfsetbuttcap%
\pgfsetroundjoin%
\definecolor{currentfill}{rgb}{0.000000,0.000000,0.000000}%
\pgfsetfillcolor{currentfill}%
\pgfsetlinewidth{0.803000pt}%
\definecolor{currentstroke}{rgb}{0.000000,0.000000,0.000000}%
\pgfsetstrokecolor{currentstroke}%
\pgfsetdash{}{0pt}%
\pgfsys@defobject{currentmarker}{\pgfqpoint{-0.048611in}{0.000000in}}{\pgfqpoint{-0.000000in}{0.000000in}}{%
\pgfpathmoveto{\pgfqpoint{-0.000000in}{0.000000in}}%
\pgfpathlineto{\pgfqpoint{-0.048611in}{0.000000in}}%
\pgfusepath{stroke,fill}%
}%
\begin{pgfscope}%
\pgfsys@transformshift{0.800000in}{3.076000in}%
\pgfsys@useobject{currentmarker}{}%
\end{pgfscope}%
\end{pgfscope}%
\begin{pgfscope}%
\definecolor{textcolor}{rgb}{0.000000,0.000000,0.000000}%
\pgfsetstrokecolor{textcolor}%
\pgfsetfillcolor{textcolor}%
\pgftext[x=0.525308in, y=3.024900in, left, base]{\color{textcolor}\rmfamily\fontsize{10.000000}{12.000000}\selectfont \(\displaystyle {0.5}\)}%
\end{pgfscope}%
\begin{pgfscope}%
\pgfpathrectangle{\pgfqpoint{0.800000in}{0.528000in}}{\pgfqpoint{4.960000in}{3.696000in}}%
\pgfusepath{clip}%
\pgfsetrectcap%
\pgfsetroundjoin%
\pgfsetlinewidth{0.803000pt}%
\definecolor{currentstroke}{rgb}{0.690196,0.690196,0.690196}%
\pgfsetstrokecolor{currentstroke}%
\pgfsetdash{}{0pt}%
\pgfpathmoveto{\pgfqpoint{0.800000in}{3.776000in}}%
\pgfpathlineto{\pgfqpoint{5.760000in}{3.776000in}}%
\pgfusepath{stroke}%
\end{pgfscope}%
\begin{pgfscope}%
\pgfsetbuttcap%
\pgfsetroundjoin%
\definecolor{currentfill}{rgb}{0.000000,0.000000,0.000000}%
\pgfsetfillcolor{currentfill}%
\pgfsetlinewidth{0.803000pt}%
\definecolor{currentstroke}{rgb}{0.000000,0.000000,0.000000}%
\pgfsetstrokecolor{currentstroke}%
\pgfsetdash{}{0pt}%
\pgfsys@defobject{currentmarker}{\pgfqpoint{-0.048611in}{0.000000in}}{\pgfqpoint{-0.000000in}{0.000000in}}{%
\pgfpathmoveto{\pgfqpoint{-0.000000in}{0.000000in}}%
\pgfpathlineto{\pgfqpoint{-0.048611in}{0.000000in}}%
\pgfusepath{stroke,fill}%
}%
\begin{pgfscope}%
\pgfsys@transformshift{0.800000in}{3.776000in}%
\pgfsys@useobject{currentmarker}{}%
\end{pgfscope}%
\end{pgfscope}%
\begin{pgfscope}%
\definecolor{textcolor}{rgb}{0.000000,0.000000,0.000000}%
\pgfsetstrokecolor{textcolor}%
\pgfsetfillcolor{textcolor}%
\pgftext[x=0.525308in, y=3.724900in, left, base]{\color{textcolor}\rmfamily\fontsize{10.000000}{12.000000}\selectfont \(\displaystyle {1.0}\)}%
\end{pgfscope}%
\begin{pgfscope}%
\pgfpathrectangle{\pgfqpoint{0.800000in}{0.528000in}}{\pgfqpoint{4.960000in}{3.696000in}}%
\pgfusepath{clip}%
\pgfsetrectcap%
\pgfsetroundjoin%
\pgfsetlinewidth{1.505625pt}%
\definecolor{currentstroke}{rgb}{0.121569,0.466667,0.705882}%
\pgfsetstrokecolor{currentstroke}%
\pgfsetdash{}{0pt}%
\pgfpathmoveto{\pgfqpoint{1.025455in}{3.636187in}}%
\pgfpathlineto{\pgfqpoint{2.527482in}{3.636173in}}%
\pgfpathlineto{\pgfqpoint{2.528986in}{2.376052in}}%
\pgfpathlineto{\pgfqpoint{2.706403in}{2.376050in}}%
\pgfpathlineto{\pgfqpoint{2.707906in}{3.908950in}}%
\pgfpathlineto{\pgfqpoint{2.724445in}{3.942698in}}%
\pgfpathlineto{\pgfqpoint{2.740984in}{3.971485in}}%
\pgfpathlineto{\pgfqpoint{2.756019in}{3.993511in}}%
\pgfpathlineto{\pgfqpoint{2.771055in}{4.011785in}}%
\pgfpathlineto{\pgfqpoint{2.786090in}{4.026524in}}%
\pgfpathlineto{\pgfqpoint{2.801125in}{4.037967in}}%
\pgfpathlineto{\pgfqpoint{2.814657in}{4.045659in}}%
\pgfpathlineto{\pgfqpoint{2.829692in}{4.051552in}}%
\pgfpathlineto{\pgfqpoint{2.844728in}{4.054911in}}%
\pgfpathlineto{\pgfqpoint{2.859763in}{4.056000in}}%
\pgfpathlineto{\pgfqpoint{2.876302in}{4.054887in}}%
\pgfpathlineto{\pgfqpoint{2.894344in}{4.051294in}}%
\pgfpathlineto{\pgfqpoint{2.913890in}{4.045061in}}%
\pgfpathlineto{\pgfqpoint{2.936443in}{4.035457in}}%
\pgfpathlineto{\pgfqpoint{2.963507in}{4.021420in}}%
\pgfpathlineto{\pgfqpoint{3.001095in}{3.999133in}}%
\pgfpathlineto{\pgfqpoint{3.112356in}{3.931507in}}%
\pgfpathlineto{\pgfqpoint{3.149945in}{3.912159in}}%
\pgfpathlineto{\pgfqpoint{3.183022in}{3.897573in}}%
\pgfpathlineto{\pgfqpoint{3.214596in}{3.886039in}}%
\pgfpathlineto{\pgfqpoint{3.244667in}{3.877367in}}%
\pgfpathlineto{\pgfqpoint{3.274738in}{3.871039in}}%
\pgfpathlineto{\pgfqpoint{3.303305in}{3.867244in}}%
\pgfpathlineto{\pgfqpoint{3.331872in}{3.865632in}}%
\pgfpathlineto{\pgfqpoint{3.360439in}{3.866210in}}%
\pgfpathlineto{\pgfqpoint{3.389006in}{3.868966in}}%
\pgfpathlineto{\pgfqpoint{3.417573in}{3.873877in}}%
\pgfpathlineto{\pgfqpoint{3.447644in}{3.881326in}}%
\pgfpathlineto{\pgfqpoint{3.477714in}{3.891038in}}%
\pgfpathlineto{\pgfqpoint{3.509289in}{3.903545in}}%
\pgfpathlineto{\pgfqpoint{3.543870in}{3.919734in}}%
\pgfpathlineto{\pgfqpoint{3.581458in}{3.939870in}}%
\pgfpathlineto{\pgfqpoint{3.628068in}{3.967504in}}%
\pgfpathlineto{\pgfqpoint{3.719783in}{4.022574in}}%
\pgfpathlineto{\pgfqpoint{3.748350in}{4.036866in}}%
\pgfpathlineto{\pgfqpoint{3.772407in}{4.046514in}}%
\pgfpathlineto{\pgfqpoint{3.793456in}{4.052526in}}%
\pgfpathlineto{\pgfqpoint{3.811498in}{4.055425in}}%
\pgfpathlineto{\pgfqpoint{3.828037in}{4.055907in}}%
\pgfpathlineto{\pgfqpoint{3.843073in}{4.054272in}}%
\pgfpathlineto{\pgfqpoint{3.858108in}{4.050417in}}%
\pgfpathlineto{\pgfqpoint{3.873143in}{4.044099in}}%
\pgfpathlineto{\pgfqpoint{3.886675in}{4.036108in}}%
\pgfpathlineto{\pgfqpoint{3.900207in}{4.025757in}}%
\pgfpathlineto{\pgfqpoint{3.915242in}{4.011287in}}%
\pgfpathlineto{\pgfqpoint{3.930277in}{3.993491in}}%
\pgfpathlineto{\pgfqpoint{3.945313in}{3.972182in}}%
\pgfpathlineto{\pgfqpoint{3.960348in}{3.947200in}}%
\pgfpathlineto{\pgfqpoint{3.976887in}{3.915336in}}%
\pgfpathlineto{\pgfqpoint{3.993426in}{3.878781in}}%
\pgfpathlineto{\pgfqpoint{4.011468in}{3.833522in}}%
\pgfpathlineto{\pgfqpoint{4.031014in}{3.778246in}}%
\pgfpathlineto{\pgfqpoint{4.052063in}{3.711762in}}%
\pgfpathlineto{\pgfqpoint{4.074616in}{3.633171in}}%
\pgfpathlineto{\pgfqpoint{4.101680in}{3.530262in}}%
\pgfpathlineto{\pgfqpoint{4.134758in}{3.395099in}}%
\pgfpathlineto{\pgfqpoint{4.233991in}{2.983003in}}%
\pgfpathlineto{\pgfqpoint{4.259551in}{2.889561in}}%
\pgfpathlineto{\pgfqpoint{4.280600in}{2.820270in}}%
\pgfpathlineto{\pgfqpoint{4.300146in}{2.763243in}}%
\pgfpathlineto{\pgfqpoint{4.318188in}{2.717573in}}%
\pgfpathlineto{\pgfqpoint{4.333224in}{2.685021in}}%
\pgfpathlineto{\pgfqpoint{4.348259in}{2.657744in}}%
\pgfpathlineto{\pgfqpoint{4.361791in}{2.637872in}}%
\pgfpathlineto{\pgfqpoint{4.373819in}{2.624026in}}%
\pgfpathlineto{\pgfqpoint{4.384344in}{2.614909in}}%
\pgfpathlineto{\pgfqpoint{4.394869in}{2.608623in}}%
\pgfpathlineto{\pgfqpoint{4.403890in}{2.605502in}}%
\pgfpathlineto{\pgfqpoint{4.412911in}{2.604481in}}%
\pgfpathlineto{\pgfqpoint{4.421932in}{2.605562in}}%
\pgfpathlineto{\pgfqpoint{4.430953in}{2.608739in}}%
\pgfpathlineto{\pgfqpoint{4.441478in}{2.615081in}}%
\pgfpathlineto{\pgfqpoint{4.452003in}{2.624238in}}%
\pgfpathlineto{\pgfqpoint{4.464031in}{2.638105in}}%
\pgfpathlineto{\pgfqpoint{4.476059in}{2.655534in}}%
\pgfpathlineto{\pgfqpoint{4.489591in}{2.679289in}}%
\pgfpathlineto{\pgfqpoint{4.504626in}{2.710654in}}%
\pgfpathlineto{\pgfqpoint{4.521165in}{2.750906in}}%
\pgfpathlineto{\pgfqpoint{4.539208in}{2.801225in}}%
\pgfpathlineto{\pgfqpoint{4.558754in}{2.862587in}}%
\pgfpathlineto{\pgfqpoint{4.581306in}{2.941073in}}%
\pgfpathlineto{\pgfqpoint{4.608370in}{3.043957in}}%
\pgfpathlineto{\pgfqpoint{4.644455in}{3.190796in}}%
\pgfpathlineto{\pgfqpoint{4.722638in}{3.511722in}}%
\pgfpathlineto{\pgfqpoint{4.751206in}{3.618615in}}%
\pgfpathlineto{\pgfqpoint{4.776766in}{3.705748in}}%
\pgfpathlineto{\pgfqpoint{4.799319in}{3.774812in}}%
\pgfpathlineto{\pgfqpoint{4.820368in}{3.832103in}}%
\pgfpathlineto{\pgfqpoint{4.839914in}{3.878889in}}%
\pgfpathlineto{\pgfqpoint{4.857956in}{3.916572in}}%
\pgfpathlineto{\pgfqpoint{4.875999in}{3.949067in}}%
\pgfpathlineto{\pgfqpoint{4.892538in}{3.974452in}}%
\pgfpathlineto{\pgfqpoint{4.909076in}{3.995829in}}%
\pgfpathlineto{\pgfqpoint{4.925615in}{4.013433in}}%
\pgfpathlineto{\pgfqpoint{4.940651in}{4.026391in}}%
\pgfpathlineto{\pgfqpoint{4.955686in}{4.036672in}}%
\pgfpathlineto{\pgfqpoint{4.970721in}{4.044503in}}%
\pgfpathlineto{\pgfqpoint{4.987260in}{4.050565in}}%
\pgfpathlineto{\pgfqpoint{5.003799in}{4.054253in}}%
\pgfpathlineto{\pgfqpoint{5.021841in}{4.055922in}}%
\pgfpathlineto{\pgfqpoint{5.041387in}{4.055389in}}%
\pgfpathlineto{\pgfqpoint{5.062437in}{4.052599in}}%
\pgfpathlineto{\pgfqpoint{5.087997in}{4.046864in}}%
\pgfpathlineto{\pgfqpoint{5.121074in}{4.036911in}}%
\pgfpathlineto{\pgfqpoint{5.253385in}{3.994273in}}%
\pgfpathlineto{\pgfqpoint{5.286463in}{3.987356in}}%
\pgfpathlineto{\pgfqpoint{5.316533in}{3.983249in}}%
\pgfpathlineto{\pgfqpoint{5.346604in}{3.981391in}}%
\pgfpathlineto{\pgfqpoint{5.376675in}{3.981850in}}%
\pgfpathlineto{\pgfqpoint{5.406745in}{3.984601in}}%
\pgfpathlineto{\pgfqpoint{5.438319in}{3.989828in}}%
\pgfpathlineto{\pgfqpoint{5.471397in}{3.997603in}}%
\pgfpathlineto{\pgfqpoint{5.510489in}{4.009219in}}%
\pgfpathlineto{\pgfqpoint{5.533042in}{4.016739in}}%
\pgfpathlineto{\pgfqpoint{5.534545in}{2.376000in}}%
\pgfpathlineto{\pgfqpoint{5.534545in}{2.376000in}}%
\pgfusepath{stroke}%
\end{pgfscope}%
\begin{pgfscope}%
\pgfpathrectangle{\pgfqpoint{0.800000in}{0.528000in}}{\pgfqpoint{4.960000in}{3.696000in}}%
\pgfusepath{clip}%
\pgfsetrectcap%
\pgfsetroundjoin%
\pgfsetlinewidth{1.505625pt}%
\definecolor{currentstroke}{rgb}{1.000000,0.498039,0.054902}%
\pgfsetstrokecolor{currentstroke}%
\pgfsetdash{}{0pt}%
\pgfpathmoveto{\pgfqpoint{1.025455in}{3.636187in}}%
\pgfpathlineto{\pgfqpoint{2.527482in}{3.636173in}}%
\pgfpathlineto{\pgfqpoint{2.528986in}{2.376052in}}%
\pgfpathlineto{\pgfqpoint{2.706403in}{2.376050in}}%
\pgfpathlineto{\pgfqpoint{2.707906in}{3.909394in}}%
\pgfpathlineto{\pgfqpoint{2.725949in}{3.948434in}}%
\pgfpathlineto{\pgfqpoint{2.740984in}{3.975906in}}%
\pgfpathlineto{\pgfqpoint{2.756019in}{3.998889in}}%
\pgfpathlineto{\pgfqpoint{2.769551in}{4.015904in}}%
\pgfpathlineto{\pgfqpoint{2.783083in}{4.029623in}}%
\pgfpathlineto{\pgfqpoint{2.796615in}{4.040238in}}%
\pgfpathlineto{\pgfqpoint{2.810146in}{4.047955in}}%
\pgfpathlineto{\pgfqpoint{2.823678in}{4.052987in}}%
\pgfpathlineto{\pgfqpoint{2.837210in}{4.055552in}}%
\pgfpathlineto{\pgfqpoint{2.850742in}{4.055868in}}%
\pgfpathlineto{\pgfqpoint{2.865777in}{4.053842in}}%
\pgfpathlineto{\pgfqpoint{2.880812in}{4.049590in}}%
\pgfpathlineto{\pgfqpoint{2.897351in}{4.042667in}}%
\pgfpathlineto{\pgfqpoint{2.916897in}{4.031907in}}%
\pgfpathlineto{\pgfqpoint{2.939450in}{4.016660in}}%
\pgfpathlineto{\pgfqpoint{2.965010in}{3.996567in}}%
\pgfpathlineto{\pgfqpoint{2.998088in}{3.967494in}}%
\pgfpathlineto{\pgfqpoint{3.046201in}{3.921848in}}%
\pgfpathlineto{\pgfqpoint{3.169490in}{3.803847in}}%
\pgfpathlineto{\pgfqpoint{3.232639in}{3.747184in}}%
\pgfpathlineto{\pgfqpoint{3.322851in}{3.669663in}}%
\pgfpathlineto{\pgfqpoint{3.398027in}{3.603944in}}%
\pgfpathlineto{\pgfqpoint{3.443133in}{3.561527in}}%
\pgfpathlineto{\pgfqpoint{3.479218in}{3.524661in}}%
\pgfpathlineto{\pgfqpoint{3.512296in}{3.487647in}}%
\pgfpathlineto{\pgfqpoint{3.542366in}{3.450527in}}%
\pgfpathlineto{\pgfqpoint{3.569430in}{3.413600in}}%
\pgfpathlineto{\pgfqpoint{3.594990in}{3.375022in}}%
\pgfpathlineto{\pgfqpoint{3.619046in}{3.334815in}}%
\pgfpathlineto{\pgfqpoint{3.643103in}{3.290175in}}%
\pgfpathlineto{\pgfqpoint{3.665656in}{3.243644in}}%
\pgfpathlineto{\pgfqpoint{3.688209in}{3.191879in}}%
\pgfpathlineto{\pgfqpoint{3.709258in}{3.138155in}}%
\pgfpathlineto{\pgfqpoint{3.730308in}{3.078475in}}%
\pgfpathlineto{\pgfqpoint{3.751357in}{3.012044in}}%
\pgfpathlineto{\pgfqpoint{3.772407in}{2.937980in}}%
\pgfpathlineto{\pgfqpoint{3.793456in}{2.855323in}}%
\pgfpathlineto{\pgfqpoint{3.814505in}{2.763040in}}%
\pgfpathlineto{\pgfqpoint{3.835555in}{2.660055in}}%
\pgfpathlineto{\pgfqpoint{3.856604in}{2.545287in}}%
\pgfpathlineto{\pgfqpoint{3.877654in}{2.417724in}}%
\pgfpathlineto{\pgfqpoint{3.898703in}{2.276541in}}%
\pgfpathlineto{\pgfqpoint{3.921256in}{2.109637in}}%
\pgfpathlineto{\pgfqpoint{3.945313in}{1.914059in}}%
\pgfpathlineto{\pgfqpoint{3.972376in}{1.674897in}}%
\pgfpathlineto{\pgfqpoint{4.053567in}{0.940464in}}%
\pgfpathlineto{\pgfqpoint{4.067099in}{0.844472in}}%
\pgfpathlineto{\pgfqpoint{4.077623in}{0.782867in}}%
\pgfpathlineto{\pgfqpoint{4.086645in}{0.741416in}}%
\pgfpathlineto{\pgfqpoint{4.094162in}{0.716260in}}%
\pgfpathlineto{\pgfqpoint{4.100176in}{0.703048in}}%
\pgfpathlineto{\pgfqpoint{4.104687in}{0.697534in}}%
\pgfpathlineto{\pgfqpoint{4.107694in}{0.696078in}}%
\pgfpathlineto{\pgfqpoint{4.110701in}{0.696476in}}%
\pgfpathlineto{\pgfqpoint{4.113708in}{0.698793in}}%
\pgfpathlineto{\pgfqpoint{4.118219in}{0.706008in}}%
\pgfpathlineto{\pgfqpoint{4.122729in}{0.717892in}}%
\pgfpathlineto{\pgfqpoint{4.128744in}{0.741336in}}%
\pgfpathlineto{\pgfqpoint{4.134758in}{0.773847in}}%
\pgfpathlineto{\pgfqpoint{4.142275in}{0.827773in}}%
\pgfpathlineto{\pgfqpoint{4.151296in}{0.912655in}}%
\pgfpathlineto{\pgfqpoint{4.160318in}{1.020001in}}%
\pgfpathlineto{\pgfqpoint{4.170842in}{1.173548in}}%
\pgfpathlineto{\pgfqpoint{4.182871in}{1.384861in}}%
\pgfpathlineto{\pgfqpoint{4.196402in}{1.663635in}}%
\pgfpathlineto{\pgfqpoint{4.214445in}{2.087727in}}%
\pgfpathlineto{\pgfqpoint{4.268572in}{3.403294in}}%
\pgfpathlineto{\pgfqpoint{4.282104in}{3.658674in}}%
\pgfpathlineto{\pgfqpoint{4.292628in}{3.818400in}}%
\pgfpathlineto{\pgfqpoint{4.301650in}{3.924497in}}%
\pgfpathlineto{\pgfqpoint{4.309167in}{3.989665in}}%
\pgfpathlineto{\pgfqpoint{4.315181in}{4.026039in}}%
\pgfpathlineto{\pgfqpoint{4.319692in}{4.043978in}}%
\pgfpathlineto{\pgfqpoint{4.324203in}{4.053865in}}%
\pgfpathlineto{\pgfqpoint{4.327210in}{4.055984in}}%
\pgfpathlineto{\pgfqpoint{4.330217in}{4.054538in}}%
\pgfpathlineto{\pgfqpoint{4.333224in}{4.049544in}}%
\pgfpathlineto{\pgfqpoint{4.337734in}{4.035463in}}%
\pgfpathlineto{\pgfqpoint{4.342245in}{4.013580in}}%
\pgfpathlineto{\pgfqpoint{4.348259in}{3.972530in}}%
\pgfpathlineto{\pgfqpoint{4.355777in}{3.902769in}}%
\pgfpathlineto{\pgfqpoint{4.364798in}{3.793399in}}%
\pgfpathlineto{\pgfqpoint{4.375323in}{3.633163in}}%
\pgfpathlineto{\pgfqpoint{4.387351in}{3.411854in}}%
\pgfpathlineto{\pgfqpoint{4.402386in}{3.087761in}}%
\pgfpathlineto{\pgfqpoint{4.421932in}{2.610378in}}%
\pgfpathlineto{\pgfqpoint{4.473052in}{1.331332in}}%
\pgfpathlineto{\pgfqpoint{4.486584in}{1.061757in}}%
\pgfpathlineto{\pgfqpoint{4.497109in}{0.894498in}}%
\pgfpathlineto{\pgfqpoint{4.504626in}{0.802958in}}%
\pgfpathlineto{\pgfqpoint{4.512144in}{0.738030in}}%
\pgfpathlineto{\pgfqpoint{4.516655in}{0.712971in}}%
\pgfpathlineto{\pgfqpoint{4.521165in}{0.698973in}}%
\pgfpathlineto{\pgfqpoint{4.524172in}{0.696012in}}%
\pgfpathlineto{\pgfqpoint{4.525676in}{0.696489in}}%
\pgfpathlineto{\pgfqpoint{4.528683in}{0.701422in}}%
\pgfpathlineto{\pgfqpoint{4.531690in}{0.711737in}}%
\pgfpathlineto{\pgfqpoint{4.536201in}{0.737461in}}%
\pgfpathlineto{\pgfqpoint{4.542215in}{0.791138in}}%
\pgfpathlineto{\pgfqpoint{4.548229in}{0.867039in}}%
\pgfpathlineto{\pgfqpoint{4.555746in}{0.992719in}}%
\pgfpathlineto{\pgfqpoint{4.564768in}{1.186674in}}%
\pgfpathlineto{\pgfqpoint{4.575292in}{1.466498in}}%
\pgfpathlineto{\pgfqpoint{4.588824in}{1.893461in}}%
\pgfpathlineto{\pgfqpoint{4.638441in}{3.541224in}}%
\pgfpathlineto{\pgfqpoint{4.648965in}{3.782974in}}%
\pgfpathlineto{\pgfqpoint{4.656483in}{3.912174in}}%
\pgfpathlineto{\pgfqpoint{4.662497in}{3.986823in}}%
\pgfpathlineto{\pgfqpoint{4.668511in}{4.034705in}}%
\pgfpathlineto{\pgfqpoint{4.673022in}{4.052653in}}%
\pgfpathlineto{\pgfqpoint{4.676029in}{4.055999in}}%
\pgfpathlineto{\pgfqpoint{4.677533in}{4.055088in}}%
\pgfpathlineto{\pgfqpoint{4.680540in}{4.048116in}}%
\pgfpathlineto{\pgfqpoint{4.683547in}{4.034323in}}%
\pgfpathlineto{\pgfqpoint{4.688057in}{4.001024in}}%
\pgfpathlineto{\pgfqpoint{4.694071in}{3.933700in}}%
\pgfpathlineto{\pgfqpoint{4.701589in}{3.814479in}}%
\pgfpathlineto{\pgfqpoint{4.710610in}{3.624237in}}%
\pgfpathlineto{\pgfqpoint{4.721135in}{3.346027in}}%
\pgfpathlineto{\pgfqpoint{4.734667in}{2.919749in}}%
\pgfpathlineto{\pgfqpoint{4.758723in}{2.069151in}}%
\pgfpathlineto{\pgfqpoint{4.778269in}{1.414080in}}%
\pgfpathlineto{\pgfqpoint{4.790297in}{1.083810in}}%
\pgfpathlineto{\pgfqpoint{4.799319in}{0.892833in}}%
\pgfpathlineto{\pgfqpoint{4.806836in}{0.779290in}}%
\pgfpathlineto{\pgfqpoint{4.812850in}{0.722059in}}%
\pgfpathlineto{\pgfqpoint{4.817361in}{0.700171in}}%
\pgfpathlineto{\pgfqpoint{4.820368in}{0.696000in}}%
\pgfpathlineto{\pgfqpoint{4.821872in}{0.697103in}}%
\pgfpathlineto{\pgfqpoint{4.824879in}{0.705767in}}%
\pgfpathlineto{\pgfqpoint{4.827886in}{0.723112in}}%
\pgfpathlineto{\pgfqpoint{4.832396in}{0.765493in}}%
\pgfpathlineto{\pgfqpoint{4.838410in}{0.852389in}}%
\pgfpathlineto{\pgfqpoint{4.844425in}{0.973158in}}%
\pgfpathlineto{\pgfqpoint{4.851942in}{1.168959in}}%
\pgfpathlineto{\pgfqpoint{4.860963in}{1.462170in}}%
\pgfpathlineto{\pgfqpoint{4.872992in}{1.928924in}}%
\pgfpathlineto{\pgfqpoint{4.912083in}{3.511510in}}%
\pgfpathlineto{\pgfqpoint{4.921105in}{3.768252in}}%
\pgfpathlineto{\pgfqpoint{4.928622in}{3.924241in}}%
\pgfpathlineto{\pgfqpoint{4.934636in}{4.006761in}}%
\pgfpathlineto{\pgfqpoint{4.939147in}{4.042703in}}%
\pgfpathlineto{\pgfqpoint{4.942154in}{4.054055in}}%
\pgfpathlineto{\pgfqpoint{4.943658in}{4.055928in}}%
\pgfpathlineto{\pgfqpoint{4.945161in}{4.055265in}}%
\pgfpathlineto{\pgfqpoint{4.948168in}{4.046355in}}%
\pgfpathlineto{\pgfqpoint{4.951175in}{4.027407in}}%
\pgfpathlineto{\pgfqpoint{4.955686in}{3.980490in}}%
\pgfpathlineto{\pgfqpoint{4.961700in}{3.884594in}}%
\pgfpathlineto{\pgfqpoint{4.969218in}{3.714693in}}%
\pgfpathlineto{\pgfqpoint{4.978239in}{3.446111in}}%
\pgfpathlineto{\pgfqpoint{4.988764in}{3.061621in}}%
\pgfpathlineto{\pgfqpoint{5.006806in}{2.300602in}}%
\pgfpathlineto{\pgfqpoint{5.027855in}{1.433821in}}%
\pgfpathlineto{\pgfqpoint{5.038380in}{1.086085in}}%
\pgfpathlineto{\pgfqpoint{5.045898in}{0.896014in}}%
\pgfpathlineto{\pgfqpoint{5.051912in}{0.786328in}}%
\pgfpathlineto{\pgfqpoint{5.056422in}{0.731451in}}%
\pgfpathlineto{\pgfqpoint{5.060933in}{0.701555in}}%
\pgfpathlineto{\pgfqpoint{5.063940in}{0.696003in}}%
\pgfpathlineto{\pgfqpoint{5.065444in}{0.697623in}}%
\pgfpathlineto{\pgfqpoint{5.068451in}{0.709739in}}%
\pgfpathlineto{\pgfqpoint{5.071458in}{0.733742in}}%
\pgfpathlineto{\pgfqpoint{5.075968in}{0.791978in}}%
\pgfpathlineto{\pgfqpoint{5.081982in}{0.910317in}}%
\pgfpathlineto{\pgfqpoint{5.089500in}{1.120013in}}%
\pgfpathlineto{\pgfqpoint{5.098521in}{1.451111in}}%
\pgfpathlineto{\pgfqpoint{5.110550in}{1.991999in}}%
\pgfpathlineto{\pgfqpoint{5.140620in}{3.408284in}}%
\pgfpathlineto{\pgfqpoint{5.149641in}{3.723140in}}%
\pgfpathlineto{\pgfqpoint{5.157159in}{3.912274in}}%
\pgfpathlineto{\pgfqpoint{5.163173in}{4.008672in}}%
\pgfpathlineto{\pgfqpoint{5.167684in}{4.046911in}}%
\pgfpathlineto{\pgfqpoint{5.170691in}{4.055813in}}%
\pgfpathlineto{\pgfqpoint{5.172194in}{4.055263in}}%
\pgfpathlineto{\pgfqpoint{5.173698in}{4.051385in}}%
\pgfpathlineto{\pgfqpoint{5.176705in}{4.033696in}}%
\pgfpathlineto{\pgfqpoint{5.181216in}{3.982708in}}%
\pgfpathlineto{\pgfqpoint{5.185726in}{3.903301in}}%
\pgfpathlineto{\pgfqpoint{5.191740in}{3.755842in}}%
\pgfpathlineto{\pgfqpoint{5.199258in}{3.511538in}}%
\pgfpathlineto{\pgfqpoint{5.209783in}{3.079443in}}%
\pgfpathlineto{\pgfqpoint{5.226322in}{2.280175in}}%
\pgfpathlineto{\pgfqpoint{5.244364in}{1.432444in}}%
\pgfpathlineto{\pgfqpoint{5.253385in}{1.092454in}}%
\pgfpathlineto{\pgfqpoint{5.260903in}{0.879272in}}%
\pgfpathlineto{\pgfqpoint{5.266917in}{0.764247in}}%
\pgfpathlineto{\pgfqpoint{5.271427in}{0.713866in}}%
\pgfpathlineto{\pgfqpoint{5.274435in}{0.698283in}}%
\pgfpathlineto{\pgfqpoint{5.275938in}{0.696021in}}%
\pgfpathlineto{\pgfqpoint{5.277442in}{0.697486in}}%
\pgfpathlineto{\pgfqpoint{5.280449in}{0.711662in}}%
\pgfpathlineto{\pgfqpoint{5.283456in}{0.740861in}}%
\pgfpathlineto{\pgfqpoint{5.287966in}{0.812593in}}%
\pgfpathlineto{\pgfqpoint{5.293980in}{0.958760in}}%
\pgfpathlineto{\pgfqpoint{5.301498in}{1.216287in}}%
\pgfpathlineto{\pgfqpoint{5.310519in}{1.616686in}}%
\pgfpathlineto{\pgfqpoint{5.324051in}{2.333133in}}%
\pgfpathlineto{\pgfqpoint{5.342093in}{3.282937in}}%
\pgfpathlineto{\pgfqpoint{5.351115in}{3.658726in}}%
\pgfpathlineto{\pgfqpoint{5.358632in}{3.886228in}}%
\pgfpathlineto{\pgfqpoint{5.364646in}{4.001755in}}%
\pgfpathlineto{\pgfqpoint{5.369157in}{4.046566in}}%
\pgfpathlineto{\pgfqpoint{5.372164in}{4.055960in}}%
\pgfpathlineto{\pgfqpoint{5.373668in}{4.054477in}}%
\pgfpathlineto{\pgfqpoint{5.376675in}{4.039197in}}%
\pgfpathlineto{\pgfqpoint{5.379682in}{4.007669in}}%
\pgfpathlineto{\pgfqpoint{5.384192in}{3.930706in}}%
\pgfpathlineto{\pgfqpoint{5.390206in}{3.775666in}}%
\pgfpathlineto{\pgfqpoint{5.397724in}{3.506611in}}%
\pgfpathlineto{\pgfqpoint{5.406745in}{3.095267in}}%
\pgfpathlineto{\pgfqpoint{5.421781in}{2.288873in}}%
\pgfpathlineto{\pgfqpoint{5.438319in}{1.426924in}}%
\pgfpathlineto{\pgfqpoint{5.447341in}{1.057085in}}%
\pgfpathlineto{\pgfqpoint{5.454858in}{0.837734in}}%
\pgfpathlineto{\pgfqpoint{5.459369in}{0.752504in}}%
\pgfpathlineto{\pgfqpoint{5.463879in}{0.705365in}}%
\pgfpathlineto{\pgfqpoint{5.466887in}{0.696004in}}%
\pgfpathlineto{\pgfqpoint{5.468390in}{0.698070in}}%
\pgfpathlineto{\pgfqpoint{5.471397in}{0.715792in}}%
\pgfpathlineto{\pgfqpoint{5.474404in}{0.751612in}}%
\pgfpathlineto{\pgfqpoint{5.478915in}{0.838781in}}%
\pgfpathlineto{\pgfqpoint{5.484929in}{1.014716in}}%
\pgfpathlineto{\pgfqpoint{5.492447in}{1.320700in}}%
\pgfpathlineto{\pgfqpoint{5.501468in}{1.786908in}}%
\pgfpathlineto{\pgfqpoint{5.533042in}{3.553567in}}%
\pgfpathlineto{\pgfqpoint{5.534545in}{2.376000in}}%
\pgfpathlineto{\pgfqpoint{5.534545in}{2.376000in}}%
\pgfusepath{stroke}%
\end{pgfscope}%
\begin{pgfscope}%
\pgfsetrectcap%
\pgfsetmiterjoin%
\pgfsetlinewidth{0.803000pt}%
\definecolor{currentstroke}{rgb}{0.000000,0.000000,0.000000}%
\pgfsetstrokecolor{currentstroke}%
\pgfsetdash{}{0pt}%
\pgfpathmoveto{\pgfqpoint{0.800000in}{0.528000in}}%
\pgfpathlineto{\pgfqpoint{0.800000in}{4.224000in}}%
\pgfusepath{stroke}%
\end{pgfscope}%
\begin{pgfscope}%
\pgfsetrectcap%
\pgfsetmiterjoin%
\pgfsetlinewidth{0.803000pt}%
\definecolor{currentstroke}{rgb}{0.000000,0.000000,0.000000}%
\pgfsetstrokecolor{currentstroke}%
\pgfsetdash{}{0pt}%
\pgfpathmoveto{\pgfqpoint{5.760000in}{0.528000in}}%
\pgfpathlineto{\pgfqpoint{5.760000in}{4.224000in}}%
\pgfusepath{stroke}%
\end{pgfscope}%
\begin{pgfscope}%
\pgfsetrectcap%
\pgfsetmiterjoin%
\pgfsetlinewidth{0.803000pt}%
\definecolor{currentstroke}{rgb}{0.000000,0.000000,0.000000}%
\pgfsetstrokecolor{currentstroke}%
\pgfsetdash{}{0pt}%
\pgfpathmoveto{\pgfqpoint{0.800000in}{0.528000in}}%
\pgfpathlineto{\pgfqpoint{5.760000in}{0.528000in}}%
\pgfusepath{stroke}%
\end{pgfscope}%
\begin{pgfscope}%
\pgfsetrectcap%
\pgfsetmiterjoin%
\pgfsetlinewidth{0.803000pt}%
\definecolor{currentstroke}{rgb}{0.000000,0.000000,0.000000}%
\pgfsetstrokecolor{currentstroke}%
\pgfsetdash{}{0pt}%
\pgfpathmoveto{\pgfqpoint{0.800000in}{4.224000in}}%
\pgfpathlineto{\pgfqpoint{5.760000in}{4.224000in}}%
\pgfusepath{stroke}%
\end{pgfscope}%
\begin{pgfscope}%
\pgfsetbuttcap%
\pgfsetmiterjoin%
\definecolor{currentfill}{rgb}{1.000000,1.000000,1.000000}%
\pgfsetfillcolor{currentfill}%
\pgfsetfillopacity{0.800000}%
\pgfsetlinewidth{1.003750pt}%
\definecolor{currentstroke}{rgb}{0.800000,0.800000,0.800000}%
\pgfsetstrokecolor{currentstroke}%
\pgfsetstrokeopacity{0.800000}%
\pgfsetdash{}{0pt}%
\pgfpathmoveto{\pgfqpoint{0.897222in}{0.597444in}}%
\pgfpathlineto{\pgfqpoint{2.740115in}{0.597444in}}%
\pgfpathquadraticcurveto{\pgfqpoint{2.767893in}{0.597444in}}{\pgfqpoint{2.767893in}{0.625222in}}%
\pgfpathlineto{\pgfqpoint{2.767893in}{1.015114in}}%
\pgfpathquadraticcurveto{\pgfqpoint{2.767893in}{1.042892in}}{\pgfqpoint{2.740115in}{1.042892in}}%
\pgfpathlineto{\pgfqpoint{0.897222in}{1.042892in}}%
\pgfpathquadraticcurveto{\pgfqpoint{0.869444in}{1.042892in}}{\pgfqpoint{0.869444in}{1.015114in}}%
\pgfpathlineto{\pgfqpoint{0.869444in}{0.625222in}}%
\pgfpathquadraticcurveto{\pgfqpoint{0.869444in}{0.597444in}}{\pgfqpoint{0.897222in}{0.597444in}}%
\pgfpathlineto{\pgfqpoint{0.897222in}{0.597444in}}%
\pgfpathclose%
\pgfusepath{stroke,fill}%
\end{pgfscope}%
\begin{pgfscope}%
\pgfsetrectcap%
\pgfsetroundjoin%
\pgfsetlinewidth{1.505625pt}%
\definecolor{currentstroke}{rgb}{0.121569,0.466667,0.705882}%
\pgfsetstrokecolor{currentstroke}%
\pgfsetdash{}{0pt}%
\pgfpathmoveto{\pgfqpoint{0.925000in}{0.933748in}}%
\pgfpathlineto{\pgfqpoint{1.063889in}{0.933748in}}%
\pgfpathlineto{\pgfqpoint{1.202778in}{0.933748in}}%
\pgfusepath{stroke}%
\end{pgfscope}%
\begin{pgfscope}%
\definecolor{textcolor}{rgb}{0.000000,0.000000,0.000000}%
\pgfsetstrokecolor{textcolor}%
\pgfsetfillcolor{textcolor}%
\pgftext[x=1.313889in,y=0.885137in,left,base]{\color{textcolor}\rmfamily\fontsize{10.000000}{12.000000}\selectfont \(\displaystyle \Delta P\) - stable scenario}%
\end{pgfscope}%
\begin{pgfscope}%
\pgfsetrectcap%
\pgfsetroundjoin%
\pgfsetlinewidth{1.505625pt}%
\definecolor{currentstroke}{rgb}{1.000000,0.498039,0.054902}%
\pgfsetstrokecolor{currentstroke}%
\pgfsetdash{}{0pt}%
\pgfpathmoveto{\pgfqpoint{0.925000in}{0.731857in}}%
\pgfpathlineto{\pgfqpoint{1.063889in}{0.731857in}}%
\pgfpathlineto{\pgfqpoint{1.202778in}{0.731857in}}%
\pgfusepath{stroke}%
\end{pgfscope}%
\begin{pgfscope}%
\definecolor{textcolor}{rgb}{0.000000,0.000000,0.000000}%
\pgfsetstrokecolor{textcolor}%
\pgfsetfillcolor{textcolor}%
\pgftext[x=1.313889in,y=0.683246in,left,base]{\color{textcolor}\rmfamily\fontsize{10.000000}{12.000000}\selectfont \(\displaystyle \Delta P\) - unstable scenario}%
\end{pgfscope}%
\end{pgfpicture}%
\makeatother%
\endgroup%


\section{Fault 2}
\label{app:fault2}

%% Creator: Matplotlib, PGF backend
%%
%% To include the figure in your LaTeX document, write
%%   \input{<filename>.pgf}
%%
%% Make sure the required packages are loaded in your preamble
%%   \usepackage{pgf}
%%
%% Also ensure that all the required font packages are loaded; for instance,
%% the lmodern package is sometimes necessary when using math font.
%%   \usepackage{lmodern}
%%
%% Figures using additional raster images can only be included by \input if
%% they are in the same directory as the main LaTeX file. For loading figures
%% from other directories you can use the `import` package
%%   \usepackage{import}
%%
%% and then include the figures with
%%   \import{<path to file>}{<filename>.pgf}
%%
%% Matplotlib used the following preamble
%%   
%%   \usepackage{fontspec}
%%   \setmainfont{Charter.ttc}[Path=\detokenize{/System/Library/Fonts/Supplemental/}]
%%   \setsansfont{DejaVuSans.ttf}[Path=\detokenize{/opt/homebrew/lib/python3.10/site-packages/matplotlib/mpl-data/fonts/ttf/}]
%%   \setmonofont{DejaVuSansMono.ttf}[Path=\detokenize{/opt/homebrew/lib/python3.10/site-packages/matplotlib/mpl-data/fonts/ttf/}]
%%   \makeatletter\@ifpackageloaded{underscore}{}{\usepackage[strings]{underscore}}\makeatother
%%
\begingroup%
\makeatletter%
\begin{pgfpicture}%
\pgfpathrectangle{\pgfpointorigin}{\pgfqpoint{6.000000in}{8.000000in}}%
\pgfusepath{use as bounding box, clip}%
\begin{pgfscope}%
\pgfsetbuttcap%
\pgfsetmiterjoin%
\definecolor{currentfill}{rgb}{1.000000,1.000000,1.000000}%
\pgfsetfillcolor{currentfill}%
\pgfsetlinewidth{0.000000pt}%
\definecolor{currentstroke}{rgb}{1.000000,1.000000,1.000000}%
\pgfsetstrokecolor{currentstroke}%
\pgfsetdash{}{0pt}%
\pgfpathmoveto{\pgfqpoint{0.000000in}{0.000000in}}%
\pgfpathlineto{\pgfqpoint{6.000000in}{0.000000in}}%
\pgfpathlineto{\pgfqpoint{6.000000in}{8.000000in}}%
\pgfpathlineto{\pgfqpoint{0.000000in}{8.000000in}}%
\pgfpathlineto{\pgfqpoint{0.000000in}{0.000000in}}%
\pgfpathclose%
\pgfusepath{fill}%
\end{pgfscope}%
\begin{pgfscope}%
\pgfsetbuttcap%
\pgfsetmiterjoin%
\definecolor{currentfill}{rgb}{1.000000,1.000000,1.000000}%
\pgfsetfillcolor{currentfill}%
\pgfsetlinewidth{0.000000pt}%
\definecolor{currentstroke}{rgb}{0.000000,0.000000,0.000000}%
\pgfsetstrokecolor{currentstroke}%
\pgfsetstrokeopacity{0.000000}%
\pgfsetdash{}{0pt}%
\pgfpathmoveto{\pgfqpoint{0.750000in}{3.960000in}}%
\pgfpathlineto{\pgfqpoint{5.400000in}{3.960000in}}%
\pgfpathlineto{\pgfqpoint{5.400000in}{7.040000in}}%
\pgfpathlineto{\pgfqpoint{0.750000in}{7.040000in}}%
\pgfpathlineto{\pgfqpoint{0.750000in}{3.960000in}}%
\pgfpathclose%
\pgfusepath{fill}%
\end{pgfscope}%
\begin{pgfscope}%
\pgfpathrectangle{\pgfqpoint{0.750000in}{3.960000in}}{\pgfqpoint{4.650000in}{3.080000in}}%
\pgfusepath{clip}%
\pgfsetbuttcap%
\pgfsetroundjoin%
\definecolor{currentfill}{rgb}{0.900000,0.900000,0.900000}%
\pgfsetfillcolor{currentfill}%
\pgfsetlinewidth{1.003750pt}%
\definecolor{currentstroke}{rgb}{0.500000,0.500000,0.500000}%
\pgfsetstrokecolor{currentstroke}%
\pgfsetdash{}{0pt}%
\pgfsys@defobject{currentmarker}{\pgfqpoint{2.208033in}{5.782230in}}{\pgfqpoint{3.057289in}{6.405701in}}{%
\pgfpathmoveto{\pgfqpoint{2.208033in}{6.405701in}}%
\pgfpathlineto{\pgfqpoint{2.208033in}{5.782230in}}%
\pgfpathlineto{\pgfqpoint{2.225365in}{5.796260in}}%
\pgfpathlineto{\pgfqpoint{2.242697in}{5.810038in}}%
\pgfpathlineto{\pgfqpoint{2.260029in}{5.823563in}}%
\pgfpathlineto{\pgfqpoint{2.277360in}{5.836832in}}%
\pgfpathlineto{\pgfqpoint{2.294692in}{5.849843in}}%
\pgfpathlineto{\pgfqpoint{2.312024in}{5.862596in}}%
\pgfpathlineto{\pgfqpoint{2.329356in}{5.875087in}}%
\pgfpathlineto{\pgfqpoint{2.346687in}{5.887316in}}%
\pgfpathlineto{\pgfqpoint{2.364019in}{5.899281in}}%
\pgfpathlineto{\pgfqpoint{2.381351in}{5.910980in}}%
\pgfpathlineto{\pgfqpoint{2.398683in}{5.922411in}}%
\pgfpathlineto{\pgfqpoint{2.416014in}{5.933573in}}%
\pgfpathlineto{\pgfqpoint{2.433346in}{5.944465in}}%
\pgfpathlineto{\pgfqpoint{2.450678in}{5.955085in}}%
\pgfpathlineto{\pgfqpoint{2.468010in}{5.965431in}}%
\pgfpathlineto{\pgfqpoint{2.485341in}{5.975502in}}%
\pgfpathlineto{\pgfqpoint{2.502673in}{5.985296in}}%
\pgfpathlineto{\pgfqpoint{2.520005in}{5.994814in}}%
\pgfpathlineto{\pgfqpoint{2.537336in}{6.004052in}}%
\pgfpathlineto{\pgfqpoint{2.554668in}{6.013009in}}%
\pgfpathlineto{\pgfqpoint{2.572000in}{6.021686in}}%
\pgfpathlineto{\pgfqpoint{2.589332in}{6.030079in}}%
\pgfpathlineto{\pgfqpoint{2.606663in}{6.038189in}}%
\pgfpathlineto{\pgfqpoint{2.623995in}{6.046014in}}%
\pgfpathlineto{\pgfqpoint{2.641327in}{6.053553in}}%
\pgfpathlineto{\pgfqpoint{2.658659in}{6.060804in}}%
\pgfpathlineto{\pgfqpoint{2.675990in}{6.067768in}}%
\pgfpathlineto{\pgfqpoint{2.693322in}{6.074443in}}%
\pgfpathlineto{\pgfqpoint{2.710654in}{6.080828in}}%
\pgfpathlineto{\pgfqpoint{2.727986in}{6.086922in}}%
\pgfpathlineto{\pgfqpoint{2.745317in}{6.092724in}}%
\pgfpathlineto{\pgfqpoint{2.762649in}{6.098234in}}%
\pgfpathlineto{\pgfqpoint{2.779981in}{6.103451in}}%
\pgfpathlineto{\pgfqpoint{2.797313in}{6.108374in}}%
\pgfpathlineto{\pgfqpoint{2.814644in}{6.113002in}}%
\pgfpathlineto{\pgfqpoint{2.831976in}{6.117335in}}%
\pgfpathlineto{\pgfqpoint{2.849308in}{6.121373in}}%
\pgfpathlineto{\pgfqpoint{2.866640in}{6.125114in}}%
\pgfpathlineto{\pgfqpoint{2.883971in}{6.128558in}}%
\pgfpathlineto{\pgfqpoint{2.901303in}{6.131704in}}%
\pgfpathlineto{\pgfqpoint{2.918635in}{6.134553in}}%
\pgfpathlineto{\pgfqpoint{2.935967in}{6.137104in}}%
\pgfpathlineto{\pgfqpoint{2.953298in}{6.139357in}}%
\pgfpathlineto{\pgfqpoint{2.970630in}{6.141310in}}%
\pgfpathlineto{\pgfqpoint{2.987962in}{6.142965in}}%
\pgfpathlineto{\pgfqpoint{3.005294in}{6.144320in}}%
\pgfpathlineto{\pgfqpoint{3.022625in}{6.145376in}}%
\pgfpathlineto{\pgfqpoint{3.039957in}{6.146132in}}%
\pgfpathlineto{\pgfqpoint{3.057289in}{6.146588in}}%
\pgfpathlineto{\pgfqpoint{3.057289in}{6.405701in}}%
\pgfpathlineto{\pgfqpoint{3.057289in}{6.405701in}}%
\pgfpathlineto{\pgfqpoint{3.039957in}{6.405701in}}%
\pgfpathlineto{\pgfqpoint{3.022625in}{6.405701in}}%
\pgfpathlineto{\pgfqpoint{3.005294in}{6.405701in}}%
\pgfpathlineto{\pgfqpoint{2.987962in}{6.405701in}}%
\pgfpathlineto{\pgfqpoint{2.970630in}{6.405701in}}%
\pgfpathlineto{\pgfqpoint{2.953298in}{6.405701in}}%
\pgfpathlineto{\pgfqpoint{2.935967in}{6.405701in}}%
\pgfpathlineto{\pgfqpoint{2.918635in}{6.405701in}}%
\pgfpathlineto{\pgfqpoint{2.901303in}{6.405701in}}%
\pgfpathlineto{\pgfqpoint{2.883971in}{6.405701in}}%
\pgfpathlineto{\pgfqpoint{2.866640in}{6.405701in}}%
\pgfpathlineto{\pgfqpoint{2.849308in}{6.405701in}}%
\pgfpathlineto{\pgfqpoint{2.831976in}{6.405701in}}%
\pgfpathlineto{\pgfqpoint{2.814644in}{6.405701in}}%
\pgfpathlineto{\pgfqpoint{2.797313in}{6.405701in}}%
\pgfpathlineto{\pgfqpoint{2.779981in}{6.405701in}}%
\pgfpathlineto{\pgfqpoint{2.762649in}{6.405701in}}%
\pgfpathlineto{\pgfqpoint{2.745317in}{6.405701in}}%
\pgfpathlineto{\pgfqpoint{2.727986in}{6.405701in}}%
\pgfpathlineto{\pgfqpoint{2.710654in}{6.405701in}}%
\pgfpathlineto{\pgfqpoint{2.693322in}{6.405701in}}%
\pgfpathlineto{\pgfqpoint{2.675990in}{6.405701in}}%
\pgfpathlineto{\pgfqpoint{2.658659in}{6.405701in}}%
\pgfpathlineto{\pgfqpoint{2.641327in}{6.405701in}}%
\pgfpathlineto{\pgfqpoint{2.623995in}{6.405701in}}%
\pgfpathlineto{\pgfqpoint{2.606663in}{6.405701in}}%
\pgfpathlineto{\pgfqpoint{2.589332in}{6.405701in}}%
\pgfpathlineto{\pgfqpoint{2.572000in}{6.405701in}}%
\pgfpathlineto{\pgfqpoint{2.554668in}{6.405701in}}%
\pgfpathlineto{\pgfqpoint{2.537336in}{6.405701in}}%
\pgfpathlineto{\pgfqpoint{2.520005in}{6.405701in}}%
\pgfpathlineto{\pgfqpoint{2.502673in}{6.405701in}}%
\pgfpathlineto{\pgfqpoint{2.485341in}{6.405701in}}%
\pgfpathlineto{\pgfqpoint{2.468010in}{6.405701in}}%
\pgfpathlineto{\pgfqpoint{2.450678in}{6.405701in}}%
\pgfpathlineto{\pgfqpoint{2.433346in}{6.405701in}}%
\pgfpathlineto{\pgfqpoint{2.416014in}{6.405701in}}%
\pgfpathlineto{\pgfqpoint{2.398683in}{6.405701in}}%
\pgfpathlineto{\pgfqpoint{2.381351in}{6.405701in}}%
\pgfpathlineto{\pgfqpoint{2.364019in}{6.405701in}}%
\pgfpathlineto{\pgfqpoint{2.346687in}{6.405701in}}%
\pgfpathlineto{\pgfqpoint{2.329356in}{6.405701in}}%
\pgfpathlineto{\pgfqpoint{2.312024in}{6.405701in}}%
\pgfpathlineto{\pgfqpoint{2.294692in}{6.405701in}}%
\pgfpathlineto{\pgfqpoint{2.277360in}{6.405701in}}%
\pgfpathlineto{\pgfqpoint{2.260029in}{6.405701in}}%
\pgfpathlineto{\pgfqpoint{2.242697in}{6.405701in}}%
\pgfpathlineto{\pgfqpoint{2.225365in}{6.405701in}}%
\pgfpathlineto{\pgfqpoint{2.208033in}{6.405701in}}%
\pgfpathlineto{\pgfqpoint{2.208033in}{6.405701in}}%
\pgfpathclose%
\pgfusepath{stroke,fill}%
}%
\begin{pgfscope}%
\pgfsys@transformshift{0.000000in}{0.000000in}%
\pgfsys@useobject{currentmarker}{}%
\end{pgfscope}%
\end{pgfscope}%
\begin{pgfscope}%
\pgfpathrectangle{\pgfqpoint{0.750000in}{3.960000in}}{\pgfqpoint{4.650000in}{3.080000in}}%
\pgfusepath{clip}%
\pgfsetbuttcap%
\pgfsetroundjoin%
\definecolor{currentfill}{rgb}{0.900000,0.900000,0.900000}%
\pgfsetfillcolor{currentfill}%
\pgfsetlinewidth{1.003750pt}%
\definecolor{currentstroke}{rgb}{0.500000,0.500000,0.500000}%
\pgfsetstrokecolor{currentstroke}%
\pgfsetdash{}{0pt}%
\pgfsys@defobject{currentmarker}{\pgfqpoint{3.057289in}{6.405625in}}{\pgfqpoint{3.941967in}{6.894841in}}{%
\pgfpathmoveto{\pgfqpoint{3.057289in}{6.405701in}}%
\pgfpathlineto{\pgfqpoint{3.057289in}{6.894631in}}%
\pgfpathlineto{\pgfqpoint{3.075343in}{6.894841in}}%
\pgfpathlineto{\pgfqpoint{3.093398in}{6.894614in}}%
\pgfpathlineto{\pgfqpoint{3.111453in}{6.893951in}}%
\pgfpathlineto{\pgfqpoint{3.129507in}{6.892851in}}%
\pgfpathlineto{\pgfqpoint{3.147562in}{6.891315in}}%
\pgfpathlineto{\pgfqpoint{3.165617in}{6.889343in}}%
\pgfpathlineto{\pgfqpoint{3.183671in}{6.886935in}}%
\pgfpathlineto{\pgfqpoint{3.201726in}{6.884091in}}%
\pgfpathlineto{\pgfqpoint{3.219781in}{6.880812in}}%
\pgfpathlineto{\pgfqpoint{3.237835in}{6.877099in}}%
\pgfpathlineto{\pgfqpoint{3.255890in}{6.872952in}}%
\pgfpathlineto{\pgfqpoint{3.273945in}{6.868371in}}%
\pgfpathlineto{\pgfqpoint{3.291999in}{6.863357in}}%
\pgfpathlineto{\pgfqpoint{3.310054in}{6.857912in}}%
\pgfpathlineto{\pgfqpoint{3.328109in}{6.852035in}}%
\pgfpathlineto{\pgfqpoint{3.346163in}{6.845728in}}%
\pgfpathlineto{\pgfqpoint{3.364218in}{6.838992in}}%
\pgfpathlineto{\pgfqpoint{3.382273in}{6.831827in}}%
\pgfpathlineto{\pgfqpoint{3.400327in}{6.824235in}}%
\pgfpathlineto{\pgfqpoint{3.418382in}{6.816217in}}%
\pgfpathlineto{\pgfqpoint{3.436436in}{6.807774in}}%
\pgfpathlineto{\pgfqpoint{3.454491in}{6.798907in}}%
\pgfpathlineto{\pgfqpoint{3.472546in}{6.789618in}}%
\pgfpathlineto{\pgfqpoint{3.490600in}{6.779908in}}%
\pgfpathlineto{\pgfqpoint{3.508655in}{6.769778in}}%
\pgfpathlineto{\pgfqpoint{3.526710in}{6.759230in}}%
\pgfpathlineto{\pgfqpoint{3.544764in}{6.748266in}}%
\pgfpathlineto{\pgfqpoint{3.562819in}{6.736887in}}%
\pgfpathlineto{\pgfqpoint{3.580874in}{6.725095in}}%
\pgfpathlineto{\pgfqpoint{3.598928in}{6.712891in}}%
\pgfpathlineto{\pgfqpoint{3.616983in}{6.700278in}}%
\pgfpathlineto{\pgfqpoint{3.635038in}{6.687256in}}%
\pgfpathlineto{\pgfqpoint{3.653092in}{6.673830in}}%
\pgfpathlineto{\pgfqpoint{3.671147in}{6.659999in}}%
\pgfpathlineto{\pgfqpoint{3.689202in}{6.645767in}}%
\pgfpathlineto{\pgfqpoint{3.707256in}{6.631135in}}%
\pgfpathlineto{\pgfqpoint{3.725311in}{6.616105in}}%
\pgfpathlineto{\pgfqpoint{3.743366in}{6.600681in}}%
\pgfpathlineto{\pgfqpoint{3.761420in}{6.584863in}}%
\pgfpathlineto{\pgfqpoint{3.779475in}{6.568655in}}%
\pgfpathlineto{\pgfqpoint{3.797529in}{6.552059in}}%
\pgfpathlineto{\pgfqpoint{3.815584in}{6.535077in}}%
\pgfpathlineto{\pgfqpoint{3.833639in}{6.517712in}}%
\pgfpathlineto{\pgfqpoint{3.851693in}{6.499967in}}%
\pgfpathlineto{\pgfqpoint{3.869748in}{6.481843in}}%
\pgfpathlineto{\pgfqpoint{3.887803in}{6.463344in}}%
\pgfpathlineto{\pgfqpoint{3.905857in}{6.444473in}}%
\pgfpathlineto{\pgfqpoint{3.923912in}{6.425232in}}%
\pgfpathlineto{\pgfqpoint{3.941967in}{6.405625in}}%
\pgfpathlineto{\pgfqpoint{3.941967in}{6.405701in}}%
\pgfpathlineto{\pgfqpoint{3.941967in}{6.405701in}}%
\pgfpathlineto{\pgfqpoint{3.923912in}{6.405701in}}%
\pgfpathlineto{\pgfqpoint{3.905857in}{6.405701in}}%
\pgfpathlineto{\pgfqpoint{3.887803in}{6.405701in}}%
\pgfpathlineto{\pgfqpoint{3.869748in}{6.405701in}}%
\pgfpathlineto{\pgfqpoint{3.851693in}{6.405701in}}%
\pgfpathlineto{\pgfqpoint{3.833639in}{6.405701in}}%
\pgfpathlineto{\pgfqpoint{3.815584in}{6.405701in}}%
\pgfpathlineto{\pgfqpoint{3.797529in}{6.405701in}}%
\pgfpathlineto{\pgfqpoint{3.779475in}{6.405701in}}%
\pgfpathlineto{\pgfqpoint{3.761420in}{6.405701in}}%
\pgfpathlineto{\pgfqpoint{3.743366in}{6.405701in}}%
\pgfpathlineto{\pgfqpoint{3.725311in}{6.405701in}}%
\pgfpathlineto{\pgfqpoint{3.707256in}{6.405701in}}%
\pgfpathlineto{\pgfqpoint{3.689202in}{6.405701in}}%
\pgfpathlineto{\pgfqpoint{3.671147in}{6.405701in}}%
\pgfpathlineto{\pgfqpoint{3.653092in}{6.405701in}}%
\pgfpathlineto{\pgfqpoint{3.635038in}{6.405701in}}%
\pgfpathlineto{\pgfqpoint{3.616983in}{6.405701in}}%
\pgfpathlineto{\pgfqpoint{3.598928in}{6.405701in}}%
\pgfpathlineto{\pgfqpoint{3.580874in}{6.405701in}}%
\pgfpathlineto{\pgfqpoint{3.562819in}{6.405701in}}%
\pgfpathlineto{\pgfqpoint{3.544764in}{6.405701in}}%
\pgfpathlineto{\pgfqpoint{3.526710in}{6.405701in}}%
\pgfpathlineto{\pgfqpoint{3.508655in}{6.405701in}}%
\pgfpathlineto{\pgfqpoint{3.490600in}{6.405701in}}%
\pgfpathlineto{\pgfqpoint{3.472546in}{6.405701in}}%
\pgfpathlineto{\pgfqpoint{3.454491in}{6.405701in}}%
\pgfpathlineto{\pgfqpoint{3.436436in}{6.405701in}}%
\pgfpathlineto{\pgfqpoint{3.418382in}{6.405701in}}%
\pgfpathlineto{\pgfqpoint{3.400327in}{6.405701in}}%
\pgfpathlineto{\pgfqpoint{3.382273in}{6.405701in}}%
\pgfpathlineto{\pgfqpoint{3.364218in}{6.405701in}}%
\pgfpathlineto{\pgfqpoint{3.346163in}{6.405701in}}%
\pgfpathlineto{\pgfqpoint{3.328109in}{6.405701in}}%
\pgfpathlineto{\pgfqpoint{3.310054in}{6.405701in}}%
\pgfpathlineto{\pgfqpoint{3.291999in}{6.405701in}}%
\pgfpathlineto{\pgfqpoint{3.273945in}{6.405701in}}%
\pgfpathlineto{\pgfqpoint{3.255890in}{6.405701in}}%
\pgfpathlineto{\pgfqpoint{3.237835in}{6.405701in}}%
\pgfpathlineto{\pgfqpoint{3.219781in}{6.405701in}}%
\pgfpathlineto{\pgfqpoint{3.201726in}{6.405701in}}%
\pgfpathlineto{\pgfqpoint{3.183671in}{6.405701in}}%
\pgfpathlineto{\pgfqpoint{3.165617in}{6.405701in}}%
\pgfpathlineto{\pgfqpoint{3.147562in}{6.405701in}}%
\pgfpathlineto{\pgfqpoint{3.129507in}{6.405701in}}%
\pgfpathlineto{\pgfqpoint{3.111453in}{6.405701in}}%
\pgfpathlineto{\pgfqpoint{3.093398in}{6.405701in}}%
\pgfpathlineto{\pgfqpoint{3.075343in}{6.405701in}}%
\pgfpathlineto{\pgfqpoint{3.057289in}{6.405701in}}%
\pgfpathlineto{\pgfqpoint{3.057289in}{6.405701in}}%
\pgfpathclose%
\pgfusepath{stroke,fill}%
}%
\begin{pgfscope}%
\pgfsys@transformshift{0.000000in}{0.000000in}%
\pgfsys@useobject{currentmarker}{}%
\end{pgfscope}%
\end{pgfscope}%
\begin{pgfscope}%
\pgfpathrectangle{\pgfqpoint{0.750000in}{3.960000in}}{\pgfqpoint{4.650000in}{3.080000in}}%
\pgfusepath{clip}%
\pgfsetrectcap%
\pgfsetroundjoin%
\pgfsetlinewidth{0.803000pt}%
\definecolor{currentstroke}{rgb}{0.690196,0.690196,0.690196}%
\pgfsetstrokecolor{currentstroke}%
\pgfsetdash{}{0pt}%
\pgfpathmoveto{\pgfqpoint{0.750000in}{3.960000in}}%
\pgfpathlineto{\pgfqpoint{0.750000in}{7.040000in}}%
\pgfusepath{stroke}%
\end{pgfscope}%
\begin{pgfscope}%
\pgfsetbuttcap%
\pgfsetroundjoin%
\definecolor{currentfill}{rgb}{0.000000,0.000000,0.000000}%
\pgfsetfillcolor{currentfill}%
\pgfsetlinewidth{0.803000pt}%
\definecolor{currentstroke}{rgb}{0.000000,0.000000,0.000000}%
\pgfsetstrokecolor{currentstroke}%
\pgfsetdash{}{0pt}%
\pgfsys@defobject{currentmarker}{\pgfqpoint{0.000000in}{-0.048611in}}{\pgfqpoint{0.000000in}{0.000000in}}{%
\pgfpathmoveto{\pgfqpoint{0.000000in}{0.000000in}}%
\pgfpathlineto{\pgfqpoint{0.000000in}{-0.048611in}}%
\pgfusepath{stroke,fill}%
}%
\begin{pgfscope}%
\pgfsys@transformshift{0.750000in}{3.960000in}%
\pgfsys@useobject{currentmarker}{}%
\end{pgfscope}%
\end{pgfscope}%
\begin{pgfscope}%
\pgfpathrectangle{\pgfqpoint{0.750000in}{3.960000in}}{\pgfqpoint{4.650000in}{3.080000in}}%
\pgfusepath{clip}%
\pgfsetrectcap%
\pgfsetroundjoin%
\pgfsetlinewidth{0.803000pt}%
\definecolor{currentstroke}{rgb}{0.690196,0.690196,0.690196}%
\pgfsetstrokecolor{currentstroke}%
\pgfsetdash{}{0pt}%
\pgfpathmoveto{\pgfqpoint{1.266667in}{3.960000in}}%
\pgfpathlineto{\pgfqpoint{1.266667in}{7.040000in}}%
\pgfusepath{stroke}%
\end{pgfscope}%
\begin{pgfscope}%
\pgfsetbuttcap%
\pgfsetroundjoin%
\definecolor{currentfill}{rgb}{0.000000,0.000000,0.000000}%
\pgfsetfillcolor{currentfill}%
\pgfsetlinewidth{0.803000pt}%
\definecolor{currentstroke}{rgb}{0.000000,0.000000,0.000000}%
\pgfsetstrokecolor{currentstroke}%
\pgfsetdash{}{0pt}%
\pgfsys@defobject{currentmarker}{\pgfqpoint{0.000000in}{-0.048611in}}{\pgfqpoint{0.000000in}{0.000000in}}{%
\pgfpathmoveto{\pgfqpoint{0.000000in}{0.000000in}}%
\pgfpathlineto{\pgfqpoint{0.000000in}{-0.048611in}}%
\pgfusepath{stroke,fill}%
}%
\begin{pgfscope}%
\pgfsys@transformshift{1.266667in}{3.960000in}%
\pgfsys@useobject{currentmarker}{}%
\end{pgfscope}%
\end{pgfscope}%
\begin{pgfscope}%
\pgfpathrectangle{\pgfqpoint{0.750000in}{3.960000in}}{\pgfqpoint{4.650000in}{3.080000in}}%
\pgfusepath{clip}%
\pgfsetrectcap%
\pgfsetroundjoin%
\pgfsetlinewidth{0.803000pt}%
\definecolor{currentstroke}{rgb}{0.690196,0.690196,0.690196}%
\pgfsetstrokecolor{currentstroke}%
\pgfsetdash{}{0pt}%
\pgfpathmoveto{\pgfqpoint{1.783333in}{3.960000in}}%
\pgfpathlineto{\pgfqpoint{1.783333in}{7.040000in}}%
\pgfusepath{stroke}%
\end{pgfscope}%
\begin{pgfscope}%
\pgfsetbuttcap%
\pgfsetroundjoin%
\definecolor{currentfill}{rgb}{0.000000,0.000000,0.000000}%
\pgfsetfillcolor{currentfill}%
\pgfsetlinewidth{0.803000pt}%
\definecolor{currentstroke}{rgb}{0.000000,0.000000,0.000000}%
\pgfsetstrokecolor{currentstroke}%
\pgfsetdash{}{0pt}%
\pgfsys@defobject{currentmarker}{\pgfqpoint{0.000000in}{-0.048611in}}{\pgfqpoint{0.000000in}{0.000000in}}{%
\pgfpathmoveto{\pgfqpoint{0.000000in}{0.000000in}}%
\pgfpathlineto{\pgfqpoint{0.000000in}{-0.048611in}}%
\pgfusepath{stroke,fill}%
}%
\begin{pgfscope}%
\pgfsys@transformshift{1.783333in}{3.960000in}%
\pgfsys@useobject{currentmarker}{}%
\end{pgfscope}%
\end{pgfscope}%
\begin{pgfscope}%
\pgfpathrectangle{\pgfqpoint{0.750000in}{3.960000in}}{\pgfqpoint{4.650000in}{3.080000in}}%
\pgfusepath{clip}%
\pgfsetrectcap%
\pgfsetroundjoin%
\pgfsetlinewidth{0.803000pt}%
\definecolor{currentstroke}{rgb}{0.690196,0.690196,0.690196}%
\pgfsetstrokecolor{currentstroke}%
\pgfsetdash{}{0pt}%
\pgfpathmoveto{\pgfqpoint{2.300000in}{3.960000in}}%
\pgfpathlineto{\pgfqpoint{2.300000in}{7.040000in}}%
\pgfusepath{stroke}%
\end{pgfscope}%
\begin{pgfscope}%
\pgfsetbuttcap%
\pgfsetroundjoin%
\definecolor{currentfill}{rgb}{0.000000,0.000000,0.000000}%
\pgfsetfillcolor{currentfill}%
\pgfsetlinewidth{0.803000pt}%
\definecolor{currentstroke}{rgb}{0.000000,0.000000,0.000000}%
\pgfsetstrokecolor{currentstroke}%
\pgfsetdash{}{0pt}%
\pgfsys@defobject{currentmarker}{\pgfqpoint{0.000000in}{-0.048611in}}{\pgfqpoint{0.000000in}{0.000000in}}{%
\pgfpathmoveto{\pgfqpoint{0.000000in}{0.000000in}}%
\pgfpathlineto{\pgfqpoint{0.000000in}{-0.048611in}}%
\pgfusepath{stroke,fill}%
}%
\begin{pgfscope}%
\pgfsys@transformshift{2.300000in}{3.960000in}%
\pgfsys@useobject{currentmarker}{}%
\end{pgfscope}%
\end{pgfscope}%
\begin{pgfscope}%
\pgfpathrectangle{\pgfqpoint{0.750000in}{3.960000in}}{\pgfqpoint{4.650000in}{3.080000in}}%
\pgfusepath{clip}%
\pgfsetrectcap%
\pgfsetroundjoin%
\pgfsetlinewidth{0.803000pt}%
\definecolor{currentstroke}{rgb}{0.690196,0.690196,0.690196}%
\pgfsetstrokecolor{currentstroke}%
\pgfsetdash{}{0pt}%
\pgfpathmoveto{\pgfqpoint{2.816667in}{3.960000in}}%
\pgfpathlineto{\pgfqpoint{2.816667in}{7.040000in}}%
\pgfusepath{stroke}%
\end{pgfscope}%
\begin{pgfscope}%
\pgfsetbuttcap%
\pgfsetroundjoin%
\definecolor{currentfill}{rgb}{0.000000,0.000000,0.000000}%
\pgfsetfillcolor{currentfill}%
\pgfsetlinewidth{0.803000pt}%
\definecolor{currentstroke}{rgb}{0.000000,0.000000,0.000000}%
\pgfsetstrokecolor{currentstroke}%
\pgfsetdash{}{0pt}%
\pgfsys@defobject{currentmarker}{\pgfqpoint{0.000000in}{-0.048611in}}{\pgfqpoint{0.000000in}{0.000000in}}{%
\pgfpathmoveto{\pgfqpoint{0.000000in}{0.000000in}}%
\pgfpathlineto{\pgfqpoint{0.000000in}{-0.048611in}}%
\pgfusepath{stroke,fill}%
}%
\begin{pgfscope}%
\pgfsys@transformshift{2.816667in}{3.960000in}%
\pgfsys@useobject{currentmarker}{}%
\end{pgfscope}%
\end{pgfscope}%
\begin{pgfscope}%
\pgfpathrectangle{\pgfqpoint{0.750000in}{3.960000in}}{\pgfqpoint{4.650000in}{3.080000in}}%
\pgfusepath{clip}%
\pgfsetrectcap%
\pgfsetroundjoin%
\pgfsetlinewidth{0.803000pt}%
\definecolor{currentstroke}{rgb}{0.690196,0.690196,0.690196}%
\pgfsetstrokecolor{currentstroke}%
\pgfsetdash{}{0pt}%
\pgfpathmoveto{\pgfqpoint{3.333333in}{3.960000in}}%
\pgfpathlineto{\pgfqpoint{3.333333in}{7.040000in}}%
\pgfusepath{stroke}%
\end{pgfscope}%
\begin{pgfscope}%
\pgfsetbuttcap%
\pgfsetroundjoin%
\definecolor{currentfill}{rgb}{0.000000,0.000000,0.000000}%
\pgfsetfillcolor{currentfill}%
\pgfsetlinewidth{0.803000pt}%
\definecolor{currentstroke}{rgb}{0.000000,0.000000,0.000000}%
\pgfsetstrokecolor{currentstroke}%
\pgfsetdash{}{0pt}%
\pgfsys@defobject{currentmarker}{\pgfqpoint{0.000000in}{-0.048611in}}{\pgfqpoint{0.000000in}{0.000000in}}{%
\pgfpathmoveto{\pgfqpoint{0.000000in}{0.000000in}}%
\pgfpathlineto{\pgfqpoint{0.000000in}{-0.048611in}}%
\pgfusepath{stroke,fill}%
}%
\begin{pgfscope}%
\pgfsys@transformshift{3.333333in}{3.960000in}%
\pgfsys@useobject{currentmarker}{}%
\end{pgfscope}%
\end{pgfscope}%
\begin{pgfscope}%
\pgfpathrectangle{\pgfqpoint{0.750000in}{3.960000in}}{\pgfqpoint{4.650000in}{3.080000in}}%
\pgfusepath{clip}%
\pgfsetrectcap%
\pgfsetroundjoin%
\pgfsetlinewidth{0.803000pt}%
\definecolor{currentstroke}{rgb}{0.690196,0.690196,0.690196}%
\pgfsetstrokecolor{currentstroke}%
\pgfsetdash{}{0pt}%
\pgfpathmoveto{\pgfqpoint{3.850000in}{3.960000in}}%
\pgfpathlineto{\pgfqpoint{3.850000in}{7.040000in}}%
\pgfusepath{stroke}%
\end{pgfscope}%
\begin{pgfscope}%
\pgfsetbuttcap%
\pgfsetroundjoin%
\definecolor{currentfill}{rgb}{0.000000,0.000000,0.000000}%
\pgfsetfillcolor{currentfill}%
\pgfsetlinewidth{0.803000pt}%
\definecolor{currentstroke}{rgb}{0.000000,0.000000,0.000000}%
\pgfsetstrokecolor{currentstroke}%
\pgfsetdash{}{0pt}%
\pgfsys@defobject{currentmarker}{\pgfqpoint{0.000000in}{-0.048611in}}{\pgfqpoint{0.000000in}{0.000000in}}{%
\pgfpathmoveto{\pgfqpoint{0.000000in}{0.000000in}}%
\pgfpathlineto{\pgfqpoint{0.000000in}{-0.048611in}}%
\pgfusepath{stroke,fill}%
}%
\begin{pgfscope}%
\pgfsys@transformshift{3.850000in}{3.960000in}%
\pgfsys@useobject{currentmarker}{}%
\end{pgfscope}%
\end{pgfscope}%
\begin{pgfscope}%
\pgfpathrectangle{\pgfqpoint{0.750000in}{3.960000in}}{\pgfqpoint{4.650000in}{3.080000in}}%
\pgfusepath{clip}%
\pgfsetrectcap%
\pgfsetroundjoin%
\pgfsetlinewidth{0.803000pt}%
\definecolor{currentstroke}{rgb}{0.690196,0.690196,0.690196}%
\pgfsetstrokecolor{currentstroke}%
\pgfsetdash{}{0pt}%
\pgfpathmoveto{\pgfqpoint{4.366667in}{3.960000in}}%
\pgfpathlineto{\pgfqpoint{4.366667in}{7.040000in}}%
\pgfusepath{stroke}%
\end{pgfscope}%
\begin{pgfscope}%
\pgfsetbuttcap%
\pgfsetroundjoin%
\definecolor{currentfill}{rgb}{0.000000,0.000000,0.000000}%
\pgfsetfillcolor{currentfill}%
\pgfsetlinewidth{0.803000pt}%
\definecolor{currentstroke}{rgb}{0.000000,0.000000,0.000000}%
\pgfsetstrokecolor{currentstroke}%
\pgfsetdash{}{0pt}%
\pgfsys@defobject{currentmarker}{\pgfqpoint{0.000000in}{-0.048611in}}{\pgfqpoint{0.000000in}{0.000000in}}{%
\pgfpathmoveto{\pgfqpoint{0.000000in}{0.000000in}}%
\pgfpathlineto{\pgfqpoint{0.000000in}{-0.048611in}}%
\pgfusepath{stroke,fill}%
}%
\begin{pgfscope}%
\pgfsys@transformshift{4.366667in}{3.960000in}%
\pgfsys@useobject{currentmarker}{}%
\end{pgfscope}%
\end{pgfscope}%
\begin{pgfscope}%
\pgfpathrectangle{\pgfqpoint{0.750000in}{3.960000in}}{\pgfqpoint{4.650000in}{3.080000in}}%
\pgfusepath{clip}%
\pgfsetrectcap%
\pgfsetroundjoin%
\pgfsetlinewidth{0.803000pt}%
\definecolor{currentstroke}{rgb}{0.690196,0.690196,0.690196}%
\pgfsetstrokecolor{currentstroke}%
\pgfsetdash{}{0pt}%
\pgfpathmoveto{\pgfqpoint{4.883333in}{3.960000in}}%
\pgfpathlineto{\pgfqpoint{4.883333in}{7.040000in}}%
\pgfusepath{stroke}%
\end{pgfscope}%
\begin{pgfscope}%
\pgfsetbuttcap%
\pgfsetroundjoin%
\definecolor{currentfill}{rgb}{0.000000,0.000000,0.000000}%
\pgfsetfillcolor{currentfill}%
\pgfsetlinewidth{0.803000pt}%
\definecolor{currentstroke}{rgb}{0.000000,0.000000,0.000000}%
\pgfsetstrokecolor{currentstroke}%
\pgfsetdash{}{0pt}%
\pgfsys@defobject{currentmarker}{\pgfqpoint{0.000000in}{-0.048611in}}{\pgfqpoint{0.000000in}{0.000000in}}{%
\pgfpathmoveto{\pgfqpoint{0.000000in}{0.000000in}}%
\pgfpathlineto{\pgfqpoint{0.000000in}{-0.048611in}}%
\pgfusepath{stroke,fill}%
}%
\begin{pgfscope}%
\pgfsys@transformshift{4.883333in}{3.960000in}%
\pgfsys@useobject{currentmarker}{}%
\end{pgfscope}%
\end{pgfscope}%
\begin{pgfscope}%
\pgfpathrectangle{\pgfqpoint{0.750000in}{3.960000in}}{\pgfqpoint{4.650000in}{3.080000in}}%
\pgfusepath{clip}%
\pgfsetrectcap%
\pgfsetroundjoin%
\pgfsetlinewidth{0.803000pt}%
\definecolor{currentstroke}{rgb}{0.690196,0.690196,0.690196}%
\pgfsetstrokecolor{currentstroke}%
\pgfsetdash{}{0pt}%
\pgfpathmoveto{\pgfqpoint{5.400000in}{3.960000in}}%
\pgfpathlineto{\pgfqpoint{5.400000in}{7.040000in}}%
\pgfusepath{stroke}%
\end{pgfscope}%
\begin{pgfscope}%
\pgfsetbuttcap%
\pgfsetroundjoin%
\definecolor{currentfill}{rgb}{0.000000,0.000000,0.000000}%
\pgfsetfillcolor{currentfill}%
\pgfsetlinewidth{0.803000pt}%
\definecolor{currentstroke}{rgb}{0.000000,0.000000,0.000000}%
\pgfsetstrokecolor{currentstroke}%
\pgfsetdash{}{0pt}%
\pgfsys@defobject{currentmarker}{\pgfqpoint{0.000000in}{-0.048611in}}{\pgfqpoint{0.000000in}{0.000000in}}{%
\pgfpathmoveto{\pgfqpoint{0.000000in}{0.000000in}}%
\pgfpathlineto{\pgfqpoint{0.000000in}{-0.048611in}}%
\pgfusepath{stroke,fill}%
}%
\begin{pgfscope}%
\pgfsys@transformshift{5.400000in}{3.960000in}%
\pgfsys@useobject{currentmarker}{}%
\end{pgfscope}%
\end{pgfscope}%
\begin{pgfscope}%
\pgfpathrectangle{\pgfqpoint{0.750000in}{3.960000in}}{\pgfqpoint{4.650000in}{3.080000in}}%
\pgfusepath{clip}%
\pgfsetrectcap%
\pgfsetroundjoin%
\pgfsetlinewidth{0.803000pt}%
\definecolor{currentstroke}{rgb}{0.690196,0.690196,0.690196}%
\pgfsetstrokecolor{currentstroke}%
\pgfsetdash{}{0pt}%
\pgfpathmoveto{\pgfqpoint{0.750000in}{3.960000in}}%
\pgfpathlineto{\pgfqpoint{5.400000in}{3.960000in}}%
\pgfusepath{stroke}%
\end{pgfscope}%
\begin{pgfscope}%
\pgfsetbuttcap%
\pgfsetroundjoin%
\definecolor{currentfill}{rgb}{0.000000,0.000000,0.000000}%
\pgfsetfillcolor{currentfill}%
\pgfsetlinewidth{0.803000pt}%
\definecolor{currentstroke}{rgb}{0.000000,0.000000,0.000000}%
\pgfsetstrokecolor{currentstroke}%
\pgfsetdash{}{0pt}%
\pgfsys@defobject{currentmarker}{\pgfqpoint{-0.048611in}{0.000000in}}{\pgfqpoint{-0.000000in}{0.000000in}}{%
\pgfpathmoveto{\pgfqpoint{-0.000000in}{0.000000in}}%
\pgfpathlineto{\pgfqpoint{-0.048611in}{0.000000in}}%
\pgfusepath{stroke,fill}%
}%
\begin{pgfscope}%
\pgfsys@transformshift{0.750000in}{3.960000in}%
\pgfsys@useobject{currentmarker}{}%
\end{pgfscope}%
\end{pgfscope}%
\begin{pgfscope}%
\definecolor{textcolor}{rgb}{0.000000,0.000000,0.000000}%
\pgfsetstrokecolor{textcolor}%
\pgfsetfillcolor{textcolor}%
\pgftext[x=0.475308in, y=3.908900in, left, base]{\color{textcolor}\rmfamily\fontsize{10.000000}{12.000000}\selectfont \(\displaystyle {0.0}\)}%
\end{pgfscope}%
\begin{pgfscope}%
\pgfpathrectangle{\pgfqpoint{0.750000in}{3.960000in}}{\pgfqpoint{4.650000in}{3.080000in}}%
\pgfusepath{clip}%
\pgfsetrectcap%
\pgfsetroundjoin%
\pgfsetlinewidth{0.803000pt}%
\definecolor{currentstroke}{rgb}{0.690196,0.690196,0.690196}%
\pgfsetstrokecolor{currentstroke}%
\pgfsetdash{}{0pt}%
\pgfpathmoveto{\pgfqpoint{0.750000in}{4.449140in}}%
\pgfpathlineto{\pgfqpoint{5.400000in}{4.449140in}}%
\pgfusepath{stroke}%
\end{pgfscope}%
\begin{pgfscope}%
\pgfsetbuttcap%
\pgfsetroundjoin%
\definecolor{currentfill}{rgb}{0.000000,0.000000,0.000000}%
\pgfsetfillcolor{currentfill}%
\pgfsetlinewidth{0.803000pt}%
\definecolor{currentstroke}{rgb}{0.000000,0.000000,0.000000}%
\pgfsetstrokecolor{currentstroke}%
\pgfsetdash{}{0pt}%
\pgfsys@defobject{currentmarker}{\pgfqpoint{-0.048611in}{0.000000in}}{\pgfqpoint{-0.000000in}{0.000000in}}{%
\pgfpathmoveto{\pgfqpoint{-0.000000in}{0.000000in}}%
\pgfpathlineto{\pgfqpoint{-0.048611in}{0.000000in}}%
\pgfusepath{stroke,fill}%
}%
\begin{pgfscope}%
\pgfsys@transformshift{0.750000in}{4.449140in}%
\pgfsys@useobject{currentmarker}{}%
\end{pgfscope}%
\end{pgfscope}%
\begin{pgfscope}%
\definecolor{textcolor}{rgb}{0.000000,0.000000,0.000000}%
\pgfsetstrokecolor{textcolor}%
\pgfsetfillcolor{textcolor}%
\pgftext[x=0.475308in, y=4.398040in, left, base]{\color{textcolor}\rmfamily\fontsize{10.000000}{12.000000}\selectfont \(\displaystyle {0.2}\)}%
\end{pgfscope}%
\begin{pgfscope}%
\pgfpathrectangle{\pgfqpoint{0.750000in}{3.960000in}}{\pgfqpoint{4.650000in}{3.080000in}}%
\pgfusepath{clip}%
\pgfsetrectcap%
\pgfsetroundjoin%
\pgfsetlinewidth{0.803000pt}%
\definecolor{currentstroke}{rgb}{0.690196,0.690196,0.690196}%
\pgfsetstrokecolor{currentstroke}%
\pgfsetdash{}{0pt}%
\pgfpathmoveto{\pgfqpoint{0.750000in}{4.938280in}}%
\pgfpathlineto{\pgfqpoint{5.400000in}{4.938280in}}%
\pgfusepath{stroke}%
\end{pgfscope}%
\begin{pgfscope}%
\pgfsetbuttcap%
\pgfsetroundjoin%
\definecolor{currentfill}{rgb}{0.000000,0.000000,0.000000}%
\pgfsetfillcolor{currentfill}%
\pgfsetlinewidth{0.803000pt}%
\definecolor{currentstroke}{rgb}{0.000000,0.000000,0.000000}%
\pgfsetstrokecolor{currentstroke}%
\pgfsetdash{}{0pt}%
\pgfsys@defobject{currentmarker}{\pgfqpoint{-0.048611in}{0.000000in}}{\pgfqpoint{-0.000000in}{0.000000in}}{%
\pgfpathmoveto{\pgfqpoint{-0.000000in}{0.000000in}}%
\pgfpathlineto{\pgfqpoint{-0.048611in}{0.000000in}}%
\pgfusepath{stroke,fill}%
}%
\begin{pgfscope}%
\pgfsys@transformshift{0.750000in}{4.938280in}%
\pgfsys@useobject{currentmarker}{}%
\end{pgfscope}%
\end{pgfscope}%
\begin{pgfscope}%
\definecolor{textcolor}{rgb}{0.000000,0.000000,0.000000}%
\pgfsetstrokecolor{textcolor}%
\pgfsetfillcolor{textcolor}%
\pgftext[x=0.475308in, y=4.887180in, left, base]{\color{textcolor}\rmfamily\fontsize{10.000000}{12.000000}\selectfont \(\displaystyle {0.4}\)}%
\end{pgfscope}%
\begin{pgfscope}%
\pgfpathrectangle{\pgfqpoint{0.750000in}{3.960000in}}{\pgfqpoint{4.650000in}{3.080000in}}%
\pgfusepath{clip}%
\pgfsetrectcap%
\pgfsetroundjoin%
\pgfsetlinewidth{0.803000pt}%
\definecolor{currentstroke}{rgb}{0.690196,0.690196,0.690196}%
\pgfsetstrokecolor{currentstroke}%
\pgfsetdash{}{0pt}%
\pgfpathmoveto{\pgfqpoint{0.750000in}{5.427421in}}%
\pgfpathlineto{\pgfqpoint{5.400000in}{5.427421in}}%
\pgfusepath{stroke}%
\end{pgfscope}%
\begin{pgfscope}%
\pgfsetbuttcap%
\pgfsetroundjoin%
\definecolor{currentfill}{rgb}{0.000000,0.000000,0.000000}%
\pgfsetfillcolor{currentfill}%
\pgfsetlinewidth{0.803000pt}%
\definecolor{currentstroke}{rgb}{0.000000,0.000000,0.000000}%
\pgfsetstrokecolor{currentstroke}%
\pgfsetdash{}{0pt}%
\pgfsys@defobject{currentmarker}{\pgfqpoint{-0.048611in}{0.000000in}}{\pgfqpoint{-0.000000in}{0.000000in}}{%
\pgfpathmoveto{\pgfqpoint{-0.000000in}{0.000000in}}%
\pgfpathlineto{\pgfqpoint{-0.048611in}{0.000000in}}%
\pgfusepath{stroke,fill}%
}%
\begin{pgfscope}%
\pgfsys@transformshift{0.750000in}{5.427421in}%
\pgfsys@useobject{currentmarker}{}%
\end{pgfscope}%
\end{pgfscope}%
\begin{pgfscope}%
\definecolor{textcolor}{rgb}{0.000000,0.000000,0.000000}%
\pgfsetstrokecolor{textcolor}%
\pgfsetfillcolor{textcolor}%
\pgftext[x=0.475308in, y=5.376321in, left, base]{\color{textcolor}\rmfamily\fontsize{10.000000}{12.000000}\selectfont \(\displaystyle {0.6}\)}%
\end{pgfscope}%
\begin{pgfscope}%
\pgfpathrectangle{\pgfqpoint{0.750000in}{3.960000in}}{\pgfqpoint{4.650000in}{3.080000in}}%
\pgfusepath{clip}%
\pgfsetrectcap%
\pgfsetroundjoin%
\pgfsetlinewidth{0.803000pt}%
\definecolor{currentstroke}{rgb}{0.690196,0.690196,0.690196}%
\pgfsetstrokecolor{currentstroke}%
\pgfsetdash{}{0pt}%
\pgfpathmoveto{\pgfqpoint{0.750000in}{5.916561in}}%
\pgfpathlineto{\pgfqpoint{5.400000in}{5.916561in}}%
\pgfusepath{stroke}%
\end{pgfscope}%
\begin{pgfscope}%
\pgfsetbuttcap%
\pgfsetroundjoin%
\definecolor{currentfill}{rgb}{0.000000,0.000000,0.000000}%
\pgfsetfillcolor{currentfill}%
\pgfsetlinewidth{0.803000pt}%
\definecolor{currentstroke}{rgb}{0.000000,0.000000,0.000000}%
\pgfsetstrokecolor{currentstroke}%
\pgfsetdash{}{0pt}%
\pgfsys@defobject{currentmarker}{\pgfqpoint{-0.048611in}{0.000000in}}{\pgfqpoint{-0.000000in}{0.000000in}}{%
\pgfpathmoveto{\pgfqpoint{-0.000000in}{0.000000in}}%
\pgfpathlineto{\pgfqpoint{-0.048611in}{0.000000in}}%
\pgfusepath{stroke,fill}%
}%
\begin{pgfscope}%
\pgfsys@transformshift{0.750000in}{5.916561in}%
\pgfsys@useobject{currentmarker}{}%
\end{pgfscope}%
\end{pgfscope}%
\begin{pgfscope}%
\definecolor{textcolor}{rgb}{0.000000,0.000000,0.000000}%
\pgfsetstrokecolor{textcolor}%
\pgfsetfillcolor{textcolor}%
\pgftext[x=0.475308in, y=5.865461in, left, base]{\color{textcolor}\rmfamily\fontsize{10.000000}{12.000000}\selectfont \(\displaystyle {0.8}\)}%
\end{pgfscope}%
\begin{pgfscope}%
\pgfpathrectangle{\pgfqpoint{0.750000in}{3.960000in}}{\pgfqpoint{4.650000in}{3.080000in}}%
\pgfusepath{clip}%
\pgfsetrectcap%
\pgfsetroundjoin%
\pgfsetlinewidth{0.803000pt}%
\definecolor{currentstroke}{rgb}{0.690196,0.690196,0.690196}%
\pgfsetstrokecolor{currentstroke}%
\pgfsetdash{}{0pt}%
\pgfpathmoveto{\pgfqpoint{0.750000in}{6.405701in}}%
\pgfpathlineto{\pgfqpoint{5.400000in}{6.405701in}}%
\pgfusepath{stroke}%
\end{pgfscope}%
\begin{pgfscope}%
\pgfsetbuttcap%
\pgfsetroundjoin%
\definecolor{currentfill}{rgb}{0.000000,0.000000,0.000000}%
\pgfsetfillcolor{currentfill}%
\pgfsetlinewidth{0.803000pt}%
\definecolor{currentstroke}{rgb}{0.000000,0.000000,0.000000}%
\pgfsetstrokecolor{currentstroke}%
\pgfsetdash{}{0pt}%
\pgfsys@defobject{currentmarker}{\pgfqpoint{-0.048611in}{0.000000in}}{\pgfqpoint{-0.000000in}{0.000000in}}{%
\pgfpathmoveto{\pgfqpoint{-0.000000in}{0.000000in}}%
\pgfpathlineto{\pgfqpoint{-0.048611in}{0.000000in}}%
\pgfusepath{stroke,fill}%
}%
\begin{pgfscope}%
\pgfsys@transformshift{0.750000in}{6.405701in}%
\pgfsys@useobject{currentmarker}{}%
\end{pgfscope}%
\end{pgfscope}%
\begin{pgfscope}%
\definecolor{textcolor}{rgb}{0.000000,0.000000,0.000000}%
\pgfsetstrokecolor{textcolor}%
\pgfsetfillcolor{textcolor}%
\pgftext[x=0.475308in, y=6.354601in, left, base]{\color{textcolor}\rmfamily\fontsize{10.000000}{12.000000}\selectfont \(\displaystyle {1.0}\)}%
\end{pgfscope}%
\begin{pgfscope}%
\pgfpathrectangle{\pgfqpoint{0.750000in}{3.960000in}}{\pgfqpoint{4.650000in}{3.080000in}}%
\pgfusepath{clip}%
\pgfsetrectcap%
\pgfsetroundjoin%
\pgfsetlinewidth{0.803000pt}%
\definecolor{currentstroke}{rgb}{0.690196,0.690196,0.690196}%
\pgfsetstrokecolor{currentstroke}%
\pgfsetdash{}{0pt}%
\pgfpathmoveto{\pgfqpoint{0.750000in}{6.894841in}}%
\pgfpathlineto{\pgfqpoint{5.400000in}{6.894841in}}%
\pgfusepath{stroke}%
\end{pgfscope}%
\begin{pgfscope}%
\pgfsetbuttcap%
\pgfsetroundjoin%
\definecolor{currentfill}{rgb}{0.000000,0.000000,0.000000}%
\pgfsetfillcolor{currentfill}%
\pgfsetlinewidth{0.803000pt}%
\definecolor{currentstroke}{rgb}{0.000000,0.000000,0.000000}%
\pgfsetstrokecolor{currentstroke}%
\pgfsetdash{}{0pt}%
\pgfsys@defobject{currentmarker}{\pgfqpoint{-0.048611in}{0.000000in}}{\pgfqpoint{-0.000000in}{0.000000in}}{%
\pgfpathmoveto{\pgfqpoint{-0.000000in}{0.000000in}}%
\pgfpathlineto{\pgfqpoint{-0.048611in}{0.000000in}}%
\pgfusepath{stroke,fill}%
}%
\begin{pgfscope}%
\pgfsys@transformshift{0.750000in}{6.894841in}%
\pgfsys@useobject{currentmarker}{}%
\end{pgfscope}%
\end{pgfscope}%
\begin{pgfscope}%
\definecolor{textcolor}{rgb}{0.000000,0.000000,0.000000}%
\pgfsetstrokecolor{textcolor}%
\pgfsetfillcolor{textcolor}%
\pgftext[x=0.475308in, y=6.843741in, left, base]{\color{textcolor}\rmfamily\fontsize{10.000000}{12.000000}\selectfont \(\displaystyle {1.2}\)}%
\end{pgfscope}%
\begin{pgfscope}%
\definecolor{textcolor}{rgb}{0.000000,0.000000,0.000000}%
\pgfsetstrokecolor{textcolor}%
\pgfsetfillcolor{textcolor}%
\pgftext[x=0.419752in,y=5.500000in,,bottom,rotate=90.000000]{\color{textcolor}\rmfamily\fontsize{10.000000}{12.000000}\selectfont power in pu}%
\end{pgfscope}%
\begin{pgfscope}%
\pgfpathrectangle{\pgfqpoint{0.750000in}{3.960000in}}{\pgfqpoint{4.650000in}{3.080000in}}%
\pgfusepath{clip}%
\pgfsetrectcap%
\pgfsetroundjoin%
\pgfsetlinewidth{2.007500pt}%
\definecolor{currentstroke}{rgb}{0.121569,0.466667,0.705882}%
\pgfsetstrokecolor{currentstroke}%
\pgfsetdash{}{0pt}%
\pgfpathmoveto{\pgfqpoint{0.750000in}{3.960000in}}%
\pgfpathlineto{\pgfqpoint{0.844898in}{4.148036in}}%
\pgfpathlineto{\pgfqpoint{0.939796in}{4.335299in}}%
\pgfpathlineto{\pgfqpoint{1.034694in}{4.521020in}}%
\pgfpathlineto{\pgfqpoint{1.129592in}{4.704436in}}%
\pgfpathlineto{\pgfqpoint{1.224490in}{4.884793in}}%
\pgfpathlineto{\pgfqpoint{1.319388in}{5.061349in}}%
\pgfpathlineto{\pgfqpoint{1.414286in}{5.233380in}}%
\pgfpathlineto{\pgfqpoint{1.509184in}{5.400178in}}%
\pgfpathlineto{\pgfqpoint{1.604082in}{5.561058in}}%
\pgfpathlineto{\pgfqpoint{1.698980in}{5.715359in}}%
\pgfpathlineto{\pgfqpoint{1.793878in}{5.862447in}}%
\pgfpathlineto{\pgfqpoint{1.888776in}{6.001718in}}%
\pgfpathlineto{\pgfqpoint{1.983673in}{6.132598in}}%
\pgfpathlineto{\pgfqpoint{2.078571in}{6.254551in}}%
\pgfpathlineto{\pgfqpoint{2.173469in}{6.367075in}}%
\pgfpathlineto{\pgfqpoint{2.268367in}{6.469708in}}%
\pgfpathlineto{\pgfqpoint{2.363265in}{6.562028in}}%
\pgfpathlineto{\pgfqpoint{2.458163in}{6.643656in}}%
\pgfpathlineto{\pgfqpoint{2.553061in}{6.714256in}}%
\pgfpathlineto{\pgfqpoint{2.647959in}{6.773538in}}%
\pgfpathlineto{\pgfqpoint{2.742857in}{6.821259in}}%
\pgfpathlineto{\pgfqpoint{2.837755in}{6.857222in}}%
\pgfpathlineto{\pgfqpoint{2.932653in}{6.881280in}}%
\pgfpathlineto{\pgfqpoint{3.027551in}{6.893333in}}%
\pgfpathlineto{\pgfqpoint{3.122449in}{6.893333in}}%
\pgfpathlineto{\pgfqpoint{3.217347in}{6.881280in}}%
\pgfpathlineto{\pgfqpoint{3.312245in}{6.857222in}}%
\pgfpathlineto{\pgfqpoint{3.407143in}{6.821259in}}%
\pgfpathlineto{\pgfqpoint{3.502041in}{6.773538in}}%
\pgfpathlineto{\pgfqpoint{3.596939in}{6.714256in}}%
\pgfpathlineto{\pgfqpoint{3.691837in}{6.643656in}}%
\pgfpathlineto{\pgfqpoint{3.786735in}{6.562028in}}%
\pgfpathlineto{\pgfqpoint{3.881633in}{6.469708in}}%
\pgfpathlineto{\pgfqpoint{3.976531in}{6.367075in}}%
\pgfpathlineto{\pgfqpoint{4.071429in}{6.254551in}}%
\pgfpathlineto{\pgfqpoint{4.166327in}{6.132598in}}%
\pgfpathlineto{\pgfqpoint{4.261224in}{6.001718in}}%
\pgfpathlineto{\pgfqpoint{4.356122in}{5.862447in}}%
\pgfpathlineto{\pgfqpoint{4.451020in}{5.715359in}}%
\pgfpathlineto{\pgfqpoint{4.545918in}{5.561058in}}%
\pgfpathlineto{\pgfqpoint{4.640816in}{5.400178in}}%
\pgfpathlineto{\pgfqpoint{4.735714in}{5.233380in}}%
\pgfpathlineto{\pgfqpoint{4.830612in}{5.061349in}}%
\pgfpathlineto{\pgfqpoint{4.925510in}{4.884793in}}%
\pgfpathlineto{\pgfqpoint{5.020408in}{4.704436in}}%
\pgfpathlineto{\pgfqpoint{5.115306in}{4.521020in}}%
\pgfpathlineto{\pgfqpoint{5.210204in}{4.335299in}}%
\pgfpathlineto{\pgfqpoint{5.305102in}{4.148036in}}%
\pgfpathlineto{\pgfqpoint{5.400000in}{3.960000in}}%
\pgfusepath{stroke}%
\end{pgfscope}%
\begin{pgfscope}%
\pgfpathrectangle{\pgfqpoint{0.750000in}{3.960000in}}{\pgfqpoint{4.650000in}{3.080000in}}%
\pgfusepath{clip}%
\pgfsetrectcap%
\pgfsetroundjoin%
\pgfsetlinewidth{2.007500pt}%
\definecolor{currentstroke}{rgb}{1.000000,0.498039,0.054902}%
\pgfsetstrokecolor{currentstroke}%
\pgfsetdash{}{0pt}%
\pgfpathmoveto{\pgfqpoint{0.750000in}{3.960000in}}%
\pgfpathlineto{\pgfqpoint{0.844898in}{4.100105in}}%
\pgfpathlineto{\pgfqpoint{0.939796in}{4.239635in}}%
\pgfpathlineto{\pgfqpoint{1.034694in}{4.378015in}}%
\pgfpathlineto{\pgfqpoint{1.129592in}{4.514678in}}%
\pgfpathlineto{\pgfqpoint{1.224490in}{4.649061in}}%
\pgfpathlineto{\pgfqpoint{1.319388in}{4.780613in}}%
\pgfpathlineto{\pgfqpoint{1.414286in}{4.908793in}}%
\pgfpathlineto{\pgfqpoint{1.509184in}{5.033074in}}%
\pgfpathlineto{\pgfqpoint{1.604082in}{5.152945in}}%
\pgfpathlineto{\pgfqpoint{1.698980in}{5.267915in}}%
\pgfpathlineto{\pgfqpoint{1.793878in}{5.377510in}}%
\pgfpathlineto{\pgfqpoint{1.888776in}{5.481280in}}%
\pgfpathlineto{\pgfqpoint{1.983673in}{5.578799in}}%
\pgfpathlineto{\pgfqpoint{2.078571in}{5.669666in}}%
\pgfpathlineto{\pgfqpoint{2.173469in}{5.753507in}}%
\pgfpathlineto{\pgfqpoint{2.268367in}{5.829979in}}%
\pgfpathlineto{\pgfqpoint{2.363265in}{5.898766in}}%
\pgfpathlineto{\pgfqpoint{2.458163in}{5.959587in}}%
\pgfpathlineto{\pgfqpoint{2.553061in}{6.012191in}}%
\pgfpathlineto{\pgfqpoint{2.647959in}{6.056362in}}%
\pgfpathlineto{\pgfqpoint{2.742857in}{6.091918in}}%
\pgfpathlineto{\pgfqpoint{2.837755in}{6.118714in}}%
\pgfpathlineto{\pgfqpoint{2.932653in}{6.136640in}}%
\pgfpathlineto{\pgfqpoint{3.027551in}{6.145621in}}%
\pgfpathlineto{\pgfqpoint{3.122449in}{6.145621in}}%
\pgfpathlineto{\pgfqpoint{3.217347in}{6.136640in}}%
\pgfpathlineto{\pgfqpoint{3.312245in}{6.118714in}}%
\pgfpathlineto{\pgfqpoint{3.407143in}{6.091918in}}%
\pgfpathlineto{\pgfqpoint{3.502041in}{6.056362in}}%
\pgfpathlineto{\pgfqpoint{3.596939in}{6.012191in}}%
\pgfpathlineto{\pgfqpoint{3.691837in}{5.959587in}}%
\pgfpathlineto{\pgfqpoint{3.786735in}{5.898766in}}%
\pgfpathlineto{\pgfqpoint{3.881633in}{5.829979in}}%
\pgfpathlineto{\pgfqpoint{3.976531in}{5.753507in}}%
\pgfpathlineto{\pgfqpoint{4.071429in}{5.669666in}}%
\pgfpathlineto{\pgfqpoint{4.166327in}{5.578799in}}%
\pgfpathlineto{\pgfqpoint{4.261224in}{5.481280in}}%
\pgfpathlineto{\pgfqpoint{4.356122in}{5.377510in}}%
\pgfpathlineto{\pgfqpoint{4.451020in}{5.267915in}}%
\pgfpathlineto{\pgfqpoint{4.545918in}{5.152945in}}%
\pgfpathlineto{\pgfqpoint{4.640816in}{5.033074in}}%
\pgfpathlineto{\pgfqpoint{4.735714in}{4.908793in}}%
\pgfpathlineto{\pgfqpoint{4.830612in}{4.780613in}}%
\pgfpathlineto{\pgfqpoint{4.925510in}{4.649061in}}%
\pgfpathlineto{\pgfqpoint{5.020408in}{4.514678in}}%
\pgfpathlineto{\pgfqpoint{5.115306in}{4.378015in}}%
\pgfpathlineto{\pgfqpoint{5.210204in}{4.239635in}}%
\pgfpathlineto{\pgfqpoint{5.305102in}{4.100105in}}%
\pgfpathlineto{\pgfqpoint{5.400000in}{3.960000in}}%
\pgfusepath{stroke}%
\end{pgfscope}%
\begin{pgfscope}%
\pgfpathrectangle{\pgfqpoint{0.750000in}{3.960000in}}{\pgfqpoint{4.650000in}{3.080000in}}%
\pgfusepath{clip}%
\pgfsetrectcap%
\pgfsetroundjoin%
\pgfsetlinewidth{2.007500pt}%
\definecolor{currentstroke}{rgb}{0.172549,0.627451,0.172549}%
\pgfsetstrokecolor{currentstroke}%
\pgfsetdash{}{0pt}%
\pgfpathmoveto{\pgfqpoint{0.750000in}{6.405701in}}%
\pgfpathlineto{\pgfqpoint{0.844898in}{6.405701in}}%
\pgfpathlineto{\pgfqpoint{0.939796in}{6.405701in}}%
\pgfpathlineto{\pgfqpoint{1.034694in}{6.405701in}}%
\pgfpathlineto{\pgfqpoint{1.129592in}{6.405701in}}%
\pgfpathlineto{\pgfqpoint{1.224490in}{6.405701in}}%
\pgfpathlineto{\pgfqpoint{1.319388in}{6.405701in}}%
\pgfpathlineto{\pgfqpoint{1.414286in}{6.405701in}}%
\pgfpathlineto{\pgfqpoint{1.509184in}{6.405701in}}%
\pgfpathlineto{\pgfqpoint{1.604082in}{6.405701in}}%
\pgfpathlineto{\pgfqpoint{1.698980in}{6.405701in}}%
\pgfpathlineto{\pgfqpoint{1.793878in}{6.405701in}}%
\pgfpathlineto{\pgfqpoint{1.888776in}{6.405701in}}%
\pgfpathlineto{\pgfqpoint{1.983673in}{6.405701in}}%
\pgfpathlineto{\pgfqpoint{2.078571in}{6.405701in}}%
\pgfpathlineto{\pgfqpoint{2.173469in}{6.405701in}}%
\pgfpathlineto{\pgfqpoint{2.268367in}{6.405701in}}%
\pgfpathlineto{\pgfqpoint{2.363265in}{6.405701in}}%
\pgfpathlineto{\pgfqpoint{2.458163in}{6.405701in}}%
\pgfpathlineto{\pgfqpoint{2.553061in}{6.405701in}}%
\pgfpathlineto{\pgfqpoint{2.647959in}{6.405701in}}%
\pgfpathlineto{\pgfqpoint{2.742857in}{6.405701in}}%
\pgfpathlineto{\pgfqpoint{2.837755in}{6.405701in}}%
\pgfpathlineto{\pgfqpoint{2.932653in}{6.405701in}}%
\pgfpathlineto{\pgfqpoint{3.027551in}{6.405701in}}%
\pgfpathlineto{\pgfqpoint{3.122449in}{6.405701in}}%
\pgfpathlineto{\pgfqpoint{3.217347in}{6.405701in}}%
\pgfpathlineto{\pgfqpoint{3.312245in}{6.405701in}}%
\pgfpathlineto{\pgfqpoint{3.407143in}{6.405701in}}%
\pgfpathlineto{\pgfqpoint{3.502041in}{6.405701in}}%
\pgfpathlineto{\pgfqpoint{3.596939in}{6.405701in}}%
\pgfpathlineto{\pgfqpoint{3.691837in}{6.405701in}}%
\pgfpathlineto{\pgfqpoint{3.786735in}{6.405701in}}%
\pgfpathlineto{\pgfqpoint{3.881633in}{6.405701in}}%
\pgfpathlineto{\pgfqpoint{3.976531in}{6.405701in}}%
\pgfpathlineto{\pgfqpoint{4.071429in}{6.405701in}}%
\pgfpathlineto{\pgfqpoint{4.166327in}{6.405701in}}%
\pgfpathlineto{\pgfqpoint{4.261224in}{6.405701in}}%
\pgfpathlineto{\pgfqpoint{4.356122in}{6.405701in}}%
\pgfpathlineto{\pgfqpoint{4.451020in}{6.405701in}}%
\pgfpathlineto{\pgfqpoint{4.545918in}{6.405701in}}%
\pgfpathlineto{\pgfqpoint{4.640816in}{6.405701in}}%
\pgfpathlineto{\pgfqpoint{4.735714in}{6.405701in}}%
\pgfpathlineto{\pgfqpoint{4.830612in}{6.405701in}}%
\pgfpathlineto{\pgfqpoint{4.925510in}{6.405701in}}%
\pgfpathlineto{\pgfqpoint{5.020408in}{6.405701in}}%
\pgfpathlineto{\pgfqpoint{5.115306in}{6.405701in}}%
\pgfpathlineto{\pgfqpoint{5.210204in}{6.405701in}}%
\pgfpathlineto{\pgfqpoint{5.305102in}{6.405701in}}%
\pgfpathlineto{\pgfqpoint{5.400000in}{6.405701in}}%
\pgfusepath{stroke}%
\end{pgfscope}%
\begin{pgfscope}%
\pgfsetrectcap%
\pgfsetmiterjoin%
\pgfsetlinewidth{0.803000pt}%
\definecolor{currentstroke}{rgb}{0.000000,0.000000,0.000000}%
\pgfsetstrokecolor{currentstroke}%
\pgfsetdash{}{0pt}%
\pgfpathmoveto{\pgfqpoint{0.750000in}{3.960000in}}%
\pgfpathlineto{\pgfqpoint{0.750000in}{7.040000in}}%
\pgfusepath{stroke}%
\end{pgfscope}%
\begin{pgfscope}%
\pgfsetrectcap%
\pgfsetmiterjoin%
\pgfsetlinewidth{0.803000pt}%
\definecolor{currentstroke}{rgb}{0.000000,0.000000,0.000000}%
\pgfsetstrokecolor{currentstroke}%
\pgfsetdash{}{0pt}%
\pgfpathmoveto{\pgfqpoint{5.400000in}{3.960000in}}%
\pgfpathlineto{\pgfqpoint{5.400000in}{7.040000in}}%
\pgfusepath{stroke}%
\end{pgfscope}%
\begin{pgfscope}%
\pgfsetrectcap%
\pgfsetmiterjoin%
\pgfsetlinewidth{0.803000pt}%
\definecolor{currentstroke}{rgb}{0.000000,0.000000,0.000000}%
\pgfsetstrokecolor{currentstroke}%
\pgfsetdash{}{0pt}%
\pgfpathmoveto{\pgfqpoint{0.750000in}{3.960000in}}%
\pgfpathlineto{\pgfqpoint{5.400000in}{3.960000in}}%
\pgfusepath{stroke}%
\end{pgfscope}%
\begin{pgfscope}%
\pgfsetrectcap%
\pgfsetmiterjoin%
\pgfsetlinewidth{0.803000pt}%
\definecolor{currentstroke}{rgb}{0.000000,0.000000,0.000000}%
\pgfsetstrokecolor{currentstroke}%
\pgfsetdash{}{0pt}%
\pgfpathmoveto{\pgfqpoint{0.750000in}{7.040000in}}%
\pgfpathlineto{\pgfqpoint{5.400000in}{7.040000in}}%
\pgfusepath{stroke}%
\end{pgfscope}%
\begin{pgfscope}%
\pgfsetbuttcap%
\pgfsetmiterjoin%
\definecolor{currentfill}{rgb}{1.000000,1.000000,1.000000}%
\pgfsetfillcolor{currentfill}%
\pgfsetfillopacity{0.800000}%
\pgfsetlinewidth{1.003750pt}%
\definecolor{currentstroke}{rgb}{0.800000,0.800000,0.800000}%
\pgfsetstrokecolor{currentstroke}%
\pgfsetstrokeopacity{0.800000}%
\pgfsetdash{}{0pt}%
\pgfpathmoveto{\pgfqpoint{2.342004in}{4.029444in}}%
\pgfpathlineto{\pgfqpoint{3.807996in}{4.029444in}}%
\pgfpathquadraticcurveto{\pgfqpoint{3.835774in}{4.029444in}}{\pgfqpoint{3.835774in}{4.057222in}}%
\pgfpathlineto{\pgfqpoint{3.835774in}{4.651311in}}%
\pgfpathquadraticcurveto{\pgfqpoint{3.835774in}{4.679088in}}{\pgfqpoint{3.807996in}{4.679088in}}%
\pgfpathlineto{\pgfqpoint{2.342004in}{4.679088in}}%
\pgfpathquadraticcurveto{\pgfqpoint{2.314226in}{4.679088in}}{\pgfqpoint{2.314226in}{4.651311in}}%
\pgfpathlineto{\pgfqpoint{2.314226in}{4.057222in}}%
\pgfpathquadraticcurveto{\pgfqpoint{2.314226in}{4.029444in}}{\pgfqpoint{2.342004in}{4.029444in}}%
\pgfpathlineto{\pgfqpoint{2.342004in}{4.029444in}}%
\pgfpathclose%
\pgfusepath{stroke,fill}%
\end{pgfscope}%
\begin{pgfscope}%
\pgfsetrectcap%
\pgfsetroundjoin%
\pgfsetlinewidth{2.007500pt}%
\definecolor{currentstroke}{rgb}{0.121569,0.466667,0.705882}%
\pgfsetstrokecolor{currentstroke}%
\pgfsetdash{}{0pt}%
\pgfpathmoveto{\pgfqpoint{2.369781in}{4.568791in}}%
\pgfpathlineto{\pgfqpoint{2.508670in}{4.568791in}}%
\pgfpathlineto{\pgfqpoint{2.647559in}{4.568791in}}%
\pgfusepath{stroke}%
\end{pgfscope}%
\begin{pgfscope}%
\definecolor{textcolor}{rgb}{0.000000,0.000000,0.000000}%
\pgfsetstrokecolor{textcolor}%
\pgfsetfillcolor{textcolor}%
\pgftext[x=2.758670in,y=4.520180in,left,base]{\color{textcolor}\rmfamily\fontsize{10.000000}{12.000000}\selectfont \(\displaystyle P_\mathrm{e}\) pre-fault}%
\end{pgfscope}%
\begin{pgfscope}%
\pgfsetrectcap%
\pgfsetroundjoin%
\pgfsetlinewidth{2.007500pt}%
\definecolor{currentstroke}{rgb}{1.000000,0.498039,0.054902}%
\pgfsetstrokecolor{currentstroke}%
\pgfsetdash{}{0pt}%
\pgfpathmoveto{\pgfqpoint{2.369781in}{4.365748in}}%
\pgfpathlineto{\pgfqpoint{2.508670in}{4.365748in}}%
\pgfpathlineto{\pgfqpoint{2.647559in}{4.365748in}}%
\pgfusepath{stroke}%
\end{pgfscope}%
\begin{pgfscope}%
\definecolor{textcolor}{rgb}{0.000000,0.000000,0.000000}%
\pgfsetstrokecolor{textcolor}%
\pgfsetfillcolor{textcolor}%
\pgftext[x=2.758670in,y=4.317137in,left,base]{\color{textcolor}\rmfamily\fontsize{10.000000}{12.000000}\selectfont \(\displaystyle P_\mathrm{e}\) post-fault}%
\end{pgfscope}%
\begin{pgfscope}%
\pgfsetrectcap%
\pgfsetroundjoin%
\pgfsetlinewidth{2.007500pt}%
\definecolor{currentstroke}{rgb}{0.172549,0.627451,0.172549}%
\pgfsetstrokecolor{currentstroke}%
\pgfsetdash{}{0pt}%
\pgfpathmoveto{\pgfqpoint{2.369781in}{4.163857in}}%
\pgfpathlineto{\pgfqpoint{2.508670in}{4.163857in}}%
\pgfpathlineto{\pgfqpoint{2.647559in}{4.163857in}}%
\pgfusepath{stroke}%
\end{pgfscope}%
\begin{pgfscope}%
\definecolor{textcolor}{rgb}{0.000000,0.000000,0.000000}%
\pgfsetstrokecolor{textcolor}%
\pgfsetfillcolor{textcolor}%
\pgftext[x=2.758670in,y=4.115246in,left,base]{\color{textcolor}\rmfamily\fontsize{10.000000}{12.000000}\selectfont \(\displaystyle P_\mathrm{T}\) of the turbine}%
\end{pgfscope}%
\begin{pgfscope}%
\pgfsetbuttcap%
\pgfsetmiterjoin%
\definecolor{currentfill}{rgb}{1.000000,1.000000,1.000000}%
\pgfsetfillcolor{currentfill}%
\pgfsetlinewidth{0.000000pt}%
\definecolor{currentstroke}{rgb}{0.000000,0.000000,0.000000}%
\pgfsetstrokecolor{currentstroke}%
\pgfsetstrokeopacity{0.000000}%
\pgfsetdash{}{0pt}%
\pgfpathmoveto{\pgfqpoint{0.750000in}{0.880000in}}%
\pgfpathlineto{\pgfqpoint{5.400000in}{0.880000in}}%
\pgfpathlineto{\pgfqpoint{5.400000in}{3.960000in}}%
\pgfpathlineto{\pgfqpoint{0.750000in}{3.960000in}}%
\pgfpathlineto{\pgfqpoint{0.750000in}{0.880000in}}%
\pgfpathclose%
\pgfusepath{fill}%
\end{pgfscope}%
\begin{pgfscope}%
\pgfpathrectangle{\pgfqpoint{0.750000in}{0.880000in}}{\pgfqpoint{4.650000in}{3.080000in}}%
\pgfusepath{clip}%
\pgfsetrectcap%
\pgfsetroundjoin%
\pgfsetlinewidth{0.803000pt}%
\definecolor{currentstroke}{rgb}{0.690196,0.690196,0.690196}%
\pgfsetstrokecolor{currentstroke}%
\pgfsetdash{}{0pt}%
\pgfpathmoveto{\pgfqpoint{0.750000in}{0.880000in}}%
\pgfpathlineto{\pgfqpoint{0.750000in}{3.960000in}}%
\pgfusepath{stroke}%
\end{pgfscope}%
\begin{pgfscope}%
\pgfsetbuttcap%
\pgfsetroundjoin%
\definecolor{currentfill}{rgb}{0.000000,0.000000,0.000000}%
\pgfsetfillcolor{currentfill}%
\pgfsetlinewidth{0.803000pt}%
\definecolor{currentstroke}{rgb}{0.000000,0.000000,0.000000}%
\pgfsetstrokecolor{currentstroke}%
\pgfsetdash{}{0pt}%
\pgfsys@defobject{currentmarker}{\pgfqpoint{0.000000in}{-0.048611in}}{\pgfqpoint{0.000000in}{0.000000in}}{%
\pgfpathmoveto{\pgfqpoint{0.000000in}{0.000000in}}%
\pgfpathlineto{\pgfqpoint{0.000000in}{-0.048611in}}%
\pgfusepath{stroke,fill}%
}%
\begin{pgfscope}%
\pgfsys@transformshift{0.750000in}{0.880000in}%
\pgfsys@useobject{currentmarker}{}%
\end{pgfscope}%
\end{pgfscope}%
\begin{pgfscope}%
\definecolor{textcolor}{rgb}{0.000000,0.000000,0.000000}%
\pgfsetstrokecolor{textcolor}%
\pgfsetfillcolor{textcolor}%
\pgftext[x=0.750000in,y=0.782778in,,top]{\color{textcolor}\rmfamily\fontsize{10.000000}{12.000000}\selectfont \(\displaystyle {0}\)}%
\end{pgfscope}%
\begin{pgfscope}%
\pgfpathrectangle{\pgfqpoint{0.750000in}{0.880000in}}{\pgfqpoint{4.650000in}{3.080000in}}%
\pgfusepath{clip}%
\pgfsetrectcap%
\pgfsetroundjoin%
\pgfsetlinewidth{0.803000pt}%
\definecolor{currentstroke}{rgb}{0.690196,0.690196,0.690196}%
\pgfsetstrokecolor{currentstroke}%
\pgfsetdash{}{0pt}%
\pgfpathmoveto{\pgfqpoint{1.266667in}{0.880000in}}%
\pgfpathlineto{\pgfqpoint{1.266667in}{3.960000in}}%
\pgfusepath{stroke}%
\end{pgfscope}%
\begin{pgfscope}%
\pgfsetbuttcap%
\pgfsetroundjoin%
\definecolor{currentfill}{rgb}{0.000000,0.000000,0.000000}%
\pgfsetfillcolor{currentfill}%
\pgfsetlinewidth{0.803000pt}%
\definecolor{currentstroke}{rgb}{0.000000,0.000000,0.000000}%
\pgfsetstrokecolor{currentstroke}%
\pgfsetdash{}{0pt}%
\pgfsys@defobject{currentmarker}{\pgfqpoint{0.000000in}{-0.048611in}}{\pgfqpoint{0.000000in}{0.000000in}}{%
\pgfpathmoveto{\pgfqpoint{0.000000in}{0.000000in}}%
\pgfpathlineto{\pgfqpoint{0.000000in}{-0.048611in}}%
\pgfusepath{stroke,fill}%
}%
\begin{pgfscope}%
\pgfsys@transformshift{1.266667in}{0.880000in}%
\pgfsys@useobject{currentmarker}{}%
\end{pgfscope}%
\end{pgfscope}%
\begin{pgfscope}%
\definecolor{textcolor}{rgb}{0.000000,0.000000,0.000000}%
\pgfsetstrokecolor{textcolor}%
\pgfsetfillcolor{textcolor}%
\pgftext[x=1.266667in,y=0.782778in,,top]{\color{textcolor}\rmfamily\fontsize{10.000000}{12.000000}\selectfont \(\displaystyle {20}\)}%
\end{pgfscope}%
\begin{pgfscope}%
\pgfpathrectangle{\pgfqpoint{0.750000in}{0.880000in}}{\pgfqpoint{4.650000in}{3.080000in}}%
\pgfusepath{clip}%
\pgfsetrectcap%
\pgfsetroundjoin%
\pgfsetlinewidth{0.803000pt}%
\definecolor{currentstroke}{rgb}{0.690196,0.690196,0.690196}%
\pgfsetstrokecolor{currentstroke}%
\pgfsetdash{}{0pt}%
\pgfpathmoveto{\pgfqpoint{1.783333in}{0.880000in}}%
\pgfpathlineto{\pgfqpoint{1.783333in}{3.960000in}}%
\pgfusepath{stroke}%
\end{pgfscope}%
\begin{pgfscope}%
\pgfsetbuttcap%
\pgfsetroundjoin%
\definecolor{currentfill}{rgb}{0.000000,0.000000,0.000000}%
\pgfsetfillcolor{currentfill}%
\pgfsetlinewidth{0.803000pt}%
\definecolor{currentstroke}{rgb}{0.000000,0.000000,0.000000}%
\pgfsetstrokecolor{currentstroke}%
\pgfsetdash{}{0pt}%
\pgfsys@defobject{currentmarker}{\pgfqpoint{0.000000in}{-0.048611in}}{\pgfqpoint{0.000000in}{0.000000in}}{%
\pgfpathmoveto{\pgfqpoint{0.000000in}{0.000000in}}%
\pgfpathlineto{\pgfqpoint{0.000000in}{-0.048611in}}%
\pgfusepath{stroke,fill}%
}%
\begin{pgfscope}%
\pgfsys@transformshift{1.783333in}{0.880000in}%
\pgfsys@useobject{currentmarker}{}%
\end{pgfscope}%
\end{pgfscope}%
\begin{pgfscope}%
\definecolor{textcolor}{rgb}{0.000000,0.000000,0.000000}%
\pgfsetstrokecolor{textcolor}%
\pgfsetfillcolor{textcolor}%
\pgftext[x=1.783333in,y=0.782778in,,top]{\color{textcolor}\rmfamily\fontsize{10.000000}{12.000000}\selectfont \(\displaystyle {40}\)}%
\end{pgfscope}%
\begin{pgfscope}%
\pgfpathrectangle{\pgfqpoint{0.750000in}{0.880000in}}{\pgfqpoint{4.650000in}{3.080000in}}%
\pgfusepath{clip}%
\pgfsetrectcap%
\pgfsetroundjoin%
\pgfsetlinewidth{0.803000pt}%
\definecolor{currentstroke}{rgb}{0.690196,0.690196,0.690196}%
\pgfsetstrokecolor{currentstroke}%
\pgfsetdash{}{0pt}%
\pgfpathmoveto{\pgfqpoint{2.300000in}{0.880000in}}%
\pgfpathlineto{\pgfqpoint{2.300000in}{3.960000in}}%
\pgfusepath{stroke}%
\end{pgfscope}%
\begin{pgfscope}%
\pgfsetbuttcap%
\pgfsetroundjoin%
\definecolor{currentfill}{rgb}{0.000000,0.000000,0.000000}%
\pgfsetfillcolor{currentfill}%
\pgfsetlinewidth{0.803000pt}%
\definecolor{currentstroke}{rgb}{0.000000,0.000000,0.000000}%
\pgfsetstrokecolor{currentstroke}%
\pgfsetdash{}{0pt}%
\pgfsys@defobject{currentmarker}{\pgfqpoint{0.000000in}{-0.048611in}}{\pgfqpoint{0.000000in}{0.000000in}}{%
\pgfpathmoveto{\pgfqpoint{0.000000in}{0.000000in}}%
\pgfpathlineto{\pgfqpoint{0.000000in}{-0.048611in}}%
\pgfusepath{stroke,fill}%
}%
\begin{pgfscope}%
\pgfsys@transformshift{2.300000in}{0.880000in}%
\pgfsys@useobject{currentmarker}{}%
\end{pgfscope}%
\end{pgfscope}%
\begin{pgfscope}%
\definecolor{textcolor}{rgb}{0.000000,0.000000,0.000000}%
\pgfsetstrokecolor{textcolor}%
\pgfsetfillcolor{textcolor}%
\pgftext[x=2.300000in,y=0.782778in,,top]{\color{textcolor}\rmfamily\fontsize{10.000000}{12.000000}\selectfont \(\displaystyle {60}\)}%
\end{pgfscope}%
\begin{pgfscope}%
\pgfpathrectangle{\pgfqpoint{0.750000in}{0.880000in}}{\pgfqpoint{4.650000in}{3.080000in}}%
\pgfusepath{clip}%
\pgfsetrectcap%
\pgfsetroundjoin%
\pgfsetlinewidth{0.803000pt}%
\definecolor{currentstroke}{rgb}{0.690196,0.690196,0.690196}%
\pgfsetstrokecolor{currentstroke}%
\pgfsetdash{}{0pt}%
\pgfpathmoveto{\pgfqpoint{2.816667in}{0.880000in}}%
\pgfpathlineto{\pgfqpoint{2.816667in}{3.960000in}}%
\pgfusepath{stroke}%
\end{pgfscope}%
\begin{pgfscope}%
\pgfsetbuttcap%
\pgfsetroundjoin%
\definecolor{currentfill}{rgb}{0.000000,0.000000,0.000000}%
\pgfsetfillcolor{currentfill}%
\pgfsetlinewidth{0.803000pt}%
\definecolor{currentstroke}{rgb}{0.000000,0.000000,0.000000}%
\pgfsetstrokecolor{currentstroke}%
\pgfsetdash{}{0pt}%
\pgfsys@defobject{currentmarker}{\pgfqpoint{0.000000in}{-0.048611in}}{\pgfqpoint{0.000000in}{0.000000in}}{%
\pgfpathmoveto{\pgfqpoint{0.000000in}{0.000000in}}%
\pgfpathlineto{\pgfqpoint{0.000000in}{-0.048611in}}%
\pgfusepath{stroke,fill}%
}%
\begin{pgfscope}%
\pgfsys@transformshift{2.816667in}{0.880000in}%
\pgfsys@useobject{currentmarker}{}%
\end{pgfscope}%
\end{pgfscope}%
\begin{pgfscope}%
\definecolor{textcolor}{rgb}{0.000000,0.000000,0.000000}%
\pgfsetstrokecolor{textcolor}%
\pgfsetfillcolor{textcolor}%
\pgftext[x=2.816667in,y=0.782778in,,top]{\color{textcolor}\rmfamily\fontsize{10.000000}{12.000000}\selectfont \(\displaystyle {80}\)}%
\end{pgfscope}%
\begin{pgfscope}%
\pgfpathrectangle{\pgfqpoint{0.750000in}{0.880000in}}{\pgfqpoint{4.650000in}{3.080000in}}%
\pgfusepath{clip}%
\pgfsetrectcap%
\pgfsetroundjoin%
\pgfsetlinewidth{0.803000pt}%
\definecolor{currentstroke}{rgb}{0.690196,0.690196,0.690196}%
\pgfsetstrokecolor{currentstroke}%
\pgfsetdash{}{0pt}%
\pgfpathmoveto{\pgfqpoint{3.333333in}{0.880000in}}%
\pgfpathlineto{\pgfqpoint{3.333333in}{3.960000in}}%
\pgfusepath{stroke}%
\end{pgfscope}%
\begin{pgfscope}%
\pgfsetbuttcap%
\pgfsetroundjoin%
\definecolor{currentfill}{rgb}{0.000000,0.000000,0.000000}%
\pgfsetfillcolor{currentfill}%
\pgfsetlinewidth{0.803000pt}%
\definecolor{currentstroke}{rgb}{0.000000,0.000000,0.000000}%
\pgfsetstrokecolor{currentstroke}%
\pgfsetdash{}{0pt}%
\pgfsys@defobject{currentmarker}{\pgfqpoint{0.000000in}{-0.048611in}}{\pgfqpoint{0.000000in}{0.000000in}}{%
\pgfpathmoveto{\pgfqpoint{0.000000in}{0.000000in}}%
\pgfpathlineto{\pgfqpoint{0.000000in}{-0.048611in}}%
\pgfusepath{stroke,fill}%
}%
\begin{pgfscope}%
\pgfsys@transformshift{3.333333in}{0.880000in}%
\pgfsys@useobject{currentmarker}{}%
\end{pgfscope}%
\end{pgfscope}%
\begin{pgfscope}%
\definecolor{textcolor}{rgb}{0.000000,0.000000,0.000000}%
\pgfsetstrokecolor{textcolor}%
\pgfsetfillcolor{textcolor}%
\pgftext[x=3.333333in,y=0.782778in,,top]{\color{textcolor}\rmfamily\fontsize{10.000000}{12.000000}\selectfont \(\displaystyle {100}\)}%
\end{pgfscope}%
\begin{pgfscope}%
\pgfpathrectangle{\pgfqpoint{0.750000in}{0.880000in}}{\pgfqpoint{4.650000in}{3.080000in}}%
\pgfusepath{clip}%
\pgfsetrectcap%
\pgfsetroundjoin%
\pgfsetlinewidth{0.803000pt}%
\definecolor{currentstroke}{rgb}{0.690196,0.690196,0.690196}%
\pgfsetstrokecolor{currentstroke}%
\pgfsetdash{}{0pt}%
\pgfpathmoveto{\pgfqpoint{3.850000in}{0.880000in}}%
\pgfpathlineto{\pgfqpoint{3.850000in}{3.960000in}}%
\pgfusepath{stroke}%
\end{pgfscope}%
\begin{pgfscope}%
\pgfsetbuttcap%
\pgfsetroundjoin%
\definecolor{currentfill}{rgb}{0.000000,0.000000,0.000000}%
\pgfsetfillcolor{currentfill}%
\pgfsetlinewidth{0.803000pt}%
\definecolor{currentstroke}{rgb}{0.000000,0.000000,0.000000}%
\pgfsetstrokecolor{currentstroke}%
\pgfsetdash{}{0pt}%
\pgfsys@defobject{currentmarker}{\pgfqpoint{0.000000in}{-0.048611in}}{\pgfqpoint{0.000000in}{0.000000in}}{%
\pgfpathmoveto{\pgfqpoint{0.000000in}{0.000000in}}%
\pgfpathlineto{\pgfqpoint{0.000000in}{-0.048611in}}%
\pgfusepath{stroke,fill}%
}%
\begin{pgfscope}%
\pgfsys@transformshift{3.850000in}{0.880000in}%
\pgfsys@useobject{currentmarker}{}%
\end{pgfscope}%
\end{pgfscope}%
\begin{pgfscope}%
\definecolor{textcolor}{rgb}{0.000000,0.000000,0.000000}%
\pgfsetstrokecolor{textcolor}%
\pgfsetfillcolor{textcolor}%
\pgftext[x=3.850000in,y=0.782778in,,top]{\color{textcolor}\rmfamily\fontsize{10.000000}{12.000000}\selectfont \(\displaystyle {120}\)}%
\end{pgfscope}%
\begin{pgfscope}%
\pgfpathrectangle{\pgfqpoint{0.750000in}{0.880000in}}{\pgfqpoint{4.650000in}{3.080000in}}%
\pgfusepath{clip}%
\pgfsetrectcap%
\pgfsetroundjoin%
\pgfsetlinewidth{0.803000pt}%
\definecolor{currentstroke}{rgb}{0.690196,0.690196,0.690196}%
\pgfsetstrokecolor{currentstroke}%
\pgfsetdash{}{0pt}%
\pgfpathmoveto{\pgfqpoint{4.366667in}{0.880000in}}%
\pgfpathlineto{\pgfqpoint{4.366667in}{3.960000in}}%
\pgfusepath{stroke}%
\end{pgfscope}%
\begin{pgfscope}%
\pgfsetbuttcap%
\pgfsetroundjoin%
\definecolor{currentfill}{rgb}{0.000000,0.000000,0.000000}%
\pgfsetfillcolor{currentfill}%
\pgfsetlinewidth{0.803000pt}%
\definecolor{currentstroke}{rgb}{0.000000,0.000000,0.000000}%
\pgfsetstrokecolor{currentstroke}%
\pgfsetdash{}{0pt}%
\pgfsys@defobject{currentmarker}{\pgfqpoint{0.000000in}{-0.048611in}}{\pgfqpoint{0.000000in}{0.000000in}}{%
\pgfpathmoveto{\pgfqpoint{0.000000in}{0.000000in}}%
\pgfpathlineto{\pgfqpoint{0.000000in}{-0.048611in}}%
\pgfusepath{stroke,fill}%
}%
\begin{pgfscope}%
\pgfsys@transformshift{4.366667in}{0.880000in}%
\pgfsys@useobject{currentmarker}{}%
\end{pgfscope}%
\end{pgfscope}%
\begin{pgfscope}%
\definecolor{textcolor}{rgb}{0.000000,0.000000,0.000000}%
\pgfsetstrokecolor{textcolor}%
\pgfsetfillcolor{textcolor}%
\pgftext[x=4.366667in,y=0.782778in,,top]{\color{textcolor}\rmfamily\fontsize{10.000000}{12.000000}\selectfont \(\displaystyle {140}\)}%
\end{pgfscope}%
\begin{pgfscope}%
\pgfpathrectangle{\pgfqpoint{0.750000in}{0.880000in}}{\pgfqpoint{4.650000in}{3.080000in}}%
\pgfusepath{clip}%
\pgfsetrectcap%
\pgfsetroundjoin%
\pgfsetlinewidth{0.803000pt}%
\definecolor{currentstroke}{rgb}{0.690196,0.690196,0.690196}%
\pgfsetstrokecolor{currentstroke}%
\pgfsetdash{}{0pt}%
\pgfpathmoveto{\pgfqpoint{4.883333in}{0.880000in}}%
\pgfpathlineto{\pgfqpoint{4.883333in}{3.960000in}}%
\pgfusepath{stroke}%
\end{pgfscope}%
\begin{pgfscope}%
\pgfsetbuttcap%
\pgfsetroundjoin%
\definecolor{currentfill}{rgb}{0.000000,0.000000,0.000000}%
\pgfsetfillcolor{currentfill}%
\pgfsetlinewidth{0.803000pt}%
\definecolor{currentstroke}{rgb}{0.000000,0.000000,0.000000}%
\pgfsetstrokecolor{currentstroke}%
\pgfsetdash{}{0pt}%
\pgfsys@defobject{currentmarker}{\pgfqpoint{0.000000in}{-0.048611in}}{\pgfqpoint{0.000000in}{0.000000in}}{%
\pgfpathmoveto{\pgfqpoint{0.000000in}{0.000000in}}%
\pgfpathlineto{\pgfqpoint{0.000000in}{-0.048611in}}%
\pgfusepath{stroke,fill}%
}%
\begin{pgfscope}%
\pgfsys@transformshift{4.883333in}{0.880000in}%
\pgfsys@useobject{currentmarker}{}%
\end{pgfscope}%
\end{pgfscope}%
\begin{pgfscope}%
\definecolor{textcolor}{rgb}{0.000000,0.000000,0.000000}%
\pgfsetstrokecolor{textcolor}%
\pgfsetfillcolor{textcolor}%
\pgftext[x=4.883333in,y=0.782778in,,top]{\color{textcolor}\rmfamily\fontsize{10.000000}{12.000000}\selectfont \(\displaystyle {160}\)}%
\end{pgfscope}%
\begin{pgfscope}%
\pgfpathrectangle{\pgfqpoint{0.750000in}{0.880000in}}{\pgfqpoint{4.650000in}{3.080000in}}%
\pgfusepath{clip}%
\pgfsetrectcap%
\pgfsetroundjoin%
\pgfsetlinewidth{0.803000pt}%
\definecolor{currentstroke}{rgb}{0.690196,0.690196,0.690196}%
\pgfsetstrokecolor{currentstroke}%
\pgfsetdash{}{0pt}%
\pgfpathmoveto{\pgfqpoint{5.400000in}{0.880000in}}%
\pgfpathlineto{\pgfqpoint{5.400000in}{3.960000in}}%
\pgfusepath{stroke}%
\end{pgfscope}%
\begin{pgfscope}%
\pgfsetbuttcap%
\pgfsetroundjoin%
\definecolor{currentfill}{rgb}{0.000000,0.000000,0.000000}%
\pgfsetfillcolor{currentfill}%
\pgfsetlinewidth{0.803000pt}%
\definecolor{currentstroke}{rgb}{0.000000,0.000000,0.000000}%
\pgfsetstrokecolor{currentstroke}%
\pgfsetdash{}{0pt}%
\pgfsys@defobject{currentmarker}{\pgfqpoint{0.000000in}{-0.048611in}}{\pgfqpoint{0.000000in}{0.000000in}}{%
\pgfpathmoveto{\pgfqpoint{0.000000in}{0.000000in}}%
\pgfpathlineto{\pgfqpoint{0.000000in}{-0.048611in}}%
\pgfusepath{stroke,fill}%
}%
\begin{pgfscope}%
\pgfsys@transformshift{5.400000in}{0.880000in}%
\pgfsys@useobject{currentmarker}{}%
\end{pgfscope}%
\end{pgfscope}%
\begin{pgfscope}%
\definecolor{textcolor}{rgb}{0.000000,0.000000,0.000000}%
\pgfsetstrokecolor{textcolor}%
\pgfsetfillcolor{textcolor}%
\pgftext[x=5.400000in,y=0.782778in,,top]{\color{textcolor}\rmfamily\fontsize{10.000000}{12.000000}\selectfont \(\displaystyle {180}\)}%
\end{pgfscope}%
\begin{pgfscope}%
\definecolor{textcolor}{rgb}{0.000000,0.000000,0.000000}%
\pgfsetstrokecolor{textcolor}%
\pgfsetfillcolor{textcolor}%
\pgftext[x=3.075000in,y=0.594776in,,top]{\color{textcolor}\rmfamily\fontsize{10.000000}{12.000000}\selectfont power angle \(\displaystyle \delta\) in deg}%
\end{pgfscope}%
\begin{pgfscope}%
\pgfpathrectangle{\pgfqpoint{0.750000in}{0.880000in}}{\pgfqpoint{4.650000in}{3.080000in}}%
\pgfusepath{clip}%
\pgfsetrectcap%
\pgfsetroundjoin%
\pgfsetlinewidth{0.803000pt}%
\definecolor{currentstroke}{rgb}{0.690196,0.690196,0.690196}%
\pgfsetstrokecolor{currentstroke}%
\pgfsetdash{}{0pt}%
\pgfpathmoveto{\pgfqpoint{0.750000in}{3.823047in}}%
\pgfpathlineto{\pgfqpoint{5.400000in}{3.823047in}}%
\pgfusepath{stroke}%
\end{pgfscope}%
\begin{pgfscope}%
\pgfsetbuttcap%
\pgfsetroundjoin%
\definecolor{currentfill}{rgb}{0.000000,0.000000,0.000000}%
\pgfsetfillcolor{currentfill}%
\pgfsetlinewidth{0.803000pt}%
\definecolor{currentstroke}{rgb}{0.000000,0.000000,0.000000}%
\pgfsetstrokecolor{currentstroke}%
\pgfsetdash{}{0pt}%
\pgfsys@defobject{currentmarker}{\pgfqpoint{-0.048611in}{0.000000in}}{\pgfqpoint{-0.000000in}{0.000000in}}{%
\pgfpathmoveto{\pgfqpoint{-0.000000in}{0.000000in}}%
\pgfpathlineto{\pgfqpoint{-0.048611in}{0.000000in}}%
\pgfusepath{stroke,fill}%
}%
\begin{pgfscope}%
\pgfsys@transformshift{0.750000in}{3.823047in}%
\pgfsys@useobject{currentmarker}{}%
\end{pgfscope}%
\end{pgfscope}%
\begin{pgfscope}%
\definecolor{textcolor}{rgb}{0.000000,0.000000,0.000000}%
\pgfsetstrokecolor{textcolor}%
\pgfsetfillcolor{textcolor}%
\pgftext[x=0.405863in, y=3.771947in, left, base]{\color{textcolor}\rmfamily\fontsize{10.000000}{12.000000}\selectfont \(\displaystyle {0.00}\)}%
\end{pgfscope}%
\begin{pgfscope}%
\pgfpathrectangle{\pgfqpoint{0.750000in}{0.880000in}}{\pgfqpoint{4.650000in}{3.080000in}}%
\pgfusepath{clip}%
\pgfsetrectcap%
\pgfsetroundjoin%
\pgfsetlinewidth{0.803000pt}%
\definecolor{currentstroke}{rgb}{0.690196,0.690196,0.690196}%
\pgfsetstrokecolor{currentstroke}%
\pgfsetdash{}{0pt}%
\pgfpathmoveto{\pgfqpoint{0.750000in}{3.480665in}}%
\pgfpathlineto{\pgfqpoint{5.400000in}{3.480665in}}%
\pgfusepath{stroke}%
\end{pgfscope}%
\begin{pgfscope}%
\pgfsetbuttcap%
\pgfsetroundjoin%
\definecolor{currentfill}{rgb}{0.000000,0.000000,0.000000}%
\pgfsetfillcolor{currentfill}%
\pgfsetlinewidth{0.803000pt}%
\definecolor{currentstroke}{rgb}{0.000000,0.000000,0.000000}%
\pgfsetstrokecolor{currentstroke}%
\pgfsetdash{}{0pt}%
\pgfsys@defobject{currentmarker}{\pgfqpoint{-0.048611in}{0.000000in}}{\pgfqpoint{-0.000000in}{0.000000in}}{%
\pgfpathmoveto{\pgfqpoint{-0.000000in}{0.000000in}}%
\pgfpathlineto{\pgfqpoint{-0.048611in}{0.000000in}}%
\pgfusepath{stroke,fill}%
}%
\begin{pgfscope}%
\pgfsys@transformshift{0.750000in}{3.480665in}%
\pgfsys@useobject{currentmarker}{}%
\end{pgfscope}%
\end{pgfscope}%
\begin{pgfscope}%
\definecolor{textcolor}{rgb}{0.000000,0.000000,0.000000}%
\pgfsetstrokecolor{textcolor}%
\pgfsetfillcolor{textcolor}%
\pgftext[x=0.405863in, y=3.429565in, left, base]{\color{textcolor}\rmfamily\fontsize{10.000000}{12.000000}\selectfont \(\displaystyle {0.25}\)}%
\end{pgfscope}%
\begin{pgfscope}%
\pgfpathrectangle{\pgfqpoint{0.750000in}{0.880000in}}{\pgfqpoint{4.650000in}{3.080000in}}%
\pgfusepath{clip}%
\pgfsetrectcap%
\pgfsetroundjoin%
\pgfsetlinewidth{0.803000pt}%
\definecolor{currentstroke}{rgb}{0.690196,0.690196,0.690196}%
\pgfsetstrokecolor{currentstroke}%
\pgfsetdash{}{0pt}%
\pgfpathmoveto{\pgfqpoint{0.750000in}{3.138283in}}%
\pgfpathlineto{\pgfqpoint{5.400000in}{3.138283in}}%
\pgfusepath{stroke}%
\end{pgfscope}%
\begin{pgfscope}%
\pgfsetbuttcap%
\pgfsetroundjoin%
\definecolor{currentfill}{rgb}{0.000000,0.000000,0.000000}%
\pgfsetfillcolor{currentfill}%
\pgfsetlinewidth{0.803000pt}%
\definecolor{currentstroke}{rgb}{0.000000,0.000000,0.000000}%
\pgfsetstrokecolor{currentstroke}%
\pgfsetdash{}{0pt}%
\pgfsys@defobject{currentmarker}{\pgfqpoint{-0.048611in}{0.000000in}}{\pgfqpoint{-0.000000in}{0.000000in}}{%
\pgfpathmoveto{\pgfqpoint{-0.000000in}{0.000000in}}%
\pgfpathlineto{\pgfqpoint{-0.048611in}{0.000000in}}%
\pgfusepath{stroke,fill}%
}%
\begin{pgfscope}%
\pgfsys@transformshift{0.750000in}{3.138283in}%
\pgfsys@useobject{currentmarker}{}%
\end{pgfscope}%
\end{pgfscope}%
\begin{pgfscope}%
\definecolor{textcolor}{rgb}{0.000000,0.000000,0.000000}%
\pgfsetstrokecolor{textcolor}%
\pgfsetfillcolor{textcolor}%
\pgftext[x=0.405863in, y=3.087183in, left, base]{\color{textcolor}\rmfamily\fontsize{10.000000}{12.000000}\selectfont \(\displaystyle {0.50}\)}%
\end{pgfscope}%
\begin{pgfscope}%
\pgfpathrectangle{\pgfqpoint{0.750000in}{0.880000in}}{\pgfqpoint{4.650000in}{3.080000in}}%
\pgfusepath{clip}%
\pgfsetrectcap%
\pgfsetroundjoin%
\pgfsetlinewidth{0.803000pt}%
\definecolor{currentstroke}{rgb}{0.690196,0.690196,0.690196}%
\pgfsetstrokecolor{currentstroke}%
\pgfsetdash{}{0pt}%
\pgfpathmoveto{\pgfqpoint{0.750000in}{2.795901in}}%
\pgfpathlineto{\pgfqpoint{5.400000in}{2.795901in}}%
\pgfusepath{stroke}%
\end{pgfscope}%
\begin{pgfscope}%
\pgfsetbuttcap%
\pgfsetroundjoin%
\definecolor{currentfill}{rgb}{0.000000,0.000000,0.000000}%
\pgfsetfillcolor{currentfill}%
\pgfsetlinewidth{0.803000pt}%
\definecolor{currentstroke}{rgb}{0.000000,0.000000,0.000000}%
\pgfsetstrokecolor{currentstroke}%
\pgfsetdash{}{0pt}%
\pgfsys@defobject{currentmarker}{\pgfqpoint{-0.048611in}{0.000000in}}{\pgfqpoint{-0.000000in}{0.000000in}}{%
\pgfpathmoveto{\pgfqpoint{-0.000000in}{0.000000in}}%
\pgfpathlineto{\pgfqpoint{-0.048611in}{0.000000in}}%
\pgfusepath{stroke,fill}%
}%
\begin{pgfscope}%
\pgfsys@transformshift{0.750000in}{2.795901in}%
\pgfsys@useobject{currentmarker}{}%
\end{pgfscope}%
\end{pgfscope}%
\begin{pgfscope}%
\definecolor{textcolor}{rgb}{0.000000,0.000000,0.000000}%
\pgfsetstrokecolor{textcolor}%
\pgfsetfillcolor{textcolor}%
\pgftext[x=0.405863in, y=2.744801in, left, base]{\color{textcolor}\rmfamily\fontsize{10.000000}{12.000000}\selectfont \(\displaystyle {0.75}\)}%
\end{pgfscope}%
\begin{pgfscope}%
\pgfpathrectangle{\pgfqpoint{0.750000in}{0.880000in}}{\pgfqpoint{4.650000in}{3.080000in}}%
\pgfusepath{clip}%
\pgfsetrectcap%
\pgfsetroundjoin%
\pgfsetlinewidth{0.803000pt}%
\definecolor{currentstroke}{rgb}{0.690196,0.690196,0.690196}%
\pgfsetstrokecolor{currentstroke}%
\pgfsetdash{}{0pt}%
\pgfpathmoveto{\pgfqpoint{0.750000in}{2.453519in}}%
\pgfpathlineto{\pgfqpoint{5.400000in}{2.453519in}}%
\pgfusepath{stroke}%
\end{pgfscope}%
\begin{pgfscope}%
\pgfsetbuttcap%
\pgfsetroundjoin%
\definecolor{currentfill}{rgb}{0.000000,0.000000,0.000000}%
\pgfsetfillcolor{currentfill}%
\pgfsetlinewidth{0.803000pt}%
\definecolor{currentstroke}{rgb}{0.000000,0.000000,0.000000}%
\pgfsetstrokecolor{currentstroke}%
\pgfsetdash{}{0pt}%
\pgfsys@defobject{currentmarker}{\pgfqpoint{-0.048611in}{0.000000in}}{\pgfqpoint{-0.000000in}{0.000000in}}{%
\pgfpathmoveto{\pgfqpoint{-0.000000in}{0.000000in}}%
\pgfpathlineto{\pgfqpoint{-0.048611in}{0.000000in}}%
\pgfusepath{stroke,fill}%
}%
\begin{pgfscope}%
\pgfsys@transformshift{0.750000in}{2.453519in}%
\pgfsys@useobject{currentmarker}{}%
\end{pgfscope}%
\end{pgfscope}%
\begin{pgfscope}%
\definecolor{textcolor}{rgb}{0.000000,0.000000,0.000000}%
\pgfsetstrokecolor{textcolor}%
\pgfsetfillcolor{textcolor}%
\pgftext[x=0.405863in, y=2.402419in, left, base]{\color{textcolor}\rmfamily\fontsize{10.000000}{12.000000}\selectfont \(\displaystyle {1.00}\)}%
\end{pgfscope}%
\begin{pgfscope}%
\pgfpathrectangle{\pgfqpoint{0.750000in}{0.880000in}}{\pgfqpoint{4.650000in}{3.080000in}}%
\pgfusepath{clip}%
\pgfsetrectcap%
\pgfsetroundjoin%
\pgfsetlinewidth{0.803000pt}%
\definecolor{currentstroke}{rgb}{0.690196,0.690196,0.690196}%
\pgfsetstrokecolor{currentstroke}%
\pgfsetdash{}{0pt}%
\pgfpathmoveto{\pgfqpoint{0.750000in}{2.111137in}}%
\pgfpathlineto{\pgfqpoint{5.400000in}{2.111137in}}%
\pgfusepath{stroke}%
\end{pgfscope}%
\begin{pgfscope}%
\pgfsetbuttcap%
\pgfsetroundjoin%
\definecolor{currentfill}{rgb}{0.000000,0.000000,0.000000}%
\pgfsetfillcolor{currentfill}%
\pgfsetlinewidth{0.803000pt}%
\definecolor{currentstroke}{rgb}{0.000000,0.000000,0.000000}%
\pgfsetstrokecolor{currentstroke}%
\pgfsetdash{}{0pt}%
\pgfsys@defobject{currentmarker}{\pgfqpoint{-0.048611in}{0.000000in}}{\pgfqpoint{-0.000000in}{0.000000in}}{%
\pgfpathmoveto{\pgfqpoint{-0.000000in}{0.000000in}}%
\pgfpathlineto{\pgfqpoint{-0.048611in}{0.000000in}}%
\pgfusepath{stroke,fill}%
}%
\begin{pgfscope}%
\pgfsys@transformshift{0.750000in}{2.111137in}%
\pgfsys@useobject{currentmarker}{}%
\end{pgfscope}%
\end{pgfscope}%
\begin{pgfscope}%
\definecolor{textcolor}{rgb}{0.000000,0.000000,0.000000}%
\pgfsetstrokecolor{textcolor}%
\pgfsetfillcolor{textcolor}%
\pgftext[x=0.405863in, y=2.060037in, left, base]{\color{textcolor}\rmfamily\fontsize{10.000000}{12.000000}\selectfont \(\displaystyle {1.25}\)}%
\end{pgfscope}%
\begin{pgfscope}%
\pgfpathrectangle{\pgfqpoint{0.750000in}{0.880000in}}{\pgfqpoint{4.650000in}{3.080000in}}%
\pgfusepath{clip}%
\pgfsetrectcap%
\pgfsetroundjoin%
\pgfsetlinewidth{0.803000pt}%
\definecolor{currentstroke}{rgb}{0.690196,0.690196,0.690196}%
\pgfsetstrokecolor{currentstroke}%
\pgfsetdash{}{0pt}%
\pgfpathmoveto{\pgfqpoint{0.750000in}{1.768755in}}%
\pgfpathlineto{\pgfqpoint{5.400000in}{1.768755in}}%
\pgfusepath{stroke}%
\end{pgfscope}%
\begin{pgfscope}%
\pgfsetbuttcap%
\pgfsetroundjoin%
\definecolor{currentfill}{rgb}{0.000000,0.000000,0.000000}%
\pgfsetfillcolor{currentfill}%
\pgfsetlinewidth{0.803000pt}%
\definecolor{currentstroke}{rgb}{0.000000,0.000000,0.000000}%
\pgfsetstrokecolor{currentstroke}%
\pgfsetdash{}{0pt}%
\pgfsys@defobject{currentmarker}{\pgfqpoint{-0.048611in}{0.000000in}}{\pgfqpoint{-0.000000in}{0.000000in}}{%
\pgfpathmoveto{\pgfqpoint{-0.000000in}{0.000000in}}%
\pgfpathlineto{\pgfqpoint{-0.048611in}{0.000000in}}%
\pgfusepath{stroke,fill}%
}%
\begin{pgfscope}%
\pgfsys@transformshift{0.750000in}{1.768755in}%
\pgfsys@useobject{currentmarker}{}%
\end{pgfscope}%
\end{pgfscope}%
\begin{pgfscope}%
\definecolor{textcolor}{rgb}{0.000000,0.000000,0.000000}%
\pgfsetstrokecolor{textcolor}%
\pgfsetfillcolor{textcolor}%
\pgftext[x=0.405863in, y=1.717655in, left, base]{\color{textcolor}\rmfamily\fontsize{10.000000}{12.000000}\selectfont \(\displaystyle {1.50}\)}%
\end{pgfscope}%
\begin{pgfscope}%
\pgfpathrectangle{\pgfqpoint{0.750000in}{0.880000in}}{\pgfqpoint{4.650000in}{3.080000in}}%
\pgfusepath{clip}%
\pgfsetrectcap%
\pgfsetroundjoin%
\pgfsetlinewidth{0.803000pt}%
\definecolor{currentstroke}{rgb}{0.690196,0.690196,0.690196}%
\pgfsetstrokecolor{currentstroke}%
\pgfsetdash{}{0pt}%
\pgfpathmoveto{\pgfqpoint{0.750000in}{1.426373in}}%
\pgfpathlineto{\pgfqpoint{5.400000in}{1.426373in}}%
\pgfusepath{stroke}%
\end{pgfscope}%
\begin{pgfscope}%
\pgfsetbuttcap%
\pgfsetroundjoin%
\definecolor{currentfill}{rgb}{0.000000,0.000000,0.000000}%
\pgfsetfillcolor{currentfill}%
\pgfsetlinewidth{0.803000pt}%
\definecolor{currentstroke}{rgb}{0.000000,0.000000,0.000000}%
\pgfsetstrokecolor{currentstroke}%
\pgfsetdash{}{0pt}%
\pgfsys@defobject{currentmarker}{\pgfqpoint{-0.048611in}{0.000000in}}{\pgfqpoint{-0.000000in}{0.000000in}}{%
\pgfpathmoveto{\pgfqpoint{-0.000000in}{0.000000in}}%
\pgfpathlineto{\pgfqpoint{-0.048611in}{0.000000in}}%
\pgfusepath{stroke,fill}%
}%
\begin{pgfscope}%
\pgfsys@transformshift{0.750000in}{1.426373in}%
\pgfsys@useobject{currentmarker}{}%
\end{pgfscope}%
\end{pgfscope}%
\begin{pgfscope}%
\definecolor{textcolor}{rgb}{0.000000,0.000000,0.000000}%
\pgfsetstrokecolor{textcolor}%
\pgfsetfillcolor{textcolor}%
\pgftext[x=0.405863in, y=1.375273in, left, base]{\color{textcolor}\rmfamily\fontsize{10.000000}{12.000000}\selectfont \(\displaystyle {1.75}\)}%
\end{pgfscope}%
\begin{pgfscope}%
\pgfpathrectangle{\pgfqpoint{0.750000in}{0.880000in}}{\pgfqpoint{4.650000in}{3.080000in}}%
\pgfusepath{clip}%
\pgfsetrectcap%
\pgfsetroundjoin%
\pgfsetlinewidth{0.803000pt}%
\definecolor{currentstroke}{rgb}{0.690196,0.690196,0.690196}%
\pgfsetstrokecolor{currentstroke}%
\pgfsetdash{}{0pt}%
\pgfpathmoveto{\pgfqpoint{0.750000in}{1.083991in}}%
\pgfpathlineto{\pgfqpoint{5.400000in}{1.083991in}}%
\pgfusepath{stroke}%
\end{pgfscope}%
\begin{pgfscope}%
\pgfsetbuttcap%
\pgfsetroundjoin%
\definecolor{currentfill}{rgb}{0.000000,0.000000,0.000000}%
\pgfsetfillcolor{currentfill}%
\pgfsetlinewidth{0.803000pt}%
\definecolor{currentstroke}{rgb}{0.000000,0.000000,0.000000}%
\pgfsetstrokecolor{currentstroke}%
\pgfsetdash{}{0pt}%
\pgfsys@defobject{currentmarker}{\pgfqpoint{-0.048611in}{0.000000in}}{\pgfqpoint{-0.000000in}{0.000000in}}{%
\pgfpathmoveto{\pgfqpoint{-0.000000in}{0.000000in}}%
\pgfpathlineto{\pgfqpoint{-0.048611in}{0.000000in}}%
\pgfusepath{stroke,fill}%
}%
\begin{pgfscope}%
\pgfsys@transformshift{0.750000in}{1.083991in}%
\pgfsys@useobject{currentmarker}{}%
\end{pgfscope}%
\end{pgfscope}%
\begin{pgfscope}%
\definecolor{textcolor}{rgb}{0.000000,0.000000,0.000000}%
\pgfsetstrokecolor{textcolor}%
\pgfsetfillcolor{textcolor}%
\pgftext[x=0.405863in, y=1.032891in, left, base]{\color{textcolor}\rmfamily\fontsize{10.000000}{12.000000}\selectfont \(\displaystyle {2.00}\)}%
\end{pgfscope}%
\begin{pgfscope}%
\definecolor{textcolor}{rgb}{0.000000,0.000000,0.000000}%
\pgfsetstrokecolor{textcolor}%
\pgfsetfillcolor{textcolor}%
\pgftext[x=0.350308in,y=2.420000in,,bottom,rotate=90.000000]{\color{textcolor}\rmfamily\fontsize{10.000000}{12.000000}\selectfont time in s}%
\end{pgfscope}%
\begin{pgfscope}%
\pgfpathrectangle{\pgfqpoint{0.750000in}{0.880000in}}{\pgfqpoint{4.650000in}{3.080000in}}%
\pgfusepath{clip}%
\pgfsetrectcap%
\pgfsetroundjoin%
\pgfsetlinewidth{1.505625pt}%
\definecolor{currentstroke}{rgb}{0.121569,0.466667,0.705882}%
\pgfsetstrokecolor{currentstroke}%
\pgfsetdash{}{0pt}%
\pgfpathmoveto{\pgfqpoint{2.208084in}{3.970000in}}%
\pgfpathlineto{\pgfqpoint{2.209136in}{3.807982in}}%
\pgfpathlineto{\pgfqpoint{2.212385in}{3.792918in}}%
\pgfpathlineto{\pgfqpoint{2.217787in}{3.777853in}}%
\pgfpathlineto{\pgfqpoint{2.226114in}{3.761418in}}%
\pgfpathlineto{\pgfqpoint{2.236948in}{3.744984in}}%
\pgfpathlineto{\pgfqpoint{2.251468in}{3.727180in}}%
\pgfpathlineto{\pgfqpoint{2.268823in}{3.709376in}}%
\pgfpathlineto{\pgfqpoint{2.290601in}{3.690203in}}%
\pgfpathlineto{\pgfqpoint{2.317378in}{3.669660in}}%
\pgfpathlineto{\pgfqpoint{2.349716in}{3.647748in}}%
\pgfpathlineto{\pgfqpoint{2.388152in}{3.624466in}}%
\pgfpathlineto{\pgfqpoint{2.433178in}{3.599814in}}%
\pgfpathlineto{\pgfqpoint{2.488095in}{3.572424in}}%
\pgfpathlineto{\pgfqpoint{2.550880in}{3.543663in}}%
\pgfpathlineto{\pgfqpoint{2.625147in}{3.512164in}}%
\pgfpathlineto{\pgfqpoint{2.711765in}{3.477926in}}%
\pgfpathlineto{\pgfqpoint{2.811490in}{3.440949in}}%
\pgfpathlineto{\pgfqpoint{2.925007in}{3.401233in}}%
\pgfpathlineto{\pgfqpoint{3.065263in}{3.354669in}}%
\pgfpathlineto{\pgfqpoint{3.152069in}{3.324539in}}%
\pgfpathlineto{\pgfqpoint{3.228369in}{3.295779in}}%
\pgfpathlineto{\pgfqpoint{3.298430in}{3.267019in}}%
\pgfpathlineto{\pgfqpoint{3.362494in}{3.238259in}}%
\pgfpathlineto{\pgfqpoint{3.418199in}{3.210868in}}%
\pgfpathlineto{\pgfqpoint{3.469014in}{3.183478in}}%
\pgfpathlineto{\pgfqpoint{3.515239in}{3.156087in}}%
\pgfpathlineto{\pgfqpoint{3.557179in}{3.128697in}}%
\pgfpathlineto{\pgfqpoint{3.595140in}{3.101306in}}%
\pgfpathlineto{\pgfqpoint{3.629422in}{3.073915in}}%
\pgfpathlineto{\pgfqpoint{3.660313in}{3.046525in}}%
\pgfpathlineto{\pgfqpoint{3.689399in}{3.017765in}}%
\pgfpathlineto{\pgfqpoint{3.715348in}{2.989005in}}%
\pgfpathlineto{\pgfqpoint{3.739472in}{2.958875in}}%
\pgfpathlineto{\pgfqpoint{3.760753in}{2.928745in}}%
\pgfpathlineto{\pgfqpoint{3.780252in}{2.897246in}}%
\pgfpathlineto{\pgfqpoint{3.797897in}{2.864378in}}%
\pgfpathlineto{\pgfqpoint{3.813056in}{2.831509in}}%
\pgfpathlineto{\pgfqpoint{3.826468in}{2.797271in}}%
\pgfpathlineto{\pgfqpoint{3.838096in}{2.761663in}}%
\pgfpathlineto{\pgfqpoint{3.848229in}{2.723316in}}%
\pgfpathlineto{\pgfqpoint{3.856355in}{2.683600in}}%
\pgfpathlineto{\pgfqpoint{3.862442in}{2.642514in}}%
\pgfpathlineto{\pgfqpoint{3.866430in}{2.600059in}}%
\pgfpathlineto{\pgfqpoint{3.868198in}{2.557603in}}%
\pgfpathlineto{\pgfqpoint{3.867758in}{2.513778in}}%
\pgfpathlineto{\pgfqpoint{3.865136in}{2.471323in}}%
\pgfpathlineto{\pgfqpoint{3.860289in}{2.428868in}}%
\pgfpathlineto{\pgfqpoint{3.853363in}{2.387782in}}%
\pgfpathlineto{\pgfqpoint{3.844419in}{2.348066in}}%
\pgfpathlineto{\pgfqpoint{3.833491in}{2.309719in}}%
\pgfpathlineto{\pgfqpoint{3.820602in}{2.272742in}}%
\pgfpathlineto{\pgfqpoint{3.805772in}{2.237134in}}%
\pgfpathlineto{\pgfqpoint{3.789027in}{2.202896in}}%
\pgfpathlineto{\pgfqpoint{3.770404in}{2.170027in}}%
\pgfpathlineto{\pgfqpoint{3.749962in}{2.138528in}}%
\pgfpathlineto{\pgfqpoint{3.727783in}{2.108398in}}%
\pgfpathlineto{\pgfqpoint{3.702777in}{2.078269in}}%
\pgfpathlineto{\pgfqpoint{3.676012in}{2.049508in}}%
\pgfpathlineto{\pgfqpoint{3.646154in}{2.020748in}}%
\pgfpathlineto{\pgfqpoint{3.614586in}{1.993358in}}%
\pgfpathlineto{\pgfqpoint{3.579704in}{1.965967in}}%
\pgfpathlineto{\pgfqpoint{3.541244in}{1.938577in}}%
\pgfpathlineto{\pgfqpoint{3.498933in}{1.911186in}}%
\pgfpathlineto{\pgfqpoint{3.452500in}{1.883796in}}%
\pgfpathlineto{\pgfqpoint{3.401673in}{1.856405in}}%
\pgfpathlineto{\pgfqpoint{3.346193in}{1.829014in}}%
\pgfpathlineto{\pgfqpoint{3.285822in}{1.801624in}}%
\pgfpathlineto{\pgfqpoint{3.216943in}{1.772864in}}%
\pgfpathlineto{\pgfqpoint{3.142256in}{1.744104in}}%
\pgfpathlineto{\pgfqpoint{3.057665in}{1.713974in}}%
\pgfpathlineto{\pgfqpoint{2.962269in}{1.682475in}}%
\pgfpathlineto{\pgfqpoint{2.855312in}{1.649606in}}%
\pgfpathlineto{\pgfqpoint{2.736305in}{1.615368in}}%
\pgfpathlineto{\pgfqpoint{2.594849in}{1.577021in}}%
\pgfpathlineto{\pgfqpoint{2.414015in}{1.530457in}}%
\pgfpathlineto{\pgfqpoint{1.842891in}{1.385287in}}%
\pgfpathlineto{\pgfqpoint{1.732003in}{1.353788in}}%
\pgfpathlineto{\pgfqpoint{1.643694in}{1.326398in}}%
\pgfpathlineto{\pgfqpoint{1.575828in}{1.303116in}}%
\pgfpathlineto{\pgfqpoint{1.522217in}{1.282573in}}%
\pgfpathlineto{\pgfqpoint{1.478018in}{1.263399in}}%
\pgfpathlineto{\pgfqpoint{1.442400in}{1.245596in}}%
\pgfpathlineto{\pgfqpoint{1.414420in}{1.229161in}}%
\pgfpathlineto{\pgfqpoint{1.393077in}{1.214096in}}%
\pgfpathlineto{\pgfqpoint{1.377359in}{1.200401in}}%
\pgfpathlineto{\pgfqpoint{1.366279in}{1.188075in}}%
\pgfpathlineto{\pgfqpoint{1.358147in}{1.175750in}}%
\pgfpathlineto{\pgfqpoint{1.353420in}{1.164793in}}%
\pgfpathlineto{\pgfqpoint{1.351061in}{1.153837in}}%
\pgfpathlineto{\pgfqpoint{1.351074in}{1.142881in}}%
\pgfpathlineto{\pgfqpoint{1.353456in}{1.131925in}}%
\pgfpathlineto{\pgfqpoint{1.358197in}{1.120968in}}%
\pgfpathlineto{\pgfqpoint{1.365278in}{1.110012in}}%
\pgfpathlineto{\pgfqpoint{1.376008in}{1.097686in}}%
\pgfpathlineto{\pgfqpoint{1.389616in}{1.085361in}}%
\pgfpathlineto{\pgfqpoint{1.389616in}{1.085361in}}%
\pgfusepath{stroke}%
\end{pgfscope}%
\begin{pgfscope}%
\pgfpathrectangle{\pgfqpoint{0.750000in}{0.880000in}}{\pgfqpoint{4.650000in}{3.080000in}}%
\pgfusepath{clip}%
\pgfsetbuttcap%
\pgfsetroundjoin%
\pgfsetlinewidth{1.505625pt}%
\definecolor{currentstroke}{rgb}{0.121569,0.466667,0.705882}%
\pgfsetstrokecolor{currentstroke}%
\pgfsetdash{{5.550000pt}{2.400000pt}}{0.000000pt}%
\pgfpathmoveto{\pgfqpoint{0.750000in}{3.357408in}}%
\pgfpathlineto{\pgfqpoint{5.400000in}{3.357408in}}%
\pgfusepath{stroke}%
\end{pgfscope}%
\begin{pgfscope}%
\pgfsetrectcap%
\pgfsetmiterjoin%
\pgfsetlinewidth{0.803000pt}%
\definecolor{currentstroke}{rgb}{0.000000,0.000000,0.000000}%
\pgfsetstrokecolor{currentstroke}%
\pgfsetdash{}{0pt}%
\pgfpathmoveto{\pgfqpoint{0.750000in}{0.880000in}}%
\pgfpathlineto{\pgfqpoint{0.750000in}{3.960000in}}%
\pgfusepath{stroke}%
\end{pgfscope}%
\begin{pgfscope}%
\pgfsetrectcap%
\pgfsetmiterjoin%
\pgfsetlinewidth{0.803000pt}%
\definecolor{currentstroke}{rgb}{0.000000,0.000000,0.000000}%
\pgfsetstrokecolor{currentstroke}%
\pgfsetdash{}{0pt}%
\pgfpathmoveto{\pgfqpoint{5.400000in}{0.880000in}}%
\pgfpathlineto{\pgfqpoint{5.400000in}{3.960000in}}%
\pgfusepath{stroke}%
\end{pgfscope}%
\begin{pgfscope}%
\pgfsetrectcap%
\pgfsetmiterjoin%
\pgfsetlinewidth{0.803000pt}%
\definecolor{currentstroke}{rgb}{0.000000,0.000000,0.000000}%
\pgfsetstrokecolor{currentstroke}%
\pgfsetdash{}{0pt}%
\pgfpathmoveto{\pgfqpoint{0.750000in}{0.880000in}}%
\pgfpathlineto{\pgfqpoint{5.400000in}{0.880000in}}%
\pgfusepath{stroke}%
\end{pgfscope}%
\begin{pgfscope}%
\pgfsetrectcap%
\pgfsetmiterjoin%
\pgfsetlinewidth{0.803000pt}%
\definecolor{currentstroke}{rgb}{0.000000,0.000000,0.000000}%
\pgfsetstrokecolor{currentstroke}%
\pgfsetdash{}{0pt}%
\pgfpathmoveto{\pgfqpoint{0.750000in}{3.960000in}}%
\pgfpathlineto{\pgfqpoint{5.400000in}{3.960000in}}%
\pgfusepath{stroke}%
\end{pgfscope}%
\begin{pgfscope}%
\pgfsetbuttcap%
\pgfsetmiterjoin%
\definecolor{currentfill}{rgb}{1.000000,1.000000,1.000000}%
\pgfsetfillcolor{currentfill}%
\pgfsetfillopacity{0.800000}%
\pgfsetlinewidth{1.003750pt}%
\definecolor{currentstroke}{rgb}{0.800000,0.800000,0.800000}%
\pgfsetstrokecolor{currentstroke}%
\pgfsetstrokeopacity{0.800000}%
\pgfsetdash{}{0pt}%
\pgfpathmoveto{\pgfqpoint{3.899064in}{3.443955in}}%
\pgfpathlineto{\pgfqpoint{5.302778in}{3.443955in}}%
\pgfpathquadraticcurveto{\pgfqpoint{5.330556in}{3.443955in}}{\pgfqpoint{5.330556in}{3.471733in}}%
\pgfpathlineto{\pgfqpoint{5.330556in}{3.862778in}}%
\pgfpathquadraticcurveto{\pgfqpoint{5.330556in}{3.890556in}}{\pgfqpoint{5.302778in}{3.890556in}}%
\pgfpathlineto{\pgfqpoint{3.899064in}{3.890556in}}%
\pgfpathquadraticcurveto{\pgfqpoint{3.871286in}{3.890556in}}{\pgfqpoint{3.871286in}{3.862778in}}%
\pgfpathlineto{\pgfqpoint{3.871286in}{3.471733in}}%
\pgfpathquadraticcurveto{\pgfqpoint{3.871286in}{3.443955in}}{\pgfqpoint{3.899064in}{3.443955in}}%
\pgfpathlineto{\pgfqpoint{3.899064in}{3.443955in}}%
\pgfpathclose%
\pgfusepath{stroke,fill}%
\end{pgfscope}%
\begin{pgfscope}%
\pgfsetrectcap%
\pgfsetroundjoin%
\pgfsetlinewidth{1.505625pt}%
\definecolor{currentstroke}{rgb}{0.121569,0.466667,0.705882}%
\pgfsetstrokecolor{currentstroke}%
\pgfsetdash{}{0pt}%
\pgfpathmoveto{\pgfqpoint{3.926842in}{3.781411in}}%
\pgfpathlineto{\pgfqpoint{4.065731in}{3.781411in}}%
\pgfpathlineto{\pgfqpoint{4.204620in}{3.781411in}}%
\pgfusepath{stroke}%
\end{pgfscope}%
\begin{pgfscope}%
\definecolor{textcolor}{rgb}{0.000000,0.000000,0.000000}%
\pgfsetstrokecolor{textcolor}%
\pgfsetfillcolor{textcolor}%
\pgftext[x=4.315731in,y=3.732800in,left,base]{\color{textcolor}\rmfamily\fontsize{10.000000}{12.000000}\selectfont delta}%
\end{pgfscope}%
\begin{pgfscope}%
\pgfsetbuttcap%
\pgfsetroundjoin%
\pgfsetlinewidth{1.505625pt}%
\definecolor{currentstroke}{rgb}{0.121569,0.466667,0.705882}%
\pgfsetstrokecolor{currentstroke}%
\pgfsetdash{{5.550000pt}{2.400000pt}}{0.000000pt}%
\pgfpathmoveto{\pgfqpoint{3.926842in}{3.578368in}}%
\pgfpathlineto{\pgfqpoint{4.065731in}{3.578368in}}%
\pgfpathlineto{\pgfqpoint{4.204620in}{3.578368in}}%
\pgfusepath{stroke}%
\end{pgfscope}%
\begin{pgfscope}%
\definecolor{textcolor}{rgb}{0.000000,0.000000,0.000000}%
\pgfsetstrokecolor{textcolor}%
\pgfsetfillcolor{textcolor}%
\pgftext[x=4.315731in,y=3.529757in,left,base]{\color{textcolor}\rmfamily\fontsize{10.000000}{12.000000}\selectfont clearing of fault}%
\end{pgfscope}%
\begin{pgfscope}%
\definecolor{textcolor}{rgb}{0.000000,0.000000,0.000000}%
\pgfsetstrokecolor{textcolor}%
\pgfsetfillcolor{textcolor}%
\pgftext[x=3.000000in,y=7.840000in,,top]{\color{textcolor}\rmfamily\fontsize{12.000000}{14.400000}\selectfont Stable scenario - fault 2}%
\end{pgfscope}%
\end{pgfpicture}%
\makeatother%
\endgroup%


%% Creator: Matplotlib, PGF backend
%%
%% To include the figure in your LaTeX document, write
%%   \input{<filename>.pgf}
%%
%% Make sure the required packages are loaded in your preamble
%%   \usepackage{pgf}
%%
%% Also ensure that all the required font packages are loaded; for instance,
%% the lmodern package is sometimes necessary when using math font.
%%   \usepackage{lmodern}
%%
%% Figures using additional raster images can only be included by \input if
%% they are in the same directory as the main LaTeX file. For loading figures
%% from other directories you can use the `import` package
%%   \usepackage{import}
%%
%% and then include the figures with
%%   \import{<path to file>}{<filename>.pgf}
%%
%% Matplotlib used the following preamble
%%   
%%   \usepackage{fontspec}
%%   \setmainfont{Charter.ttc}[Path=\detokenize{/System/Library/Fonts/Supplemental/}]
%%   \setsansfont{DejaVuSans.ttf}[Path=\detokenize{/opt/homebrew/lib/python3.10/site-packages/matplotlib/mpl-data/fonts/ttf/}]
%%   \setmonofont{DejaVuSansMono.ttf}[Path=\detokenize{/opt/homebrew/lib/python3.10/site-packages/matplotlib/mpl-data/fonts/ttf/}]
%%   \makeatletter\@ifpackageloaded{underscore}{}{\usepackage[strings]{underscore}}\makeatother
%%
\begingroup%
\makeatletter%
\begin{pgfpicture}%
\pgfpathrectangle{\pgfpointorigin}{\pgfqpoint{6.000000in}{8.000000in}}%
\pgfusepath{use as bounding box, clip}%
\begin{pgfscope}%
\pgfsetbuttcap%
\pgfsetmiterjoin%
\definecolor{currentfill}{rgb}{1.000000,1.000000,1.000000}%
\pgfsetfillcolor{currentfill}%
\pgfsetlinewidth{0.000000pt}%
\definecolor{currentstroke}{rgb}{1.000000,1.000000,1.000000}%
\pgfsetstrokecolor{currentstroke}%
\pgfsetdash{}{0pt}%
\pgfpathmoveto{\pgfqpoint{0.000000in}{0.000000in}}%
\pgfpathlineto{\pgfqpoint{6.000000in}{0.000000in}}%
\pgfpathlineto{\pgfqpoint{6.000000in}{8.000000in}}%
\pgfpathlineto{\pgfqpoint{0.000000in}{8.000000in}}%
\pgfpathlineto{\pgfqpoint{0.000000in}{0.000000in}}%
\pgfpathclose%
\pgfusepath{fill}%
\end{pgfscope}%
\begin{pgfscope}%
\pgfsetbuttcap%
\pgfsetmiterjoin%
\definecolor{currentfill}{rgb}{1.000000,1.000000,1.000000}%
\pgfsetfillcolor{currentfill}%
\pgfsetlinewidth{0.000000pt}%
\definecolor{currentstroke}{rgb}{0.000000,0.000000,0.000000}%
\pgfsetstrokecolor{currentstroke}%
\pgfsetstrokeopacity{0.000000}%
\pgfsetdash{}{0pt}%
\pgfpathmoveto{\pgfqpoint{0.750000in}{3.960000in}}%
\pgfpathlineto{\pgfqpoint{5.400000in}{3.960000in}}%
\pgfpathlineto{\pgfqpoint{5.400000in}{7.040000in}}%
\pgfpathlineto{\pgfqpoint{0.750000in}{7.040000in}}%
\pgfpathlineto{\pgfqpoint{0.750000in}{3.960000in}}%
\pgfpathclose%
\pgfusepath{fill}%
\end{pgfscope}%
\begin{pgfscope}%
\pgfpathrectangle{\pgfqpoint{0.750000in}{3.960000in}}{\pgfqpoint{4.650000in}{3.080000in}}%
\pgfusepath{clip}%
\pgfsetbuttcap%
\pgfsetroundjoin%
\definecolor{currentfill}{rgb}{0.900000,0.900000,0.900000}%
\pgfsetfillcolor{currentfill}%
\pgfsetlinewidth{1.003750pt}%
\definecolor{currentstroke}{rgb}{0.500000,0.500000,0.500000}%
\pgfsetstrokecolor{currentstroke}%
\pgfsetdash{}{0pt}%
\pgfsys@defobject{currentmarker}{\pgfqpoint{2.005500in}{5.441634in}}{\pgfqpoint{3.228434in}{6.161131in}}{%
\pgfpathmoveto{\pgfqpoint{2.005500in}{6.161131in}}%
\pgfpathlineto{\pgfqpoint{2.005500in}{5.441634in}}%
\pgfpathlineto{\pgfqpoint{2.030458in}{5.463448in}}%
\pgfpathlineto{\pgfqpoint{2.055416in}{5.484835in}}%
\pgfpathlineto{\pgfqpoint{2.080374in}{5.505788in}}%
\pgfpathlineto{\pgfqpoint{2.105331in}{5.526301in}}%
\pgfpathlineto{\pgfqpoint{2.130289in}{5.546369in}}%
\pgfpathlineto{\pgfqpoint{2.155247in}{5.565986in}}%
\pgfpathlineto{\pgfqpoint{2.180205in}{5.585146in}}%
\pgfpathlineto{\pgfqpoint{2.205163in}{5.603845in}}%
\pgfpathlineto{\pgfqpoint{2.230121in}{5.622076in}}%
\pgfpathlineto{\pgfqpoint{2.255078in}{5.639834in}}%
\pgfpathlineto{\pgfqpoint{2.280036in}{5.657115in}}%
\pgfpathlineto{\pgfqpoint{2.304994in}{5.673913in}}%
\pgfpathlineto{\pgfqpoint{2.329952in}{5.690224in}}%
\pgfpathlineto{\pgfqpoint{2.354910in}{5.706043in}}%
\pgfpathlineto{\pgfqpoint{2.379868in}{5.721366in}}%
\pgfpathlineto{\pgfqpoint{2.404825in}{5.736188in}}%
\pgfpathlineto{\pgfqpoint{2.429783in}{5.750505in}}%
\pgfpathlineto{\pgfqpoint{2.454741in}{5.764313in}}%
\pgfpathlineto{\pgfqpoint{2.479699in}{5.777608in}}%
\pgfpathlineto{\pgfqpoint{2.504657in}{5.790386in}}%
\pgfpathlineto{\pgfqpoint{2.529615in}{5.802643in}}%
\pgfpathlineto{\pgfqpoint{2.554572in}{5.814377in}}%
\pgfpathlineto{\pgfqpoint{2.579530in}{5.825584in}}%
\pgfpathlineto{\pgfqpoint{2.604488in}{5.836260in}}%
\pgfpathlineto{\pgfqpoint{2.629446in}{5.846403in}}%
\pgfpathlineto{\pgfqpoint{2.654404in}{5.856009in}}%
\pgfpathlineto{\pgfqpoint{2.679362in}{5.865076in}}%
\pgfpathlineto{\pgfqpoint{2.704319in}{5.873602in}}%
\pgfpathlineto{\pgfqpoint{2.729277in}{5.881584in}}%
\pgfpathlineto{\pgfqpoint{2.754235in}{5.889019in}}%
\pgfpathlineto{\pgfqpoint{2.779193in}{5.895906in}}%
\pgfpathlineto{\pgfqpoint{2.804151in}{5.902242in}}%
\pgfpathlineto{\pgfqpoint{2.829109in}{5.908026in}}%
\pgfpathlineto{\pgfqpoint{2.854066in}{5.913257in}}%
\pgfpathlineto{\pgfqpoint{2.879024in}{5.917932in}}%
\pgfpathlineto{\pgfqpoint{2.903982in}{5.922050in}}%
\pgfpathlineto{\pgfqpoint{2.928940in}{5.925611in}}%
\pgfpathlineto{\pgfqpoint{2.953898in}{5.928613in}}%
\pgfpathlineto{\pgfqpoint{2.978856in}{5.931055in}}%
\pgfpathlineto{\pgfqpoint{3.003813in}{5.932936in}}%
\pgfpathlineto{\pgfqpoint{3.028771in}{5.934257in}}%
\pgfpathlineto{\pgfqpoint{3.053729in}{5.935016in}}%
\pgfpathlineto{\pgfqpoint{3.078687in}{5.935214in}}%
\pgfpathlineto{\pgfqpoint{3.103645in}{5.934850in}}%
\pgfpathlineto{\pgfqpoint{3.128603in}{5.933925in}}%
\pgfpathlineto{\pgfqpoint{3.153560in}{5.932439in}}%
\pgfpathlineto{\pgfqpoint{3.178518in}{5.930391in}}%
\pgfpathlineto{\pgfqpoint{3.203476in}{5.927784in}}%
\pgfpathlineto{\pgfqpoint{3.228434in}{5.924617in}}%
\pgfpathlineto{\pgfqpoint{3.228434in}{6.161131in}}%
\pgfpathlineto{\pgfqpoint{3.228434in}{6.161131in}}%
\pgfpathlineto{\pgfqpoint{3.203476in}{6.161131in}}%
\pgfpathlineto{\pgfqpoint{3.178518in}{6.161131in}}%
\pgfpathlineto{\pgfqpoint{3.153560in}{6.161131in}}%
\pgfpathlineto{\pgfqpoint{3.128603in}{6.161131in}}%
\pgfpathlineto{\pgfqpoint{3.103645in}{6.161131in}}%
\pgfpathlineto{\pgfqpoint{3.078687in}{6.161131in}}%
\pgfpathlineto{\pgfqpoint{3.053729in}{6.161131in}}%
\pgfpathlineto{\pgfqpoint{3.028771in}{6.161131in}}%
\pgfpathlineto{\pgfqpoint{3.003813in}{6.161131in}}%
\pgfpathlineto{\pgfqpoint{2.978856in}{6.161131in}}%
\pgfpathlineto{\pgfqpoint{2.953898in}{6.161131in}}%
\pgfpathlineto{\pgfqpoint{2.928940in}{6.161131in}}%
\pgfpathlineto{\pgfqpoint{2.903982in}{6.161131in}}%
\pgfpathlineto{\pgfqpoint{2.879024in}{6.161131in}}%
\pgfpathlineto{\pgfqpoint{2.854066in}{6.161131in}}%
\pgfpathlineto{\pgfqpoint{2.829109in}{6.161131in}}%
\pgfpathlineto{\pgfqpoint{2.804151in}{6.161131in}}%
\pgfpathlineto{\pgfqpoint{2.779193in}{6.161131in}}%
\pgfpathlineto{\pgfqpoint{2.754235in}{6.161131in}}%
\pgfpathlineto{\pgfqpoint{2.729277in}{6.161131in}}%
\pgfpathlineto{\pgfqpoint{2.704319in}{6.161131in}}%
\pgfpathlineto{\pgfqpoint{2.679362in}{6.161131in}}%
\pgfpathlineto{\pgfqpoint{2.654404in}{6.161131in}}%
\pgfpathlineto{\pgfqpoint{2.629446in}{6.161131in}}%
\pgfpathlineto{\pgfqpoint{2.604488in}{6.161131in}}%
\pgfpathlineto{\pgfqpoint{2.579530in}{6.161131in}}%
\pgfpathlineto{\pgfqpoint{2.554572in}{6.161131in}}%
\pgfpathlineto{\pgfqpoint{2.529615in}{6.161131in}}%
\pgfpathlineto{\pgfqpoint{2.504657in}{6.161131in}}%
\pgfpathlineto{\pgfqpoint{2.479699in}{6.161131in}}%
\pgfpathlineto{\pgfqpoint{2.454741in}{6.161131in}}%
\pgfpathlineto{\pgfqpoint{2.429783in}{6.161131in}}%
\pgfpathlineto{\pgfqpoint{2.404825in}{6.161131in}}%
\pgfpathlineto{\pgfqpoint{2.379868in}{6.161131in}}%
\pgfpathlineto{\pgfqpoint{2.354910in}{6.161131in}}%
\pgfpathlineto{\pgfqpoint{2.329952in}{6.161131in}}%
\pgfpathlineto{\pgfqpoint{2.304994in}{6.161131in}}%
\pgfpathlineto{\pgfqpoint{2.280036in}{6.161131in}}%
\pgfpathlineto{\pgfqpoint{2.255078in}{6.161131in}}%
\pgfpathlineto{\pgfqpoint{2.230121in}{6.161131in}}%
\pgfpathlineto{\pgfqpoint{2.205163in}{6.161131in}}%
\pgfpathlineto{\pgfqpoint{2.180205in}{6.161131in}}%
\pgfpathlineto{\pgfqpoint{2.155247in}{6.161131in}}%
\pgfpathlineto{\pgfqpoint{2.130289in}{6.161131in}}%
\pgfpathlineto{\pgfqpoint{2.105331in}{6.161131in}}%
\pgfpathlineto{\pgfqpoint{2.080374in}{6.161131in}}%
\pgfpathlineto{\pgfqpoint{2.055416in}{6.161131in}}%
\pgfpathlineto{\pgfqpoint{2.030458in}{6.161131in}}%
\pgfpathlineto{\pgfqpoint{2.005500in}{6.161131in}}%
\pgfpathlineto{\pgfqpoint{2.005500in}{6.161131in}}%
\pgfpathclose%
\pgfusepath{stroke,fill}%
}%
\begin{pgfscope}%
\pgfsys@transformshift{0.000000in}{0.000000in}%
\pgfsys@useobject{currentmarker}{}%
\end{pgfscope}%
\end{pgfscope}%
\begin{pgfscope}%
\pgfpathrectangle{\pgfqpoint{0.750000in}{3.960000in}}{\pgfqpoint{4.650000in}{3.080000in}}%
\pgfusepath{clip}%
\pgfsetbuttcap%
\pgfsetroundjoin%
\definecolor{currentfill}{rgb}{0.900000,0.900000,0.900000}%
\pgfsetfillcolor{currentfill}%
\pgfsetlinewidth{1.003750pt}%
\definecolor{currentstroke}{rgb}{0.500000,0.500000,0.500000}%
\pgfsetstrokecolor{currentstroke}%
\pgfsetdash{}{0pt}%
\pgfsys@defobject{currentmarker}{\pgfqpoint{3.228434in}{6.161131in}}{\pgfqpoint{4.144500in}{6.879087in}}{%
\pgfpathmoveto{\pgfqpoint{3.228434in}{6.161131in}}%
\pgfpathlineto{\pgfqpoint{3.228434in}{6.879087in}}%
\pgfpathlineto{\pgfqpoint{3.247129in}{6.875018in}}%
\pgfpathlineto{\pgfqpoint{3.265824in}{6.870485in}}%
\pgfpathlineto{\pgfqpoint{3.284520in}{6.865487in}}%
\pgfpathlineto{\pgfqpoint{3.303215in}{6.860025in}}%
\pgfpathlineto{\pgfqpoint{3.321910in}{6.854101in}}%
\pgfpathlineto{\pgfqpoint{3.340605in}{6.847716in}}%
\pgfpathlineto{\pgfqpoint{3.359301in}{6.840869in}}%
\pgfpathlineto{\pgfqpoint{3.377996in}{6.833563in}}%
\pgfpathlineto{\pgfqpoint{3.396691in}{6.825799in}}%
\pgfpathlineto{\pgfqpoint{3.415386in}{6.817577in}}%
\pgfpathlineto{\pgfqpoint{3.434081in}{6.808900in}}%
\pgfpathlineto{\pgfqpoint{3.452777in}{6.799768in}}%
\pgfpathlineto{\pgfqpoint{3.471472in}{6.790183in}}%
\pgfpathlineto{\pgfqpoint{3.490167in}{6.780146in}}%
\pgfpathlineto{\pgfqpoint{3.508862in}{6.769660in}}%
\pgfpathlineto{\pgfqpoint{3.527558in}{6.758725in}}%
\pgfpathlineto{\pgfqpoint{3.546253in}{6.747344in}}%
\pgfpathlineto{\pgfqpoint{3.564948in}{6.735518in}}%
\pgfpathlineto{\pgfqpoint{3.583643in}{6.723249in}}%
\pgfpathlineto{\pgfqpoint{3.602338in}{6.710540in}}%
\pgfpathlineto{\pgfqpoint{3.621034in}{6.697392in}}%
\pgfpathlineto{\pgfqpoint{3.639729in}{6.683807in}}%
\pgfpathlineto{\pgfqpoint{3.658424in}{6.669787in}}%
\pgfpathlineto{\pgfqpoint{3.677119in}{6.655335in}}%
\pgfpathlineto{\pgfqpoint{3.695815in}{6.640453in}}%
\pgfpathlineto{\pgfqpoint{3.714510in}{6.625144in}}%
\pgfpathlineto{\pgfqpoint{3.733205in}{6.609409in}}%
\pgfpathlineto{\pgfqpoint{3.751900in}{6.593252in}}%
\pgfpathlineto{\pgfqpoint{3.770596in}{6.576675in}}%
\pgfpathlineto{\pgfqpoint{3.789291in}{6.559680in}}%
\pgfpathlineto{\pgfqpoint{3.807986in}{6.542271in}}%
\pgfpathlineto{\pgfqpoint{3.826681in}{6.524449in}}%
\pgfpathlineto{\pgfqpoint{3.845376in}{6.506218in}}%
\pgfpathlineto{\pgfqpoint{3.864072in}{6.487582in}}%
\pgfpathlineto{\pgfqpoint{3.882767in}{6.468542in}}%
\pgfpathlineto{\pgfqpoint{3.901462in}{6.449101in}}%
\pgfpathlineto{\pgfqpoint{3.920157in}{6.429264in}}%
\pgfpathlineto{\pgfqpoint{3.938853in}{6.409033in}}%
\pgfpathlineto{\pgfqpoint{3.957548in}{6.388411in}}%
\pgfpathlineto{\pgfqpoint{3.976243in}{6.367402in}}%
\pgfpathlineto{\pgfqpoint{3.994938in}{6.346008in}}%
\pgfpathlineto{\pgfqpoint{4.013633in}{6.324234in}}%
\pgfpathlineto{\pgfqpoint{4.032329in}{6.302083in}}%
\pgfpathlineto{\pgfqpoint{4.051024in}{6.279558in}}%
\pgfpathlineto{\pgfqpoint{4.069719in}{6.256663in}}%
\pgfpathlineto{\pgfqpoint{4.088414in}{6.233402in}}%
\pgfpathlineto{\pgfqpoint{4.107110in}{6.209778in}}%
\pgfpathlineto{\pgfqpoint{4.125805in}{6.185795in}}%
\pgfpathlineto{\pgfqpoint{4.144500in}{6.161457in}}%
\pgfpathlineto{\pgfqpoint{4.144500in}{6.161131in}}%
\pgfpathlineto{\pgfqpoint{4.144500in}{6.161131in}}%
\pgfpathlineto{\pgfqpoint{4.125805in}{6.161131in}}%
\pgfpathlineto{\pgfqpoint{4.107110in}{6.161131in}}%
\pgfpathlineto{\pgfqpoint{4.088414in}{6.161131in}}%
\pgfpathlineto{\pgfqpoint{4.069719in}{6.161131in}}%
\pgfpathlineto{\pgfqpoint{4.051024in}{6.161131in}}%
\pgfpathlineto{\pgfqpoint{4.032329in}{6.161131in}}%
\pgfpathlineto{\pgfqpoint{4.013633in}{6.161131in}}%
\pgfpathlineto{\pgfqpoint{3.994938in}{6.161131in}}%
\pgfpathlineto{\pgfqpoint{3.976243in}{6.161131in}}%
\pgfpathlineto{\pgfqpoint{3.957548in}{6.161131in}}%
\pgfpathlineto{\pgfqpoint{3.938853in}{6.161131in}}%
\pgfpathlineto{\pgfqpoint{3.920157in}{6.161131in}}%
\pgfpathlineto{\pgfqpoint{3.901462in}{6.161131in}}%
\pgfpathlineto{\pgfqpoint{3.882767in}{6.161131in}}%
\pgfpathlineto{\pgfqpoint{3.864072in}{6.161131in}}%
\pgfpathlineto{\pgfqpoint{3.845376in}{6.161131in}}%
\pgfpathlineto{\pgfqpoint{3.826681in}{6.161131in}}%
\pgfpathlineto{\pgfqpoint{3.807986in}{6.161131in}}%
\pgfpathlineto{\pgfqpoint{3.789291in}{6.161131in}}%
\pgfpathlineto{\pgfqpoint{3.770596in}{6.161131in}}%
\pgfpathlineto{\pgfqpoint{3.751900in}{6.161131in}}%
\pgfpathlineto{\pgfqpoint{3.733205in}{6.161131in}}%
\pgfpathlineto{\pgfqpoint{3.714510in}{6.161131in}}%
\pgfpathlineto{\pgfqpoint{3.695815in}{6.161131in}}%
\pgfpathlineto{\pgfqpoint{3.677119in}{6.161131in}}%
\pgfpathlineto{\pgfqpoint{3.658424in}{6.161131in}}%
\pgfpathlineto{\pgfqpoint{3.639729in}{6.161131in}}%
\pgfpathlineto{\pgfqpoint{3.621034in}{6.161131in}}%
\pgfpathlineto{\pgfqpoint{3.602338in}{6.161131in}}%
\pgfpathlineto{\pgfqpoint{3.583643in}{6.161131in}}%
\pgfpathlineto{\pgfqpoint{3.564948in}{6.161131in}}%
\pgfpathlineto{\pgfqpoint{3.546253in}{6.161131in}}%
\pgfpathlineto{\pgfqpoint{3.527558in}{6.161131in}}%
\pgfpathlineto{\pgfqpoint{3.508862in}{6.161131in}}%
\pgfpathlineto{\pgfqpoint{3.490167in}{6.161131in}}%
\pgfpathlineto{\pgfqpoint{3.471472in}{6.161131in}}%
\pgfpathlineto{\pgfqpoint{3.452777in}{6.161131in}}%
\pgfpathlineto{\pgfqpoint{3.434081in}{6.161131in}}%
\pgfpathlineto{\pgfqpoint{3.415386in}{6.161131in}}%
\pgfpathlineto{\pgfqpoint{3.396691in}{6.161131in}}%
\pgfpathlineto{\pgfqpoint{3.377996in}{6.161131in}}%
\pgfpathlineto{\pgfqpoint{3.359301in}{6.161131in}}%
\pgfpathlineto{\pgfqpoint{3.340605in}{6.161131in}}%
\pgfpathlineto{\pgfqpoint{3.321910in}{6.161131in}}%
\pgfpathlineto{\pgfqpoint{3.303215in}{6.161131in}}%
\pgfpathlineto{\pgfqpoint{3.284520in}{6.161131in}}%
\pgfpathlineto{\pgfqpoint{3.265824in}{6.161131in}}%
\pgfpathlineto{\pgfqpoint{3.247129in}{6.161131in}}%
\pgfpathlineto{\pgfqpoint{3.228434in}{6.161131in}}%
\pgfpathlineto{\pgfqpoint{3.228434in}{6.161131in}}%
\pgfpathclose%
\pgfusepath{stroke,fill}%
}%
\begin{pgfscope}%
\pgfsys@transformshift{0.000000in}{0.000000in}%
\pgfsys@useobject{currentmarker}{}%
\end{pgfscope}%
\end{pgfscope}%
\begin{pgfscope}%
\pgfpathrectangle{\pgfqpoint{0.750000in}{3.960000in}}{\pgfqpoint{4.650000in}{3.080000in}}%
\pgfusepath{clip}%
\pgfsetrectcap%
\pgfsetroundjoin%
\pgfsetlinewidth{0.803000pt}%
\definecolor{currentstroke}{rgb}{0.690196,0.690196,0.690196}%
\pgfsetstrokecolor{currentstroke}%
\pgfsetdash{}{0pt}%
\pgfpathmoveto{\pgfqpoint{0.750000in}{3.960000in}}%
\pgfpathlineto{\pgfqpoint{0.750000in}{7.040000in}}%
\pgfusepath{stroke}%
\end{pgfscope}%
\begin{pgfscope}%
\pgfsetbuttcap%
\pgfsetroundjoin%
\definecolor{currentfill}{rgb}{0.000000,0.000000,0.000000}%
\pgfsetfillcolor{currentfill}%
\pgfsetlinewidth{0.803000pt}%
\definecolor{currentstroke}{rgb}{0.000000,0.000000,0.000000}%
\pgfsetstrokecolor{currentstroke}%
\pgfsetdash{}{0pt}%
\pgfsys@defobject{currentmarker}{\pgfqpoint{0.000000in}{-0.048611in}}{\pgfqpoint{0.000000in}{0.000000in}}{%
\pgfpathmoveto{\pgfqpoint{0.000000in}{0.000000in}}%
\pgfpathlineto{\pgfqpoint{0.000000in}{-0.048611in}}%
\pgfusepath{stroke,fill}%
}%
\begin{pgfscope}%
\pgfsys@transformshift{0.750000in}{3.960000in}%
\pgfsys@useobject{currentmarker}{}%
\end{pgfscope}%
\end{pgfscope}%
\begin{pgfscope}%
\pgfpathrectangle{\pgfqpoint{0.750000in}{3.960000in}}{\pgfqpoint{4.650000in}{3.080000in}}%
\pgfusepath{clip}%
\pgfsetrectcap%
\pgfsetroundjoin%
\pgfsetlinewidth{0.803000pt}%
\definecolor{currentstroke}{rgb}{0.690196,0.690196,0.690196}%
\pgfsetstrokecolor{currentstroke}%
\pgfsetdash{}{0pt}%
\pgfpathmoveto{\pgfqpoint{1.266667in}{3.960000in}}%
\pgfpathlineto{\pgfqpoint{1.266667in}{7.040000in}}%
\pgfusepath{stroke}%
\end{pgfscope}%
\begin{pgfscope}%
\pgfsetbuttcap%
\pgfsetroundjoin%
\definecolor{currentfill}{rgb}{0.000000,0.000000,0.000000}%
\pgfsetfillcolor{currentfill}%
\pgfsetlinewidth{0.803000pt}%
\definecolor{currentstroke}{rgb}{0.000000,0.000000,0.000000}%
\pgfsetstrokecolor{currentstroke}%
\pgfsetdash{}{0pt}%
\pgfsys@defobject{currentmarker}{\pgfqpoint{0.000000in}{-0.048611in}}{\pgfqpoint{0.000000in}{0.000000in}}{%
\pgfpathmoveto{\pgfqpoint{0.000000in}{0.000000in}}%
\pgfpathlineto{\pgfqpoint{0.000000in}{-0.048611in}}%
\pgfusepath{stroke,fill}%
}%
\begin{pgfscope}%
\pgfsys@transformshift{1.266667in}{3.960000in}%
\pgfsys@useobject{currentmarker}{}%
\end{pgfscope}%
\end{pgfscope}%
\begin{pgfscope}%
\pgfpathrectangle{\pgfqpoint{0.750000in}{3.960000in}}{\pgfqpoint{4.650000in}{3.080000in}}%
\pgfusepath{clip}%
\pgfsetrectcap%
\pgfsetroundjoin%
\pgfsetlinewidth{0.803000pt}%
\definecolor{currentstroke}{rgb}{0.690196,0.690196,0.690196}%
\pgfsetstrokecolor{currentstroke}%
\pgfsetdash{}{0pt}%
\pgfpathmoveto{\pgfqpoint{1.783333in}{3.960000in}}%
\pgfpathlineto{\pgfqpoint{1.783333in}{7.040000in}}%
\pgfusepath{stroke}%
\end{pgfscope}%
\begin{pgfscope}%
\pgfsetbuttcap%
\pgfsetroundjoin%
\definecolor{currentfill}{rgb}{0.000000,0.000000,0.000000}%
\pgfsetfillcolor{currentfill}%
\pgfsetlinewidth{0.803000pt}%
\definecolor{currentstroke}{rgb}{0.000000,0.000000,0.000000}%
\pgfsetstrokecolor{currentstroke}%
\pgfsetdash{}{0pt}%
\pgfsys@defobject{currentmarker}{\pgfqpoint{0.000000in}{-0.048611in}}{\pgfqpoint{0.000000in}{0.000000in}}{%
\pgfpathmoveto{\pgfqpoint{0.000000in}{0.000000in}}%
\pgfpathlineto{\pgfqpoint{0.000000in}{-0.048611in}}%
\pgfusepath{stroke,fill}%
}%
\begin{pgfscope}%
\pgfsys@transformshift{1.783333in}{3.960000in}%
\pgfsys@useobject{currentmarker}{}%
\end{pgfscope}%
\end{pgfscope}%
\begin{pgfscope}%
\pgfpathrectangle{\pgfqpoint{0.750000in}{3.960000in}}{\pgfqpoint{4.650000in}{3.080000in}}%
\pgfusepath{clip}%
\pgfsetrectcap%
\pgfsetroundjoin%
\pgfsetlinewidth{0.803000pt}%
\definecolor{currentstroke}{rgb}{0.690196,0.690196,0.690196}%
\pgfsetstrokecolor{currentstroke}%
\pgfsetdash{}{0pt}%
\pgfpathmoveto{\pgfqpoint{2.300000in}{3.960000in}}%
\pgfpathlineto{\pgfqpoint{2.300000in}{7.040000in}}%
\pgfusepath{stroke}%
\end{pgfscope}%
\begin{pgfscope}%
\pgfsetbuttcap%
\pgfsetroundjoin%
\definecolor{currentfill}{rgb}{0.000000,0.000000,0.000000}%
\pgfsetfillcolor{currentfill}%
\pgfsetlinewidth{0.803000pt}%
\definecolor{currentstroke}{rgb}{0.000000,0.000000,0.000000}%
\pgfsetstrokecolor{currentstroke}%
\pgfsetdash{}{0pt}%
\pgfsys@defobject{currentmarker}{\pgfqpoint{0.000000in}{-0.048611in}}{\pgfqpoint{0.000000in}{0.000000in}}{%
\pgfpathmoveto{\pgfqpoint{0.000000in}{0.000000in}}%
\pgfpathlineto{\pgfqpoint{0.000000in}{-0.048611in}}%
\pgfusepath{stroke,fill}%
}%
\begin{pgfscope}%
\pgfsys@transformshift{2.300000in}{3.960000in}%
\pgfsys@useobject{currentmarker}{}%
\end{pgfscope}%
\end{pgfscope}%
\begin{pgfscope}%
\pgfpathrectangle{\pgfqpoint{0.750000in}{3.960000in}}{\pgfqpoint{4.650000in}{3.080000in}}%
\pgfusepath{clip}%
\pgfsetrectcap%
\pgfsetroundjoin%
\pgfsetlinewidth{0.803000pt}%
\definecolor{currentstroke}{rgb}{0.690196,0.690196,0.690196}%
\pgfsetstrokecolor{currentstroke}%
\pgfsetdash{}{0pt}%
\pgfpathmoveto{\pgfqpoint{2.816667in}{3.960000in}}%
\pgfpathlineto{\pgfqpoint{2.816667in}{7.040000in}}%
\pgfusepath{stroke}%
\end{pgfscope}%
\begin{pgfscope}%
\pgfsetbuttcap%
\pgfsetroundjoin%
\definecolor{currentfill}{rgb}{0.000000,0.000000,0.000000}%
\pgfsetfillcolor{currentfill}%
\pgfsetlinewidth{0.803000pt}%
\definecolor{currentstroke}{rgb}{0.000000,0.000000,0.000000}%
\pgfsetstrokecolor{currentstroke}%
\pgfsetdash{}{0pt}%
\pgfsys@defobject{currentmarker}{\pgfqpoint{0.000000in}{-0.048611in}}{\pgfqpoint{0.000000in}{0.000000in}}{%
\pgfpathmoveto{\pgfqpoint{0.000000in}{0.000000in}}%
\pgfpathlineto{\pgfqpoint{0.000000in}{-0.048611in}}%
\pgfusepath{stroke,fill}%
}%
\begin{pgfscope}%
\pgfsys@transformshift{2.816667in}{3.960000in}%
\pgfsys@useobject{currentmarker}{}%
\end{pgfscope}%
\end{pgfscope}%
\begin{pgfscope}%
\pgfpathrectangle{\pgfqpoint{0.750000in}{3.960000in}}{\pgfqpoint{4.650000in}{3.080000in}}%
\pgfusepath{clip}%
\pgfsetrectcap%
\pgfsetroundjoin%
\pgfsetlinewidth{0.803000pt}%
\definecolor{currentstroke}{rgb}{0.690196,0.690196,0.690196}%
\pgfsetstrokecolor{currentstroke}%
\pgfsetdash{}{0pt}%
\pgfpathmoveto{\pgfqpoint{3.333333in}{3.960000in}}%
\pgfpathlineto{\pgfqpoint{3.333333in}{7.040000in}}%
\pgfusepath{stroke}%
\end{pgfscope}%
\begin{pgfscope}%
\pgfsetbuttcap%
\pgfsetroundjoin%
\definecolor{currentfill}{rgb}{0.000000,0.000000,0.000000}%
\pgfsetfillcolor{currentfill}%
\pgfsetlinewidth{0.803000pt}%
\definecolor{currentstroke}{rgb}{0.000000,0.000000,0.000000}%
\pgfsetstrokecolor{currentstroke}%
\pgfsetdash{}{0pt}%
\pgfsys@defobject{currentmarker}{\pgfqpoint{0.000000in}{-0.048611in}}{\pgfqpoint{0.000000in}{0.000000in}}{%
\pgfpathmoveto{\pgfqpoint{0.000000in}{0.000000in}}%
\pgfpathlineto{\pgfqpoint{0.000000in}{-0.048611in}}%
\pgfusepath{stroke,fill}%
}%
\begin{pgfscope}%
\pgfsys@transformshift{3.333333in}{3.960000in}%
\pgfsys@useobject{currentmarker}{}%
\end{pgfscope}%
\end{pgfscope}%
\begin{pgfscope}%
\pgfpathrectangle{\pgfqpoint{0.750000in}{3.960000in}}{\pgfqpoint{4.650000in}{3.080000in}}%
\pgfusepath{clip}%
\pgfsetrectcap%
\pgfsetroundjoin%
\pgfsetlinewidth{0.803000pt}%
\definecolor{currentstroke}{rgb}{0.690196,0.690196,0.690196}%
\pgfsetstrokecolor{currentstroke}%
\pgfsetdash{}{0pt}%
\pgfpathmoveto{\pgfqpoint{3.850000in}{3.960000in}}%
\pgfpathlineto{\pgfqpoint{3.850000in}{7.040000in}}%
\pgfusepath{stroke}%
\end{pgfscope}%
\begin{pgfscope}%
\pgfsetbuttcap%
\pgfsetroundjoin%
\definecolor{currentfill}{rgb}{0.000000,0.000000,0.000000}%
\pgfsetfillcolor{currentfill}%
\pgfsetlinewidth{0.803000pt}%
\definecolor{currentstroke}{rgb}{0.000000,0.000000,0.000000}%
\pgfsetstrokecolor{currentstroke}%
\pgfsetdash{}{0pt}%
\pgfsys@defobject{currentmarker}{\pgfqpoint{0.000000in}{-0.048611in}}{\pgfqpoint{0.000000in}{0.000000in}}{%
\pgfpathmoveto{\pgfqpoint{0.000000in}{0.000000in}}%
\pgfpathlineto{\pgfqpoint{0.000000in}{-0.048611in}}%
\pgfusepath{stroke,fill}%
}%
\begin{pgfscope}%
\pgfsys@transformshift{3.850000in}{3.960000in}%
\pgfsys@useobject{currentmarker}{}%
\end{pgfscope}%
\end{pgfscope}%
\begin{pgfscope}%
\pgfpathrectangle{\pgfqpoint{0.750000in}{3.960000in}}{\pgfqpoint{4.650000in}{3.080000in}}%
\pgfusepath{clip}%
\pgfsetrectcap%
\pgfsetroundjoin%
\pgfsetlinewidth{0.803000pt}%
\definecolor{currentstroke}{rgb}{0.690196,0.690196,0.690196}%
\pgfsetstrokecolor{currentstroke}%
\pgfsetdash{}{0pt}%
\pgfpathmoveto{\pgfqpoint{4.366667in}{3.960000in}}%
\pgfpathlineto{\pgfqpoint{4.366667in}{7.040000in}}%
\pgfusepath{stroke}%
\end{pgfscope}%
\begin{pgfscope}%
\pgfsetbuttcap%
\pgfsetroundjoin%
\definecolor{currentfill}{rgb}{0.000000,0.000000,0.000000}%
\pgfsetfillcolor{currentfill}%
\pgfsetlinewidth{0.803000pt}%
\definecolor{currentstroke}{rgb}{0.000000,0.000000,0.000000}%
\pgfsetstrokecolor{currentstroke}%
\pgfsetdash{}{0pt}%
\pgfsys@defobject{currentmarker}{\pgfqpoint{0.000000in}{-0.048611in}}{\pgfqpoint{0.000000in}{0.000000in}}{%
\pgfpathmoveto{\pgfqpoint{0.000000in}{0.000000in}}%
\pgfpathlineto{\pgfqpoint{0.000000in}{-0.048611in}}%
\pgfusepath{stroke,fill}%
}%
\begin{pgfscope}%
\pgfsys@transformshift{4.366667in}{3.960000in}%
\pgfsys@useobject{currentmarker}{}%
\end{pgfscope}%
\end{pgfscope}%
\begin{pgfscope}%
\pgfpathrectangle{\pgfqpoint{0.750000in}{3.960000in}}{\pgfqpoint{4.650000in}{3.080000in}}%
\pgfusepath{clip}%
\pgfsetrectcap%
\pgfsetroundjoin%
\pgfsetlinewidth{0.803000pt}%
\definecolor{currentstroke}{rgb}{0.690196,0.690196,0.690196}%
\pgfsetstrokecolor{currentstroke}%
\pgfsetdash{}{0pt}%
\pgfpathmoveto{\pgfqpoint{4.883333in}{3.960000in}}%
\pgfpathlineto{\pgfqpoint{4.883333in}{7.040000in}}%
\pgfusepath{stroke}%
\end{pgfscope}%
\begin{pgfscope}%
\pgfsetbuttcap%
\pgfsetroundjoin%
\definecolor{currentfill}{rgb}{0.000000,0.000000,0.000000}%
\pgfsetfillcolor{currentfill}%
\pgfsetlinewidth{0.803000pt}%
\definecolor{currentstroke}{rgb}{0.000000,0.000000,0.000000}%
\pgfsetstrokecolor{currentstroke}%
\pgfsetdash{}{0pt}%
\pgfsys@defobject{currentmarker}{\pgfqpoint{0.000000in}{-0.048611in}}{\pgfqpoint{0.000000in}{0.000000in}}{%
\pgfpathmoveto{\pgfqpoint{0.000000in}{0.000000in}}%
\pgfpathlineto{\pgfqpoint{0.000000in}{-0.048611in}}%
\pgfusepath{stroke,fill}%
}%
\begin{pgfscope}%
\pgfsys@transformshift{4.883333in}{3.960000in}%
\pgfsys@useobject{currentmarker}{}%
\end{pgfscope}%
\end{pgfscope}%
\begin{pgfscope}%
\pgfpathrectangle{\pgfqpoint{0.750000in}{3.960000in}}{\pgfqpoint{4.650000in}{3.080000in}}%
\pgfusepath{clip}%
\pgfsetrectcap%
\pgfsetroundjoin%
\pgfsetlinewidth{0.803000pt}%
\definecolor{currentstroke}{rgb}{0.690196,0.690196,0.690196}%
\pgfsetstrokecolor{currentstroke}%
\pgfsetdash{}{0pt}%
\pgfpathmoveto{\pgfqpoint{5.400000in}{3.960000in}}%
\pgfpathlineto{\pgfqpoint{5.400000in}{7.040000in}}%
\pgfusepath{stroke}%
\end{pgfscope}%
\begin{pgfscope}%
\pgfsetbuttcap%
\pgfsetroundjoin%
\definecolor{currentfill}{rgb}{0.000000,0.000000,0.000000}%
\pgfsetfillcolor{currentfill}%
\pgfsetlinewidth{0.803000pt}%
\definecolor{currentstroke}{rgb}{0.000000,0.000000,0.000000}%
\pgfsetstrokecolor{currentstroke}%
\pgfsetdash{}{0pt}%
\pgfsys@defobject{currentmarker}{\pgfqpoint{0.000000in}{-0.048611in}}{\pgfqpoint{0.000000in}{0.000000in}}{%
\pgfpathmoveto{\pgfqpoint{0.000000in}{0.000000in}}%
\pgfpathlineto{\pgfqpoint{0.000000in}{-0.048611in}}%
\pgfusepath{stroke,fill}%
}%
\begin{pgfscope}%
\pgfsys@transformshift{5.400000in}{3.960000in}%
\pgfsys@useobject{currentmarker}{}%
\end{pgfscope}%
\end{pgfscope}%
\begin{pgfscope}%
\pgfpathrectangle{\pgfqpoint{0.750000in}{3.960000in}}{\pgfqpoint{4.650000in}{3.080000in}}%
\pgfusepath{clip}%
\pgfsetrectcap%
\pgfsetroundjoin%
\pgfsetlinewidth{0.803000pt}%
\definecolor{currentstroke}{rgb}{0.690196,0.690196,0.690196}%
\pgfsetstrokecolor{currentstroke}%
\pgfsetdash{}{0pt}%
\pgfpathmoveto{\pgfqpoint{0.750000in}{3.960000in}}%
\pgfpathlineto{\pgfqpoint{5.400000in}{3.960000in}}%
\pgfusepath{stroke}%
\end{pgfscope}%
\begin{pgfscope}%
\pgfsetbuttcap%
\pgfsetroundjoin%
\definecolor{currentfill}{rgb}{0.000000,0.000000,0.000000}%
\pgfsetfillcolor{currentfill}%
\pgfsetlinewidth{0.803000pt}%
\definecolor{currentstroke}{rgb}{0.000000,0.000000,0.000000}%
\pgfsetstrokecolor{currentstroke}%
\pgfsetdash{}{0pt}%
\pgfsys@defobject{currentmarker}{\pgfqpoint{-0.048611in}{0.000000in}}{\pgfqpoint{-0.000000in}{0.000000in}}{%
\pgfpathmoveto{\pgfqpoint{-0.000000in}{0.000000in}}%
\pgfpathlineto{\pgfqpoint{-0.048611in}{0.000000in}}%
\pgfusepath{stroke,fill}%
}%
\begin{pgfscope}%
\pgfsys@transformshift{0.750000in}{3.960000in}%
\pgfsys@useobject{currentmarker}{}%
\end{pgfscope}%
\end{pgfscope}%
\begin{pgfscope}%
\definecolor{textcolor}{rgb}{0.000000,0.000000,0.000000}%
\pgfsetstrokecolor{textcolor}%
\pgfsetfillcolor{textcolor}%
\pgftext[x=0.475308in, y=3.908900in, left, base]{\color{textcolor}\rmfamily\fontsize{10.000000}{12.000000}\selectfont \(\displaystyle {0.0}\)}%
\end{pgfscope}%
\begin{pgfscope}%
\pgfpathrectangle{\pgfqpoint{0.750000in}{3.960000in}}{\pgfqpoint{4.650000in}{3.080000in}}%
\pgfusepath{clip}%
\pgfsetrectcap%
\pgfsetroundjoin%
\pgfsetlinewidth{0.803000pt}%
\definecolor{currentstroke}{rgb}{0.690196,0.690196,0.690196}%
\pgfsetstrokecolor{currentstroke}%
\pgfsetdash{}{0pt}%
\pgfpathmoveto{\pgfqpoint{0.750000in}{4.449140in}}%
\pgfpathlineto{\pgfqpoint{5.400000in}{4.449140in}}%
\pgfusepath{stroke}%
\end{pgfscope}%
\begin{pgfscope}%
\pgfsetbuttcap%
\pgfsetroundjoin%
\definecolor{currentfill}{rgb}{0.000000,0.000000,0.000000}%
\pgfsetfillcolor{currentfill}%
\pgfsetlinewidth{0.803000pt}%
\definecolor{currentstroke}{rgb}{0.000000,0.000000,0.000000}%
\pgfsetstrokecolor{currentstroke}%
\pgfsetdash{}{0pt}%
\pgfsys@defobject{currentmarker}{\pgfqpoint{-0.048611in}{0.000000in}}{\pgfqpoint{-0.000000in}{0.000000in}}{%
\pgfpathmoveto{\pgfqpoint{-0.000000in}{0.000000in}}%
\pgfpathlineto{\pgfqpoint{-0.048611in}{0.000000in}}%
\pgfusepath{stroke,fill}%
}%
\begin{pgfscope}%
\pgfsys@transformshift{0.750000in}{4.449140in}%
\pgfsys@useobject{currentmarker}{}%
\end{pgfscope}%
\end{pgfscope}%
\begin{pgfscope}%
\definecolor{textcolor}{rgb}{0.000000,0.000000,0.000000}%
\pgfsetstrokecolor{textcolor}%
\pgfsetfillcolor{textcolor}%
\pgftext[x=0.475308in, y=4.398040in, left, base]{\color{textcolor}\rmfamily\fontsize{10.000000}{12.000000}\selectfont \(\displaystyle {0.2}\)}%
\end{pgfscope}%
\begin{pgfscope}%
\pgfpathrectangle{\pgfqpoint{0.750000in}{3.960000in}}{\pgfqpoint{4.650000in}{3.080000in}}%
\pgfusepath{clip}%
\pgfsetrectcap%
\pgfsetroundjoin%
\pgfsetlinewidth{0.803000pt}%
\definecolor{currentstroke}{rgb}{0.690196,0.690196,0.690196}%
\pgfsetstrokecolor{currentstroke}%
\pgfsetdash{}{0pt}%
\pgfpathmoveto{\pgfqpoint{0.750000in}{4.938280in}}%
\pgfpathlineto{\pgfqpoint{5.400000in}{4.938280in}}%
\pgfusepath{stroke}%
\end{pgfscope}%
\begin{pgfscope}%
\pgfsetbuttcap%
\pgfsetroundjoin%
\definecolor{currentfill}{rgb}{0.000000,0.000000,0.000000}%
\pgfsetfillcolor{currentfill}%
\pgfsetlinewidth{0.803000pt}%
\definecolor{currentstroke}{rgb}{0.000000,0.000000,0.000000}%
\pgfsetstrokecolor{currentstroke}%
\pgfsetdash{}{0pt}%
\pgfsys@defobject{currentmarker}{\pgfqpoint{-0.048611in}{0.000000in}}{\pgfqpoint{-0.000000in}{0.000000in}}{%
\pgfpathmoveto{\pgfqpoint{-0.000000in}{0.000000in}}%
\pgfpathlineto{\pgfqpoint{-0.048611in}{0.000000in}}%
\pgfusepath{stroke,fill}%
}%
\begin{pgfscope}%
\pgfsys@transformshift{0.750000in}{4.938280in}%
\pgfsys@useobject{currentmarker}{}%
\end{pgfscope}%
\end{pgfscope}%
\begin{pgfscope}%
\definecolor{textcolor}{rgb}{0.000000,0.000000,0.000000}%
\pgfsetstrokecolor{textcolor}%
\pgfsetfillcolor{textcolor}%
\pgftext[x=0.475308in, y=4.887180in, left, base]{\color{textcolor}\rmfamily\fontsize{10.000000}{12.000000}\selectfont \(\displaystyle {0.4}\)}%
\end{pgfscope}%
\begin{pgfscope}%
\pgfpathrectangle{\pgfqpoint{0.750000in}{3.960000in}}{\pgfqpoint{4.650000in}{3.080000in}}%
\pgfusepath{clip}%
\pgfsetrectcap%
\pgfsetroundjoin%
\pgfsetlinewidth{0.803000pt}%
\definecolor{currentstroke}{rgb}{0.690196,0.690196,0.690196}%
\pgfsetstrokecolor{currentstroke}%
\pgfsetdash{}{0pt}%
\pgfpathmoveto{\pgfqpoint{0.750000in}{5.427421in}}%
\pgfpathlineto{\pgfqpoint{5.400000in}{5.427421in}}%
\pgfusepath{stroke}%
\end{pgfscope}%
\begin{pgfscope}%
\pgfsetbuttcap%
\pgfsetroundjoin%
\definecolor{currentfill}{rgb}{0.000000,0.000000,0.000000}%
\pgfsetfillcolor{currentfill}%
\pgfsetlinewidth{0.803000pt}%
\definecolor{currentstroke}{rgb}{0.000000,0.000000,0.000000}%
\pgfsetstrokecolor{currentstroke}%
\pgfsetdash{}{0pt}%
\pgfsys@defobject{currentmarker}{\pgfqpoint{-0.048611in}{0.000000in}}{\pgfqpoint{-0.000000in}{0.000000in}}{%
\pgfpathmoveto{\pgfqpoint{-0.000000in}{0.000000in}}%
\pgfpathlineto{\pgfqpoint{-0.048611in}{0.000000in}}%
\pgfusepath{stroke,fill}%
}%
\begin{pgfscope}%
\pgfsys@transformshift{0.750000in}{5.427421in}%
\pgfsys@useobject{currentmarker}{}%
\end{pgfscope}%
\end{pgfscope}%
\begin{pgfscope}%
\definecolor{textcolor}{rgb}{0.000000,0.000000,0.000000}%
\pgfsetstrokecolor{textcolor}%
\pgfsetfillcolor{textcolor}%
\pgftext[x=0.475308in, y=5.376321in, left, base]{\color{textcolor}\rmfamily\fontsize{10.000000}{12.000000}\selectfont \(\displaystyle {0.6}\)}%
\end{pgfscope}%
\begin{pgfscope}%
\pgfpathrectangle{\pgfqpoint{0.750000in}{3.960000in}}{\pgfqpoint{4.650000in}{3.080000in}}%
\pgfusepath{clip}%
\pgfsetrectcap%
\pgfsetroundjoin%
\pgfsetlinewidth{0.803000pt}%
\definecolor{currentstroke}{rgb}{0.690196,0.690196,0.690196}%
\pgfsetstrokecolor{currentstroke}%
\pgfsetdash{}{0pt}%
\pgfpathmoveto{\pgfqpoint{0.750000in}{5.916561in}}%
\pgfpathlineto{\pgfqpoint{5.400000in}{5.916561in}}%
\pgfusepath{stroke}%
\end{pgfscope}%
\begin{pgfscope}%
\pgfsetbuttcap%
\pgfsetroundjoin%
\definecolor{currentfill}{rgb}{0.000000,0.000000,0.000000}%
\pgfsetfillcolor{currentfill}%
\pgfsetlinewidth{0.803000pt}%
\definecolor{currentstroke}{rgb}{0.000000,0.000000,0.000000}%
\pgfsetstrokecolor{currentstroke}%
\pgfsetdash{}{0pt}%
\pgfsys@defobject{currentmarker}{\pgfqpoint{-0.048611in}{0.000000in}}{\pgfqpoint{-0.000000in}{0.000000in}}{%
\pgfpathmoveto{\pgfqpoint{-0.000000in}{0.000000in}}%
\pgfpathlineto{\pgfqpoint{-0.048611in}{0.000000in}}%
\pgfusepath{stroke,fill}%
}%
\begin{pgfscope}%
\pgfsys@transformshift{0.750000in}{5.916561in}%
\pgfsys@useobject{currentmarker}{}%
\end{pgfscope}%
\end{pgfscope}%
\begin{pgfscope}%
\definecolor{textcolor}{rgb}{0.000000,0.000000,0.000000}%
\pgfsetstrokecolor{textcolor}%
\pgfsetfillcolor{textcolor}%
\pgftext[x=0.475308in, y=5.865461in, left, base]{\color{textcolor}\rmfamily\fontsize{10.000000}{12.000000}\selectfont \(\displaystyle {0.8}\)}%
\end{pgfscope}%
\begin{pgfscope}%
\pgfpathrectangle{\pgfqpoint{0.750000in}{3.960000in}}{\pgfqpoint{4.650000in}{3.080000in}}%
\pgfusepath{clip}%
\pgfsetrectcap%
\pgfsetroundjoin%
\pgfsetlinewidth{0.803000pt}%
\definecolor{currentstroke}{rgb}{0.690196,0.690196,0.690196}%
\pgfsetstrokecolor{currentstroke}%
\pgfsetdash{}{0pt}%
\pgfpathmoveto{\pgfqpoint{0.750000in}{6.405701in}}%
\pgfpathlineto{\pgfqpoint{5.400000in}{6.405701in}}%
\pgfusepath{stroke}%
\end{pgfscope}%
\begin{pgfscope}%
\pgfsetbuttcap%
\pgfsetroundjoin%
\definecolor{currentfill}{rgb}{0.000000,0.000000,0.000000}%
\pgfsetfillcolor{currentfill}%
\pgfsetlinewidth{0.803000pt}%
\definecolor{currentstroke}{rgb}{0.000000,0.000000,0.000000}%
\pgfsetstrokecolor{currentstroke}%
\pgfsetdash{}{0pt}%
\pgfsys@defobject{currentmarker}{\pgfqpoint{-0.048611in}{0.000000in}}{\pgfqpoint{-0.000000in}{0.000000in}}{%
\pgfpathmoveto{\pgfqpoint{-0.000000in}{0.000000in}}%
\pgfpathlineto{\pgfqpoint{-0.048611in}{0.000000in}}%
\pgfusepath{stroke,fill}%
}%
\begin{pgfscope}%
\pgfsys@transformshift{0.750000in}{6.405701in}%
\pgfsys@useobject{currentmarker}{}%
\end{pgfscope}%
\end{pgfscope}%
\begin{pgfscope}%
\definecolor{textcolor}{rgb}{0.000000,0.000000,0.000000}%
\pgfsetstrokecolor{textcolor}%
\pgfsetfillcolor{textcolor}%
\pgftext[x=0.475308in, y=6.354601in, left, base]{\color{textcolor}\rmfamily\fontsize{10.000000}{12.000000}\selectfont \(\displaystyle {1.0}\)}%
\end{pgfscope}%
\begin{pgfscope}%
\pgfpathrectangle{\pgfqpoint{0.750000in}{3.960000in}}{\pgfqpoint{4.650000in}{3.080000in}}%
\pgfusepath{clip}%
\pgfsetrectcap%
\pgfsetroundjoin%
\pgfsetlinewidth{0.803000pt}%
\definecolor{currentstroke}{rgb}{0.690196,0.690196,0.690196}%
\pgfsetstrokecolor{currentstroke}%
\pgfsetdash{}{0pt}%
\pgfpathmoveto{\pgfqpoint{0.750000in}{6.894841in}}%
\pgfpathlineto{\pgfqpoint{5.400000in}{6.894841in}}%
\pgfusepath{stroke}%
\end{pgfscope}%
\begin{pgfscope}%
\pgfsetbuttcap%
\pgfsetroundjoin%
\definecolor{currentfill}{rgb}{0.000000,0.000000,0.000000}%
\pgfsetfillcolor{currentfill}%
\pgfsetlinewidth{0.803000pt}%
\definecolor{currentstroke}{rgb}{0.000000,0.000000,0.000000}%
\pgfsetstrokecolor{currentstroke}%
\pgfsetdash{}{0pt}%
\pgfsys@defobject{currentmarker}{\pgfqpoint{-0.048611in}{0.000000in}}{\pgfqpoint{-0.000000in}{0.000000in}}{%
\pgfpathmoveto{\pgfqpoint{-0.000000in}{0.000000in}}%
\pgfpathlineto{\pgfqpoint{-0.048611in}{0.000000in}}%
\pgfusepath{stroke,fill}%
}%
\begin{pgfscope}%
\pgfsys@transformshift{0.750000in}{6.894841in}%
\pgfsys@useobject{currentmarker}{}%
\end{pgfscope}%
\end{pgfscope}%
\begin{pgfscope}%
\definecolor{textcolor}{rgb}{0.000000,0.000000,0.000000}%
\pgfsetstrokecolor{textcolor}%
\pgfsetfillcolor{textcolor}%
\pgftext[x=0.475308in, y=6.843741in, left, base]{\color{textcolor}\rmfamily\fontsize{10.000000}{12.000000}\selectfont \(\displaystyle {1.2}\)}%
\end{pgfscope}%
\begin{pgfscope}%
\definecolor{textcolor}{rgb}{0.000000,0.000000,0.000000}%
\pgfsetstrokecolor{textcolor}%
\pgfsetfillcolor{textcolor}%
\pgftext[x=0.419752in,y=5.500000in,,bottom,rotate=90.000000]{\color{textcolor}\rmfamily\fontsize{10.000000}{12.000000}\selectfont power in pu}%
\end{pgfscope}%
\begin{pgfscope}%
\pgfpathrectangle{\pgfqpoint{0.750000in}{3.960000in}}{\pgfqpoint{4.650000in}{3.080000in}}%
\pgfusepath{clip}%
\pgfsetrectcap%
\pgfsetroundjoin%
\pgfsetlinewidth{2.007500pt}%
\definecolor{currentstroke}{rgb}{0.121569,0.466667,0.705882}%
\pgfsetstrokecolor{currentstroke}%
\pgfsetdash{}{0pt}%
\pgfpathmoveto{\pgfqpoint{0.750000in}{3.960000in}}%
\pgfpathlineto{\pgfqpoint{0.844898in}{4.148036in}}%
\pgfpathlineto{\pgfqpoint{0.939796in}{4.335299in}}%
\pgfpathlineto{\pgfqpoint{1.034694in}{4.521020in}}%
\pgfpathlineto{\pgfqpoint{1.129592in}{4.704436in}}%
\pgfpathlineto{\pgfqpoint{1.224490in}{4.884793in}}%
\pgfpathlineto{\pgfqpoint{1.319388in}{5.061349in}}%
\pgfpathlineto{\pgfqpoint{1.414286in}{5.233380in}}%
\pgfpathlineto{\pgfqpoint{1.509184in}{5.400178in}}%
\pgfpathlineto{\pgfqpoint{1.604082in}{5.561058in}}%
\pgfpathlineto{\pgfqpoint{1.698980in}{5.715359in}}%
\pgfpathlineto{\pgfqpoint{1.793878in}{5.862447in}}%
\pgfpathlineto{\pgfqpoint{1.888776in}{6.001718in}}%
\pgfpathlineto{\pgfqpoint{1.983673in}{6.132598in}}%
\pgfpathlineto{\pgfqpoint{2.078571in}{6.254551in}}%
\pgfpathlineto{\pgfqpoint{2.173469in}{6.367075in}}%
\pgfpathlineto{\pgfqpoint{2.268367in}{6.469708in}}%
\pgfpathlineto{\pgfqpoint{2.363265in}{6.562028in}}%
\pgfpathlineto{\pgfqpoint{2.458163in}{6.643656in}}%
\pgfpathlineto{\pgfqpoint{2.553061in}{6.714256in}}%
\pgfpathlineto{\pgfqpoint{2.647959in}{6.773538in}}%
\pgfpathlineto{\pgfqpoint{2.742857in}{6.821259in}}%
\pgfpathlineto{\pgfqpoint{2.837755in}{6.857222in}}%
\pgfpathlineto{\pgfqpoint{2.932653in}{6.881280in}}%
\pgfpathlineto{\pgfqpoint{3.027551in}{6.893333in}}%
\pgfpathlineto{\pgfqpoint{3.122449in}{6.893333in}}%
\pgfpathlineto{\pgfqpoint{3.217347in}{6.881280in}}%
\pgfpathlineto{\pgfqpoint{3.312245in}{6.857222in}}%
\pgfpathlineto{\pgfqpoint{3.407143in}{6.821259in}}%
\pgfpathlineto{\pgfqpoint{3.502041in}{6.773538in}}%
\pgfpathlineto{\pgfqpoint{3.596939in}{6.714256in}}%
\pgfpathlineto{\pgfqpoint{3.691837in}{6.643656in}}%
\pgfpathlineto{\pgfqpoint{3.786735in}{6.562028in}}%
\pgfpathlineto{\pgfqpoint{3.881633in}{6.469708in}}%
\pgfpathlineto{\pgfqpoint{3.976531in}{6.367075in}}%
\pgfpathlineto{\pgfqpoint{4.071429in}{6.254551in}}%
\pgfpathlineto{\pgfqpoint{4.166327in}{6.132598in}}%
\pgfpathlineto{\pgfqpoint{4.261224in}{6.001718in}}%
\pgfpathlineto{\pgfqpoint{4.356122in}{5.862447in}}%
\pgfpathlineto{\pgfqpoint{4.451020in}{5.715359in}}%
\pgfpathlineto{\pgfqpoint{4.545918in}{5.561058in}}%
\pgfpathlineto{\pgfqpoint{4.640816in}{5.400178in}}%
\pgfpathlineto{\pgfqpoint{4.735714in}{5.233380in}}%
\pgfpathlineto{\pgfqpoint{4.830612in}{5.061349in}}%
\pgfpathlineto{\pgfqpoint{4.925510in}{4.884793in}}%
\pgfpathlineto{\pgfqpoint{5.020408in}{4.704436in}}%
\pgfpathlineto{\pgfqpoint{5.115306in}{4.521020in}}%
\pgfpathlineto{\pgfqpoint{5.210204in}{4.335299in}}%
\pgfpathlineto{\pgfqpoint{5.305102in}{4.148036in}}%
\pgfpathlineto{\pgfqpoint{5.400000in}{3.960000in}}%
\pgfusepath{stroke}%
\end{pgfscope}%
\begin{pgfscope}%
\pgfpathrectangle{\pgfqpoint{0.750000in}{3.960000in}}{\pgfqpoint{4.650000in}{3.080000in}}%
\pgfusepath{clip}%
\pgfsetrectcap%
\pgfsetroundjoin%
\pgfsetlinewidth{2.007500pt}%
\definecolor{currentstroke}{rgb}{1.000000,0.498039,0.054902}%
\pgfsetstrokecolor{currentstroke}%
\pgfsetdash{}{0pt}%
\pgfpathmoveto{\pgfqpoint{0.750000in}{3.960000in}}%
\pgfpathlineto{\pgfqpoint{0.844898in}{4.086553in}}%
\pgfpathlineto{\pgfqpoint{0.939796in}{4.212586in}}%
\pgfpathlineto{\pgfqpoint{1.034694in}{4.337580in}}%
\pgfpathlineto{\pgfqpoint{1.129592in}{4.461024in}}%
\pgfpathlineto{\pgfqpoint{1.224490in}{4.582408in}}%
\pgfpathlineto{\pgfqpoint{1.319388in}{4.701235in}}%
\pgfpathlineto{\pgfqpoint{1.414286in}{4.817016in}}%
\pgfpathlineto{\pgfqpoint{1.509184in}{4.929275in}}%
\pgfpathlineto{\pgfqpoint{1.604082in}{5.037552in}}%
\pgfpathlineto{\pgfqpoint{1.698980in}{5.141400in}}%
\pgfpathlineto{\pgfqpoint{1.793878in}{5.240394in}}%
\pgfpathlineto{\pgfqpoint{1.888776in}{5.334126in}}%
\pgfpathlineto{\pgfqpoint{1.983673in}{5.422212in}}%
\pgfpathlineto{\pgfqpoint{2.078571in}{5.504289in}}%
\pgfpathlineto{\pgfqpoint{2.173469in}{5.580021in}}%
\pgfpathlineto{\pgfqpoint{2.268367in}{5.649095in}}%
\pgfpathlineto{\pgfqpoint{2.363265in}{5.711229in}}%
\pgfpathlineto{\pgfqpoint{2.458163in}{5.766166in}}%
\pgfpathlineto{\pgfqpoint{2.553061in}{5.813682in}}%
\pgfpathlineto{\pgfqpoint{2.647959in}{5.853580in}}%
\pgfpathlineto{\pgfqpoint{2.742857in}{5.885697in}}%
\pgfpathlineto{\pgfqpoint{2.837755in}{5.909901in}}%
\pgfpathlineto{\pgfqpoint{2.932653in}{5.926093in}}%
\pgfpathlineto{\pgfqpoint{3.027551in}{5.934205in}}%
\pgfpathlineto{\pgfqpoint{3.122449in}{5.934205in}}%
\pgfpathlineto{\pgfqpoint{3.217347in}{5.926093in}}%
\pgfpathlineto{\pgfqpoint{3.312245in}{5.909901in}}%
\pgfpathlineto{\pgfqpoint{3.407143in}{5.885697in}}%
\pgfpathlineto{\pgfqpoint{3.502041in}{5.853580in}}%
\pgfpathlineto{\pgfqpoint{3.596939in}{5.813682in}}%
\pgfpathlineto{\pgfqpoint{3.691837in}{5.766166in}}%
\pgfpathlineto{\pgfqpoint{3.786735in}{5.711229in}}%
\pgfpathlineto{\pgfqpoint{3.881633in}{5.649095in}}%
\pgfpathlineto{\pgfqpoint{3.976531in}{5.580021in}}%
\pgfpathlineto{\pgfqpoint{4.071429in}{5.504289in}}%
\pgfpathlineto{\pgfqpoint{4.166327in}{5.422212in}}%
\pgfpathlineto{\pgfqpoint{4.261224in}{5.334126in}}%
\pgfpathlineto{\pgfqpoint{4.356122in}{5.240394in}}%
\pgfpathlineto{\pgfqpoint{4.451020in}{5.141400in}}%
\pgfpathlineto{\pgfqpoint{4.545918in}{5.037552in}}%
\pgfpathlineto{\pgfqpoint{4.640816in}{4.929275in}}%
\pgfpathlineto{\pgfqpoint{4.735714in}{4.817016in}}%
\pgfpathlineto{\pgfqpoint{4.830612in}{4.701235in}}%
\pgfpathlineto{\pgfqpoint{4.925510in}{4.582408in}}%
\pgfpathlineto{\pgfqpoint{5.020408in}{4.461024in}}%
\pgfpathlineto{\pgfqpoint{5.115306in}{4.337580in}}%
\pgfpathlineto{\pgfqpoint{5.210204in}{4.212586in}}%
\pgfpathlineto{\pgfqpoint{5.305102in}{4.086553in}}%
\pgfpathlineto{\pgfqpoint{5.400000in}{3.960000in}}%
\pgfusepath{stroke}%
\end{pgfscope}%
\begin{pgfscope}%
\pgfpathrectangle{\pgfqpoint{0.750000in}{3.960000in}}{\pgfqpoint{4.650000in}{3.080000in}}%
\pgfusepath{clip}%
\pgfsetrectcap%
\pgfsetroundjoin%
\pgfsetlinewidth{2.007500pt}%
\definecolor{currentstroke}{rgb}{0.172549,0.627451,0.172549}%
\pgfsetstrokecolor{currentstroke}%
\pgfsetdash{}{0pt}%
\pgfpathmoveto{\pgfqpoint{0.750000in}{6.161131in}}%
\pgfpathlineto{\pgfqpoint{0.844898in}{6.161131in}}%
\pgfpathlineto{\pgfqpoint{0.939796in}{6.161131in}}%
\pgfpathlineto{\pgfqpoint{1.034694in}{6.161131in}}%
\pgfpathlineto{\pgfqpoint{1.129592in}{6.161131in}}%
\pgfpathlineto{\pgfqpoint{1.224490in}{6.161131in}}%
\pgfpathlineto{\pgfqpoint{1.319388in}{6.161131in}}%
\pgfpathlineto{\pgfqpoint{1.414286in}{6.161131in}}%
\pgfpathlineto{\pgfqpoint{1.509184in}{6.161131in}}%
\pgfpathlineto{\pgfqpoint{1.604082in}{6.161131in}}%
\pgfpathlineto{\pgfqpoint{1.698980in}{6.161131in}}%
\pgfpathlineto{\pgfqpoint{1.793878in}{6.161131in}}%
\pgfpathlineto{\pgfqpoint{1.888776in}{6.161131in}}%
\pgfpathlineto{\pgfqpoint{1.983673in}{6.161131in}}%
\pgfpathlineto{\pgfqpoint{2.078571in}{6.161131in}}%
\pgfpathlineto{\pgfqpoint{2.173469in}{6.161131in}}%
\pgfpathlineto{\pgfqpoint{2.268367in}{6.161131in}}%
\pgfpathlineto{\pgfqpoint{2.363265in}{6.161131in}}%
\pgfpathlineto{\pgfqpoint{2.458163in}{6.161131in}}%
\pgfpathlineto{\pgfqpoint{2.553061in}{6.161131in}}%
\pgfpathlineto{\pgfqpoint{2.647959in}{6.161131in}}%
\pgfpathlineto{\pgfqpoint{2.742857in}{6.161131in}}%
\pgfpathlineto{\pgfqpoint{2.837755in}{6.161131in}}%
\pgfpathlineto{\pgfqpoint{2.932653in}{6.161131in}}%
\pgfpathlineto{\pgfqpoint{3.027551in}{6.161131in}}%
\pgfpathlineto{\pgfqpoint{3.122449in}{6.161131in}}%
\pgfpathlineto{\pgfqpoint{3.217347in}{6.161131in}}%
\pgfpathlineto{\pgfqpoint{3.312245in}{6.161131in}}%
\pgfpathlineto{\pgfqpoint{3.407143in}{6.161131in}}%
\pgfpathlineto{\pgfqpoint{3.502041in}{6.161131in}}%
\pgfpathlineto{\pgfqpoint{3.596939in}{6.161131in}}%
\pgfpathlineto{\pgfqpoint{3.691837in}{6.161131in}}%
\pgfpathlineto{\pgfqpoint{3.786735in}{6.161131in}}%
\pgfpathlineto{\pgfqpoint{3.881633in}{6.161131in}}%
\pgfpathlineto{\pgfqpoint{3.976531in}{6.161131in}}%
\pgfpathlineto{\pgfqpoint{4.071429in}{6.161131in}}%
\pgfpathlineto{\pgfqpoint{4.166327in}{6.161131in}}%
\pgfpathlineto{\pgfqpoint{4.261224in}{6.161131in}}%
\pgfpathlineto{\pgfqpoint{4.356122in}{6.161131in}}%
\pgfpathlineto{\pgfqpoint{4.451020in}{6.161131in}}%
\pgfpathlineto{\pgfqpoint{4.545918in}{6.161131in}}%
\pgfpathlineto{\pgfqpoint{4.640816in}{6.161131in}}%
\pgfpathlineto{\pgfqpoint{4.735714in}{6.161131in}}%
\pgfpathlineto{\pgfqpoint{4.830612in}{6.161131in}}%
\pgfpathlineto{\pgfqpoint{4.925510in}{6.161131in}}%
\pgfpathlineto{\pgfqpoint{5.020408in}{6.161131in}}%
\pgfpathlineto{\pgfqpoint{5.115306in}{6.161131in}}%
\pgfpathlineto{\pgfqpoint{5.210204in}{6.161131in}}%
\pgfpathlineto{\pgfqpoint{5.305102in}{6.161131in}}%
\pgfpathlineto{\pgfqpoint{5.400000in}{6.161131in}}%
\pgfusepath{stroke}%
\end{pgfscope}%
\begin{pgfscope}%
\pgfsetrectcap%
\pgfsetmiterjoin%
\pgfsetlinewidth{0.803000pt}%
\definecolor{currentstroke}{rgb}{0.000000,0.000000,0.000000}%
\pgfsetstrokecolor{currentstroke}%
\pgfsetdash{}{0pt}%
\pgfpathmoveto{\pgfqpoint{0.750000in}{3.960000in}}%
\pgfpathlineto{\pgfqpoint{0.750000in}{7.040000in}}%
\pgfusepath{stroke}%
\end{pgfscope}%
\begin{pgfscope}%
\pgfsetrectcap%
\pgfsetmiterjoin%
\pgfsetlinewidth{0.803000pt}%
\definecolor{currentstroke}{rgb}{0.000000,0.000000,0.000000}%
\pgfsetstrokecolor{currentstroke}%
\pgfsetdash{}{0pt}%
\pgfpathmoveto{\pgfqpoint{5.400000in}{3.960000in}}%
\pgfpathlineto{\pgfqpoint{5.400000in}{7.040000in}}%
\pgfusepath{stroke}%
\end{pgfscope}%
\begin{pgfscope}%
\pgfsetrectcap%
\pgfsetmiterjoin%
\pgfsetlinewidth{0.803000pt}%
\definecolor{currentstroke}{rgb}{0.000000,0.000000,0.000000}%
\pgfsetstrokecolor{currentstroke}%
\pgfsetdash{}{0pt}%
\pgfpathmoveto{\pgfqpoint{0.750000in}{3.960000in}}%
\pgfpathlineto{\pgfqpoint{5.400000in}{3.960000in}}%
\pgfusepath{stroke}%
\end{pgfscope}%
\begin{pgfscope}%
\pgfsetrectcap%
\pgfsetmiterjoin%
\pgfsetlinewidth{0.803000pt}%
\definecolor{currentstroke}{rgb}{0.000000,0.000000,0.000000}%
\pgfsetstrokecolor{currentstroke}%
\pgfsetdash{}{0pt}%
\pgfpathmoveto{\pgfqpoint{0.750000in}{7.040000in}}%
\pgfpathlineto{\pgfqpoint{5.400000in}{7.040000in}}%
\pgfusepath{stroke}%
\end{pgfscope}%
\begin{pgfscope}%
\pgfsetbuttcap%
\pgfsetmiterjoin%
\definecolor{currentfill}{rgb}{1.000000,1.000000,1.000000}%
\pgfsetfillcolor{currentfill}%
\pgfsetfillopacity{0.800000}%
\pgfsetlinewidth{1.003750pt}%
\definecolor{currentstroke}{rgb}{0.800000,0.800000,0.800000}%
\pgfsetstrokecolor{currentstroke}%
\pgfsetstrokeopacity{0.800000}%
\pgfsetdash{}{0pt}%
\pgfpathmoveto{\pgfqpoint{2.342004in}{4.029444in}}%
\pgfpathlineto{\pgfqpoint{3.807996in}{4.029444in}}%
\pgfpathquadraticcurveto{\pgfqpoint{3.835774in}{4.029444in}}{\pgfqpoint{3.835774in}{4.057222in}}%
\pgfpathlineto{\pgfqpoint{3.835774in}{4.651311in}}%
\pgfpathquadraticcurveto{\pgfqpoint{3.835774in}{4.679088in}}{\pgfqpoint{3.807996in}{4.679088in}}%
\pgfpathlineto{\pgfqpoint{2.342004in}{4.679088in}}%
\pgfpathquadraticcurveto{\pgfqpoint{2.314226in}{4.679088in}}{\pgfqpoint{2.314226in}{4.651311in}}%
\pgfpathlineto{\pgfqpoint{2.314226in}{4.057222in}}%
\pgfpathquadraticcurveto{\pgfqpoint{2.314226in}{4.029444in}}{\pgfqpoint{2.342004in}{4.029444in}}%
\pgfpathlineto{\pgfqpoint{2.342004in}{4.029444in}}%
\pgfpathclose%
\pgfusepath{stroke,fill}%
\end{pgfscope}%
\begin{pgfscope}%
\pgfsetrectcap%
\pgfsetroundjoin%
\pgfsetlinewidth{2.007500pt}%
\definecolor{currentstroke}{rgb}{0.121569,0.466667,0.705882}%
\pgfsetstrokecolor{currentstroke}%
\pgfsetdash{}{0pt}%
\pgfpathmoveto{\pgfqpoint{2.369781in}{4.568791in}}%
\pgfpathlineto{\pgfqpoint{2.508670in}{4.568791in}}%
\pgfpathlineto{\pgfqpoint{2.647559in}{4.568791in}}%
\pgfusepath{stroke}%
\end{pgfscope}%
\begin{pgfscope}%
\definecolor{textcolor}{rgb}{0.000000,0.000000,0.000000}%
\pgfsetstrokecolor{textcolor}%
\pgfsetfillcolor{textcolor}%
\pgftext[x=2.758670in,y=4.520180in,left,base]{\color{textcolor}\rmfamily\fontsize{10.000000}{12.000000}\selectfont \(\displaystyle P_\mathrm{e}\) pre-fault}%
\end{pgfscope}%
\begin{pgfscope}%
\pgfsetrectcap%
\pgfsetroundjoin%
\pgfsetlinewidth{2.007500pt}%
\definecolor{currentstroke}{rgb}{1.000000,0.498039,0.054902}%
\pgfsetstrokecolor{currentstroke}%
\pgfsetdash{}{0pt}%
\pgfpathmoveto{\pgfqpoint{2.369781in}{4.365748in}}%
\pgfpathlineto{\pgfqpoint{2.508670in}{4.365748in}}%
\pgfpathlineto{\pgfqpoint{2.647559in}{4.365748in}}%
\pgfusepath{stroke}%
\end{pgfscope}%
\begin{pgfscope}%
\definecolor{textcolor}{rgb}{0.000000,0.000000,0.000000}%
\pgfsetstrokecolor{textcolor}%
\pgfsetfillcolor{textcolor}%
\pgftext[x=2.758670in,y=4.317137in,left,base]{\color{textcolor}\rmfamily\fontsize{10.000000}{12.000000}\selectfont \(\displaystyle P_\mathrm{e}\) post-fault}%
\end{pgfscope}%
\begin{pgfscope}%
\pgfsetrectcap%
\pgfsetroundjoin%
\pgfsetlinewidth{2.007500pt}%
\definecolor{currentstroke}{rgb}{0.172549,0.627451,0.172549}%
\pgfsetstrokecolor{currentstroke}%
\pgfsetdash{}{0pt}%
\pgfpathmoveto{\pgfqpoint{2.369781in}{4.163857in}}%
\pgfpathlineto{\pgfqpoint{2.508670in}{4.163857in}}%
\pgfpathlineto{\pgfqpoint{2.647559in}{4.163857in}}%
\pgfusepath{stroke}%
\end{pgfscope}%
\begin{pgfscope}%
\definecolor{textcolor}{rgb}{0.000000,0.000000,0.000000}%
\pgfsetstrokecolor{textcolor}%
\pgfsetfillcolor{textcolor}%
\pgftext[x=2.758670in,y=4.115246in,left,base]{\color{textcolor}\rmfamily\fontsize{10.000000}{12.000000}\selectfont \(\displaystyle P_\mathrm{T}\) of the turbine}%
\end{pgfscope}%
\begin{pgfscope}%
\pgfsetbuttcap%
\pgfsetmiterjoin%
\definecolor{currentfill}{rgb}{1.000000,1.000000,1.000000}%
\pgfsetfillcolor{currentfill}%
\pgfsetlinewidth{0.000000pt}%
\definecolor{currentstroke}{rgb}{0.000000,0.000000,0.000000}%
\pgfsetstrokecolor{currentstroke}%
\pgfsetstrokeopacity{0.000000}%
\pgfsetdash{}{0pt}%
\pgfpathmoveto{\pgfqpoint{0.750000in}{0.880000in}}%
\pgfpathlineto{\pgfqpoint{5.400000in}{0.880000in}}%
\pgfpathlineto{\pgfqpoint{5.400000in}{3.960000in}}%
\pgfpathlineto{\pgfqpoint{0.750000in}{3.960000in}}%
\pgfpathlineto{\pgfqpoint{0.750000in}{0.880000in}}%
\pgfpathclose%
\pgfusepath{fill}%
\end{pgfscope}%
\begin{pgfscope}%
\pgfpathrectangle{\pgfqpoint{0.750000in}{0.880000in}}{\pgfqpoint{4.650000in}{3.080000in}}%
\pgfusepath{clip}%
\pgfsetrectcap%
\pgfsetroundjoin%
\pgfsetlinewidth{0.803000pt}%
\definecolor{currentstroke}{rgb}{0.690196,0.690196,0.690196}%
\pgfsetstrokecolor{currentstroke}%
\pgfsetdash{}{0pt}%
\pgfpathmoveto{\pgfqpoint{0.750000in}{0.880000in}}%
\pgfpathlineto{\pgfqpoint{0.750000in}{3.960000in}}%
\pgfusepath{stroke}%
\end{pgfscope}%
\begin{pgfscope}%
\pgfsetbuttcap%
\pgfsetroundjoin%
\definecolor{currentfill}{rgb}{0.000000,0.000000,0.000000}%
\pgfsetfillcolor{currentfill}%
\pgfsetlinewidth{0.803000pt}%
\definecolor{currentstroke}{rgb}{0.000000,0.000000,0.000000}%
\pgfsetstrokecolor{currentstroke}%
\pgfsetdash{}{0pt}%
\pgfsys@defobject{currentmarker}{\pgfqpoint{0.000000in}{-0.048611in}}{\pgfqpoint{0.000000in}{0.000000in}}{%
\pgfpathmoveto{\pgfqpoint{0.000000in}{0.000000in}}%
\pgfpathlineto{\pgfqpoint{0.000000in}{-0.048611in}}%
\pgfusepath{stroke,fill}%
}%
\begin{pgfscope}%
\pgfsys@transformshift{0.750000in}{0.880000in}%
\pgfsys@useobject{currentmarker}{}%
\end{pgfscope}%
\end{pgfscope}%
\begin{pgfscope}%
\definecolor{textcolor}{rgb}{0.000000,0.000000,0.000000}%
\pgfsetstrokecolor{textcolor}%
\pgfsetfillcolor{textcolor}%
\pgftext[x=0.750000in,y=0.782778in,,top]{\color{textcolor}\rmfamily\fontsize{10.000000}{12.000000}\selectfont \(\displaystyle {0}\)}%
\end{pgfscope}%
\begin{pgfscope}%
\pgfpathrectangle{\pgfqpoint{0.750000in}{0.880000in}}{\pgfqpoint{4.650000in}{3.080000in}}%
\pgfusepath{clip}%
\pgfsetrectcap%
\pgfsetroundjoin%
\pgfsetlinewidth{0.803000pt}%
\definecolor{currentstroke}{rgb}{0.690196,0.690196,0.690196}%
\pgfsetstrokecolor{currentstroke}%
\pgfsetdash{}{0pt}%
\pgfpathmoveto{\pgfqpoint{1.266667in}{0.880000in}}%
\pgfpathlineto{\pgfqpoint{1.266667in}{3.960000in}}%
\pgfusepath{stroke}%
\end{pgfscope}%
\begin{pgfscope}%
\pgfsetbuttcap%
\pgfsetroundjoin%
\definecolor{currentfill}{rgb}{0.000000,0.000000,0.000000}%
\pgfsetfillcolor{currentfill}%
\pgfsetlinewidth{0.803000pt}%
\definecolor{currentstroke}{rgb}{0.000000,0.000000,0.000000}%
\pgfsetstrokecolor{currentstroke}%
\pgfsetdash{}{0pt}%
\pgfsys@defobject{currentmarker}{\pgfqpoint{0.000000in}{-0.048611in}}{\pgfqpoint{0.000000in}{0.000000in}}{%
\pgfpathmoveto{\pgfqpoint{0.000000in}{0.000000in}}%
\pgfpathlineto{\pgfqpoint{0.000000in}{-0.048611in}}%
\pgfusepath{stroke,fill}%
}%
\begin{pgfscope}%
\pgfsys@transformshift{1.266667in}{0.880000in}%
\pgfsys@useobject{currentmarker}{}%
\end{pgfscope}%
\end{pgfscope}%
\begin{pgfscope}%
\definecolor{textcolor}{rgb}{0.000000,0.000000,0.000000}%
\pgfsetstrokecolor{textcolor}%
\pgfsetfillcolor{textcolor}%
\pgftext[x=1.266667in,y=0.782778in,,top]{\color{textcolor}\rmfamily\fontsize{10.000000}{12.000000}\selectfont \(\displaystyle {20}\)}%
\end{pgfscope}%
\begin{pgfscope}%
\pgfpathrectangle{\pgfqpoint{0.750000in}{0.880000in}}{\pgfqpoint{4.650000in}{3.080000in}}%
\pgfusepath{clip}%
\pgfsetrectcap%
\pgfsetroundjoin%
\pgfsetlinewidth{0.803000pt}%
\definecolor{currentstroke}{rgb}{0.690196,0.690196,0.690196}%
\pgfsetstrokecolor{currentstroke}%
\pgfsetdash{}{0pt}%
\pgfpathmoveto{\pgfqpoint{1.783333in}{0.880000in}}%
\pgfpathlineto{\pgfqpoint{1.783333in}{3.960000in}}%
\pgfusepath{stroke}%
\end{pgfscope}%
\begin{pgfscope}%
\pgfsetbuttcap%
\pgfsetroundjoin%
\definecolor{currentfill}{rgb}{0.000000,0.000000,0.000000}%
\pgfsetfillcolor{currentfill}%
\pgfsetlinewidth{0.803000pt}%
\definecolor{currentstroke}{rgb}{0.000000,0.000000,0.000000}%
\pgfsetstrokecolor{currentstroke}%
\pgfsetdash{}{0pt}%
\pgfsys@defobject{currentmarker}{\pgfqpoint{0.000000in}{-0.048611in}}{\pgfqpoint{0.000000in}{0.000000in}}{%
\pgfpathmoveto{\pgfqpoint{0.000000in}{0.000000in}}%
\pgfpathlineto{\pgfqpoint{0.000000in}{-0.048611in}}%
\pgfusepath{stroke,fill}%
}%
\begin{pgfscope}%
\pgfsys@transformshift{1.783333in}{0.880000in}%
\pgfsys@useobject{currentmarker}{}%
\end{pgfscope}%
\end{pgfscope}%
\begin{pgfscope}%
\definecolor{textcolor}{rgb}{0.000000,0.000000,0.000000}%
\pgfsetstrokecolor{textcolor}%
\pgfsetfillcolor{textcolor}%
\pgftext[x=1.783333in,y=0.782778in,,top]{\color{textcolor}\rmfamily\fontsize{10.000000}{12.000000}\selectfont \(\displaystyle {40}\)}%
\end{pgfscope}%
\begin{pgfscope}%
\pgfpathrectangle{\pgfqpoint{0.750000in}{0.880000in}}{\pgfqpoint{4.650000in}{3.080000in}}%
\pgfusepath{clip}%
\pgfsetrectcap%
\pgfsetroundjoin%
\pgfsetlinewidth{0.803000pt}%
\definecolor{currentstroke}{rgb}{0.690196,0.690196,0.690196}%
\pgfsetstrokecolor{currentstroke}%
\pgfsetdash{}{0pt}%
\pgfpathmoveto{\pgfqpoint{2.300000in}{0.880000in}}%
\pgfpathlineto{\pgfqpoint{2.300000in}{3.960000in}}%
\pgfusepath{stroke}%
\end{pgfscope}%
\begin{pgfscope}%
\pgfsetbuttcap%
\pgfsetroundjoin%
\definecolor{currentfill}{rgb}{0.000000,0.000000,0.000000}%
\pgfsetfillcolor{currentfill}%
\pgfsetlinewidth{0.803000pt}%
\definecolor{currentstroke}{rgb}{0.000000,0.000000,0.000000}%
\pgfsetstrokecolor{currentstroke}%
\pgfsetdash{}{0pt}%
\pgfsys@defobject{currentmarker}{\pgfqpoint{0.000000in}{-0.048611in}}{\pgfqpoint{0.000000in}{0.000000in}}{%
\pgfpathmoveto{\pgfqpoint{0.000000in}{0.000000in}}%
\pgfpathlineto{\pgfqpoint{0.000000in}{-0.048611in}}%
\pgfusepath{stroke,fill}%
}%
\begin{pgfscope}%
\pgfsys@transformshift{2.300000in}{0.880000in}%
\pgfsys@useobject{currentmarker}{}%
\end{pgfscope}%
\end{pgfscope}%
\begin{pgfscope}%
\definecolor{textcolor}{rgb}{0.000000,0.000000,0.000000}%
\pgfsetstrokecolor{textcolor}%
\pgfsetfillcolor{textcolor}%
\pgftext[x=2.300000in,y=0.782778in,,top]{\color{textcolor}\rmfamily\fontsize{10.000000}{12.000000}\selectfont \(\displaystyle {60}\)}%
\end{pgfscope}%
\begin{pgfscope}%
\pgfpathrectangle{\pgfqpoint{0.750000in}{0.880000in}}{\pgfqpoint{4.650000in}{3.080000in}}%
\pgfusepath{clip}%
\pgfsetrectcap%
\pgfsetroundjoin%
\pgfsetlinewidth{0.803000pt}%
\definecolor{currentstroke}{rgb}{0.690196,0.690196,0.690196}%
\pgfsetstrokecolor{currentstroke}%
\pgfsetdash{}{0pt}%
\pgfpathmoveto{\pgfqpoint{2.816667in}{0.880000in}}%
\pgfpathlineto{\pgfqpoint{2.816667in}{3.960000in}}%
\pgfusepath{stroke}%
\end{pgfscope}%
\begin{pgfscope}%
\pgfsetbuttcap%
\pgfsetroundjoin%
\definecolor{currentfill}{rgb}{0.000000,0.000000,0.000000}%
\pgfsetfillcolor{currentfill}%
\pgfsetlinewidth{0.803000pt}%
\definecolor{currentstroke}{rgb}{0.000000,0.000000,0.000000}%
\pgfsetstrokecolor{currentstroke}%
\pgfsetdash{}{0pt}%
\pgfsys@defobject{currentmarker}{\pgfqpoint{0.000000in}{-0.048611in}}{\pgfqpoint{0.000000in}{0.000000in}}{%
\pgfpathmoveto{\pgfqpoint{0.000000in}{0.000000in}}%
\pgfpathlineto{\pgfqpoint{0.000000in}{-0.048611in}}%
\pgfusepath{stroke,fill}%
}%
\begin{pgfscope}%
\pgfsys@transformshift{2.816667in}{0.880000in}%
\pgfsys@useobject{currentmarker}{}%
\end{pgfscope}%
\end{pgfscope}%
\begin{pgfscope}%
\definecolor{textcolor}{rgb}{0.000000,0.000000,0.000000}%
\pgfsetstrokecolor{textcolor}%
\pgfsetfillcolor{textcolor}%
\pgftext[x=2.816667in,y=0.782778in,,top]{\color{textcolor}\rmfamily\fontsize{10.000000}{12.000000}\selectfont \(\displaystyle {80}\)}%
\end{pgfscope}%
\begin{pgfscope}%
\pgfpathrectangle{\pgfqpoint{0.750000in}{0.880000in}}{\pgfqpoint{4.650000in}{3.080000in}}%
\pgfusepath{clip}%
\pgfsetrectcap%
\pgfsetroundjoin%
\pgfsetlinewidth{0.803000pt}%
\definecolor{currentstroke}{rgb}{0.690196,0.690196,0.690196}%
\pgfsetstrokecolor{currentstroke}%
\pgfsetdash{}{0pt}%
\pgfpathmoveto{\pgfqpoint{3.333333in}{0.880000in}}%
\pgfpathlineto{\pgfqpoint{3.333333in}{3.960000in}}%
\pgfusepath{stroke}%
\end{pgfscope}%
\begin{pgfscope}%
\pgfsetbuttcap%
\pgfsetroundjoin%
\definecolor{currentfill}{rgb}{0.000000,0.000000,0.000000}%
\pgfsetfillcolor{currentfill}%
\pgfsetlinewidth{0.803000pt}%
\definecolor{currentstroke}{rgb}{0.000000,0.000000,0.000000}%
\pgfsetstrokecolor{currentstroke}%
\pgfsetdash{}{0pt}%
\pgfsys@defobject{currentmarker}{\pgfqpoint{0.000000in}{-0.048611in}}{\pgfqpoint{0.000000in}{0.000000in}}{%
\pgfpathmoveto{\pgfqpoint{0.000000in}{0.000000in}}%
\pgfpathlineto{\pgfqpoint{0.000000in}{-0.048611in}}%
\pgfusepath{stroke,fill}%
}%
\begin{pgfscope}%
\pgfsys@transformshift{3.333333in}{0.880000in}%
\pgfsys@useobject{currentmarker}{}%
\end{pgfscope}%
\end{pgfscope}%
\begin{pgfscope}%
\definecolor{textcolor}{rgb}{0.000000,0.000000,0.000000}%
\pgfsetstrokecolor{textcolor}%
\pgfsetfillcolor{textcolor}%
\pgftext[x=3.333333in,y=0.782778in,,top]{\color{textcolor}\rmfamily\fontsize{10.000000}{12.000000}\selectfont \(\displaystyle {100}\)}%
\end{pgfscope}%
\begin{pgfscope}%
\pgfpathrectangle{\pgfqpoint{0.750000in}{0.880000in}}{\pgfqpoint{4.650000in}{3.080000in}}%
\pgfusepath{clip}%
\pgfsetrectcap%
\pgfsetroundjoin%
\pgfsetlinewidth{0.803000pt}%
\definecolor{currentstroke}{rgb}{0.690196,0.690196,0.690196}%
\pgfsetstrokecolor{currentstroke}%
\pgfsetdash{}{0pt}%
\pgfpathmoveto{\pgfqpoint{3.850000in}{0.880000in}}%
\pgfpathlineto{\pgfqpoint{3.850000in}{3.960000in}}%
\pgfusepath{stroke}%
\end{pgfscope}%
\begin{pgfscope}%
\pgfsetbuttcap%
\pgfsetroundjoin%
\definecolor{currentfill}{rgb}{0.000000,0.000000,0.000000}%
\pgfsetfillcolor{currentfill}%
\pgfsetlinewidth{0.803000pt}%
\definecolor{currentstroke}{rgb}{0.000000,0.000000,0.000000}%
\pgfsetstrokecolor{currentstroke}%
\pgfsetdash{}{0pt}%
\pgfsys@defobject{currentmarker}{\pgfqpoint{0.000000in}{-0.048611in}}{\pgfqpoint{0.000000in}{0.000000in}}{%
\pgfpathmoveto{\pgfqpoint{0.000000in}{0.000000in}}%
\pgfpathlineto{\pgfqpoint{0.000000in}{-0.048611in}}%
\pgfusepath{stroke,fill}%
}%
\begin{pgfscope}%
\pgfsys@transformshift{3.850000in}{0.880000in}%
\pgfsys@useobject{currentmarker}{}%
\end{pgfscope}%
\end{pgfscope}%
\begin{pgfscope}%
\definecolor{textcolor}{rgb}{0.000000,0.000000,0.000000}%
\pgfsetstrokecolor{textcolor}%
\pgfsetfillcolor{textcolor}%
\pgftext[x=3.850000in,y=0.782778in,,top]{\color{textcolor}\rmfamily\fontsize{10.000000}{12.000000}\selectfont \(\displaystyle {120}\)}%
\end{pgfscope}%
\begin{pgfscope}%
\pgfpathrectangle{\pgfqpoint{0.750000in}{0.880000in}}{\pgfqpoint{4.650000in}{3.080000in}}%
\pgfusepath{clip}%
\pgfsetrectcap%
\pgfsetroundjoin%
\pgfsetlinewidth{0.803000pt}%
\definecolor{currentstroke}{rgb}{0.690196,0.690196,0.690196}%
\pgfsetstrokecolor{currentstroke}%
\pgfsetdash{}{0pt}%
\pgfpathmoveto{\pgfqpoint{4.366667in}{0.880000in}}%
\pgfpathlineto{\pgfqpoint{4.366667in}{3.960000in}}%
\pgfusepath{stroke}%
\end{pgfscope}%
\begin{pgfscope}%
\pgfsetbuttcap%
\pgfsetroundjoin%
\definecolor{currentfill}{rgb}{0.000000,0.000000,0.000000}%
\pgfsetfillcolor{currentfill}%
\pgfsetlinewidth{0.803000pt}%
\definecolor{currentstroke}{rgb}{0.000000,0.000000,0.000000}%
\pgfsetstrokecolor{currentstroke}%
\pgfsetdash{}{0pt}%
\pgfsys@defobject{currentmarker}{\pgfqpoint{0.000000in}{-0.048611in}}{\pgfqpoint{0.000000in}{0.000000in}}{%
\pgfpathmoveto{\pgfqpoint{0.000000in}{0.000000in}}%
\pgfpathlineto{\pgfqpoint{0.000000in}{-0.048611in}}%
\pgfusepath{stroke,fill}%
}%
\begin{pgfscope}%
\pgfsys@transformshift{4.366667in}{0.880000in}%
\pgfsys@useobject{currentmarker}{}%
\end{pgfscope}%
\end{pgfscope}%
\begin{pgfscope}%
\definecolor{textcolor}{rgb}{0.000000,0.000000,0.000000}%
\pgfsetstrokecolor{textcolor}%
\pgfsetfillcolor{textcolor}%
\pgftext[x=4.366667in,y=0.782778in,,top]{\color{textcolor}\rmfamily\fontsize{10.000000}{12.000000}\selectfont \(\displaystyle {140}\)}%
\end{pgfscope}%
\begin{pgfscope}%
\pgfpathrectangle{\pgfqpoint{0.750000in}{0.880000in}}{\pgfqpoint{4.650000in}{3.080000in}}%
\pgfusepath{clip}%
\pgfsetrectcap%
\pgfsetroundjoin%
\pgfsetlinewidth{0.803000pt}%
\definecolor{currentstroke}{rgb}{0.690196,0.690196,0.690196}%
\pgfsetstrokecolor{currentstroke}%
\pgfsetdash{}{0pt}%
\pgfpathmoveto{\pgfqpoint{4.883333in}{0.880000in}}%
\pgfpathlineto{\pgfqpoint{4.883333in}{3.960000in}}%
\pgfusepath{stroke}%
\end{pgfscope}%
\begin{pgfscope}%
\pgfsetbuttcap%
\pgfsetroundjoin%
\definecolor{currentfill}{rgb}{0.000000,0.000000,0.000000}%
\pgfsetfillcolor{currentfill}%
\pgfsetlinewidth{0.803000pt}%
\definecolor{currentstroke}{rgb}{0.000000,0.000000,0.000000}%
\pgfsetstrokecolor{currentstroke}%
\pgfsetdash{}{0pt}%
\pgfsys@defobject{currentmarker}{\pgfqpoint{0.000000in}{-0.048611in}}{\pgfqpoint{0.000000in}{0.000000in}}{%
\pgfpathmoveto{\pgfqpoint{0.000000in}{0.000000in}}%
\pgfpathlineto{\pgfqpoint{0.000000in}{-0.048611in}}%
\pgfusepath{stroke,fill}%
}%
\begin{pgfscope}%
\pgfsys@transformshift{4.883333in}{0.880000in}%
\pgfsys@useobject{currentmarker}{}%
\end{pgfscope}%
\end{pgfscope}%
\begin{pgfscope}%
\definecolor{textcolor}{rgb}{0.000000,0.000000,0.000000}%
\pgfsetstrokecolor{textcolor}%
\pgfsetfillcolor{textcolor}%
\pgftext[x=4.883333in,y=0.782778in,,top]{\color{textcolor}\rmfamily\fontsize{10.000000}{12.000000}\selectfont \(\displaystyle {160}\)}%
\end{pgfscope}%
\begin{pgfscope}%
\pgfpathrectangle{\pgfqpoint{0.750000in}{0.880000in}}{\pgfqpoint{4.650000in}{3.080000in}}%
\pgfusepath{clip}%
\pgfsetrectcap%
\pgfsetroundjoin%
\pgfsetlinewidth{0.803000pt}%
\definecolor{currentstroke}{rgb}{0.690196,0.690196,0.690196}%
\pgfsetstrokecolor{currentstroke}%
\pgfsetdash{}{0pt}%
\pgfpathmoveto{\pgfqpoint{5.400000in}{0.880000in}}%
\pgfpathlineto{\pgfqpoint{5.400000in}{3.960000in}}%
\pgfusepath{stroke}%
\end{pgfscope}%
\begin{pgfscope}%
\pgfsetbuttcap%
\pgfsetroundjoin%
\definecolor{currentfill}{rgb}{0.000000,0.000000,0.000000}%
\pgfsetfillcolor{currentfill}%
\pgfsetlinewidth{0.803000pt}%
\definecolor{currentstroke}{rgb}{0.000000,0.000000,0.000000}%
\pgfsetstrokecolor{currentstroke}%
\pgfsetdash{}{0pt}%
\pgfsys@defobject{currentmarker}{\pgfqpoint{0.000000in}{-0.048611in}}{\pgfqpoint{0.000000in}{0.000000in}}{%
\pgfpathmoveto{\pgfqpoint{0.000000in}{0.000000in}}%
\pgfpathlineto{\pgfqpoint{0.000000in}{-0.048611in}}%
\pgfusepath{stroke,fill}%
}%
\begin{pgfscope}%
\pgfsys@transformshift{5.400000in}{0.880000in}%
\pgfsys@useobject{currentmarker}{}%
\end{pgfscope}%
\end{pgfscope}%
\begin{pgfscope}%
\definecolor{textcolor}{rgb}{0.000000,0.000000,0.000000}%
\pgfsetstrokecolor{textcolor}%
\pgfsetfillcolor{textcolor}%
\pgftext[x=5.400000in,y=0.782778in,,top]{\color{textcolor}\rmfamily\fontsize{10.000000}{12.000000}\selectfont \(\displaystyle {180}\)}%
\end{pgfscope}%
\begin{pgfscope}%
\definecolor{textcolor}{rgb}{0.000000,0.000000,0.000000}%
\pgfsetstrokecolor{textcolor}%
\pgfsetfillcolor{textcolor}%
\pgftext[x=3.075000in,y=0.594776in,,top]{\color{textcolor}\rmfamily\fontsize{10.000000}{12.000000}\selectfont power angle \(\displaystyle \delta\) in deg}%
\end{pgfscope}%
\begin{pgfscope}%
\pgfpathrectangle{\pgfqpoint{0.750000in}{0.880000in}}{\pgfqpoint{4.650000in}{3.080000in}}%
\pgfusepath{clip}%
\pgfsetrectcap%
\pgfsetroundjoin%
\pgfsetlinewidth{0.803000pt}%
\definecolor{currentstroke}{rgb}{0.690196,0.690196,0.690196}%
\pgfsetstrokecolor{currentstroke}%
\pgfsetdash{}{0pt}%
\pgfpathmoveto{\pgfqpoint{0.750000in}{3.823047in}}%
\pgfpathlineto{\pgfqpoint{5.400000in}{3.823047in}}%
\pgfusepath{stroke}%
\end{pgfscope}%
\begin{pgfscope}%
\pgfsetbuttcap%
\pgfsetroundjoin%
\definecolor{currentfill}{rgb}{0.000000,0.000000,0.000000}%
\pgfsetfillcolor{currentfill}%
\pgfsetlinewidth{0.803000pt}%
\definecolor{currentstroke}{rgb}{0.000000,0.000000,0.000000}%
\pgfsetstrokecolor{currentstroke}%
\pgfsetdash{}{0pt}%
\pgfsys@defobject{currentmarker}{\pgfqpoint{-0.048611in}{0.000000in}}{\pgfqpoint{-0.000000in}{0.000000in}}{%
\pgfpathmoveto{\pgfqpoint{-0.000000in}{0.000000in}}%
\pgfpathlineto{\pgfqpoint{-0.048611in}{0.000000in}}%
\pgfusepath{stroke,fill}%
}%
\begin{pgfscope}%
\pgfsys@transformshift{0.750000in}{3.823047in}%
\pgfsys@useobject{currentmarker}{}%
\end{pgfscope}%
\end{pgfscope}%
\begin{pgfscope}%
\definecolor{textcolor}{rgb}{0.000000,0.000000,0.000000}%
\pgfsetstrokecolor{textcolor}%
\pgfsetfillcolor{textcolor}%
\pgftext[x=0.405863in, y=3.771947in, left, base]{\color{textcolor}\rmfamily\fontsize{10.000000}{12.000000}\selectfont \(\displaystyle {0.00}\)}%
\end{pgfscope}%
\begin{pgfscope}%
\pgfpathrectangle{\pgfqpoint{0.750000in}{0.880000in}}{\pgfqpoint{4.650000in}{3.080000in}}%
\pgfusepath{clip}%
\pgfsetrectcap%
\pgfsetroundjoin%
\pgfsetlinewidth{0.803000pt}%
\definecolor{currentstroke}{rgb}{0.690196,0.690196,0.690196}%
\pgfsetstrokecolor{currentstroke}%
\pgfsetdash{}{0pt}%
\pgfpathmoveto{\pgfqpoint{0.750000in}{3.480665in}}%
\pgfpathlineto{\pgfqpoint{5.400000in}{3.480665in}}%
\pgfusepath{stroke}%
\end{pgfscope}%
\begin{pgfscope}%
\pgfsetbuttcap%
\pgfsetroundjoin%
\definecolor{currentfill}{rgb}{0.000000,0.000000,0.000000}%
\pgfsetfillcolor{currentfill}%
\pgfsetlinewidth{0.803000pt}%
\definecolor{currentstroke}{rgb}{0.000000,0.000000,0.000000}%
\pgfsetstrokecolor{currentstroke}%
\pgfsetdash{}{0pt}%
\pgfsys@defobject{currentmarker}{\pgfqpoint{-0.048611in}{0.000000in}}{\pgfqpoint{-0.000000in}{0.000000in}}{%
\pgfpathmoveto{\pgfqpoint{-0.000000in}{0.000000in}}%
\pgfpathlineto{\pgfqpoint{-0.048611in}{0.000000in}}%
\pgfusepath{stroke,fill}%
}%
\begin{pgfscope}%
\pgfsys@transformshift{0.750000in}{3.480665in}%
\pgfsys@useobject{currentmarker}{}%
\end{pgfscope}%
\end{pgfscope}%
\begin{pgfscope}%
\definecolor{textcolor}{rgb}{0.000000,0.000000,0.000000}%
\pgfsetstrokecolor{textcolor}%
\pgfsetfillcolor{textcolor}%
\pgftext[x=0.405863in, y=3.429565in, left, base]{\color{textcolor}\rmfamily\fontsize{10.000000}{12.000000}\selectfont \(\displaystyle {0.25}\)}%
\end{pgfscope}%
\begin{pgfscope}%
\pgfpathrectangle{\pgfqpoint{0.750000in}{0.880000in}}{\pgfqpoint{4.650000in}{3.080000in}}%
\pgfusepath{clip}%
\pgfsetrectcap%
\pgfsetroundjoin%
\pgfsetlinewidth{0.803000pt}%
\definecolor{currentstroke}{rgb}{0.690196,0.690196,0.690196}%
\pgfsetstrokecolor{currentstroke}%
\pgfsetdash{}{0pt}%
\pgfpathmoveto{\pgfqpoint{0.750000in}{3.138283in}}%
\pgfpathlineto{\pgfqpoint{5.400000in}{3.138283in}}%
\pgfusepath{stroke}%
\end{pgfscope}%
\begin{pgfscope}%
\pgfsetbuttcap%
\pgfsetroundjoin%
\definecolor{currentfill}{rgb}{0.000000,0.000000,0.000000}%
\pgfsetfillcolor{currentfill}%
\pgfsetlinewidth{0.803000pt}%
\definecolor{currentstroke}{rgb}{0.000000,0.000000,0.000000}%
\pgfsetstrokecolor{currentstroke}%
\pgfsetdash{}{0pt}%
\pgfsys@defobject{currentmarker}{\pgfqpoint{-0.048611in}{0.000000in}}{\pgfqpoint{-0.000000in}{0.000000in}}{%
\pgfpathmoveto{\pgfqpoint{-0.000000in}{0.000000in}}%
\pgfpathlineto{\pgfqpoint{-0.048611in}{0.000000in}}%
\pgfusepath{stroke,fill}%
}%
\begin{pgfscope}%
\pgfsys@transformshift{0.750000in}{3.138283in}%
\pgfsys@useobject{currentmarker}{}%
\end{pgfscope}%
\end{pgfscope}%
\begin{pgfscope}%
\definecolor{textcolor}{rgb}{0.000000,0.000000,0.000000}%
\pgfsetstrokecolor{textcolor}%
\pgfsetfillcolor{textcolor}%
\pgftext[x=0.405863in, y=3.087183in, left, base]{\color{textcolor}\rmfamily\fontsize{10.000000}{12.000000}\selectfont \(\displaystyle {0.50}\)}%
\end{pgfscope}%
\begin{pgfscope}%
\pgfpathrectangle{\pgfqpoint{0.750000in}{0.880000in}}{\pgfqpoint{4.650000in}{3.080000in}}%
\pgfusepath{clip}%
\pgfsetrectcap%
\pgfsetroundjoin%
\pgfsetlinewidth{0.803000pt}%
\definecolor{currentstroke}{rgb}{0.690196,0.690196,0.690196}%
\pgfsetstrokecolor{currentstroke}%
\pgfsetdash{}{0pt}%
\pgfpathmoveto{\pgfqpoint{0.750000in}{2.795901in}}%
\pgfpathlineto{\pgfqpoint{5.400000in}{2.795901in}}%
\pgfusepath{stroke}%
\end{pgfscope}%
\begin{pgfscope}%
\pgfsetbuttcap%
\pgfsetroundjoin%
\definecolor{currentfill}{rgb}{0.000000,0.000000,0.000000}%
\pgfsetfillcolor{currentfill}%
\pgfsetlinewidth{0.803000pt}%
\definecolor{currentstroke}{rgb}{0.000000,0.000000,0.000000}%
\pgfsetstrokecolor{currentstroke}%
\pgfsetdash{}{0pt}%
\pgfsys@defobject{currentmarker}{\pgfqpoint{-0.048611in}{0.000000in}}{\pgfqpoint{-0.000000in}{0.000000in}}{%
\pgfpathmoveto{\pgfqpoint{-0.000000in}{0.000000in}}%
\pgfpathlineto{\pgfqpoint{-0.048611in}{0.000000in}}%
\pgfusepath{stroke,fill}%
}%
\begin{pgfscope}%
\pgfsys@transformshift{0.750000in}{2.795901in}%
\pgfsys@useobject{currentmarker}{}%
\end{pgfscope}%
\end{pgfscope}%
\begin{pgfscope}%
\definecolor{textcolor}{rgb}{0.000000,0.000000,0.000000}%
\pgfsetstrokecolor{textcolor}%
\pgfsetfillcolor{textcolor}%
\pgftext[x=0.405863in, y=2.744801in, left, base]{\color{textcolor}\rmfamily\fontsize{10.000000}{12.000000}\selectfont \(\displaystyle {0.75}\)}%
\end{pgfscope}%
\begin{pgfscope}%
\pgfpathrectangle{\pgfqpoint{0.750000in}{0.880000in}}{\pgfqpoint{4.650000in}{3.080000in}}%
\pgfusepath{clip}%
\pgfsetrectcap%
\pgfsetroundjoin%
\pgfsetlinewidth{0.803000pt}%
\definecolor{currentstroke}{rgb}{0.690196,0.690196,0.690196}%
\pgfsetstrokecolor{currentstroke}%
\pgfsetdash{}{0pt}%
\pgfpathmoveto{\pgfqpoint{0.750000in}{2.453519in}}%
\pgfpathlineto{\pgfqpoint{5.400000in}{2.453519in}}%
\pgfusepath{stroke}%
\end{pgfscope}%
\begin{pgfscope}%
\pgfsetbuttcap%
\pgfsetroundjoin%
\definecolor{currentfill}{rgb}{0.000000,0.000000,0.000000}%
\pgfsetfillcolor{currentfill}%
\pgfsetlinewidth{0.803000pt}%
\definecolor{currentstroke}{rgb}{0.000000,0.000000,0.000000}%
\pgfsetstrokecolor{currentstroke}%
\pgfsetdash{}{0pt}%
\pgfsys@defobject{currentmarker}{\pgfqpoint{-0.048611in}{0.000000in}}{\pgfqpoint{-0.000000in}{0.000000in}}{%
\pgfpathmoveto{\pgfqpoint{-0.000000in}{0.000000in}}%
\pgfpathlineto{\pgfqpoint{-0.048611in}{0.000000in}}%
\pgfusepath{stroke,fill}%
}%
\begin{pgfscope}%
\pgfsys@transformshift{0.750000in}{2.453519in}%
\pgfsys@useobject{currentmarker}{}%
\end{pgfscope}%
\end{pgfscope}%
\begin{pgfscope}%
\definecolor{textcolor}{rgb}{0.000000,0.000000,0.000000}%
\pgfsetstrokecolor{textcolor}%
\pgfsetfillcolor{textcolor}%
\pgftext[x=0.405863in, y=2.402419in, left, base]{\color{textcolor}\rmfamily\fontsize{10.000000}{12.000000}\selectfont \(\displaystyle {1.00}\)}%
\end{pgfscope}%
\begin{pgfscope}%
\pgfpathrectangle{\pgfqpoint{0.750000in}{0.880000in}}{\pgfqpoint{4.650000in}{3.080000in}}%
\pgfusepath{clip}%
\pgfsetrectcap%
\pgfsetroundjoin%
\pgfsetlinewidth{0.803000pt}%
\definecolor{currentstroke}{rgb}{0.690196,0.690196,0.690196}%
\pgfsetstrokecolor{currentstroke}%
\pgfsetdash{}{0pt}%
\pgfpathmoveto{\pgfqpoint{0.750000in}{2.111137in}}%
\pgfpathlineto{\pgfqpoint{5.400000in}{2.111137in}}%
\pgfusepath{stroke}%
\end{pgfscope}%
\begin{pgfscope}%
\pgfsetbuttcap%
\pgfsetroundjoin%
\definecolor{currentfill}{rgb}{0.000000,0.000000,0.000000}%
\pgfsetfillcolor{currentfill}%
\pgfsetlinewidth{0.803000pt}%
\definecolor{currentstroke}{rgb}{0.000000,0.000000,0.000000}%
\pgfsetstrokecolor{currentstroke}%
\pgfsetdash{}{0pt}%
\pgfsys@defobject{currentmarker}{\pgfqpoint{-0.048611in}{0.000000in}}{\pgfqpoint{-0.000000in}{0.000000in}}{%
\pgfpathmoveto{\pgfqpoint{-0.000000in}{0.000000in}}%
\pgfpathlineto{\pgfqpoint{-0.048611in}{0.000000in}}%
\pgfusepath{stroke,fill}%
}%
\begin{pgfscope}%
\pgfsys@transformshift{0.750000in}{2.111137in}%
\pgfsys@useobject{currentmarker}{}%
\end{pgfscope}%
\end{pgfscope}%
\begin{pgfscope}%
\definecolor{textcolor}{rgb}{0.000000,0.000000,0.000000}%
\pgfsetstrokecolor{textcolor}%
\pgfsetfillcolor{textcolor}%
\pgftext[x=0.405863in, y=2.060037in, left, base]{\color{textcolor}\rmfamily\fontsize{10.000000}{12.000000}\selectfont \(\displaystyle {1.25}\)}%
\end{pgfscope}%
\begin{pgfscope}%
\pgfpathrectangle{\pgfqpoint{0.750000in}{0.880000in}}{\pgfqpoint{4.650000in}{3.080000in}}%
\pgfusepath{clip}%
\pgfsetrectcap%
\pgfsetroundjoin%
\pgfsetlinewidth{0.803000pt}%
\definecolor{currentstroke}{rgb}{0.690196,0.690196,0.690196}%
\pgfsetstrokecolor{currentstroke}%
\pgfsetdash{}{0pt}%
\pgfpathmoveto{\pgfqpoint{0.750000in}{1.768755in}}%
\pgfpathlineto{\pgfqpoint{5.400000in}{1.768755in}}%
\pgfusepath{stroke}%
\end{pgfscope}%
\begin{pgfscope}%
\pgfsetbuttcap%
\pgfsetroundjoin%
\definecolor{currentfill}{rgb}{0.000000,0.000000,0.000000}%
\pgfsetfillcolor{currentfill}%
\pgfsetlinewidth{0.803000pt}%
\definecolor{currentstroke}{rgb}{0.000000,0.000000,0.000000}%
\pgfsetstrokecolor{currentstroke}%
\pgfsetdash{}{0pt}%
\pgfsys@defobject{currentmarker}{\pgfqpoint{-0.048611in}{0.000000in}}{\pgfqpoint{-0.000000in}{0.000000in}}{%
\pgfpathmoveto{\pgfqpoint{-0.000000in}{0.000000in}}%
\pgfpathlineto{\pgfqpoint{-0.048611in}{0.000000in}}%
\pgfusepath{stroke,fill}%
}%
\begin{pgfscope}%
\pgfsys@transformshift{0.750000in}{1.768755in}%
\pgfsys@useobject{currentmarker}{}%
\end{pgfscope}%
\end{pgfscope}%
\begin{pgfscope}%
\definecolor{textcolor}{rgb}{0.000000,0.000000,0.000000}%
\pgfsetstrokecolor{textcolor}%
\pgfsetfillcolor{textcolor}%
\pgftext[x=0.405863in, y=1.717655in, left, base]{\color{textcolor}\rmfamily\fontsize{10.000000}{12.000000}\selectfont \(\displaystyle {1.50}\)}%
\end{pgfscope}%
\begin{pgfscope}%
\pgfpathrectangle{\pgfqpoint{0.750000in}{0.880000in}}{\pgfqpoint{4.650000in}{3.080000in}}%
\pgfusepath{clip}%
\pgfsetrectcap%
\pgfsetroundjoin%
\pgfsetlinewidth{0.803000pt}%
\definecolor{currentstroke}{rgb}{0.690196,0.690196,0.690196}%
\pgfsetstrokecolor{currentstroke}%
\pgfsetdash{}{0pt}%
\pgfpathmoveto{\pgfqpoint{0.750000in}{1.426373in}}%
\pgfpathlineto{\pgfqpoint{5.400000in}{1.426373in}}%
\pgfusepath{stroke}%
\end{pgfscope}%
\begin{pgfscope}%
\pgfsetbuttcap%
\pgfsetroundjoin%
\definecolor{currentfill}{rgb}{0.000000,0.000000,0.000000}%
\pgfsetfillcolor{currentfill}%
\pgfsetlinewidth{0.803000pt}%
\definecolor{currentstroke}{rgb}{0.000000,0.000000,0.000000}%
\pgfsetstrokecolor{currentstroke}%
\pgfsetdash{}{0pt}%
\pgfsys@defobject{currentmarker}{\pgfqpoint{-0.048611in}{0.000000in}}{\pgfqpoint{-0.000000in}{0.000000in}}{%
\pgfpathmoveto{\pgfqpoint{-0.000000in}{0.000000in}}%
\pgfpathlineto{\pgfqpoint{-0.048611in}{0.000000in}}%
\pgfusepath{stroke,fill}%
}%
\begin{pgfscope}%
\pgfsys@transformshift{0.750000in}{1.426373in}%
\pgfsys@useobject{currentmarker}{}%
\end{pgfscope}%
\end{pgfscope}%
\begin{pgfscope}%
\definecolor{textcolor}{rgb}{0.000000,0.000000,0.000000}%
\pgfsetstrokecolor{textcolor}%
\pgfsetfillcolor{textcolor}%
\pgftext[x=0.405863in, y=1.375273in, left, base]{\color{textcolor}\rmfamily\fontsize{10.000000}{12.000000}\selectfont \(\displaystyle {1.75}\)}%
\end{pgfscope}%
\begin{pgfscope}%
\pgfpathrectangle{\pgfqpoint{0.750000in}{0.880000in}}{\pgfqpoint{4.650000in}{3.080000in}}%
\pgfusepath{clip}%
\pgfsetrectcap%
\pgfsetroundjoin%
\pgfsetlinewidth{0.803000pt}%
\definecolor{currentstroke}{rgb}{0.690196,0.690196,0.690196}%
\pgfsetstrokecolor{currentstroke}%
\pgfsetdash{}{0pt}%
\pgfpathmoveto{\pgfqpoint{0.750000in}{1.083991in}}%
\pgfpathlineto{\pgfqpoint{5.400000in}{1.083991in}}%
\pgfusepath{stroke}%
\end{pgfscope}%
\begin{pgfscope}%
\pgfsetbuttcap%
\pgfsetroundjoin%
\definecolor{currentfill}{rgb}{0.000000,0.000000,0.000000}%
\pgfsetfillcolor{currentfill}%
\pgfsetlinewidth{0.803000pt}%
\definecolor{currentstroke}{rgb}{0.000000,0.000000,0.000000}%
\pgfsetstrokecolor{currentstroke}%
\pgfsetdash{}{0pt}%
\pgfsys@defobject{currentmarker}{\pgfqpoint{-0.048611in}{0.000000in}}{\pgfqpoint{-0.000000in}{0.000000in}}{%
\pgfpathmoveto{\pgfqpoint{-0.000000in}{0.000000in}}%
\pgfpathlineto{\pgfqpoint{-0.048611in}{0.000000in}}%
\pgfusepath{stroke,fill}%
}%
\begin{pgfscope}%
\pgfsys@transformshift{0.750000in}{1.083991in}%
\pgfsys@useobject{currentmarker}{}%
\end{pgfscope}%
\end{pgfscope}%
\begin{pgfscope}%
\definecolor{textcolor}{rgb}{0.000000,0.000000,0.000000}%
\pgfsetstrokecolor{textcolor}%
\pgfsetfillcolor{textcolor}%
\pgftext[x=0.405863in, y=1.032891in, left, base]{\color{textcolor}\rmfamily\fontsize{10.000000}{12.000000}\selectfont \(\displaystyle {2.00}\)}%
\end{pgfscope}%
\begin{pgfscope}%
\definecolor{textcolor}{rgb}{0.000000,0.000000,0.000000}%
\pgfsetstrokecolor{textcolor}%
\pgfsetfillcolor{textcolor}%
\pgftext[x=0.350308in,y=2.420000in,,bottom,rotate=90.000000]{\color{textcolor}\rmfamily\fontsize{10.000000}{12.000000}\selectfont time in s}%
\end{pgfscope}%
\begin{pgfscope}%
\pgfpathrectangle{\pgfqpoint{0.750000in}{0.880000in}}{\pgfqpoint{4.650000in}{3.080000in}}%
\pgfusepath{clip}%
\pgfsetrectcap%
\pgfsetroundjoin%
\pgfsetlinewidth{1.505625pt}%
\definecolor{currentstroke}{rgb}{0.121569,0.466667,0.705882}%
\pgfsetstrokecolor{currentstroke}%
\pgfsetdash{}{0pt}%
\pgfpathmoveto{\pgfqpoint{2.005414in}{3.970000in}}%
\pgfpathlineto{\pgfqpoint{2.006519in}{3.809352in}}%
\pgfpathlineto{\pgfqpoint{2.009623in}{3.795657in}}%
\pgfpathlineto{\pgfqpoint{2.015412in}{3.780592in}}%
\pgfpathlineto{\pgfqpoint{2.023668in}{3.765527in}}%
\pgfpathlineto{\pgfqpoint{2.034361in}{3.750462in}}%
\pgfpathlineto{\pgfqpoint{2.048764in}{3.734028in}}%
\pgfpathlineto{\pgfqpoint{2.067528in}{3.716224in}}%
\pgfpathlineto{\pgfqpoint{2.089498in}{3.698420in}}%
\pgfpathlineto{\pgfqpoint{2.116634in}{3.679247in}}%
\pgfpathlineto{\pgfqpoint{2.149563in}{3.658704in}}%
\pgfpathlineto{\pgfqpoint{2.188887in}{3.636791in}}%
\pgfpathlineto{\pgfqpoint{2.235164in}{3.613509in}}%
\pgfpathlineto{\pgfqpoint{2.292014in}{3.587488in}}%
\pgfpathlineto{\pgfqpoint{2.357254in}{3.560098in}}%
\pgfpathlineto{\pgfqpoint{2.434824in}{3.529968in}}%
\pgfpathlineto{\pgfqpoint{2.525639in}{3.497100in}}%
\pgfpathlineto{\pgfqpoint{2.634561in}{3.460122in}}%
\pgfpathlineto{\pgfqpoint{2.762787in}{3.419036in}}%
\pgfpathlineto{\pgfqpoint{2.911296in}{3.373842in}}%
\pgfpathlineto{\pgfqpoint{3.081087in}{3.324539in}}%
\pgfpathlineto{\pgfqpoint{3.405465in}{3.231411in}}%
\pgfpathlineto{\pgfqpoint{3.483216in}{3.205390in}}%
\pgfpathlineto{\pgfqpoint{3.554302in}{3.179369in}}%
\pgfpathlineto{\pgfqpoint{3.619171in}{3.153348in}}%
\pgfpathlineto{\pgfqpoint{3.678291in}{3.127327in}}%
\pgfpathlineto{\pgfqpoint{3.732135in}{3.101306in}}%
\pgfpathlineto{\pgfqpoint{3.783632in}{3.073915in}}%
\pgfpathlineto{\pgfqpoint{3.830341in}{3.046525in}}%
\pgfpathlineto{\pgfqpoint{3.872783in}{3.019134in}}%
\pgfpathlineto{\pgfqpoint{3.913295in}{2.990374in}}%
\pgfpathlineto{\pgfqpoint{3.951871in}{2.960245in}}%
\pgfpathlineto{\pgfqpoint{3.988608in}{2.928745in}}%
\pgfpathlineto{\pgfqpoint{4.025109in}{2.894507in}}%
\pgfpathlineto{\pgfqpoint{4.062685in}{2.856160in}}%
\pgfpathlineto{\pgfqpoint{4.105056in}{2.809596in}}%
\pgfpathlineto{\pgfqpoint{4.252689in}{2.643884in}}%
\pgfpathlineto{\pgfqpoint{4.289179in}{2.608276in}}%
\pgfpathlineto{\pgfqpoint{4.324560in}{2.576777in}}%
\pgfpathlineto{\pgfqpoint{4.360096in}{2.548017in}}%
\pgfpathlineto{\pgfqpoint{4.397435in}{2.520626in}}%
\pgfpathlineto{\pgfqpoint{4.436666in}{2.494605in}}%
\pgfpathlineto{\pgfqpoint{4.477787in}{2.469954in}}%
\pgfpathlineto{\pgfqpoint{4.523369in}{2.445302in}}%
\pgfpathlineto{\pgfqpoint{4.571129in}{2.422020in}}%
\pgfpathlineto{\pgfqpoint{4.624127in}{2.398738in}}%
\pgfpathlineto{\pgfqpoint{4.679452in}{2.376826in}}%
\pgfpathlineto{\pgfqpoint{4.740766in}{2.354913in}}%
\pgfpathlineto{\pgfqpoint{4.808858in}{2.333001in}}%
\pgfpathlineto{\pgfqpoint{4.884622in}{2.311088in}}%
\pgfpathlineto{\pgfqpoint{4.963518in}{2.290545in}}%
\pgfpathlineto{\pgfqpoint{5.050996in}{2.270002in}}%
\pgfpathlineto{\pgfqpoint{5.148121in}{2.249460in}}%
\pgfpathlineto{\pgfqpoint{5.256098in}{2.228917in}}%
\pgfpathlineto{\pgfqpoint{5.376281in}{2.208374in}}%
\pgfpathlineto{\pgfqpoint{5.410000in}{2.202987in}}%
\pgfpathlineto{\pgfqpoint{5.410000in}{2.202987in}}%
\pgfusepath{stroke}%
\end{pgfscope}%
\begin{pgfscope}%
\pgfpathrectangle{\pgfqpoint{0.750000in}{0.880000in}}{\pgfqpoint{4.650000in}{3.080000in}}%
\pgfusepath{clip}%
\pgfsetbuttcap%
\pgfsetroundjoin%
\pgfsetlinewidth{1.505625pt}%
\definecolor{currentstroke}{rgb}{0.121569,0.466667,0.705882}%
\pgfsetstrokecolor{currentstroke}%
\pgfsetdash{{5.550000pt}{2.400000pt}}{0.000000pt}%
\pgfpathmoveto{\pgfqpoint{0.750000in}{3.283453in}}%
\pgfpathlineto{\pgfqpoint{5.400000in}{3.283453in}}%
\pgfusepath{stroke}%
\end{pgfscope}%
\begin{pgfscope}%
\pgfsetrectcap%
\pgfsetmiterjoin%
\pgfsetlinewidth{0.803000pt}%
\definecolor{currentstroke}{rgb}{0.000000,0.000000,0.000000}%
\pgfsetstrokecolor{currentstroke}%
\pgfsetdash{}{0pt}%
\pgfpathmoveto{\pgfqpoint{0.750000in}{0.880000in}}%
\pgfpathlineto{\pgfqpoint{0.750000in}{3.960000in}}%
\pgfusepath{stroke}%
\end{pgfscope}%
\begin{pgfscope}%
\pgfsetrectcap%
\pgfsetmiterjoin%
\pgfsetlinewidth{0.803000pt}%
\definecolor{currentstroke}{rgb}{0.000000,0.000000,0.000000}%
\pgfsetstrokecolor{currentstroke}%
\pgfsetdash{}{0pt}%
\pgfpathmoveto{\pgfqpoint{5.400000in}{0.880000in}}%
\pgfpathlineto{\pgfqpoint{5.400000in}{3.960000in}}%
\pgfusepath{stroke}%
\end{pgfscope}%
\begin{pgfscope}%
\pgfsetrectcap%
\pgfsetmiterjoin%
\pgfsetlinewidth{0.803000pt}%
\definecolor{currentstroke}{rgb}{0.000000,0.000000,0.000000}%
\pgfsetstrokecolor{currentstroke}%
\pgfsetdash{}{0pt}%
\pgfpathmoveto{\pgfqpoint{0.750000in}{0.880000in}}%
\pgfpathlineto{\pgfqpoint{5.400000in}{0.880000in}}%
\pgfusepath{stroke}%
\end{pgfscope}%
\begin{pgfscope}%
\pgfsetrectcap%
\pgfsetmiterjoin%
\pgfsetlinewidth{0.803000pt}%
\definecolor{currentstroke}{rgb}{0.000000,0.000000,0.000000}%
\pgfsetstrokecolor{currentstroke}%
\pgfsetdash{}{0pt}%
\pgfpathmoveto{\pgfqpoint{0.750000in}{3.960000in}}%
\pgfpathlineto{\pgfqpoint{5.400000in}{3.960000in}}%
\pgfusepath{stroke}%
\end{pgfscope}%
\begin{pgfscope}%
\pgfsetbuttcap%
\pgfsetmiterjoin%
\definecolor{currentfill}{rgb}{1.000000,1.000000,1.000000}%
\pgfsetfillcolor{currentfill}%
\pgfsetfillopacity{0.800000}%
\pgfsetlinewidth{1.003750pt}%
\definecolor{currentstroke}{rgb}{0.800000,0.800000,0.800000}%
\pgfsetstrokecolor{currentstroke}%
\pgfsetstrokeopacity{0.800000}%
\pgfsetdash{}{0pt}%
\pgfpathmoveto{\pgfqpoint{3.899064in}{3.443955in}}%
\pgfpathlineto{\pgfqpoint{5.302778in}{3.443955in}}%
\pgfpathquadraticcurveto{\pgfqpoint{5.330556in}{3.443955in}}{\pgfqpoint{5.330556in}{3.471733in}}%
\pgfpathlineto{\pgfqpoint{5.330556in}{3.862778in}}%
\pgfpathquadraticcurveto{\pgfqpoint{5.330556in}{3.890556in}}{\pgfqpoint{5.302778in}{3.890556in}}%
\pgfpathlineto{\pgfqpoint{3.899064in}{3.890556in}}%
\pgfpathquadraticcurveto{\pgfqpoint{3.871286in}{3.890556in}}{\pgfqpoint{3.871286in}{3.862778in}}%
\pgfpathlineto{\pgfqpoint{3.871286in}{3.471733in}}%
\pgfpathquadraticcurveto{\pgfqpoint{3.871286in}{3.443955in}}{\pgfqpoint{3.899064in}{3.443955in}}%
\pgfpathlineto{\pgfqpoint{3.899064in}{3.443955in}}%
\pgfpathclose%
\pgfusepath{stroke,fill}%
\end{pgfscope}%
\begin{pgfscope}%
\pgfsetrectcap%
\pgfsetroundjoin%
\pgfsetlinewidth{1.505625pt}%
\definecolor{currentstroke}{rgb}{0.121569,0.466667,0.705882}%
\pgfsetstrokecolor{currentstroke}%
\pgfsetdash{}{0pt}%
\pgfpathmoveto{\pgfqpoint{3.926842in}{3.781411in}}%
\pgfpathlineto{\pgfqpoint{4.065731in}{3.781411in}}%
\pgfpathlineto{\pgfqpoint{4.204620in}{3.781411in}}%
\pgfusepath{stroke}%
\end{pgfscope}%
\begin{pgfscope}%
\definecolor{textcolor}{rgb}{0.000000,0.000000,0.000000}%
\pgfsetstrokecolor{textcolor}%
\pgfsetfillcolor{textcolor}%
\pgftext[x=4.315731in,y=3.732800in,left,base]{\color{textcolor}\rmfamily\fontsize{10.000000}{12.000000}\selectfont delta}%
\end{pgfscope}%
\begin{pgfscope}%
\pgfsetbuttcap%
\pgfsetroundjoin%
\pgfsetlinewidth{1.505625pt}%
\definecolor{currentstroke}{rgb}{0.121569,0.466667,0.705882}%
\pgfsetstrokecolor{currentstroke}%
\pgfsetdash{{5.550000pt}{2.400000pt}}{0.000000pt}%
\pgfpathmoveto{\pgfqpoint{3.926842in}{3.578368in}}%
\pgfpathlineto{\pgfqpoint{4.065731in}{3.578368in}}%
\pgfpathlineto{\pgfqpoint{4.204620in}{3.578368in}}%
\pgfusepath{stroke}%
\end{pgfscope}%
\begin{pgfscope}%
\definecolor{textcolor}{rgb}{0.000000,0.000000,0.000000}%
\pgfsetstrokecolor{textcolor}%
\pgfsetfillcolor{textcolor}%
\pgftext[x=4.315731in,y=3.529757in,left,base]{\color{textcolor}\rmfamily\fontsize{10.000000}{12.000000}\selectfont clearing of fault}%
\end{pgfscope}%
\begin{pgfscope}%
\definecolor{textcolor}{rgb}{0.000000,0.000000,0.000000}%
\pgfsetstrokecolor{textcolor}%
\pgfsetfillcolor{textcolor}%
\pgftext[x=3.000000in,y=7.840000in,,top]{\color{textcolor}\rmfamily\fontsize{12.000000}{14.400000}\selectfont Unstable scenario - fault 2}%
\end{pgfscope}%
\end{pgfpicture}%
\makeatother%
\endgroup%


%% Creator: Matplotlib, PGF backend
%%
%% To include the figure in your LaTeX document, write
%%   \input{<filename>.pgf}
%%
%% Make sure the required packages are loaded in your preamble
%%   \usepackage{pgf}
%%
%% Also ensure that all the required font packages are loaded; for instance,
%% the lmodern package is sometimes necessary when using math font.
%%   \usepackage{lmodern}
%%
%% Figures using additional raster images can only be included by \input if
%% they are in the same directory as the main LaTeX file. For loading figures
%% from other directories you can use the `import` package
%%   \usepackage{import}
%%
%% and then include the figures with
%%   \import{<path to file>}{<filename>.pgf}
%%
%% Matplotlib used the following preamble
%%   
%%   \usepackage{fontspec}
%%   \setmainfont{Charter.ttc}[Path=\detokenize{/System/Library/Fonts/Supplemental/}]
%%   \setsansfont{DejaVuSans.ttf}[Path=\detokenize{/opt/homebrew/lib/python3.10/site-packages/matplotlib/mpl-data/fonts/ttf/}]
%%   \setmonofont{DejaVuSansMono.ttf}[Path=\detokenize{/opt/homebrew/lib/python3.10/site-packages/matplotlib/mpl-data/fonts/ttf/}]
%%   \makeatletter\@ifpackageloaded{underscore}{}{\usepackage[strings]{underscore}}\makeatother
%%
\begingroup%
\makeatletter%
\begin{pgfpicture}%
\pgfpathrectangle{\pgfpointorigin}{\pgfqpoint{6.400000in}{4.800000in}}%
\pgfusepath{use as bounding box, clip}%
\begin{pgfscope}%
\pgfsetbuttcap%
\pgfsetmiterjoin%
\definecolor{currentfill}{rgb}{1.000000,1.000000,1.000000}%
\pgfsetfillcolor{currentfill}%
\pgfsetlinewidth{0.000000pt}%
\definecolor{currentstroke}{rgb}{1.000000,1.000000,1.000000}%
\pgfsetstrokecolor{currentstroke}%
\pgfsetdash{}{0pt}%
\pgfpathmoveto{\pgfqpoint{0.000000in}{0.000000in}}%
\pgfpathlineto{\pgfqpoint{6.400000in}{0.000000in}}%
\pgfpathlineto{\pgfqpoint{6.400000in}{4.800000in}}%
\pgfpathlineto{\pgfqpoint{0.000000in}{4.800000in}}%
\pgfpathlineto{\pgfqpoint{0.000000in}{0.000000in}}%
\pgfpathclose%
\pgfusepath{fill}%
\end{pgfscope}%
\begin{pgfscope}%
\pgfsetbuttcap%
\pgfsetmiterjoin%
\definecolor{currentfill}{rgb}{1.000000,1.000000,1.000000}%
\pgfsetfillcolor{currentfill}%
\pgfsetlinewidth{0.000000pt}%
\definecolor{currentstroke}{rgb}{0.000000,0.000000,0.000000}%
\pgfsetstrokecolor{currentstroke}%
\pgfsetstrokeopacity{0.000000}%
\pgfsetdash{}{0pt}%
\pgfpathmoveto{\pgfqpoint{0.800000in}{0.528000in}}%
\pgfpathlineto{\pgfqpoint{5.760000in}{0.528000in}}%
\pgfpathlineto{\pgfqpoint{5.760000in}{4.224000in}}%
\pgfpathlineto{\pgfqpoint{0.800000in}{4.224000in}}%
\pgfpathlineto{\pgfqpoint{0.800000in}{0.528000in}}%
\pgfpathclose%
\pgfusepath{fill}%
\end{pgfscope}%
\begin{pgfscope}%
\pgfpathrectangle{\pgfqpoint{0.800000in}{0.528000in}}{\pgfqpoint{4.960000in}{3.696000in}}%
\pgfusepath{clip}%
\pgfsetrectcap%
\pgfsetroundjoin%
\pgfsetlinewidth{0.803000pt}%
\definecolor{currentstroke}{rgb}{0.690196,0.690196,0.690196}%
\pgfsetstrokecolor{currentstroke}%
\pgfsetdash{}{0pt}%
\pgfpathmoveto{\pgfqpoint{1.043857in}{0.528000in}}%
\pgfpathlineto{\pgfqpoint{1.043857in}{4.224000in}}%
\pgfusepath{stroke}%
\end{pgfscope}%
\begin{pgfscope}%
\pgfsetbuttcap%
\pgfsetroundjoin%
\definecolor{currentfill}{rgb}{0.000000,0.000000,0.000000}%
\pgfsetfillcolor{currentfill}%
\pgfsetlinewidth{0.803000pt}%
\definecolor{currentstroke}{rgb}{0.000000,0.000000,0.000000}%
\pgfsetstrokecolor{currentstroke}%
\pgfsetdash{}{0pt}%
\pgfsys@defobject{currentmarker}{\pgfqpoint{0.000000in}{-0.048611in}}{\pgfqpoint{0.000000in}{0.000000in}}{%
\pgfpathmoveto{\pgfqpoint{0.000000in}{0.000000in}}%
\pgfpathlineto{\pgfqpoint{0.000000in}{-0.048611in}}%
\pgfusepath{stroke,fill}%
}%
\begin{pgfscope}%
\pgfsys@transformshift{1.043857in}{0.528000in}%
\pgfsys@useobject{currentmarker}{}%
\end{pgfscope}%
\end{pgfscope}%
\begin{pgfscope}%
\definecolor{textcolor}{rgb}{0.000000,0.000000,0.000000}%
\pgfsetstrokecolor{textcolor}%
\pgfsetfillcolor{textcolor}%
\pgftext[x=1.043857in,y=0.430778in,,top]{\color{textcolor}\rmfamily\fontsize{10.000000}{12.000000}\selectfont \(\displaystyle {\ensuremath{-}1.0}\)}%
\end{pgfscope}%
\begin{pgfscope}%
\pgfpathrectangle{\pgfqpoint{0.800000in}{0.528000in}}{\pgfqpoint{4.960000in}{3.696000in}}%
\pgfusepath{clip}%
\pgfsetrectcap%
\pgfsetroundjoin%
\pgfsetlinewidth{0.803000pt}%
\definecolor{currentstroke}{rgb}{0.690196,0.690196,0.690196}%
\pgfsetstrokecolor{currentstroke}%
\pgfsetdash{}{0pt}%
\pgfpathmoveto{\pgfqpoint{1.856985in}{0.528000in}}%
\pgfpathlineto{\pgfqpoint{1.856985in}{4.224000in}}%
\pgfusepath{stroke}%
\end{pgfscope}%
\begin{pgfscope}%
\pgfsetbuttcap%
\pgfsetroundjoin%
\definecolor{currentfill}{rgb}{0.000000,0.000000,0.000000}%
\pgfsetfillcolor{currentfill}%
\pgfsetlinewidth{0.803000pt}%
\definecolor{currentstroke}{rgb}{0.000000,0.000000,0.000000}%
\pgfsetstrokecolor{currentstroke}%
\pgfsetdash{}{0pt}%
\pgfsys@defobject{currentmarker}{\pgfqpoint{0.000000in}{-0.048611in}}{\pgfqpoint{0.000000in}{0.000000in}}{%
\pgfpathmoveto{\pgfqpoint{0.000000in}{0.000000in}}%
\pgfpathlineto{\pgfqpoint{0.000000in}{-0.048611in}}%
\pgfusepath{stroke,fill}%
}%
\begin{pgfscope}%
\pgfsys@transformshift{1.856985in}{0.528000in}%
\pgfsys@useobject{currentmarker}{}%
\end{pgfscope}%
\end{pgfscope}%
\begin{pgfscope}%
\definecolor{textcolor}{rgb}{0.000000,0.000000,0.000000}%
\pgfsetstrokecolor{textcolor}%
\pgfsetfillcolor{textcolor}%
\pgftext[x=1.856985in,y=0.430778in,,top]{\color{textcolor}\rmfamily\fontsize{10.000000}{12.000000}\selectfont \(\displaystyle {\ensuremath{-}0.5}\)}%
\end{pgfscope}%
\begin{pgfscope}%
\pgfpathrectangle{\pgfqpoint{0.800000in}{0.528000in}}{\pgfqpoint{4.960000in}{3.696000in}}%
\pgfusepath{clip}%
\pgfsetrectcap%
\pgfsetroundjoin%
\pgfsetlinewidth{0.803000pt}%
\definecolor{currentstroke}{rgb}{0.690196,0.690196,0.690196}%
\pgfsetstrokecolor{currentstroke}%
\pgfsetdash{}{0pt}%
\pgfpathmoveto{\pgfqpoint{2.670113in}{0.528000in}}%
\pgfpathlineto{\pgfqpoint{2.670113in}{4.224000in}}%
\pgfusepath{stroke}%
\end{pgfscope}%
\begin{pgfscope}%
\pgfsetbuttcap%
\pgfsetroundjoin%
\definecolor{currentfill}{rgb}{0.000000,0.000000,0.000000}%
\pgfsetfillcolor{currentfill}%
\pgfsetlinewidth{0.803000pt}%
\definecolor{currentstroke}{rgb}{0.000000,0.000000,0.000000}%
\pgfsetstrokecolor{currentstroke}%
\pgfsetdash{}{0pt}%
\pgfsys@defobject{currentmarker}{\pgfqpoint{0.000000in}{-0.048611in}}{\pgfqpoint{0.000000in}{0.000000in}}{%
\pgfpathmoveto{\pgfqpoint{0.000000in}{0.000000in}}%
\pgfpathlineto{\pgfqpoint{0.000000in}{-0.048611in}}%
\pgfusepath{stroke,fill}%
}%
\begin{pgfscope}%
\pgfsys@transformshift{2.670113in}{0.528000in}%
\pgfsys@useobject{currentmarker}{}%
\end{pgfscope}%
\end{pgfscope}%
\begin{pgfscope}%
\definecolor{textcolor}{rgb}{0.000000,0.000000,0.000000}%
\pgfsetstrokecolor{textcolor}%
\pgfsetfillcolor{textcolor}%
\pgftext[x=2.670113in,y=0.430778in,,top]{\color{textcolor}\rmfamily\fontsize{10.000000}{12.000000}\selectfont \(\displaystyle {0.0}\)}%
\end{pgfscope}%
\begin{pgfscope}%
\pgfpathrectangle{\pgfqpoint{0.800000in}{0.528000in}}{\pgfqpoint{4.960000in}{3.696000in}}%
\pgfusepath{clip}%
\pgfsetrectcap%
\pgfsetroundjoin%
\pgfsetlinewidth{0.803000pt}%
\definecolor{currentstroke}{rgb}{0.690196,0.690196,0.690196}%
\pgfsetstrokecolor{currentstroke}%
\pgfsetdash{}{0pt}%
\pgfpathmoveto{\pgfqpoint{3.483241in}{0.528000in}}%
\pgfpathlineto{\pgfqpoint{3.483241in}{4.224000in}}%
\pgfusepath{stroke}%
\end{pgfscope}%
\begin{pgfscope}%
\pgfsetbuttcap%
\pgfsetroundjoin%
\definecolor{currentfill}{rgb}{0.000000,0.000000,0.000000}%
\pgfsetfillcolor{currentfill}%
\pgfsetlinewidth{0.803000pt}%
\definecolor{currentstroke}{rgb}{0.000000,0.000000,0.000000}%
\pgfsetstrokecolor{currentstroke}%
\pgfsetdash{}{0pt}%
\pgfsys@defobject{currentmarker}{\pgfqpoint{0.000000in}{-0.048611in}}{\pgfqpoint{0.000000in}{0.000000in}}{%
\pgfpathmoveto{\pgfqpoint{0.000000in}{0.000000in}}%
\pgfpathlineto{\pgfqpoint{0.000000in}{-0.048611in}}%
\pgfusepath{stroke,fill}%
}%
\begin{pgfscope}%
\pgfsys@transformshift{3.483241in}{0.528000in}%
\pgfsys@useobject{currentmarker}{}%
\end{pgfscope}%
\end{pgfscope}%
\begin{pgfscope}%
\definecolor{textcolor}{rgb}{0.000000,0.000000,0.000000}%
\pgfsetstrokecolor{textcolor}%
\pgfsetfillcolor{textcolor}%
\pgftext[x=3.483241in,y=0.430778in,,top]{\color{textcolor}\rmfamily\fontsize{10.000000}{12.000000}\selectfont \(\displaystyle {0.5}\)}%
\end{pgfscope}%
\begin{pgfscope}%
\pgfpathrectangle{\pgfqpoint{0.800000in}{0.528000in}}{\pgfqpoint{4.960000in}{3.696000in}}%
\pgfusepath{clip}%
\pgfsetrectcap%
\pgfsetroundjoin%
\pgfsetlinewidth{0.803000pt}%
\definecolor{currentstroke}{rgb}{0.690196,0.690196,0.690196}%
\pgfsetstrokecolor{currentstroke}%
\pgfsetdash{}{0pt}%
\pgfpathmoveto{\pgfqpoint{4.296369in}{0.528000in}}%
\pgfpathlineto{\pgfqpoint{4.296369in}{4.224000in}}%
\pgfusepath{stroke}%
\end{pgfscope}%
\begin{pgfscope}%
\pgfsetbuttcap%
\pgfsetroundjoin%
\definecolor{currentfill}{rgb}{0.000000,0.000000,0.000000}%
\pgfsetfillcolor{currentfill}%
\pgfsetlinewidth{0.803000pt}%
\definecolor{currentstroke}{rgb}{0.000000,0.000000,0.000000}%
\pgfsetstrokecolor{currentstroke}%
\pgfsetdash{}{0pt}%
\pgfsys@defobject{currentmarker}{\pgfqpoint{0.000000in}{-0.048611in}}{\pgfqpoint{0.000000in}{0.000000in}}{%
\pgfpathmoveto{\pgfqpoint{0.000000in}{0.000000in}}%
\pgfpathlineto{\pgfqpoint{0.000000in}{-0.048611in}}%
\pgfusepath{stroke,fill}%
}%
\begin{pgfscope}%
\pgfsys@transformshift{4.296369in}{0.528000in}%
\pgfsys@useobject{currentmarker}{}%
\end{pgfscope}%
\end{pgfscope}%
\begin{pgfscope}%
\definecolor{textcolor}{rgb}{0.000000,0.000000,0.000000}%
\pgfsetstrokecolor{textcolor}%
\pgfsetfillcolor{textcolor}%
\pgftext[x=4.296369in,y=0.430778in,,top]{\color{textcolor}\rmfamily\fontsize{10.000000}{12.000000}\selectfont \(\displaystyle {1.0}\)}%
\end{pgfscope}%
\begin{pgfscope}%
\pgfpathrectangle{\pgfqpoint{0.800000in}{0.528000in}}{\pgfqpoint{4.960000in}{3.696000in}}%
\pgfusepath{clip}%
\pgfsetrectcap%
\pgfsetroundjoin%
\pgfsetlinewidth{0.803000pt}%
\definecolor{currentstroke}{rgb}{0.690196,0.690196,0.690196}%
\pgfsetstrokecolor{currentstroke}%
\pgfsetdash{}{0pt}%
\pgfpathmoveto{\pgfqpoint{5.109498in}{0.528000in}}%
\pgfpathlineto{\pgfqpoint{5.109498in}{4.224000in}}%
\pgfusepath{stroke}%
\end{pgfscope}%
\begin{pgfscope}%
\pgfsetbuttcap%
\pgfsetroundjoin%
\definecolor{currentfill}{rgb}{0.000000,0.000000,0.000000}%
\pgfsetfillcolor{currentfill}%
\pgfsetlinewidth{0.803000pt}%
\definecolor{currentstroke}{rgb}{0.000000,0.000000,0.000000}%
\pgfsetstrokecolor{currentstroke}%
\pgfsetdash{}{0pt}%
\pgfsys@defobject{currentmarker}{\pgfqpoint{0.000000in}{-0.048611in}}{\pgfqpoint{0.000000in}{0.000000in}}{%
\pgfpathmoveto{\pgfqpoint{0.000000in}{0.000000in}}%
\pgfpathlineto{\pgfqpoint{0.000000in}{-0.048611in}}%
\pgfusepath{stroke,fill}%
}%
\begin{pgfscope}%
\pgfsys@transformshift{5.109498in}{0.528000in}%
\pgfsys@useobject{currentmarker}{}%
\end{pgfscope}%
\end{pgfscope}%
\begin{pgfscope}%
\definecolor{textcolor}{rgb}{0.000000,0.000000,0.000000}%
\pgfsetstrokecolor{textcolor}%
\pgfsetfillcolor{textcolor}%
\pgftext[x=5.109498in,y=0.430778in,,top]{\color{textcolor}\rmfamily\fontsize{10.000000}{12.000000}\selectfont \(\displaystyle {1.5}\)}%
\end{pgfscope}%
\begin{pgfscope}%
\definecolor{textcolor}{rgb}{0.000000,0.000000,0.000000}%
\pgfsetstrokecolor{textcolor}%
\pgfsetfillcolor{textcolor}%
\pgftext[x=3.280000in,y=0.242776in,,top]{\color{textcolor}\rmfamily\fontsize{10.000000}{12.000000}\selectfont time in s}%
\end{pgfscope}%
\begin{pgfscope}%
\pgfpathrectangle{\pgfqpoint{0.800000in}{0.528000in}}{\pgfqpoint{4.960000in}{3.696000in}}%
\pgfusepath{clip}%
\pgfsetrectcap%
\pgfsetroundjoin%
\pgfsetlinewidth{0.803000pt}%
\definecolor{currentstroke}{rgb}{0.690196,0.690196,0.690196}%
\pgfsetstrokecolor{currentstroke}%
\pgfsetdash{}{0pt}%
\pgfpathmoveto{\pgfqpoint{0.800000in}{0.975975in}}%
\pgfpathlineto{\pgfqpoint{5.760000in}{0.975975in}}%
\pgfusepath{stroke}%
\end{pgfscope}%
\begin{pgfscope}%
\pgfsetbuttcap%
\pgfsetroundjoin%
\definecolor{currentfill}{rgb}{0.000000,0.000000,0.000000}%
\pgfsetfillcolor{currentfill}%
\pgfsetlinewidth{0.803000pt}%
\definecolor{currentstroke}{rgb}{0.000000,0.000000,0.000000}%
\pgfsetstrokecolor{currentstroke}%
\pgfsetdash{}{0pt}%
\pgfsys@defobject{currentmarker}{\pgfqpoint{-0.048611in}{0.000000in}}{\pgfqpoint{-0.000000in}{0.000000in}}{%
\pgfpathmoveto{\pgfqpoint{-0.000000in}{0.000000in}}%
\pgfpathlineto{\pgfqpoint{-0.048611in}{0.000000in}}%
\pgfusepath{stroke,fill}%
}%
\begin{pgfscope}%
\pgfsys@transformshift{0.800000in}{0.975975in}%
\pgfsys@useobject{currentmarker}{}%
\end{pgfscope}%
\end{pgfscope}%
\begin{pgfscope}%
\definecolor{textcolor}{rgb}{0.000000,0.000000,0.000000}%
\pgfsetstrokecolor{textcolor}%
\pgfsetfillcolor{textcolor}%
\pgftext[x=0.417283in, y=0.924875in, left, base]{\color{textcolor}\rmfamily\fontsize{10.000000}{12.000000}\selectfont \(\displaystyle {\ensuremath{-}1.0}\)}%
\end{pgfscope}%
\begin{pgfscope}%
\pgfpathrectangle{\pgfqpoint{0.800000in}{0.528000in}}{\pgfqpoint{4.960000in}{3.696000in}}%
\pgfusepath{clip}%
\pgfsetrectcap%
\pgfsetroundjoin%
\pgfsetlinewidth{0.803000pt}%
\definecolor{currentstroke}{rgb}{0.690196,0.690196,0.690196}%
\pgfsetstrokecolor{currentstroke}%
\pgfsetdash{}{0pt}%
\pgfpathmoveto{\pgfqpoint{0.800000in}{1.675981in}}%
\pgfpathlineto{\pgfqpoint{5.760000in}{1.675981in}}%
\pgfusepath{stroke}%
\end{pgfscope}%
\begin{pgfscope}%
\pgfsetbuttcap%
\pgfsetroundjoin%
\definecolor{currentfill}{rgb}{0.000000,0.000000,0.000000}%
\pgfsetfillcolor{currentfill}%
\pgfsetlinewidth{0.803000pt}%
\definecolor{currentstroke}{rgb}{0.000000,0.000000,0.000000}%
\pgfsetstrokecolor{currentstroke}%
\pgfsetdash{}{0pt}%
\pgfsys@defobject{currentmarker}{\pgfqpoint{-0.048611in}{0.000000in}}{\pgfqpoint{-0.000000in}{0.000000in}}{%
\pgfpathmoveto{\pgfqpoint{-0.000000in}{0.000000in}}%
\pgfpathlineto{\pgfqpoint{-0.048611in}{0.000000in}}%
\pgfusepath{stroke,fill}%
}%
\begin{pgfscope}%
\pgfsys@transformshift{0.800000in}{1.675981in}%
\pgfsys@useobject{currentmarker}{}%
\end{pgfscope}%
\end{pgfscope}%
\begin{pgfscope}%
\definecolor{textcolor}{rgb}{0.000000,0.000000,0.000000}%
\pgfsetstrokecolor{textcolor}%
\pgfsetfillcolor{textcolor}%
\pgftext[x=0.417283in, y=1.624881in, left, base]{\color{textcolor}\rmfamily\fontsize{10.000000}{12.000000}\selectfont \(\displaystyle {\ensuremath{-}0.5}\)}%
\end{pgfscope}%
\begin{pgfscope}%
\pgfpathrectangle{\pgfqpoint{0.800000in}{0.528000in}}{\pgfqpoint{4.960000in}{3.696000in}}%
\pgfusepath{clip}%
\pgfsetrectcap%
\pgfsetroundjoin%
\pgfsetlinewidth{0.803000pt}%
\definecolor{currentstroke}{rgb}{0.690196,0.690196,0.690196}%
\pgfsetstrokecolor{currentstroke}%
\pgfsetdash{}{0pt}%
\pgfpathmoveto{\pgfqpoint{0.800000in}{2.375986in}}%
\pgfpathlineto{\pgfqpoint{5.760000in}{2.375986in}}%
\pgfusepath{stroke}%
\end{pgfscope}%
\begin{pgfscope}%
\pgfsetbuttcap%
\pgfsetroundjoin%
\definecolor{currentfill}{rgb}{0.000000,0.000000,0.000000}%
\pgfsetfillcolor{currentfill}%
\pgfsetlinewidth{0.803000pt}%
\definecolor{currentstroke}{rgb}{0.000000,0.000000,0.000000}%
\pgfsetstrokecolor{currentstroke}%
\pgfsetdash{}{0pt}%
\pgfsys@defobject{currentmarker}{\pgfqpoint{-0.048611in}{0.000000in}}{\pgfqpoint{-0.000000in}{0.000000in}}{%
\pgfpathmoveto{\pgfqpoint{-0.000000in}{0.000000in}}%
\pgfpathlineto{\pgfqpoint{-0.048611in}{0.000000in}}%
\pgfusepath{stroke,fill}%
}%
\begin{pgfscope}%
\pgfsys@transformshift{0.800000in}{2.375986in}%
\pgfsys@useobject{currentmarker}{}%
\end{pgfscope}%
\end{pgfscope}%
\begin{pgfscope}%
\definecolor{textcolor}{rgb}{0.000000,0.000000,0.000000}%
\pgfsetstrokecolor{textcolor}%
\pgfsetfillcolor{textcolor}%
\pgftext[x=0.525308in, y=2.324886in, left, base]{\color{textcolor}\rmfamily\fontsize{10.000000}{12.000000}\selectfont \(\displaystyle {0.0}\)}%
\end{pgfscope}%
\begin{pgfscope}%
\pgfpathrectangle{\pgfqpoint{0.800000in}{0.528000in}}{\pgfqpoint{4.960000in}{3.696000in}}%
\pgfusepath{clip}%
\pgfsetrectcap%
\pgfsetroundjoin%
\pgfsetlinewidth{0.803000pt}%
\definecolor{currentstroke}{rgb}{0.690196,0.690196,0.690196}%
\pgfsetstrokecolor{currentstroke}%
\pgfsetdash{}{0pt}%
\pgfpathmoveto{\pgfqpoint{0.800000in}{3.075992in}}%
\pgfpathlineto{\pgfqpoint{5.760000in}{3.075992in}}%
\pgfusepath{stroke}%
\end{pgfscope}%
\begin{pgfscope}%
\pgfsetbuttcap%
\pgfsetroundjoin%
\definecolor{currentfill}{rgb}{0.000000,0.000000,0.000000}%
\pgfsetfillcolor{currentfill}%
\pgfsetlinewidth{0.803000pt}%
\definecolor{currentstroke}{rgb}{0.000000,0.000000,0.000000}%
\pgfsetstrokecolor{currentstroke}%
\pgfsetdash{}{0pt}%
\pgfsys@defobject{currentmarker}{\pgfqpoint{-0.048611in}{0.000000in}}{\pgfqpoint{-0.000000in}{0.000000in}}{%
\pgfpathmoveto{\pgfqpoint{-0.000000in}{0.000000in}}%
\pgfpathlineto{\pgfqpoint{-0.048611in}{0.000000in}}%
\pgfusepath{stroke,fill}%
}%
\begin{pgfscope}%
\pgfsys@transformshift{0.800000in}{3.075992in}%
\pgfsys@useobject{currentmarker}{}%
\end{pgfscope}%
\end{pgfscope}%
\begin{pgfscope}%
\definecolor{textcolor}{rgb}{0.000000,0.000000,0.000000}%
\pgfsetstrokecolor{textcolor}%
\pgfsetfillcolor{textcolor}%
\pgftext[x=0.525308in, y=3.024892in, left, base]{\color{textcolor}\rmfamily\fontsize{10.000000}{12.000000}\selectfont \(\displaystyle {0.5}\)}%
\end{pgfscope}%
\begin{pgfscope}%
\pgfpathrectangle{\pgfqpoint{0.800000in}{0.528000in}}{\pgfqpoint{4.960000in}{3.696000in}}%
\pgfusepath{clip}%
\pgfsetrectcap%
\pgfsetroundjoin%
\pgfsetlinewidth{0.803000pt}%
\definecolor{currentstroke}{rgb}{0.690196,0.690196,0.690196}%
\pgfsetstrokecolor{currentstroke}%
\pgfsetdash{}{0pt}%
\pgfpathmoveto{\pgfqpoint{0.800000in}{3.775998in}}%
\pgfpathlineto{\pgfqpoint{5.760000in}{3.775998in}}%
\pgfusepath{stroke}%
\end{pgfscope}%
\begin{pgfscope}%
\pgfsetbuttcap%
\pgfsetroundjoin%
\definecolor{currentfill}{rgb}{0.000000,0.000000,0.000000}%
\pgfsetfillcolor{currentfill}%
\pgfsetlinewidth{0.803000pt}%
\definecolor{currentstroke}{rgb}{0.000000,0.000000,0.000000}%
\pgfsetstrokecolor{currentstroke}%
\pgfsetdash{}{0pt}%
\pgfsys@defobject{currentmarker}{\pgfqpoint{-0.048611in}{0.000000in}}{\pgfqpoint{-0.000000in}{0.000000in}}{%
\pgfpathmoveto{\pgfqpoint{-0.000000in}{0.000000in}}%
\pgfpathlineto{\pgfqpoint{-0.048611in}{0.000000in}}%
\pgfusepath{stroke,fill}%
}%
\begin{pgfscope}%
\pgfsys@transformshift{0.800000in}{3.775998in}%
\pgfsys@useobject{currentmarker}{}%
\end{pgfscope}%
\end{pgfscope}%
\begin{pgfscope}%
\definecolor{textcolor}{rgb}{0.000000,0.000000,0.000000}%
\pgfsetstrokecolor{textcolor}%
\pgfsetfillcolor{textcolor}%
\pgftext[x=0.525308in, y=3.724898in, left, base]{\color{textcolor}\rmfamily\fontsize{10.000000}{12.000000}\selectfont \(\displaystyle {1.0}\)}%
\end{pgfscope}%
\begin{pgfscope}%
\definecolor{textcolor}{rgb}{0.000000,0.000000,0.000000}%
\pgfsetstrokecolor{textcolor}%
\pgfsetfillcolor{textcolor}%
\pgftext[x=0.361727in,y=2.376000in,,bottom,rotate=90.000000]{\color{textcolor}\rmfamily\fontsize{10.000000}{12.000000}\selectfont electrical power in \(\displaystyle \mathrm{p.u.}\)}%
\end{pgfscope}%
\begin{pgfscope}%
\pgfpathrectangle{\pgfqpoint{0.800000in}{0.528000in}}{\pgfqpoint{4.960000in}{3.696000in}}%
\pgfusepath{clip}%
\pgfsetrectcap%
\pgfsetroundjoin%
\pgfsetlinewidth{1.505625pt}%
\definecolor{currentstroke}{rgb}{0.121569,0.466667,0.705882}%
\pgfsetstrokecolor{currentstroke}%
\pgfsetdash{}{0pt}%
\pgfpathmoveto{\pgfqpoint{1.043857in}{3.775954in}}%
\pgfpathlineto{\pgfqpoint{2.668487in}{3.775966in}}%
\pgfpathlineto{\pgfqpoint{2.670113in}{3.419109in}}%
\pgfpathlineto{\pgfqpoint{2.696133in}{3.420179in}}%
\pgfpathlineto{\pgfqpoint{2.722153in}{3.423369in}}%
\pgfpathlineto{\pgfqpoint{2.749800in}{3.429002in}}%
\pgfpathlineto{\pgfqpoint{2.779072in}{3.437341in}}%
\pgfpathlineto{\pgfqpoint{2.809971in}{3.448556in}}%
\pgfpathlineto{\pgfqpoint{2.844123in}{3.463449in}}%
\pgfpathlineto{\pgfqpoint{2.883153in}{3.483033in}}%
\pgfpathlineto{\pgfqpoint{2.935193in}{3.511909in}}%
\pgfpathlineto{\pgfqpoint{3.040900in}{3.571236in}}%
\pgfpathlineto{\pgfqpoint{3.079930in}{3.590257in}}%
\pgfpathlineto{\pgfqpoint{3.114081in}{3.604435in}}%
\pgfpathlineto{\pgfqpoint{3.143354in}{3.614295in}}%
\pgfpathlineto{\pgfqpoint{3.171000in}{3.621340in}}%
\pgfpathlineto{\pgfqpoint{3.197020in}{3.625710in}}%
\pgfpathlineto{\pgfqpoint{3.221414in}{3.627622in}}%
\pgfpathlineto{\pgfqpoint{3.223040in}{4.055876in}}%
\pgfpathlineto{\pgfqpoint{3.244182in}{4.055542in}}%
\pgfpathlineto{\pgfqpoint{3.266949in}{4.053008in}}%
\pgfpathlineto{\pgfqpoint{3.292969in}{4.047792in}}%
\pgfpathlineto{\pgfqpoint{3.322242in}{4.039550in}}%
\pgfpathlineto{\pgfqpoint{3.356393in}{4.027553in}}%
\pgfpathlineto{\pgfqpoint{3.400302in}{4.009649in}}%
\pgfpathlineto{\pgfqpoint{3.478363in}{3.974919in}}%
\pgfpathlineto{\pgfqpoint{3.562928in}{3.938166in}}%
\pgfpathlineto{\pgfqpoint{3.621473in}{3.915136in}}%
\pgfpathlineto{\pgfqpoint{3.676766in}{3.895778in}}%
\pgfpathlineto{\pgfqpoint{3.730432in}{3.879357in}}%
\pgfpathlineto{\pgfqpoint{3.784099in}{3.865253in}}%
\pgfpathlineto{\pgfqpoint{3.839391in}{3.853039in}}%
\pgfpathlineto{\pgfqpoint{3.896310in}{3.842758in}}%
\pgfpathlineto{\pgfqpoint{3.954856in}{3.834417in}}%
\pgfpathlineto{\pgfqpoint{4.016653in}{3.827864in}}%
\pgfpathlineto{\pgfqpoint{4.080077in}{3.823349in}}%
\pgfpathlineto{\pgfqpoint{4.146754in}{3.820855in}}%
\pgfpathlineto{\pgfqpoint{4.213430in}{3.820569in}}%
\pgfpathlineto{\pgfqpoint{4.280107in}{3.822469in}}%
\pgfpathlineto{\pgfqpoint{4.346783in}{3.826608in}}%
\pgfpathlineto{\pgfqpoint{4.410207in}{3.832739in}}%
\pgfpathlineto{\pgfqpoint{4.472005in}{3.840930in}}%
\pgfpathlineto{\pgfqpoint{4.530550in}{3.850882in}}%
\pgfpathlineto{\pgfqpoint{4.587469in}{3.862775in}}%
\pgfpathlineto{\pgfqpoint{4.642762in}{3.876568in}}%
\pgfpathlineto{\pgfqpoint{4.698055in}{3.892678in}}%
\pgfpathlineto{\pgfqpoint{4.753347in}{3.911133in}}%
\pgfpathlineto{\pgfqpoint{4.810266in}{3.932466in}}%
\pgfpathlineto{\pgfqpoint{4.875317in}{3.959275in}}%
\pgfpathlineto{\pgfqpoint{5.041195in}{4.029069in}}%
\pgfpathlineto{\pgfqpoint{5.078599in}{4.041451in}}%
\pgfpathlineto{\pgfqpoint{5.109498in}{4.049361in}}%
\pgfpathlineto{\pgfqpoint{5.135518in}{4.053866in}}%
\pgfpathlineto{\pgfqpoint{5.159911in}{4.055875in}}%
\pgfpathlineto{\pgfqpoint{5.182679in}{4.055462in}}%
\pgfpathlineto{\pgfqpoint{5.203820in}{4.052799in}}%
\pgfpathlineto{\pgfqpoint{5.223335in}{4.048148in}}%
\pgfpathlineto{\pgfqpoint{5.242851in}{4.041172in}}%
\pgfpathlineto{\pgfqpoint{5.262366in}{4.031661in}}%
\pgfpathlineto{\pgfqpoint{5.280254in}{4.020544in}}%
\pgfpathlineto{\pgfqpoint{5.298143in}{4.006988in}}%
\pgfpathlineto{\pgfqpoint{5.317658in}{3.989273in}}%
\pgfpathlineto{\pgfqpoint{5.337173in}{3.968377in}}%
\pgfpathlineto{\pgfqpoint{5.356688in}{3.944204in}}%
\pgfpathlineto{\pgfqpoint{5.377830in}{3.914263in}}%
\pgfpathlineto{\pgfqpoint{5.398971in}{3.880425in}}%
\pgfpathlineto{\pgfqpoint{5.421739in}{3.839725in}}%
\pgfpathlineto{\pgfqpoint{5.446133in}{3.791465in}}%
\pgfpathlineto{\pgfqpoint{5.473779in}{3.731490in}}%
\pgfpathlineto{\pgfqpoint{5.504678in}{3.658876in}}%
\pgfpathlineto{\pgfqpoint{5.545334in}{3.557024in}}%
\pgfpathlineto{\pgfqpoint{5.639657in}{3.318140in}}%
\pgfpathlineto{\pgfqpoint{5.668930in}{3.251060in}}%
\pgfpathlineto{\pgfqpoint{5.693323in}{3.200210in}}%
\pgfpathlineto{\pgfqpoint{5.716091in}{3.157835in}}%
\pgfpathlineto{\pgfqpoint{5.735606in}{3.125983in}}%
\pgfpathlineto{\pgfqpoint{5.753495in}{3.100766in}}%
\pgfpathlineto{\pgfqpoint{5.760000in}{3.092594in}}%
\pgfpathlineto{\pgfqpoint{5.760000in}{3.092594in}}%
\pgfusepath{stroke}%
\end{pgfscope}%
\begin{pgfscope}%
\pgfpathrectangle{\pgfqpoint{0.800000in}{0.528000in}}{\pgfqpoint{4.960000in}{3.696000in}}%
\pgfusepath{clip}%
\pgfsetrectcap%
\pgfsetroundjoin%
\pgfsetlinewidth{1.505625pt}%
\definecolor{currentstroke}{rgb}{1.000000,0.498039,0.054902}%
\pgfsetstrokecolor{currentstroke}%
\pgfsetdash{}{0pt}%
\pgfpathmoveto{\pgfqpoint{1.043857in}{3.775954in}}%
\pgfpathlineto{\pgfqpoint{2.668487in}{3.775966in}}%
\pgfpathlineto{\pgfqpoint{2.670113in}{3.419109in}}%
\pgfpathlineto{\pgfqpoint{2.696133in}{3.420179in}}%
\pgfpathlineto{\pgfqpoint{2.722153in}{3.423369in}}%
\pgfpathlineto{\pgfqpoint{2.749800in}{3.429002in}}%
\pgfpathlineto{\pgfqpoint{2.779072in}{3.437341in}}%
\pgfpathlineto{\pgfqpoint{2.809971in}{3.448556in}}%
\pgfpathlineto{\pgfqpoint{2.844123in}{3.463449in}}%
\pgfpathlineto{\pgfqpoint{2.883153in}{3.483033in}}%
\pgfpathlineto{\pgfqpoint{2.935193in}{3.511909in}}%
\pgfpathlineto{\pgfqpoint{3.040900in}{3.571236in}}%
\pgfpathlineto{\pgfqpoint{3.079930in}{3.590257in}}%
\pgfpathlineto{\pgfqpoint{3.114081in}{3.604435in}}%
\pgfpathlineto{\pgfqpoint{3.143354in}{3.614295in}}%
\pgfpathlineto{\pgfqpoint{3.171000in}{3.621340in}}%
\pgfpathlineto{\pgfqpoint{3.197020in}{3.625710in}}%
\pgfpathlineto{\pgfqpoint{3.221414in}{3.627624in}}%
\pgfpathlineto{\pgfqpoint{3.223040in}{4.055880in}}%
\pgfpathlineto{\pgfqpoint{3.244182in}{4.055483in}}%
\pgfpathlineto{\pgfqpoint{3.266949in}{4.052710in}}%
\pgfpathlineto{\pgfqpoint{3.291343in}{4.047449in}}%
\pgfpathlineto{\pgfqpoint{3.318989in}{4.039142in}}%
\pgfpathlineto{\pgfqpoint{3.351515in}{4.026895in}}%
\pgfpathlineto{\pgfqpoint{3.390545in}{4.009703in}}%
\pgfpathlineto{\pgfqpoint{3.445837in}{3.982667in}}%
\pgfpathlineto{\pgfqpoint{3.614968in}{3.898461in}}%
\pgfpathlineto{\pgfqpoint{3.684897in}{3.866983in}}%
\pgfpathlineto{\pgfqpoint{3.759705in}{3.835844in}}%
\pgfpathlineto{\pgfqpoint{3.860533in}{3.796572in}}%
\pgfpathlineto{\pgfqpoint{3.982502in}{3.748711in}}%
\pgfpathlineto{\pgfqpoint{4.042673in}{3.722621in}}%
\pgfpathlineto{\pgfqpoint{4.091461in}{3.699058in}}%
\pgfpathlineto{\pgfqpoint{4.133744in}{3.676159in}}%
\pgfpathlineto{\pgfqpoint{4.171148in}{3.653396in}}%
\pgfpathlineto{\pgfqpoint{4.205299in}{3.630048in}}%
\pgfpathlineto{\pgfqpoint{4.237824in}{3.605039in}}%
\pgfpathlineto{\pgfqpoint{4.267097in}{3.579770in}}%
\pgfpathlineto{\pgfqpoint{4.294743in}{3.553072in}}%
\pgfpathlineto{\pgfqpoint{4.320763in}{3.525020in}}%
\pgfpathlineto{\pgfqpoint{4.346783in}{3.493685in}}%
\pgfpathlineto{\pgfqpoint{4.371177in}{3.460878in}}%
\pgfpathlineto{\pgfqpoint{4.395571in}{3.424260in}}%
\pgfpathlineto{\pgfqpoint{4.418339in}{3.386160in}}%
\pgfpathlineto{\pgfqpoint{4.441106in}{3.343747in}}%
\pgfpathlineto{\pgfqpoint{4.463874in}{3.296438in}}%
\pgfpathlineto{\pgfqpoint{4.485015in}{3.247547in}}%
\pgfpathlineto{\pgfqpoint{4.506156in}{3.193271in}}%
\pgfpathlineto{\pgfqpoint{4.527298in}{3.132945in}}%
\pgfpathlineto{\pgfqpoint{4.548439in}{3.065834in}}%
\pgfpathlineto{\pgfqpoint{4.569580in}{2.991133in}}%
\pgfpathlineto{\pgfqpoint{4.590722in}{2.907968in}}%
\pgfpathlineto{\pgfqpoint{4.611863in}{2.815411in}}%
\pgfpathlineto{\pgfqpoint{4.633004in}{2.712503in}}%
\pgfpathlineto{\pgfqpoint{4.654146in}{2.598293in}}%
\pgfpathlineto{\pgfqpoint{4.675287in}{2.471903in}}%
\pgfpathlineto{\pgfqpoint{4.698055in}{2.321365in}}%
\pgfpathlineto{\pgfqpoint{4.720822in}{2.155423in}}%
\pgfpathlineto{\pgfqpoint{4.745216in}{1.960863in}}%
\pgfpathlineto{\pgfqpoint{4.772863in}{1.721790in}}%
\pgfpathlineto{\pgfqpoint{4.820024in}{1.289368in}}%
\pgfpathlineto{\pgfqpoint{4.847670in}{1.046703in}}%
\pgfpathlineto{\pgfqpoint{4.865559in}{0.908612in}}%
\pgfpathlineto{\pgfqpoint{4.878569in}{0.823879in}}%
\pgfpathlineto{\pgfqpoint{4.889953in}{0.764314in}}%
\pgfpathlineto{\pgfqpoint{4.898084in}{0.731880in}}%
\pgfpathlineto{\pgfqpoint{4.904589in}{0.712889in}}%
\pgfpathlineto{\pgfqpoint{4.911094in}{0.700735in}}%
\pgfpathlineto{\pgfqpoint{4.915973in}{0.696465in}}%
\pgfpathlineto{\pgfqpoint{4.919226in}{0.696063in}}%
\pgfpathlineto{\pgfqpoint{4.922478in}{0.697702in}}%
\pgfpathlineto{\pgfqpoint{4.925731in}{0.701451in}}%
\pgfpathlineto{\pgfqpoint{4.930609in}{0.711180in}}%
\pgfpathlineto{\pgfqpoint{4.935488in}{0.726026in}}%
\pgfpathlineto{\pgfqpoint{4.941993in}{0.754130in}}%
\pgfpathlineto{\pgfqpoint{4.948498in}{0.792116in}}%
\pgfpathlineto{\pgfqpoint{4.956629in}{0.854019in}}%
\pgfpathlineto{\pgfqpoint{4.964761in}{0.932385in}}%
\pgfpathlineto{\pgfqpoint{4.974518in}{1.048444in}}%
\pgfpathlineto{\pgfqpoint{4.985902in}{1.213877in}}%
\pgfpathlineto{\pgfqpoint{4.998912in}{1.440504in}}%
\pgfpathlineto{\pgfqpoint{5.013548in}{1.737593in}}%
\pgfpathlineto{\pgfqpoint{5.033063in}{2.185004in}}%
\pgfpathlineto{\pgfqpoint{5.080225in}{3.294429in}}%
\pgfpathlineto{\pgfqpoint{5.094861in}{3.577555in}}%
\pgfpathlineto{\pgfqpoint{5.106245in}{3.759069in}}%
\pgfpathlineto{\pgfqpoint{5.116003in}{3.882940in}}%
\pgfpathlineto{\pgfqpoint{5.124134in}{3.961731in}}%
\pgfpathlineto{\pgfqpoint{5.130639in}{4.007961in}}%
\pgfpathlineto{\pgfqpoint{5.137144in}{4.038846in}}%
\pgfpathlineto{\pgfqpoint{5.142023in}{4.051811in}}%
\pgfpathlineto{\pgfqpoint{5.145275in}{4.055577in}}%
\pgfpathlineto{\pgfqpoint{5.148528in}{4.055445in}}%
\pgfpathlineto{\pgfqpoint{5.151780in}{4.051428in}}%
\pgfpathlineto{\pgfqpoint{5.155033in}{4.043546in}}%
\pgfpathlineto{\pgfqpoint{5.159911in}{4.024548in}}%
\pgfpathlineto{\pgfqpoint{5.166416in}{3.986041in}}%
\pgfpathlineto{\pgfqpoint{5.172922in}{3.932876in}}%
\pgfpathlineto{\pgfqpoint{5.181053in}{3.846638in}}%
\pgfpathlineto{\pgfqpoint{5.190810in}{3.715848in}}%
\pgfpathlineto{\pgfqpoint{5.202194in}{3.528946in}}%
\pgfpathlineto{\pgfqpoint{5.215204in}{3.276020in}}%
\pgfpathlineto{\pgfqpoint{5.231467in}{2.912952in}}%
\pgfpathlineto{\pgfqpoint{5.255861in}{2.308738in}}%
\pgfpathlineto{\pgfqpoint{5.288386in}{1.506106in}}%
\pgfpathlineto{\pgfqpoint{5.303022in}{1.194336in}}%
\pgfpathlineto{\pgfqpoint{5.314406in}{0.991868in}}%
\pgfpathlineto{\pgfqpoint{5.324163in}{0.854440in}}%
\pgfpathlineto{\pgfqpoint{5.332295in}{0.769975in}}%
\pgfpathlineto{\pgfqpoint{5.338800in}{0.724398in}}%
\pgfpathlineto{\pgfqpoint{5.343678in}{0.704016in}}%
\pgfpathlineto{\pgfqpoint{5.346931in}{0.697306in}}%
\pgfpathlineto{\pgfqpoint{5.348557in}{0.696068in}}%
\pgfpathlineto{\pgfqpoint{5.350183in}{0.696263in}}%
\pgfpathlineto{\pgfqpoint{5.353436in}{0.701010in}}%
\pgfpathlineto{\pgfqpoint{5.356688in}{0.711648in}}%
\pgfpathlineto{\pgfqpoint{5.361567in}{0.738824in}}%
\pgfpathlineto{\pgfqpoint{5.366446in}{0.779609in}}%
\pgfpathlineto{\pgfqpoint{5.372951in}{0.855225in}}%
\pgfpathlineto{\pgfqpoint{5.381082in}{0.983403in}}%
\pgfpathlineto{\pgfqpoint{5.389214in}{1.147437in}}%
\pgfpathlineto{\pgfqpoint{5.398971in}{1.387529in}}%
\pgfpathlineto{\pgfqpoint{5.411981in}{1.768588in}}%
\pgfpathlineto{\pgfqpoint{5.431496in}{2.419811in}}%
\pgfpathlineto{\pgfqpoint{5.455890in}{3.224039in}}%
\pgfpathlineto{\pgfqpoint{5.468900in}{3.580129in}}%
\pgfpathlineto{\pgfqpoint{5.478658in}{3.791253in}}%
\pgfpathlineto{\pgfqpoint{5.486789in}{3.923168in}}%
\pgfpathlineto{\pgfqpoint{5.493294in}{3.997129in}}%
\pgfpathlineto{\pgfqpoint{5.498173in}{4.033313in}}%
\pgfpathlineto{\pgfqpoint{5.503052in}{4.052597in}}%
\pgfpathlineto{\pgfqpoint{5.506304in}{4.055990in}}%
\pgfpathlineto{\pgfqpoint{5.507930in}{4.054846in}}%
\pgfpathlineto{\pgfqpoint{5.511183in}{4.046894in}}%
\pgfpathlineto{\pgfqpoint{5.514435in}{4.031441in}}%
\pgfpathlineto{\pgfqpoint{5.519314in}{3.994393in}}%
\pgfpathlineto{\pgfqpoint{5.525819in}{3.919812in}}%
\pgfpathlineto{\pgfqpoint{5.532324in}{3.817774in}}%
\pgfpathlineto{\pgfqpoint{5.540455in}{3.654475in}}%
\pgfpathlineto{\pgfqpoint{5.550213in}{3.412080in}}%
\pgfpathlineto{\pgfqpoint{5.563223in}{3.025595in}}%
\pgfpathlineto{\pgfqpoint{5.582738in}{2.364015in}}%
\pgfpathlineto{\pgfqpoint{5.608758in}{1.488905in}}%
\pgfpathlineto{\pgfqpoint{5.621768in}{1.127923in}}%
\pgfpathlineto{\pgfqpoint{5.631526in}{0.917253in}}%
\pgfpathlineto{\pgfqpoint{5.639657in}{0.790971in}}%
\pgfpathlineto{\pgfqpoint{5.646162in}{0.726600in}}%
\pgfpathlineto{\pgfqpoint{5.651041in}{0.701374in}}%
\pgfpathlineto{\pgfqpoint{5.654293in}{0.696000in}}%
\pgfpathlineto{\pgfqpoint{5.655920in}{0.696818in}}%
\pgfpathlineto{\pgfqpoint{5.659172in}{0.705550in}}%
\pgfpathlineto{\pgfqpoint{5.662425in}{0.723824in}}%
\pgfpathlineto{\pgfqpoint{5.667303in}{0.769204in}}%
\pgfpathlineto{\pgfqpoint{5.672182in}{0.836031in}}%
\pgfpathlineto{\pgfqpoint{5.678687in}{0.957710in}}%
\pgfpathlineto{\pgfqpoint{5.686818in}{1.159245in}}%
\pgfpathlineto{\pgfqpoint{5.696576in}{1.465115in}}%
\pgfpathlineto{\pgfqpoint{5.709586in}{1.955243in}}%
\pgfpathlineto{\pgfqpoint{5.746990in}{3.435778in}}%
\pgfpathlineto{\pgfqpoint{5.756747in}{3.720994in}}%
\pgfpathlineto{\pgfqpoint{5.760000in}{3.798820in}}%
\pgfpathlineto{\pgfqpoint{5.760000in}{3.798820in}}%
\pgfusepath{stroke}%
\end{pgfscope}%
\begin{pgfscope}%
\pgfsetrectcap%
\pgfsetmiterjoin%
\pgfsetlinewidth{0.803000pt}%
\definecolor{currentstroke}{rgb}{0.000000,0.000000,0.000000}%
\pgfsetstrokecolor{currentstroke}%
\pgfsetdash{}{0pt}%
\pgfpathmoveto{\pgfqpoint{0.800000in}{0.528000in}}%
\pgfpathlineto{\pgfqpoint{0.800000in}{4.224000in}}%
\pgfusepath{stroke}%
\end{pgfscope}%
\begin{pgfscope}%
\pgfsetrectcap%
\pgfsetmiterjoin%
\pgfsetlinewidth{0.803000pt}%
\definecolor{currentstroke}{rgb}{0.000000,0.000000,0.000000}%
\pgfsetstrokecolor{currentstroke}%
\pgfsetdash{}{0pt}%
\pgfpathmoveto{\pgfqpoint{5.760000in}{0.528000in}}%
\pgfpathlineto{\pgfqpoint{5.760000in}{4.224000in}}%
\pgfusepath{stroke}%
\end{pgfscope}%
\begin{pgfscope}%
\pgfsetrectcap%
\pgfsetmiterjoin%
\pgfsetlinewidth{0.803000pt}%
\definecolor{currentstroke}{rgb}{0.000000,0.000000,0.000000}%
\pgfsetstrokecolor{currentstroke}%
\pgfsetdash{}{0pt}%
\pgfpathmoveto{\pgfqpoint{0.800000in}{0.528000in}}%
\pgfpathlineto{\pgfqpoint{5.760000in}{0.528000in}}%
\pgfusepath{stroke}%
\end{pgfscope}%
\begin{pgfscope}%
\pgfsetrectcap%
\pgfsetmiterjoin%
\pgfsetlinewidth{0.803000pt}%
\definecolor{currentstroke}{rgb}{0.000000,0.000000,0.000000}%
\pgfsetstrokecolor{currentstroke}%
\pgfsetdash{}{0pt}%
\pgfpathmoveto{\pgfqpoint{0.800000in}{4.224000in}}%
\pgfpathlineto{\pgfqpoint{5.760000in}{4.224000in}}%
\pgfusepath{stroke}%
\end{pgfscope}%
\begin{pgfscope}%
\definecolor{textcolor}{rgb}{0.000000,0.000000,0.000000}%
\pgfsetstrokecolor{textcolor}%
\pgfsetfillcolor{textcolor}%
\pgftext[x=3.280000in,y=4.307333in,,base]{\color{textcolor}\rmfamily\fontsize{12.000000}{14.400000}\selectfont Electrical power over time - fault 2}%
\end{pgfscope}%
\begin{pgfscope}%
\pgfsetbuttcap%
\pgfsetmiterjoin%
\definecolor{currentfill}{rgb}{1.000000,1.000000,1.000000}%
\pgfsetfillcolor{currentfill}%
\pgfsetfillopacity{0.800000}%
\pgfsetlinewidth{1.003750pt}%
\definecolor{currentstroke}{rgb}{0.800000,0.800000,0.800000}%
\pgfsetstrokecolor{currentstroke}%
\pgfsetstrokeopacity{0.800000}%
\pgfsetdash{}{0pt}%
\pgfpathmoveto{\pgfqpoint{0.897222in}{0.597444in}}%
\pgfpathlineto{\pgfqpoint{2.740115in}{0.597444in}}%
\pgfpathquadraticcurveto{\pgfqpoint{2.767893in}{0.597444in}}{\pgfqpoint{2.767893in}{0.625222in}}%
\pgfpathlineto{\pgfqpoint{2.767893in}{1.015114in}}%
\pgfpathquadraticcurveto{\pgfqpoint{2.767893in}{1.042892in}}{\pgfqpoint{2.740115in}{1.042892in}}%
\pgfpathlineto{\pgfqpoint{0.897222in}{1.042892in}}%
\pgfpathquadraticcurveto{\pgfqpoint{0.869444in}{1.042892in}}{\pgfqpoint{0.869444in}{1.015114in}}%
\pgfpathlineto{\pgfqpoint{0.869444in}{0.625222in}}%
\pgfpathquadraticcurveto{\pgfqpoint{0.869444in}{0.597444in}}{\pgfqpoint{0.897222in}{0.597444in}}%
\pgfpathlineto{\pgfqpoint{0.897222in}{0.597444in}}%
\pgfpathclose%
\pgfusepath{stroke,fill}%
\end{pgfscope}%
\begin{pgfscope}%
\pgfsetrectcap%
\pgfsetroundjoin%
\pgfsetlinewidth{1.505625pt}%
\definecolor{currentstroke}{rgb}{0.121569,0.466667,0.705882}%
\pgfsetstrokecolor{currentstroke}%
\pgfsetdash{}{0pt}%
\pgfpathmoveto{\pgfqpoint{0.925000in}{0.933748in}}%
\pgfpathlineto{\pgfqpoint{1.063889in}{0.933748in}}%
\pgfpathlineto{\pgfqpoint{1.202778in}{0.933748in}}%
\pgfusepath{stroke}%
\end{pgfscope}%
\begin{pgfscope}%
\definecolor{textcolor}{rgb}{0.000000,0.000000,0.000000}%
\pgfsetstrokecolor{textcolor}%
\pgfsetfillcolor{textcolor}%
\pgftext[x=1.313889in,y=0.885137in,left,base]{\color{textcolor}\rmfamily\fontsize{10.000000}{12.000000}\selectfont \(\displaystyle \Delta P\) - stable scenario}%
\end{pgfscope}%
\begin{pgfscope}%
\pgfsetrectcap%
\pgfsetroundjoin%
\pgfsetlinewidth{1.505625pt}%
\definecolor{currentstroke}{rgb}{1.000000,0.498039,0.054902}%
\pgfsetstrokecolor{currentstroke}%
\pgfsetdash{}{0pt}%
\pgfpathmoveto{\pgfqpoint{0.925000in}{0.731857in}}%
\pgfpathlineto{\pgfqpoint{1.063889in}{0.731857in}}%
\pgfpathlineto{\pgfqpoint{1.202778in}{0.731857in}}%
\pgfusepath{stroke}%
\end{pgfscope}%
\begin{pgfscope}%
\definecolor{textcolor}{rgb}{0.000000,0.000000,0.000000}%
\pgfsetstrokecolor{textcolor}%
\pgfsetfillcolor{textcolor}%
\pgftext[x=1.313889in,y=0.683246in,left,base]{\color{textcolor}\rmfamily\fontsize{10.000000}{12.000000}\selectfont \(\displaystyle \Delta P\) - unstable scenario}%
\end{pgfscope}%
\end{pgfpicture}%
\makeatother%
\endgroup%


\section{Fault 3}
\label{app:fault3}

%% Creator: Matplotlib, PGF backend
%%
%% To include the figure in your LaTeX document, write
%%   \input{<filename>.pgf}
%%
%% Make sure the required packages are loaded in your preamble
%%   \usepackage{pgf}
%%
%% Also ensure that all the required font packages are loaded; for instance,
%% the lmodern package is sometimes necessary when using math font.
%%   \usepackage{lmodern}
%%
%% Figures using additional raster images can only be included by \input if
%% they are in the same directory as the main LaTeX file. For loading figures
%% from other directories you can use the `import` package
%%   \usepackage{import}
%%
%% and then include the figures with
%%   \import{<path to file>}{<filename>.pgf}
%%
%% Matplotlib used the following preamble
%%   
%%   \usepackage{fontspec}
%%   \setmainfont{Charter.ttc}[Path=\detokenize{/System/Library/Fonts/Supplemental/}]
%%   \setsansfont{DejaVuSans.ttf}[Path=\detokenize{/opt/homebrew/lib/python3.10/site-packages/matplotlib/mpl-data/fonts/ttf/}]
%%   \setmonofont{DejaVuSansMono.ttf}[Path=\detokenize{/opt/homebrew/lib/python3.10/site-packages/matplotlib/mpl-data/fonts/ttf/}]
%%   \makeatletter\@ifpackageloaded{underscore}{}{\usepackage[strings]{underscore}}\makeatother
%%
\begingroup%
\makeatletter%
\begin{pgfpicture}%
\pgfpathrectangle{\pgfpointorigin}{\pgfqpoint{5.000000in}{6.000000in}}%
\pgfusepath{use as bounding box, clip}%
\begin{pgfscope}%
\pgfsetbuttcap%
\pgfsetmiterjoin%
\definecolor{currentfill}{rgb}{1.000000,1.000000,1.000000}%
\pgfsetfillcolor{currentfill}%
\pgfsetlinewidth{0.000000pt}%
\definecolor{currentstroke}{rgb}{1.000000,1.000000,1.000000}%
\pgfsetstrokecolor{currentstroke}%
\pgfsetdash{}{0pt}%
\pgfpathmoveto{\pgfqpoint{0.000000in}{0.000000in}}%
\pgfpathlineto{\pgfqpoint{5.000000in}{0.000000in}}%
\pgfpathlineto{\pgfqpoint{5.000000in}{6.000000in}}%
\pgfpathlineto{\pgfqpoint{0.000000in}{6.000000in}}%
\pgfpathlineto{\pgfqpoint{0.000000in}{0.000000in}}%
\pgfpathclose%
\pgfusepath{fill}%
\end{pgfscope}%
\begin{pgfscope}%
\pgfsetbuttcap%
\pgfsetmiterjoin%
\definecolor{currentfill}{rgb}{1.000000,1.000000,1.000000}%
\pgfsetfillcolor{currentfill}%
\pgfsetlinewidth{0.000000pt}%
\definecolor{currentstroke}{rgb}{0.000000,0.000000,0.000000}%
\pgfsetstrokecolor{currentstroke}%
\pgfsetstrokeopacity{0.000000}%
\pgfsetdash{}{0pt}%
\pgfpathmoveto{\pgfqpoint{0.625000in}{2.970000in}}%
\pgfpathlineto{\pgfqpoint{4.500000in}{2.970000in}}%
\pgfpathlineto{\pgfqpoint{4.500000in}{5.280000in}}%
\pgfpathlineto{\pgfqpoint{0.625000in}{5.280000in}}%
\pgfpathlineto{\pgfqpoint{0.625000in}{2.970000in}}%
\pgfpathclose%
\pgfusepath{fill}%
\end{pgfscope}%
\begin{pgfscope}%
\pgfpathrectangle{\pgfqpoint{0.625000in}{2.970000in}}{\pgfqpoint{3.875000in}{2.310000in}}%
\pgfusepath{clip}%
\pgfsetbuttcap%
\pgfsetroundjoin%
\definecolor{currentfill}{rgb}{0.900000,0.900000,0.900000}%
\pgfsetfillcolor{currentfill}%
\pgfsetlinewidth{1.003750pt}%
\definecolor{currentstroke}{rgb}{0.500000,0.500000,0.500000}%
\pgfsetstrokecolor{currentstroke}%
\pgfsetdash{}{0pt}%
\pgfsys@defobject{currentmarker}{\pgfqpoint{1.392896in}{3.926344in}}{\pgfqpoint{1.747563in}{4.264932in}}{%
\pgfpathmoveto{\pgfqpoint{1.392896in}{4.253993in}}%
\pgfpathlineto{\pgfqpoint{1.392896in}{3.926344in}}%
\pgfpathlineto{\pgfqpoint{1.400134in}{3.934146in}}%
\pgfpathlineto{\pgfqpoint{1.407372in}{3.941915in}}%
\pgfpathlineto{\pgfqpoint{1.414610in}{3.949650in}}%
\pgfpathlineto{\pgfqpoint{1.421848in}{3.957352in}}%
\pgfpathlineto{\pgfqpoint{1.429086in}{3.965020in}}%
\pgfpathlineto{\pgfqpoint{1.436324in}{3.972653in}}%
\pgfpathlineto{\pgfqpoint{1.443563in}{3.980252in}}%
\pgfpathlineto{\pgfqpoint{1.450801in}{3.987816in}}%
\pgfpathlineto{\pgfqpoint{1.458039in}{3.995345in}}%
\pgfpathlineto{\pgfqpoint{1.465277in}{4.002839in}}%
\pgfpathlineto{\pgfqpoint{1.472515in}{4.010297in}}%
\pgfpathlineto{\pgfqpoint{1.479753in}{4.017719in}}%
\pgfpathlineto{\pgfqpoint{1.486991in}{4.025105in}}%
\pgfpathlineto{\pgfqpoint{1.494229in}{4.032455in}}%
\pgfpathlineto{\pgfqpoint{1.501467in}{4.039768in}}%
\pgfpathlineto{\pgfqpoint{1.508705in}{4.047045in}}%
\pgfpathlineto{\pgfqpoint{1.515944in}{4.054284in}}%
\pgfpathlineto{\pgfqpoint{1.523182in}{4.061486in}}%
\pgfpathlineto{\pgfqpoint{1.530420in}{4.068651in}}%
\pgfpathlineto{\pgfqpoint{1.537658in}{4.075777in}}%
\pgfpathlineto{\pgfqpoint{1.544896in}{4.082866in}}%
\pgfpathlineto{\pgfqpoint{1.552134in}{4.089916in}}%
\pgfpathlineto{\pgfqpoint{1.559372in}{4.096928in}}%
\pgfpathlineto{\pgfqpoint{1.566610in}{4.103901in}}%
\pgfpathlineto{\pgfqpoint{1.573848in}{4.110835in}}%
\pgfpathlineto{\pgfqpoint{1.581086in}{4.117729in}}%
\pgfpathlineto{\pgfqpoint{1.588325in}{4.124584in}}%
\pgfpathlineto{\pgfqpoint{1.595563in}{4.131399in}}%
\pgfpathlineto{\pgfqpoint{1.602801in}{4.138175in}}%
\pgfpathlineto{\pgfqpoint{1.610039in}{4.144910in}}%
\pgfpathlineto{\pgfqpoint{1.617277in}{4.151604in}}%
\pgfpathlineto{\pgfqpoint{1.624515in}{4.158258in}}%
\pgfpathlineto{\pgfqpoint{1.631753in}{4.164871in}}%
\pgfpathlineto{\pgfqpoint{1.638991in}{4.171443in}}%
\pgfpathlineto{\pgfqpoint{1.646229in}{4.177973in}}%
\pgfpathlineto{\pgfqpoint{1.653467in}{4.184462in}}%
\pgfpathlineto{\pgfqpoint{1.660705in}{4.190909in}}%
\pgfpathlineto{\pgfqpoint{1.667944in}{4.197314in}}%
\pgfpathlineto{\pgfqpoint{1.675182in}{4.203677in}}%
\pgfpathlineto{\pgfqpoint{1.682420in}{4.209997in}}%
\pgfpathlineto{\pgfqpoint{1.689658in}{4.216275in}}%
\pgfpathlineto{\pgfqpoint{1.696896in}{4.222509in}}%
\pgfpathlineto{\pgfqpoint{1.704134in}{4.228701in}}%
\pgfpathlineto{\pgfqpoint{1.711372in}{4.234849in}}%
\pgfpathlineto{\pgfqpoint{1.718610in}{4.240953in}}%
\pgfpathlineto{\pgfqpoint{1.725848in}{4.247014in}}%
\pgfpathlineto{\pgfqpoint{1.733086in}{4.253031in}}%
\pgfpathlineto{\pgfqpoint{1.740325in}{4.259004in}}%
\pgfpathlineto{\pgfqpoint{1.747563in}{4.264932in}}%
\pgfpathlineto{\pgfqpoint{1.747563in}{4.253993in}}%
\pgfpathlineto{\pgfqpoint{1.747563in}{4.253993in}}%
\pgfpathlineto{\pgfqpoint{1.740325in}{4.253993in}}%
\pgfpathlineto{\pgfqpoint{1.733086in}{4.253993in}}%
\pgfpathlineto{\pgfqpoint{1.725848in}{4.253993in}}%
\pgfpathlineto{\pgfqpoint{1.718610in}{4.253993in}}%
\pgfpathlineto{\pgfqpoint{1.711372in}{4.253993in}}%
\pgfpathlineto{\pgfqpoint{1.704134in}{4.253993in}}%
\pgfpathlineto{\pgfqpoint{1.696896in}{4.253993in}}%
\pgfpathlineto{\pgfqpoint{1.689658in}{4.253993in}}%
\pgfpathlineto{\pgfqpoint{1.682420in}{4.253993in}}%
\pgfpathlineto{\pgfqpoint{1.675182in}{4.253993in}}%
\pgfpathlineto{\pgfqpoint{1.667944in}{4.253993in}}%
\pgfpathlineto{\pgfqpoint{1.660705in}{4.253993in}}%
\pgfpathlineto{\pgfqpoint{1.653467in}{4.253993in}}%
\pgfpathlineto{\pgfqpoint{1.646229in}{4.253993in}}%
\pgfpathlineto{\pgfqpoint{1.638991in}{4.253993in}}%
\pgfpathlineto{\pgfqpoint{1.631753in}{4.253993in}}%
\pgfpathlineto{\pgfqpoint{1.624515in}{4.253993in}}%
\pgfpathlineto{\pgfqpoint{1.617277in}{4.253993in}}%
\pgfpathlineto{\pgfqpoint{1.610039in}{4.253993in}}%
\pgfpathlineto{\pgfqpoint{1.602801in}{4.253993in}}%
\pgfpathlineto{\pgfqpoint{1.595563in}{4.253993in}}%
\pgfpathlineto{\pgfqpoint{1.588325in}{4.253993in}}%
\pgfpathlineto{\pgfqpoint{1.581086in}{4.253993in}}%
\pgfpathlineto{\pgfqpoint{1.573848in}{4.253993in}}%
\pgfpathlineto{\pgfqpoint{1.566610in}{4.253993in}}%
\pgfpathlineto{\pgfqpoint{1.559372in}{4.253993in}}%
\pgfpathlineto{\pgfqpoint{1.552134in}{4.253993in}}%
\pgfpathlineto{\pgfqpoint{1.544896in}{4.253993in}}%
\pgfpathlineto{\pgfqpoint{1.537658in}{4.253993in}}%
\pgfpathlineto{\pgfqpoint{1.530420in}{4.253993in}}%
\pgfpathlineto{\pgfqpoint{1.523182in}{4.253993in}}%
\pgfpathlineto{\pgfqpoint{1.515944in}{4.253993in}}%
\pgfpathlineto{\pgfqpoint{1.508705in}{4.253993in}}%
\pgfpathlineto{\pgfqpoint{1.501467in}{4.253993in}}%
\pgfpathlineto{\pgfqpoint{1.494229in}{4.253993in}}%
\pgfpathlineto{\pgfqpoint{1.486991in}{4.253993in}}%
\pgfpathlineto{\pgfqpoint{1.479753in}{4.253993in}}%
\pgfpathlineto{\pgfqpoint{1.472515in}{4.253993in}}%
\pgfpathlineto{\pgfqpoint{1.465277in}{4.253993in}}%
\pgfpathlineto{\pgfqpoint{1.458039in}{4.253993in}}%
\pgfpathlineto{\pgfqpoint{1.450801in}{4.253993in}}%
\pgfpathlineto{\pgfqpoint{1.443563in}{4.253993in}}%
\pgfpathlineto{\pgfqpoint{1.436324in}{4.253993in}}%
\pgfpathlineto{\pgfqpoint{1.429086in}{4.253993in}}%
\pgfpathlineto{\pgfqpoint{1.421848in}{4.253993in}}%
\pgfpathlineto{\pgfqpoint{1.414610in}{4.253993in}}%
\pgfpathlineto{\pgfqpoint{1.407372in}{4.253993in}}%
\pgfpathlineto{\pgfqpoint{1.400134in}{4.253993in}}%
\pgfpathlineto{\pgfqpoint{1.392896in}{4.253993in}}%
\pgfpathlineto{\pgfqpoint{1.392896in}{4.253993in}}%
\pgfpathclose%
\pgfusepath{stroke,fill}%
}%
\begin{pgfscope}%
\pgfsys@transformshift{0.000000in}{0.000000in}%
\pgfsys@useobject{currentmarker}{}%
\end{pgfscope}%
\end{pgfscope}%
\begin{pgfscope}%
\pgfpathrectangle{\pgfqpoint{0.625000in}{2.970000in}}{\pgfqpoint{3.875000in}{2.310000in}}%
\pgfusepath{clip}%
\pgfsetbuttcap%
\pgfsetroundjoin%
\definecolor{currentfill}{rgb}{0.900000,0.900000,0.900000}%
\pgfsetfillcolor{currentfill}%
\pgfsetlinewidth{1.003750pt}%
\definecolor{currentstroke}{rgb}{0.500000,0.500000,0.500000}%
\pgfsetstrokecolor{currentstroke}%
\pgfsetdash{}{0pt}%
\pgfsys@defobject{currentmarker}{\pgfqpoint{1.747563in}{4.253993in}}{\pgfqpoint{2.142267in}{4.515791in}}{%
\pgfpathmoveto{\pgfqpoint{1.747563in}{4.253993in}}%
\pgfpathlineto{\pgfqpoint{1.747563in}{4.264932in}}%
\pgfpathlineto{\pgfqpoint{1.755618in}{4.271477in}}%
\pgfpathlineto{\pgfqpoint{1.763673in}{4.277967in}}%
\pgfpathlineto{\pgfqpoint{1.771728in}{4.284401in}}%
\pgfpathlineto{\pgfqpoint{1.779783in}{4.290778in}}%
\pgfpathlineto{\pgfqpoint{1.787839in}{4.297100in}}%
\pgfpathlineto{\pgfqpoint{1.795894in}{4.303364in}}%
\pgfpathlineto{\pgfqpoint{1.803949in}{4.309572in}}%
\pgfpathlineto{\pgfqpoint{1.812004in}{4.315723in}}%
\pgfpathlineto{\pgfqpoint{1.820059in}{4.321817in}}%
\pgfpathlineto{\pgfqpoint{1.828115in}{4.327852in}}%
\pgfpathlineto{\pgfqpoint{1.836170in}{4.333830in}}%
\pgfpathlineto{\pgfqpoint{1.844225in}{4.339750in}}%
\pgfpathlineto{\pgfqpoint{1.852280in}{4.345611in}}%
\pgfpathlineto{\pgfqpoint{1.860335in}{4.351414in}}%
\pgfpathlineto{\pgfqpoint{1.868391in}{4.357157in}}%
\pgfpathlineto{\pgfqpoint{1.876446in}{4.362842in}}%
\pgfpathlineto{\pgfqpoint{1.884501in}{4.368467in}}%
\pgfpathlineto{\pgfqpoint{1.892556in}{4.374033in}}%
\pgfpathlineto{\pgfqpoint{1.900611in}{4.379538in}}%
\pgfpathlineto{\pgfqpoint{1.908666in}{4.384984in}}%
\pgfpathlineto{\pgfqpoint{1.916722in}{4.390369in}}%
\pgfpathlineto{\pgfqpoint{1.924777in}{4.395693in}}%
\pgfpathlineto{\pgfqpoint{1.932832in}{4.400957in}}%
\pgfpathlineto{\pgfqpoint{1.940887in}{4.406160in}}%
\pgfpathlineto{\pgfqpoint{1.948942in}{4.411301in}}%
\pgfpathlineto{\pgfqpoint{1.956998in}{4.416381in}}%
\pgfpathlineto{\pgfqpoint{1.965053in}{4.421400in}}%
\pgfpathlineto{\pgfqpoint{1.973108in}{4.426356in}}%
\pgfpathlineto{\pgfqpoint{1.981163in}{4.431250in}}%
\pgfpathlineto{\pgfqpoint{1.989218in}{4.436082in}}%
\pgfpathlineto{\pgfqpoint{1.997274in}{4.440852in}}%
\pgfpathlineto{\pgfqpoint{2.005329in}{4.445558in}}%
\pgfpathlineto{\pgfqpoint{2.013384in}{4.450202in}}%
\pgfpathlineto{\pgfqpoint{2.021439in}{4.454783in}}%
\pgfpathlineto{\pgfqpoint{2.029494in}{4.459300in}}%
\pgfpathlineto{\pgfqpoint{2.037549in}{4.463754in}}%
\pgfpathlineto{\pgfqpoint{2.045605in}{4.468144in}}%
\pgfpathlineto{\pgfqpoint{2.053660in}{4.472470in}}%
\pgfpathlineto{\pgfqpoint{2.061715in}{4.476733in}}%
\pgfpathlineto{\pgfqpoint{2.069770in}{4.480930in}}%
\pgfpathlineto{\pgfqpoint{2.077825in}{4.485064in}}%
\pgfpathlineto{\pgfqpoint{2.085881in}{4.489133in}}%
\pgfpathlineto{\pgfqpoint{2.093936in}{4.493137in}}%
\pgfpathlineto{\pgfqpoint{2.101991in}{4.497076in}}%
\pgfpathlineto{\pgfqpoint{2.110046in}{4.500950in}}%
\pgfpathlineto{\pgfqpoint{2.118101in}{4.504758in}}%
\pgfpathlineto{\pgfqpoint{2.126157in}{4.508502in}}%
\pgfpathlineto{\pgfqpoint{2.134212in}{4.512179in}}%
\pgfpathlineto{\pgfqpoint{2.142267in}{4.515791in}}%
\pgfpathlineto{\pgfqpoint{2.142267in}{4.253993in}}%
\pgfpathlineto{\pgfqpoint{2.142267in}{4.253993in}}%
\pgfpathlineto{\pgfqpoint{2.134212in}{4.253993in}}%
\pgfpathlineto{\pgfqpoint{2.126157in}{4.253993in}}%
\pgfpathlineto{\pgfqpoint{2.118101in}{4.253993in}}%
\pgfpathlineto{\pgfqpoint{2.110046in}{4.253993in}}%
\pgfpathlineto{\pgfqpoint{2.101991in}{4.253993in}}%
\pgfpathlineto{\pgfqpoint{2.093936in}{4.253993in}}%
\pgfpathlineto{\pgfqpoint{2.085881in}{4.253993in}}%
\pgfpathlineto{\pgfqpoint{2.077825in}{4.253993in}}%
\pgfpathlineto{\pgfqpoint{2.069770in}{4.253993in}}%
\pgfpathlineto{\pgfqpoint{2.061715in}{4.253993in}}%
\pgfpathlineto{\pgfqpoint{2.053660in}{4.253993in}}%
\pgfpathlineto{\pgfqpoint{2.045605in}{4.253993in}}%
\pgfpathlineto{\pgfqpoint{2.037549in}{4.253993in}}%
\pgfpathlineto{\pgfqpoint{2.029494in}{4.253993in}}%
\pgfpathlineto{\pgfqpoint{2.021439in}{4.253993in}}%
\pgfpathlineto{\pgfqpoint{2.013384in}{4.253993in}}%
\pgfpathlineto{\pgfqpoint{2.005329in}{4.253993in}}%
\pgfpathlineto{\pgfqpoint{1.997274in}{4.253993in}}%
\pgfpathlineto{\pgfqpoint{1.989218in}{4.253993in}}%
\pgfpathlineto{\pgfqpoint{1.981163in}{4.253993in}}%
\pgfpathlineto{\pgfqpoint{1.973108in}{4.253993in}}%
\pgfpathlineto{\pgfqpoint{1.965053in}{4.253993in}}%
\pgfpathlineto{\pgfqpoint{1.956998in}{4.253993in}}%
\pgfpathlineto{\pgfqpoint{1.948942in}{4.253993in}}%
\pgfpathlineto{\pgfqpoint{1.940887in}{4.253993in}}%
\pgfpathlineto{\pgfqpoint{1.932832in}{4.253993in}}%
\pgfpathlineto{\pgfqpoint{1.924777in}{4.253993in}}%
\pgfpathlineto{\pgfqpoint{1.916722in}{4.253993in}}%
\pgfpathlineto{\pgfqpoint{1.908666in}{4.253993in}}%
\pgfpathlineto{\pgfqpoint{1.900611in}{4.253993in}}%
\pgfpathlineto{\pgfqpoint{1.892556in}{4.253993in}}%
\pgfpathlineto{\pgfqpoint{1.884501in}{4.253993in}}%
\pgfpathlineto{\pgfqpoint{1.876446in}{4.253993in}}%
\pgfpathlineto{\pgfqpoint{1.868391in}{4.253993in}}%
\pgfpathlineto{\pgfqpoint{1.860335in}{4.253993in}}%
\pgfpathlineto{\pgfqpoint{1.852280in}{4.253993in}}%
\pgfpathlineto{\pgfqpoint{1.844225in}{4.253993in}}%
\pgfpathlineto{\pgfqpoint{1.836170in}{4.253993in}}%
\pgfpathlineto{\pgfqpoint{1.828115in}{4.253993in}}%
\pgfpathlineto{\pgfqpoint{1.820059in}{4.253993in}}%
\pgfpathlineto{\pgfqpoint{1.812004in}{4.253993in}}%
\pgfpathlineto{\pgfqpoint{1.803949in}{4.253993in}}%
\pgfpathlineto{\pgfqpoint{1.795894in}{4.253993in}}%
\pgfpathlineto{\pgfqpoint{1.787839in}{4.253993in}}%
\pgfpathlineto{\pgfqpoint{1.779783in}{4.253993in}}%
\pgfpathlineto{\pgfqpoint{1.771728in}{4.253993in}}%
\pgfpathlineto{\pgfqpoint{1.763673in}{4.253993in}}%
\pgfpathlineto{\pgfqpoint{1.755618in}{4.253993in}}%
\pgfpathlineto{\pgfqpoint{1.747563in}{4.253993in}}%
\pgfpathlineto{\pgfqpoint{1.747563in}{4.253993in}}%
\pgfpathclose%
\pgfusepath{stroke,fill}%
}%
\begin{pgfscope}%
\pgfsys@transformshift{0.000000in}{0.000000in}%
\pgfsys@useobject{currentmarker}{}%
\end{pgfscope}%
\end{pgfscope}%
\begin{pgfscope}%
\pgfpathrectangle{\pgfqpoint{0.625000in}{2.970000in}}{\pgfqpoint{3.875000in}{2.310000in}}%
\pgfusepath{clip}%
\pgfsetrectcap%
\pgfsetroundjoin%
\pgfsetlinewidth{0.803000pt}%
\definecolor{currentstroke}{rgb}{0.690196,0.690196,0.690196}%
\pgfsetstrokecolor{currentstroke}%
\pgfsetdash{}{0pt}%
\pgfpathmoveto{\pgfqpoint{0.625000in}{2.970000in}}%
\pgfpathlineto{\pgfqpoint{0.625000in}{5.280000in}}%
\pgfusepath{stroke}%
\end{pgfscope}%
\begin{pgfscope}%
\pgfsetbuttcap%
\pgfsetroundjoin%
\definecolor{currentfill}{rgb}{0.000000,0.000000,0.000000}%
\pgfsetfillcolor{currentfill}%
\pgfsetlinewidth{0.803000pt}%
\definecolor{currentstroke}{rgb}{0.000000,0.000000,0.000000}%
\pgfsetstrokecolor{currentstroke}%
\pgfsetdash{}{0pt}%
\pgfsys@defobject{currentmarker}{\pgfqpoint{0.000000in}{-0.048611in}}{\pgfqpoint{0.000000in}{0.000000in}}{%
\pgfpathmoveto{\pgfqpoint{0.000000in}{0.000000in}}%
\pgfpathlineto{\pgfqpoint{0.000000in}{-0.048611in}}%
\pgfusepath{stroke,fill}%
}%
\begin{pgfscope}%
\pgfsys@transformshift{0.625000in}{2.970000in}%
\pgfsys@useobject{currentmarker}{}%
\end{pgfscope}%
\end{pgfscope}%
\begin{pgfscope}%
\pgfpathrectangle{\pgfqpoint{0.625000in}{2.970000in}}{\pgfqpoint{3.875000in}{2.310000in}}%
\pgfusepath{clip}%
\pgfsetrectcap%
\pgfsetroundjoin%
\pgfsetlinewidth{0.803000pt}%
\definecolor{currentstroke}{rgb}{0.690196,0.690196,0.690196}%
\pgfsetstrokecolor{currentstroke}%
\pgfsetdash{}{0pt}%
\pgfpathmoveto{\pgfqpoint{1.055556in}{2.970000in}}%
\pgfpathlineto{\pgfqpoint{1.055556in}{5.280000in}}%
\pgfusepath{stroke}%
\end{pgfscope}%
\begin{pgfscope}%
\pgfsetbuttcap%
\pgfsetroundjoin%
\definecolor{currentfill}{rgb}{0.000000,0.000000,0.000000}%
\pgfsetfillcolor{currentfill}%
\pgfsetlinewidth{0.803000pt}%
\definecolor{currentstroke}{rgb}{0.000000,0.000000,0.000000}%
\pgfsetstrokecolor{currentstroke}%
\pgfsetdash{}{0pt}%
\pgfsys@defobject{currentmarker}{\pgfqpoint{0.000000in}{-0.048611in}}{\pgfqpoint{0.000000in}{0.000000in}}{%
\pgfpathmoveto{\pgfqpoint{0.000000in}{0.000000in}}%
\pgfpathlineto{\pgfqpoint{0.000000in}{-0.048611in}}%
\pgfusepath{stroke,fill}%
}%
\begin{pgfscope}%
\pgfsys@transformshift{1.055556in}{2.970000in}%
\pgfsys@useobject{currentmarker}{}%
\end{pgfscope}%
\end{pgfscope}%
\begin{pgfscope}%
\pgfpathrectangle{\pgfqpoint{0.625000in}{2.970000in}}{\pgfqpoint{3.875000in}{2.310000in}}%
\pgfusepath{clip}%
\pgfsetrectcap%
\pgfsetroundjoin%
\pgfsetlinewidth{0.803000pt}%
\definecolor{currentstroke}{rgb}{0.690196,0.690196,0.690196}%
\pgfsetstrokecolor{currentstroke}%
\pgfsetdash{}{0pt}%
\pgfpathmoveto{\pgfqpoint{1.486111in}{2.970000in}}%
\pgfpathlineto{\pgfqpoint{1.486111in}{5.280000in}}%
\pgfusepath{stroke}%
\end{pgfscope}%
\begin{pgfscope}%
\pgfsetbuttcap%
\pgfsetroundjoin%
\definecolor{currentfill}{rgb}{0.000000,0.000000,0.000000}%
\pgfsetfillcolor{currentfill}%
\pgfsetlinewidth{0.803000pt}%
\definecolor{currentstroke}{rgb}{0.000000,0.000000,0.000000}%
\pgfsetstrokecolor{currentstroke}%
\pgfsetdash{}{0pt}%
\pgfsys@defobject{currentmarker}{\pgfqpoint{0.000000in}{-0.048611in}}{\pgfqpoint{0.000000in}{0.000000in}}{%
\pgfpathmoveto{\pgfqpoint{0.000000in}{0.000000in}}%
\pgfpathlineto{\pgfqpoint{0.000000in}{-0.048611in}}%
\pgfusepath{stroke,fill}%
}%
\begin{pgfscope}%
\pgfsys@transformshift{1.486111in}{2.970000in}%
\pgfsys@useobject{currentmarker}{}%
\end{pgfscope}%
\end{pgfscope}%
\begin{pgfscope}%
\pgfpathrectangle{\pgfqpoint{0.625000in}{2.970000in}}{\pgfqpoint{3.875000in}{2.310000in}}%
\pgfusepath{clip}%
\pgfsetrectcap%
\pgfsetroundjoin%
\pgfsetlinewidth{0.803000pt}%
\definecolor{currentstroke}{rgb}{0.690196,0.690196,0.690196}%
\pgfsetstrokecolor{currentstroke}%
\pgfsetdash{}{0pt}%
\pgfpathmoveto{\pgfqpoint{1.916667in}{2.970000in}}%
\pgfpathlineto{\pgfqpoint{1.916667in}{5.280000in}}%
\pgfusepath{stroke}%
\end{pgfscope}%
\begin{pgfscope}%
\pgfsetbuttcap%
\pgfsetroundjoin%
\definecolor{currentfill}{rgb}{0.000000,0.000000,0.000000}%
\pgfsetfillcolor{currentfill}%
\pgfsetlinewidth{0.803000pt}%
\definecolor{currentstroke}{rgb}{0.000000,0.000000,0.000000}%
\pgfsetstrokecolor{currentstroke}%
\pgfsetdash{}{0pt}%
\pgfsys@defobject{currentmarker}{\pgfqpoint{0.000000in}{-0.048611in}}{\pgfqpoint{0.000000in}{0.000000in}}{%
\pgfpathmoveto{\pgfqpoint{0.000000in}{0.000000in}}%
\pgfpathlineto{\pgfqpoint{0.000000in}{-0.048611in}}%
\pgfusepath{stroke,fill}%
}%
\begin{pgfscope}%
\pgfsys@transformshift{1.916667in}{2.970000in}%
\pgfsys@useobject{currentmarker}{}%
\end{pgfscope}%
\end{pgfscope}%
\begin{pgfscope}%
\pgfpathrectangle{\pgfqpoint{0.625000in}{2.970000in}}{\pgfqpoint{3.875000in}{2.310000in}}%
\pgfusepath{clip}%
\pgfsetrectcap%
\pgfsetroundjoin%
\pgfsetlinewidth{0.803000pt}%
\definecolor{currentstroke}{rgb}{0.690196,0.690196,0.690196}%
\pgfsetstrokecolor{currentstroke}%
\pgfsetdash{}{0pt}%
\pgfpathmoveto{\pgfqpoint{2.347222in}{2.970000in}}%
\pgfpathlineto{\pgfqpoint{2.347222in}{5.280000in}}%
\pgfusepath{stroke}%
\end{pgfscope}%
\begin{pgfscope}%
\pgfsetbuttcap%
\pgfsetroundjoin%
\definecolor{currentfill}{rgb}{0.000000,0.000000,0.000000}%
\pgfsetfillcolor{currentfill}%
\pgfsetlinewidth{0.803000pt}%
\definecolor{currentstroke}{rgb}{0.000000,0.000000,0.000000}%
\pgfsetstrokecolor{currentstroke}%
\pgfsetdash{}{0pt}%
\pgfsys@defobject{currentmarker}{\pgfqpoint{0.000000in}{-0.048611in}}{\pgfqpoint{0.000000in}{0.000000in}}{%
\pgfpathmoveto{\pgfqpoint{0.000000in}{0.000000in}}%
\pgfpathlineto{\pgfqpoint{0.000000in}{-0.048611in}}%
\pgfusepath{stroke,fill}%
}%
\begin{pgfscope}%
\pgfsys@transformshift{2.347222in}{2.970000in}%
\pgfsys@useobject{currentmarker}{}%
\end{pgfscope}%
\end{pgfscope}%
\begin{pgfscope}%
\pgfpathrectangle{\pgfqpoint{0.625000in}{2.970000in}}{\pgfqpoint{3.875000in}{2.310000in}}%
\pgfusepath{clip}%
\pgfsetrectcap%
\pgfsetroundjoin%
\pgfsetlinewidth{0.803000pt}%
\definecolor{currentstroke}{rgb}{0.690196,0.690196,0.690196}%
\pgfsetstrokecolor{currentstroke}%
\pgfsetdash{}{0pt}%
\pgfpathmoveto{\pgfqpoint{2.777778in}{2.970000in}}%
\pgfpathlineto{\pgfqpoint{2.777778in}{5.280000in}}%
\pgfusepath{stroke}%
\end{pgfscope}%
\begin{pgfscope}%
\pgfsetbuttcap%
\pgfsetroundjoin%
\definecolor{currentfill}{rgb}{0.000000,0.000000,0.000000}%
\pgfsetfillcolor{currentfill}%
\pgfsetlinewidth{0.803000pt}%
\definecolor{currentstroke}{rgb}{0.000000,0.000000,0.000000}%
\pgfsetstrokecolor{currentstroke}%
\pgfsetdash{}{0pt}%
\pgfsys@defobject{currentmarker}{\pgfqpoint{0.000000in}{-0.048611in}}{\pgfqpoint{0.000000in}{0.000000in}}{%
\pgfpathmoveto{\pgfqpoint{0.000000in}{0.000000in}}%
\pgfpathlineto{\pgfqpoint{0.000000in}{-0.048611in}}%
\pgfusepath{stroke,fill}%
}%
\begin{pgfscope}%
\pgfsys@transformshift{2.777778in}{2.970000in}%
\pgfsys@useobject{currentmarker}{}%
\end{pgfscope}%
\end{pgfscope}%
\begin{pgfscope}%
\pgfpathrectangle{\pgfqpoint{0.625000in}{2.970000in}}{\pgfqpoint{3.875000in}{2.310000in}}%
\pgfusepath{clip}%
\pgfsetrectcap%
\pgfsetroundjoin%
\pgfsetlinewidth{0.803000pt}%
\definecolor{currentstroke}{rgb}{0.690196,0.690196,0.690196}%
\pgfsetstrokecolor{currentstroke}%
\pgfsetdash{}{0pt}%
\pgfpathmoveto{\pgfqpoint{3.208333in}{2.970000in}}%
\pgfpathlineto{\pgfqpoint{3.208333in}{5.280000in}}%
\pgfusepath{stroke}%
\end{pgfscope}%
\begin{pgfscope}%
\pgfsetbuttcap%
\pgfsetroundjoin%
\definecolor{currentfill}{rgb}{0.000000,0.000000,0.000000}%
\pgfsetfillcolor{currentfill}%
\pgfsetlinewidth{0.803000pt}%
\definecolor{currentstroke}{rgb}{0.000000,0.000000,0.000000}%
\pgfsetstrokecolor{currentstroke}%
\pgfsetdash{}{0pt}%
\pgfsys@defobject{currentmarker}{\pgfqpoint{0.000000in}{-0.048611in}}{\pgfqpoint{0.000000in}{0.000000in}}{%
\pgfpathmoveto{\pgfqpoint{0.000000in}{0.000000in}}%
\pgfpathlineto{\pgfqpoint{0.000000in}{-0.048611in}}%
\pgfusepath{stroke,fill}%
}%
\begin{pgfscope}%
\pgfsys@transformshift{3.208333in}{2.970000in}%
\pgfsys@useobject{currentmarker}{}%
\end{pgfscope}%
\end{pgfscope}%
\begin{pgfscope}%
\pgfpathrectangle{\pgfqpoint{0.625000in}{2.970000in}}{\pgfqpoint{3.875000in}{2.310000in}}%
\pgfusepath{clip}%
\pgfsetrectcap%
\pgfsetroundjoin%
\pgfsetlinewidth{0.803000pt}%
\definecolor{currentstroke}{rgb}{0.690196,0.690196,0.690196}%
\pgfsetstrokecolor{currentstroke}%
\pgfsetdash{}{0pt}%
\pgfpathmoveto{\pgfqpoint{3.638889in}{2.970000in}}%
\pgfpathlineto{\pgfqpoint{3.638889in}{5.280000in}}%
\pgfusepath{stroke}%
\end{pgfscope}%
\begin{pgfscope}%
\pgfsetbuttcap%
\pgfsetroundjoin%
\definecolor{currentfill}{rgb}{0.000000,0.000000,0.000000}%
\pgfsetfillcolor{currentfill}%
\pgfsetlinewidth{0.803000pt}%
\definecolor{currentstroke}{rgb}{0.000000,0.000000,0.000000}%
\pgfsetstrokecolor{currentstroke}%
\pgfsetdash{}{0pt}%
\pgfsys@defobject{currentmarker}{\pgfqpoint{0.000000in}{-0.048611in}}{\pgfqpoint{0.000000in}{0.000000in}}{%
\pgfpathmoveto{\pgfqpoint{0.000000in}{0.000000in}}%
\pgfpathlineto{\pgfqpoint{0.000000in}{-0.048611in}}%
\pgfusepath{stroke,fill}%
}%
\begin{pgfscope}%
\pgfsys@transformshift{3.638889in}{2.970000in}%
\pgfsys@useobject{currentmarker}{}%
\end{pgfscope}%
\end{pgfscope}%
\begin{pgfscope}%
\pgfpathrectangle{\pgfqpoint{0.625000in}{2.970000in}}{\pgfqpoint{3.875000in}{2.310000in}}%
\pgfusepath{clip}%
\pgfsetrectcap%
\pgfsetroundjoin%
\pgfsetlinewidth{0.803000pt}%
\definecolor{currentstroke}{rgb}{0.690196,0.690196,0.690196}%
\pgfsetstrokecolor{currentstroke}%
\pgfsetdash{}{0pt}%
\pgfpathmoveto{\pgfqpoint{4.069444in}{2.970000in}}%
\pgfpathlineto{\pgfqpoint{4.069444in}{5.280000in}}%
\pgfusepath{stroke}%
\end{pgfscope}%
\begin{pgfscope}%
\pgfsetbuttcap%
\pgfsetroundjoin%
\definecolor{currentfill}{rgb}{0.000000,0.000000,0.000000}%
\pgfsetfillcolor{currentfill}%
\pgfsetlinewidth{0.803000pt}%
\definecolor{currentstroke}{rgb}{0.000000,0.000000,0.000000}%
\pgfsetstrokecolor{currentstroke}%
\pgfsetdash{}{0pt}%
\pgfsys@defobject{currentmarker}{\pgfqpoint{0.000000in}{-0.048611in}}{\pgfqpoint{0.000000in}{0.000000in}}{%
\pgfpathmoveto{\pgfqpoint{0.000000in}{0.000000in}}%
\pgfpathlineto{\pgfqpoint{0.000000in}{-0.048611in}}%
\pgfusepath{stroke,fill}%
}%
\begin{pgfscope}%
\pgfsys@transformshift{4.069444in}{2.970000in}%
\pgfsys@useobject{currentmarker}{}%
\end{pgfscope}%
\end{pgfscope}%
\begin{pgfscope}%
\pgfpathrectangle{\pgfqpoint{0.625000in}{2.970000in}}{\pgfqpoint{3.875000in}{2.310000in}}%
\pgfusepath{clip}%
\pgfsetrectcap%
\pgfsetroundjoin%
\pgfsetlinewidth{0.803000pt}%
\definecolor{currentstroke}{rgb}{0.690196,0.690196,0.690196}%
\pgfsetstrokecolor{currentstroke}%
\pgfsetdash{}{0pt}%
\pgfpathmoveto{\pgfqpoint{4.500000in}{2.970000in}}%
\pgfpathlineto{\pgfqpoint{4.500000in}{5.280000in}}%
\pgfusepath{stroke}%
\end{pgfscope}%
\begin{pgfscope}%
\pgfsetbuttcap%
\pgfsetroundjoin%
\definecolor{currentfill}{rgb}{0.000000,0.000000,0.000000}%
\pgfsetfillcolor{currentfill}%
\pgfsetlinewidth{0.803000pt}%
\definecolor{currentstroke}{rgb}{0.000000,0.000000,0.000000}%
\pgfsetstrokecolor{currentstroke}%
\pgfsetdash{}{0pt}%
\pgfsys@defobject{currentmarker}{\pgfqpoint{0.000000in}{-0.048611in}}{\pgfqpoint{0.000000in}{0.000000in}}{%
\pgfpathmoveto{\pgfqpoint{0.000000in}{0.000000in}}%
\pgfpathlineto{\pgfqpoint{0.000000in}{-0.048611in}}%
\pgfusepath{stroke,fill}%
}%
\begin{pgfscope}%
\pgfsys@transformshift{4.500000in}{2.970000in}%
\pgfsys@useobject{currentmarker}{}%
\end{pgfscope}%
\end{pgfscope}%
\begin{pgfscope}%
\pgfpathrectangle{\pgfqpoint{0.625000in}{2.970000in}}{\pgfqpoint{3.875000in}{2.310000in}}%
\pgfusepath{clip}%
\pgfsetrectcap%
\pgfsetroundjoin%
\pgfsetlinewidth{0.803000pt}%
\definecolor{currentstroke}{rgb}{0.690196,0.690196,0.690196}%
\pgfsetstrokecolor{currentstroke}%
\pgfsetdash{}{0pt}%
\pgfpathmoveto{\pgfqpoint{0.625000in}{2.970000in}}%
\pgfpathlineto{\pgfqpoint{4.500000in}{2.970000in}}%
\pgfusepath{stroke}%
\end{pgfscope}%
\begin{pgfscope}%
\pgfsetbuttcap%
\pgfsetroundjoin%
\definecolor{currentfill}{rgb}{0.000000,0.000000,0.000000}%
\pgfsetfillcolor{currentfill}%
\pgfsetlinewidth{0.803000pt}%
\definecolor{currentstroke}{rgb}{0.000000,0.000000,0.000000}%
\pgfsetstrokecolor{currentstroke}%
\pgfsetdash{}{0pt}%
\pgfsys@defobject{currentmarker}{\pgfqpoint{-0.048611in}{0.000000in}}{\pgfqpoint{-0.000000in}{0.000000in}}{%
\pgfpathmoveto{\pgfqpoint{-0.000000in}{0.000000in}}%
\pgfpathlineto{\pgfqpoint{-0.048611in}{0.000000in}}%
\pgfusepath{stroke,fill}%
}%
\begin{pgfscope}%
\pgfsys@transformshift{0.625000in}{2.970000in}%
\pgfsys@useobject{currentmarker}{}%
\end{pgfscope}%
\end{pgfscope}%
\begin{pgfscope}%
\definecolor{textcolor}{rgb}{0.000000,0.000000,0.000000}%
\pgfsetstrokecolor{textcolor}%
\pgfsetfillcolor{textcolor}%
\pgftext[x=0.350308in, y=2.918900in, left, base]{\color{textcolor}\rmfamily\fontsize{10.000000}{12.000000}\selectfont \(\displaystyle {0.0}\)}%
\end{pgfscope}%
\begin{pgfscope}%
\pgfpathrectangle{\pgfqpoint{0.625000in}{2.970000in}}{\pgfqpoint{3.875000in}{2.310000in}}%
\pgfusepath{clip}%
\pgfsetrectcap%
\pgfsetroundjoin%
\pgfsetlinewidth{0.803000pt}%
\definecolor{currentstroke}{rgb}{0.690196,0.690196,0.690196}%
\pgfsetstrokecolor{currentstroke}%
\pgfsetdash{}{0pt}%
\pgfpathmoveto{\pgfqpoint{0.625000in}{3.336855in}}%
\pgfpathlineto{\pgfqpoint{4.500000in}{3.336855in}}%
\pgfusepath{stroke}%
\end{pgfscope}%
\begin{pgfscope}%
\pgfsetbuttcap%
\pgfsetroundjoin%
\definecolor{currentfill}{rgb}{0.000000,0.000000,0.000000}%
\pgfsetfillcolor{currentfill}%
\pgfsetlinewidth{0.803000pt}%
\definecolor{currentstroke}{rgb}{0.000000,0.000000,0.000000}%
\pgfsetstrokecolor{currentstroke}%
\pgfsetdash{}{0pt}%
\pgfsys@defobject{currentmarker}{\pgfqpoint{-0.048611in}{0.000000in}}{\pgfqpoint{-0.000000in}{0.000000in}}{%
\pgfpathmoveto{\pgfqpoint{-0.000000in}{0.000000in}}%
\pgfpathlineto{\pgfqpoint{-0.048611in}{0.000000in}}%
\pgfusepath{stroke,fill}%
}%
\begin{pgfscope}%
\pgfsys@transformshift{0.625000in}{3.336855in}%
\pgfsys@useobject{currentmarker}{}%
\end{pgfscope}%
\end{pgfscope}%
\begin{pgfscope}%
\definecolor{textcolor}{rgb}{0.000000,0.000000,0.000000}%
\pgfsetstrokecolor{textcolor}%
\pgfsetfillcolor{textcolor}%
\pgftext[x=0.350308in, y=3.285755in, left, base]{\color{textcolor}\rmfamily\fontsize{10.000000}{12.000000}\selectfont \(\displaystyle {0.2}\)}%
\end{pgfscope}%
\begin{pgfscope}%
\pgfpathrectangle{\pgfqpoint{0.625000in}{2.970000in}}{\pgfqpoint{3.875000in}{2.310000in}}%
\pgfusepath{clip}%
\pgfsetrectcap%
\pgfsetroundjoin%
\pgfsetlinewidth{0.803000pt}%
\definecolor{currentstroke}{rgb}{0.690196,0.690196,0.690196}%
\pgfsetstrokecolor{currentstroke}%
\pgfsetdash{}{0pt}%
\pgfpathmoveto{\pgfqpoint{0.625000in}{3.703710in}}%
\pgfpathlineto{\pgfqpoint{4.500000in}{3.703710in}}%
\pgfusepath{stroke}%
\end{pgfscope}%
\begin{pgfscope}%
\pgfsetbuttcap%
\pgfsetroundjoin%
\definecolor{currentfill}{rgb}{0.000000,0.000000,0.000000}%
\pgfsetfillcolor{currentfill}%
\pgfsetlinewidth{0.803000pt}%
\definecolor{currentstroke}{rgb}{0.000000,0.000000,0.000000}%
\pgfsetstrokecolor{currentstroke}%
\pgfsetdash{}{0pt}%
\pgfsys@defobject{currentmarker}{\pgfqpoint{-0.048611in}{0.000000in}}{\pgfqpoint{-0.000000in}{0.000000in}}{%
\pgfpathmoveto{\pgfqpoint{-0.000000in}{0.000000in}}%
\pgfpathlineto{\pgfqpoint{-0.048611in}{0.000000in}}%
\pgfusepath{stroke,fill}%
}%
\begin{pgfscope}%
\pgfsys@transformshift{0.625000in}{3.703710in}%
\pgfsys@useobject{currentmarker}{}%
\end{pgfscope}%
\end{pgfscope}%
\begin{pgfscope}%
\definecolor{textcolor}{rgb}{0.000000,0.000000,0.000000}%
\pgfsetstrokecolor{textcolor}%
\pgfsetfillcolor{textcolor}%
\pgftext[x=0.350308in, y=3.652610in, left, base]{\color{textcolor}\rmfamily\fontsize{10.000000}{12.000000}\selectfont \(\displaystyle {0.4}\)}%
\end{pgfscope}%
\begin{pgfscope}%
\pgfpathrectangle{\pgfqpoint{0.625000in}{2.970000in}}{\pgfqpoint{3.875000in}{2.310000in}}%
\pgfusepath{clip}%
\pgfsetrectcap%
\pgfsetroundjoin%
\pgfsetlinewidth{0.803000pt}%
\definecolor{currentstroke}{rgb}{0.690196,0.690196,0.690196}%
\pgfsetstrokecolor{currentstroke}%
\pgfsetdash{}{0pt}%
\pgfpathmoveto{\pgfqpoint{0.625000in}{4.070565in}}%
\pgfpathlineto{\pgfqpoint{4.500000in}{4.070565in}}%
\pgfusepath{stroke}%
\end{pgfscope}%
\begin{pgfscope}%
\pgfsetbuttcap%
\pgfsetroundjoin%
\definecolor{currentfill}{rgb}{0.000000,0.000000,0.000000}%
\pgfsetfillcolor{currentfill}%
\pgfsetlinewidth{0.803000pt}%
\definecolor{currentstroke}{rgb}{0.000000,0.000000,0.000000}%
\pgfsetstrokecolor{currentstroke}%
\pgfsetdash{}{0pt}%
\pgfsys@defobject{currentmarker}{\pgfqpoint{-0.048611in}{0.000000in}}{\pgfqpoint{-0.000000in}{0.000000in}}{%
\pgfpathmoveto{\pgfqpoint{-0.000000in}{0.000000in}}%
\pgfpathlineto{\pgfqpoint{-0.048611in}{0.000000in}}%
\pgfusepath{stroke,fill}%
}%
\begin{pgfscope}%
\pgfsys@transformshift{0.625000in}{4.070565in}%
\pgfsys@useobject{currentmarker}{}%
\end{pgfscope}%
\end{pgfscope}%
\begin{pgfscope}%
\definecolor{textcolor}{rgb}{0.000000,0.000000,0.000000}%
\pgfsetstrokecolor{textcolor}%
\pgfsetfillcolor{textcolor}%
\pgftext[x=0.350308in, y=4.019465in, left, base]{\color{textcolor}\rmfamily\fontsize{10.000000}{12.000000}\selectfont \(\displaystyle {0.6}\)}%
\end{pgfscope}%
\begin{pgfscope}%
\pgfpathrectangle{\pgfqpoint{0.625000in}{2.970000in}}{\pgfqpoint{3.875000in}{2.310000in}}%
\pgfusepath{clip}%
\pgfsetrectcap%
\pgfsetroundjoin%
\pgfsetlinewidth{0.803000pt}%
\definecolor{currentstroke}{rgb}{0.690196,0.690196,0.690196}%
\pgfsetstrokecolor{currentstroke}%
\pgfsetdash{}{0pt}%
\pgfpathmoveto{\pgfqpoint{0.625000in}{4.437421in}}%
\pgfpathlineto{\pgfqpoint{4.500000in}{4.437421in}}%
\pgfusepath{stroke}%
\end{pgfscope}%
\begin{pgfscope}%
\pgfsetbuttcap%
\pgfsetroundjoin%
\definecolor{currentfill}{rgb}{0.000000,0.000000,0.000000}%
\pgfsetfillcolor{currentfill}%
\pgfsetlinewidth{0.803000pt}%
\definecolor{currentstroke}{rgb}{0.000000,0.000000,0.000000}%
\pgfsetstrokecolor{currentstroke}%
\pgfsetdash{}{0pt}%
\pgfsys@defobject{currentmarker}{\pgfqpoint{-0.048611in}{0.000000in}}{\pgfqpoint{-0.000000in}{0.000000in}}{%
\pgfpathmoveto{\pgfqpoint{-0.000000in}{0.000000in}}%
\pgfpathlineto{\pgfqpoint{-0.048611in}{0.000000in}}%
\pgfusepath{stroke,fill}%
}%
\begin{pgfscope}%
\pgfsys@transformshift{0.625000in}{4.437421in}%
\pgfsys@useobject{currentmarker}{}%
\end{pgfscope}%
\end{pgfscope}%
\begin{pgfscope}%
\definecolor{textcolor}{rgb}{0.000000,0.000000,0.000000}%
\pgfsetstrokecolor{textcolor}%
\pgfsetfillcolor{textcolor}%
\pgftext[x=0.350308in, y=4.386321in, left, base]{\color{textcolor}\rmfamily\fontsize{10.000000}{12.000000}\selectfont \(\displaystyle {0.8}\)}%
\end{pgfscope}%
\begin{pgfscope}%
\pgfpathrectangle{\pgfqpoint{0.625000in}{2.970000in}}{\pgfqpoint{3.875000in}{2.310000in}}%
\pgfusepath{clip}%
\pgfsetrectcap%
\pgfsetroundjoin%
\pgfsetlinewidth{0.803000pt}%
\definecolor{currentstroke}{rgb}{0.690196,0.690196,0.690196}%
\pgfsetstrokecolor{currentstroke}%
\pgfsetdash{}{0pt}%
\pgfpathmoveto{\pgfqpoint{0.625000in}{4.804276in}}%
\pgfpathlineto{\pgfqpoint{4.500000in}{4.804276in}}%
\pgfusepath{stroke}%
\end{pgfscope}%
\begin{pgfscope}%
\pgfsetbuttcap%
\pgfsetroundjoin%
\definecolor{currentfill}{rgb}{0.000000,0.000000,0.000000}%
\pgfsetfillcolor{currentfill}%
\pgfsetlinewidth{0.803000pt}%
\definecolor{currentstroke}{rgb}{0.000000,0.000000,0.000000}%
\pgfsetstrokecolor{currentstroke}%
\pgfsetdash{}{0pt}%
\pgfsys@defobject{currentmarker}{\pgfqpoint{-0.048611in}{0.000000in}}{\pgfqpoint{-0.000000in}{0.000000in}}{%
\pgfpathmoveto{\pgfqpoint{-0.000000in}{0.000000in}}%
\pgfpathlineto{\pgfqpoint{-0.048611in}{0.000000in}}%
\pgfusepath{stroke,fill}%
}%
\begin{pgfscope}%
\pgfsys@transformshift{0.625000in}{4.804276in}%
\pgfsys@useobject{currentmarker}{}%
\end{pgfscope}%
\end{pgfscope}%
\begin{pgfscope}%
\definecolor{textcolor}{rgb}{0.000000,0.000000,0.000000}%
\pgfsetstrokecolor{textcolor}%
\pgfsetfillcolor{textcolor}%
\pgftext[x=0.350308in, y=4.753176in, left, base]{\color{textcolor}\rmfamily\fontsize{10.000000}{12.000000}\selectfont \(\displaystyle {1.0}\)}%
\end{pgfscope}%
\begin{pgfscope}%
\pgfpathrectangle{\pgfqpoint{0.625000in}{2.970000in}}{\pgfqpoint{3.875000in}{2.310000in}}%
\pgfusepath{clip}%
\pgfsetrectcap%
\pgfsetroundjoin%
\pgfsetlinewidth{0.803000pt}%
\definecolor{currentstroke}{rgb}{0.690196,0.690196,0.690196}%
\pgfsetstrokecolor{currentstroke}%
\pgfsetdash{}{0pt}%
\pgfpathmoveto{\pgfqpoint{0.625000in}{5.171131in}}%
\pgfpathlineto{\pgfqpoint{4.500000in}{5.171131in}}%
\pgfusepath{stroke}%
\end{pgfscope}%
\begin{pgfscope}%
\pgfsetbuttcap%
\pgfsetroundjoin%
\definecolor{currentfill}{rgb}{0.000000,0.000000,0.000000}%
\pgfsetfillcolor{currentfill}%
\pgfsetlinewidth{0.803000pt}%
\definecolor{currentstroke}{rgb}{0.000000,0.000000,0.000000}%
\pgfsetstrokecolor{currentstroke}%
\pgfsetdash{}{0pt}%
\pgfsys@defobject{currentmarker}{\pgfqpoint{-0.048611in}{0.000000in}}{\pgfqpoint{-0.000000in}{0.000000in}}{%
\pgfpathmoveto{\pgfqpoint{-0.000000in}{0.000000in}}%
\pgfpathlineto{\pgfqpoint{-0.048611in}{0.000000in}}%
\pgfusepath{stroke,fill}%
}%
\begin{pgfscope}%
\pgfsys@transformshift{0.625000in}{5.171131in}%
\pgfsys@useobject{currentmarker}{}%
\end{pgfscope}%
\end{pgfscope}%
\begin{pgfscope}%
\definecolor{textcolor}{rgb}{0.000000,0.000000,0.000000}%
\pgfsetstrokecolor{textcolor}%
\pgfsetfillcolor{textcolor}%
\pgftext[x=0.350308in, y=5.120031in, left, base]{\color{textcolor}\rmfamily\fontsize{10.000000}{12.000000}\selectfont \(\displaystyle {1.2}\)}%
\end{pgfscope}%
\begin{pgfscope}%
\definecolor{textcolor}{rgb}{0.000000,0.000000,0.000000}%
\pgfsetstrokecolor{textcolor}%
\pgfsetfillcolor{textcolor}%
\pgftext[x=0.294753in,y=4.125000in,,bottom,rotate=90.000000]{\color{textcolor}\rmfamily\fontsize{10.000000}{12.000000}\selectfont power in p.u.}%
\end{pgfscope}%
\begin{pgfscope}%
\pgfpathrectangle{\pgfqpoint{0.625000in}{2.970000in}}{\pgfqpoint{3.875000in}{2.310000in}}%
\pgfusepath{clip}%
\pgfsetrectcap%
\pgfsetroundjoin%
\pgfsetlinewidth{2.007500pt}%
\definecolor{currentstroke}{rgb}{0.121569,0.466667,0.705882}%
\pgfsetstrokecolor{currentstroke}%
\pgfsetdash{}{0pt}%
\pgfpathmoveto{\pgfqpoint{0.625000in}{2.970000in}}%
\pgfpathlineto{\pgfqpoint{0.704082in}{3.111027in}}%
\pgfpathlineto{\pgfqpoint{0.783163in}{3.251474in}}%
\pgfpathlineto{\pgfqpoint{0.862245in}{3.390765in}}%
\pgfpathlineto{\pgfqpoint{0.941327in}{3.528327in}}%
\pgfpathlineto{\pgfqpoint{1.020408in}{3.663594in}}%
\pgfpathlineto{\pgfqpoint{1.099490in}{3.796012in}}%
\pgfpathlineto{\pgfqpoint{1.178571in}{3.925035in}}%
\pgfpathlineto{\pgfqpoint{1.257653in}{4.050134in}}%
\pgfpathlineto{\pgfqpoint{1.336735in}{4.170794in}}%
\pgfpathlineto{\pgfqpoint{1.415816in}{4.286520in}}%
\pgfpathlineto{\pgfqpoint{1.494898in}{4.396836in}}%
\pgfpathlineto{\pgfqpoint{1.573980in}{4.501288in}}%
\pgfpathlineto{\pgfqpoint{1.653061in}{4.599449in}}%
\pgfpathlineto{\pgfqpoint{1.732143in}{4.690913in}}%
\pgfpathlineto{\pgfqpoint{1.811224in}{4.775306in}}%
\pgfpathlineto{\pgfqpoint{1.890306in}{4.852281in}}%
\pgfpathlineto{\pgfqpoint{1.969388in}{4.921521in}}%
\pgfpathlineto{\pgfqpoint{2.048469in}{4.982742in}}%
\pgfpathlineto{\pgfqpoint{2.127551in}{5.035692in}}%
\pgfpathlineto{\pgfqpoint{2.206633in}{5.080153in}}%
\pgfpathlineto{\pgfqpoint{2.285714in}{5.115944in}}%
\pgfpathlineto{\pgfqpoint{2.364796in}{5.142916in}}%
\pgfpathlineto{\pgfqpoint{2.443878in}{5.160960in}}%
\pgfpathlineto{\pgfqpoint{2.522959in}{5.170000in}}%
\pgfpathlineto{\pgfqpoint{2.602041in}{5.170000in}}%
\pgfpathlineto{\pgfqpoint{2.681122in}{5.160960in}}%
\pgfpathlineto{\pgfqpoint{2.760204in}{5.142916in}}%
\pgfpathlineto{\pgfqpoint{2.839286in}{5.115944in}}%
\pgfpathlineto{\pgfqpoint{2.918367in}{5.080153in}}%
\pgfpathlineto{\pgfqpoint{2.997449in}{5.035692in}}%
\pgfpathlineto{\pgfqpoint{3.076531in}{4.982742in}}%
\pgfpathlineto{\pgfqpoint{3.155612in}{4.921521in}}%
\pgfpathlineto{\pgfqpoint{3.234694in}{4.852281in}}%
\pgfpathlineto{\pgfqpoint{3.313776in}{4.775306in}}%
\pgfpathlineto{\pgfqpoint{3.392857in}{4.690913in}}%
\pgfpathlineto{\pgfqpoint{3.471939in}{4.599449in}}%
\pgfpathlineto{\pgfqpoint{3.551020in}{4.501288in}}%
\pgfpathlineto{\pgfqpoint{3.630102in}{4.396836in}}%
\pgfpathlineto{\pgfqpoint{3.709184in}{4.286520in}}%
\pgfpathlineto{\pgfqpoint{3.788265in}{4.170794in}}%
\pgfpathlineto{\pgfqpoint{3.867347in}{4.050134in}}%
\pgfpathlineto{\pgfqpoint{3.946429in}{3.925035in}}%
\pgfpathlineto{\pgfqpoint{4.025510in}{3.796012in}}%
\pgfpathlineto{\pgfqpoint{4.104592in}{3.663594in}}%
\pgfpathlineto{\pgfqpoint{4.183673in}{3.528327in}}%
\pgfpathlineto{\pgfqpoint{4.262755in}{3.390765in}}%
\pgfpathlineto{\pgfqpoint{4.341837in}{3.251474in}}%
\pgfpathlineto{\pgfqpoint{4.420918in}{3.111027in}}%
\pgfpathlineto{\pgfqpoint{4.500000in}{2.970000in}}%
\pgfusepath{stroke}%
\end{pgfscope}%
\begin{pgfscope}%
\pgfpathrectangle{\pgfqpoint{0.625000in}{2.970000in}}{\pgfqpoint{3.875000in}{2.310000in}}%
\pgfusepath{clip}%
\pgfsetrectcap%
\pgfsetroundjoin%
\pgfsetlinewidth{2.007500pt}%
\definecolor{currentstroke}{rgb}{1.000000,0.498039,0.054902}%
\pgfsetstrokecolor{currentstroke}%
\pgfsetdash{}{0pt}%
\pgfpathmoveto{\pgfqpoint{0.625000in}{2.970000in}}%
\pgfpathlineto{\pgfqpoint{0.704082in}{3.075079in}}%
\pgfpathlineto{\pgfqpoint{0.783163in}{3.179726in}}%
\pgfpathlineto{\pgfqpoint{0.862245in}{3.283511in}}%
\pgfpathlineto{\pgfqpoint{0.941327in}{3.386008in}}%
\pgfpathlineto{\pgfqpoint{1.020408in}{3.486796in}}%
\pgfpathlineto{\pgfqpoint{1.099490in}{3.585460in}}%
\pgfpathlineto{\pgfqpoint{1.178571in}{3.681595in}}%
\pgfpathlineto{\pgfqpoint{1.257653in}{3.774805in}}%
\pgfpathlineto{\pgfqpoint{1.336735in}{3.864709in}}%
\pgfpathlineto{\pgfqpoint{1.415816in}{3.950936in}}%
\pgfpathlineto{\pgfqpoint{1.494898in}{4.033132in}}%
\pgfpathlineto{\pgfqpoint{1.573980in}{4.110960in}}%
\pgfpathlineto{\pgfqpoint{1.653061in}{4.184099in}}%
\pgfpathlineto{\pgfqpoint{1.732143in}{4.252249in}}%
\pgfpathlineto{\pgfqpoint{1.811224in}{4.315130in}}%
\pgfpathlineto{\pgfqpoint{1.890306in}{4.372484in}}%
\pgfpathlineto{\pgfqpoint{1.969388in}{4.424075in}}%
\pgfpathlineto{\pgfqpoint{2.048469in}{4.469690in}}%
\pgfpathlineto{\pgfqpoint{2.127551in}{4.509143in}}%
\pgfpathlineto{\pgfqpoint{2.206633in}{4.542271in}}%
\pgfpathlineto{\pgfqpoint{2.285714in}{4.568939in}}%
\pgfpathlineto{\pgfqpoint{2.364796in}{4.589036in}}%
\pgfpathlineto{\pgfqpoint{2.443878in}{4.602480in}}%
\pgfpathlineto{\pgfqpoint{2.522959in}{4.609216in}}%
\pgfpathlineto{\pgfqpoint{2.602041in}{4.609216in}}%
\pgfpathlineto{\pgfqpoint{2.681122in}{4.602480in}}%
\pgfpathlineto{\pgfqpoint{2.760204in}{4.589036in}}%
\pgfpathlineto{\pgfqpoint{2.839286in}{4.568939in}}%
\pgfpathlineto{\pgfqpoint{2.918367in}{4.542271in}}%
\pgfpathlineto{\pgfqpoint{2.997449in}{4.509143in}}%
\pgfpathlineto{\pgfqpoint{3.076531in}{4.469690in}}%
\pgfpathlineto{\pgfqpoint{3.155612in}{4.424075in}}%
\pgfpathlineto{\pgfqpoint{3.234694in}{4.372484in}}%
\pgfpathlineto{\pgfqpoint{3.313776in}{4.315130in}}%
\pgfpathlineto{\pgfqpoint{3.392857in}{4.252249in}}%
\pgfpathlineto{\pgfqpoint{3.471939in}{4.184099in}}%
\pgfpathlineto{\pgfqpoint{3.551020in}{4.110960in}}%
\pgfpathlineto{\pgfqpoint{3.630102in}{4.033132in}}%
\pgfpathlineto{\pgfqpoint{3.709184in}{3.950936in}}%
\pgfpathlineto{\pgfqpoint{3.788265in}{3.864709in}}%
\pgfpathlineto{\pgfqpoint{3.867347in}{3.774805in}}%
\pgfpathlineto{\pgfqpoint{3.946429in}{3.681595in}}%
\pgfpathlineto{\pgfqpoint{4.025510in}{3.585460in}}%
\pgfpathlineto{\pgfqpoint{4.104592in}{3.486796in}}%
\pgfpathlineto{\pgfqpoint{4.183673in}{3.386008in}}%
\pgfpathlineto{\pgfqpoint{4.262755in}{3.283511in}}%
\pgfpathlineto{\pgfqpoint{4.341837in}{3.179726in}}%
\pgfpathlineto{\pgfqpoint{4.420918in}{3.075079in}}%
\pgfpathlineto{\pgfqpoint{4.500000in}{2.970000in}}%
\pgfusepath{stroke}%
\end{pgfscope}%
\begin{pgfscope}%
\pgfpathrectangle{\pgfqpoint{0.625000in}{2.970000in}}{\pgfqpoint{3.875000in}{2.310000in}}%
\pgfusepath{clip}%
\pgfsetrectcap%
\pgfsetroundjoin%
\pgfsetlinewidth{2.007500pt}%
\definecolor{currentstroke}{rgb}{0.172549,0.627451,0.172549}%
\pgfsetstrokecolor{currentstroke}%
\pgfsetdash{}{0pt}%
\pgfpathmoveto{\pgfqpoint{0.625000in}{4.253993in}}%
\pgfpathlineto{\pgfqpoint{0.704082in}{4.253993in}}%
\pgfpathlineto{\pgfqpoint{0.783163in}{4.253993in}}%
\pgfpathlineto{\pgfqpoint{0.862245in}{4.253993in}}%
\pgfpathlineto{\pgfqpoint{0.941327in}{4.253993in}}%
\pgfpathlineto{\pgfqpoint{1.020408in}{4.253993in}}%
\pgfpathlineto{\pgfqpoint{1.099490in}{4.253993in}}%
\pgfpathlineto{\pgfqpoint{1.178571in}{4.253993in}}%
\pgfpathlineto{\pgfqpoint{1.257653in}{4.253993in}}%
\pgfpathlineto{\pgfqpoint{1.336735in}{4.253993in}}%
\pgfpathlineto{\pgfqpoint{1.415816in}{4.253993in}}%
\pgfpathlineto{\pgfqpoint{1.494898in}{4.253993in}}%
\pgfpathlineto{\pgfqpoint{1.573980in}{4.253993in}}%
\pgfpathlineto{\pgfqpoint{1.653061in}{4.253993in}}%
\pgfpathlineto{\pgfqpoint{1.732143in}{4.253993in}}%
\pgfpathlineto{\pgfqpoint{1.811224in}{4.253993in}}%
\pgfpathlineto{\pgfqpoint{1.890306in}{4.253993in}}%
\pgfpathlineto{\pgfqpoint{1.969388in}{4.253993in}}%
\pgfpathlineto{\pgfqpoint{2.048469in}{4.253993in}}%
\pgfpathlineto{\pgfqpoint{2.127551in}{4.253993in}}%
\pgfpathlineto{\pgfqpoint{2.206633in}{4.253993in}}%
\pgfpathlineto{\pgfqpoint{2.285714in}{4.253993in}}%
\pgfpathlineto{\pgfqpoint{2.364796in}{4.253993in}}%
\pgfpathlineto{\pgfqpoint{2.443878in}{4.253993in}}%
\pgfpathlineto{\pgfqpoint{2.522959in}{4.253993in}}%
\pgfpathlineto{\pgfqpoint{2.602041in}{4.253993in}}%
\pgfpathlineto{\pgfqpoint{2.681122in}{4.253993in}}%
\pgfpathlineto{\pgfqpoint{2.760204in}{4.253993in}}%
\pgfpathlineto{\pgfqpoint{2.839286in}{4.253993in}}%
\pgfpathlineto{\pgfqpoint{2.918367in}{4.253993in}}%
\pgfpathlineto{\pgfqpoint{2.997449in}{4.253993in}}%
\pgfpathlineto{\pgfqpoint{3.076531in}{4.253993in}}%
\pgfpathlineto{\pgfqpoint{3.155612in}{4.253993in}}%
\pgfpathlineto{\pgfqpoint{3.234694in}{4.253993in}}%
\pgfpathlineto{\pgfqpoint{3.313776in}{4.253993in}}%
\pgfpathlineto{\pgfqpoint{3.392857in}{4.253993in}}%
\pgfpathlineto{\pgfqpoint{3.471939in}{4.253993in}}%
\pgfpathlineto{\pgfqpoint{3.551020in}{4.253993in}}%
\pgfpathlineto{\pgfqpoint{3.630102in}{4.253993in}}%
\pgfpathlineto{\pgfqpoint{3.709184in}{4.253993in}}%
\pgfpathlineto{\pgfqpoint{3.788265in}{4.253993in}}%
\pgfpathlineto{\pgfqpoint{3.867347in}{4.253993in}}%
\pgfpathlineto{\pgfqpoint{3.946429in}{4.253993in}}%
\pgfpathlineto{\pgfqpoint{4.025510in}{4.253993in}}%
\pgfpathlineto{\pgfqpoint{4.104592in}{4.253993in}}%
\pgfpathlineto{\pgfqpoint{4.183673in}{4.253993in}}%
\pgfpathlineto{\pgfqpoint{4.262755in}{4.253993in}}%
\pgfpathlineto{\pgfqpoint{4.341837in}{4.253993in}}%
\pgfpathlineto{\pgfqpoint{4.420918in}{4.253993in}}%
\pgfpathlineto{\pgfqpoint{4.500000in}{4.253993in}}%
\pgfusepath{stroke}%
\end{pgfscope}%
\begin{pgfscope}%
\pgfsetrectcap%
\pgfsetmiterjoin%
\pgfsetlinewidth{0.803000pt}%
\definecolor{currentstroke}{rgb}{0.000000,0.000000,0.000000}%
\pgfsetstrokecolor{currentstroke}%
\pgfsetdash{}{0pt}%
\pgfpathmoveto{\pgfqpoint{0.625000in}{2.970000in}}%
\pgfpathlineto{\pgfqpoint{0.625000in}{5.280000in}}%
\pgfusepath{stroke}%
\end{pgfscope}%
\begin{pgfscope}%
\pgfsetrectcap%
\pgfsetmiterjoin%
\pgfsetlinewidth{0.803000pt}%
\definecolor{currentstroke}{rgb}{0.000000,0.000000,0.000000}%
\pgfsetstrokecolor{currentstroke}%
\pgfsetdash{}{0pt}%
\pgfpathmoveto{\pgfqpoint{4.500000in}{2.970000in}}%
\pgfpathlineto{\pgfqpoint{4.500000in}{5.280000in}}%
\pgfusepath{stroke}%
\end{pgfscope}%
\begin{pgfscope}%
\pgfsetrectcap%
\pgfsetmiterjoin%
\pgfsetlinewidth{0.803000pt}%
\definecolor{currentstroke}{rgb}{0.000000,0.000000,0.000000}%
\pgfsetstrokecolor{currentstroke}%
\pgfsetdash{}{0pt}%
\pgfpathmoveto{\pgfqpoint{0.625000in}{2.970000in}}%
\pgfpathlineto{\pgfqpoint{4.500000in}{2.970000in}}%
\pgfusepath{stroke}%
\end{pgfscope}%
\begin{pgfscope}%
\pgfsetrectcap%
\pgfsetmiterjoin%
\pgfsetlinewidth{0.803000pt}%
\definecolor{currentstroke}{rgb}{0.000000,0.000000,0.000000}%
\pgfsetstrokecolor{currentstroke}%
\pgfsetdash{}{0pt}%
\pgfpathmoveto{\pgfqpoint{0.625000in}{5.280000in}}%
\pgfpathlineto{\pgfqpoint{4.500000in}{5.280000in}}%
\pgfusepath{stroke}%
\end{pgfscope}%
\begin{pgfscope}%
\pgfsetbuttcap%
\pgfsetmiterjoin%
\definecolor{currentfill}{rgb}{1.000000,1.000000,1.000000}%
\pgfsetfillcolor{currentfill}%
\pgfsetfillopacity{0.800000}%
\pgfsetlinewidth{1.003750pt}%
\definecolor{currentstroke}{rgb}{0.800000,0.800000,0.800000}%
\pgfsetstrokecolor{currentstroke}%
\pgfsetstrokeopacity{0.800000}%
\pgfsetdash{}{0pt}%
\pgfpathmoveto{\pgfqpoint{1.829504in}{3.039444in}}%
\pgfpathlineto{\pgfqpoint{3.295496in}{3.039444in}}%
\pgfpathquadraticcurveto{\pgfqpoint{3.323274in}{3.039444in}}{\pgfqpoint{3.323274in}{3.067222in}}%
\pgfpathlineto{\pgfqpoint{3.323274in}{3.661311in}}%
\pgfpathquadraticcurveto{\pgfqpoint{3.323274in}{3.689088in}}{\pgfqpoint{3.295496in}{3.689088in}}%
\pgfpathlineto{\pgfqpoint{1.829504in}{3.689088in}}%
\pgfpathquadraticcurveto{\pgfqpoint{1.801726in}{3.689088in}}{\pgfqpoint{1.801726in}{3.661311in}}%
\pgfpathlineto{\pgfqpoint{1.801726in}{3.067222in}}%
\pgfpathquadraticcurveto{\pgfqpoint{1.801726in}{3.039444in}}{\pgfqpoint{1.829504in}{3.039444in}}%
\pgfpathlineto{\pgfqpoint{1.829504in}{3.039444in}}%
\pgfpathclose%
\pgfusepath{stroke,fill}%
\end{pgfscope}%
\begin{pgfscope}%
\pgfsetrectcap%
\pgfsetroundjoin%
\pgfsetlinewidth{2.007500pt}%
\definecolor{currentstroke}{rgb}{0.121569,0.466667,0.705882}%
\pgfsetstrokecolor{currentstroke}%
\pgfsetdash{}{0pt}%
\pgfpathmoveto{\pgfqpoint{1.857281in}{3.578791in}}%
\pgfpathlineto{\pgfqpoint{1.996170in}{3.578791in}}%
\pgfpathlineto{\pgfqpoint{2.135059in}{3.578791in}}%
\pgfusepath{stroke}%
\end{pgfscope}%
\begin{pgfscope}%
\definecolor{textcolor}{rgb}{0.000000,0.000000,0.000000}%
\pgfsetstrokecolor{textcolor}%
\pgfsetfillcolor{textcolor}%
\pgftext[x=2.246170in,y=3.530180in,left,base]{\color{textcolor}\rmfamily\fontsize{10.000000}{12.000000}\selectfont \(\displaystyle P_\mathrm{e}\) pre-fault}%
\end{pgfscope}%
\begin{pgfscope}%
\pgfsetrectcap%
\pgfsetroundjoin%
\pgfsetlinewidth{2.007500pt}%
\definecolor{currentstroke}{rgb}{1.000000,0.498039,0.054902}%
\pgfsetstrokecolor{currentstroke}%
\pgfsetdash{}{0pt}%
\pgfpathmoveto{\pgfqpoint{1.857281in}{3.375748in}}%
\pgfpathlineto{\pgfqpoint{1.996170in}{3.375748in}}%
\pgfpathlineto{\pgfqpoint{2.135059in}{3.375748in}}%
\pgfusepath{stroke}%
\end{pgfscope}%
\begin{pgfscope}%
\definecolor{textcolor}{rgb}{0.000000,0.000000,0.000000}%
\pgfsetstrokecolor{textcolor}%
\pgfsetfillcolor{textcolor}%
\pgftext[x=2.246170in,y=3.327137in,left,base]{\color{textcolor}\rmfamily\fontsize{10.000000}{12.000000}\selectfont \(\displaystyle P_\mathrm{e}\) post-fault}%
\end{pgfscope}%
\begin{pgfscope}%
\pgfsetrectcap%
\pgfsetroundjoin%
\pgfsetlinewidth{2.007500pt}%
\definecolor{currentstroke}{rgb}{0.172549,0.627451,0.172549}%
\pgfsetstrokecolor{currentstroke}%
\pgfsetdash{}{0pt}%
\pgfpathmoveto{\pgfqpoint{1.857281in}{3.173857in}}%
\pgfpathlineto{\pgfqpoint{1.996170in}{3.173857in}}%
\pgfpathlineto{\pgfqpoint{2.135059in}{3.173857in}}%
\pgfusepath{stroke}%
\end{pgfscope}%
\begin{pgfscope}%
\definecolor{textcolor}{rgb}{0.000000,0.000000,0.000000}%
\pgfsetstrokecolor{textcolor}%
\pgfsetfillcolor{textcolor}%
\pgftext[x=2.246170in,y=3.125246in,left,base]{\color{textcolor}\rmfamily\fontsize{10.000000}{12.000000}\selectfont \(\displaystyle P_\mathrm{T}\) of the turbine}%
\end{pgfscope}%
\begin{pgfscope}%
\pgfsetbuttcap%
\pgfsetmiterjoin%
\definecolor{currentfill}{rgb}{1.000000,1.000000,1.000000}%
\pgfsetfillcolor{currentfill}%
\pgfsetlinewidth{0.000000pt}%
\definecolor{currentstroke}{rgb}{0.000000,0.000000,0.000000}%
\pgfsetstrokecolor{currentstroke}%
\pgfsetstrokeopacity{0.000000}%
\pgfsetdash{}{0pt}%
\pgfpathmoveto{\pgfqpoint{0.625000in}{0.660000in}}%
\pgfpathlineto{\pgfqpoint{4.500000in}{0.660000in}}%
\pgfpathlineto{\pgfqpoint{4.500000in}{2.970000in}}%
\pgfpathlineto{\pgfqpoint{0.625000in}{2.970000in}}%
\pgfpathlineto{\pgfqpoint{0.625000in}{0.660000in}}%
\pgfpathclose%
\pgfusepath{fill}%
\end{pgfscope}%
\begin{pgfscope}%
\pgfpathrectangle{\pgfqpoint{0.625000in}{0.660000in}}{\pgfqpoint{3.875000in}{2.310000in}}%
\pgfusepath{clip}%
\pgfsetrectcap%
\pgfsetroundjoin%
\pgfsetlinewidth{0.803000pt}%
\definecolor{currentstroke}{rgb}{0.690196,0.690196,0.690196}%
\pgfsetstrokecolor{currentstroke}%
\pgfsetdash{}{0pt}%
\pgfpathmoveto{\pgfqpoint{0.625000in}{0.660000in}}%
\pgfpathlineto{\pgfqpoint{0.625000in}{2.970000in}}%
\pgfusepath{stroke}%
\end{pgfscope}%
\begin{pgfscope}%
\pgfsetbuttcap%
\pgfsetroundjoin%
\definecolor{currentfill}{rgb}{0.000000,0.000000,0.000000}%
\pgfsetfillcolor{currentfill}%
\pgfsetlinewidth{0.803000pt}%
\definecolor{currentstroke}{rgb}{0.000000,0.000000,0.000000}%
\pgfsetstrokecolor{currentstroke}%
\pgfsetdash{}{0pt}%
\pgfsys@defobject{currentmarker}{\pgfqpoint{0.000000in}{-0.048611in}}{\pgfqpoint{0.000000in}{0.000000in}}{%
\pgfpathmoveto{\pgfqpoint{0.000000in}{0.000000in}}%
\pgfpathlineto{\pgfqpoint{0.000000in}{-0.048611in}}%
\pgfusepath{stroke,fill}%
}%
\begin{pgfscope}%
\pgfsys@transformshift{0.625000in}{0.660000in}%
\pgfsys@useobject{currentmarker}{}%
\end{pgfscope}%
\end{pgfscope}%
\begin{pgfscope}%
\definecolor{textcolor}{rgb}{0.000000,0.000000,0.000000}%
\pgfsetstrokecolor{textcolor}%
\pgfsetfillcolor{textcolor}%
\pgftext[x=0.625000in,y=0.562778in,,top]{\color{textcolor}\rmfamily\fontsize{10.000000}{12.000000}\selectfont \(\displaystyle {0}\)}%
\end{pgfscope}%
\begin{pgfscope}%
\pgfpathrectangle{\pgfqpoint{0.625000in}{0.660000in}}{\pgfqpoint{3.875000in}{2.310000in}}%
\pgfusepath{clip}%
\pgfsetrectcap%
\pgfsetroundjoin%
\pgfsetlinewidth{0.803000pt}%
\definecolor{currentstroke}{rgb}{0.690196,0.690196,0.690196}%
\pgfsetstrokecolor{currentstroke}%
\pgfsetdash{}{0pt}%
\pgfpathmoveto{\pgfqpoint{1.055556in}{0.660000in}}%
\pgfpathlineto{\pgfqpoint{1.055556in}{2.970000in}}%
\pgfusepath{stroke}%
\end{pgfscope}%
\begin{pgfscope}%
\pgfsetbuttcap%
\pgfsetroundjoin%
\definecolor{currentfill}{rgb}{0.000000,0.000000,0.000000}%
\pgfsetfillcolor{currentfill}%
\pgfsetlinewidth{0.803000pt}%
\definecolor{currentstroke}{rgb}{0.000000,0.000000,0.000000}%
\pgfsetstrokecolor{currentstroke}%
\pgfsetdash{}{0pt}%
\pgfsys@defobject{currentmarker}{\pgfqpoint{0.000000in}{-0.048611in}}{\pgfqpoint{0.000000in}{0.000000in}}{%
\pgfpathmoveto{\pgfqpoint{0.000000in}{0.000000in}}%
\pgfpathlineto{\pgfqpoint{0.000000in}{-0.048611in}}%
\pgfusepath{stroke,fill}%
}%
\begin{pgfscope}%
\pgfsys@transformshift{1.055556in}{0.660000in}%
\pgfsys@useobject{currentmarker}{}%
\end{pgfscope}%
\end{pgfscope}%
\begin{pgfscope}%
\definecolor{textcolor}{rgb}{0.000000,0.000000,0.000000}%
\pgfsetstrokecolor{textcolor}%
\pgfsetfillcolor{textcolor}%
\pgftext[x=1.055556in,y=0.562778in,,top]{\color{textcolor}\rmfamily\fontsize{10.000000}{12.000000}\selectfont \(\displaystyle {20}\)}%
\end{pgfscope}%
\begin{pgfscope}%
\pgfpathrectangle{\pgfqpoint{0.625000in}{0.660000in}}{\pgfqpoint{3.875000in}{2.310000in}}%
\pgfusepath{clip}%
\pgfsetrectcap%
\pgfsetroundjoin%
\pgfsetlinewidth{0.803000pt}%
\definecolor{currentstroke}{rgb}{0.690196,0.690196,0.690196}%
\pgfsetstrokecolor{currentstroke}%
\pgfsetdash{}{0pt}%
\pgfpathmoveto{\pgfqpoint{1.486111in}{0.660000in}}%
\pgfpathlineto{\pgfqpoint{1.486111in}{2.970000in}}%
\pgfusepath{stroke}%
\end{pgfscope}%
\begin{pgfscope}%
\pgfsetbuttcap%
\pgfsetroundjoin%
\definecolor{currentfill}{rgb}{0.000000,0.000000,0.000000}%
\pgfsetfillcolor{currentfill}%
\pgfsetlinewidth{0.803000pt}%
\definecolor{currentstroke}{rgb}{0.000000,0.000000,0.000000}%
\pgfsetstrokecolor{currentstroke}%
\pgfsetdash{}{0pt}%
\pgfsys@defobject{currentmarker}{\pgfqpoint{0.000000in}{-0.048611in}}{\pgfqpoint{0.000000in}{0.000000in}}{%
\pgfpathmoveto{\pgfqpoint{0.000000in}{0.000000in}}%
\pgfpathlineto{\pgfqpoint{0.000000in}{-0.048611in}}%
\pgfusepath{stroke,fill}%
}%
\begin{pgfscope}%
\pgfsys@transformshift{1.486111in}{0.660000in}%
\pgfsys@useobject{currentmarker}{}%
\end{pgfscope}%
\end{pgfscope}%
\begin{pgfscope}%
\definecolor{textcolor}{rgb}{0.000000,0.000000,0.000000}%
\pgfsetstrokecolor{textcolor}%
\pgfsetfillcolor{textcolor}%
\pgftext[x=1.486111in,y=0.562778in,,top]{\color{textcolor}\rmfamily\fontsize{10.000000}{12.000000}\selectfont \(\displaystyle {40}\)}%
\end{pgfscope}%
\begin{pgfscope}%
\pgfpathrectangle{\pgfqpoint{0.625000in}{0.660000in}}{\pgfqpoint{3.875000in}{2.310000in}}%
\pgfusepath{clip}%
\pgfsetrectcap%
\pgfsetroundjoin%
\pgfsetlinewidth{0.803000pt}%
\definecolor{currentstroke}{rgb}{0.690196,0.690196,0.690196}%
\pgfsetstrokecolor{currentstroke}%
\pgfsetdash{}{0pt}%
\pgfpathmoveto{\pgfqpoint{1.916667in}{0.660000in}}%
\pgfpathlineto{\pgfqpoint{1.916667in}{2.970000in}}%
\pgfusepath{stroke}%
\end{pgfscope}%
\begin{pgfscope}%
\pgfsetbuttcap%
\pgfsetroundjoin%
\definecolor{currentfill}{rgb}{0.000000,0.000000,0.000000}%
\pgfsetfillcolor{currentfill}%
\pgfsetlinewidth{0.803000pt}%
\definecolor{currentstroke}{rgb}{0.000000,0.000000,0.000000}%
\pgfsetstrokecolor{currentstroke}%
\pgfsetdash{}{0pt}%
\pgfsys@defobject{currentmarker}{\pgfqpoint{0.000000in}{-0.048611in}}{\pgfqpoint{0.000000in}{0.000000in}}{%
\pgfpathmoveto{\pgfqpoint{0.000000in}{0.000000in}}%
\pgfpathlineto{\pgfqpoint{0.000000in}{-0.048611in}}%
\pgfusepath{stroke,fill}%
}%
\begin{pgfscope}%
\pgfsys@transformshift{1.916667in}{0.660000in}%
\pgfsys@useobject{currentmarker}{}%
\end{pgfscope}%
\end{pgfscope}%
\begin{pgfscope}%
\definecolor{textcolor}{rgb}{0.000000,0.000000,0.000000}%
\pgfsetstrokecolor{textcolor}%
\pgfsetfillcolor{textcolor}%
\pgftext[x=1.916667in,y=0.562778in,,top]{\color{textcolor}\rmfamily\fontsize{10.000000}{12.000000}\selectfont \(\displaystyle {60}\)}%
\end{pgfscope}%
\begin{pgfscope}%
\pgfpathrectangle{\pgfqpoint{0.625000in}{0.660000in}}{\pgfqpoint{3.875000in}{2.310000in}}%
\pgfusepath{clip}%
\pgfsetrectcap%
\pgfsetroundjoin%
\pgfsetlinewidth{0.803000pt}%
\definecolor{currentstroke}{rgb}{0.690196,0.690196,0.690196}%
\pgfsetstrokecolor{currentstroke}%
\pgfsetdash{}{0pt}%
\pgfpathmoveto{\pgfqpoint{2.347222in}{0.660000in}}%
\pgfpathlineto{\pgfqpoint{2.347222in}{2.970000in}}%
\pgfusepath{stroke}%
\end{pgfscope}%
\begin{pgfscope}%
\pgfsetbuttcap%
\pgfsetroundjoin%
\definecolor{currentfill}{rgb}{0.000000,0.000000,0.000000}%
\pgfsetfillcolor{currentfill}%
\pgfsetlinewidth{0.803000pt}%
\definecolor{currentstroke}{rgb}{0.000000,0.000000,0.000000}%
\pgfsetstrokecolor{currentstroke}%
\pgfsetdash{}{0pt}%
\pgfsys@defobject{currentmarker}{\pgfqpoint{0.000000in}{-0.048611in}}{\pgfqpoint{0.000000in}{0.000000in}}{%
\pgfpathmoveto{\pgfqpoint{0.000000in}{0.000000in}}%
\pgfpathlineto{\pgfqpoint{0.000000in}{-0.048611in}}%
\pgfusepath{stroke,fill}%
}%
\begin{pgfscope}%
\pgfsys@transformshift{2.347222in}{0.660000in}%
\pgfsys@useobject{currentmarker}{}%
\end{pgfscope}%
\end{pgfscope}%
\begin{pgfscope}%
\definecolor{textcolor}{rgb}{0.000000,0.000000,0.000000}%
\pgfsetstrokecolor{textcolor}%
\pgfsetfillcolor{textcolor}%
\pgftext[x=2.347222in,y=0.562778in,,top]{\color{textcolor}\rmfamily\fontsize{10.000000}{12.000000}\selectfont \(\displaystyle {80}\)}%
\end{pgfscope}%
\begin{pgfscope}%
\pgfpathrectangle{\pgfqpoint{0.625000in}{0.660000in}}{\pgfqpoint{3.875000in}{2.310000in}}%
\pgfusepath{clip}%
\pgfsetrectcap%
\pgfsetroundjoin%
\pgfsetlinewidth{0.803000pt}%
\definecolor{currentstroke}{rgb}{0.690196,0.690196,0.690196}%
\pgfsetstrokecolor{currentstroke}%
\pgfsetdash{}{0pt}%
\pgfpathmoveto{\pgfqpoint{2.777778in}{0.660000in}}%
\pgfpathlineto{\pgfqpoint{2.777778in}{2.970000in}}%
\pgfusepath{stroke}%
\end{pgfscope}%
\begin{pgfscope}%
\pgfsetbuttcap%
\pgfsetroundjoin%
\definecolor{currentfill}{rgb}{0.000000,0.000000,0.000000}%
\pgfsetfillcolor{currentfill}%
\pgfsetlinewidth{0.803000pt}%
\definecolor{currentstroke}{rgb}{0.000000,0.000000,0.000000}%
\pgfsetstrokecolor{currentstroke}%
\pgfsetdash{}{0pt}%
\pgfsys@defobject{currentmarker}{\pgfqpoint{0.000000in}{-0.048611in}}{\pgfqpoint{0.000000in}{0.000000in}}{%
\pgfpathmoveto{\pgfqpoint{0.000000in}{0.000000in}}%
\pgfpathlineto{\pgfqpoint{0.000000in}{-0.048611in}}%
\pgfusepath{stroke,fill}%
}%
\begin{pgfscope}%
\pgfsys@transformshift{2.777778in}{0.660000in}%
\pgfsys@useobject{currentmarker}{}%
\end{pgfscope}%
\end{pgfscope}%
\begin{pgfscope}%
\definecolor{textcolor}{rgb}{0.000000,0.000000,0.000000}%
\pgfsetstrokecolor{textcolor}%
\pgfsetfillcolor{textcolor}%
\pgftext[x=2.777778in,y=0.562778in,,top]{\color{textcolor}\rmfamily\fontsize{10.000000}{12.000000}\selectfont \(\displaystyle {100}\)}%
\end{pgfscope}%
\begin{pgfscope}%
\pgfpathrectangle{\pgfqpoint{0.625000in}{0.660000in}}{\pgfqpoint{3.875000in}{2.310000in}}%
\pgfusepath{clip}%
\pgfsetrectcap%
\pgfsetroundjoin%
\pgfsetlinewidth{0.803000pt}%
\definecolor{currentstroke}{rgb}{0.690196,0.690196,0.690196}%
\pgfsetstrokecolor{currentstroke}%
\pgfsetdash{}{0pt}%
\pgfpathmoveto{\pgfqpoint{3.208333in}{0.660000in}}%
\pgfpathlineto{\pgfqpoint{3.208333in}{2.970000in}}%
\pgfusepath{stroke}%
\end{pgfscope}%
\begin{pgfscope}%
\pgfsetbuttcap%
\pgfsetroundjoin%
\definecolor{currentfill}{rgb}{0.000000,0.000000,0.000000}%
\pgfsetfillcolor{currentfill}%
\pgfsetlinewidth{0.803000pt}%
\definecolor{currentstroke}{rgb}{0.000000,0.000000,0.000000}%
\pgfsetstrokecolor{currentstroke}%
\pgfsetdash{}{0pt}%
\pgfsys@defobject{currentmarker}{\pgfqpoint{0.000000in}{-0.048611in}}{\pgfqpoint{0.000000in}{0.000000in}}{%
\pgfpathmoveto{\pgfqpoint{0.000000in}{0.000000in}}%
\pgfpathlineto{\pgfqpoint{0.000000in}{-0.048611in}}%
\pgfusepath{stroke,fill}%
}%
\begin{pgfscope}%
\pgfsys@transformshift{3.208333in}{0.660000in}%
\pgfsys@useobject{currentmarker}{}%
\end{pgfscope}%
\end{pgfscope}%
\begin{pgfscope}%
\definecolor{textcolor}{rgb}{0.000000,0.000000,0.000000}%
\pgfsetstrokecolor{textcolor}%
\pgfsetfillcolor{textcolor}%
\pgftext[x=3.208333in,y=0.562778in,,top]{\color{textcolor}\rmfamily\fontsize{10.000000}{12.000000}\selectfont \(\displaystyle {120}\)}%
\end{pgfscope}%
\begin{pgfscope}%
\pgfpathrectangle{\pgfqpoint{0.625000in}{0.660000in}}{\pgfqpoint{3.875000in}{2.310000in}}%
\pgfusepath{clip}%
\pgfsetrectcap%
\pgfsetroundjoin%
\pgfsetlinewidth{0.803000pt}%
\definecolor{currentstroke}{rgb}{0.690196,0.690196,0.690196}%
\pgfsetstrokecolor{currentstroke}%
\pgfsetdash{}{0pt}%
\pgfpathmoveto{\pgfqpoint{3.638889in}{0.660000in}}%
\pgfpathlineto{\pgfqpoint{3.638889in}{2.970000in}}%
\pgfusepath{stroke}%
\end{pgfscope}%
\begin{pgfscope}%
\pgfsetbuttcap%
\pgfsetroundjoin%
\definecolor{currentfill}{rgb}{0.000000,0.000000,0.000000}%
\pgfsetfillcolor{currentfill}%
\pgfsetlinewidth{0.803000pt}%
\definecolor{currentstroke}{rgb}{0.000000,0.000000,0.000000}%
\pgfsetstrokecolor{currentstroke}%
\pgfsetdash{}{0pt}%
\pgfsys@defobject{currentmarker}{\pgfqpoint{0.000000in}{-0.048611in}}{\pgfqpoint{0.000000in}{0.000000in}}{%
\pgfpathmoveto{\pgfqpoint{0.000000in}{0.000000in}}%
\pgfpathlineto{\pgfqpoint{0.000000in}{-0.048611in}}%
\pgfusepath{stroke,fill}%
}%
\begin{pgfscope}%
\pgfsys@transformshift{3.638889in}{0.660000in}%
\pgfsys@useobject{currentmarker}{}%
\end{pgfscope}%
\end{pgfscope}%
\begin{pgfscope}%
\definecolor{textcolor}{rgb}{0.000000,0.000000,0.000000}%
\pgfsetstrokecolor{textcolor}%
\pgfsetfillcolor{textcolor}%
\pgftext[x=3.638889in,y=0.562778in,,top]{\color{textcolor}\rmfamily\fontsize{10.000000}{12.000000}\selectfont \(\displaystyle {140}\)}%
\end{pgfscope}%
\begin{pgfscope}%
\pgfpathrectangle{\pgfqpoint{0.625000in}{0.660000in}}{\pgfqpoint{3.875000in}{2.310000in}}%
\pgfusepath{clip}%
\pgfsetrectcap%
\pgfsetroundjoin%
\pgfsetlinewidth{0.803000pt}%
\definecolor{currentstroke}{rgb}{0.690196,0.690196,0.690196}%
\pgfsetstrokecolor{currentstroke}%
\pgfsetdash{}{0pt}%
\pgfpathmoveto{\pgfqpoint{4.069444in}{0.660000in}}%
\pgfpathlineto{\pgfqpoint{4.069444in}{2.970000in}}%
\pgfusepath{stroke}%
\end{pgfscope}%
\begin{pgfscope}%
\pgfsetbuttcap%
\pgfsetroundjoin%
\definecolor{currentfill}{rgb}{0.000000,0.000000,0.000000}%
\pgfsetfillcolor{currentfill}%
\pgfsetlinewidth{0.803000pt}%
\definecolor{currentstroke}{rgb}{0.000000,0.000000,0.000000}%
\pgfsetstrokecolor{currentstroke}%
\pgfsetdash{}{0pt}%
\pgfsys@defobject{currentmarker}{\pgfqpoint{0.000000in}{-0.048611in}}{\pgfqpoint{0.000000in}{0.000000in}}{%
\pgfpathmoveto{\pgfqpoint{0.000000in}{0.000000in}}%
\pgfpathlineto{\pgfqpoint{0.000000in}{-0.048611in}}%
\pgfusepath{stroke,fill}%
}%
\begin{pgfscope}%
\pgfsys@transformshift{4.069444in}{0.660000in}%
\pgfsys@useobject{currentmarker}{}%
\end{pgfscope}%
\end{pgfscope}%
\begin{pgfscope}%
\definecolor{textcolor}{rgb}{0.000000,0.000000,0.000000}%
\pgfsetstrokecolor{textcolor}%
\pgfsetfillcolor{textcolor}%
\pgftext[x=4.069444in,y=0.562778in,,top]{\color{textcolor}\rmfamily\fontsize{10.000000}{12.000000}\selectfont \(\displaystyle {160}\)}%
\end{pgfscope}%
\begin{pgfscope}%
\pgfpathrectangle{\pgfqpoint{0.625000in}{0.660000in}}{\pgfqpoint{3.875000in}{2.310000in}}%
\pgfusepath{clip}%
\pgfsetrectcap%
\pgfsetroundjoin%
\pgfsetlinewidth{0.803000pt}%
\definecolor{currentstroke}{rgb}{0.690196,0.690196,0.690196}%
\pgfsetstrokecolor{currentstroke}%
\pgfsetdash{}{0pt}%
\pgfpathmoveto{\pgfqpoint{4.500000in}{0.660000in}}%
\pgfpathlineto{\pgfqpoint{4.500000in}{2.970000in}}%
\pgfusepath{stroke}%
\end{pgfscope}%
\begin{pgfscope}%
\pgfsetbuttcap%
\pgfsetroundjoin%
\definecolor{currentfill}{rgb}{0.000000,0.000000,0.000000}%
\pgfsetfillcolor{currentfill}%
\pgfsetlinewidth{0.803000pt}%
\definecolor{currentstroke}{rgb}{0.000000,0.000000,0.000000}%
\pgfsetstrokecolor{currentstroke}%
\pgfsetdash{}{0pt}%
\pgfsys@defobject{currentmarker}{\pgfqpoint{0.000000in}{-0.048611in}}{\pgfqpoint{0.000000in}{0.000000in}}{%
\pgfpathmoveto{\pgfqpoint{0.000000in}{0.000000in}}%
\pgfpathlineto{\pgfqpoint{0.000000in}{-0.048611in}}%
\pgfusepath{stroke,fill}%
}%
\begin{pgfscope}%
\pgfsys@transformshift{4.500000in}{0.660000in}%
\pgfsys@useobject{currentmarker}{}%
\end{pgfscope}%
\end{pgfscope}%
\begin{pgfscope}%
\definecolor{textcolor}{rgb}{0.000000,0.000000,0.000000}%
\pgfsetstrokecolor{textcolor}%
\pgfsetfillcolor{textcolor}%
\pgftext[x=4.500000in,y=0.562778in,,top]{\color{textcolor}\rmfamily\fontsize{10.000000}{12.000000}\selectfont \(\displaystyle {180}\)}%
\end{pgfscope}%
\begin{pgfscope}%
\definecolor{textcolor}{rgb}{0.000000,0.000000,0.000000}%
\pgfsetstrokecolor{textcolor}%
\pgfsetfillcolor{textcolor}%
\pgftext[x=2.562500in,y=0.374776in,,top]{\color{textcolor}\rmfamily\fontsize{10.000000}{12.000000}\selectfont power angle \(\displaystyle \delta\) in deg}%
\end{pgfscope}%
\begin{pgfscope}%
\pgfpathrectangle{\pgfqpoint{0.625000in}{0.660000in}}{\pgfqpoint{3.875000in}{2.310000in}}%
\pgfusepath{clip}%
\pgfsetrectcap%
\pgfsetroundjoin%
\pgfsetlinewidth{0.803000pt}%
\definecolor{currentstroke}{rgb}{0.690196,0.690196,0.690196}%
\pgfsetstrokecolor{currentstroke}%
\pgfsetdash{}{0pt}%
\pgfpathmoveto{\pgfqpoint{0.625000in}{2.770826in}}%
\pgfpathlineto{\pgfqpoint{4.500000in}{2.770826in}}%
\pgfusepath{stroke}%
\end{pgfscope}%
\begin{pgfscope}%
\pgfsetbuttcap%
\pgfsetroundjoin%
\definecolor{currentfill}{rgb}{0.000000,0.000000,0.000000}%
\pgfsetfillcolor{currentfill}%
\pgfsetlinewidth{0.803000pt}%
\definecolor{currentstroke}{rgb}{0.000000,0.000000,0.000000}%
\pgfsetstrokecolor{currentstroke}%
\pgfsetdash{}{0pt}%
\pgfsys@defobject{currentmarker}{\pgfqpoint{-0.048611in}{0.000000in}}{\pgfqpoint{-0.000000in}{0.000000in}}{%
\pgfpathmoveto{\pgfqpoint{-0.000000in}{0.000000in}}%
\pgfpathlineto{\pgfqpoint{-0.048611in}{0.000000in}}%
\pgfusepath{stroke,fill}%
}%
\begin{pgfscope}%
\pgfsys@transformshift{0.625000in}{2.770826in}%
\pgfsys@useobject{currentmarker}{}%
\end{pgfscope}%
\end{pgfscope}%
\begin{pgfscope}%
\definecolor{textcolor}{rgb}{0.000000,0.000000,0.000000}%
\pgfsetstrokecolor{textcolor}%
\pgfsetfillcolor{textcolor}%
\pgftext[x=0.458333in, y=2.719726in, left, base]{\color{textcolor}\rmfamily\fontsize{10.000000}{12.000000}\selectfont \(\displaystyle {0}\)}%
\end{pgfscope}%
\begin{pgfscope}%
\pgfpathrectangle{\pgfqpoint{0.625000in}{0.660000in}}{\pgfqpoint{3.875000in}{2.310000in}}%
\pgfusepath{clip}%
\pgfsetrectcap%
\pgfsetroundjoin%
\pgfsetlinewidth{0.803000pt}%
\definecolor{currentstroke}{rgb}{0.690196,0.690196,0.690196}%
\pgfsetstrokecolor{currentstroke}%
\pgfsetdash{}{0pt}%
\pgfpathmoveto{\pgfqpoint{0.625000in}{2.372478in}}%
\pgfpathlineto{\pgfqpoint{4.500000in}{2.372478in}}%
\pgfusepath{stroke}%
\end{pgfscope}%
\begin{pgfscope}%
\pgfsetbuttcap%
\pgfsetroundjoin%
\definecolor{currentfill}{rgb}{0.000000,0.000000,0.000000}%
\pgfsetfillcolor{currentfill}%
\pgfsetlinewidth{0.803000pt}%
\definecolor{currentstroke}{rgb}{0.000000,0.000000,0.000000}%
\pgfsetstrokecolor{currentstroke}%
\pgfsetdash{}{0pt}%
\pgfsys@defobject{currentmarker}{\pgfqpoint{-0.048611in}{0.000000in}}{\pgfqpoint{-0.000000in}{0.000000in}}{%
\pgfpathmoveto{\pgfqpoint{-0.000000in}{0.000000in}}%
\pgfpathlineto{\pgfqpoint{-0.048611in}{0.000000in}}%
\pgfusepath{stroke,fill}%
}%
\begin{pgfscope}%
\pgfsys@transformshift{0.625000in}{2.372478in}%
\pgfsys@useobject{currentmarker}{}%
\end{pgfscope}%
\end{pgfscope}%
\begin{pgfscope}%
\definecolor{textcolor}{rgb}{0.000000,0.000000,0.000000}%
\pgfsetstrokecolor{textcolor}%
\pgfsetfillcolor{textcolor}%
\pgftext[x=0.458333in, y=2.321378in, left, base]{\color{textcolor}\rmfamily\fontsize{10.000000}{12.000000}\selectfont \(\displaystyle {1}\)}%
\end{pgfscope}%
\begin{pgfscope}%
\pgfpathrectangle{\pgfqpoint{0.625000in}{0.660000in}}{\pgfqpoint{3.875000in}{2.310000in}}%
\pgfusepath{clip}%
\pgfsetrectcap%
\pgfsetroundjoin%
\pgfsetlinewidth{0.803000pt}%
\definecolor{currentstroke}{rgb}{0.690196,0.690196,0.690196}%
\pgfsetstrokecolor{currentstroke}%
\pgfsetdash{}{0pt}%
\pgfpathmoveto{\pgfqpoint{0.625000in}{1.974130in}}%
\pgfpathlineto{\pgfqpoint{4.500000in}{1.974130in}}%
\pgfusepath{stroke}%
\end{pgfscope}%
\begin{pgfscope}%
\pgfsetbuttcap%
\pgfsetroundjoin%
\definecolor{currentfill}{rgb}{0.000000,0.000000,0.000000}%
\pgfsetfillcolor{currentfill}%
\pgfsetlinewidth{0.803000pt}%
\definecolor{currentstroke}{rgb}{0.000000,0.000000,0.000000}%
\pgfsetstrokecolor{currentstroke}%
\pgfsetdash{}{0pt}%
\pgfsys@defobject{currentmarker}{\pgfqpoint{-0.048611in}{0.000000in}}{\pgfqpoint{-0.000000in}{0.000000in}}{%
\pgfpathmoveto{\pgfqpoint{-0.000000in}{0.000000in}}%
\pgfpathlineto{\pgfqpoint{-0.048611in}{0.000000in}}%
\pgfusepath{stroke,fill}%
}%
\begin{pgfscope}%
\pgfsys@transformshift{0.625000in}{1.974130in}%
\pgfsys@useobject{currentmarker}{}%
\end{pgfscope}%
\end{pgfscope}%
\begin{pgfscope}%
\definecolor{textcolor}{rgb}{0.000000,0.000000,0.000000}%
\pgfsetstrokecolor{textcolor}%
\pgfsetfillcolor{textcolor}%
\pgftext[x=0.458333in, y=1.923030in, left, base]{\color{textcolor}\rmfamily\fontsize{10.000000}{12.000000}\selectfont \(\displaystyle {2}\)}%
\end{pgfscope}%
\begin{pgfscope}%
\pgfpathrectangle{\pgfqpoint{0.625000in}{0.660000in}}{\pgfqpoint{3.875000in}{2.310000in}}%
\pgfusepath{clip}%
\pgfsetrectcap%
\pgfsetroundjoin%
\pgfsetlinewidth{0.803000pt}%
\definecolor{currentstroke}{rgb}{0.690196,0.690196,0.690196}%
\pgfsetstrokecolor{currentstroke}%
\pgfsetdash{}{0pt}%
\pgfpathmoveto{\pgfqpoint{0.625000in}{1.575782in}}%
\pgfpathlineto{\pgfqpoint{4.500000in}{1.575782in}}%
\pgfusepath{stroke}%
\end{pgfscope}%
\begin{pgfscope}%
\pgfsetbuttcap%
\pgfsetroundjoin%
\definecolor{currentfill}{rgb}{0.000000,0.000000,0.000000}%
\pgfsetfillcolor{currentfill}%
\pgfsetlinewidth{0.803000pt}%
\definecolor{currentstroke}{rgb}{0.000000,0.000000,0.000000}%
\pgfsetstrokecolor{currentstroke}%
\pgfsetdash{}{0pt}%
\pgfsys@defobject{currentmarker}{\pgfqpoint{-0.048611in}{0.000000in}}{\pgfqpoint{-0.000000in}{0.000000in}}{%
\pgfpathmoveto{\pgfqpoint{-0.000000in}{0.000000in}}%
\pgfpathlineto{\pgfqpoint{-0.048611in}{0.000000in}}%
\pgfusepath{stroke,fill}%
}%
\begin{pgfscope}%
\pgfsys@transformshift{0.625000in}{1.575782in}%
\pgfsys@useobject{currentmarker}{}%
\end{pgfscope}%
\end{pgfscope}%
\begin{pgfscope}%
\definecolor{textcolor}{rgb}{0.000000,0.000000,0.000000}%
\pgfsetstrokecolor{textcolor}%
\pgfsetfillcolor{textcolor}%
\pgftext[x=0.458333in, y=1.524682in, left, base]{\color{textcolor}\rmfamily\fontsize{10.000000}{12.000000}\selectfont \(\displaystyle {3}\)}%
\end{pgfscope}%
\begin{pgfscope}%
\pgfpathrectangle{\pgfqpoint{0.625000in}{0.660000in}}{\pgfqpoint{3.875000in}{2.310000in}}%
\pgfusepath{clip}%
\pgfsetrectcap%
\pgfsetroundjoin%
\pgfsetlinewidth{0.803000pt}%
\definecolor{currentstroke}{rgb}{0.690196,0.690196,0.690196}%
\pgfsetstrokecolor{currentstroke}%
\pgfsetdash{}{0pt}%
\pgfpathmoveto{\pgfqpoint{0.625000in}{1.177434in}}%
\pgfpathlineto{\pgfqpoint{4.500000in}{1.177434in}}%
\pgfusepath{stroke}%
\end{pgfscope}%
\begin{pgfscope}%
\pgfsetbuttcap%
\pgfsetroundjoin%
\definecolor{currentfill}{rgb}{0.000000,0.000000,0.000000}%
\pgfsetfillcolor{currentfill}%
\pgfsetlinewidth{0.803000pt}%
\definecolor{currentstroke}{rgb}{0.000000,0.000000,0.000000}%
\pgfsetstrokecolor{currentstroke}%
\pgfsetdash{}{0pt}%
\pgfsys@defobject{currentmarker}{\pgfqpoint{-0.048611in}{0.000000in}}{\pgfqpoint{-0.000000in}{0.000000in}}{%
\pgfpathmoveto{\pgfqpoint{-0.000000in}{0.000000in}}%
\pgfpathlineto{\pgfqpoint{-0.048611in}{0.000000in}}%
\pgfusepath{stroke,fill}%
}%
\begin{pgfscope}%
\pgfsys@transformshift{0.625000in}{1.177434in}%
\pgfsys@useobject{currentmarker}{}%
\end{pgfscope}%
\end{pgfscope}%
\begin{pgfscope}%
\definecolor{textcolor}{rgb}{0.000000,0.000000,0.000000}%
\pgfsetstrokecolor{textcolor}%
\pgfsetfillcolor{textcolor}%
\pgftext[x=0.458333in, y=1.126334in, left, base]{\color{textcolor}\rmfamily\fontsize{10.000000}{12.000000}\selectfont \(\displaystyle {4}\)}%
\end{pgfscope}%
\begin{pgfscope}%
\pgfpathrectangle{\pgfqpoint{0.625000in}{0.660000in}}{\pgfqpoint{3.875000in}{2.310000in}}%
\pgfusepath{clip}%
\pgfsetrectcap%
\pgfsetroundjoin%
\pgfsetlinewidth{0.803000pt}%
\definecolor{currentstroke}{rgb}{0.690196,0.690196,0.690196}%
\pgfsetstrokecolor{currentstroke}%
\pgfsetdash{}{0pt}%
\pgfpathmoveto{\pgfqpoint{0.625000in}{0.779086in}}%
\pgfpathlineto{\pgfqpoint{4.500000in}{0.779086in}}%
\pgfusepath{stroke}%
\end{pgfscope}%
\begin{pgfscope}%
\pgfsetbuttcap%
\pgfsetroundjoin%
\definecolor{currentfill}{rgb}{0.000000,0.000000,0.000000}%
\pgfsetfillcolor{currentfill}%
\pgfsetlinewidth{0.803000pt}%
\definecolor{currentstroke}{rgb}{0.000000,0.000000,0.000000}%
\pgfsetstrokecolor{currentstroke}%
\pgfsetdash{}{0pt}%
\pgfsys@defobject{currentmarker}{\pgfqpoint{-0.048611in}{0.000000in}}{\pgfqpoint{-0.000000in}{0.000000in}}{%
\pgfpathmoveto{\pgfqpoint{-0.000000in}{0.000000in}}%
\pgfpathlineto{\pgfqpoint{-0.048611in}{0.000000in}}%
\pgfusepath{stroke,fill}%
}%
\begin{pgfscope}%
\pgfsys@transformshift{0.625000in}{0.779086in}%
\pgfsys@useobject{currentmarker}{}%
\end{pgfscope}%
\end{pgfscope}%
\begin{pgfscope}%
\definecolor{textcolor}{rgb}{0.000000,0.000000,0.000000}%
\pgfsetstrokecolor{textcolor}%
\pgfsetfillcolor{textcolor}%
\pgftext[x=0.458333in, y=0.727986in, left, base]{\color{textcolor}\rmfamily\fontsize{10.000000}{12.000000}\selectfont \(\displaystyle {5}\)}%
\end{pgfscope}%
\begin{pgfscope}%
\definecolor{textcolor}{rgb}{0.000000,0.000000,0.000000}%
\pgfsetstrokecolor{textcolor}%
\pgfsetfillcolor{textcolor}%
\pgftext[x=0.402777in,y=1.815000in,,bottom,rotate=90.000000]{\color{textcolor}\rmfamily\fontsize{10.000000}{12.000000}\selectfont time in s}%
\end{pgfscope}%
\begin{pgfscope}%
\pgfpathrectangle{\pgfqpoint{0.625000in}{0.660000in}}{\pgfqpoint{3.875000in}{2.310000in}}%
\pgfusepath{clip}%
\pgfsetrectcap%
\pgfsetroundjoin%
\pgfsetlinewidth{1.505625pt}%
\definecolor{currentstroke}{rgb}{0.121569,0.466667,0.705882}%
\pgfsetstrokecolor{currentstroke}%
\pgfsetdash{}{0pt}%
\pgfpathmoveto{\pgfqpoint{1.393547in}{2.980000in}}%
\pgfpathlineto{\pgfqpoint{1.394485in}{2.764054in}}%
\pgfpathlineto{\pgfqpoint{1.398334in}{2.758079in}}%
\pgfpathlineto{\pgfqpoint{1.405485in}{2.751307in}}%
\pgfpathlineto{\pgfqpoint{1.416889in}{2.743738in}}%
\pgfpathlineto{\pgfqpoint{1.434403in}{2.734975in}}%
\pgfpathlineto{\pgfqpoint{1.459552in}{2.725016in}}%
\pgfpathlineto{\pgfqpoint{1.495025in}{2.713464in}}%
\pgfpathlineto{\pgfqpoint{1.545478in}{2.699522in}}%
\pgfpathlineto{\pgfqpoint{1.617833in}{2.681994in}}%
\pgfpathlineto{\pgfqpoint{1.750732in}{2.652517in}}%
\pgfpathlineto{\pgfqpoint{1.883778in}{2.622242in}}%
\pgfpathlineto{\pgfqpoint{1.954765in}{2.603918in}}%
\pgfpathlineto{\pgfqpoint{2.005103in}{2.588781in}}%
\pgfpathlineto{\pgfqpoint{2.041552in}{2.575636in}}%
\pgfpathlineto{\pgfqpoint{2.067420in}{2.564083in}}%
\pgfpathlineto{\pgfqpoint{2.085402in}{2.553726in}}%
\pgfpathlineto{\pgfqpoint{2.097046in}{2.544564in}}%
\pgfpathlineto{\pgfqpoint{2.103844in}{2.536597in}}%
\pgfpathlineto{\pgfqpoint{2.107285in}{2.529427in}}%
\pgfpathlineto{\pgfqpoint{2.108192in}{2.522655in}}%
\pgfpathlineto{\pgfqpoint{2.106824in}{2.515883in}}%
\pgfpathlineto{\pgfqpoint{2.103191in}{2.509111in}}%
\pgfpathlineto{\pgfqpoint{2.096476in}{2.501543in}}%
\pgfpathlineto{\pgfqpoint{2.085846in}{2.493177in}}%
\pgfpathlineto{\pgfqpoint{2.070432in}{2.484015in}}%
\pgfpathlineto{\pgfqpoint{2.048438in}{2.473658in}}%
\pgfpathlineto{\pgfqpoint{2.018501in}{2.462106in}}%
\pgfpathlineto{\pgfqpoint{1.978077in}{2.448961in}}%
\pgfpathlineto{\pgfqpoint{1.922717in}{2.433425in}}%
\pgfpathlineto{\pgfqpoint{1.845794in}{2.414305in}}%
\pgfpathlineto{\pgfqpoint{1.703187in}{2.381640in}}%
\pgfpathlineto{\pgfqpoint{1.589138in}{2.354552in}}%
\pgfpathlineto{\pgfqpoint{1.526271in}{2.337423in}}%
\pgfpathlineto{\pgfqpoint{1.484496in}{2.323880in}}%
\pgfpathlineto{\pgfqpoint{1.455669in}{2.312327in}}%
\pgfpathlineto{\pgfqpoint{1.436486in}{2.302369in}}%
\pgfpathlineto{\pgfqpoint{1.424275in}{2.293605in}}%
\pgfpathlineto{\pgfqpoint{1.417405in}{2.286037in}}%
\pgfpathlineto{\pgfqpoint{1.414214in}{2.279265in}}%
\pgfpathlineto{\pgfqpoint{1.413740in}{2.273289in}}%
\pgfpathlineto{\pgfqpoint{1.415462in}{2.267314in}}%
\pgfpathlineto{\pgfqpoint{1.419700in}{2.260941in}}%
\pgfpathlineto{\pgfqpoint{1.427381in}{2.253770in}}%
\pgfpathlineto{\pgfqpoint{1.439422in}{2.245803in}}%
\pgfpathlineto{\pgfqpoint{1.457618in}{2.236641in}}%
\pgfpathlineto{\pgfqpoint{1.483382in}{2.226284in}}%
\pgfpathlineto{\pgfqpoint{1.520510in}{2.213936in}}%
\pgfpathlineto{\pgfqpoint{1.572385in}{2.199197in}}%
\pgfpathlineto{\pgfqpoint{1.648432in}{2.180076in}}%
\pgfpathlineto{\pgfqpoint{1.920357in}{2.113552in}}%
\pgfpathlineto{\pgfqpoint{1.972807in}{2.097618in}}%
\pgfpathlineto{\pgfqpoint{2.011375in}{2.083676in}}%
\pgfpathlineto{\pgfqpoint{2.038285in}{2.071725in}}%
\pgfpathlineto{\pgfqpoint{2.057280in}{2.060970in}}%
\pgfpathlineto{\pgfqpoint{2.069818in}{2.051410in}}%
\pgfpathlineto{\pgfqpoint{2.077340in}{2.043044in}}%
\pgfpathlineto{\pgfqpoint{2.081330in}{2.035476in}}%
\pgfpathlineto{\pgfqpoint{2.082624in}{2.028305in}}%
\pgfpathlineto{\pgfqpoint{2.081623in}{2.021533in}}%
\pgfpathlineto{\pgfqpoint{2.078220in}{2.014363in}}%
\pgfpathlineto{\pgfqpoint{2.072037in}{2.006795in}}%
\pgfpathlineto{\pgfqpoint{2.062158in}{1.998429in}}%
\pgfpathlineto{\pgfqpoint{2.047762in}{1.989267in}}%
\pgfpathlineto{\pgfqpoint{2.027158in}{1.978910in}}%
\pgfpathlineto{\pgfqpoint{1.999064in}{1.967358in}}%
\pgfpathlineto{\pgfqpoint{1.961096in}{1.954213in}}%
\pgfpathlineto{\pgfqpoint{1.909092in}{1.938677in}}%
\pgfpathlineto{\pgfqpoint{1.836870in}{1.919556in}}%
\pgfpathlineto{\pgfqpoint{1.698294in}{1.885697in}}%
\pgfpathlineto{\pgfqpoint{1.595087in}{1.859406in}}%
\pgfpathlineto{\pgfqpoint{1.537867in}{1.842675in}}%
\pgfpathlineto{\pgfqpoint{1.498924in}{1.829131in}}%
\pgfpathlineto{\pgfqpoint{1.472082in}{1.817579in}}%
\pgfpathlineto{\pgfqpoint{1.454241in}{1.807621in}}%
\pgfpathlineto{\pgfqpoint{1.442904in}{1.798857in}}%
\pgfpathlineto{\pgfqpoint{1.436542in}{1.791288in}}%
\pgfpathlineto{\pgfqpoint{1.433608in}{1.784516in}}%
\pgfpathlineto{\pgfqpoint{1.433248in}{1.778143in}}%
\pgfpathlineto{\pgfqpoint{1.435217in}{1.771769in}}%
\pgfpathlineto{\pgfqpoint{1.439837in}{1.764997in}}%
\pgfpathlineto{\pgfqpoint{1.447513in}{1.757827in}}%
\pgfpathlineto{\pgfqpoint{1.459970in}{1.749462in}}%
\pgfpathlineto{\pgfqpoint{1.478584in}{1.739901in}}%
\pgfpathlineto{\pgfqpoint{1.504667in}{1.729146in}}%
\pgfpathlineto{\pgfqpoint{1.541782in}{1.716399in}}%
\pgfpathlineto{\pgfqpoint{1.594403in}{1.700863in}}%
\pgfpathlineto{\pgfqpoint{1.674678in}{1.679751in}}%
\pgfpathlineto{\pgfqpoint{1.885953in}{1.625177in}}%
\pgfpathlineto{\pgfqpoint{1.941716in}{1.608048in}}%
\pgfpathlineto{\pgfqpoint{1.981379in}{1.593708in}}%
\pgfpathlineto{\pgfqpoint{2.010385in}{1.580961in}}%
\pgfpathlineto{\pgfqpoint{2.030489in}{1.569807in}}%
\pgfpathlineto{\pgfqpoint{2.044041in}{1.559848in}}%
\pgfpathlineto{\pgfqpoint{2.052419in}{1.551085in}}%
\pgfpathlineto{\pgfqpoint{2.057097in}{1.543118in}}%
\pgfpathlineto{\pgfqpoint{2.058892in}{1.535947in}}%
\pgfpathlineto{\pgfqpoint{2.058392in}{1.528777in}}%
\pgfpathlineto{\pgfqpoint{2.055604in}{1.521607in}}%
\pgfpathlineto{\pgfqpoint{2.050201in}{1.514038in}}%
\pgfpathlineto{\pgfqpoint{2.041337in}{1.505673in}}%
\pgfpathlineto{\pgfqpoint{2.028229in}{1.496511in}}%
\pgfpathlineto{\pgfqpoint{2.009296in}{1.486154in}}%
\pgfpathlineto{\pgfqpoint{1.983318in}{1.474602in}}%
\pgfpathlineto{\pgfqpoint{1.948053in}{1.461456in}}%
\pgfpathlineto{\pgfqpoint{1.899594in}{1.445921in}}%
\pgfpathlineto{\pgfqpoint{1.832130in}{1.426800in}}%
\pgfpathlineto{\pgfqpoint{1.706970in}{1.394135in}}%
\pgfpathlineto{\pgfqpoint{1.605476in}{1.366649in}}%
\pgfpathlineto{\pgfqpoint{1.550427in}{1.349520in}}%
\pgfpathlineto{\pgfqpoint{1.513843in}{1.335977in}}%
\pgfpathlineto{\pgfqpoint{1.488574in}{1.324425in}}%
\pgfpathlineto{\pgfqpoint{1.471725in}{1.314466in}}%
\pgfpathlineto{\pgfqpoint{1.460959in}{1.305702in}}%
\pgfpathlineto{\pgfqpoint{1.454857in}{1.298134in}}%
\pgfpathlineto{\pgfqpoint{1.451873in}{1.290963in}}%
\pgfpathlineto{\pgfqpoint{1.451529in}{1.284590in}}%
\pgfpathlineto{\pgfqpoint{1.453357in}{1.278216in}}%
\pgfpathlineto{\pgfqpoint{1.457657in}{1.271444in}}%
\pgfpathlineto{\pgfqpoint{1.465281in}{1.263876in}}%
\pgfpathlineto{\pgfqpoint{1.477055in}{1.255510in}}%
\pgfpathlineto{\pgfqpoint{1.494589in}{1.245950in}}%
\pgfpathlineto{\pgfqpoint{1.520101in}{1.234796in}}%
\pgfpathlineto{\pgfqpoint{1.555111in}{1.222049in}}%
\pgfpathlineto{\pgfqpoint{1.604613in}{1.206514in}}%
\pgfpathlineto{\pgfqpoint{1.679960in}{1.185401in}}%
\pgfpathlineto{\pgfqpoint{1.875088in}{1.131624in}}%
\pgfpathlineto{\pgfqpoint{1.928574in}{1.114097in}}%
\pgfpathlineto{\pgfqpoint{1.965634in}{1.099756in}}%
\pgfpathlineto{\pgfqpoint{1.992667in}{1.087009in}}%
\pgfpathlineto{\pgfqpoint{2.011327in}{1.075855in}}%
\pgfpathlineto{\pgfqpoint{2.023822in}{1.065897in}}%
\pgfpathlineto{\pgfqpoint{2.031457in}{1.057133in}}%
\pgfpathlineto{\pgfqpoint{2.035614in}{1.049166in}}%
\pgfpathlineto{\pgfqpoint{2.037084in}{1.041597in}}%
\pgfpathlineto{\pgfqpoint{2.036244in}{1.034427in}}%
\pgfpathlineto{\pgfqpoint{2.033012in}{1.026859in}}%
\pgfpathlineto{\pgfqpoint{2.027033in}{1.018892in}}%
\pgfpathlineto{\pgfqpoint{2.017463in}{1.010128in}}%
\pgfpathlineto{\pgfqpoint{2.003552in}{1.000568in}}%
\pgfpathlineto{\pgfqpoint{1.983763in}{0.989812in}}%
\pgfpathlineto{\pgfqpoint{1.956978in}{0.977862in}}%
\pgfpathlineto{\pgfqpoint{1.919984in}{0.963920in}}%
\pgfpathlineto{\pgfqpoint{1.870029in}{0.947587in}}%
\pgfpathlineto{\pgfqpoint{1.796410in}{0.926077in}}%
\pgfpathlineto{\pgfqpoint{1.594586in}{0.868316in}}%
\pgfpathlineto{\pgfqpoint{1.549036in}{0.852382in}}%
\pgfpathlineto{\pgfqpoint{1.517924in}{0.839237in}}%
\pgfpathlineto{\pgfqpoint{1.496957in}{0.828083in}}%
\pgfpathlineto{\pgfqpoint{1.483373in}{0.818523in}}%
\pgfpathlineto{\pgfqpoint{1.475017in}{0.810157in}}%
\pgfpathlineto{\pgfqpoint{1.470397in}{0.802589in}}%
\pgfpathlineto{\pgfqpoint{1.468673in}{0.795817in}}%
\pgfpathlineto{\pgfqpoint{1.469240in}{0.789045in}}%
\pgfpathlineto{\pgfqpoint{1.472088in}{0.782273in}}%
\pgfpathlineto{\pgfqpoint{1.473916in}{0.779484in}}%
\pgfpathlineto{\pgfqpoint{1.473916in}{0.779484in}}%
\pgfusepath{stroke}%
\end{pgfscope}%
\begin{pgfscope}%
\pgfpathrectangle{\pgfqpoint{0.625000in}{0.660000in}}{\pgfqpoint{3.875000in}{2.310000in}}%
\pgfusepath{clip}%
\pgfsetbuttcap%
\pgfsetroundjoin%
\pgfsetlinewidth{1.505625pt}%
\definecolor{currentstroke}{rgb}{0.121569,0.466667,0.705882}%
\pgfsetstrokecolor{currentstroke}%
\pgfsetdash{{5.550000pt}{2.400000pt}}{0.000000pt}%
\pgfpathmoveto{\pgfqpoint{1.747563in}{0.660000in}}%
\pgfpathlineto{\pgfqpoint{1.747563in}{2.970000in}}%
\pgfusepath{stroke}%
\end{pgfscope}%
\begin{pgfscope}%
\pgfsetrectcap%
\pgfsetmiterjoin%
\pgfsetlinewidth{0.803000pt}%
\definecolor{currentstroke}{rgb}{0.000000,0.000000,0.000000}%
\pgfsetstrokecolor{currentstroke}%
\pgfsetdash{}{0pt}%
\pgfpathmoveto{\pgfqpoint{0.625000in}{0.660000in}}%
\pgfpathlineto{\pgfqpoint{0.625000in}{2.970000in}}%
\pgfusepath{stroke}%
\end{pgfscope}%
\begin{pgfscope}%
\pgfsetrectcap%
\pgfsetmiterjoin%
\pgfsetlinewidth{0.803000pt}%
\definecolor{currentstroke}{rgb}{0.000000,0.000000,0.000000}%
\pgfsetstrokecolor{currentstroke}%
\pgfsetdash{}{0pt}%
\pgfpathmoveto{\pgfqpoint{4.500000in}{0.660000in}}%
\pgfpathlineto{\pgfqpoint{4.500000in}{2.970000in}}%
\pgfusepath{stroke}%
\end{pgfscope}%
\begin{pgfscope}%
\pgfsetrectcap%
\pgfsetmiterjoin%
\pgfsetlinewidth{0.803000pt}%
\definecolor{currentstroke}{rgb}{0.000000,0.000000,0.000000}%
\pgfsetstrokecolor{currentstroke}%
\pgfsetdash{}{0pt}%
\pgfpathmoveto{\pgfqpoint{0.625000in}{0.660000in}}%
\pgfpathlineto{\pgfqpoint{4.500000in}{0.660000in}}%
\pgfusepath{stroke}%
\end{pgfscope}%
\begin{pgfscope}%
\pgfsetrectcap%
\pgfsetmiterjoin%
\pgfsetlinewidth{0.803000pt}%
\definecolor{currentstroke}{rgb}{0.000000,0.000000,0.000000}%
\pgfsetstrokecolor{currentstroke}%
\pgfsetdash{}{0pt}%
\pgfpathmoveto{\pgfqpoint{0.625000in}{2.970000in}}%
\pgfpathlineto{\pgfqpoint{4.500000in}{2.970000in}}%
\pgfusepath{stroke}%
\end{pgfscope}%
\begin{pgfscope}%
\pgfsetbuttcap%
\pgfsetmiterjoin%
\definecolor{currentfill}{rgb}{1.000000,1.000000,1.000000}%
\pgfsetfillcolor{currentfill}%
\pgfsetfillopacity{0.800000}%
\pgfsetlinewidth{1.003750pt}%
\definecolor{currentstroke}{rgb}{0.800000,0.800000,0.800000}%
\pgfsetstrokecolor{currentstroke}%
\pgfsetstrokeopacity{0.800000}%
\pgfsetdash{}{0pt}%
\pgfpathmoveto{\pgfqpoint{2.600708in}{2.455108in}}%
\pgfpathlineto{\pgfqpoint{4.402778in}{2.455108in}}%
\pgfpathquadraticcurveto{\pgfqpoint{4.430556in}{2.455108in}}{\pgfqpoint{4.430556in}{2.482886in}}%
\pgfpathlineto{\pgfqpoint{4.430556in}{2.872778in}}%
\pgfpathquadraticcurveto{\pgfqpoint{4.430556in}{2.900556in}}{\pgfqpoint{4.402778in}{2.900556in}}%
\pgfpathlineto{\pgfqpoint{2.600708in}{2.900556in}}%
\pgfpathquadraticcurveto{\pgfqpoint{2.572930in}{2.900556in}}{\pgfqpoint{2.572930in}{2.872778in}}%
\pgfpathlineto{\pgfqpoint{2.572930in}{2.482886in}}%
\pgfpathquadraticcurveto{\pgfqpoint{2.572930in}{2.455108in}}{\pgfqpoint{2.600708in}{2.455108in}}%
\pgfpathlineto{\pgfqpoint{2.600708in}{2.455108in}}%
\pgfpathclose%
\pgfusepath{stroke,fill}%
\end{pgfscope}%
\begin{pgfscope}%
\pgfsetrectcap%
\pgfsetroundjoin%
\pgfsetlinewidth{1.505625pt}%
\definecolor{currentstroke}{rgb}{0.121569,0.466667,0.705882}%
\pgfsetstrokecolor{currentstroke}%
\pgfsetdash{}{0pt}%
\pgfpathmoveto{\pgfqpoint{2.628486in}{2.791411in}}%
\pgfpathlineto{\pgfqpoint{2.767375in}{2.791411in}}%
\pgfpathlineto{\pgfqpoint{2.906264in}{2.791411in}}%
\pgfusepath{stroke}%
\end{pgfscope}%
\begin{pgfscope}%
\definecolor{textcolor}{rgb}{0.000000,0.000000,0.000000}%
\pgfsetstrokecolor{textcolor}%
\pgfsetfillcolor{textcolor}%
\pgftext[x=3.017375in,y=2.742800in,left,base]{\color{textcolor}\rmfamily\fontsize{10.000000}{12.000000}\selectfont delta}%
\end{pgfscope}%
\begin{pgfscope}%
\pgfsetbuttcap%
\pgfsetroundjoin%
\pgfsetlinewidth{1.505625pt}%
\definecolor{currentstroke}{rgb}{0.121569,0.466667,0.705882}%
\pgfsetstrokecolor{currentstroke}%
\pgfsetdash{{5.550000pt}{2.400000pt}}{0.000000pt}%
\pgfpathmoveto{\pgfqpoint{2.628486in}{2.589521in}}%
\pgfpathlineto{\pgfqpoint{2.767375in}{2.589521in}}%
\pgfpathlineto{\pgfqpoint{2.906264in}{2.589521in}}%
\pgfusepath{stroke}%
\end{pgfscope}%
\begin{pgfscope}%
\definecolor{textcolor}{rgb}{0.000000,0.000000,0.000000}%
\pgfsetstrokecolor{textcolor}%
\pgfsetfillcolor{textcolor}%
\pgftext[x=3.017375in,y=2.540909in,left,base]{\color{textcolor}\rmfamily\fontsize{10.000000}{12.000000}\selectfont new stable convergent}%
\end{pgfscope}%
\begin{pgfscope}%
\definecolor{textcolor}{rgb}{0.000000,0.000000,0.000000}%
\pgfsetstrokecolor{textcolor}%
\pgfsetfillcolor{textcolor}%
\pgftext[x=2.500000in,y=5.880000in,,top]{\color{textcolor}\rmfamily\fontsize{12.000000}{14.400000}\selectfont Stable scenario - fault 3}%
\end{pgfscope}%
\end{pgfpicture}%
\makeatother%
\endgroup%


%% Creator: Matplotlib, PGF backend
%%
%% To include the figure in your LaTeX document, write
%%   \input{<filename>.pgf}
%%
%% Make sure the required packages are loaded in your preamble
%%   \usepackage{pgf}
%%
%% Also ensure that all the required font packages are loaded; for instance,
%% the lmodern package is sometimes necessary when using math font.
%%   \usepackage{lmodern}
%%
%% Figures using additional raster images can only be included by \input if
%% they are in the same directory as the main LaTeX file. For loading figures
%% from other directories you can use the `import` package
%%   \usepackage{import}
%%
%% and then include the figures with
%%   \import{<path to file>}{<filename>.pgf}
%%
%% Matplotlib used the following preamble
%%   
%%   \usepackage{fontspec}
%%   \setmainfont{Charter.ttc}[Path=\detokenize{/System/Library/Fonts/Supplemental/}]
%%   \setsansfont{DejaVuSans.ttf}[Path=\detokenize{/opt/homebrew/lib/python3.10/site-packages/matplotlib/mpl-data/fonts/ttf/}]
%%   \setmonofont{DejaVuSansMono.ttf}[Path=\detokenize{/opt/homebrew/lib/python3.10/site-packages/matplotlib/mpl-data/fonts/ttf/}]
%%   \makeatletter\@ifpackageloaded{underscore}{}{\usepackage[strings]{underscore}}\makeatother
%%
\begingroup%
\makeatletter%
\begin{pgfpicture}%
\pgfpathrectangle{\pgfpointorigin}{\pgfqpoint{6.400000in}{4.800000in}}%
\pgfusepath{use as bounding box, clip}%
\begin{pgfscope}%
\pgfsetbuttcap%
\pgfsetmiterjoin%
\definecolor{currentfill}{rgb}{1.000000,1.000000,1.000000}%
\pgfsetfillcolor{currentfill}%
\pgfsetlinewidth{0.000000pt}%
\definecolor{currentstroke}{rgb}{1.000000,1.000000,1.000000}%
\pgfsetstrokecolor{currentstroke}%
\pgfsetdash{}{0pt}%
\pgfpathmoveto{\pgfqpoint{0.000000in}{0.000000in}}%
\pgfpathlineto{\pgfqpoint{6.400000in}{0.000000in}}%
\pgfpathlineto{\pgfqpoint{6.400000in}{4.800000in}}%
\pgfpathlineto{\pgfqpoint{0.000000in}{4.800000in}}%
\pgfpathlineto{\pgfqpoint{0.000000in}{0.000000in}}%
\pgfpathclose%
\pgfusepath{fill}%
\end{pgfscope}%
\begin{pgfscope}%
\pgfsetbuttcap%
\pgfsetmiterjoin%
\definecolor{currentfill}{rgb}{1.000000,1.000000,1.000000}%
\pgfsetfillcolor{currentfill}%
\pgfsetlinewidth{0.000000pt}%
\definecolor{currentstroke}{rgb}{0.000000,0.000000,0.000000}%
\pgfsetstrokecolor{currentstroke}%
\pgfsetstrokeopacity{0.000000}%
\pgfsetdash{}{0pt}%
\pgfpathmoveto{\pgfqpoint{0.800000in}{0.528000in}}%
\pgfpathlineto{\pgfqpoint{5.760000in}{0.528000in}}%
\pgfpathlineto{\pgfqpoint{5.760000in}{4.224000in}}%
\pgfpathlineto{\pgfqpoint{0.800000in}{4.224000in}}%
\pgfpathlineto{\pgfqpoint{0.800000in}{0.528000in}}%
\pgfpathclose%
\pgfusepath{fill}%
\end{pgfscope}%
\begin{pgfscope}%
\pgfpathrectangle{\pgfqpoint{0.800000in}{0.528000in}}{\pgfqpoint{4.960000in}{3.696000in}}%
\pgfusepath{clip}%
\pgfsetrectcap%
\pgfsetroundjoin%
\pgfsetlinewidth{0.803000pt}%
\definecolor{currentstroke}{rgb}{0.690196,0.690196,0.690196}%
\pgfsetstrokecolor{currentstroke}%
\pgfsetdash{}{0pt}%
\pgfpathmoveto{\pgfqpoint{1.056511in}{0.528000in}}%
\pgfpathlineto{\pgfqpoint{1.056511in}{4.224000in}}%
\pgfusepath{stroke}%
\end{pgfscope}%
\begin{pgfscope}%
\pgfsetbuttcap%
\pgfsetroundjoin%
\definecolor{currentfill}{rgb}{0.000000,0.000000,0.000000}%
\pgfsetfillcolor{currentfill}%
\pgfsetlinewidth{0.803000pt}%
\definecolor{currentstroke}{rgb}{0.000000,0.000000,0.000000}%
\pgfsetstrokecolor{currentstroke}%
\pgfsetdash{}{0pt}%
\pgfsys@defobject{currentmarker}{\pgfqpoint{0.000000in}{-0.048611in}}{\pgfqpoint{0.000000in}{0.000000in}}{%
\pgfpathmoveto{\pgfqpoint{0.000000in}{0.000000in}}%
\pgfpathlineto{\pgfqpoint{0.000000in}{-0.048611in}}%
\pgfusepath{stroke,fill}%
}%
\begin{pgfscope}%
\pgfsys@transformshift{1.056511in}{0.528000in}%
\pgfsys@useobject{currentmarker}{}%
\end{pgfscope}%
\end{pgfscope}%
\begin{pgfscope}%
\definecolor{textcolor}{rgb}{0.000000,0.000000,0.000000}%
\pgfsetstrokecolor{textcolor}%
\pgfsetfillcolor{textcolor}%
\pgftext[x=1.056511in,y=0.430778in,,top]{\color{textcolor}\rmfamily\fontsize{10.000000}{12.000000}\selectfont \(\displaystyle {\ensuremath{-}1}\)}%
\end{pgfscope}%
\begin{pgfscope}%
\pgfpathrectangle{\pgfqpoint{0.800000in}{0.528000in}}{\pgfqpoint{4.960000in}{3.696000in}}%
\pgfusepath{clip}%
\pgfsetrectcap%
\pgfsetroundjoin%
\pgfsetlinewidth{0.803000pt}%
\definecolor{currentstroke}{rgb}{0.690196,0.690196,0.690196}%
\pgfsetstrokecolor{currentstroke}%
\pgfsetdash{}{0pt}%
\pgfpathmoveto{\pgfqpoint{1.911691in}{0.528000in}}%
\pgfpathlineto{\pgfqpoint{1.911691in}{4.224000in}}%
\pgfusepath{stroke}%
\end{pgfscope}%
\begin{pgfscope}%
\pgfsetbuttcap%
\pgfsetroundjoin%
\definecolor{currentfill}{rgb}{0.000000,0.000000,0.000000}%
\pgfsetfillcolor{currentfill}%
\pgfsetlinewidth{0.803000pt}%
\definecolor{currentstroke}{rgb}{0.000000,0.000000,0.000000}%
\pgfsetstrokecolor{currentstroke}%
\pgfsetdash{}{0pt}%
\pgfsys@defobject{currentmarker}{\pgfqpoint{0.000000in}{-0.048611in}}{\pgfqpoint{0.000000in}{0.000000in}}{%
\pgfpathmoveto{\pgfqpoint{0.000000in}{0.000000in}}%
\pgfpathlineto{\pgfqpoint{0.000000in}{-0.048611in}}%
\pgfusepath{stroke,fill}%
}%
\begin{pgfscope}%
\pgfsys@transformshift{1.911691in}{0.528000in}%
\pgfsys@useobject{currentmarker}{}%
\end{pgfscope}%
\end{pgfscope}%
\begin{pgfscope}%
\definecolor{textcolor}{rgb}{0.000000,0.000000,0.000000}%
\pgfsetstrokecolor{textcolor}%
\pgfsetfillcolor{textcolor}%
\pgftext[x=1.911691in,y=0.430778in,,top]{\color{textcolor}\rmfamily\fontsize{10.000000}{12.000000}\selectfont \(\displaystyle {0}\)}%
\end{pgfscope}%
\begin{pgfscope}%
\pgfpathrectangle{\pgfqpoint{0.800000in}{0.528000in}}{\pgfqpoint{4.960000in}{3.696000in}}%
\pgfusepath{clip}%
\pgfsetrectcap%
\pgfsetroundjoin%
\pgfsetlinewidth{0.803000pt}%
\definecolor{currentstroke}{rgb}{0.690196,0.690196,0.690196}%
\pgfsetstrokecolor{currentstroke}%
\pgfsetdash{}{0pt}%
\pgfpathmoveto{\pgfqpoint{2.766871in}{0.528000in}}%
\pgfpathlineto{\pgfqpoint{2.766871in}{4.224000in}}%
\pgfusepath{stroke}%
\end{pgfscope}%
\begin{pgfscope}%
\pgfsetbuttcap%
\pgfsetroundjoin%
\definecolor{currentfill}{rgb}{0.000000,0.000000,0.000000}%
\pgfsetfillcolor{currentfill}%
\pgfsetlinewidth{0.803000pt}%
\definecolor{currentstroke}{rgb}{0.000000,0.000000,0.000000}%
\pgfsetstrokecolor{currentstroke}%
\pgfsetdash{}{0pt}%
\pgfsys@defobject{currentmarker}{\pgfqpoint{0.000000in}{-0.048611in}}{\pgfqpoint{0.000000in}{0.000000in}}{%
\pgfpathmoveto{\pgfqpoint{0.000000in}{0.000000in}}%
\pgfpathlineto{\pgfqpoint{0.000000in}{-0.048611in}}%
\pgfusepath{stroke,fill}%
}%
\begin{pgfscope}%
\pgfsys@transformshift{2.766871in}{0.528000in}%
\pgfsys@useobject{currentmarker}{}%
\end{pgfscope}%
\end{pgfscope}%
\begin{pgfscope}%
\definecolor{textcolor}{rgb}{0.000000,0.000000,0.000000}%
\pgfsetstrokecolor{textcolor}%
\pgfsetfillcolor{textcolor}%
\pgftext[x=2.766871in,y=0.430778in,,top]{\color{textcolor}\rmfamily\fontsize{10.000000}{12.000000}\selectfont \(\displaystyle {1}\)}%
\end{pgfscope}%
\begin{pgfscope}%
\pgfpathrectangle{\pgfqpoint{0.800000in}{0.528000in}}{\pgfqpoint{4.960000in}{3.696000in}}%
\pgfusepath{clip}%
\pgfsetrectcap%
\pgfsetroundjoin%
\pgfsetlinewidth{0.803000pt}%
\definecolor{currentstroke}{rgb}{0.690196,0.690196,0.690196}%
\pgfsetstrokecolor{currentstroke}%
\pgfsetdash{}{0pt}%
\pgfpathmoveto{\pgfqpoint{3.622051in}{0.528000in}}%
\pgfpathlineto{\pgfqpoint{3.622051in}{4.224000in}}%
\pgfusepath{stroke}%
\end{pgfscope}%
\begin{pgfscope}%
\pgfsetbuttcap%
\pgfsetroundjoin%
\definecolor{currentfill}{rgb}{0.000000,0.000000,0.000000}%
\pgfsetfillcolor{currentfill}%
\pgfsetlinewidth{0.803000pt}%
\definecolor{currentstroke}{rgb}{0.000000,0.000000,0.000000}%
\pgfsetstrokecolor{currentstroke}%
\pgfsetdash{}{0pt}%
\pgfsys@defobject{currentmarker}{\pgfqpoint{0.000000in}{-0.048611in}}{\pgfqpoint{0.000000in}{0.000000in}}{%
\pgfpathmoveto{\pgfqpoint{0.000000in}{0.000000in}}%
\pgfpathlineto{\pgfqpoint{0.000000in}{-0.048611in}}%
\pgfusepath{stroke,fill}%
}%
\begin{pgfscope}%
\pgfsys@transformshift{3.622051in}{0.528000in}%
\pgfsys@useobject{currentmarker}{}%
\end{pgfscope}%
\end{pgfscope}%
\begin{pgfscope}%
\definecolor{textcolor}{rgb}{0.000000,0.000000,0.000000}%
\pgfsetstrokecolor{textcolor}%
\pgfsetfillcolor{textcolor}%
\pgftext[x=3.622051in,y=0.430778in,,top]{\color{textcolor}\rmfamily\fontsize{10.000000}{12.000000}\selectfont \(\displaystyle {2}\)}%
\end{pgfscope}%
\begin{pgfscope}%
\pgfpathrectangle{\pgfqpoint{0.800000in}{0.528000in}}{\pgfqpoint{4.960000in}{3.696000in}}%
\pgfusepath{clip}%
\pgfsetrectcap%
\pgfsetroundjoin%
\pgfsetlinewidth{0.803000pt}%
\definecolor{currentstroke}{rgb}{0.690196,0.690196,0.690196}%
\pgfsetstrokecolor{currentstroke}%
\pgfsetdash{}{0pt}%
\pgfpathmoveto{\pgfqpoint{4.477230in}{0.528000in}}%
\pgfpathlineto{\pgfqpoint{4.477230in}{4.224000in}}%
\pgfusepath{stroke}%
\end{pgfscope}%
\begin{pgfscope}%
\pgfsetbuttcap%
\pgfsetroundjoin%
\definecolor{currentfill}{rgb}{0.000000,0.000000,0.000000}%
\pgfsetfillcolor{currentfill}%
\pgfsetlinewidth{0.803000pt}%
\definecolor{currentstroke}{rgb}{0.000000,0.000000,0.000000}%
\pgfsetstrokecolor{currentstroke}%
\pgfsetdash{}{0pt}%
\pgfsys@defobject{currentmarker}{\pgfqpoint{0.000000in}{-0.048611in}}{\pgfqpoint{0.000000in}{0.000000in}}{%
\pgfpathmoveto{\pgfqpoint{0.000000in}{0.000000in}}%
\pgfpathlineto{\pgfqpoint{0.000000in}{-0.048611in}}%
\pgfusepath{stroke,fill}%
}%
\begin{pgfscope}%
\pgfsys@transformshift{4.477230in}{0.528000in}%
\pgfsys@useobject{currentmarker}{}%
\end{pgfscope}%
\end{pgfscope}%
\begin{pgfscope}%
\definecolor{textcolor}{rgb}{0.000000,0.000000,0.000000}%
\pgfsetstrokecolor{textcolor}%
\pgfsetfillcolor{textcolor}%
\pgftext[x=4.477230in,y=0.430778in,,top]{\color{textcolor}\rmfamily\fontsize{10.000000}{12.000000}\selectfont \(\displaystyle {3}\)}%
\end{pgfscope}%
\begin{pgfscope}%
\pgfpathrectangle{\pgfqpoint{0.800000in}{0.528000in}}{\pgfqpoint{4.960000in}{3.696000in}}%
\pgfusepath{clip}%
\pgfsetrectcap%
\pgfsetroundjoin%
\pgfsetlinewidth{0.803000pt}%
\definecolor{currentstroke}{rgb}{0.690196,0.690196,0.690196}%
\pgfsetstrokecolor{currentstroke}%
\pgfsetdash{}{0pt}%
\pgfpathmoveto{\pgfqpoint{5.332410in}{0.528000in}}%
\pgfpathlineto{\pgfqpoint{5.332410in}{4.224000in}}%
\pgfusepath{stroke}%
\end{pgfscope}%
\begin{pgfscope}%
\pgfsetbuttcap%
\pgfsetroundjoin%
\definecolor{currentfill}{rgb}{0.000000,0.000000,0.000000}%
\pgfsetfillcolor{currentfill}%
\pgfsetlinewidth{0.803000pt}%
\definecolor{currentstroke}{rgb}{0.000000,0.000000,0.000000}%
\pgfsetstrokecolor{currentstroke}%
\pgfsetdash{}{0pt}%
\pgfsys@defobject{currentmarker}{\pgfqpoint{0.000000in}{-0.048611in}}{\pgfqpoint{0.000000in}{0.000000in}}{%
\pgfpathmoveto{\pgfqpoint{0.000000in}{0.000000in}}%
\pgfpathlineto{\pgfqpoint{0.000000in}{-0.048611in}}%
\pgfusepath{stroke,fill}%
}%
\begin{pgfscope}%
\pgfsys@transformshift{5.332410in}{0.528000in}%
\pgfsys@useobject{currentmarker}{}%
\end{pgfscope}%
\end{pgfscope}%
\begin{pgfscope}%
\definecolor{textcolor}{rgb}{0.000000,0.000000,0.000000}%
\pgfsetstrokecolor{textcolor}%
\pgfsetfillcolor{textcolor}%
\pgftext[x=5.332410in,y=0.430778in,,top]{\color{textcolor}\rmfamily\fontsize{10.000000}{12.000000}\selectfont \(\displaystyle {4}\)}%
\end{pgfscope}%
\begin{pgfscope}%
\definecolor{textcolor}{rgb}{0.000000,0.000000,0.000000}%
\pgfsetstrokecolor{textcolor}%
\pgfsetfillcolor{textcolor}%
\pgftext[x=3.280000in,y=0.242776in,,top]{\color{textcolor}\rmfamily\fontsize{10.000000}{12.000000}\selectfont time in s}%
\end{pgfscope}%
\begin{pgfscope}%
\pgfpathrectangle{\pgfqpoint{0.800000in}{0.528000in}}{\pgfqpoint{4.960000in}{3.696000in}}%
\pgfusepath{clip}%
\pgfsetrectcap%
\pgfsetroundjoin%
\pgfsetlinewidth{0.803000pt}%
\definecolor{currentstroke}{rgb}{0.690196,0.690196,0.690196}%
\pgfsetstrokecolor{currentstroke}%
\pgfsetdash{}{0pt}%
\pgfpathmoveto{\pgfqpoint{0.800000in}{0.528000in}}%
\pgfpathlineto{\pgfqpoint{5.760000in}{0.528000in}}%
\pgfusepath{stroke}%
\end{pgfscope}%
\begin{pgfscope}%
\pgfsetbuttcap%
\pgfsetroundjoin%
\definecolor{currentfill}{rgb}{0.000000,0.000000,0.000000}%
\pgfsetfillcolor{currentfill}%
\pgfsetlinewidth{0.803000pt}%
\definecolor{currentstroke}{rgb}{0.000000,0.000000,0.000000}%
\pgfsetstrokecolor{currentstroke}%
\pgfsetdash{}{0pt}%
\pgfsys@defobject{currentmarker}{\pgfqpoint{-0.048611in}{0.000000in}}{\pgfqpoint{-0.000000in}{0.000000in}}{%
\pgfpathmoveto{\pgfqpoint{-0.000000in}{0.000000in}}%
\pgfpathlineto{\pgfqpoint{-0.048611in}{0.000000in}}%
\pgfusepath{stroke,fill}%
}%
\begin{pgfscope}%
\pgfsys@transformshift{0.800000in}{0.528000in}%
\pgfsys@useobject{currentmarker}{}%
\end{pgfscope}%
\end{pgfscope}%
\begin{pgfscope}%
\definecolor{textcolor}{rgb}{0.000000,0.000000,0.000000}%
\pgfsetstrokecolor{textcolor}%
\pgfsetfillcolor{textcolor}%
\pgftext[x=0.525308in, y=0.476900in, left, base]{\color{textcolor}\rmfamily\fontsize{10.000000}{12.000000}\selectfont \(\displaystyle {0.0}\)}%
\end{pgfscope}%
\begin{pgfscope}%
\pgfpathrectangle{\pgfqpoint{0.800000in}{0.528000in}}{\pgfqpoint{4.960000in}{3.696000in}}%
\pgfusepath{clip}%
\pgfsetrectcap%
\pgfsetroundjoin%
\pgfsetlinewidth{0.803000pt}%
\definecolor{currentstroke}{rgb}{0.690196,0.690196,0.690196}%
\pgfsetstrokecolor{currentstroke}%
\pgfsetdash{}{0pt}%
\pgfpathmoveto{\pgfqpoint{0.800000in}{0.949984in}}%
\pgfpathlineto{\pgfqpoint{5.760000in}{0.949984in}}%
\pgfusepath{stroke}%
\end{pgfscope}%
\begin{pgfscope}%
\pgfsetbuttcap%
\pgfsetroundjoin%
\definecolor{currentfill}{rgb}{0.000000,0.000000,0.000000}%
\pgfsetfillcolor{currentfill}%
\pgfsetlinewidth{0.803000pt}%
\definecolor{currentstroke}{rgb}{0.000000,0.000000,0.000000}%
\pgfsetstrokecolor{currentstroke}%
\pgfsetdash{}{0pt}%
\pgfsys@defobject{currentmarker}{\pgfqpoint{-0.048611in}{0.000000in}}{\pgfqpoint{-0.000000in}{0.000000in}}{%
\pgfpathmoveto{\pgfqpoint{-0.000000in}{0.000000in}}%
\pgfpathlineto{\pgfqpoint{-0.048611in}{0.000000in}}%
\pgfusepath{stroke,fill}%
}%
\begin{pgfscope}%
\pgfsys@transformshift{0.800000in}{0.949984in}%
\pgfsys@useobject{currentmarker}{}%
\end{pgfscope}%
\end{pgfscope}%
\begin{pgfscope}%
\definecolor{textcolor}{rgb}{0.000000,0.000000,0.000000}%
\pgfsetstrokecolor{textcolor}%
\pgfsetfillcolor{textcolor}%
\pgftext[x=0.525308in, y=0.898884in, left, base]{\color{textcolor}\rmfamily\fontsize{10.000000}{12.000000}\selectfont \(\displaystyle {0.1}\)}%
\end{pgfscope}%
\begin{pgfscope}%
\pgfpathrectangle{\pgfqpoint{0.800000in}{0.528000in}}{\pgfqpoint{4.960000in}{3.696000in}}%
\pgfusepath{clip}%
\pgfsetrectcap%
\pgfsetroundjoin%
\pgfsetlinewidth{0.803000pt}%
\definecolor{currentstroke}{rgb}{0.690196,0.690196,0.690196}%
\pgfsetstrokecolor{currentstroke}%
\pgfsetdash{}{0pt}%
\pgfpathmoveto{\pgfqpoint{0.800000in}{1.371968in}}%
\pgfpathlineto{\pgfqpoint{5.760000in}{1.371968in}}%
\pgfusepath{stroke}%
\end{pgfscope}%
\begin{pgfscope}%
\pgfsetbuttcap%
\pgfsetroundjoin%
\definecolor{currentfill}{rgb}{0.000000,0.000000,0.000000}%
\pgfsetfillcolor{currentfill}%
\pgfsetlinewidth{0.803000pt}%
\definecolor{currentstroke}{rgb}{0.000000,0.000000,0.000000}%
\pgfsetstrokecolor{currentstroke}%
\pgfsetdash{}{0pt}%
\pgfsys@defobject{currentmarker}{\pgfqpoint{-0.048611in}{0.000000in}}{\pgfqpoint{-0.000000in}{0.000000in}}{%
\pgfpathmoveto{\pgfqpoint{-0.000000in}{0.000000in}}%
\pgfpathlineto{\pgfqpoint{-0.048611in}{0.000000in}}%
\pgfusepath{stroke,fill}%
}%
\begin{pgfscope}%
\pgfsys@transformshift{0.800000in}{1.371968in}%
\pgfsys@useobject{currentmarker}{}%
\end{pgfscope}%
\end{pgfscope}%
\begin{pgfscope}%
\definecolor{textcolor}{rgb}{0.000000,0.000000,0.000000}%
\pgfsetstrokecolor{textcolor}%
\pgfsetfillcolor{textcolor}%
\pgftext[x=0.525308in, y=1.320868in, left, base]{\color{textcolor}\rmfamily\fontsize{10.000000}{12.000000}\selectfont \(\displaystyle {0.2}\)}%
\end{pgfscope}%
\begin{pgfscope}%
\pgfpathrectangle{\pgfqpoint{0.800000in}{0.528000in}}{\pgfqpoint{4.960000in}{3.696000in}}%
\pgfusepath{clip}%
\pgfsetrectcap%
\pgfsetroundjoin%
\pgfsetlinewidth{0.803000pt}%
\definecolor{currentstroke}{rgb}{0.690196,0.690196,0.690196}%
\pgfsetstrokecolor{currentstroke}%
\pgfsetdash{}{0pt}%
\pgfpathmoveto{\pgfqpoint{0.800000in}{1.793953in}}%
\pgfpathlineto{\pgfqpoint{5.760000in}{1.793953in}}%
\pgfusepath{stroke}%
\end{pgfscope}%
\begin{pgfscope}%
\pgfsetbuttcap%
\pgfsetroundjoin%
\definecolor{currentfill}{rgb}{0.000000,0.000000,0.000000}%
\pgfsetfillcolor{currentfill}%
\pgfsetlinewidth{0.803000pt}%
\definecolor{currentstroke}{rgb}{0.000000,0.000000,0.000000}%
\pgfsetstrokecolor{currentstroke}%
\pgfsetdash{}{0pt}%
\pgfsys@defobject{currentmarker}{\pgfqpoint{-0.048611in}{0.000000in}}{\pgfqpoint{-0.000000in}{0.000000in}}{%
\pgfpathmoveto{\pgfqpoint{-0.000000in}{0.000000in}}%
\pgfpathlineto{\pgfqpoint{-0.048611in}{0.000000in}}%
\pgfusepath{stroke,fill}%
}%
\begin{pgfscope}%
\pgfsys@transformshift{0.800000in}{1.793953in}%
\pgfsys@useobject{currentmarker}{}%
\end{pgfscope}%
\end{pgfscope}%
\begin{pgfscope}%
\definecolor{textcolor}{rgb}{0.000000,0.000000,0.000000}%
\pgfsetstrokecolor{textcolor}%
\pgfsetfillcolor{textcolor}%
\pgftext[x=0.525308in, y=1.742853in, left, base]{\color{textcolor}\rmfamily\fontsize{10.000000}{12.000000}\selectfont \(\displaystyle {0.3}\)}%
\end{pgfscope}%
\begin{pgfscope}%
\pgfpathrectangle{\pgfqpoint{0.800000in}{0.528000in}}{\pgfqpoint{4.960000in}{3.696000in}}%
\pgfusepath{clip}%
\pgfsetrectcap%
\pgfsetroundjoin%
\pgfsetlinewidth{0.803000pt}%
\definecolor{currentstroke}{rgb}{0.690196,0.690196,0.690196}%
\pgfsetstrokecolor{currentstroke}%
\pgfsetdash{}{0pt}%
\pgfpathmoveto{\pgfqpoint{0.800000in}{2.215937in}}%
\pgfpathlineto{\pgfqpoint{5.760000in}{2.215937in}}%
\pgfusepath{stroke}%
\end{pgfscope}%
\begin{pgfscope}%
\pgfsetbuttcap%
\pgfsetroundjoin%
\definecolor{currentfill}{rgb}{0.000000,0.000000,0.000000}%
\pgfsetfillcolor{currentfill}%
\pgfsetlinewidth{0.803000pt}%
\definecolor{currentstroke}{rgb}{0.000000,0.000000,0.000000}%
\pgfsetstrokecolor{currentstroke}%
\pgfsetdash{}{0pt}%
\pgfsys@defobject{currentmarker}{\pgfqpoint{-0.048611in}{0.000000in}}{\pgfqpoint{-0.000000in}{0.000000in}}{%
\pgfpathmoveto{\pgfqpoint{-0.000000in}{0.000000in}}%
\pgfpathlineto{\pgfqpoint{-0.048611in}{0.000000in}}%
\pgfusepath{stroke,fill}%
}%
\begin{pgfscope}%
\pgfsys@transformshift{0.800000in}{2.215937in}%
\pgfsys@useobject{currentmarker}{}%
\end{pgfscope}%
\end{pgfscope}%
\begin{pgfscope}%
\definecolor{textcolor}{rgb}{0.000000,0.000000,0.000000}%
\pgfsetstrokecolor{textcolor}%
\pgfsetfillcolor{textcolor}%
\pgftext[x=0.525308in, y=2.164837in, left, base]{\color{textcolor}\rmfamily\fontsize{10.000000}{12.000000}\selectfont \(\displaystyle {0.4}\)}%
\end{pgfscope}%
\begin{pgfscope}%
\pgfpathrectangle{\pgfqpoint{0.800000in}{0.528000in}}{\pgfqpoint{4.960000in}{3.696000in}}%
\pgfusepath{clip}%
\pgfsetrectcap%
\pgfsetroundjoin%
\pgfsetlinewidth{0.803000pt}%
\definecolor{currentstroke}{rgb}{0.690196,0.690196,0.690196}%
\pgfsetstrokecolor{currentstroke}%
\pgfsetdash{}{0pt}%
\pgfpathmoveto{\pgfqpoint{0.800000in}{2.637921in}}%
\pgfpathlineto{\pgfqpoint{5.760000in}{2.637921in}}%
\pgfusepath{stroke}%
\end{pgfscope}%
\begin{pgfscope}%
\pgfsetbuttcap%
\pgfsetroundjoin%
\definecolor{currentfill}{rgb}{0.000000,0.000000,0.000000}%
\pgfsetfillcolor{currentfill}%
\pgfsetlinewidth{0.803000pt}%
\definecolor{currentstroke}{rgb}{0.000000,0.000000,0.000000}%
\pgfsetstrokecolor{currentstroke}%
\pgfsetdash{}{0pt}%
\pgfsys@defobject{currentmarker}{\pgfqpoint{-0.048611in}{0.000000in}}{\pgfqpoint{-0.000000in}{0.000000in}}{%
\pgfpathmoveto{\pgfqpoint{-0.000000in}{0.000000in}}%
\pgfpathlineto{\pgfqpoint{-0.048611in}{0.000000in}}%
\pgfusepath{stroke,fill}%
}%
\begin{pgfscope}%
\pgfsys@transformshift{0.800000in}{2.637921in}%
\pgfsys@useobject{currentmarker}{}%
\end{pgfscope}%
\end{pgfscope}%
\begin{pgfscope}%
\definecolor{textcolor}{rgb}{0.000000,0.000000,0.000000}%
\pgfsetstrokecolor{textcolor}%
\pgfsetfillcolor{textcolor}%
\pgftext[x=0.525308in, y=2.586821in, left, base]{\color{textcolor}\rmfamily\fontsize{10.000000}{12.000000}\selectfont \(\displaystyle {0.5}\)}%
\end{pgfscope}%
\begin{pgfscope}%
\pgfpathrectangle{\pgfqpoint{0.800000in}{0.528000in}}{\pgfqpoint{4.960000in}{3.696000in}}%
\pgfusepath{clip}%
\pgfsetrectcap%
\pgfsetroundjoin%
\pgfsetlinewidth{0.803000pt}%
\definecolor{currentstroke}{rgb}{0.690196,0.690196,0.690196}%
\pgfsetstrokecolor{currentstroke}%
\pgfsetdash{}{0pt}%
\pgfpathmoveto{\pgfqpoint{0.800000in}{3.059905in}}%
\pgfpathlineto{\pgfqpoint{5.760000in}{3.059905in}}%
\pgfusepath{stroke}%
\end{pgfscope}%
\begin{pgfscope}%
\pgfsetbuttcap%
\pgfsetroundjoin%
\definecolor{currentfill}{rgb}{0.000000,0.000000,0.000000}%
\pgfsetfillcolor{currentfill}%
\pgfsetlinewidth{0.803000pt}%
\definecolor{currentstroke}{rgb}{0.000000,0.000000,0.000000}%
\pgfsetstrokecolor{currentstroke}%
\pgfsetdash{}{0pt}%
\pgfsys@defobject{currentmarker}{\pgfqpoint{-0.048611in}{0.000000in}}{\pgfqpoint{-0.000000in}{0.000000in}}{%
\pgfpathmoveto{\pgfqpoint{-0.000000in}{0.000000in}}%
\pgfpathlineto{\pgfqpoint{-0.048611in}{0.000000in}}%
\pgfusepath{stroke,fill}%
}%
\begin{pgfscope}%
\pgfsys@transformshift{0.800000in}{3.059905in}%
\pgfsys@useobject{currentmarker}{}%
\end{pgfscope}%
\end{pgfscope}%
\begin{pgfscope}%
\definecolor{textcolor}{rgb}{0.000000,0.000000,0.000000}%
\pgfsetstrokecolor{textcolor}%
\pgfsetfillcolor{textcolor}%
\pgftext[x=0.525308in, y=3.008805in, left, base]{\color{textcolor}\rmfamily\fontsize{10.000000}{12.000000}\selectfont \(\displaystyle {0.6}\)}%
\end{pgfscope}%
\begin{pgfscope}%
\pgfpathrectangle{\pgfqpoint{0.800000in}{0.528000in}}{\pgfqpoint{4.960000in}{3.696000in}}%
\pgfusepath{clip}%
\pgfsetrectcap%
\pgfsetroundjoin%
\pgfsetlinewidth{0.803000pt}%
\definecolor{currentstroke}{rgb}{0.690196,0.690196,0.690196}%
\pgfsetstrokecolor{currentstroke}%
\pgfsetdash{}{0pt}%
\pgfpathmoveto{\pgfqpoint{0.800000in}{3.481890in}}%
\pgfpathlineto{\pgfqpoint{5.760000in}{3.481890in}}%
\pgfusepath{stroke}%
\end{pgfscope}%
\begin{pgfscope}%
\pgfsetbuttcap%
\pgfsetroundjoin%
\definecolor{currentfill}{rgb}{0.000000,0.000000,0.000000}%
\pgfsetfillcolor{currentfill}%
\pgfsetlinewidth{0.803000pt}%
\definecolor{currentstroke}{rgb}{0.000000,0.000000,0.000000}%
\pgfsetstrokecolor{currentstroke}%
\pgfsetdash{}{0pt}%
\pgfsys@defobject{currentmarker}{\pgfqpoint{-0.048611in}{0.000000in}}{\pgfqpoint{-0.000000in}{0.000000in}}{%
\pgfpathmoveto{\pgfqpoint{-0.000000in}{0.000000in}}%
\pgfpathlineto{\pgfqpoint{-0.048611in}{0.000000in}}%
\pgfusepath{stroke,fill}%
}%
\begin{pgfscope}%
\pgfsys@transformshift{0.800000in}{3.481890in}%
\pgfsys@useobject{currentmarker}{}%
\end{pgfscope}%
\end{pgfscope}%
\begin{pgfscope}%
\definecolor{textcolor}{rgb}{0.000000,0.000000,0.000000}%
\pgfsetstrokecolor{textcolor}%
\pgfsetfillcolor{textcolor}%
\pgftext[x=0.525308in, y=3.430790in, left, base]{\color{textcolor}\rmfamily\fontsize{10.000000}{12.000000}\selectfont \(\displaystyle {0.7}\)}%
\end{pgfscope}%
\begin{pgfscope}%
\pgfpathrectangle{\pgfqpoint{0.800000in}{0.528000in}}{\pgfqpoint{4.960000in}{3.696000in}}%
\pgfusepath{clip}%
\pgfsetrectcap%
\pgfsetroundjoin%
\pgfsetlinewidth{0.803000pt}%
\definecolor{currentstroke}{rgb}{0.690196,0.690196,0.690196}%
\pgfsetstrokecolor{currentstroke}%
\pgfsetdash{}{0pt}%
\pgfpathmoveto{\pgfqpoint{0.800000in}{3.903874in}}%
\pgfpathlineto{\pgfqpoint{5.760000in}{3.903874in}}%
\pgfusepath{stroke}%
\end{pgfscope}%
\begin{pgfscope}%
\pgfsetbuttcap%
\pgfsetroundjoin%
\definecolor{currentfill}{rgb}{0.000000,0.000000,0.000000}%
\pgfsetfillcolor{currentfill}%
\pgfsetlinewidth{0.803000pt}%
\definecolor{currentstroke}{rgb}{0.000000,0.000000,0.000000}%
\pgfsetstrokecolor{currentstroke}%
\pgfsetdash{}{0pt}%
\pgfsys@defobject{currentmarker}{\pgfqpoint{-0.048611in}{0.000000in}}{\pgfqpoint{-0.000000in}{0.000000in}}{%
\pgfpathmoveto{\pgfqpoint{-0.000000in}{0.000000in}}%
\pgfpathlineto{\pgfqpoint{-0.048611in}{0.000000in}}%
\pgfusepath{stroke,fill}%
}%
\begin{pgfscope}%
\pgfsys@transformshift{0.800000in}{3.903874in}%
\pgfsys@useobject{currentmarker}{}%
\end{pgfscope}%
\end{pgfscope}%
\begin{pgfscope}%
\definecolor{textcolor}{rgb}{0.000000,0.000000,0.000000}%
\pgfsetstrokecolor{textcolor}%
\pgfsetfillcolor{textcolor}%
\pgftext[x=0.525308in, y=3.852774in, left, base]{\color{textcolor}\rmfamily\fontsize{10.000000}{12.000000}\selectfont \(\displaystyle {0.8}\)}%
\end{pgfscope}%
\begin{pgfscope}%
\definecolor{textcolor}{rgb}{0.000000,0.000000,0.000000}%
\pgfsetstrokecolor{textcolor}%
\pgfsetfillcolor{textcolor}%
\pgftext[x=0.469752in,y=2.376000in,,bottom,rotate=90.000000]{\color{textcolor}\rmfamily\fontsize{10.000000}{12.000000}\selectfont electrical power in \(\displaystyle \mathrm{p.u.}\)}%
\end{pgfscope}%
\begin{pgfscope}%
\pgfpathrectangle{\pgfqpoint{0.800000in}{0.528000in}}{\pgfqpoint{4.960000in}{3.696000in}}%
\pgfusepath{clip}%
\pgfsetrectcap%
\pgfsetroundjoin%
\pgfsetlinewidth{1.505625pt}%
\definecolor{currentstroke}{rgb}{0.121569,0.466667,0.705882}%
\pgfsetstrokecolor{currentstroke}%
\pgfsetdash{}{0pt}%
\pgfpathmoveto{\pgfqpoint{1.056511in}{3.480789in}}%
\pgfpathlineto{\pgfqpoint{1.255768in}{3.481899in}}%
\pgfpathlineto{\pgfqpoint{1.497784in}{3.482892in}}%
\pgfpathlineto{\pgfqpoint{1.910836in}{3.480979in}}%
\pgfpathlineto{\pgfqpoint{1.911691in}{2.728262in}}%
\pgfpathlineto{\pgfqpoint{1.912546in}{2.728277in}}%
\pgfpathlineto{\pgfqpoint{1.920243in}{2.729590in}}%
\pgfpathlineto{\pgfqpoint{1.928795in}{2.733513in}}%
\pgfpathlineto{\pgfqpoint{1.937346in}{2.740006in}}%
\pgfpathlineto{\pgfqpoint{1.946753in}{2.750075in}}%
\pgfpathlineto{\pgfqpoint{1.957015in}{2.764476in}}%
\pgfpathlineto{\pgfqpoint{1.968133in}{2.783972in}}%
\pgfpathlineto{\pgfqpoint{1.980961in}{2.811274in}}%
\pgfpathlineto{\pgfqpoint{1.994643in}{2.845717in}}%
\pgfpathlineto{\pgfqpoint{2.010037in}{2.890468in}}%
\pgfpathlineto{\pgfqpoint{2.027995in}{2.949751in}}%
\pgfpathlineto{\pgfqpoint{2.049375in}{3.028523in}}%
\pgfpathlineto{\pgfqpoint{2.075885in}{3.135200in}}%
\pgfpathlineto{\pgfqpoint{2.124631in}{3.342801in}}%
\pgfpathlineto{\pgfqpoint{2.164824in}{3.509592in}}%
\pgfpathlineto{\pgfqpoint{2.193045in}{3.617446in}}%
\pgfpathlineto{\pgfqpoint{2.216990in}{3.700539in}}%
\pgfpathlineto{\pgfqpoint{2.239225in}{3.769776in}}%
\pgfpathlineto{\pgfqpoint{2.259749in}{3.826509in}}%
\pgfpathlineto{\pgfqpoint{2.279418in}{3.874303in}}%
\pgfpathlineto{\pgfqpoint{2.297377in}{3.912374in}}%
\pgfpathlineto{\pgfqpoint{2.315336in}{3.945270in}}%
\pgfpathlineto{\pgfqpoint{2.332439in}{3.971963in}}%
\pgfpathlineto{\pgfqpoint{2.348688in}{3.993308in}}%
\pgfpathlineto{\pgfqpoint{2.364081in}{4.010081in}}%
\pgfpathlineto{\pgfqpoint{2.379474in}{4.023640in}}%
\pgfpathlineto{\pgfqpoint{2.394012in}{4.033612in}}%
\pgfpathlineto{\pgfqpoint{2.408550in}{4.040923in}}%
\pgfpathlineto{\pgfqpoint{2.422233in}{4.045438in}}%
\pgfpathlineto{\pgfqpoint{2.435916in}{4.047703in}}%
\pgfpathlineto{\pgfqpoint{2.449599in}{4.047741in}}%
\pgfpathlineto{\pgfqpoint{2.463282in}{4.045557in}}%
\pgfpathlineto{\pgfqpoint{2.476965in}{4.041134in}}%
\pgfpathlineto{\pgfqpoint{2.490648in}{4.034440in}}%
\pgfpathlineto{\pgfqpoint{2.505186in}{4.024782in}}%
\pgfpathlineto{\pgfqpoint{2.519724in}{4.012424in}}%
\pgfpathlineto{\pgfqpoint{2.535117in}{3.996293in}}%
\pgfpathlineto{\pgfqpoint{2.550510in}{3.976908in}}%
\pgfpathlineto{\pgfqpoint{2.566759in}{3.952776in}}%
\pgfpathlineto{\pgfqpoint{2.583862in}{3.923148in}}%
\pgfpathlineto{\pgfqpoint{2.600966in}{3.889043in}}%
\pgfpathlineto{\pgfqpoint{2.618925in}{3.848302in}}%
\pgfpathlineto{\pgfqpoint{2.637739in}{3.800151in}}%
\pgfpathlineto{\pgfqpoint{2.658263in}{3.741312in}}%
\pgfpathlineto{\pgfqpoint{2.680498in}{3.670453in}}%
\pgfpathlineto{\pgfqpoint{2.705298in}{3.583494in}}%
\pgfpathlineto{\pgfqpoint{2.734374in}{3.472850in}}%
\pgfpathlineto{\pgfqpoint{2.777988in}{3.296695in}}%
\pgfpathlineto{\pgfqpoint{2.825023in}{3.109247in}}%
\pgfpathlineto{\pgfqpoint{2.850678in}{3.016075in}}%
\pgfpathlineto{\pgfqpoint{2.871203in}{2.949474in}}%
\pgfpathlineto{\pgfqpoint{2.889161in}{2.898546in}}%
\pgfpathlineto{\pgfqpoint{2.904555in}{2.861223in}}%
\pgfpathlineto{\pgfqpoint{2.918238in}{2.833465in}}%
\pgfpathlineto{\pgfqpoint{2.931065in}{2.812402in}}%
\pgfpathlineto{\pgfqpoint{2.942183in}{2.798216in}}%
\pgfpathlineto{\pgfqpoint{2.952445in}{2.788585in}}%
\pgfpathlineto{\pgfqpoint{2.961852in}{2.782737in}}%
\pgfpathlineto{\pgfqpoint{2.970404in}{2.779924in}}%
\pgfpathlineto{\pgfqpoint{2.978955in}{2.779508in}}%
\pgfpathlineto{\pgfqpoint{2.987507in}{2.781489in}}%
\pgfpathlineto{\pgfqpoint{2.996059in}{2.785854in}}%
\pgfpathlineto{\pgfqpoint{3.005466in}{2.793379in}}%
\pgfpathlineto{\pgfqpoint{3.015728in}{2.804783in}}%
\pgfpathlineto{\pgfqpoint{3.026845in}{2.820799in}}%
\pgfpathlineto{\pgfqpoint{3.038818in}{2.842144in}}%
\pgfpathlineto{\pgfqpoint{3.052501in}{2.871464in}}%
\pgfpathlineto{\pgfqpoint{3.067894in}{2.910263in}}%
\pgfpathlineto{\pgfqpoint{3.084998in}{2.959844in}}%
\pgfpathlineto{\pgfqpoint{3.104667in}{3.024049in}}%
\pgfpathlineto{\pgfqpoint{3.128612in}{3.110328in}}%
\pgfpathlineto{\pgfqpoint{3.161964in}{3.239863in}}%
\pgfpathlineto{\pgfqpoint{3.233799in}{3.521152in}}%
\pgfpathlineto{\pgfqpoint{3.262020in}{3.621374in}}%
\pgfpathlineto{\pgfqpoint{3.285965in}{3.698633in}}%
\pgfpathlineto{\pgfqpoint{3.308200in}{3.763069in}}%
\pgfpathlineto{\pgfqpoint{3.328724in}{3.815920in}}%
\pgfpathlineto{\pgfqpoint{3.348393in}{3.860478in}}%
\pgfpathlineto{\pgfqpoint{3.367207in}{3.897548in}}%
\pgfpathlineto{\pgfqpoint{3.385166in}{3.927977in}}%
\pgfpathlineto{\pgfqpoint{3.402269in}{3.952593in}}%
\pgfpathlineto{\pgfqpoint{3.418518in}{3.972175in}}%
\pgfpathlineto{\pgfqpoint{3.433911in}{3.987431in}}%
\pgfpathlineto{\pgfqpoint{3.449304in}{3.999591in}}%
\pgfpathlineto{\pgfqpoint{3.463842in}{4.008321in}}%
\pgfpathlineto{\pgfqpoint{3.478380in}{4.014446in}}%
\pgfpathlineto{\pgfqpoint{3.492063in}{4.017877in}}%
\pgfpathlineto{\pgfqpoint{3.505746in}{4.019074in}}%
\pgfpathlineto{\pgfqpoint{3.519429in}{4.018049in}}%
\pgfpathlineto{\pgfqpoint{3.533112in}{4.014800in}}%
\pgfpathlineto{\pgfqpoint{3.546795in}{4.009306in}}%
\pgfpathlineto{\pgfqpoint{3.561333in}{4.000973in}}%
\pgfpathlineto{\pgfqpoint{3.575871in}{3.990012in}}%
\pgfpathlineto{\pgfqpoint{3.591264in}{3.975466in}}%
\pgfpathlineto{\pgfqpoint{3.606657in}{3.957806in}}%
\pgfpathlineto{\pgfqpoint{3.622906in}{3.935677in}}%
\pgfpathlineto{\pgfqpoint{3.640009in}{3.908397in}}%
\pgfpathlineto{\pgfqpoint{3.657113in}{3.876920in}}%
\pgfpathlineto{\pgfqpoint{3.675072in}{3.839268in}}%
\pgfpathlineto{\pgfqpoint{3.694741in}{3.792592in}}%
\pgfpathlineto{\pgfqpoint{3.715265in}{3.737908in}}%
\pgfpathlineto{\pgfqpoint{3.737500in}{3.672081in}}%
\pgfpathlineto{\pgfqpoint{3.762300in}{3.591317in}}%
\pgfpathlineto{\pgfqpoint{3.791376in}{3.488521in}}%
\pgfpathlineto{\pgfqpoint{3.833280in}{3.331056in}}%
\pgfpathlineto{\pgfqpoint{3.883736in}{3.142967in}}%
\pgfpathlineto{\pgfqpoint{3.909391in}{3.055600in}}%
\pgfpathlineto{\pgfqpoint{3.930770in}{2.990411in}}%
\pgfpathlineto{\pgfqpoint{3.948729in}{2.942481in}}%
\pgfpathlineto{\pgfqpoint{3.964978in}{2.905380in}}%
\pgfpathlineto{\pgfqpoint{3.978661in}{2.879249in}}%
\pgfpathlineto{\pgfqpoint{3.991488in}{2.859304in}}%
\pgfpathlineto{\pgfqpoint{4.002606in}{2.845756in}}%
\pgfpathlineto{\pgfqpoint{4.012868in}{2.836432in}}%
\pgfpathlineto{\pgfqpoint{4.022275in}{2.830627in}}%
\pgfpathlineto{\pgfqpoint{4.031682in}{2.827477in}}%
\pgfpathlineto{\pgfqpoint{4.040233in}{2.826931in}}%
\pgfpathlineto{\pgfqpoint{4.048785in}{2.828595in}}%
\pgfpathlineto{\pgfqpoint{4.057337in}{2.832456in}}%
\pgfpathlineto{\pgfqpoint{4.066744in}{2.839215in}}%
\pgfpathlineto{\pgfqpoint{4.077006in}{2.849540in}}%
\pgfpathlineto{\pgfqpoint{4.088124in}{2.864108in}}%
\pgfpathlineto{\pgfqpoint{4.100096in}{2.883589in}}%
\pgfpathlineto{\pgfqpoint{4.113779in}{2.910419in}}%
\pgfpathlineto{\pgfqpoint{4.129172in}{2.946007in}}%
\pgfpathlineto{\pgfqpoint{4.146276in}{2.991589in}}%
\pgfpathlineto{\pgfqpoint{4.165945in}{3.050753in}}%
\pgfpathlineto{\pgfqpoint{4.189035in}{3.127513in}}%
\pgfpathlineto{\pgfqpoint{4.220676in}{3.241159in}}%
\pgfpathlineto{\pgfqpoint{4.304484in}{3.546099in}}%
\pgfpathlineto{\pgfqpoint{4.331850in}{3.635294in}}%
\pgfpathlineto{\pgfqpoint{4.355795in}{3.706069in}}%
\pgfpathlineto{\pgfqpoint{4.378029in}{3.764947in}}%
\pgfpathlineto{\pgfqpoint{4.398554in}{3.813119in}}%
\pgfpathlineto{\pgfqpoint{4.418223in}{3.853615in}}%
\pgfpathlineto{\pgfqpoint{4.436182in}{3.885759in}}%
\pgfpathlineto{\pgfqpoint{4.454140in}{3.913371in}}%
\pgfpathlineto{\pgfqpoint{4.471244in}{3.935561in}}%
\pgfpathlineto{\pgfqpoint{4.487492in}{3.953036in}}%
\pgfpathlineto{\pgfqpoint{4.502886in}{3.966447in}}%
\pgfpathlineto{\pgfqpoint{4.518279in}{3.976881in}}%
\pgfpathlineto{\pgfqpoint{4.532817in}{3.984067in}}%
\pgfpathlineto{\pgfqpoint{4.547355in}{3.988710in}}%
\pgfpathlineto{\pgfqpoint{4.561038in}{3.990789in}}%
\pgfpathlineto{\pgfqpoint{4.574721in}{3.990661in}}%
\pgfpathlineto{\pgfqpoint{4.588404in}{3.988331in}}%
\pgfpathlineto{\pgfqpoint{4.602087in}{3.983787in}}%
\pgfpathlineto{\pgfqpoint{4.616625in}{3.976512in}}%
\pgfpathlineto{\pgfqpoint{4.631163in}{3.966674in}}%
\pgfpathlineto{\pgfqpoint{4.646556in}{3.953411in}}%
\pgfpathlineto{\pgfqpoint{4.661949in}{3.937151in}}%
\pgfpathlineto{\pgfqpoint{4.678198in}{3.916658in}}%
\pgfpathlineto{\pgfqpoint{4.695301in}{3.891305in}}%
\pgfpathlineto{\pgfqpoint{4.713260in}{3.860424in}}%
\pgfpathlineto{\pgfqpoint{4.732074in}{3.823333in}}%
\pgfpathlineto{\pgfqpoint{4.751743in}{3.779371in}}%
\pgfpathlineto{\pgfqpoint{4.772267in}{3.727959in}}%
\pgfpathlineto{\pgfqpoint{4.794502in}{3.666189in}}%
\pgfpathlineto{\pgfqpoint{4.819302in}{3.590558in}}%
\pgfpathlineto{\pgfqpoint{4.848378in}{3.494477in}}%
\pgfpathlineto{\pgfqpoint{4.891137in}{3.344482in}}%
\pgfpathlineto{\pgfqpoint{4.941593in}{3.169099in}}%
\pgfpathlineto{\pgfqpoint{4.968104in}{3.085012in}}%
\pgfpathlineto{\pgfqpoint{4.989483in}{3.024379in}}%
\pgfpathlineto{\pgfqpoint{5.007442in}{2.979763in}}%
\pgfpathlineto{\pgfqpoint{5.023690in}{2.945176in}}%
\pgfpathlineto{\pgfqpoint{5.038228in}{2.919385in}}%
\pgfpathlineto{\pgfqpoint{5.051056in}{2.900966in}}%
\pgfpathlineto{\pgfqpoint{5.063028in}{2.887628in}}%
\pgfpathlineto{\pgfqpoint{5.073291in}{2.879253in}}%
\pgfpathlineto{\pgfqpoint{5.083553in}{2.873761in}}%
\pgfpathlineto{\pgfqpoint{5.092960in}{2.871290in}}%
\pgfpathlineto{\pgfqpoint{5.101512in}{2.871181in}}%
\pgfpathlineto{\pgfqpoint{5.110919in}{2.873412in}}%
\pgfpathlineto{\pgfqpoint{5.120326in}{2.878089in}}%
\pgfpathlineto{\pgfqpoint{5.130588in}{2.885944in}}%
\pgfpathlineto{\pgfqpoint{5.140850in}{2.896614in}}%
\pgfpathlineto{\pgfqpoint{5.152822in}{2.912512in}}%
\pgfpathlineto{\pgfqpoint{5.165650in}{2.933497in}}%
\pgfpathlineto{\pgfqpoint{5.180188in}{2.961927in}}%
\pgfpathlineto{\pgfqpoint{5.196437in}{2.999064in}}%
\pgfpathlineto{\pgfqpoint{5.214395in}{3.045945in}}%
\pgfpathlineto{\pgfqpoint{5.235775in}{3.108440in}}%
\pgfpathlineto{\pgfqpoint{5.261430in}{3.190671in}}%
\pgfpathlineto{\pgfqpoint{5.300768in}{3.325267in}}%
\pgfpathlineto{\pgfqpoint{5.356355in}{3.514578in}}%
\pgfpathlineto{\pgfqpoint{5.386286in}{3.608380in}}%
\pgfpathlineto{\pgfqpoint{5.411087in}{3.679168in}}%
\pgfpathlineto{\pgfqpoint{5.434177in}{3.738468in}}%
\pgfpathlineto{\pgfqpoint{5.455556in}{3.787251in}}%
\pgfpathlineto{\pgfqpoint{5.475225in}{3.826756in}}%
\pgfpathlineto{\pgfqpoint{5.494039in}{3.859685in}}%
\pgfpathlineto{\pgfqpoint{5.511998in}{3.886720in}}%
\pgfpathlineto{\pgfqpoint{5.529101in}{3.908543in}}%
\pgfpathlineto{\pgfqpoint{5.545350in}{3.925803in}}%
\pgfpathlineto{\pgfqpoint{5.560743in}{3.939107in}}%
\pgfpathlineto{\pgfqpoint{5.576136in}{3.949508in}}%
\pgfpathlineto{\pgfqpoint{5.590674in}{3.956719in}}%
\pgfpathlineto{\pgfqpoint{5.605212in}{3.961432in}}%
\pgfpathlineto{\pgfqpoint{5.618895in}{3.963611in}}%
\pgfpathlineto{\pgfqpoint{5.632578in}{3.963616in}}%
\pgfpathlineto{\pgfqpoint{5.646261in}{3.961451in}}%
\pgfpathlineto{\pgfqpoint{5.659944in}{3.957110in}}%
\pgfpathlineto{\pgfqpoint{5.674482in}{3.950095in}}%
\pgfpathlineto{\pgfqpoint{5.689020in}{3.940576in}}%
\pgfpathlineto{\pgfqpoint{5.704413in}{3.927727in}}%
\pgfpathlineto{\pgfqpoint{5.719807in}{3.911980in}}%
\pgfpathlineto{\pgfqpoint{5.736055in}{3.892157in}}%
\pgfpathlineto{\pgfqpoint{5.753159in}{3.867679in}}%
\pgfpathlineto{\pgfqpoint{5.760000in}{3.856840in}}%
\pgfpathlineto{\pgfqpoint{5.760000in}{3.856840in}}%
\pgfusepath{stroke}%
\end{pgfscope}%
\begin{pgfscope}%
\pgfsetrectcap%
\pgfsetmiterjoin%
\pgfsetlinewidth{0.803000pt}%
\definecolor{currentstroke}{rgb}{0.000000,0.000000,0.000000}%
\pgfsetstrokecolor{currentstroke}%
\pgfsetdash{}{0pt}%
\pgfpathmoveto{\pgfqpoint{0.800000in}{0.528000in}}%
\pgfpathlineto{\pgfqpoint{0.800000in}{4.224000in}}%
\pgfusepath{stroke}%
\end{pgfscope}%
\begin{pgfscope}%
\pgfsetrectcap%
\pgfsetmiterjoin%
\pgfsetlinewidth{0.803000pt}%
\definecolor{currentstroke}{rgb}{0.000000,0.000000,0.000000}%
\pgfsetstrokecolor{currentstroke}%
\pgfsetdash{}{0pt}%
\pgfpathmoveto{\pgfqpoint{5.760000in}{0.528000in}}%
\pgfpathlineto{\pgfqpoint{5.760000in}{4.224000in}}%
\pgfusepath{stroke}%
\end{pgfscope}%
\begin{pgfscope}%
\pgfsetrectcap%
\pgfsetmiterjoin%
\pgfsetlinewidth{0.803000pt}%
\definecolor{currentstroke}{rgb}{0.000000,0.000000,0.000000}%
\pgfsetstrokecolor{currentstroke}%
\pgfsetdash{}{0pt}%
\pgfpathmoveto{\pgfqpoint{0.800000in}{0.528000in}}%
\pgfpathlineto{\pgfqpoint{5.760000in}{0.528000in}}%
\pgfusepath{stroke}%
\end{pgfscope}%
\begin{pgfscope}%
\pgfsetrectcap%
\pgfsetmiterjoin%
\pgfsetlinewidth{0.803000pt}%
\definecolor{currentstroke}{rgb}{0.000000,0.000000,0.000000}%
\pgfsetstrokecolor{currentstroke}%
\pgfsetdash{}{0pt}%
\pgfpathmoveto{\pgfqpoint{0.800000in}{4.224000in}}%
\pgfpathlineto{\pgfqpoint{5.760000in}{4.224000in}}%
\pgfusepath{stroke}%
\end{pgfscope}%
\begin{pgfscope}%
\definecolor{textcolor}{rgb}{0.000000,0.000000,0.000000}%
\pgfsetstrokecolor{textcolor}%
\pgfsetfillcolor{textcolor}%
\pgftext[x=3.280000in,y=4.307333in,,base]{\color{textcolor}\rmfamily\fontsize{12.000000}{14.400000}\selectfont Electrical power over time - fault 3}%
\end{pgfscope}%
\begin{pgfscope}%
\pgfsetbuttcap%
\pgfsetmiterjoin%
\definecolor{currentfill}{rgb}{1.000000,1.000000,1.000000}%
\pgfsetfillcolor{currentfill}%
\pgfsetfillopacity{0.800000}%
\pgfsetlinewidth{1.003750pt}%
\definecolor{currentstroke}{rgb}{0.800000,0.800000,0.800000}%
\pgfsetstrokecolor{currentstroke}%
\pgfsetstrokeopacity{0.800000}%
\pgfsetdash{}{0pt}%
\pgfpathmoveto{\pgfqpoint{0.897222in}{0.597444in}}%
\pgfpathlineto{\pgfqpoint{2.654872in}{0.597444in}}%
\pgfpathquadraticcurveto{\pgfqpoint{2.682650in}{0.597444in}}{\pgfqpoint{2.682650in}{0.625222in}}%
\pgfpathlineto{\pgfqpoint{2.682650in}{0.813224in}}%
\pgfpathquadraticcurveto{\pgfqpoint{2.682650in}{0.841002in}}{\pgfqpoint{2.654872in}{0.841002in}}%
\pgfpathlineto{\pgfqpoint{0.897222in}{0.841002in}}%
\pgfpathquadraticcurveto{\pgfqpoint{0.869444in}{0.841002in}}{\pgfqpoint{0.869444in}{0.813224in}}%
\pgfpathlineto{\pgfqpoint{0.869444in}{0.625222in}}%
\pgfpathquadraticcurveto{\pgfqpoint{0.869444in}{0.597444in}}{\pgfqpoint{0.897222in}{0.597444in}}%
\pgfpathlineto{\pgfqpoint{0.897222in}{0.597444in}}%
\pgfpathclose%
\pgfusepath{stroke,fill}%
\end{pgfscope}%
\begin{pgfscope}%
\pgfsetrectcap%
\pgfsetroundjoin%
\pgfsetlinewidth{1.505625pt}%
\definecolor{currentstroke}{rgb}{0.121569,0.466667,0.705882}%
\pgfsetstrokecolor{currentstroke}%
\pgfsetdash{}{0pt}%
\pgfpathmoveto{\pgfqpoint{0.925000in}{0.731857in}}%
\pgfpathlineto{\pgfqpoint{1.063889in}{0.731857in}}%
\pgfpathlineto{\pgfqpoint{1.202778in}{0.731857in}}%
\pgfusepath{stroke}%
\end{pgfscope}%
\begin{pgfscope}%
\definecolor{textcolor}{rgb}{0.000000,0.000000,0.000000}%
\pgfsetstrokecolor{textcolor}%
\pgfsetfillcolor{textcolor}%
\pgftext[x=1.313889in,y=0.683246in,left,base]{\color{textcolor}\rmfamily\fontsize{10.000000}{12.000000}\selectfont power stable scenario}%
\end{pgfscope}%
\end{pgfpicture}%
\makeatother%
\endgroup%


\section{Comparison algebraic vs. non-algebraic}
\label{app:alg-non-alg-comparison}

%% Creator: Matplotlib, PGF backend
%%
%% To include the figure in your LaTeX document, write
%%   \input{<filename>.pgf}
%%
%% Make sure the required packages are loaded in your preamble
%%   \usepackage{pgf}
%%
%% Also ensure that all the required font packages are loaded; for instance,
%% the lmodern package is sometimes necessary when using math font.
%%   \usepackage{lmodern}
%%
%% Figures using additional raster images can only be included by \input if
%% they are in the same directory as the main LaTeX file. For loading figures
%% from other directories you can use the `import` package
%%   \usepackage{import}
%%
%% and then include the figures with
%%   \import{<path to file>}{<filename>.pgf}
%%
%% Matplotlib used the following preamble
%%   
%%   \usepackage{fontspec}
%%   \setmainfont{Charter.ttc}[Path=\detokenize{/System/Library/Fonts/Supplemental/}]
%%   \setsansfont{DejaVuSans.ttf}[Path=\detokenize{/opt/homebrew/lib/python3.10/site-packages/matplotlib/mpl-data/fonts/ttf/}]
%%   \setmonofont{DejaVuSansMono.ttf}[Path=\detokenize{/opt/homebrew/lib/python3.10/site-packages/matplotlib/mpl-data/fonts/ttf/}]
%%   \makeatletter\@ifpackageloaded{underscore}{}{\usepackage[strings]{underscore}}\makeatother
%%
\begingroup%
\makeatletter%
\begin{pgfpicture}%
\pgfpathrectangle{\pgfpointorigin}{\pgfqpoint{6.400000in}{4.800000in}}%
\pgfusepath{use as bounding box, clip}%
\begin{pgfscope}%
\pgfsetbuttcap%
\pgfsetmiterjoin%
\definecolor{currentfill}{rgb}{1.000000,1.000000,1.000000}%
\pgfsetfillcolor{currentfill}%
\pgfsetlinewidth{0.000000pt}%
\definecolor{currentstroke}{rgb}{1.000000,1.000000,1.000000}%
\pgfsetstrokecolor{currentstroke}%
\pgfsetdash{}{0pt}%
\pgfpathmoveto{\pgfqpoint{0.000000in}{0.000000in}}%
\pgfpathlineto{\pgfqpoint{6.400000in}{0.000000in}}%
\pgfpathlineto{\pgfqpoint{6.400000in}{4.800000in}}%
\pgfpathlineto{\pgfqpoint{0.000000in}{4.800000in}}%
\pgfpathlineto{\pgfqpoint{0.000000in}{0.000000in}}%
\pgfpathclose%
\pgfusepath{fill}%
\end{pgfscope}%
\begin{pgfscope}%
\pgfsetbuttcap%
\pgfsetmiterjoin%
\definecolor{currentfill}{rgb}{1.000000,1.000000,1.000000}%
\pgfsetfillcolor{currentfill}%
\pgfsetlinewidth{0.000000pt}%
\definecolor{currentstroke}{rgb}{0.000000,0.000000,0.000000}%
\pgfsetstrokecolor{currentstroke}%
\pgfsetstrokeopacity{0.000000}%
\pgfsetdash{}{0pt}%
\pgfpathmoveto{\pgfqpoint{0.800000in}{0.528000in}}%
\pgfpathlineto{\pgfqpoint{5.760000in}{0.528000in}}%
\pgfpathlineto{\pgfqpoint{5.760000in}{4.224000in}}%
\pgfpathlineto{\pgfqpoint{0.800000in}{4.224000in}}%
\pgfpathlineto{\pgfqpoint{0.800000in}{0.528000in}}%
\pgfpathclose%
\pgfusepath{fill}%
\end{pgfscope}%
\begin{pgfscope}%
\pgfpathrectangle{\pgfqpoint{0.800000in}{0.528000in}}{\pgfqpoint{4.960000in}{3.696000in}}%
\pgfusepath{clip}%
\pgfsetrectcap%
\pgfsetroundjoin%
\pgfsetlinewidth{0.803000pt}%
\definecolor{currentstroke}{rgb}{0.690196,0.690196,0.690196}%
\pgfsetstrokecolor{currentstroke}%
\pgfsetdash{}{0pt}%
\pgfpathmoveto{\pgfqpoint{1.025455in}{0.528000in}}%
\pgfpathlineto{\pgfqpoint{1.025455in}{4.224000in}}%
\pgfusepath{stroke}%
\end{pgfscope}%
\begin{pgfscope}%
\pgfsetbuttcap%
\pgfsetroundjoin%
\definecolor{currentfill}{rgb}{0.000000,0.000000,0.000000}%
\pgfsetfillcolor{currentfill}%
\pgfsetlinewidth{0.803000pt}%
\definecolor{currentstroke}{rgb}{0.000000,0.000000,0.000000}%
\pgfsetstrokecolor{currentstroke}%
\pgfsetdash{}{0pt}%
\pgfsys@defobject{currentmarker}{\pgfqpoint{0.000000in}{-0.048611in}}{\pgfqpoint{0.000000in}{0.000000in}}{%
\pgfpathmoveto{\pgfqpoint{0.000000in}{0.000000in}}%
\pgfpathlineto{\pgfqpoint{0.000000in}{-0.048611in}}%
\pgfusepath{stroke,fill}%
}%
\begin{pgfscope}%
\pgfsys@transformshift{1.025455in}{0.528000in}%
\pgfsys@useobject{currentmarker}{}%
\end{pgfscope}%
\end{pgfscope}%
\begin{pgfscope}%
\definecolor{textcolor}{rgb}{0.000000,0.000000,0.000000}%
\pgfsetstrokecolor{textcolor}%
\pgfsetfillcolor{textcolor}%
\pgftext[x=1.025455in,y=0.430778in,,top]{\color{textcolor}\rmfamily\fontsize{10.000000}{12.000000}\selectfont \(\displaystyle {\ensuremath{-}1}\)}%
\end{pgfscope}%
\begin{pgfscope}%
\pgfpathrectangle{\pgfqpoint{0.800000in}{0.528000in}}{\pgfqpoint{4.960000in}{3.696000in}}%
\pgfusepath{clip}%
\pgfsetrectcap%
\pgfsetroundjoin%
\pgfsetlinewidth{0.803000pt}%
\definecolor{currentstroke}{rgb}{0.690196,0.690196,0.690196}%
\pgfsetstrokecolor{currentstroke}%
\pgfsetdash{}{0pt}%
\pgfpathmoveto{\pgfqpoint{1.777095in}{0.528000in}}%
\pgfpathlineto{\pgfqpoint{1.777095in}{4.224000in}}%
\pgfusepath{stroke}%
\end{pgfscope}%
\begin{pgfscope}%
\pgfsetbuttcap%
\pgfsetroundjoin%
\definecolor{currentfill}{rgb}{0.000000,0.000000,0.000000}%
\pgfsetfillcolor{currentfill}%
\pgfsetlinewidth{0.803000pt}%
\definecolor{currentstroke}{rgb}{0.000000,0.000000,0.000000}%
\pgfsetstrokecolor{currentstroke}%
\pgfsetdash{}{0pt}%
\pgfsys@defobject{currentmarker}{\pgfqpoint{0.000000in}{-0.048611in}}{\pgfqpoint{0.000000in}{0.000000in}}{%
\pgfpathmoveto{\pgfqpoint{0.000000in}{0.000000in}}%
\pgfpathlineto{\pgfqpoint{0.000000in}{-0.048611in}}%
\pgfusepath{stroke,fill}%
}%
\begin{pgfscope}%
\pgfsys@transformshift{1.777095in}{0.528000in}%
\pgfsys@useobject{currentmarker}{}%
\end{pgfscope}%
\end{pgfscope}%
\begin{pgfscope}%
\definecolor{textcolor}{rgb}{0.000000,0.000000,0.000000}%
\pgfsetstrokecolor{textcolor}%
\pgfsetfillcolor{textcolor}%
\pgftext[x=1.777095in,y=0.430778in,,top]{\color{textcolor}\rmfamily\fontsize{10.000000}{12.000000}\selectfont \(\displaystyle {0}\)}%
\end{pgfscope}%
\begin{pgfscope}%
\pgfpathrectangle{\pgfqpoint{0.800000in}{0.528000in}}{\pgfqpoint{4.960000in}{3.696000in}}%
\pgfusepath{clip}%
\pgfsetrectcap%
\pgfsetroundjoin%
\pgfsetlinewidth{0.803000pt}%
\definecolor{currentstroke}{rgb}{0.690196,0.690196,0.690196}%
\pgfsetstrokecolor{currentstroke}%
\pgfsetdash{}{0pt}%
\pgfpathmoveto{\pgfqpoint{2.528735in}{0.528000in}}%
\pgfpathlineto{\pgfqpoint{2.528735in}{4.224000in}}%
\pgfusepath{stroke}%
\end{pgfscope}%
\begin{pgfscope}%
\pgfsetbuttcap%
\pgfsetroundjoin%
\definecolor{currentfill}{rgb}{0.000000,0.000000,0.000000}%
\pgfsetfillcolor{currentfill}%
\pgfsetlinewidth{0.803000pt}%
\definecolor{currentstroke}{rgb}{0.000000,0.000000,0.000000}%
\pgfsetstrokecolor{currentstroke}%
\pgfsetdash{}{0pt}%
\pgfsys@defobject{currentmarker}{\pgfqpoint{0.000000in}{-0.048611in}}{\pgfqpoint{0.000000in}{0.000000in}}{%
\pgfpathmoveto{\pgfqpoint{0.000000in}{0.000000in}}%
\pgfpathlineto{\pgfqpoint{0.000000in}{-0.048611in}}%
\pgfusepath{stroke,fill}%
}%
\begin{pgfscope}%
\pgfsys@transformshift{2.528735in}{0.528000in}%
\pgfsys@useobject{currentmarker}{}%
\end{pgfscope}%
\end{pgfscope}%
\begin{pgfscope}%
\definecolor{textcolor}{rgb}{0.000000,0.000000,0.000000}%
\pgfsetstrokecolor{textcolor}%
\pgfsetfillcolor{textcolor}%
\pgftext[x=2.528735in,y=0.430778in,,top]{\color{textcolor}\rmfamily\fontsize{10.000000}{12.000000}\selectfont \(\displaystyle {1}\)}%
\end{pgfscope}%
\begin{pgfscope}%
\pgfpathrectangle{\pgfqpoint{0.800000in}{0.528000in}}{\pgfqpoint{4.960000in}{3.696000in}}%
\pgfusepath{clip}%
\pgfsetrectcap%
\pgfsetroundjoin%
\pgfsetlinewidth{0.803000pt}%
\definecolor{currentstroke}{rgb}{0.690196,0.690196,0.690196}%
\pgfsetstrokecolor{currentstroke}%
\pgfsetdash{}{0pt}%
\pgfpathmoveto{\pgfqpoint{3.280376in}{0.528000in}}%
\pgfpathlineto{\pgfqpoint{3.280376in}{4.224000in}}%
\pgfusepath{stroke}%
\end{pgfscope}%
\begin{pgfscope}%
\pgfsetbuttcap%
\pgfsetroundjoin%
\definecolor{currentfill}{rgb}{0.000000,0.000000,0.000000}%
\pgfsetfillcolor{currentfill}%
\pgfsetlinewidth{0.803000pt}%
\definecolor{currentstroke}{rgb}{0.000000,0.000000,0.000000}%
\pgfsetstrokecolor{currentstroke}%
\pgfsetdash{}{0pt}%
\pgfsys@defobject{currentmarker}{\pgfqpoint{0.000000in}{-0.048611in}}{\pgfqpoint{0.000000in}{0.000000in}}{%
\pgfpathmoveto{\pgfqpoint{0.000000in}{0.000000in}}%
\pgfpathlineto{\pgfqpoint{0.000000in}{-0.048611in}}%
\pgfusepath{stroke,fill}%
}%
\begin{pgfscope}%
\pgfsys@transformshift{3.280376in}{0.528000in}%
\pgfsys@useobject{currentmarker}{}%
\end{pgfscope}%
\end{pgfscope}%
\begin{pgfscope}%
\definecolor{textcolor}{rgb}{0.000000,0.000000,0.000000}%
\pgfsetstrokecolor{textcolor}%
\pgfsetfillcolor{textcolor}%
\pgftext[x=3.280376in,y=0.430778in,,top]{\color{textcolor}\rmfamily\fontsize{10.000000}{12.000000}\selectfont \(\displaystyle {2}\)}%
\end{pgfscope}%
\begin{pgfscope}%
\pgfpathrectangle{\pgfqpoint{0.800000in}{0.528000in}}{\pgfqpoint{4.960000in}{3.696000in}}%
\pgfusepath{clip}%
\pgfsetrectcap%
\pgfsetroundjoin%
\pgfsetlinewidth{0.803000pt}%
\definecolor{currentstroke}{rgb}{0.690196,0.690196,0.690196}%
\pgfsetstrokecolor{currentstroke}%
\pgfsetdash{}{0pt}%
\pgfpathmoveto{\pgfqpoint{4.032016in}{0.528000in}}%
\pgfpathlineto{\pgfqpoint{4.032016in}{4.224000in}}%
\pgfusepath{stroke}%
\end{pgfscope}%
\begin{pgfscope}%
\pgfsetbuttcap%
\pgfsetroundjoin%
\definecolor{currentfill}{rgb}{0.000000,0.000000,0.000000}%
\pgfsetfillcolor{currentfill}%
\pgfsetlinewidth{0.803000pt}%
\definecolor{currentstroke}{rgb}{0.000000,0.000000,0.000000}%
\pgfsetstrokecolor{currentstroke}%
\pgfsetdash{}{0pt}%
\pgfsys@defobject{currentmarker}{\pgfqpoint{0.000000in}{-0.048611in}}{\pgfqpoint{0.000000in}{0.000000in}}{%
\pgfpathmoveto{\pgfqpoint{0.000000in}{0.000000in}}%
\pgfpathlineto{\pgfqpoint{0.000000in}{-0.048611in}}%
\pgfusepath{stroke,fill}%
}%
\begin{pgfscope}%
\pgfsys@transformshift{4.032016in}{0.528000in}%
\pgfsys@useobject{currentmarker}{}%
\end{pgfscope}%
\end{pgfscope}%
\begin{pgfscope}%
\definecolor{textcolor}{rgb}{0.000000,0.000000,0.000000}%
\pgfsetstrokecolor{textcolor}%
\pgfsetfillcolor{textcolor}%
\pgftext[x=4.032016in,y=0.430778in,,top]{\color{textcolor}\rmfamily\fontsize{10.000000}{12.000000}\selectfont \(\displaystyle {3}\)}%
\end{pgfscope}%
\begin{pgfscope}%
\pgfpathrectangle{\pgfqpoint{0.800000in}{0.528000in}}{\pgfqpoint{4.960000in}{3.696000in}}%
\pgfusepath{clip}%
\pgfsetrectcap%
\pgfsetroundjoin%
\pgfsetlinewidth{0.803000pt}%
\definecolor{currentstroke}{rgb}{0.690196,0.690196,0.690196}%
\pgfsetstrokecolor{currentstroke}%
\pgfsetdash{}{0pt}%
\pgfpathmoveto{\pgfqpoint{4.783657in}{0.528000in}}%
\pgfpathlineto{\pgfqpoint{4.783657in}{4.224000in}}%
\pgfusepath{stroke}%
\end{pgfscope}%
\begin{pgfscope}%
\pgfsetbuttcap%
\pgfsetroundjoin%
\definecolor{currentfill}{rgb}{0.000000,0.000000,0.000000}%
\pgfsetfillcolor{currentfill}%
\pgfsetlinewidth{0.803000pt}%
\definecolor{currentstroke}{rgb}{0.000000,0.000000,0.000000}%
\pgfsetstrokecolor{currentstroke}%
\pgfsetdash{}{0pt}%
\pgfsys@defobject{currentmarker}{\pgfqpoint{0.000000in}{-0.048611in}}{\pgfqpoint{0.000000in}{0.000000in}}{%
\pgfpathmoveto{\pgfqpoint{0.000000in}{0.000000in}}%
\pgfpathlineto{\pgfqpoint{0.000000in}{-0.048611in}}%
\pgfusepath{stroke,fill}%
}%
\begin{pgfscope}%
\pgfsys@transformshift{4.783657in}{0.528000in}%
\pgfsys@useobject{currentmarker}{}%
\end{pgfscope}%
\end{pgfscope}%
\begin{pgfscope}%
\definecolor{textcolor}{rgb}{0.000000,0.000000,0.000000}%
\pgfsetstrokecolor{textcolor}%
\pgfsetfillcolor{textcolor}%
\pgftext[x=4.783657in,y=0.430778in,,top]{\color{textcolor}\rmfamily\fontsize{10.000000}{12.000000}\selectfont \(\displaystyle {4}\)}%
\end{pgfscope}%
\begin{pgfscope}%
\pgfpathrectangle{\pgfqpoint{0.800000in}{0.528000in}}{\pgfqpoint{4.960000in}{3.696000in}}%
\pgfusepath{clip}%
\pgfsetrectcap%
\pgfsetroundjoin%
\pgfsetlinewidth{0.803000pt}%
\definecolor{currentstroke}{rgb}{0.690196,0.690196,0.690196}%
\pgfsetstrokecolor{currentstroke}%
\pgfsetdash{}{0pt}%
\pgfpathmoveto{\pgfqpoint{5.535297in}{0.528000in}}%
\pgfpathlineto{\pgfqpoint{5.535297in}{4.224000in}}%
\pgfusepath{stroke}%
\end{pgfscope}%
\begin{pgfscope}%
\pgfsetbuttcap%
\pgfsetroundjoin%
\definecolor{currentfill}{rgb}{0.000000,0.000000,0.000000}%
\pgfsetfillcolor{currentfill}%
\pgfsetlinewidth{0.803000pt}%
\definecolor{currentstroke}{rgb}{0.000000,0.000000,0.000000}%
\pgfsetstrokecolor{currentstroke}%
\pgfsetdash{}{0pt}%
\pgfsys@defobject{currentmarker}{\pgfqpoint{0.000000in}{-0.048611in}}{\pgfqpoint{0.000000in}{0.000000in}}{%
\pgfpathmoveto{\pgfqpoint{0.000000in}{0.000000in}}%
\pgfpathlineto{\pgfqpoint{0.000000in}{-0.048611in}}%
\pgfusepath{stroke,fill}%
}%
\begin{pgfscope}%
\pgfsys@transformshift{5.535297in}{0.528000in}%
\pgfsys@useobject{currentmarker}{}%
\end{pgfscope}%
\end{pgfscope}%
\begin{pgfscope}%
\definecolor{textcolor}{rgb}{0.000000,0.000000,0.000000}%
\pgfsetstrokecolor{textcolor}%
\pgfsetfillcolor{textcolor}%
\pgftext[x=5.535297in,y=0.430778in,,top]{\color{textcolor}\rmfamily\fontsize{10.000000}{12.000000}\selectfont \(\displaystyle {5}\)}%
\end{pgfscope}%
\begin{pgfscope}%
\definecolor{textcolor}{rgb}{0.000000,0.000000,0.000000}%
\pgfsetstrokecolor{textcolor}%
\pgfsetfillcolor{textcolor}%
\pgftext[x=3.280000in,y=0.242776in,,top]{\color{textcolor}\rmfamily\fontsize{10.000000}{12.000000}\selectfont time in s}%
\end{pgfscope}%
\begin{pgfscope}%
\pgfpathrectangle{\pgfqpoint{0.800000in}{0.528000in}}{\pgfqpoint{4.960000in}{3.696000in}}%
\pgfusepath{clip}%
\pgfsetrectcap%
\pgfsetroundjoin%
\pgfsetlinewidth{0.803000pt}%
\definecolor{currentstroke}{rgb}{0.690196,0.690196,0.690196}%
\pgfsetstrokecolor{currentstroke}%
\pgfsetdash{}{0pt}%
\pgfpathmoveto{\pgfqpoint{0.800000in}{1.070775in}}%
\pgfpathlineto{\pgfqpoint{5.760000in}{1.070775in}}%
\pgfusepath{stroke}%
\end{pgfscope}%
\begin{pgfscope}%
\pgfsetbuttcap%
\pgfsetroundjoin%
\definecolor{currentfill}{rgb}{0.000000,0.000000,0.000000}%
\pgfsetfillcolor{currentfill}%
\pgfsetlinewidth{0.803000pt}%
\definecolor{currentstroke}{rgb}{0.000000,0.000000,0.000000}%
\pgfsetstrokecolor{currentstroke}%
\pgfsetdash{}{0pt}%
\pgfsys@defobject{currentmarker}{\pgfqpoint{-0.048611in}{0.000000in}}{\pgfqpoint{-0.000000in}{0.000000in}}{%
\pgfpathmoveto{\pgfqpoint{-0.000000in}{0.000000in}}%
\pgfpathlineto{\pgfqpoint{-0.048611in}{0.000000in}}%
\pgfusepath{stroke,fill}%
}%
\begin{pgfscope}%
\pgfsys@transformshift{0.800000in}{1.070775in}%
\pgfsys@useobject{currentmarker}{}%
\end{pgfscope}%
\end{pgfscope}%
\begin{pgfscope}%
\definecolor{textcolor}{rgb}{0.000000,0.000000,0.000000}%
\pgfsetstrokecolor{textcolor}%
\pgfsetfillcolor{textcolor}%
\pgftext[x=0.563888in, y=1.019675in, left, base]{\color{textcolor}\rmfamily\fontsize{10.000000}{12.000000}\selectfont \(\displaystyle {20}\)}%
\end{pgfscope}%
\begin{pgfscope}%
\pgfpathrectangle{\pgfqpoint{0.800000in}{0.528000in}}{\pgfqpoint{4.960000in}{3.696000in}}%
\pgfusepath{clip}%
\pgfsetrectcap%
\pgfsetroundjoin%
\pgfsetlinewidth{0.803000pt}%
\definecolor{currentstroke}{rgb}{0.690196,0.690196,0.690196}%
\pgfsetstrokecolor{currentstroke}%
\pgfsetdash{}{0pt}%
\pgfpathmoveto{\pgfqpoint{0.800000in}{1.677284in}}%
\pgfpathlineto{\pgfqpoint{5.760000in}{1.677284in}}%
\pgfusepath{stroke}%
\end{pgfscope}%
\begin{pgfscope}%
\pgfsetbuttcap%
\pgfsetroundjoin%
\definecolor{currentfill}{rgb}{0.000000,0.000000,0.000000}%
\pgfsetfillcolor{currentfill}%
\pgfsetlinewidth{0.803000pt}%
\definecolor{currentstroke}{rgb}{0.000000,0.000000,0.000000}%
\pgfsetstrokecolor{currentstroke}%
\pgfsetdash{}{0pt}%
\pgfsys@defobject{currentmarker}{\pgfqpoint{-0.048611in}{0.000000in}}{\pgfqpoint{-0.000000in}{0.000000in}}{%
\pgfpathmoveto{\pgfqpoint{-0.000000in}{0.000000in}}%
\pgfpathlineto{\pgfqpoint{-0.048611in}{0.000000in}}%
\pgfusepath{stroke,fill}%
}%
\begin{pgfscope}%
\pgfsys@transformshift{0.800000in}{1.677284in}%
\pgfsys@useobject{currentmarker}{}%
\end{pgfscope}%
\end{pgfscope}%
\begin{pgfscope}%
\definecolor{textcolor}{rgb}{0.000000,0.000000,0.000000}%
\pgfsetstrokecolor{textcolor}%
\pgfsetfillcolor{textcolor}%
\pgftext[x=0.563888in, y=1.626184in, left, base]{\color{textcolor}\rmfamily\fontsize{10.000000}{12.000000}\selectfont \(\displaystyle {40}\)}%
\end{pgfscope}%
\begin{pgfscope}%
\pgfpathrectangle{\pgfqpoint{0.800000in}{0.528000in}}{\pgfqpoint{4.960000in}{3.696000in}}%
\pgfusepath{clip}%
\pgfsetrectcap%
\pgfsetroundjoin%
\pgfsetlinewidth{0.803000pt}%
\definecolor{currentstroke}{rgb}{0.690196,0.690196,0.690196}%
\pgfsetstrokecolor{currentstroke}%
\pgfsetdash{}{0pt}%
\pgfpathmoveto{\pgfqpoint{0.800000in}{2.283792in}}%
\pgfpathlineto{\pgfqpoint{5.760000in}{2.283792in}}%
\pgfusepath{stroke}%
\end{pgfscope}%
\begin{pgfscope}%
\pgfsetbuttcap%
\pgfsetroundjoin%
\definecolor{currentfill}{rgb}{0.000000,0.000000,0.000000}%
\pgfsetfillcolor{currentfill}%
\pgfsetlinewidth{0.803000pt}%
\definecolor{currentstroke}{rgb}{0.000000,0.000000,0.000000}%
\pgfsetstrokecolor{currentstroke}%
\pgfsetdash{}{0pt}%
\pgfsys@defobject{currentmarker}{\pgfqpoint{-0.048611in}{0.000000in}}{\pgfqpoint{-0.000000in}{0.000000in}}{%
\pgfpathmoveto{\pgfqpoint{-0.000000in}{0.000000in}}%
\pgfpathlineto{\pgfqpoint{-0.048611in}{0.000000in}}%
\pgfusepath{stroke,fill}%
}%
\begin{pgfscope}%
\pgfsys@transformshift{0.800000in}{2.283792in}%
\pgfsys@useobject{currentmarker}{}%
\end{pgfscope}%
\end{pgfscope}%
\begin{pgfscope}%
\definecolor{textcolor}{rgb}{0.000000,0.000000,0.000000}%
\pgfsetstrokecolor{textcolor}%
\pgfsetfillcolor{textcolor}%
\pgftext[x=0.563888in, y=2.232692in, left, base]{\color{textcolor}\rmfamily\fontsize{10.000000}{12.000000}\selectfont \(\displaystyle {60}\)}%
\end{pgfscope}%
\begin{pgfscope}%
\pgfpathrectangle{\pgfqpoint{0.800000in}{0.528000in}}{\pgfqpoint{4.960000in}{3.696000in}}%
\pgfusepath{clip}%
\pgfsetrectcap%
\pgfsetroundjoin%
\pgfsetlinewidth{0.803000pt}%
\definecolor{currentstroke}{rgb}{0.690196,0.690196,0.690196}%
\pgfsetstrokecolor{currentstroke}%
\pgfsetdash{}{0pt}%
\pgfpathmoveto{\pgfqpoint{0.800000in}{2.890301in}}%
\pgfpathlineto{\pgfqpoint{5.760000in}{2.890301in}}%
\pgfusepath{stroke}%
\end{pgfscope}%
\begin{pgfscope}%
\pgfsetbuttcap%
\pgfsetroundjoin%
\definecolor{currentfill}{rgb}{0.000000,0.000000,0.000000}%
\pgfsetfillcolor{currentfill}%
\pgfsetlinewidth{0.803000pt}%
\definecolor{currentstroke}{rgb}{0.000000,0.000000,0.000000}%
\pgfsetstrokecolor{currentstroke}%
\pgfsetdash{}{0pt}%
\pgfsys@defobject{currentmarker}{\pgfqpoint{-0.048611in}{0.000000in}}{\pgfqpoint{-0.000000in}{0.000000in}}{%
\pgfpathmoveto{\pgfqpoint{-0.000000in}{0.000000in}}%
\pgfpathlineto{\pgfqpoint{-0.048611in}{0.000000in}}%
\pgfusepath{stroke,fill}%
}%
\begin{pgfscope}%
\pgfsys@transformshift{0.800000in}{2.890301in}%
\pgfsys@useobject{currentmarker}{}%
\end{pgfscope}%
\end{pgfscope}%
\begin{pgfscope}%
\definecolor{textcolor}{rgb}{0.000000,0.000000,0.000000}%
\pgfsetstrokecolor{textcolor}%
\pgfsetfillcolor{textcolor}%
\pgftext[x=0.563888in, y=2.839201in, left, base]{\color{textcolor}\rmfamily\fontsize{10.000000}{12.000000}\selectfont \(\displaystyle {80}\)}%
\end{pgfscope}%
\begin{pgfscope}%
\pgfpathrectangle{\pgfqpoint{0.800000in}{0.528000in}}{\pgfqpoint{4.960000in}{3.696000in}}%
\pgfusepath{clip}%
\pgfsetrectcap%
\pgfsetroundjoin%
\pgfsetlinewidth{0.803000pt}%
\definecolor{currentstroke}{rgb}{0.690196,0.690196,0.690196}%
\pgfsetstrokecolor{currentstroke}%
\pgfsetdash{}{0pt}%
\pgfpathmoveto{\pgfqpoint{0.800000in}{3.496810in}}%
\pgfpathlineto{\pgfqpoint{5.760000in}{3.496810in}}%
\pgfusepath{stroke}%
\end{pgfscope}%
\begin{pgfscope}%
\pgfsetbuttcap%
\pgfsetroundjoin%
\definecolor{currentfill}{rgb}{0.000000,0.000000,0.000000}%
\pgfsetfillcolor{currentfill}%
\pgfsetlinewidth{0.803000pt}%
\definecolor{currentstroke}{rgb}{0.000000,0.000000,0.000000}%
\pgfsetstrokecolor{currentstroke}%
\pgfsetdash{}{0pt}%
\pgfsys@defobject{currentmarker}{\pgfqpoint{-0.048611in}{0.000000in}}{\pgfqpoint{-0.000000in}{0.000000in}}{%
\pgfpathmoveto{\pgfqpoint{-0.000000in}{0.000000in}}%
\pgfpathlineto{\pgfqpoint{-0.048611in}{0.000000in}}%
\pgfusepath{stroke,fill}%
}%
\begin{pgfscope}%
\pgfsys@transformshift{0.800000in}{3.496810in}%
\pgfsys@useobject{currentmarker}{}%
\end{pgfscope}%
\end{pgfscope}%
\begin{pgfscope}%
\definecolor{textcolor}{rgb}{0.000000,0.000000,0.000000}%
\pgfsetstrokecolor{textcolor}%
\pgfsetfillcolor{textcolor}%
\pgftext[x=0.494444in, y=3.445710in, left, base]{\color{textcolor}\rmfamily\fontsize{10.000000}{12.000000}\selectfont \(\displaystyle {100}\)}%
\end{pgfscope}%
\begin{pgfscope}%
\pgfpathrectangle{\pgfqpoint{0.800000in}{0.528000in}}{\pgfqpoint{4.960000in}{3.696000in}}%
\pgfusepath{clip}%
\pgfsetrectcap%
\pgfsetroundjoin%
\pgfsetlinewidth{0.803000pt}%
\definecolor{currentstroke}{rgb}{0.690196,0.690196,0.690196}%
\pgfsetstrokecolor{currentstroke}%
\pgfsetdash{}{0pt}%
\pgfpathmoveto{\pgfqpoint{0.800000in}{4.103319in}}%
\pgfpathlineto{\pgfqpoint{5.760000in}{4.103319in}}%
\pgfusepath{stroke}%
\end{pgfscope}%
\begin{pgfscope}%
\pgfsetbuttcap%
\pgfsetroundjoin%
\definecolor{currentfill}{rgb}{0.000000,0.000000,0.000000}%
\pgfsetfillcolor{currentfill}%
\pgfsetlinewidth{0.803000pt}%
\definecolor{currentstroke}{rgb}{0.000000,0.000000,0.000000}%
\pgfsetstrokecolor{currentstroke}%
\pgfsetdash{}{0pt}%
\pgfsys@defobject{currentmarker}{\pgfqpoint{-0.048611in}{0.000000in}}{\pgfqpoint{-0.000000in}{0.000000in}}{%
\pgfpathmoveto{\pgfqpoint{-0.000000in}{0.000000in}}%
\pgfpathlineto{\pgfqpoint{-0.048611in}{0.000000in}}%
\pgfusepath{stroke,fill}%
}%
\begin{pgfscope}%
\pgfsys@transformshift{0.800000in}{4.103319in}%
\pgfsys@useobject{currentmarker}{}%
\end{pgfscope}%
\end{pgfscope}%
\begin{pgfscope}%
\definecolor{textcolor}{rgb}{0.000000,0.000000,0.000000}%
\pgfsetstrokecolor{textcolor}%
\pgfsetfillcolor{textcolor}%
\pgftext[x=0.494444in, y=4.052219in, left, base]{\color{textcolor}\rmfamily\fontsize{10.000000}{12.000000}\selectfont \(\displaystyle {120}\)}%
\end{pgfscope}%
\begin{pgfscope}%
\definecolor{textcolor}{rgb}{0.000000,0.000000,0.000000}%
\pgfsetstrokecolor{textcolor}%
\pgfsetfillcolor{textcolor}%
\pgftext[x=0.438888in,y=2.376000in,,bottom,rotate=90.000000]{\color{textcolor}\rmfamily\fontsize{10.000000}{12.000000}\selectfont power angle \(\displaystyle \delta\) in deg}%
\end{pgfscope}%
\begin{pgfscope}%
\pgfpathrectangle{\pgfqpoint{0.800000in}{0.528000in}}{\pgfqpoint{4.960000in}{3.696000in}}%
\pgfusepath{clip}%
\pgfsetrectcap%
\pgfsetroundjoin%
\pgfsetlinewidth{1.505625pt}%
\definecolor{currentstroke}{rgb}{0.121569,0.466667,0.705882}%
\pgfsetstrokecolor{currentstroke}%
\pgfsetdash{}{0pt}%
\pgfpathmoveto{\pgfqpoint{1.025455in}{1.938082in}}%
\pgfpathlineto{\pgfqpoint{1.780853in}{1.938992in}}%
\pgfpathlineto{\pgfqpoint{1.785363in}{1.942564in}}%
\pgfpathlineto{\pgfqpoint{1.790624in}{1.950113in}}%
\pgfpathlineto{\pgfqpoint{1.796638in}{1.963195in}}%
\pgfpathlineto{\pgfqpoint{1.803402in}{1.983586in}}%
\pgfpathlineto{\pgfqpoint{1.810919in}{2.013280in}}%
\pgfpathlineto{\pgfqpoint{1.819938in}{2.058677in}}%
\pgfpathlineto{\pgfqpoint{1.829710in}{2.119859in}}%
\pgfpathlineto{\pgfqpoint{1.840984in}{2.205936in}}%
\pgfpathlineto{\pgfqpoint{1.853011in}{2.316000in}}%
\pgfpathlineto{\pgfqpoint{1.867292in}{2.469493in}}%
\pgfpathlineto{\pgfqpoint{1.894351in}{2.756389in}}%
\pgfpathlineto{\pgfqpoint{1.917652in}{2.980255in}}%
\pgfpathlineto{\pgfqpoint{1.938698in}{3.161958in}}%
\pgfpathlineto{\pgfqpoint{1.958992in}{3.318208in}}%
\pgfpathlineto{\pgfqpoint{1.977783in}{3.446465in}}%
\pgfpathlineto{\pgfqpoint{1.995822in}{3.555245in}}%
\pgfpathlineto{\pgfqpoint{2.013862in}{3.650645in}}%
\pgfpathlineto{\pgfqpoint{2.031149in}{3.730225in}}%
\pgfpathlineto{\pgfqpoint{2.047686in}{3.796162in}}%
\pgfpathlineto{\pgfqpoint{2.063470in}{3.850385in}}%
\pgfpathlineto{\pgfqpoint{2.078503in}{3.894584in}}%
\pgfpathlineto{\pgfqpoint{2.093536in}{3.931932in}}%
\pgfpathlineto{\pgfqpoint{2.107817in}{3.961395in}}%
\pgfpathlineto{\pgfqpoint{2.121346in}{3.984150in}}%
\pgfpathlineto{\pgfqpoint{2.133373in}{4.000325in}}%
\pgfpathlineto{\pgfqpoint{2.145399in}{4.012805in}}%
\pgfpathlineto{\pgfqpoint{2.156673in}{4.021230in}}%
\pgfpathlineto{\pgfqpoint{2.167196in}{4.026284in}}%
\pgfpathlineto{\pgfqpoint{2.176968in}{4.028578in}}%
\pgfpathlineto{\pgfqpoint{2.186739in}{4.028575in}}%
\pgfpathlineto{\pgfqpoint{2.196510in}{4.026281in}}%
\pgfpathlineto{\pgfqpoint{2.206282in}{4.021688in}}%
\pgfpathlineto{\pgfqpoint{2.216805in}{4.014152in}}%
\pgfpathlineto{\pgfqpoint{2.227328in}{4.003899in}}%
\pgfpathlineto{\pgfqpoint{2.238602in}{3.989844in}}%
\pgfpathlineto{\pgfqpoint{2.250628in}{3.971271in}}%
\pgfpathlineto{\pgfqpoint{2.263406in}{3.947363in}}%
\pgfpathlineto{\pgfqpoint{2.276936in}{3.917190in}}%
\pgfpathlineto{\pgfqpoint{2.291217in}{3.879682in}}%
\pgfpathlineto{\pgfqpoint{2.306250in}{3.833607in}}%
\pgfpathlineto{\pgfqpoint{2.322034in}{3.777549in}}%
\pgfpathlineto{\pgfqpoint{2.337819in}{3.713171in}}%
\pgfpathlineto{\pgfqpoint{2.354355in}{3.636278in}}%
\pgfpathlineto{\pgfqpoint{2.371643in}{3.544968in}}%
\pgfpathlineto{\pgfqpoint{2.389682in}{3.437162in}}%
\pgfpathlineto{\pgfqpoint{2.408473in}{3.310675in}}%
\pgfpathlineto{\pgfqpoint{2.428016in}{3.163344in}}%
\pgfpathlineto{\pgfqpoint{2.448310in}{2.993247in}}%
\pgfpathlineto{\pgfqpoint{2.470859in}{2.784513in}}%
\pgfpathlineto{\pgfqpoint{2.495663in}{2.533282in}}%
\pgfpathlineto{\pgfqpoint{2.526480in}{2.196998in}}%
\pgfpathlineto{\pgfqpoint{2.600893in}{1.374416in}}%
\pgfpathlineto{\pgfqpoint{2.621939in}{1.172321in}}%
\pgfpathlineto{\pgfqpoint{2.639227in}{1.027066in}}%
\pgfpathlineto{\pgfqpoint{2.653508in}{0.924295in}}%
\pgfpathlineto{\pgfqpoint{2.666286in}{0.847305in}}%
\pgfpathlineto{\pgfqpoint{2.677560in}{0.792114in}}%
\pgfpathlineto{\pgfqpoint{2.687332in}{0.754491in}}%
\pgfpathlineto{\pgfqpoint{2.695600in}{0.730333in}}%
\pgfpathlineto{\pgfqpoint{2.703116in}{0.714617in}}%
\pgfpathlineto{\pgfqpoint{2.709129in}{0.706385in}}%
\pgfpathlineto{\pgfqpoint{2.714391in}{0.702370in}}%
\pgfpathlineto{\pgfqpoint{2.718900in}{0.701305in}}%
\pgfpathlineto{\pgfqpoint{2.723410in}{0.702433in}}%
\pgfpathlineto{\pgfqpoint{2.727920in}{0.705750in}}%
\pgfpathlineto{\pgfqpoint{2.733182in}{0.712375in}}%
\pgfpathlineto{\pgfqpoint{2.739195in}{0.723554in}}%
\pgfpathlineto{\pgfqpoint{2.746711in}{0.742870in}}%
\pgfpathlineto{\pgfqpoint{2.754979in}{0.770841in}}%
\pgfpathlineto{\pgfqpoint{2.764750in}{0.812709in}}%
\pgfpathlineto{\pgfqpoint{2.775273in}{0.868015in}}%
\pgfpathlineto{\pgfqpoint{2.787300in}{0.943411in}}%
\pgfpathlineto{\pgfqpoint{2.801581in}{1.048363in}}%
\pgfpathlineto{\pgfqpoint{2.818117in}{1.188157in}}%
\pgfpathlineto{\pgfqpoint{2.837660in}{1.373857in}}%
\pgfpathlineto{\pgfqpoint{2.863215in}{1.639959in}}%
\pgfpathlineto{\pgfqpoint{2.953412in}{2.599543in}}%
\pgfpathlineto{\pgfqpoint{2.977465in}{2.821418in}}%
\pgfpathlineto{\pgfqpoint{2.998511in}{2.996073in}}%
\pgfpathlineto{\pgfqpoint{3.018053in}{3.140798in}}%
\pgfpathlineto{\pgfqpoint{3.036844in}{3.263752in}}%
\pgfpathlineto{\pgfqpoint{3.054132in}{3.362886in}}%
\pgfpathlineto{\pgfqpoint{3.070668in}{3.445397in}}%
\pgfpathlineto{\pgfqpoint{3.086453in}{3.513210in}}%
\pgfpathlineto{\pgfqpoint{3.101485in}{3.568138in}}%
\pgfpathlineto{\pgfqpoint{3.115015in}{3.609747in}}%
\pgfpathlineto{\pgfqpoint{3.127793in}{3.642418in}}%
\pgfpathlineto{\pgfqpoint{3.139819in}{3.667419in}}%
\pgfpathlineto{\pgfqpoint{3.151094in}{3.685895in}}%
\pgfpathlineto{\pgfqpoint{3.161617in}{3.698873in}}%
\pgfpathlineto{\pgfqpoint{3.170636in}{3.706758in}}%
\pgfpathlineto{\pgfqpoint{3.178904in}{3.711381in}}%
\pgfpathlineto{\pgfqpoint{3.187172in}{3.713526in}}%
\pgfpathlineto{\pgfqpoint{3.194689in}{3.713331in}}%
\pgfpathlineto{\pgfqpoint{3.202957in}{3.710763in}}%
\pgfpathlineto{\pgfqpoint{3.211225in}{3.705725in}}%
\pgfpathlineto{\pgfqpoint{3.220245in}{3.697406in}}%
\pgfpathlineto{\pgfqpoint{3.230016in}{3.685053in}}%
\pgfpathlineto{\pgfqpoint{3.240539in}{3.667831in}}%
\pgfpathlineto{\pgfqpoint{3.251813in}{3.644820in}}%
\pgfpathlineto{\pgfqpoint{3.263840in}{3.615003in}}%
\pgfpathlineto{\pgfqpoint{3.276618in}{3.577251in}}%
\pgfpathlineto{\pgfqpoint{3.290147in}{3.530323in}}%
\pgfpathlineto{\pgfqpoint{3.304428in}{3.472851in}}%
\pgfpathlineto{\pgfqpoint{3.320213in}{3.399628in}}%
\pgfpathlineto{\pgfqpoint{3.336749in}{3.311757in}}%
\pgfpathlineto{\pgfqpoint{3.354037in}{3.207481in}}%
\pgfpathlineto{\pgfqpoint{3.372076in}{3.085073in}}%
\pgfpathlineto{\pgfqpoint{3.391619in}{2.936990in}}%
\pgfpathlineto{\pgfqpoint{3.412665in}{2.760290in}}%
\pgfpathlineto{\pgfqpoint{3.436717in}{2.538597in}}%
\pgfpathlineto{\pgfqpoint{3.465279in}{2.253609in}}%
\pgfpathlineto{\pgfqpoint{3.554725in}{1.343829in}}%
\pgfpathlineto{\pgfqpoint{3.574267in}{1.176818in}}%
\pgfpathlineto{\pgfqpoint{3.590803in}{1.054093in}}%
\pgfpathlineto{\pgfqpoint{3.605084in}{0.964366in}}%
\pgfpathlineto{\pgfqpoint{3.617111in}{0.901843in}}%
\pgfpathlineto{\pgfqpoint{3.627634in}{0.857638in}}%
\pgfpathlineto{\pgfqpoint{3.637405in}{0.825795in}}%
\pgfpathlineto{\pgfqpoint{3.645673in}{0.805993in}}%
\pgfpathlineto{\pgfqpoint{3.652438in}{0.794757in}}%
\pgfpathlineto{\pgfqpoint{3.658451in}{0.788562in}}%
\pgfpathlineto{\pgfqpoint{3.663712in}{0.786085in}}%
\pgfpathlineto{\pgfqpoint{3.668222in}{0.786153in}}%
\pgfpathlineto{\pgfqpoint{3.672732in}{0.788241in}}%
\pgfpathlineto{\pgfqpoint{3.677994in}{0.793222in}}%
\pgfpathlineto{\pgfqpoint{3.684007in}{0.802250in}}%
\pgfpathlineto{\pgfqpoint{3.690771in}{0.816617in}}%
\pgfpathlineto{\pgfqpoint{3.699040in}{0.840126in}}%
\pgfpathlineto{\pgfqpoint{3.708059in}{0.873049in}}%
\pgfpathlineto{\pgfqpoint{3.718582in}{0.920692in}}%
\pgfpathlineto{\pgfqpoint{3.730608in}{0.986652in}}%
\pgfpathlineto{\pgfqpoint{3.744138in}{1.074388in}}%
\pgfpathlineto{\pgfqpoint{3.759922in}{1.192818in}}%
\pgfpathlineto{\pgfqpoint{3.777962in}{1.345906in}}%
\pgfpathlineto{\pgfqpoint{3.800511in}{1.557192in}}%
\pgfpathlineto{\pgfqpoint{3.836590in}{1.919772in}}%
\pgfpathlineto{\pgfqpoint{3.883191in}{2.383654in}}%
\pgfpathlineto{\pgfqpoint{3.909499in}{2.623297in}}%
\pgfpathlineto{\pgfqpoint{3.932048in}{2.809165in}}%
\pgfpathlineto{\pgfqpoint{3.952342in}{2.958734in}}%
\pgfpathlineto{\pgfqpoint{3.971133in}{3.081283in}}%
\pgfpathlineto{\pgfqpoint{3.988421in}{3.180131in}}%
\pgfpathlineto{\pgfqpoint{4.004957in}{3.262143in}}%
\pgfpathlineto{\pgfqpoint{4.019990in}{3.326116in}}%
\pgfpathlineto{\pgfqpoint{4.034271in}{3.377657in}}%
\pgfpathlineto{\pgfqpoint{4.047801in}{3.418295in}}%
\pgfpathlineto{\pgfqpoint{4.059827in}{3.447816in}}%
\pgfpathlineto{\pgfqpoint{4.071102in}{3.469915in}}%
\pgfpathlineto{\pgfqpoint{4.081625in}{3.485721in}}%
\pgfpathlineto{\pgfqpoint{4.090644in}{3.495595in}}%
\pgfpathlineto{\pgfqpoint{4.098912in}{3.501684in}}%
\pgfpathlineto{\pgfqpoint{4.106429in}{3.504771in}}%
\pgfpathlineto{\pgfqpoint{4.113945in}{3.505532in}}%
\pgfpathlineto{\pgfqpoint{4.121461in}{3.503973in}}%
\pgfpathlineto{\pgfqpoint{4.128978in}{3.500094in}}%
\pgfpathlineto{\pgfqpoint{4.137246in}{3.493146in}}%
\pgfpathlineto{\pgfqpoint{4.146266in}{3.482360in}}%
\pgfpathlineto{\pgfqpoint{4.156037in}{3.466888in}}%
\pgfpathlineto{\pgfqpoint{4.166560in}{3.445802in}}%
\pgfpathlineto{\pgfqpoint{4.177834in}{3.418088in}}%
\pgfpathlineto{\pgfqpoint{4.189861in}{3.382647in}}%
\pgfpathlineto{\pgfqpoint{4.203390in}{3.335461in}}%
\pgfpathlineto{\pgfqpoint{4.217671in}{3.277186in}}%
\pgfpathlineto{\pgfqpoint{4.232704in}{3.206391in}}%
\pgfpathlineto{\pgfqpoint{4.249240in}{3.117304in}}%
\pgfpathlineto{\pgfqpoint{4.266528in}{3.011701in}}%
\pgfpathlineto{\pgfqpoint{4.285319in}{2.882795in}}%
\pgfpathlineto{\pgfqpoint{4.305613in}{2.727871in}}%
\pgfpathlineto{\pgfqpoint{4.328914in}{2.531827in}}%
\pgfpathlineto{\pgfqpoint{4.355973in}{2.284393in}}%
\pgfpathlineto{\pgfqpoint{4.401072in}{1.847410in}}%
\pgfpathlineto{\pgfqpoint{4.437150in}{1.507246in}}%
\pgfpathlineto{\pgfqpoint{4.459700in}{1.315028in}}%
\pgfpathlineto{\pgfqpoint{4.477739in}{1.179357in}}%
\pgfpathlineto{\pgfqpoint{4.492772in}{1.081712in}}%
\pgfpathlineto{\pgfqpoint{4.506301in}{1.007617in}}%
\pgfpathlineto{\pgfqpoint{4.518328in}{0.953784in}}%
\pgfpathlineto{\pgfqpoint{4.528851in}{0.916572in}}%
\pgfpathlineto{\pgfqpoint{4.537870in}{0.892333in}}%
\pgfpathlineto{\pgfqpoint{4.545387in}{0.877676in}}%
\pgfpathlineto{\pgfqpoint{4.552151in}{0.868859in}}%
\pgfpathlineto{\pgfqpoint{4.558165in}{0.864531in}}%
\pgfpathlineto{\pgfqpoint{4.563426in}{0.863464in}}%
\pgfpathlineto{\pgfqpoint{4.568688in}{0.864936in}}%
\pgfpathlineto{\pgfqpoint{4.573949in}{0.868940in}}%
\pgfpathlineto{\pgfqpoint{4.579962in}{0.876600in}}%
\pgfpathlineto{\pgfqpoint{4.586727in}{0.889112in}}%
\pgfpathlineto{\pgfqpoint{4.594243in}{0.907777in}}%
\pgfpathlineto{\pgfqpoint{4.603263in}{0.936641in}}%
\pgfpathlineto{\pgfqpoint{4.613034in}{0.975601in}}%
\pgfpathlineto{\pgfqpoint{4.624309in}{1.030013in}}%
\pgfpathlineto{\pgfqpoint{4.637087in}{1.103094in}}%
\pgfpathlineto{\pgfqpoint{4.652120in}{1.203070in}}%
\pgfpathlineto{\pgfqpoint{4.669407in}{1.334085in}}%
\pgfpathlineto{\pgfqpoint{4.690453in}{1.511748in}}%
\pgfpathlineto{\pgfqpoint{4.719767in}{1.780285in}}%
\pgfpathlineto{\pgfqpoint{4.785912in}{2.392218in}}%
\pgfpathlineto{\pgfqpoint{4.811467in}{2.604685in}}%
\pgfpathlineto{\pgfqpoint{4.833265in}{2.767950in}}%
\pgfpathlineto{\pgfqpoint{4.852808in}{2.898297in}}%
\pgfpathlineto{\pgfqpoint{4.870847in}{3.004254in}}%
\pgfpathlineto{\pgfqpoint{4.887383in}{3.088865in}}%
\pgfpathlineto{\pgfqpoint{4.903167in}{3.158302in}}%
\pgfpathlineto{\pgfqpoint{4.917449in}{3.211552in}}%
\pgfpathlineto{\pgfqpoint{4.930978in}{3.253624in}}%
\pgfpathlineto{\pgfqpoint{4.943004in}{3.284211in}}%
\pgfpathlineto{\pgfqpoint{4.954279in}{3.307093in}}%
\pgfpathlineto{\pgfqpoint{4.964802in}{3.323419in}}%
\pgfpathlineto{\pgfqpoint{4.973822in}{3.333563in}}%
\pgfpathlineto{\pgfqpoint{4.982090in}{3.339752in}}%
\pgfpathlineto{\pgfqpoint{4.989606in}{3.342803in}}%
\pgfpathlineto{\pgfqpoint{4.997123in}{3.343407in}}%
\pgfpathlineto{\pgfqpoint{5.004639in}{3.341567in}}%
\pgfpathlineto{\pgfqpoint{5.012155in}{3.337286in}}%
\pgfpathlineto{\pgfqpoint{5.020423in}{3.329759in}}%
\pgfpathlineto{\pgfqpoint{5.029443in}{3.318182in}}%
\pgfpathlineto{\pgfqpoint{5.039214in}{3.301673in}}%
\pgfpathlineto{\pgfqpoint{5.049737in}{3.279278in}}%
\pgfpathlineto{\pgfqpoint{5.061012in}{3.249966in}}%
\pgfpathlineto{\pgfqpoint{5.073790in}{3.210092in}}%
\pgfpathlineto{\pgfqpoint{5.087319in}{3.160169in}}%
\pgfpathlineto{\pgfqpoint{5.101601in}{3.098907in}}%
\pgfpathlineto{\pgfqpoint{5.117385in}{3.021046in}}%
\pgfpathlineto{\pgfqpoint{5.134673in}{2.923760in}}%
\pgfpathlineto{\pgfqpoint{5.152712in}{2.809303in}}%
\pgfpathlineto{\pgfqpoint{5.173006in}{2.665683in}}%
\pgfpathlineto{\pgfqpoint{5.195556in}{2.489597in}}%
\pgfpathlineto{\pgfqpoint{5.222615in}{2.259863in}}%
\pgfpathlineto{\pgfqpoint{5.267713in}{1.853847in}}%
\pgfpathlineto{\pgfqpoint{5.303792in}{1.537725in}}%
\pgfpathlineto{\pgfqpoint{5.326341in}{1.358961in}}%
\pgfpathlineto{\pgfqpoint{5.344380in}{1.232624in}}%
\pgfpathlineto{\pgfqpoint{5.360165in}{1.137356in}}%
\pgfpathlineto{\pgfqpoint{5.373694in}{1.068770in}}%
\pgfpathlineto{\pgfqpoint{5.385721in}{1.018929in}}%
\pgfpathlineto{\pgfqpoint{5.396244in}{0.984460in}}%
\pgfpathlineto{\pgfqpoint{5.405263in}{0.961989in}}%
\pgfpathlineto{\pgfqpoint{5.412780in}{0.948381in}}%
\pgfpathlineto{\pgfqpoint{5.419544in}{0.940175in}}%
\pgfpathlineto{\pgfqpoint{5.425558in}{0.936121in}}%
\pgfpathlineto{\pgfqpoint{5.430819in}{0.935086in}}%
\pgfpathlineto{\pgfqpoint{5.436081in}{0.936395in}}%
\pgfpathlineto{\pgfqpoint{5.441342in}{0.940044in}}%
\pgfpathlineto{\pgfqpoint{5.447355in}{0.947062in}}%
\pgfpathlineto{\pgfqpoint{5.454120in}{0.958556in}}%
\pgfpathlineto{\pgfqpoint{5.461636in}{0.975728in}}%
\pgfpathlineto{\pgfqpoint{5.470656in}{1.002311in}}%
\pgfpathlineto{\pgfqpoint{5.480427in}{1.038223in}}%
\pgfpathlineto{\pgfqpoint{5.491702in}{1.088413in}}%
\pgfpathlineto{\pgfqpoint{5.504480in}{1.155870in}}%
\pgfpathlineto{\pgfqpoint{5.519513in}{1.248222in}}%
\pgfpathlineto{\pgfqpoint{5.534545in}{1.352755in}}%
\pgfpathlineto{\pgfqpoint{5.534545in}{1.352755in}}%
\pgfusepath{stroke}%
\end{pgfscope}%
\begin{pgfscope}%
\pgfpathrectangle{\pgfqpoint{0.800000in}{0.528000in}}{\pgfqpoint{4.960000in}{3.696000in}}%
\pgfusepath{clip}%
\pgfsetrectcap%
\pgfsetroundjoin%
\pgfsetlinewidth{1.505625pt}%
\definecolor{currentstroke}{rgb}{1.000000,0.498039,0.054902}%
\pgfsetstrokecolor{currentstroke}%
\pgfsetdash{}{0pt}%
\pgfpathmoveto{\pgfqpoint{1.025455in}{1.954352in}}%
\pgfpathlineto{\pgfqpoint{1.070553in}{1.953247in}}%
\pgfpathlineto{\pgfqpoint{1.118658in}{1.949818in}}%
\pgfpathlineto{\pgfqpoint{1.177286in}{1.943291in}}%
\pgfpathlineto{\pgfqpoint{1.329117in}{1.925221in}}%
\pgfpathlineto{\pgfqpoint{1.377974in}{1.922390in}}%
\pgfpathlineto{\pgfqpoint{1.423824in}{1.921960in}}%
\pgfpathlineto{\pgfqpoint{1.470426in}{1.923782in}}%
\pgfpathlineto{\pgfqpoint{1.521537in}{1.928055in}}%
\pgfpathlineto{\pgfqpoint{1.591440in}{1.936345in}}%
\pgfpathlineto{\pgfqpoint{1.693663in}{1.948327in}}%
\pgfpathlineto{\pgfqpoint{1.746278in}{1.952083in}}%
\pgfpathlineto{\pgfqpoint{1.781605in}{1.954507in}}%
\pgfpathlineto{\pgfqpoint{1.786115in}{1.958602in}}%
\pgfpathlineto{\pgfqpoint{1.791376in}{1.966760in}}%
\pgfpathlineto{\pgfqpoint{1.797389in}{1.980537in}}%
\pgfpathlineto{\pgfqpoint{1.804154in}{2.001710in}}%
\pgfpathlineto{\pgfqpoint{1.812422in}{2.035735in}}%
\pgfpathlineto{\pgfqpoint{1.821442in}{2.083059in}}%
\pgfpathlineto{\pgfqpoint{1.831213in}{2.146326in}}%
\pgfpathlineto{\pgfqpoint{1.842488in}{2.234804in}}%
\pgfpathlineto{\pgfqpoint{1.854514in}{2.347424in}}%
\pgfpathlineto{\pgfqpoint{1.870298in}{2.519224in}}%
\pgfpathlineto{\pgfqpoint{1.896606in}{2.795765in}}%
\pgfpathlineto{\pgfqpoint{1.919155in}{3.010489in}}%
\pgfpathlineto{\pgfqpoint{1.940201in}{3.190571in}}%
\pgfpathlineto{\pgfqpoint{1.960495in}{3.345248in}}%
\pgfpathlineto{\pgfqpoint{1.979286in}{3.472130in}}%
\pgfpathlineto{\pgfqpoint{1.997326in}{3.579722in}}%
\pgfpathlineto{\pgfqpoint{2.015365in}{3.674110in}}%
\pgfpathlineto{\pgfqpoint{2.032653in}{3.752915in}}%
\pgfpathlineto{\pgfqpoint{2.049189in}{3.818311in}}%
\pgfpathlineto{\pgfqpoint{2.064973in}{3.872217in}}%
\pgfpathlineto{\pgfqpoint{2.080758in}{3.918332in}}%
\pgfpathlineto{\pgfqpoint{2.095791in}{3.955443in}}%
\pgfpathlineto{\pgfqpoint{2.110072in}{3.984892in}}%
\pgfpathlineto{\pgfqpoint{2.123601in}{4.007825in}}%
\pgfpathlineto{\pgfqpoint{2.136379in}{4.025235in}}%
\pgfpathlineto{\pgfqpoint{2.148405in}{4.037978in}}%
\pgfpathlineto{\pgfqpoint{2.159680in}{4.046803in}}%
\pgfpathlineto{\pgfqpoint{2.170203in}{4.052368in}}%
\pgfpathlineto{\pgfqpoint{2.180726in}{4.055388in}}%
\pgfpathlineto{\pgfqpoint{2.190497in}{4.055933in}}%
\pgfpathlineto{\pgfqpoint{2.200269in}{4.054309in}}%
\pgfpathlineto{\pgfqpoint{2.210040in}{4.050512in}}%
\pgfpathlineto{\pgfqpoint{2.220563in}{4.043977in}}%
\pgfpathlineto{\pgfqpoint{2.231837in}{4.034128in}}%
\pgfpathlineto{\pgfqpoint{2.243112in}{4.021281in}}%
\pgfpathlineto{\pgfqpoint{2.255138in}{4.004193in}}%
\pgfpathlineto{\pgfqpoint{2.267916in}{3.982093in}}%
\pgfpathlineto{\pgfqpoint{2.281446in}{3.954097in}}%
\pgfpathlineto{\pgfqpoint{2.295727in}{3.919188in}}%
\pgfpathlineto{\pgfqpoint{2.310760in}{3.876189in}}%
\pgfpathlineto{\pgfqpoint{2.325792in}{3.826413in}}%
\pgfpathlineto{\pgfqpoint{2.341577in}{3.766419in}}%
\pgfpathlineto{\pgfqpoint{2.358113in}{3.694548in}}%
\pgfpathlineto{\pgfqpoint{2.375401in}{3.608930in}}%
\pgfpathlineto{\pgfqpoint{2.392688in}{3.511974in}}%
\pgfpathlineto{\pgfqpoint{2.410728in}{3.398095in}}%
\pgfpathlineto{\pgfqpoint{2.429519in}{3.265132in}}%
\pgfpathlineto{\pgfqpoint{2.449062in}{3.111003in}}%
\pgfpathlineto{\pgfqpoint{2.470107in}{2.927068in}}%
\pgfpathlineto{\pgfqpoint{2.493408in}{2.702846in}}%
\pgfpathlineto{\pgfqpoint{2.519716in}{2.427196in}}%
\pgfpathlineto{\pgfqpoint{2.555794in}{2.023341in}}%
\pgfpathlineto{\pgfqpoint{2.606154in}{1.461554in}}%
\pgfpathlineto{\pgfqpoint{2.628704in}{1.234901in}}%
\pgfpathlineto{\pgfqpoint{2.646743in}{1.074343in}}%
\pgfpathlineto{\pgfqpoint{2.662527in}{0.953111in}}%
\pgfpathlineto{\pgfqpoint{2.676057in}{0.865782in}}%
\pgfpathlineto{\pgfqpoint{2.688083in}{0.802323in}}%
\pgfpathlineto{\pgfqpoint{2.698606in}{0.758461in}}%
\pgfpathlineto{\pgfqpoint{2.707626in}{0.729904in}}%
\pgfpathlineto{\pgfqpoint{2.715142in}{0.712651in}}%
\pgfpathlineto{\pgfqpoint{2.721907in}{0.702291in}}%
\pgfpathlineto{\pgfqpoint{2.727168in}{0.697646in}}%
\pgfpathlineto{\pgfqpoint{2.731678in}{0.696051in}}%
\pgfpathlineto{\pgfqpoint{2.736188in}{0.696661in}}%
\pgfpathlineto{\pgfqpoint{2.740698in}{0.699472in}}%
\pgfpathlineto{\pgfqpoint{2.745959in}{0.705524in}}%
\pgfpathlineto{\pgfqpoint{2.751973in}{0.716071in}}%
\pgfpathlineto{\pgfqpoint{2.758737in}{0.732515in}}%
\pgfpathlineto{\pgfqpoint{2.767005in}{0.759077in}}%
\pgfpathlineto{\pgfqpoint{2.776025in}{0.795951in}}%
\pgfpathlineto{\pgfqpoint{2.786548in}{0.848973in}}%
\pgfpathlineto{\pgfqpoint{2.798574in}{0.922019in}}%
\pgfpathlineto{\pgfqpoint{2.812104in}{1.018784in}}%
\pgfpathlineto{\pgfqpoint{2.827888in}{1.148931in}}%
\pgfpathlineto{\pgfqpoint{2.846679in}{1.323946in}}%
\pgfpathlineto{\pgfqpoint{2.869980in}{1.563140in}}%
\pgfpathlineto{\pgfqpoint{2.910569in}{2.007358in}}%
\pgfpathlineto{\pgfqpoint{2.950406in}{2.434467in}}%
\pgfpathlineto{\pgfqpoint{2.976713in}{2.693385in}}%
\pgfpathlineto{\pgfqpoint{2.999262in}{2.894532in}}%
\pgfpathlineto{\pgfqpoint{3.020308in}{3.062739in}}%
\pgfpathlineto{\pgfqpoint{3.039851in}{3.201234in}}%
\pgfpathlineto{\pgfqpoint{3.057890in}{3.313808in}}%
\pgfpathlineto{\pgfqpoint{3.075178in}{3.408098in}}%
\pgfpathlineto{\pgfqpoint{3.091714in}{3.486127in}}%
\pgfpathlineto{\pgfqpoint{3.107499in}{3.549836in}}%
\pgfpathlineto{\pgfqpoint{3.121780in}{3.598686in}}%
\pgfpathlineto{\pgfqpoint{3.135309in}{3.637486in}}%
\pgfpathlineto{\pgfqpoint{3.148087in}{3.667624in}}%
\pgfpathlineto{\pgfqpoint{3.160113in}{3.690346in}}%
\pgfpathlineto{\pgfqpoint{3.170636in}{3.705824in}}%
\pgfpathlineto{\pgfqpoint{3.180408in}{3.716569in}}%
\pgfpathlineto{\pgfqpoint{3.189427in}{3.723419in}}%
\pgfpathlineto{\pgfqpoint{3.197695in}{3.727130in}}%
\pgfpathlineto{\pgfqpoint{3.205963in}{3.728396in}}%
\pgfpathlineto{\pgfqpoint{3.214231in}{3.727225in}}%
\pgfpathlineto{\pgfqpoint{3.222500in}{3.723616in}}%
\pgfpathlineto{\pgfqpoint{3.230768in}{3.717566in}}%
\pgfpathlineto{\pgfqpoint{3.239787in}{3.708172in}}%
\pgfpathlineto{\pgfqpoint{3.249559in}{3.694685in}}%
\pgfpathlineto{\pgfqpoint{3.260082in}{3.676273in}}%
\pgfpathlineto{\pgfqpoint{3.271356in}{3.652019in}}%
\pgfpathlineto{\pgfqpoint{3.283382in}{3.620903in}}%
\pgfpathlineto{\pgfqpoint{3.296912in}{3.579302in}}%
\pgfpathlineto{\pgfqpoint{3.311193in}{3.527641in}}%
\pgfpathlineto{\pgfqpoint{3.326226in}{3.464451in}}%
\pgfpathlineto{\pgfqpoint{3.342010in}{3.388138in}}%
\pgfpathlineto{\pgfqpoint{3.358546in}{3.297004in}}%
\pgfpathlineto{\pgfqpoint{3.375834in}{3.189302in}}%
\pgfpathlineto{\pgfqpoint{3.394625in}{3.057783in}}%
\pgfpathlineto{\pgfqpoint{3.414919in}{2.899132in}}%
\pgfpathlineto{\pgfqpoint{3.436717in}{2.710541in}}%
\pgfpathlineto{\pgfqpoint{3.461521in}{2.475711in}}%
\pgfpathlineto{\pgfqpoint{3.493090in}{2.153713in}}%
\pgfpathlineto{\pgfqpoint{3.562993in}{1.432696in}}%
\pgfpathlineto{\pgfqpoint{3.584039in}{1.242071in}}%
\pgfpathlineto{\pgfqpoint{3.601326in}{1.104112in}}%
\pgfpathlineto{\pgfqpoint{3.616359in}{1.000967in}}%
\pgfpathlineto{\pgfqpoint{3.629137in}{0.927320in}}%
\pgfpathlineto{\pgfqpoint{3.640412in}{0.874011in}}%
\pgfpathlineto{\pgfqpoint{3.650935in}{0.834727in}}%
\pgfpathlineto{\pgfqpoint{3.659954in}{0.809422in}}%
\pgfpathlineto{\pgfqpoint{3.667471in}{0.794385in}}%
\pgfpathlineto{\pgfqpoint{3.674235in}{0.785624in}}%
\pgfpathlineto{\pgfqpoint{3.679497in}{0.781959in}}%
\pgfpathlineto{\pgfqpoint{3.684758in}{0.781059in}}%
\pgfpathlineto{\pgfqpoint{3.689268in}{0.782488in}}%
\pgfpathlineto{\pgfqpoint{3.694530in}{0.786716in}}%
\pgfpathlineto{\pgfqpoint{3.700543in}{0.794906in}}%
\pgfpathlineto{\pgfqpoint{3.707308in}{0.808363in}}%
\pgfpathlineto{\pgfqpoint{3.714824in}{0.828503in}}%
\pgfpathlineto{\pgfqpoint{3.723844in}{0.859713in}}%
\pgfpathlineto{\pgfqpoint{3.733615in}{0.901902in}}%
\pgfpathlineto{\pgfqpoint{3.744890in}{0.960880in}}%
\pgfpathlineto{\pgfqpoint{3.757667in}{1.040141in}}%
\pgfpathlineto{\pgfqpoint{3.772700in}{1.148608in}}%
\pgfpathlineto{\pgfqpoint{3.789988in}{1.290751in}}%
\pgfpathlineto{\pgfqpoint{3.811034in}{1.483437in}}%
\pgfpathlineto{\pgfqpoint{3.840348in}{1.774416in}}%
\pgfpathlineto{\pgfqpoint{3.905741in}{2.429117in}}%
\pgfpathlineto{\pgfqpoint{3.931296in}{2.659486in}}%
\pgfpathlineto{\pgfqpoint{3.953094in}{2.837063in}}%
\pgfpathlineto{\pgfqpoint{3.973388in}{2.984720in}}%
\pgfpathlineto{\pgfqpoint{3.992179in}{3.105378in}}%
\pgfpathlineto{\pgfqpoint{4.009467in}{3.202444in}}%
\pgfpathlineto{\pgfqpoint{4.026003in}{3.282756in}}%
\pgfpathlineto{\pgfqpoint{4.041036in}{3.345210in}}%
\pgfpathlineto{\pgfqpoint{4.055317in}{3.395343in}}%
\pgfpathlineto{\pgfqpoint{4.068095in}{3.432708in}}%
\pgfpathlineto{\pgfqpoint{4.080121in}{3.461499in}}%
\pgfpathlineto{\pgfqpoint{4.091396in}{3.482943in}}%
\pgfpathlineto{\pgfqpoint{4.101919in}{3.498163in}}%
\pgfpathlineto{\pgfqpoint{4.110938in}{3.507554in}}%
\pgfpathlineto{\pgfqpoint{4.119207in}{3.513216in}}%
\pgfpathlineto{\pgfqpoint{4.126723in}{3.515928in}}%
\pgfpathlineto{\pgfqpoint{4.134239in}{3.516328in}}%
\pgfpathlineto{\pgfqpoint{4.141756in}{3.514417in}}%
\pgfpathlineto{\pgfqpoint{4.149272in}{3.510198in}}%
\pgfpathlineto{\pgfqpoint{4.157540in}{3.502890in}}%
\pgfpathlineto{\pgfqpoint{4.166560in}{3.491723in}}%
\pgfpathlineto{\pgfqpoint{4.176331in}{3.475855in}}%
\pgfpathlineto{\pgfqpoint{4.186854in}{3.454357in}}%
\pgfpathlineto{\pgfqpoint{4.198129in}{3.426220in}}%
\pgfpathlineto{\pgfqpoint{4.210907in}{3.387897in}}%
\pgfpathlineto{\pgfqpoint{4.224436in}{3.339806in}}%
\pgfpathlineto{\pgfqpoint{4.238717in}{3.280584in}}%
\pgfpathlineto{\pgfqpoint{4.253750in}{3.208799in}}%
\pgfpathlineto{\pgfqpoint{4.270286in}{3.118627in}}%
\pgfpathlineto{\pgfqpoint{4.287574in}{3.011891in}}%
\pgfpathlineto{\pgfqpoint{4.306365in}{2.881758in}}%
\pgfpathlineto{\pgfqpoint{4.326659in}{2.725521in}}%
\pgfpathlineto{\pgfqpoint{4.349960in}{2.528011in}}%
\pgfpathlineto{\pgfqpoint{4.377771in}{2.271835in}}%
\pgfpathlineto{\pgfqpoint{4.427379in}{1.788634in}}%
\pgfpathlineto{\pgfqpoint{4.459700in}{1.485296in}}%
\pgfpathlineto{\pgfqpoint{4.481497in}{1.300407in}}%
\pgfpathlineto{\pgfqpoint{4.499537in}{1.165692in}}%
\pgfpathlineto{\pgfqpoint{4.514569in}{1.069157in}}%
\pgfpathlineto{\pgfqpoint{4.528099in}{0.996288in}}%
\pgfpathlineto{\pgfqpoint{4.539374in}{0.946645in}}%
\pgfpathlineto{\pgfqpoint{4.549896in}{0.909963in}}%
\pgfpathlineto{\pgfqpoint{4.558916in}{0.886235in}}%
\pgfpathlineto{\pgfqpoint{4.566433in}{0.872040in}}%
\pgfpathlineto{\pgfqpoint{4.573197in}{0.863666in}}%
\pgfpathlineto{\pgfqpoint{4.579210in}{0.859750in}}%
\pgfpathlineto{\pgfqpoint{4.584472in}{0.859058in}}%
\pgfpathlineto{\pgfqpoint{4.589733in}{0.860916in}}%
\pgfpathlineto{\pgfqpoint{4.594995in}{0.865317in}}%
\pgfpathlineto{\pgfqpoint{4.601008in}{0.873442in}}%
\pgfpathlineto{\pgfqpoint{4.607773in}{0.886490in}}%
\pgfpathlineto{\pgfqpoint{4.616041in}{0.907963in}}%
\pgfpathlineto{\pgfqpoint{4.625061in}{0.938149in}}%
\pgfpathlineto{\pgfqpoint{4.635584in}{0.981953in}}%
\pgfpathlineto{\pgfqpoint{4.647610in}{1.042733in}}%
\pgfpathlineto{\pgfqpoint{4.661139in}{1.123729in}}%
\pgfpathlineto{\pgfqpoint{4.676172in}{1.227703in}}%
\pgfpathlineto{\pgfqpoint{4.694211in}{1.368860in}}%
\pgfpathlineto{\pgfqpoint{4.716761in}{1.564407in}}%
\pgfpathlineto{\pgfqpoint{4.750584in}{1.880127in}}%
\pgfpathlineto{\pgfqpoint{4.801696in}{2.355306in}}%
\pgfpathlineto{\pgfqpoint{4.828003in}{2.578295in}}%
\pgfpathlineto{\pgfqpoint{4.850553in}{2.750770in}}%
\pgfpathlineto{\pgfqpoint{4.870847in}{2.888932in}}%
\pgfpathlineto{\pgfqpoint{4.888886in}{2.997153in}}%
\pgfpathlineto{\pgfqpoint{4.906174in}{3.087511in}}%
\pgfpathlineto{\pgfqpoint{4.921959in}{3.158419in}}%
\pgfpathlineto{\pgfqpoint{4.936240in}{3.212997in}}%
\pgfpathlineto{\pgfqpoint{4.949769in}{3.256325in}}%
\pgfpathlineto{\pgfqpoint{4.961795in}{3.288027in}}%
\pgfpathlineto{\pgfqpoint{4.973070in}{3.311955in}}%
\pgfpathlineto{\pgfqpoint{4.983593in}{3.329259in}}%
\pgfpathlineto{\pgfqpoint{4.992613in}{3.340244in}}%
\pgfpathlineto{\pgfqpoint{5.000881in}{3.347206in}}%
\pgfpathlineto{\pgfqpoint{5.008397in}{3.350963in}}%
\pgfpathlineto{\pgfqpoint{5.015914in}{3.352276in}}%
\pgfpathlineto{\pgfqpoint{5.023430in}{3.351148in}}%
\pgfpathlineto{\pgfqpoint{5.030946in}{3.347584in}}%
\pgfpathlineto{\pgfqpoint{5.038463in}{3.341584in}}%
\pgfpathlineto{\pgfqpoint{5.046731in}{3.332171in}}%
\pgfpathlineto{\pgfqpoint{5.055751in}{3.318540in}}%
\pgfpathlineto{\pgfqpoint{5.065522in}{3.299811in}}%
\pgfpathlineto{\pgfqpoint{5.076796in}{3.273075in}}%
\pgfpathlineto{\pgfqpoint{5.088823in}{3.238497in}}%
\pgfpathlineto{\pgfqpoint{5.101601in}{3.194898in}}%
\pgfpathlineto{\pgfqpoint{5.115882in}{3.137817in}}%
\pgfpathlineto{\pgfqpoint{5.130915in}{3.068254in}}%
\pgfpathlineto{\pgfqpoint{5.147451in}{2.980643in}}%
\pgfpathlineto{\pgfqpoint{5.164738in}{2.876918in}}%
\pgfpathlineto{\pgfqpoint{5.184281in}{2.745389in}}%
\pgfpathlineto{\pgfqpoint{5.205327in}{2.588118in}}%
\pgfpathlineto{\pgfqpoint{5.230131in}{2.384918in}}%
\pgfpathlineto{\pgfqpoint{5.261700in}{2.106317in}}%
\pgfpathlineto{\pgfqpoint{5.328596in}{1.510665in}}%
\pgfpathlineto{\pgfqpoint{5.350394in}{1.339426in}}%
\pgfpathlineto{\pgfqpoint{5.368433in}{1.214749in}}%
\pgfpathlineto{\pgfqpoint{5.383466in}{1.125444in}}%
\pgfpathlineto{\pgfqpoint{5.396995in}{1.058045in}}%
\pgfpathlineto{\pgfqpoint{5.408270in}{1.012129in}}%
\pgfpathlineto{\pgfqpoint{5.418793in}{0.978195in}}%
\pgfpathlineto{\pgfqpoint{5.427813in}{0.956235in}}%
\pgfpathlineto{\pgfqpoint{5.435329in}{0.943088in}}%
\pgfpathlineto{\pgfqpoint{5.442094in}{0.935320in}}%
\pgfpathlineto{\pgfqpoint{5.448107in}{0.931674in}}%
\pgfpathlineto{\pgfqpoint{5.453368in}{0.931008in}}%
\pgfpathlineto{\pgfqpoint{5.458630in}{0.932697in}}%
\pgfpathlineto{\pgfqpoint{5.463891in}{0.936735in}}%
\pgfpathlineto{\pgfqpoint{5.469904in}{0.944209in}}%
\pgfpathlineto{\pgfqpoint{5.476669in}{0.956227in}}%
\pgfpathlineto{\pgfqpoint{5.484937in}{0.976020in}}%
\pgfpathlineto{\pgfqpoint{5.493957in}{1.003862in}}%
\pgfpathlineto{\pgfqpoint{5.504480in}{1.044284in}}%
\pgfpathlineto{\pgfqpoint{5.516506in}{1.100398in}}%
\pgfpathlineto{\pgfqpoint{5.530036in}{1.175218in}}%
\pgfpathlineto{\pgfqpoint{5.534545in}{1.202697in}}%
\pgfpathlineto{\pgfqpoint{5.534545in}{1.202697in}}%
\pgfusepath{stroke}%
\end{pgfscope}%
\begin{pgfscope}%
\pgfsetrectcap%
\pgfsetmiterjoin%
\pgfsetlinewidth{0.803000pt}%
\definecolor{currentstroke}{rgb}{0.000000,0.000000,0.000000}%
\pgfsetstrokecolor{currentstroke}%
\pgfsetdash{}{0pt}%
\pgfpathmoveto{\pgfqpoint{0.800000in}{0.528000in}}%
\pgfpathlineto{\pgfqpoint{0.800000in}{4.224000in}}%
\pgfusepath{stroke}%
\end{pgfscope}%
\begin{pgfscope}%
\pgfsetrectcap%
\pgfsetmiterjoin%
\pgfsetlinewidth{0.803000pt}%
\definecolor{currentstroke}{rgb}{0.000000,0.000000,0.000000}%
\pgfsetstrokecolor{currentstroke}%
\pgfsetdash{}{0pt}%
\pgfpathmoveto{\pgfqpoint{5.760000in}{0.528000in}}%
\pgfpathlineto{\pgfqpoint{5.760000in}{4.224000in}}%
\pgfusepath{stroke}%
\end{pgfscope}%
\begin{pgfscope}%
\pgfsetrectcap%
\pgfsetmiterjoin%
\pgfsetlinewidth{0.803000pt}%
\definecolor{currentstroke}{rgb}{0.000000,0.000000,0.000000}%
\pgfsetstrokecolor{currentstroke}%
\pgfsetdash{}{0pt}%
\pgfpathmoveto{\pgfqpoint{0.800000in}{0.528000in}}%
\pgfpathlineto{\pgfqpoint{5.760000in}{0.528000in}}%
\pgfusepath{stroke}%
\end{pgfscope}%
\begin{pgfscope}%
\pgfsetrectcap%
\pgfsetmiterjoin%
\pgfsetlinewidth{0.803000pt}%
\definecolor{currentstroke}{rgb}{0.000000,0.000000,0.000000}%
\pgfsetstrokecolor{currentstroke}%
\pgfsetdash{}{0pt}%
\pgfpathmoveto{\pgfqpoint{0.800000in}{4.224000in}}%
\pgfpathlineto{\pgfqpoint{5.760000in}{4.224000in}}%
\pgfusepath{stroke}%
\end{pgfscope}%
\begin{pgfscope}%
\definecolor{textcolor}{rgb}{0.000000,0.000000,0.000000}%
\pgfsetstrokecolor{textcolor}%
\pgfsetfillcolor{textcolor}%
\pgftext[x=3.280000in,y=4.307333in,,base]{\color{textcolor}\rmfamily\fontsize{12.000000}{14.400000}\selectfont Power angle - comparison algebraic vs. non-algebraic}%
\end{pgfscope}%
\begin{pgfscope}%
\pgfsetbuttcap%
\pgfsetmiterjoin%
\definecolor{currentfill}{rgb}{1.000000,1.000000,1.000000}%
\pgfsetfillcolor{currentfill}%
\pgfsetfillopacity{0.800000}%
\pgfsetlinewidth{1.003750pt}%
\definecolor{currentstroke}{rgb}{0.800000,0.800000,0.800000}%
\pgfsetstrokecolor{currentstroke}%
\pgfsetstrokeopacity{0.800000}%
\pgfsetdash{}{0pt}%
\pgfpathmoveto{\pgfqpoint{4.391917in}{3.709108in}}%
\pgfpathlineto{\pgfqpoint{5.662778in}{3.709108in}}%
\pgfpathquadraticcurveto{\pgfqpoint{5.690556in}{3.709108in}}{\pgfqpoint{5.690556in}{3.736886in}}%
\pgfpathlineto{\pgfqpoint{5.690556in}{4.126778in}}%
\pgfpathquadraticcurveto{\pgfqpoint{5.690556in}{4.154556in}}{\pgfqpoint{5.662778in}{4.154556in}}%
\pgfpathlineto{\pgfqpoint{4.391917in}{4.154556in}}%
\pgfpathquadraticcurveto{\pgfqpoint{4.364140in}{4.154556in}}{\pgfqpoint{4.364140in}{4.126778in}}%
\pgfpathlineto{\pgfqpoint{4.364140in}{3.736886in}}%
\pgfpathquadraticcurveto{\pgfqpoint{4.364140in}{3.709108in}}{\pgfqpoint{4.391917in}{3.709108in}}%
\pgfpathlineto{\pgfqpoint{4.391917in}{3.709108in}}%
\pgfpathclose%
\pgfusepath{stroke,fill}%
\end{pgfscope}%
\begin{pgfscope}%
\pgfsetrectcap%
\pgfsetroundjoin%
\pgfsetlinewidth{1.505625pt}%
\definecolor{currentstroke}{rgb}{0.121569,0.466667,0.705882}%
\pgfsetstrokecolor{currentstroke}%
\pgfsetdash{}{0pt}%
\pgfpathmoveto{\pgfqpoint{4.419695in}{4.045411in}}%
\pgfpathlineto{\pgfqpoint{4.558584in}{4.045411in}}%
\pgfpathlineto{\pgfqpoint{4.697473in}{4.045411in}}%
\pgfusepath{stroke}%
\end{pgfscope}%
\begin{pgfscope}%
\definecolor{textcolor}{rgb}{0.000000,0.000000,0.000000}%
\pgfsetstrokecolor{textcolor}%
\pgfsetfillcolor{textcolor}%
\pgftext[x=4.808584in,y=3.996800in,left,base]{\color{textcolor}\rmfamily\fontsize{10.000000}{12.000000}\selectfont algebraic}%
\end{pgfscope}%
\begin{pgfscope}%
\pgfsetrectcap%
\pgfsetroundjoin%
\pgfsetlinewidth{1.505625pt}%
\definecolor{currentstroke}{rgb}{1.000000,0.498039,0.054902}%
\pgfsetstrokecolor{currentstroke}%
\pgfsetdash{}{0pt}%
\pgfpathmoveto{\pgfqpoint{4.419695in}{3.843521in}}%
\pgfpathlineto{\pgfqpoint{4.558584in}{3.843521in}}%
\pgfpathlineto{\pgfqpoint{4.697473in}{3.843521in}}%
\pgfusepath{stroke}%
\end{pgfscope}%
\begin{pgfscope}%
\definecolor{textcolor}{rgb}{0.000000,0.000000,0.000000}%
\pgfsetstrokecolor{textcolor}%
\pgfsetfillcolor{textcolor}%
\pgftext[x=4.808584in,y=3.794909in,left,base]{\color{textcolor}\rmfamily\fontsize{10.000000}{12.000000}\selectfont non-algebraic}%
\end{pgfscope}%
\end{pgfpicture}%
\makeatother%
\endgroup%


%% Creator: Matplotlib, PGF backend
%%
%% To include the figure in your LaTeX document, write
%%   \input{<filename>.pgf}
%%
%% Make sure the required packages are loaded in your preamble
%%   \usepackage{pgf}
%%
%% Also ensure that all the required font packages are loaded; for instance,
%% the lmodern package is sometimes necessary when using math font.
%%   \usepackage{lmodern}
%%
%% Figures using additional raster images can only be included by \input if
%% they are in the same directory as the main LaTeX file. For loading figures
%% from other directories you can use the `import` package
%%   \usepackage{import}
%%
%% and then include the figures with
%%   \import{<path to file>}{<filename>.pgf}
%%
%% Matplotlib used the following preamble
%%   
%%   \usepackage{fontspec}
%%   \setmainfont{Charter.ttc}[Path=\detokenize{/System/Library/Fonts/Supplemental/}]
%%   \setsansfont{DejaVuSans.ttf}[Path=\detokenize{/opt/homebrew/lib/python3.10/site-packages/matplotlib/mpl-data/fonts/ttf/}]
%%   \setmonofont{DejaVuSansMono.ttf}[Path=\detokenize{/opt/homebrew/lib/python3.10/site-packages/matplotlib/mpl-data/fonts/ttf/}]
%%   \makeatletter\@ifpackageloaded{underscore}{}{\usepackage[strings]{underscore}}\makeatother
%%
\begingroup%
\makeatletter%
\begin{pgfpicture}%
\pgfpathrectangle{\pgfpointorigin}{\pgfqpoint{6.400000in}{4.800000in}}%
\pgfusepath{use as bounding box, clip}%
\begin{pgfscope}%
\pgfsetbuttcap%
\pgfsetmiterjoin%
\definecolor{currentfill}{rgb}{1.000000,1.000000,1.000000}%
\pgfsetfillcolor{currentfill}%
\pgfsetlinewidth{0.000000pt}%
\definecolor{currentstroke}{rgb}{1.000000,1.000000,1.000000}%
\pgfsetstrokecolor{currentstroke}%
\pgfsetdash{}{0pt}%
\pgfpathmoveto{\pgfqpoint{0.000000in}{0.000000in}}%
\pgfpathlineto{\pgfqpoint{6.400000in}{0.000000in}}%
\pgfpathlineto{\pgfqpoint{6.400000in}{4.800000in}}%
\pgfpathlineto{\pgfqpoint{0.000000in}{4.800000in}}%
\pgfpathlineto{\pgfqpoint{0.000000in}{0.000000in}}%
\pgfpathclose%
\pgfusepath{fill}%
\end{pgfscope}%
\begin{pgfscope}%
\pgfsetbuttcap%
\pgfsetmiterjoin%
\definecolor{currentfill}{rgb}{1.000000,1.000000,1.000000}%
\pgfsetfillcolor{currentfill}%
\pgfsetlinewidth{0.000000pt}%
\definecolor{currentstroke}{rgb}{0.000000,0.000000,0.000000}%
\pgfsetstrokecolor{currentstroke}%
\pgfsetstrokeopacity{0.000000}%
\pgfsetdash{}{0pt}%
\pgfpathmoveto{\pgfqpoint{0.800000in}{0.528000in}}%
\pgfpathlineto{\pgfqpoint{5.760000in}{0.528000in}}%
\pgfpathlineto{\pgfqpoint{5.760000in}{4.224000in}}%
\pgfpathlineto{\pgfqpoint{0.800000in}{4.224000in}}%
\pgfpathlineto{\pgfqpoint{0.800000in}{0.528000in}}%
\pgfpathclose%
\pgfusepath{fill}%
\end{pgfscope}%
\begin{pgfscope}%
\pgfpathrectangle{\pgfqpoint{0.800000in}{0.528000in}}{\pgfqpoint{4.960000in}{3.696000in}}%
\pgfusepath{clip}%
\pgfsetrectcap%
\pgfsetroundjoin%
\pgfsetlinewidth{0.803000pt}%
\definecolor{currentstroke}{rgb}{0.690196,0.690196,0.690196}%
\pgfsetstrokecolor{currentstroke}%
\pgfsetdash{}{0pt}%
\pgfpathmoveto{\pgfqpoint{1.025455in}{0.528000in}}%
\pgfpathlineto{\pgfqpoint{1.025455in}{4.224000in}}%
\pgfusepath{stroke}%
\end{pgfscope}%
\begin{pgfscope}%
\pgfsetbuttcap%
\pgfsetroundjoin%
\definecolor{currentfill}{rgb}{0.000000,0.000000,0.000000}%
\pgfsetfillcolor{currentfill}%
\pgfsetlinewidth{0.803000pt}%
\definecolor{currentstroke}{rgb}{0.000000,0.000000,0.000000}%
\pgfsetstrokecolor{currentstroke}%
\pgfsetdash{}{0pt}%
\pgfsys@defobject{currentmarker}{\pgfqpoint{0.000000in}{-0.048611in}}{\pgfqpoint{0.000000in}{0.000000in}}{%
\pgfpathmoveto{\pgfqpoint{0.000000in}{0.000000in}}%
\pgfpathlineto{\pgfqpoint{0.000000in}{-0.048611in}}%
\pgfusepath{stroke,fill}%
}%
\begin{pgfscope}%
\pgfsys@transformshift{1.025455in}{0.528000in}%
\pgfsys@useobject{currentmarker}{}%
\end{pgfscope}%
\end{pgfscope}%
\begin{pgfscope}%
\definecolor{textcolor}{rgb}{0.000000,0.000000,0.000000}%
\pgfsetstrokecolor{textcolor}%
\pgfsetfillcolor{textcolor}%
\pgftext[x=1.025455in,y=0.430778in,,top]{\color{textcolor}\rmfamily\fontsize{10.000000}{12.000000}\selectfont \(\displaystyle {\ensuremath{-}1}\)}%
\end{pgfscope}%
\begin{pgfscope}%
\pgfpathrectangle{\pgfqpoint{0.800000in}{0.528000in}}{\pgfqpoint{4.960000in}{3.696000in}}%
\pgfusepath{clip}%
\pgfsetrectcap%
\pgfsetroundjoin%
\pgfsetlinewidth{0.803000pt}%
\definecolor{currentstroke}{rgb}{0.690196,0.690196,0.690196}%
\pgfsetstrokecolor{currentstroke}%
\pgfsetdash{}{0pt}%
\pgfpathmoveto{\pgfqpoint{1.777095in}{0.528000in}}%
\pgfpathlineto{\pgfqpoint{1.777095in}{4.224000in}}%
\pgfusepath{stroke}%
\end{pgfscope}%
\begin{pgfscope}%
\pgfsetbuttcap%
\pgfsetroundjoin%
\definecolor{currentfill}{rgb}{0.000000,0.000000,0.000000}%
\pgfsetfillcolor{currentfill}%
\pgfsetlinewidth{0.803000pt}%
\definecolor{currentstroke}{rgb}{0.000000,0.000000,0.000000}%
\pgfsetstrokecolor{currentstroke}%
\pgfsetdash{}{0pt}%
\pgfsys@defobject{currentmarker}{\pgfqpoint{0.000000in}{-0.048611in}}{\pgfqpoint{0.000000in}{0.000000in}}{%
\pgfpathmoveto{\pgfqpoint{0.000000in}{0.000000in}}%
\pgfpathlineto{\pgfqpoint{0.000000in}{-0.048611in}}%
\pgfusepath{stroke,fill}%
}%
\begin{pgfscope}%
\pgfsys@transformshift{1.777095in}{0.528000in}%
\pgfsys@useobject{currentmarker}{}%
\end{pgfscope}%
\end{pgfscope}%
\begin{pgfscope}%
\definecolor{textcolor}{rgb}{0.000000,0.000000,0.000000}%
\pgfsetstrokecolor{textcolor}%
\pgfsetfillcolor{textcolor}%
\pgftext[x=1.777095in,y=0.430778in,,top]{\color{textcolor}\rmfamily\fontsize{10.000000}{12.000000}\selectfont \(\displaystyle {0}\)}%
\end{pgfscope}%
\begin{pgfscope}%
\pgfpathrectangle{\pgfqpoint{0.800000in}{0.528000in}}{\pgfqpoint{4.960000in}{3.696000in}}%
\pgfusepath{clip}%
\pgfsetrectcap%
\pgfsetroundjoin%
\pgfsetlinewidth{0.803000pt}%
\definecolor{currentstroke}{rgb}{0.690196,0.690196,0.690196}%
\pgfsetstrokecolor{currentstroke}%
\pgfsetdash{}{0pt}%
\pgfpathmoveto{\pgfqpoint{2.528735in}{0.528000in}}%
\pgfpathlineto{\pgfqpoint{2.528735in}{4.224000in}}%
\pgfusepath{stroke}%
\end{pgfscope}%
\begin{pgfscope}%
\pgfsetbuttcap%
\pgfsetroundjoin%
\definecolor{currentfill}{rgb}{0.000000,0.000000,0.000000}%
\pgfsetfillcolor{currentfill}%
\pgfsetlinewidth{0.803000pt}%
\definecolor{currentstroke}{rgb}{0.000000,0.000000,0.000000}%
\pgfsetstrokecolor{currentstroke}%
\pgfsetdash{}{0pt}%
\pgfsys@defobject{currentmarker}{\pgfqpoint{0.000000in}{-0.048611in}}{\pgfqpoint{0.000000in}{0.000000in}}{%
\pgfpathmoveto{\pgfqpoint{0.000000in}{0.000000in}}%
\pgfpathlineto{\pgfqpoint{0.000000in}{-0.048611in}}%
\pgfusepath{stroke,fill}%
}%
\begin{pgfscope}%
\pgfsys@transformshift{2.528735in}{0.528000in}%
\pgfsys@useobject{currentmarker}{}%
\end{pgfscope}%
\end{pgfscope}%
\begin{pgfscope}%
\definecolor{textcolor}{rgb}{0.000000,0.000000,0.000000}%
\pgfsetstrokecolor{textcolor}%
\pgfsetfillcolor{textcolor}%
\pgftext[x=2.528735in,y=0.430778in,,top]{\color{textcolor}\rmfamily\fontsize{10.000000}{12.000000}\selectfont \(\displaystyle {1}\)}%
\end{pgfscope}%
\begin{pgfscope}%
\pgfpathrectangle{\pgfqpoint{0.800000in}{0.528000in}}{\pgfqpoint{4.960000in}{3.696000in}}%
\pgfusepath{clip}%
\pgfsetrectcap%
\pgfsetroundjoin%
\pgfsetlinewidth{0.803000pt}%
\definecolor{currentstroke}{rgb}{0.690196,0.690196,0.690196}%
\pgfsetstrokecolor{currentstroke}%
\pgfsetdash{}{0pt}%
\pgfpathmoveto{\pgfqpoint{3.280376in}{0.528000in}}%
\pgfpathlineto{\pgfqpoint{3.280376in}{4.224000in}}%
\pgfusepath{stroke}%
\end{pgfscope}%
\begin{pgfscope}%
\pgfsetbuttcap%
\pgfsetroundjoin%
\definecolor{currentfill}{rgb}{0.000000,0.000000,0.000000}%
\pgfsetfillcolor{currentfill}%
\pgfsetlinewidth{0.803000pt}%
\definecolor{currentstroke}{rgb}{0.000000,0.000000,0.000000}%
\pgfsetstrokecolor{currentstroke}%
\pgfsetdash{}{0pt}%
\pgfsys@defobject{currentmarker}{\pgfqpoint{0.000000in}{-0.048611in}}{\pgfqpoint{0.000000in}{0.000000in}}{%
\pgfpathmoveto{\pgfqpoint{0.000000in}{0.000000in}}%
\pgfpathlineto{\pgfqpoint{0.000000in}{-0.048611in}}%
\pgfusepath{stroke,fill}%
}%
\begin{pgfscope}%
\pgfsys@transformshift{3.280376in}{0.528000in}%
\pgfsys@useobject{currentmarker}{}%
\end{pgfscope}%
\end{pgfscope}%
\begin{pgfscope}%
\definecolor{textcolor}{rgb}{0.000000,0.000000,0.000000}%
\pgfsetstrokecolor{textcolor}%
\pgfsetfillcolor{textcolor}%
\pgftext[x=3.280376in,y=0.430778in,,top]{\color{textcolor}\rmfamily\fontsize{10.000000}{12.000000}\selectfont \(\displaystyle {2}\)}%
\end{pgfscope}%
\begin{pgfscope}%
\pgfpathrectangle{\pgfqpoint{0.800000in}{0.528000in}}{\pgfqpoint{4.960000in}{3.696000in}}%
\pgfusepath{clip}%
\pgfsetrectcap%
\pgfsetroundjoin%
\pgfsetlinewidth{0.803000pt}%
\definecolor{currentstroke}{rgb}{0.690196,0.690196,0.690196}%
\pgfsetstrokecolor{currentstroke}%
\pgfsetdash{}{0pt}%
\pgfpathmoveto{\pgfqpoint{4.032016in}{0.528000in}}%
\pgfpathlineto{\pgfqpoint{4.032016in}{4.224000in}}%
\pgfusepath{stroke}%
\end{pgfscope}%
\begin{pgfscope}%
\pgfsetbuttcap%
\pgfsetroundjoin%
\definecolor{currentfill}{rgb}{0.000000,0.000000,0.000000}%
\pgfsetfillcolor{currentfill}%
\pgfsetlinewidth{0.803000pt}%
\definecolor{currentstroke}{rgb}{0.000000,0.000000,0.000000}%
\pgfsetstrokecolor{currentstroke}%
\pgfsetdash{}{0pt}%
\pgfsys@defobject{currentmarker}{\pgfqpoint{0.000000in}{-0.048611in}}{\pgfqpoint{0.000000in}{0.000000in}}{%
\pgfpathmoveto{\pgfqpoint{0.000000in}{0.000000in}}%
\pgfpathlineto{\pgfqpoint{0.000000in}{-0.048611in}}%
\pgfusepath{stroke,fill}%
}%
\begin{pgfscope}%
\pgfsys@transformshift{4.032016in}{0.528000in}%
\pgfsys@useobject{currentmarker}{}%
\end{pgfscope}%
\end{pgfscope}%
\begin{pgfscope}%
\definecolor{textcolor}{rgb}{0.000000,0.000000,0.000000}%
\pgfsetstrokecolor{textcolor}%
\pgfsetfillcolor{textcolor}%
\pgftext[x=4.032016in,y=0.430778in,,top]{\color{textcolor}\rmfamily\fontsize{10.000000}{12.000000}\selectfont \(\displaystyle {3}\)}%
\end{pgfscope}%
\begin{pgfscope}%
\pgfpathrectangle{\pgfqpoint{0.800000in}{0.528000in}}{\pgfqpoint{4.960000in}{3.696000in}}%
\pgfusepath{clip}%
\pgfsetrectcap%
\pgfsetroundjoin%
\pgfsetlinewidth{0.803000pt}%
\definecolor{currentstroke}{rgb}{0.690196,0.690196,0.690196}%
\pgfsetstrokecolor{currentstroke}%
\pgfsetdash{}{0pt}%
\pgfpathmoveto{\pgfqpoint{4.783657in}{0.528000in}}%
\pgfpathlineto{\pgfqpoint{4.783657in}{4.224000in}}%
\pgfusepath{stroke}%
\end{pgfscope}%
\begin{pgfscope}%
\pgfsetbuttcap%
\pgfsetroundjoin%
\definecolor{currentfill}{rgb}{0.000000,0.000000,0.000000}%
\pgfsetfillcolor{currentfill}%
\pgfsetlinewidth{0.803000pt}%
\definecolor{currentstroke}{rgb}{0.000000,0.000000,0.000000}%
\pgfsetstrokecolor{currentstroke}%
\pgfsetdash{}{0pt}%
\pgfsys@defobject{currentmarker}{\pgfqpoint{0.000000in}{-0.048611in}}{\pgfqpoint{0.000000in}{0.000000in}}{%
\pgfpathmoveto{\pgfqpoint{0.000000in}{0.000000in}}%
\pgfpathlineto{\pgfqpoint{0.000000in}{-0.048611in}}%
\pgfusepath{stroke,fill}%
}%
\begin{pgfscope}%
\pgfsys@transformshift{4.783657in}{0.528000in}%
\pgfsys@useobject{currentmarker}{}%
\end{pgfscope}%
\end{pgfscope}%
\begin{pgfscope}%
\definecolor{textcolor}{rgb}{0.000000,0.000000,0.000000}%
\pgfsetstrokecolor{textcolor}%
\pgfsetfillcolor{textcolor}%
\pgftext[x=4.783657in,y=0.430778in,,top]{\color{textcolor}\rmfamily\fontsize{10.000000}{12.000000}\selectfont \(\displaystyle {4}\)}%
\end{pgfscope}%
\begin{pgfscope}%
\pgfpathrectangle{\pgfqpoint{0.800000in}{0.528000in}}{\pgfqpoint{4.960000in}{3.696000in}}%
\pgfusepath{clip}%
\pgfsetrectcap%
\pgfsetroundjoin%
\pgfsetlinewidth{0.803000pt}%
\definecolor{currentstroke}{rgb}{0.690196,0.690196,0.690196}%
\pgfsetstrokecolor{currentstroke}%
\pgfsetdash{}{0pt}%
\pgfpathmoveto{\pgfqpoint{5.535297in}{0.528000in}}%
\pgfpathlineto{\pgfqpoint{5.535297in}{4.224000in}}%
\pgfusepath{stroke}%
\end{pgfscope}%
\begin{pgfscope}%
\pgfsetbuttcap%
\pgfsetroundjoin%
\definecolor{currentfill}{rgb}{0.000000,0.000000,0.000000}%
\pgfsetfillcolor{currentfill}%
\pgfsetlinewidth{0.803000pt}%
\definecolor{currentstroke}{rgb}{0.000000,0.000000,0.000000}%
\pgfsetstrokecolor{currentstroke}%
\pgfsetdash{}{0pt}%
\pgfsys@defobject{currentmarker}{\pgfqpoint{0.000000in}{-0.048611in}}{\pgfqpoint{0.000000in}{0.000000in}}{%
\pgfpathmoveto{\pgfqpoint{0.000000in}{0.000000in}}%
\pgfpathlineto{\pgfqpoint{0.000000in}{-0.048611in}}%
\pgfusepath{stroke,fill}%
}%
\begin{pgfscope}%
\pgfsys@transformshift{5.535297in}{0.528000in}%
\pgfsys@useobject{currentmarker}{}%
\end{pgfscope}%
\end{pgfscope}%
\begin{pgfscope}%
\definecolor{textcolor}{rgb}{0.000000,0.000000,0.000000}%
\pgfsetstrokecolor{textcolor}%
\pgfsetfillcolor{textcolor}%
\pgftext[x=5.535297in,y=0.430778in,,top]{\color{textcolor}\rmfamily\fontsize{10.000000}{12.000000}\selectfont \(\displaystyle {5}\)}%
\end{pgfscope}%
\begin{pgfscope}%
\definecolor{textcolor}{rgb}{0.000000,0.000000,0.000000}%
\pgfsetstrokecolor{textcolor}%
\pgfsetfillcolor{textcolor}%
\pgftext[x=3.280000in,y=0.242776in,,top]{\color{textcolor}\rmfamily\fontsize{10.000000}{12.000000}\selectfont time in s}%
\end{pgfscope}%
\begin{pgfscope}%
\pgfpathrectangle{\pgfqpoint{0.800000in}{0.528000in}}{\pgfqpoint{4.960000in}{3.696000in}}%
\pgfusepath{clip}%
\pgfsetrectcap%
\pgfsetroundjoin%
\pgfsetlinewidth{0.803000pt}%
\definecolor{currentstroke}{rgb}{0.690196,0.690196,0.690196}%
\pgfsetstrokecolor{currentstroke}%
\pgfsetdash{}{0pt}%
\pgfpathmoveto{\pgfqpoint{0.800000in}{0.696000in}}%
\pgfpathlineto{\pgfqpoint{5.760000in}{0.696000in}}%
\pgfusepath{stroke}%
\end{pgfscope}%
\begin{pgfscope}%
\pgfsetbuttcap%
\pgfsetroundjoin%
\definecolor{currentfill}{rgb}{0.000000,0.000000,0.000000}%
\pgfsetfillcolor{currentfill}%
\pgfsetlinewidth{0.803000pt}%
\definecolor{currentstroke}{rgb}{0.000000,0.000000,0.000000}%
\pgfsetstrokecolor{currentstroke}%
\pgfsetdash{}{0pt}%
\pgfsys@defobject{currentmarker}{\pgfqpoint{-0.048611in}{0.000000in}}{\pgfqpoint{-0.000000in}{0.000000in}}{%
\pgfpathmoveto{\pgfqpoint{-0.000000in}{0.000000in}}%
\pgfpathlineto{\pgfqpoint{-0.048611in}{0.000000in}}%
\pgfusepath{stroke,fill}%
}%
\begin{pgfscope}%
\pgfsys@transformshift{0.800000in}{0.696000in}%
\pgfsys@useobject{currentmarker}{}%
\end{pgfscope}%
\end{pgfscope}%
\begin{pgfscope}%
\definecolor{textcolor}{rgb}{0.000000,0.000000,0.000000}%
\pgfsetstrokecolor{textcolor}%
\pgfsetfillcolor{textcolor}%
\pgftext[x=0.525308in, y=0.644900in, left, base]{\color{textcolor}\rmfamily\fontsize{10.000000}{12.000000}\selectfont \(\displaystyle {0.0}\)}%
\end{pgfscope}%
\begin{pgfscope}%
\pgfpathrectangle{\pgfqpoint{0.800000in}{0.528000in}}{\pgfqpoint{4.960000in}{3.696000in}}%
\pgfusepath{clip}%
\pgfsetrectcap%
\pgfsetroundjoin%
\pgfsetlinewidth{0.803000pt}%
\definecolor{currentstroke}{rgb}{0.690196,0.690196,0.690196}%
\pgfsetstrokecolor{currentstroke}%
\pgfsetdash{}{0pt}%
\pgfpathmoveto{\pgfqpoint{0.800000in}{1.256000in}}%
\pgfpathlineto{\pgfqpoint{5.760000in}{1.256000in}}%
\pgfusepath{stroke}%
\end{pgfscope}%
\begin{pgfscope}%
\pgfsetbuttcap%
\pgfsetroundjoin%
\definecolor{currentfill}{rgb}{0.000000,0.000000,0.000000}%
\pgfsetfillcolor{currentfill}%
\pgfsetlinewidth{0.803000pt}%
\definecolor{currentstroke}{rgb}{0.000000,0.000000,0.000000}%
\pgfsetstrokecolor{currentstroke}%
\pgfsetdash{}{0pt}%
\pgfsys@defobject{currentmarker}{\pgfqpoint{-0.048611in}{0.000000in}}{\pgfqpoint{-0.000000in}{0.000000in}}{%
\pgfpathmoveto{\pgfqpoint{-0.000000in}{0.000000in}}%
\pgfpathlineto{\pgfqpoint{-0.048611in}{0.000000in}}%
\pgfusepath{stroke,fill}%
}%
\begin{pgfscope}%
\pgfsys@transformshift{0.800000in}{1.256000in}%
\pgfsys@useobject{currentmarker}{}%
\end{pgfscope}%
\end{pgfscope}%
\begin{pgfscope}%
\definecolor{textcolor}{rgb}{0.000000,0.000000,0.000000}%
\pgfsetstrokecolor{textcolor}%
\pgfsetfillcolor{textcolor}%
\pgftext[x=0.525308in, y=1.204900in, left, base]{\color{textcolor}\rmfamily\fontsize{10.000000}{12.000000}\selectfont \(\displaystyle {0.2}\)}%
\end{pgfscope}%
\begin{pgfscope}%
\pgfpathrectangle{\pgfqpoint{0.800000in}{0.528000in}}{\pgfqpoint{4.960000in}{3.696000in}}%
\pgfusepath{clip}%
\pgfsetrectcap%
\pgfsetroundjoin%
\pgfsetlinewidth{0.803000pt}%
\definecolor{currentstroke}{rgb}{0.690196,0.690196,0.690196}%
\pgfsetstrokecolor{currentstroke}%
\pgfsetdash{}{0pt}%
\pgfpathmoveto{\pgfqpoint{0.800000in}{1.816000in}}%
\pgfpathlineto{\pgfqpoint{5.760000in}{1.816000in}}%
\pgfusepath{stroke}%
\end{pgfscope}%
\begin{pgfscope}%
\pgfsetbuttcap%
\pgfsetroundjoin%
\definecolor{currentfill}{rgb}{0.000000,0.000000,0.000000}%
\pgfsetfillcolor{currentfill}%
\pgfsetlinewidth{0.803000pt}%
\definecolor{currentstroke}{rgb}{0.000000,0.000000,0.000000}%
\pgfsetstrokecolor{currentstroke}%
\pgfsetdash{}{0pt}%
\pgfsys@defobject{currentmarker}{\pgfqpoint{-0.048611in}{0.000000in}}{\pgfqpoint{-0.000000in}{0.000000in}}{%
\pgfpathmoveto{\pgfqpoint{-0.000000in}{0.000000in}}%
\pgfpathlineto{\pgfqpoint{-0.048611in}{0.000000in}}%
\pgfusepath{stroke,fill}%
}%
\begin{pgfscope}%
\pgfsys@transformshift{0.800000in}{1.816000in}%
\pgfsys@useobject{currentmarker}{}%
\end{pgfscope}%
\end{pgfscope}%
\begin{pgfscope}%
\definecolor{textcolor}{rgb}{0.000000,0.000000,0.000000}%
\pgfsetstrokecolor{textcolor}%
\pgfsetfillcolor{textcolor}%
\pgftext[x=0.525308in, y=1.764900in, left, base]{\color{textcolor}\rmfamily\fontsize{10.000000}{12.000000}\selectfont \(\displaystyle {0.4}\)}%
\end{pgfscope}%
\begin{pgfscope}%
\pgfpathrectangle{\pgfqpoint{0.800000in}{0.528000in}}{\pgfqpoint{4.960000in}{3.696000in}}%
\pgfusepath{clip}%
\pgfsetrectcap%
\pgfsetroundjoin%
\pgfsetlinewidth{0.803000pt}%
\definecolor{currentstroke}{rgb}{0.690196,0.690196,0.690196}%
\pgfsetstrokecolor{currentstroke}%
\pgfsetdash{}{0pt}%
\pgfpathmoveto{\pgfqpoint{0.800000in}{2.376000in}}%
\pgfpathlineto{\pgfqpoint{5.760000in}{2.376000in}}%
\pgfusepath{stroke}%
\end{pgfscope}%
\begin{pgfscope}%
\pgfsetbuttcap%
\pgfsetroundjoin%
\definecolor{currentfill}{rgb}{0.000000,0.000000,0.000000}%
\pgfsetfillcolor{currentfill}%
\pgfsetlinewidth{0.803000pt}%
\definecolor{currentstroke}{rgb}{0.000000,0.000000,0.000000}%
\pgfsetstrokecolor{currentstroke}%
\pgfsetdash{}{0pt}%
\pgfsys@defobject{currentmarker}{\pgfqpoint{-0.048611in}{0.000000in}}{\pgfqpoint{-0.000000in}{0.000000in}}{%
\pgfpathmoveto{\pgfqpoint{-0.000000in}{0.000000in}}%
\pgfpathlineto{\pgfqpoint{-0.048611in}{0.000000in}}%
\pgfusepath{stroke,fill}%
}%
\begin{pgfscope}%
\pgfsys@transformshift{0.800000in}{2.376000in}%
\pgfsys@useobject{currentmarker}{}%
\end{pgfscope}%
\end{pgfscope}%
\begin{pgfscope}%
\definecolor{textcolor}{rgb}{0.000000,0.000000,0.000000}%
\pgfsetstrokecolor{textcolor}%
\pgfsetfillcolor{textcolor}%
\pgftext[x=0.525308in, y=2.324900in, left, base]{\color{textcolor}\rmfamily\fontsize{10.000000}{12.000000}\selectfont \(\displaystyle {0.6}\)}%
\end{pgfscope}%
\begin{pgfscope}%
\pgfpathrectangle{\pgfqpoint{0.800000in}{0.528000in}}{\pgfqpoint{4.960000in}{3.696000in}}%
\pgfusepath{clip}%
\pgfsetrectcap%
\pgfsetroundjoin%
\pgfsetlinewidth{0.803000pt}%
\definecolor{currentstroke}{rgb}{0.690196,0.690196,0.690196}%
\pgfsetstrokecolor{currentstroke}%
\pgfsetdash{}{0pt}%
\pgfpathmoveto{\pgfqpoint{0.800000in}{2.936000in}}%
\pgfpathlineto{\pgfqpoint{5.760000in}{2.936000in}}%
\pgfusepath{stroke}%
\end{pgfscope}%
\begin{pgfscope}%
\pgfsetbuttcap%
\pgfsetroundjoin%
\definecolor{currentfill}{rgb}{0.000000,0.000000,0.000000}%
\pgfsetfillcolor{currentfill}%
\pgfsetlinewidth{0.803000pt}%
\definecolor{currentstroke}{rgb}{0.000000,0.000000,0.000000}%
\pgfsetstrokecolor{currentstroke}%
\pgfsetdash{}{0pt}%
\pgfsys@defobject{currentmarker}{\pgfqpoint{-0.048611in}{0.000000in}}{\pgfqpoint{-0.000000in}{0.000000in}}{%
\pgfpathmoveto{\pgfqpoint{-0.000000in}{0.000000in}}%
\pgfpathlineto{\pgfqpoint{-0.048611in}{0.000000in}}%
\pgfusepath{stroke,fill}%
}%
\begin{pgfscope}%
\pgfsys@transformshift{0.800000in}{2.936000in}%
\pgfsys@useobject{currentmarker}{}%
\end{pgfscope}%
\end{pgfscope}%
\begin{pgfscope}%
\definecolor{textcolor}{rgb}{0.000000,0.000000,0.000000}%
\pgfsetstrokecolor{textcolor}%
\pgfsetfillcolor{textcolor}%
\pgftext[x=0.525308in, y=2.884900in, left, base]{\color{textcolor}\rmfamily\fontsize{10.000000}{12.000000}\selectfont \(\displaystyle {0.8}\)}%
\end{pgfscope}%
\begin{pgfscope}%
\pgfpathrectangle{\pgfqpoint{0.800000in}{0.528000in}}{\pgfqpoint{4.960000in}{3.696000in}}%
\pgfusepath{clip}%
\pgfsetrectcap%
\pgfsetroundjoin%
\pgfsetlinewidth{0.803000pt}%
\definecolor{currentstroke}{rgb}{0.690196,0.690196,0.690196}%
\pgfsetstrokecolor{currentstroke}%
\pgfsetdash{}{0pt}%
\pgfpathmoveto{\pgfqpoint{0.800000in}{3.496000in}}%
\pgfpathlineto{\pgfqpoint{5.760000in}{3.496000in}}%
\pgfusepath{stroke}%
\end{pgfscope}%
\begin{pgfscope}%
\pgfsetbuttcap%
\pgfsetroundjoin%
\definecolor{currentfill}{rgb}{0.000000,0.000000,0.000000}%
\pgfsetfillcolor{currentfill}%
\pgfsetlinewidth{0.803000pt}%
\definecolor{currentstroke}{rgb}{0.000000,0.000000,0.000000}%
\pgfsetstrokecolor{currentstroke}%
\pgfsetdash{}{0pt}%
\pgfsys@defobject{currentmarker}{\pgfqpoint{-0.048611in}{0.000000in}}{\pgfqpoint{-0.000000in}{0.000000in}}{%
\pgfpathmoveto{\pgfqpoint{-0.000000in}{0.000000in}}%
\pgfpathlineto{\pgfqpoint{-0.048611in}{0.000000in}}%
\pgfusepath{stroke,fill}%
}%
\begin{pgfscope}%
\pgfsys@transformshift{0.800000in}{3.496000in}%
\pgfsys@useobject{currentmarker}{}%
\end{pgfscope}%
\end{pgfscope}%
\begin{pgfscope}%
\definecolor{textcolor}{rgb}{0.000000,0.000000,0.000000}%
\pgfsetstrokecolor{textcolor}%
\pgfsetfillcolor{textcolor}%
\pgftext[x=0.525308in, y=3.444900in, left, base]{\color{textcolor}\rmfamily\fontsize{10.000000}{12.000000}\selectfont \(\displaystyle {1.0}\)}%
\end{pgfscope}%
\begin{pgfscope}%
\pgfpathrectangle{\pgfqpoint{0.800000in}{0.528000in}}{\pgfqpoint{4.960000in}{3.696000in}}%
\pgfusepath{clip}%
\pgfsetrectcap%
\pgfsetroundjoin%
\pgfsetlinewidth{0.803000pt}%
\definecolor{currentstroke}{rgb}{0.690196,0.690196,0.690196}%
\pgfsetstrokecolor{currentstroke}%
\pgfsetdash{}{0pt}%
\pgfpathmoveto{\pgfqpoint{0.800000in}{4.056000in}}%
\pgfpathlineto{\pgfqpoint{5.760000in}{4.056000in}}%
\pgfusepath{stroke}%
\end{pgfscope}%
\begin{pgfscope}%
\pgfsetbuttcap%
\pgfsetroundjoin%
\definecolor{currentfill}{rgb}{0.000000,0.000000,0.000000}%
\pgfsetfillcolor{currentfill}%
\pgfsetlinewidth{0.803000pt}%
\definecolor{currentstroke}{rgb}{0.000000,0.000000,0.000000}%
\pgfsetstrokecolor{currentstroke}%
\pgfsetdash{}{0pt}%
\pgfsys@defobject{currentmarker}{\pgfqpoint{-0.048611in}{0.000000in}}{\pgfqpoint{-0.000000in}{0.000000in}}{%
\pgfpathmoveto{\pgfqpoint{-0.000000in}{0.000000in}}%
\pgfpathlineto{\pgfqpoint{-0.048611in}{0.000000in}}%
\pgfusepath{stroke,fill}%
}%
\begin{pgfscope}%
\pgfsys@transformshift{0.800000in}{4.056000in}%
\pgfsys@useobject{currentmarker}{}%
\end{pgfscope}%
\end{pgfscope}%
\begin{pgfscope}%
\definecolor{textcolor}{rgb}{0.000000,0.000000,0.000000}%
\pgfsetstrokecolor{textcolor}%
\pgfsetfillcolor{textcolor}%
\pgftext[x=0.525308in, y=4.004900in, left, base]{\color{textcolor}\rmfamily\fontsize{10.000000}{12.000000}\selectfont \(\displaystyle {1.2}\)}%
\end{pgfscope}%
\begin{pgfscope}%
\definecolor{textcolor}{rgb}{0.000000,0.000000,0.000000}%
\pgfsetstrokecolor{textcolor}%
\pgfsetfillcolor{textcolor}%
\pgftext[x=0.469752in,y=2.376000in,,bottom,rotate=90.000000]{\color{textcolor}\rmfamily\fontsize{10.000000}{12.000000}\selectfont electrical power in \(\displaystyle \mathrm{p.u.}\)}%
\end{pgfscope}%
\begin{pgfscope}%
\pgfpathrectangle{\pgfqpoint{0.800000in}{0.528000in}}{\pgfqpoint{4.960000in}{3.696000in}}%
\pgfusepath{clip}%
\pgfsetrectcap%
\pgfsetroundjoin%
\pgfsetlinewidth{1.505625pt}%
\definecolor{currentstroke}{rgb}{0.121569,0.466667,0.705882}%
\pgfsetstrokecolor{currentstroke}%
\pgfsetdash{}{0pt}%
\pgfpathmoveto{\pgfqpoint{1.025455in}{3.237068in}}%
\pgfpathlineto{\pgfqpoint{1.065291in}{3.235975in}}%
\pgfpathlineto{\pgfqpoint{1.107383in}{3.232579in}}%
\pgfpathlineto{\pgfqpoint{1.155488in}{3.226399in}}%
\pgfpathlineto{\pgfqpoint{1.229149in}{3.214340in}}%
\pgfpathlineto{\pgfqpoint{1.303562in}{3.202857in}}%
\pgfpathlineto{\pgfqpoint{1.350163in}{3.197920in}}%
\pgfpathlineto{\pgfqpoint{1.391503in}{3.195738in}}%
\pgfpathlineto{\pgfqpoint{1.431340in}{3.195832in}}%
\pgfpathlineto{\pgfqpoint{1.472681in}{3.198184in}}%
\pgfpathlineto{\pgfqpoint{1.518531in}{3.203114in}}%
\pgfpathlineto{\pgfqpoint{1.577910in}{3.211945in}}%
\pgfpathlineto{\pgfqpoint{1.690656in}{3.229072in}}%
\pgfpathlineto{\pgfqpoint{1.738010in}{3.233652in}}%
\pgfpathlineto{\pgfqpoint{1.776343in}{3.235455in}}%
\pgfpathlineto{\pgfqpoint{1.777847in}{0.696000in}}%
\pgfpathlineto{\pgfqpoint{1.865789in}{0.696000in}}%
\pgfpathlineto{\pgfqpoint{1.867292in}{3.781272in}}%
\pgfpathlineto{\pgfqpoint{1.878566in}{3.867567in}}%
\pgfpathlineto{\pgfqpoint{1.889089in}{3.931686in}}%
\pgfpathlineto{\pgfqpoint{1.898861in}{3.977907in}}%
\pgfpathlineto{\pgfqpoint{1.907880in}{4.010049in}}%
\pgfpathlineto{\pgfqpoint{1.916148in}{4.031392in}}%
\pgfpathlineto{\pgfqpoint{1.923665in}{4.044659in}}%
\pgfpathlineto{\pgfqpoint{1.930430in}{4.052064in}}%
\pgfpathlineto{\pgfqpoint{1.936443in}{4.055369in}}%
\pgfpathlineto{\pgfqpoint{1.941704in}{4.055954in}}%
\pgfpathlineto{\pgfqpoint{1.947717in}{4.054226in}}%
\pgfpathlineto{\pgfqpoint{1.954482in}{4.049536in}}%
\pgfpathlineto{\pgfqpoint{1.961999in}{4.041332in}}%
\pgfpathlineto{\pgfqpoint{1.971018in}{4.027955in}}%
\pgfpathlineto{\pgfqpoint{1.982293in}{4.006883in}}%
\pgfpathlineto{\pgfqpoint{1.996574in}{3.975143in}}%
\pgfpathlineto{\pgfqpoint{2.018372in}{3.920537in}}%
\pgfpathlineto{\pgfqpoint{2.064222in}{3.804692in}}%
\pgfpathlineto{\pgfqpoint{2.084516in}{3.760101in}}%
\pgfpathlineto{\pgfqpoint{2.101804in}{3.727303in}}%
\pgfpathlineto{\pgfqpoint{2.117588in}{3.702086in}}%
\pgfpathlineto{\pgfqpoint{2.131869in}{3.683426in}}%
\pgfpathlineto{\pgfqpoint{2.144647in}{3.670204in}}%
\pgfpathlineto{\pgfqpoint{2.156673in}{3.660819in}}%
\pgfpathlineto{\pgfqpoint{2.167948in}{3.654748in}}%
\pgfpathlineto{\pgfqpoint{2.178471in}{3.651476in}}%
\pgfpathlineto{\pgfqpoint{2.188994in}{3.650522in}}%
\pgfpathlineto{\pgfqpoint{2.199517in}{3.651883in}}%
\pgfpathlineto{\pgfqpoint{2.210040in}{3.655552in}}%
\pgfpathlineto{\pgfqpoint{2.221314in}{3.662027in}}%
\pgfpathlineto{\pgfqpoint{2.232589in}{3.671106in}}%
\pgfpathlineto{\pgfqpoint{2.244615in}{3.683616in}}%
\pgfpathlineto{\pgfqpoint{2.258145in}{3.701093in}}%
\pgfpathlineto{\pgfqpoint{2.272426in}{3.723303in}}%
\pgfpathlineto{\pgfqpoint{2.288210in}{3.752077in}}%
\pgfpathlineto{\pgfqpoint{2.306250in}{3.789846in}}%
\pgfpathlineto{\pgfqpoint{2.328047in}{3.841107in}}%
\pgfpathlineto{\pgfqpoint{2.397198in}{4.008926in}}%
\pgfpathlineto{\pgfqpoint{2.409225in}{4.030420in}}%
\pgfpathlineto{\pgfqpoint{2.418996in}{4.043841in}}%
\pgfpathlineto{\pgfqpoint{2.427264in}{4.051660in}}%
\pgfpathlineto{\pgfqpoint{2.434029in}{4.055220in}}%
\pgfpathlineto{\pgfqpoint{2.440042in}{4.055951in}}%
\pgfpathlineto{\pgfqpoint{2.445303in}{4.054515in}}%
\pgfpathlineto{\pgfqpoint{2.450565in}{4.050971in}}%
\pgfpathlineto{\pgfqpoint{2.456578in}{4.044123in}}%
\pgfpathlineto{\pgfqpoint{2.463343in}{4.032556in}}%
\pgfpathlineto{\pgfqpoint{2.470859in}{4.014493in}}%
\pgfpathlineto{\pgfqpoint{2.478375in}{3.990487in}}%
\pgfpathlineto{\pgfqpoint{2.486644in}{3.956683in}}%
\pgfpathlineto{\pgfqpoint{2.495663in}{3.910347in}}%
\pgfpathlineto{\pgfqpoint{2.505435in}{3.848330in}}%
\pgfpathlineto{\pgfqpoint{2.515958in}{3.767142in}}%
\pgfpathlineto{\pgfqpoint{2.527984in}{3.655566in}}%
\pgfpathlineto{\pgfqpoint{2.540762in}{3.515100in}}%
\pgfpathlineto{\pgfqpoint{2.555043in}{3.332630in}}%
\pgfpathlineto{\pgfqpoint{2.571579in}{3.091690in}}%
\pgfpathlineto{\pgfqpoint{2.593376in}{2.738091in}}%
\pgfpathlineto{\pgfqpoint{2.648998in}{1.818014in}}%
\pgfpathlineto{\pgfqpoint{2.665534in}{1.590203in}}%
\pgfpathlineto{\pgfqpoint{2.679063in}{1.433225in}}%
\pgfpathlineto{\pgfqpoint{2.690338in}{1.325995in}}%
\pgfpathlineto{\pgfqpoint{2.700109in}{1.252004in}}%
\pgfpathlineto{\pgfqpoint{2.709129in}{1.200177in}}%
\pgfpathlineto{\pgfqpoint{2.716646in}{1.169469in}}%
\pgfpathlineto{\pgfqpoint{2.722659in}{1.153229in}}%
\pgfpathlineto{\pgfqpoint{2.727920in}{1.145150in}}%
\pgfpathlineto{\pgfqpoint{2.731678in}{1.142896in}}%
\pgfpathlineto{\pgfqpoint{2.734685in}{1.143206in}}%
\pgfpathlineto{\pgfqpoint{2.738443in}{1.146232in}}%
\pgfpathlineto{\pgfqpoint{2.742953in}{1.153723in}}%
\pgfpathlineto{\pgfqpoint{2.748214in}{1.167753in}}%
\pgfpathlineto{\pgfqpoint{2.754228in}{1.190698in}}%
\pgfpathlineto{\pgfqpoint{2.761744in}{1.229571in}}%
\pgfpathlineto{\pgfqpoint{2.770012in}{1.285067in}}%
\pgfpathlineto{\pgfqpoint{2.779783in}{1.367142in}}%
\pgfpathlineto{\pgfqpoint{2.791058in}{1.482631in}}%
\pgfpathlineto{\pgfqpoint{2.803836in}{1.637792in}}%
\pgfpathlineto{\pgfqpoint{2.818869in}{1.848219in}}%
\pgfpathlineto{\pgfqpoint{2.838411in}{2.154735in}}%
\pgfpathlineto{\pgfqpoint{2.904556in}{3.220103in}}%
\pgfpathlineto{\pgfqpoint{2.921092in}{3.437953in}}%
\pgfpathlineto{\pgfqpoint{2.935373in}{3.599264in}}%
\pgfpathlineto{\pgfqpoint{2.948902in}{3.727633in}}%
\pgfpathlineto{\pgfqpoint{2.960929in}{3.821705in}}%
\pgfpathlineto{\pgfqpoint{2.972203in}{3.893358in}}%
\pgfpathlineto{\pgfqpoint{2.982726in}{3.946641in}}%
\pgfpathlineto{\pgfqpoint{2.992498in}{3.985260in}}%
\pgfpathlineto{\pgfqpoint{3.001517in}{4.012443in}}%
\pgfpathlineto{\pgfqpoint{3.009785in}{4.030907in}}%
\pgfpathlineto{\pgfqpoint{3.017302in}{4.042871in}}%
\pgfpathlineto{\pgfqpoint{3.024818in}{4.050730in}}%
\pgfpathlineto{\pgfqpoint{3.031583in}{4.054674in}}%
\pgfpathlineto{\pgfqpoint{3.038348in}{4.055997in}}%
\pgfpathlineto{\pgfqpoint{3.045112in}{4.055017in}}%
\pgfpathlineto{\pgfqpoint{3.052629in}{4.051600in}}%
\pgfpathlineto{\pgfqpoint{3.061648in}{4.044805in}}%
\pgfpathlineto{\pgfqpoint{3.072171in}{4.033968in}}%
\pgfpathlineto{\pgfqpoint{3.086453in}{4.015809in}}%
\pgfpathlineto{\pgfqpoint{3.114263in}{3.975777in}}%
\pgfpathlineto{\pgfqpoint{3.137564in}{3.944088in}}%
\pgfpathlineto{\pgfqpoint{3.153349in}{3.926122in}}%
\pgfpathlineto{\pgfqpoint{3.166878in}{3.913853in}}%
\pgfpathlineto{\pgfqpoint{3.178904in}{3.905742in}}%
\pgfpathlineto{\pgfqpoint{3.190179in}{3.900718in}}%
\pgfpathlineto{\pgfqpoint{3.200702in}{3.898376in}}%
\pgfpathlineto{\pgfqpoint{3.211225in}{3.898341in}}%
\pgfpathlineto{\pgfqpoint{3.221748in}{3.900609in}}%
\pgfpathlineto{\pgfqpoint{3.232271in}{3.905134in}}%
\pgfpathlineto{\pgfqpoint{3.243545in}{3.912386in}}%
\pgfpathlineto{\pgfqpoint{3.256323in}{3.923402in}}%
\pgfpathlineto{\pgfqpoint{3.270604in}{3.938820in}}%
\pgfpathlineto{\pgfqpoint{3.287892in}{3.960952in}}%
\pgfpathlineto{\pgfqpoint{3.323219in}{4.011132in}}%
\pgfpathlineto{\pgfqpoint{3.341259in}{4.034108in}}%
\pgfpathlineto{\pgfqpoint{3.353285in}{4.046058in}}%
\pgfpathlineto{\pgfqpoint{3.363056in}{4.052737in}}%
\pgfpathlineto{\pgfqpoint{3.371324in}{4.055650in}}%
\pgfpathlineto{\pgfqpoint{3.378089in}{4.055794in}}%
\pgfpathlineto{\pgfqpoint{3.384854in}{4.053612in}}%
\pgfpathlineto{\pgfqpoint{3.391619in}{4.048806in}}%
\pgfpathlineto{\pgfqpoint{3.398383in}{4.041068in}}%
\pgfpathlineto{\pgfqpoint{3.405900in}{4.028655in}}%
\pgfpathlineto{\pgfqpoint{3.413416in}{4.011814in}}%
\pgfpathlineto{\pgfqpoint{3.421684in}{3.987671in}}%
\pgfpathlineto{\pgfqpoint{3.430704in}{3.954008in}}%
\pgfpathlineto{\pgfqpoint{3.440475in}{3.908183in}}%
\pgfpathlineto{\pgfqpoint{3.450998in}{3.847146in}}%
\pgfpathlineto{\pgfqpoint{3.462273in}{3.767523in}}%
\pgfpathlineto{\pgfqpoint{3.474299in}{3.665785in}}%
\pgfpathlineto{\pgfqpoint{3.487077in}{3.538569in}}%
\pgfpathlineto{\pgfqpoint{3.501358in}{3.373911in}}%
\pgfpathlineto{\pgfqpoint{3.517894in}{3.156618in}}%
\pgfpathlineto{\pgfqpoint{3.538940in}{2.848093in}}%
\pgfpathlineto{\pgfqpoint{3.603581in}{1.875668in}}%
\pgfpathlineto{\pgfqpoint{3.619366in}{1.683228in}}%
\pgfpathlineto{\pgfqpoint{3.632144in}{1.552278in}}%
\pgfpathlineto{\pgfqpoint{3.643418in}{1.457935in}}%
\pgfpathlineto{\pgfqpoint{3.653189in}{1.393629in}}%
\pgfpathlineto{\pgfqpoint{3.661458in}{1.352535in}}%
\pgfpathlineto{\pgfqpoint{3.668974in}{1.326104in}}%
\pgfpathlineto{\pgfqpoint{3.674987in}{1.312585in}}%
\pgfpathlineto{\pgfqpoint{3.679497in}{1.306934in}}%
\pgfpathlineto{\pgfqpoint{3.683255in}{1.305175in}}%
\pgfpathlineto{\pgfqpoint{3.687013in}{1.306099in}}%
\pgfpathlineto{\pgfqpoint{3.690771in}{1.309702in}}%
\pgfpathlineto{\pgfqpoint{3.695281in}{1.317551in}}%
\pgfpathlineto{\pgfqpoint{3.700543in}{1.331538in}}%
\pgfpathlineto{\pgfqpoint{3.706556in}{1.353818in}}%
\pgfpathlineto{\pgfqpoint{3.714072in}{1.390942in}}%
\pgfpathlineto{\pgfqpoint{3.722340in}{1.443337in}}%
\pgfpathlineto{\pgfqpoint{3.732112in}{1.520181in}}%
\pgfpathlineto{\pgfqpoint{3.743386in}{1.627586in}}%
\pgfpathlineto{\pgfqpoint{3.756164in}{1.771093in}}%
\pgfpathlineto{\pgfqpoint{3.771949in}{1.975092in}}%
\pgfpathlineto{\pgfqpoint{3.792243in}{2.268463in}}%
\pgfpathlineto{\pgfqpoint{3.855381in}{3.203576in}}%
\pgfpathlineto{\pgfqpoint{3.872668in}{3.417212in}}%
\pgfpathlineto{\pgfqpoint{3.887701in}{3.577463in}}%
\pgfpathlineto{\pgfqpoint{3.901231in}{3.699869in}}%
\pgfpathlineto{\pgfqpoint{3.914009in}{3.796350in}}%
\pgfpathlineto{\pgfqpoint{3.926035in}{3.870740in}}%
\pgfpathlineto{\pgfqpoint{3.937310in}{3.926898in}}%
\pgfpathlineto{\pgfqpoint{3.947833in}{3.968405in}}%
\pgfpathlineto{\pgfqpoint{3.957604in}{3.998410in}}%
\pgfpathlineto{\pgfqpoint{3.966624in}{4.019572in}}%
\pgfpathlineto{\pgfqpoint{3.975643in}{4.035179in}}%
\pgfpathlineto{\pgfqpoint{3.983911in}{4.045202in}}%
\pgfpathlineto{\pgfqpoint{3.992179in}{4.051674in}}%
\pgfpathlineto{\pgfqpoint{3.999696in}{4.054907in}}%
\pgfpathlineto{\pgfqpoint{4.007964in}{4.056000in}}%
\pgfpathlineto{\pgfqpoint{4.016983in}{4.054776in}}%
\pgfpathlineto{\pgfqpoint{4.027506in}{4.050884in}}%
\pgfpathlineto{\pgfqpoint{4.041036in}{4.043210in}}%
\pgfpathlineto{\pgfqpoint{4.065088in}{4.026351in}}%
\pgfpathlineto{\pgfqpoint{4.087638in}{4.011583in}}%
\pgfpathlineto{\pgfqpoint{4.103422in}{4.003890in}}%
\pgfpathlineto{\pgfqpoint{4.116952in}{3.999744in}}%
\pgfpathlineto{\pgfqpoint{4.129730in}{3.998178in}}%
\pgfpathlineto{\pgfqpoint{4.141756in}{3.998872in}}%
\pgfpathlineto{\pgfqpoint{4.154534in}{4.001866in}}%
\pgfpathlineto{\pgfqpoint{4.168063in}{4.007367in}}%
\pgfpathlineto{\pgfqpoint{4.183848in}{4.016285in}}%
\pgfpathlineto{\pgfqpoint{4.207900in}{4.032873in}}%
\pgfpathlineto{\pgfqpoint{4.231201in}{4.048085in}}%
\pgfpathlineto{\pgfqpoint{4.243979in}{4.053791in}}%
\pgfpathlineto{\pgfqpoint{4.253750in}{4.055871in}}%
\pgfpathlineto{\pgfqpoint{4.262018in}{4.055552in}}%
\pgfpathlineto{\pgfqpoint{4.269535in}{4.053229in}}%
\pgfpathlineto{\pgfqpoint{4.277051in}{4.048619in}}%
\pgfpathlineto{\pgfqpoint{4.284567in}{4.041371in}}%
\pgfpathlineto{\pgfqpoint{4.292084in}{4.031121in}}%
\pgfpathlineto{\pgfqpoint{4.300352in}{4.015935in}}%
\pgfpathlineto{\pgfqpoint{4.309372in}{3.994133in}}%
\pgfpathlineto{\pgfqpoint{4.318391in}{3.966252in}}%
\pgfpathlineto{\pgfqpoint{4.328163in}{3.928484in}}%
\pgfpathlineto{\pgfqpoint{4.338686in}{3.878225in}}%
\pgfpathlineto{\pgfqpoint{4.349960in}{3.812506in}}%
\pgfpathlineto{\pgfqpoint{4.361986in}{3.728104in}}%
\pgfpathlineto{\pgfqpoint{4.374764in}{3.621747in}}%
\pgfpathlineto{\pgfqpoint{4.389045in}{3.482627in}}%
\pgfpathlineto{\pgfqpoint{4.404830in}{3.305382in}}%
\pgfpathlineto{\pgfqpoint{4.423621in}{3.066804in}}%
\pgfpathlineto{\pgfqpoint{4.449177in}{2.710119in}}%
\pgfpathlineto{\pgfqpoint{4.492020in}{2.110931in}}%
\pgfpathlineto{\pgfqpoint{4.510060in}{1.891763in}}%
\pgfpathlineto{\pgfqpoint{4.524341in}{1.743475in}}%
\pgfpathlineto{\pgfqpoint{4.536367in}{1.639573in}}%
\pgfpathlineto{\pgfqpoint{4.546890in}{1.566229in}}%
\pgfpathlineto{\pgfqpoint{4.555910in}{1.517331in}}%
\pgfpathlineto{\pgfqpoint{4.564178in}{1.484327in}}%
\pgfpathlineto{\pgfqpoint{4.570942in}{1.465950in}}%
\pgfpathlineto{\pgfqpoint{4.576204in}{1.457101in}}%
\pgfpathlineto{\pgfqpoint{4.580714in}{1.453334in}}%
\pgfpathlineto{\pgfqpoint{4.584472in}{1.452892in}}%
\pgfpathlineto{\pgfqpoint{4.588230in}{1.454903in}}%
\pgfpathlineto{\pgfqpoint{4.592740in}{1.460544in}}%
\pgfpathlineto{\pgfqpoint{4.598001in}{1.471553in}}%
\pgfpathlineto{\pgfqpoint{4.604015in}{1.489914in}}%
\pgfpathlineto{\pgfqpoint{4.610779in}{1.517822in}}%
\pgfpathlineto{\pgfqpoint{4.619047in}{1.562091in}}%
\pgfpathlineto{\pgfqpoint{4.628819in}{1.628237in}}%
\pgfpathlineto{\pgfqpoint{4.639342in}{1.715205in}}%
\pgfpathlineto{\pgfqpoint{4.652120in}{1.840704in}}%
\pgfpathlineto{\pgfqpoint{4.667152in}{2.012260in}}%
\pgfpathlineto{\pgfqpoint{4.685943in}{2.254263in}}%
\pgfpathlineto{\pgfqpoint{4.719016in}{2.715823in}}%
\pgfpathlineto{\pgfqpoint{4.748330in}{3.112269in}}%
\pgfpathlineto{\pgfqpoint{4.767872in}{3.347287in}}%
\pgfpathlineto{\pgfqpoint{4.784408in}{3.520221in}}%
\pgfpathlineto{\pgfqpoint{4.799441in}{3.654341in}}%
\pgfpathlineto{\pgfqpoint{4.812971in}{3.755746in}}%
\pgfpathlineto{\pgfqpoint{4.825749in}{3.835122in}}%
\pgfpathlineto{\pgfqpoint{4.837775in}{3.896093in}}%
\pgfpathlineto{\pgfqpoint{4.849049in}{3.942109in}}%
\pgfpathlineto{\pgfqpoint{4.859572in}{3.976261in}}%
\pgfpathlineto{\pgfqpoint{4.870095in}{4.002857in}}%
\pgfpathlineto{\pgfqpoint{4.879867in}{4.021624in}}%
\pgfpathlineto{\pgfqpoint{4.888886in}{4.034557in}}%
\pgfpathlineto{\pgfqpoint{4.897906in}{4.043882in}}%
\pgfpathlineto{\pgfqpoint{4.906926in}{4.050165in}}%
\pgfpathlineto{\pgfqpoint{4.916697in}{4.054158in}}%
\pgfpathlineto{\pgfqpoint{4.927220in}{4.055891in}}%
\pgfpathlineto{\pgfqpoint{4.939246in}{4.055507in}}%
\pgfpathlineto{\pgfqpoint{4.955782in}{4.052491in}}%
\pgfpathlineto{\pgfqpoint{5.002384in}{4.042709in}}%
\pgfpathlineto{\pgfqpoint{5.019672in}{4.042036in}}%
\pgfpathlineto{\pgfqpoint{5.036959in}{4.043583in}}%
\pgfpathlineto{\pgfqpoint{5.058005in}{4.047852in}}%
\pgfpathlineto{\pgfqpoint{5.094084in}{4.055569in}}%
\pgfpathlineto{\pgfqpoint{5.106862in}{4.055806in}}%
\pgfpathlineto{\pgfqpoint{5.117385in}{4.053843in}}%
\pgfpathlineto{\pgfqpoint{5.126405in}{4.050050in}}%
\pgfpathlineto{\pgfqpoint{5.135424in}{4.043810in}}%
\pgfpathlineto{\pgfqpoint{5.144444in}{4.034617in}}%
\pgfpathlineto{\pgfqpoint{5.153464in}{4.021941in}}%
\pgfpathlineto{\pgfqpoint{5.162483in}{4.005234in}}%
\pgfpathlineto{\pgfqpoint{5.172255in}{3.981941in}}%
\pgfpathlineto{\pgfqpoint{5.182026in}{3.952563in}}%
\pgfpathlineto{\pgfqpoint{5.192549in}{3.913347in}}%
\pgfpathlineto{\pgfqpoint{5.203072in}{3.865514in}}%
\pgfpathlineto{\pgfqpoint{5.214347in}{3.803938in}}%
\pgfpathlineto{\pgfqpoint{5.226373in}{3.725785in}}%
\pgfpathlineto{\pgfqpoint{5.239151in}{3.628149in}}%
\pgfpathlineto{\pgfqpoint{5.253432in}{3.501197in}}%
\pgfpathlineto{\pgfqpoint{5.269216in}{3.340007in}}%
\pgfpathlineto{\pgfqpoint{5.287256in}{3.132287in}}%
\pgfpathlineto{\pgfqpoint{5.311308in}{2.827394in}}%
\pgfpathlineto{\pgfqpoint{5.362420in}{2.171719in}}%
\pgfpathlineto{\pgfqpoint{5.379708in}{1.981496in}}%
\pgfpathlineto{\pgfqpoint{5.393989in}{1.847223in}}%
\pgfpathlineto{\pgfqpoint{5.406015in}{1.753428in}}%
\pgfpathlineto{\pgfqpoint{5.416538in}{1.687469in}}%
\pgfpathlineto{\pgfqpoint{5.425558in}{1.643729in}}%
\pgfpathlineto{\pgfqpoint{5.433074in}{1.616679in}}%
\pgfpathlineto{\pgfqpoint{5.439839in}{1.599825in}}%
\pgfpathlineto{\pgfqpoint{5.445100in}{1.591689in}}%
\pgfpathlineto{\pgfqpoint{5.449610in}{1.588204in}}%
\pgfpathlineto{\pgfqpoint{5.453368in}{1.587765in}}%
\pgfpathlineto{\pgfqpoint{5.457126in}{1.589567in}}%
\pgfpathlineto{\pgfqpoint{5.461636in}{1.594680in}}%
\pgfpathlineto{\pgfqpoint{5.466898in}{1.604691in}}%
\pgfpathlineto{\pgfqpoint{5.472911in}{1.621414in}}%
\pgfpathlineto{\pgfqpoint{5.479676in}{1.646851in}}%
\pgfpathlineto{\pgfqpoint{5.487944in}{1.687219in}}%
\pgfpathlineto{\pgfqpoint{5.497715in}{1.747552in}}%
\pgfpathlineto{\pgfqpoint{5.508238in}{1.826887in}}%
\pgfpathlineto{\pgfqpoint{5.521016in}{1.941387in}}%
\pgfpathlineto{\pgfqpoint{5.533794in}{2.073151in}}%
\pgfpathlineto{\pgfqpoint{5.534545in}{0.696000in}}%
\pgfpathlineto{\pgfqpoint{5.534545in}{0.696000in}}%
\pgfusepath{stroke}%
\end{pgfscope}%
\begin{pgfscope}%
\pgfpathrectangle{\pgfqpoint{0.800000in}{0.528000in}}{\pgfqpoint{4.960000in}{3.696000in}}%
\pgfusepath{clip}%
\pgfsetrectcap%
\pgfsetroundjoin%
\pgfsetlinewidth{1.505625pt}%
\definecolor{currentstroke}{rgb}{1.000000,0.498039,0.054902}%
\pgfsetstrokecolor{currentstroke}%
\pgfsetdash{}{0pt}%
\pgfpathmoveto{\pgfqpoint{1.025455in}{3.237068in}}%
\pgfpathlineto{\pgfqpoint{1.065291in}{3.235975in}}%
\pgfpathlineto{\pgfqpoint{1.107383in}{3.232579in}}%
\pgfpathlineto{\pgfqpoint{1.155488in}{3.226399in}}%
\pgfpathlineto{\pgfqpoint{1.229149in}{3.214340in}}%
\pgfpathlineto{\pgfqpoint{1.303562in}{3.202857in}}%
\pgfpathlineto{\pgfqpoint{1.350163in}{3.197920in}}%
\pgfpathlineto{\pgfqpoint{1.391503in}{3.195738in}}%
\pgfpathlineto{\pgfqpoint{1.431340in}{3.195832in}}%
\pgfpathlineto{\pgfqpoint{1.472681in}{3.198184in}}%
\pgfpathlineto{\pgfqpoint{1.518531in}{3.203114in}}%
\pgfpathlineto{\pgfqpoint{1.577910in}{3.211945in}}%
\pgfpathlineto{\pgfqpoint{1.690656in}{3.229072in}}%
\pgfpathlineto{\pgfqpoint{1.738010in}{3.233652in}}%
\pgfpathlineto{\pgfqpoint{1.776343in}{3.235455in}}%
\pgfpathlineto{\pgfqpoint{1.777847in}{0.696105in}}%
\pgfpathlineto{\pgfqpoint{1.865789in}{0.696100in}}%
\pgfpathlineto{\pgfqpoint{1.867292in}{3.781272in}}%
\pgfpathlineto{\pgfqpoint{1.878566in}{3.867567in}}%
\pgfpathlineto{\pgfqpoint{1.889089in}{3.931686in}}%
\pgfpathlineto{\pgfqpoint{1.898861in}{3.977907in}}%
\pgfpathlineto{\pgfqpoint{1.907880in}{4.010049in}}%
\pgfpathlineto{\pgfqpoint{1.916148in}{4.031392in}}%
\pgfpathlineto{\pgfqpoint{1.923665in}{4.044659in}}%
\pgfpathlineto{\pgfqpoint{1.930430in}{4.052064in}}%
\pgfpathlineto{\pgfqpoint{1.936443in}{4.055369in}}%
\pgfpathlineto{\pgfqpoint{1.941704in}{4.055954in}}%
\pgfpathlineto{\pgfqpoint{1.947717in}{4.054226in}}%
\pgfpathlineto{\pgfqpoint{1.954482in}{4.049536in}}%
\pgfpathlineto{\pgfqpoint{1.961999in}{4.041332in}}%
\pgfpathlineto{\pgfqpoint{1.971018in}{4.027955in}}%
\pgfpathlineto{\pgfqpoint{1.982293in}{4.006883in}}%
\pgfpathlineto{\pgfqpoint{1.996574in}{3.975143in}}%
\pgfpathlineto{\pgfqpoint{2.018372in}{3.920537in}}%
\pgfpathlineto{\pgfqpoint{2.064222in}{3.804692in}}%
\pgfpathlineto{\pgfqpoint{2.084516in}{3.760101in}}%
\pgfpathlineto{\pgfqpoint{2.101804in}{3.727303in}}%
\pgfpathlineto{\pgfqpoint{2.117588in}{3.702086in}}%
\pgfpathlineto{\pgfqpoint{2.131869in}{3.683426in}}%
\pgfpathlineto{\pgfqpoint{2.144647in}{3.670204in}}%
\pgfpathlineto{\pgfqpoint{2.156673in}{3.660819in}}%
\pgfpathlineto{\pgfqpoint{2.167948in}{3.654748in}}%
\pgfpathlineto{\pgfqpoint{2.178471in}{3.651476in}}%
\pgfpathlineto{\pgfqpoint{2.188994in}{3.650522in}}%
\pgfpathlineto{\pgfqpoint{2.199517in}{3.651883in}}%
\pgfpathlineto{\pgfqpoint{2.210040in}{3.655552in}}%
\pgfpathlineto{\pgfqpoint{2.221314in}{3.662027in}}%
\pgfpathlineto{\pgfqpoint{2.232589in}{3.671106in}}%
\pgfpathlineto{\pgfqpoint{2.244615in}{3.683616in}}%
\pgfpathlineto{\pgfqpoint{2.258145in}{3.701093in}}%
\pgfpathlineto{\pgfqpoint{2.272426in}{3.723303in}}%
\pgfpathlineto{\pgfqpoint{2.288210in}{3.752077in}}%
\pgfpathlineto{\pgfqpoint{2.306250in}{3.789846in}}%
\pgfpathlineto{\pgfqpoint{2.328047in}{3.841107in}}%
\pgfpathlineto{\pgfqpoint{2.397198in}{4.008926in}}%
\pgfpathlineto{\pgfqpoint{2.409225in}{4.030420in}}%
\pgfpathlineto{\pgfqpoint{2.418996in}{4.043841in}}%
\pgfpathlineto{\pgfqpoint{2.427264in}{4.051660in}}%
\pgfpathlineto{\pgfqpoint{2.434029in}{4.055220in}}%
\pgfpathlineto{\pgfqpoint{2.440042in}{4.055951in}}%
\pgfpathlineto{\pgfqpoint{2.445303in}{4.054515in}}%
\pgfpathlineto{\pgfqpoint{2.450565in}{4.050971in}}%
\pgfpathlineto{\pgfqpoint{2.456578in}{4.044123in}}%
\pgfpathlineto{\pgfqpoint{2.463343in}{4.032556in}}%
\pgfpathlineto{\pgfqpoint{2.470859in}{4.014493in}}%
\pgfpathlineto{\pgfqpoint{2.478375in}{3.990487in}}%
\pgfpathlineto{\pgfqpoint{2.486644in}{3.956683in}}%
\pgfpathlineto{\pgfqpoint{2.495663in}{3.910347in}}%
\pgfpathlineto{\pgfqpoint{2.505435in}{3.848330in}}%
\pgfpathlineto{\pgfqpoint{2.515958in}{3.767142in}}%
\pgfpathlineto{\pgfqpoint{2.527984in}{3.655566in}}%
\pgfpathlineto{\pgfqpoint{2.540762in}{3.515100in}}%
\pgfpathlineto{\pgfqpoint{2.555043in}{3.332630in}}%
\pgfpathlineto{\pgfqpoint{2.571579in}{3.091690in}}%
\pgfpathlineto{\pgfqpoint{2.593376in}{2.738091in}}%
\pgfpathlineto{\pgfqpoint{2.648998in}{1.818014in}}%
\pgfpathlineto{\pgfqpoint{2.665534in}{1.590203in}}%
\pgfpathlineto{\pgfqpoint{2.679063in}{1.433225in}}%
\pgfpathlineto{\pgfqpoint{2.690338in}{1.325995in}}%
\pgfpathlineto{\pgfqpoint{2.700109in}{1.252004in}}%
\pgfpathlineto{\pgfqpoint{2.709129in}{1.200177in}}%
\pgfpathlineto{\pgfqpoint{2.716646in}{1.169469in}}%
\pgfpathlineto{\pgfqpoint{2.722659in}{1.153229in}}%
\pgfpathlineto{\pgfqpoint{2.727920in}{1.145150in}}%
\pgfpathlineto{\pgfqpoint{2.731678in}{1.142896in}}%
\pgfpathlineto{\pgfqpoint{2.734685in}{1.143206in}}%
\pgfpathlineto{\pgfqpoint{2.738443in}{1.146232in}}%
\pgfpathlineto{\pgfqpoint{2.742953in}{1.153723in}}%
\pgfpathlineto{\pgfqpoint{2.748214in}{1.167753in}}%
\pgfpathlineto{\pgfqpoint{2.754228in}{1.190698in}}%
\pgfpathlineto{\pgfqpoint{2.761744in}{1.229571in}}%
\pgfpathlineto{\pgfqpoint{2.770012in}{1.285067in}}%
\pgfpathlineto{\pgfqpoint{2.779783in}{1.367142in}}%
\pgfpathlineto{\pgfqpoint{2.791058in}{1.482631in}}%
\pgfpathlineto{\pgfqpoint{2.803836in}{1.637792in}}%
\pgfpathlineto{\pgfqpoint{2.818869in}{1.848219in}}%
\pgfpathlineto{\pgfqpoint{2.838411in}{2.154735in}}%
\pgfpathlineto{\pgfqpoint{2.904556in}{3.220103in}}%
\pgfpathlineto{\pgfqpoint{2.921092in}{3.437953in}}%
\pgfpathlineto{\pgfqpoint{2.935373in}{3.599264in}}%
\pgfpathlineto{\pgfqpoint{2.948902in}{3.727633in}}%
\pgfpathlineto{\pgfqpoint{2.960929in}{3.821705in}}%
\pgfpathlineto{\pgfqpoint{2.972203in}{3.893358in}}%
\pgfpathlineto{\pgfqpoint{2.982726in}{3.946641in}}%
\pgfpathlineto{\pgfqpoint{2.992498in}{3.985260in}}%
\pgfpathlineto{\pgfqpoint{3.001517in}{4.012443in}}%
\pgfpathlineto{\pgfqpoint{3.009785in}{4.030907in}}%
\pgfpathlineto{\pgfqpoint{3.017302in}{4.042871in}}%
\pgfpathlineto{\pgfqpoint{3.024818in}{4.050730in}}%
\pgfpathlineto{\pgfqpoint{3.031583in}{4.054674in}}%
\pgfpathlineto{\pgfqpoint{3.038348in}{4.055997in}}%
\pgfpathlineto{\pgfqpoint{3.045112in}{4.055017in}}%
\pgfpathlineto{\pgfqpoint{3.052629in}{4.051600in}}%
\pgfpathlineto{\pgfqpoint{3.061648in}{4.044805in}}%
\pgfpathlineto{\pgfqpoint{3.072171in}{4.033968in}}%
\pgfpathlineto{\pgfqpoint{3.086453in}{4.015809in}}%
\pgfpathlineto{\pgfqpoint{3.114263in}{3.975777in}}%
\pgfpathlineto{\pgfqpoint{3.137564in}{3.944088in}}%
\pgfpathlineto{\pgfqpoint{3.153349in}{3.926122in}}%
\pgfpathlineto{\pgfqpoint{3.166878in}{3.913853in}}%
\pgfpathlineto{\pgfqpoint{3.178904in}{3.905742in}}%
\pgfpathlineto{\pgfqpoint{3.190179in}{3.900718in}}%
\pgfpathlineto{\pgfqpoint{3.200702in}{3.898376in}}%
\pgfpathlineto{\pgfqpoint{3.211225in}{3.898341in}}%
\pgfpathlineto{\pgfqpoint{3.221748in}{3.900609in}}%
\pgfpathlineto{\pgfqpoint{3.232271in}{3.905134in}}%
\pgfpathlineto{\pgfqpoint{3.243545in}{3.912386in}}%
\pgfpathlineto{\pgfqpoint{3.256323in}{3.923402in}}%
\pgfpathlineto{\pgfqpoint{3.270604in}{3.938820in}}%
\pgfpathlineto{\pgfqpoint{3.287892in}{3.960952in}}%
\pgfpathlineto{\pgfqpoint{3.323219in}{4.011132in}}%
\pgfpathlineto{\pgfqpoint{3.341259in}{4.034108in}}%
\pgfpathlineto{\pgfqpoint{3.353285in}{4.046058in}}%
\pgfpathlineto{\pgfqpoint{3.363056in}{4.052737in}}%
\pgfpathlineto{\pgfqpoint{3.371324in}{4.055650in}}%
\pgfpathlineto{\pgfqpoint{3.378089in}{4.055794in}}%
\pgfpathlineto{\pgfqpoint{3.384854in}{4.053612in}}%
\pgfpathlineto{\pgfqpoint{3.391619in}{4.048806in}}%
\pgfpathlineto{\pgfqpoint{3.398383in}{4.041068in}}%
\pgfpathlineto{\pgfqpoint{3.405900in}{4.028655in}}%
\pgfpathlineto{\pgfqpoint{3.413416in}{4.011814in}}%
\pgfpathlineto{\pgfqpoint{3.421684in}{3.987671in}}%
\pgfpathlineto{\pgfqpoint{3.430704in}{3.954008in}}%
\pgfpathlineto{\pgfqpoint{3.440475in}{3.908183in}}%
\pgfpathlineto{\pgfqpoint{3.450998in}{3.847146in}}%
\pgfpathlineto{\pgfqpoint{3.462273in}{3.767523in}}%
\pgfpathlineto{\pgfqpoint{3.474299in}{3.665785in}}%
\pgfpathlineto{\pgfqpoint{3.487077in}{3.538569in}}%
\pgfpathlineto{\pgfqpoint{3.501358in}{3.373911in}}%
\pgfpathlineto{\pgfqpoint{3.517894in}{3.156618in}}%
\pgfpathlineto{\pgfqpoint{3.538940in}{2.848093in}}%
\pgfpathlineto{\pgfqpoint{3.603581in}{1.875668in}}%
\pgfpathlineto{\pgfqpoint{3.619366in}{1.683228in}}%
\pgfpathlineto{\pgfqpoint{3.632144in}{1.552278in}}%
\pgfpathlineto{\pgfqpoint{3.643418in}{1.457935in}}%
\pgfpathlineto{\pgfqpoint{3.653189in}{1.393629in}}%
\pgfpathlineto{\pgfqpoint{3.661458in}{1.352535in}}%
\pgfpathlineto{\pgfqpoint{3.668974in}{1.326104in}}%
\pgfpathlineto{\pgfqpoint{3.674987in}{1.312585in}}%
\pgfpathlineto{\pgfqpoint{3.679497in}{1.306934in}}%
\pgfpathlineto{\pgfqpoint{3.683255in}{1.305175in}}%
\pgfpathlineto{\pgfqpoint{3.687013in}{1.306099in}}%
\pgfpathlineto{\pgfqpoint{3.690771in}{1.309702in}}%
\pgfpathlineto{\pgfqpoint{3.695281in}{1.317551in}}%
\pgfpathlineto{\pgfqpoint{3.700543in}{1.331538in}}%
\pgfpathlineto{\pgfqpoint{3.706556in}{1.353818in}}%
\pgfpathlineto{\pgfqpoint{3.714072in}{1.390942in}}%
\pgfpathlineto{\pgfqpoint{3.722340in}{1.443337in}}%
\pgfpathlineto{\pgfqpoint{3.732112in}{1.520181in}}%
\pgfpathlineto{\pgfqpoint{3.743386in}{1.627586in}}%
\pgfpathlineto{\pgfqpoint{3.756164in}{1.771093in}}%
\pgfpathlineto{\pgfqpoint{3.771949in}{1.975092in}}%
\pgfpathlineto{\pgfqpoint{3.792243in}{2.268463in}}%
\pgfpathlineto{\pgfqpoint{3.855381in}{3.203576in}}%
\pgfpathlineto{\pgfqpoint{3.872668in}{3.417212in}}%
\pgfpathlineto{\pgfqpoint{3.887701in}{3.577463in}}%
\pgfpathlineto{\pgfqpoint{3.901231in}{3.699869in}}%
\pgfpathlineto{\pgfqpoint{3.914009in}{3.796350in}}%
\pgfpathlineto{\pgfqpoint{3.926035in}{3.870740in}}%
\pgfpathlineto{\pgfqpoint{3.937310in}{3.926898in}}%
\pgfpathlineto{\pgfqpoint{3.947833in}{3.968405in}}%
\pgfpathlineto{\pgfqpoint{3.957604in}{3.998410in}}%
\pgfpathlineto{\pgfqpoint{3.966624in}{4.019572in}}%
\pgfpathlineto{\pgfqpoint{3.975643in}{4.035179in}}%
\pgfpathlineto{\pgfqpoint{3.983911in}{4.045202in}}%
\pgfpathlineto{\pgfqpoint{3.992179in}{4.051674in}}%
\pgfpathlineto{\pgfqpoint{3.999696in}{4.054907in}}%
\pgfpathlineto{\pgfqpoint{4.007964in}{4.056000in}}%
\pgfpathlineto{\pgfqpoint{4.016983in}{4.054776in}}%
\pgfpathlineto{\pgfqpoint{4.027506in}{4.050884in}}%
\pgfpathlineto{\pgfqpoint{4.041036in}{4.043210in}}%
\pgfpathlineto{\pgfqpoint{4.065088in}{4.026351in}}%
\pgfpathlineto{\pgfqpoint{4.087638in}{4.011583in}}%
\pgfpathlineto{\pgfqpoint{4.103422in}{4.003890in}}%
\pgfpathlineto{\pgfqpoint{4.116952in}{3.999744in}}%
\pgfpathlineto{\pgfqpoint{4.129730in}{3.998178in}}%
\pgfpathlineto{\pgfqpoint{4.141756in}{3.998872in}}%
\pgfpathlineto{\pgfqpoint{4.154534in}{4.001866in}}%
\pgfpathlineto{\pgfqpoint{4.168063in}{4.007367in}}%
\pgfpathlineto{\pgfqpoint{4.183848in}{4.016285in}}%
\pgfpathlineto{\pgfqpoint{4.207900in}{4.032873in}}%
\pgfpathlineto{\pgfqpoint{4.231201in}{4.048085in}}%
\pgfpathlineto{\pgfqpoint{4.243979in}{4.053791in}}%
\pgfpathlineto{\pgfqpoint{4.253750in}{4.055871in}}%
\pgfpathlineto{\pgfqpoint{4.262018in}{4.055552in}}%
\pgfpathlineto{\pgfqpoint{4.269535in}{4.053229in}}%
\pgfpathlineto{\pgfqpoint{4.277051in}{4.048619in}}%
\pgfpathlineto{\pgfqpoint{4.284567in}{4.041371in}}%
\pgfpathlineto{\pgfqpoint{4.292084in}{4.031121in}}%
\pgfpathlineto{\pgfqpoint{4.300352in}{4.015935in}}%
\pgfpathlineto{\pgfqpoint{4.309372in}{3.994133in}}%
\pgfpathlineto{\pgfqpoint{4.318391in}{3.966252in}}%
\pgfpathlineto{\pgfqpoint{4.328163in}{3.928484in}}%
\pgfpathlineto{\pgfqpoint{4.338686in}{3.878225in}}%
\pgfpathlineto{\pgfqpoint{4.349960in}{3.812506in}}%
\pgfpathlineto{\pgfqpoint{4.361986in}{3.728104in}}%
\pgfpathlineto{\pgfqpoint{4.374764in}{3.621747in}}%
\pgfpathlineto{\pgfqpoint{4.389045in}{3.482627in}}%
\pgfpathlineto{\pgfqpoint{4.404830in}{3.305382in}}%
\pgfpathlineto{\pgfqpoint{4.423621in}{3.066804in}}%
\pgfpathlineto{\pgfqpoint{4.449177in}{2.710119in}}%
\pgfpathlineto{\pgfqpoint{4.492020in}{2.110931in}}%
\pgfpathlineto{\pgfqpoint{4.510060in}{1.891763in}}%
\pgfpathlineto{\pgfqpoint{4.524341in}{1.743475in}}%
\pgfpathlineto{\pgfqpoint{4.536367in}{1.639573in}}%
\pgfpathlineto{\pgfqpoint{4.546890in}{1.566229in}}%
\pgfpathlineto{\pgfqpoint{4.555910in}{1.517331in}}%
\pgfpathlineto{\pgfqpoint{4.564178in}{1.484327in}}%
\pgfpathlineto{\pgfqpoint{4.570942in}{1.465950in}}%
\pgfpathlineto{\pgfqpoint{4.576204in}{1.457101in}}%
\pgfpathlineto{\pgfqpoint{4.580714in}{1.453334in}}%
\pgfpathlineto{\pgfqpoint{4.584472in}{1.452892in}}%
\pgfpathlineto{\pgfqpoint{4.588230in}{1.454903in}}%
\pgfpathlineto{\pgfqpoint{4.592740in}{1.460544in}}%
\pgfpathlineto{\pgfqpoint{4.598001in}{1.471553in}}%
\pgfpathlineto{\pgfqpoint{4.604015in}{1.489914in}}%
\pgfpathlineto{\pgfqpoint{4.610779in}{1.517822in}}%
\pgfpathlineto{\pgfqpoint{4.619047in}{1.562091in}}%
\pgfpathlineto{\pgfqpoint{4.628819in}{1.628237in}}%
\pgfpathlineto{\pgfqpoint{4.639342in}{1.715205in}}%
\pgfpathlineto{\pgfqpoint{4.652120in}{1.840704in}}%
\pgfpathlineto{\pgfqpoint{4.667152in}{2.012260in}}%
\pgfpathlineto{\pgfqpoint{4.685943in}{2.254263in}}%
\pgfpathlineto{\pgfqpoint{4.719016in}{2.715823in}}%
\pgfpathlineto{\pgfqpoint{4.748330in}{3.112269in}}%
\pgfpathlineto{\pgfqpoint{4.767872in}{3.347287in}}%
\pgfpathlineto{\pgfqpoint{4.784408in}{3.520221in}}%
\pgfpathlineto{\pgfqpoint{4.799441in}{3.654341in}}%
\pgfpathlineto{\pgfqpoint{4.812971in}{3.755746in}}%
\pgfpathlineto{\pgfqpoint{4.825749in}{3.835122in}}%
\pgfpathlineto{\pgfqpoint{4.837775in}{3.896093in}}%
\pgfpathlineto{\pgfqpoint{4.849049in}{3.942109in}}%
\pgfpathlineto{\pgfqpoint{4.859572in}{3.976261in}}%
\pgfpathlineto{\pgfqpoint{4.870095in}{4.002857in}}%
\pgfpathlineto{\pgfqpoint{4.879867in}{4.021624in}}%
\pgfpathlineto{\pgfqpoint{4.888886in}{4.034557in}}%
\pgfpathlineto{\pgfqpoint{4.897906in}{4.043882in}}%
\pgfpathlineto{\pgfqpoint{4.906926in}{4.050165in}}%
\pgfpathlineto{\pgfqpoint{4.916697in}{4.054158in}}%
\pgfpathlineto{\pgfqpoint{4.927220in}{4.055891in}}%
\pgfpathlineto{\pgfqpoint{4.939246in}{4.055507in}}%
\pgfpathlineto{\pgfqpoint{4.955782in}{4.052491in}}%
\pgfpathlineto{\pgfqpoint{5.002384in}{4.042709in}}%
\pgfpathlineto{\pgfqpoint{5.019672in}{4.042036in}}%
\pgfpathlineto{\pgfqpoint{5.036959in}{4.043583in}}%
\pgfpathlineto{\pgfqpoint{5.058005in}{4.047852in}}%
\pgfpathlineto{\pgfqpoint{5.094084in}{4.055569in}}%
\pgfpathlineto{\pgfqpoint{5.106862in}{4.055806in}}%
\pgfpathlineto{\pgfqpoint{5.117385in}{4.053843in}}%
\pgfpathlineto{\pgfqpoint{5.126405in}{4.050050in}}%
\pgfpathlineto{\pgfqpoint{5.135424in}{4.043810in}}%
\pgfpathlineto{\pgfqpoint{5.144444in}{4.034617in}}%
\pgfpathlineto{\pgfqpoint{5.153464in}{4.021941in}}%
\pgfpathlineto{\pgfqpoint{5.162483in}{4.005234in}}%
\pgfpathlineto{\pgfqpoint{5.172255in}{3.981941in}}%
\pgfpathlineto{\pgfqpoint{5.182026in}{3.952563in}}%
\pgfpathlineto{\pgfqpoint{5.192549in}{3.913347in}}%
\pgfpathlineto{\pgfqpoint{5.203072in}{3.865514in}}%
\pgfpathlineto{\pgfqpoint{5.214347in}{3.803938in}}%
\pgfpathlineto{\pgfqpoint{5.226373in}{3.725785in}}%
\pgfpathlineto{\pgfqpoint{5.239151in}{3.628149in}}%
\pgfpathlineto{\pgfqpoint{5.253432in}{3.501197in}}%
\pgfpathlineto{\pgfqpoint{5.269216in}{3.340007in}}%
\pgfpathlineto{\pgfqpoint{5.287256in}{3.132287in}}%
\pgfpathlineto{\pgfqpoint{5.311308in}{2.827394in}}%
\pgfpathlineto{\pgfqpoint{5.362420in}{2.171719in}}%
\pgfpathlineto{\pgfqpoint{5.379708in}{1.981496in}}%
\pgfpathlineto{\pgfqpoint{5.393989in}{1.847223in}}%
\pgfpathlineto{\pgfqpoint{5.406015in}{1.753428in}}%
\pgfpathlineto{\pgfqpoint{5.416538in}{1.687469in}}%
\pgfpathlineto{\pgfqpoint{5.425558in}{1.643729in}}%
\pgfpathlineto{\pgfqpoint{5.433074in}{1.616679in}}%
\pgfpathlineto{\pgfqpoint{5.439839in}{1.599825in}}%
\pgfpathlineto{\pgfqpoint{5.445100in}{1.591689in}}%
\pgfpathlineto{\pgfqpoint{5.449610in}{1.588204in}}%
\pgfpathlineto{\pgfqpoint{5.453368in}{1.587765in}}%
\pgfpathlineto{\pgfqpoint{5.457126in}{1.589567in}}%
\pgfpathlineto{\pgfqpoint{5.461636in}{1.594680in}}%
\pgfpathlineto{\pgfqpoint{5.466898in}{1.604691in}}%
\pgfpathlineto{\pgfqpoint{5.472911in}{1.621414in}}%
\pgfpathlineto{\pgfqpoint{5.479676in}{1.646851in}}%
\pgfpathlineto{\pgfqpoint{5.487944in}{1.687219in}}%
\pgfpathlineto{\pgfqpoint{5.497715in}{1.747552in}}%
\pgfpathlineto{\pgfqpoint{5.508238in}{1.826887in}}%
\pgfpathlineto{\pgfqpoint{5.521016in}{1.941387in}}%
\pgfpathlineto{\pgfqpoint{5.533794in}{2.073151in}}%
\pgfpathlineto{\pgfqpoint{5.534545in}{0.696000in}}%
\pgfpathlineto{\pgfqpoint{5.534545in}{0.696000in}}%
\pgfusepath{stroke}%
\end{pgfscope}%
\begin{pgfscope}%
\pgfsetrectcap%
\pgfsetmiterjoin%
\pgfsetlinewidth{0.803000pt}%
\definecolor{currentstroke}{rgb}{0.000000,0.000000,0.000000}%
\pgfsetstrokecolor{currentstroke}%
\pgfsetdash{}{0pt}%
\pgfpathmoveto{\pgfqpoint{0.800000in}{0.528000in}}%
\pgfpathlineto{\pgfqpoint{0.800000in}{4.224000in}}%
\pgfusepath{stroke}%
\end{pgfscope}%
\begin{pgfscope}%
\pgfsetrectcap%
\pgfsetmiterjoin%
\pgfsetlinewidth{0.803000pt}%
\definecolor{currentstroke}{rgb}{0.000000,0.000000,0.000000}%
\pgfsetstrokecolor{currentstroke}%
\pgfsetdash{}{0pt}%
\pgfpathmoveto{\pgfqpoint{5.760000in}{0.528000in}}%
\pgfpathlineto{\pgfqpoint{5.760000in}{4.224000in}}%
\pgfusepath{stroke}%
\end{pgfscope}%
\begin{pgfscope}%
\pgfsetrectcap%
\pgfsetmiterjoin%
\pgfsetlinewidth{0.803000pt}%
\definecolor{currentstroke}{rgb}{0.000000,0.000000,0.000000}%
\pgfsetstrokecolor{currentstroke}%
\pgfsetdash{}{0pt}%
\pgfpathmoveto{\pgfqpoint{0.800000in}{0.528000in}}%
\pgfpathlineto{\pgfqpoint{5.760000in}{0.528000in}}%
\pgfusepath{stroke}%
\end{pgfscope}%
\begin{pgfscope}%
\pgfsetrectcap%
\pgfsetmiterjoin%
\pgfsetlinewidth{0.803000pt}%
\definecolor{currentstroke}{rgb}{0.000000,0.000000,0.000000}%
\pgfsetstrokecolor{currentstroke}%
\pgfsetdash{}{0pt}%
\pgfpathmoveto{\pgfqpoint{0.800000in}{4.224000in}}%
\pgfpathlineto{\pgfqpoint{5.760000in}{4.224000in}}%
\pgfusepath{stroke}%
\end{pgfscope}%
\begin{pgfscope}%
\definecolor{textcolor}{rgb}{0.000000,0.000000,0.000000}%
\pgfsetstrokecolor{textcolor}%
\pgfsetfillcolor{textcolor}%
\pgftext[x=3.280000in,y=4.307333in,,base]{\color{textcolor}\rmfamily\fontsize{12.000000}{14.400000}\selectfont Electrical power - comparison algebraic vs. non-algebraic}%
\end{pgfscope}%
\begin{pgfscope}%
\pgfsetbuttcap%
\pgfsetmiterjoin%
\definecolor{currentfill}{rgb}{1.000000,1.000000,1.000000}%
\pgfsetfillcolor{currentfill}%
\pgfsetfillopacity{0.800000}%
\pgfsetlinewidth{1.003750pt}%
\definecolor{currentstroke}{rgb}{0.800000,0.800000,0.800000}%
\pgfsetstrokecolor{currentstroke}%
\pgfsetstrokeopacity{0.800000}%
\pgfsetdash{}{0pt}%
\pgfpathmoveto{\pgfqpoint{2.435423in}{0.597444in}}%
\pgfpathlineto{\pgfqpoint{4.124577in}{0.597444in}}%
\pgfpathquadraticcurveto{\pgfqpoint{4.152355in}{0.597444in}}{\pgfqpoint{4.152355in}{0.625222in}}%
\pgfpathlineto{\pgfqpoint{4.152355in}{1.015114in}}%
\pgfpathquadraticcurveto{\pgfqpoint{4.152355in}{1.042892in}}{\pgfqpoint{4.124577in}{1.042892in}}%
\pgfpathlineto{\pgfqpoint{2.435423in}{1.042892in}}%
\pgfpathquadraticcurveto{\pgfqpoint{2.407645in}{1.042892in}}{\pgfqpoint{2.407645in}{1.015114in}}%
\pgfpathlineto{\pgfqpoint{2.407645in}{0.625222in}}%
\pgfpathquadraticcurveto{\pgfqpoint{2.407645in}{0.597444in}}{\pgfqpoint{2.435423in}{0.597444in}}%
\pgfpathlineto{\pgfqpoint{2.435423in}{0.597444in}}%
\pgfpathclose%
\pgfusepath{stroke,fill}%
\end{pgfscope}%
\begin{pgfscope}%
\pgfsetrectcap%
\pgfsetroundjoin%
\pgfsetlinewidth{1.505625pt}%
\definecolor{currentstroke}{rgb}{0.121569,0.466667,0.705882}%
\pgfsetstrokecolor{currentstroke}%
\pgfsetdash{}{0pt}%
\pgfpathmoveto{\pgfqpoint{2.463200in}{0.933748in}}%
\pgfpathlineto{\pgfqpoint{2.602089in}{0.933748in}}%
\pgfpathlineto{\pgfqpoint{2.740978in}{0.933748in}}%
\pgfusepath{stroke}%
\end{pgfscope}%
\begin{pgfscope}%
\definecolor{textcolor}{rgb}{0.000000,0.000000,0.000000}%
\pgfsetstrokecolor{textcolor}%
\pgfsetfillcolor{textcolor}%
\pgftext[x=2.852089in,y=0.885137in,left,base]{\color{textcolor}\rmfamily\fontsize{10.000000}{12.000000}\selectfont power non-algebraic}%
\end{pgfscope}%
\begin{pgfscope}%
\pgfsetrectcap%
\pgfsetroundjoin%
\pgfsetlinewidth{1.505625pt}%
\definecolor{currentstroke}{rgb}{1.000000,0.498039,0.054902}%
\pgfsetstrokecolor{currentstroke}%
\pgfsetdash{}{0pt}%
\pgfpathmoveto{\pgfqpoint{2.463200in}{0.731857in}}%
\pgfpathlineto{\pgfqpoint{2.602089in}{0.731857in}}%
\pgfpathlineto{\pgfqpoint{2.740978in}{0.731857in}}%
\pgfusepath{stroke}%
\end{pgfscope}%
\begin{pgfscope}%
\definecolor{textcolor}{rgb}{0.000000,0.000000,0.000000}%
\pgfsetstrokecolor{textcolor}%
\pgfsetfillcolor{textcolor}%
\pgftext[x=2.852089in,y=0.683246in,left,base]{\color{textcolor}\rmfamily\fontsize{10.000000}{12.000000}\selectfont power algebraic}%
\end{pgfscope}%
\end{pgfpicture}%
\makeatother%
\endgroup%


% \chapter{Derivations}
% \label{app:derivation-analytical}

% Analytical solution of the \acs{CCT} from \autoref{sec:analytical-method}.

% \begin{align}
%     \int_{\delta_\mathrm{0}}^{\delta_\mathrm{1}}\Delta P~d\delta = 0 \label{eq:gen-eac},
% \end{align}
% while the more expressive can be achieved through splitting up the integral borders and equalizing both areas:
% \begin{align}
%     % A_\mathrm{acc}=A_\mathrm{dec} \nonumber \\
%     \int_{\delta_\mathrm{0}}^{\delta_\mathrm{c}}(P_\mathrm{m}-P_\mathrm{e})~d\delta = \int_{\delta_\mathrm{c}}^{\delta_\mathrm{max}}(P_\mathrm{e}-P_\mathrm{m})~d\delta \label{eq:big-eac}
% \end{align}
% With consideration of $\delta_\mathrm{max}=\pi-\delta_\mathrm{0}$, $P_\mathrm{e,normal}=P_\mathrm{max} \cdot sin(\delta_\mathrm{0})$, $P_\mathrm{e,fault}=0$, and some rearrangements, this leads to the final expression of the critical clearing angle:
% \begin{align}
%     \delta_\mathrm{cc}=arccos\big[~sin(\delta_\mathrm{0}) \cdot (\pi-2 \cdot \delta_\mathrm{0})-cos(\delta_\mathrm{0})~\big] \label{eq:delta-cc}
% \end{align}

% The second step is the calculation of the \acs{CCT} dependent on the critical clearing angle. Splitting the differentiated variables $d^2\delta$ and $dt$ in the combined swing equation and integrating twice, leads to the equation
% \begin{align}
%     \delta=\frac{\omega \cdot \Delta P}{4 H_\mathrm{gen}} \cdot t^2 + \delta_\mathrm{0}. \nonumber
% \end{align}
% Rearranging this gives an expression for calculating the critical clearing time $t_\mathrm{cc}$ (see \autoref{eq:tcc}; similar to \textcite{oedingElektrischeKraftwerkeUnd2016}) for a full line fault (meaning $P_\mathrm{e,fault}=0$) at the bus bar. The derivation of this formula is added in \autoref{app:derivation-analytical}.\mycomment[MK]{Vollständigen Rechenweg in den Anhang mit aufnehmen?}
% \begin{align}
%     t_\mathrm{cc}=\sqrt{\frac{4 H_\mathrm{gen} \cdot (\delta_\mathrm{cc}-\delta_\mathrm{0})}{\omega \cdot \Delta P}} \label{eq:tcc}
% \end{align}