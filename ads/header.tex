%!TEX root = ../main.tex

%
% Nahezu alle Einstellungen koennen hier getaetigt werden
%

\RequirePackage[l2tabu, orthodox]{nag}	% weist in Commandozeile bzw. log auf veraltete LaTeX Syntax hin

\documentclass[%
    % final,
	pdftex,
	oneside,			% Einseitiger Druck (oneside) oder zweiseitig (twoside)
	% headings=openright,	% Kapitelanfänge immer auf rechter Seite (bei zweiseitig)
	cleardoublepage=empty,	% Leere Vakatseiten
	12pt,				% Schriftgroesse
	parskip=half,		% Halbe Zeile Abstand zwischen Absätzen (half).
%	topmargin = 10pt,	% Abstand Seitenrand (Std:1in) zu Kopfzeile [laut log: unused]
	headheight = 20pt,	% Höhe der Kopfzeile
%	headsep = 30pt,	% Abstand zwischen Kopfzeile und Text Body  [laut log: unused]
	headsepline,		% Linie nach Kopfzeile.
	% footsepline,		% Linie vor Fusszeile.
	footheight = 20pt,	% Höhe der Fusszeile
	abstracton,		% Abstract Überschriften
	DIV=calc,		% Satzspiegel berechnen
	headinclude=false,	% Kopfzeile nicht in den Satzspiegel einbeziehen
	footinclude=false,	% Fußzeile nicht in den Satzspiegel einbeziehen
	listof=totoc,		% Abbildungs-/ Tabellenverzeichnis im Inhaltsverzeichnis darstellen
	toc=bibliography,	% Literaturverzeichnis im Inhaltsverzeichnis darstellen
	pointlessnumbers,
	% chapterprefix=true,
	% bibliography=openstyle
]{scrreprt}	% Koma-Script report-Klasse (scrreprt), fuer laengere Bachelorarbeiten alternativ auch: scrbook

\raggedbottom

% Einstellungen laden
\usepackage{xstring}
\usepackage[utf8]{inputenc}
\usepackage[T1]{fontenc}

\newcommand{\einstellung}[1]{%
  \expandafter\newcommand\csname #1\endcsname{}
  \expandafter\newcommand\csname setze#1\endcsname[1]{\expandafter\renewcommand\csname#1\endcsname{##1}}
}
\newcommand{\langstr}[1]{\einstellung{lang#1}}

\input{ads/einstellungen_liste} % verfügbare Einstellungen
%%%%%%%%%%%%%%%%%%%%%%%%%%%%%%%%%%%%%%%%%%%%%%%%%%%%%%%%%%%%%%%%%%%%%%%%%%%%%%%
%                                   Einstellungen
%
% Hier können alle relevanten Einstellungen für diese Arbeit gesetzt werden.
% Dazu gehören Angaben u.a. über den Autor sowie Formatierungen.
%%%%%%%%%%%%%%%%%%%%%%%%%%%%%%%%%%%%%%%%%%%%%%%%%%%%%%%%%%%%%%%%%%%%%%%%%%%%%%%


%%%%%%%%%%%%%%%%%%%%%%%%%%%%%%%%%%%% Sprache %%%%%%%%%%%%%%%%%%%%%%%%%%%%%%%%%%%
%% Aktuell sind Deutsch und Englisch unterstützt.
%% Es werden nicht nur alle vom Dokument erzeugten Texte in
%% der entsprechenden Sprache angezeigt, sondern auch weitere
%% Aspekte angepasst, wie z.B. die Anführungszeichen und
%% Datumsformate.
\setzesprache{de} % de oder en
%%%%%%%%%%%%%%%%%%%%%%%%%%%%%%%%%%%%%%%%%%%%%%%%%%%%%%%%%%%%%%%%%%%%%%%%%%%%%%%%



%%%%%%%%%%%%%%%%%%%%%%%%%%%%%%%%%%% Angaben  %%%%%%%%%%%%%%%%%%%%%%%%%%%%%%%%%%%
%% Die meisten der folgenden Daten werden auf dem
%% Deckblatt angezeigt, einige auch im weiteren Verlauf
%% des Dokuments.
\setzematrikelnr{23176975}
\setzekurs{TMB19 AM2}
\setzetitel{Critical clearing time of synchronous generators}
\setzedatumAbgabe{\today}
\setzefirma{Schaeffler Technologies AG \& Co. KG}
\setzefirmenort{Industriestraße 1-3, 91074 Herzogenaurach}
\setzeabgabeort{Erlangen}
\setzestudiengang{Energy Technology}
\setzedhbw{Mannheim}
\setzebetreuer{Dipl.-Phys. Philipp Hörning}
\setzegutachter{Prof. Dr.-Ing. Simon Thiele}
\setzezeitraum{27.06.2022\ -\ 19.09.2022}
\setzearbeit{Seminararbeit}
\setzeautor{Maximilian Köhler} 




%%%%%%%%%%%%%%%%%%%%%%%%%%%% Literaturverzeichnis %%%%%%%%%%%%%%%%%%%%%%%%%%%%%%
%% Bei Fehlern während der Verarbeitung bitte in ads/header.tex bei der
%% Einbindung des Pakets biblatex (ungefähr ab Zeile 110,
%% einmal für jede Sprache), biber in bibtex ändern.
\newcommand{\ladeliteratur}{%
\addbibresource{literatur.bib}
%\addbibresource{weitereDatei.bib}
}
%% Zitierstil
%% siehe: http://ctan.mirrorcatalogs.com/macros/latex/contrib/biblatex/doc/biblatex.pdf (3.3.1 Citation Styles)
%% mögliche Werte z.B numeric-comp, alphabetic, authoryear, numeric
\setzezitierstil{ieee}
%\setzezitierstil{alphabetic}
%\setzezitierstil{authoryear}




%%%%%%%%%%%%%%%%%%%%%%%%%%%%%%%%% Layout %%%%%%%%%%%%%%%%%%%%%%%%%%%%%%%%%%%%%%%
%% Verschiedene Schriftarten
% laut nag Warnung: palatino obsolete, use mathpazo, helvet (option scaled=.95), courier instead
\setzeschriftart{charter} % palatino oder goudysans, lmodern, libertine, mathptmx
% palatino oder lmodern ganz nett

%% Paket um Textteile drehen zu können
%\usepackage{rotating}
%% Paket um Seite im Querformat anzuzeigen
%\usepackage{lscape}

%% Seitenränder
% \setzeseitenrand{2.5cm}

%% Abstand vor Kapitelüberschriften zum oberen Seitenrand
\setzekapitelabstand{0pt}

%% Spaltenabstand
\setzespaltenabstand{10pt}
%%Zeilenabstand innerhalb einer Tabelle
\setzezeilenabstand{1.5}




%%%%%%%%%%%%%%%%%%%%%%%%%%%%% Verschiedenes  %%%%%%%%%%%%%%%%%%%%%%%%%%%%%%%%%%%
%% Farben (Angabe in HTML-Notation mit großen Buchstaben)
\newcommand{\ladefarben}{%
	\definecolor{LinkColor}{HTML}{00007A}
	\definecolor{ListingBackground}{HTML}{FCF7DE}
}

%%Mathematikpakete benutzen (Pakete aktivieren)
\usepackage{amsmath}
\usepackage{amssymb}

%% Programmiersprachen Highlighting (Listings)
\newcommand{\listingsettings}{%
	\lstset{%
		language=Java,			% Standardsprache des Quellcodes
		numbers=left,			% Zeilennummern links
		stepnumber=1,			% Jede Zeile nummerieren.
		numbersep=5pt,			% 5pt Abstand zum Quellcode
		numberstyle=\tiny,		% Zeichengrösse 'tiny' für die Nummern.
		breaklines=true,		% Zeilen umbrechen wenn notwendig.
		breakautoindent=true,	% Nach dem Zeilenumbruch Zeile einrücken.
		postbreak=\space,		% Bei Leerzeichen umbrechen.
		tabsize=2,				% Tabulatorgrösse 2
		basicstyle=\ttfamily\footnotesize, % Nichtproportionale Schrift, klein für den Quellcode
		showspaces=false,		% Leerzeichen nicht anzeigen.
		showstringspaces=false,	% Leerzeichen auch in Strings ('') nicht anzeigen.
		extendedchars=true,		% Alle Zeichen vom Latin1 Zeichensatz anzeigen.
		captionpos=b,			% sets the caption-position to bottom
		backgroundcolor=\color{ListingBackground}, % Hintergrundfarbe des Quellcodes setzen.
		xleftmargin=0pt,		% Rand links
		xrightmargin=0pt,		% Rand rechts
		frame=single,			% Rahmen an
		frameround=ffff,
		rulecolor=\color{darkgray},	% Rahmenfarbe
		fillcolor=\color{ListingBackground},
		keywordstyle=\color[rgb]{0.133,0.133,0.6},
		commentstyle=\color[rgb]{0.133,0.545,0.133},
		%stringstyle=\color[rgb]{0.627,0.126,0.941}
		stringstyle=\color{red}
	}
}




%%%%%%%%%%%%%%%%%%%%%%%%%%%%%%%% Eigenes %%%%%%%%%%%%%%%%%%%%%%%%%%%%%%%%%%%%%%%
%% Hier können Ergänzungen zur Präambel vorgenommen werden (eigene Pakete, Einstellungen)

%colorpackage
% \usepackage{color}
\usepackage[table,dvipsnames]{xcolor}


%listing package for defining a new language
\usepackage{listings}

% starting footnotes at 1
% \usepackage{perpage}
% \MakePerPage[1]{footnote}

%Functions
\def\bsq#1{%both single quotes
\lq{#1}\rq}


%%%%%%%%%%%%%%%%%%%%%%%%%%%%%%%%%%%%%%%%%%%%
% selfmade language pakets
% JavaScript language
\definecolor{lightgray}{RGB}{227, 227, 227}
\definecolor{darkgray}{RGB}{100, 100, 100}
\definecolor{purple}{rgb}{0.65, 0.12, 0.82}
\definecolor{schaeffler}{RGB}{0, 137, 61}
\definecolor{dark-green}{RGB}{0, 110, 93}
\definecolor{middle-green}{RGB}{115, 161, 149}
\definecolor{light-green}{RGB}{199, 222, 160}

\definecolor{pie1}{RGB}{115, 161, 149}
\definecolor{pie2}{RGB}{192, 198, 191}
\definecolor{pie3}{RGB}{135, 135, 135}
\definecolor{pie4}{RGB}{29, 155, 178}
\definecolor{pie5}{RGB}{182, 186, 194}
\definecolor{pie6}{RGB}{161, 200, 97}
\definecolor{pie7}{RGB}{67, 99, 91}
\definecolor{pie8}{RGB}{112, 123, 110}
\definecolor{pie9}{RGB}{113, 113, 113}

\lstdefinelanguage{JS}{
  keywords={break, case, catch, continue, debugger, default, delete, do, else, false, finally, for, function, if, in, instanceof, new, null, return, switch, this, throw, true, try, typeof, var, void, while, with},
  morecomment=[l]{//},
  morecomment=[s]{/*}{*/},
  morestring=[b]',
  morestring=[b]",
  ndkeywords={class, export, boolean, throw, implements, import, this},
  keywordstyle=\color{blue}\bfseries,
  ndkeywordstyle=\color{darkgray}\bfseries,
  identifierstyle=\color{black},
  commentstyle=\color[rgb]{0.133,0.545,0.133},
  %commentstyle=\color{purple}\ttfamily,
  stringstyle=\color{red}\ttfamily,
  sensitive=true
}
 % lese Einstellungen

\input{lang/strings} % verfügbare Strings
\input{lang/\sprache} % Übersetzung einlesen

% Einstellung der Sprache des Paketes Babel und der Verzeichnisüberschriften
\iflang{de}{\usepackage[english, ngerman]{babel}}
\iflang{en}{\usepackage[ngerman, english]{babel}}

%%%%%%% Package Includes %%%%%%%

\usepackage[left=3cm,right=2cm,top=2.5cm,bottom=2.5cm,foot=1cm]{geometry}	% Seitenränder und Abstände
\usepackage[activate]{microtype} %Zeilenumbruch und mehr
\usepackage[onehalfspacing]{setspace}
\usepackage{makeidx}
\usepackage[autostyle=true,german=quotes]{csquotes}
\usepackage{longtable}
\usepackage{enumitem}	% mehr Optionen bei Aufzählungen
\usepackage{graphicx}
\usepackage{pdfpages}   % zum Einbinden von PDFs
% \usepackage[table]{xcolor} 	% für HTML-Notation
\usepackage{float}
\usepackage{array}
\usepackage{calc}		% zum Rechnen (Bildtabelle in Deckblatt)
\usepackage[right]{eurosym}
\DeclareUnicodeCharacter{20AC}{\euro}
\usepackage{wrapfig}
\usepackage{pgffor} % für automatische Kapiteldateieinbindung
\usepackage[hang, multiple, stable]{footmisc} % Fussnoten; perpage für jede Seite
\usepackage{chngcntr}
\counterwithout{footnote}{chapter}
\usepackage[printonlyused]{acronym} % falls gewünscht kann die Option footnote eingefügt werden, dann wird die Erklärung nicht inline sondern in einer Fußnote dargestellt
\usepackage{listings}
\usepackage{tabularx}
% \usepackage{pdflscape}
\usepackage{lscape}
\usepackage{rotating}
\usepackage[labelfont={bf},font={small},format={hang},singlelinecheck={false}]{caption} % Einstellungen für Bildunterschriften/Tabellenüberschriften
\setcapwidth[c]{.9\textwidth}
\usepackage{subcaption} 
\usepackage{ltxtable}
% \usepackage{filecontents}
\setlength{\skip\footins}{10pt plus 6pt minus 0pt} % Abstand zwischen Fußnoten und Fließtext erhöhen. Latex-Standard: 10pt plus 4pt minus 2pt
\usepackage{mathtools}
\usepackage{tikz}
\usetikzlibrary{positioning}
\usepackage{pgf-pie} % https://www.namsu.de/Extra/pakete/Pie_Chart.html
\usepackage{pgfplots}
\pgfplotsset{compat=1.16}
\usepackage{multirow}
\usepackage[absolute]{textpos}
\usepackage{tcolorbox}
% \usepackage{minitoc}
\usepackage{afterpage}
% \usepackage[Glenn]{fncychap}
% Sonny, Lenny, Glenn, Conny, Rejne, Bjarne, Bjornstrup

% Notizen. Einsatz mit \todo{Notiz} oder \todo[inline]{Notiz}.
\usepackage[obeyFinal,backgroundcolor=yellow,linecolor=black]{todonotes}
% Alle Notizen ausblenden mit der Option "final" in \documentclass[...] oder durch das auskommentieren folgender Zeile
% \usepackage[disable]{todonotes}

% Kommentarumgebung. Einsatz mit \comment{}. Alle Kommentare ausblenden mit dem Auskommentieren der folgenden und dem aktivieren der nächsten Zeile.
% \newcommand{\comment}[1]{\par {\bfseries \itshape \color{blue} #1 \par}} %Kommentar anzeigen
%\newcommand{\comment}[1]{} %Kommentar ausblenden


%%%%%% Configuration %%%%%

%% Anwenden der Einstellungen

\usepackage{\schriftart}
% Überschriften auch in gesetzter Schriftart
\setkomafont{disposition}{%
	\normalfont\bfseries
}
\setkomafont{dictum}{\normalfont}
% Verwendung der Schrift ohne Serifen
% \renewcommand*{\familydefault}{\sfdefault}
% \addtokomafont{disposition}{\sffamily}

\ladefarben{}

% Titel, Autor und Datum
\title{\titel}
\author{\autor}
\date{\datum}

% PDF Einstellungen
\usepackage[%
	pdftitle={\titel},
	pdfauthor={\autor},
	pdfsubject={\arbeit},
	pdfcreator={pdflatex, LaTeX with KOMA-Script},
	pdfpagemode=UseOutlines, 		% Beim Oeffnen Inhaltsverzeichnis anzeigen
	pdfdisplaydoctitle=true, 		% Dokumenttitel statt Dateiname anzeigen.
	pdflang={\sprache}, 			% Sprache des Dokuments.
	%hidelinks,						% entfernt Umrandung von verlinkten Stellen, ohne Verlinkung zu löschen
]{hyperref}

% (Farb-)einstellungen für die Links im PDF
\hypersetup{%
	colorlinks=true, 		% Aktivieren von farbigen Links im Dokument
	linkcolor=black, 	% Farbe festlegen
	citecolor=black,
	filecolor=black,
	menucolor=black,
	urlcolor=black,
	linktocpage=true, 		% Nicht der Text sondern die Seitenzahlen in Verzeichnissen klickbar
	bookmarksnumbered=true 	% Überschriftsnummerierung im PDF Inhalt anzeigen.
}
% Workaround um Fehler in Hyperref, muss hier stehen bleiben
\usepackage{bookmark} %nur ein latex-Durchlauf für die Aktualisierung von Verzeichnissen nötig

% Schriftart in Captions etwas kleiner
\addtokomafont{caption}{\small}

% Literaturverweise (sowohl deutsch als auch englisch)
\iflang{de}{%
\usepackage[
	backend=biber,		% empfohlen. Falls biber Probleme macht: bibtex
	bibwarn=true,
	bibencoding=utf8,	% wenn .bib in utf8, sonst ascii
	% sortlocale=de_DE,
	sorting=none,		% Altenativen: https://tex.stackexchange.com/questions/51434/biblatex-citation-order
	style=\zitierstil,
]{biblatex}
}
\iflang{en}{%
\usepackage[
	backend=biber,		% empfohlen. Falls biber Probleme macht: bibtex
	bibwarn=true,
	bibencoding=utf8,	% wenn .bib in utf8, sonst ascii
	% sortlocale=en_US,
	sorting=none,
	style=\zitierstil,
]{biblatex}
}

\setcounter{biburlnumpenalty}{100}
\setcounter{biburlucpenalty}{100}
\setcounter{biburllcpenalty}{100}

\ladeliteratur{}

% Glossar
\usepackage[nonumberlist,toc,automake]{glossaries}
\addtokomafont{descriptionlabel}{\normalfont\bfseries}

%Kopf- und Fußzeilen
\usepackage[plainfootsepline=yes]{scrlayer-scrpage}

%%%%%% Additional settings %%%%%%

% Hurenkinder und Schusterjungen verhindern
% http://projekte.dante.de/DanteFAQ/Silbentrennung
\clubpenalty = 10000 % schließt Schusterjungen aus (Seitenumbruch nach der ersten Zeile eines neuen Absatzes)
\widowpenalty = 10000 % schließt Hurenkinder aus (die letzte Zeile eines Absatzes steht auf einer neuen Seite)
\displaywidowpenalty=10000

% Bildpfad
\graphicspath{{images/}}

% Einige häufig verwendete Sprachen
\lstloadlanguages{PHP,Python,Java,C,C++,bash}
\listingsettings{}
% Umbennung des Listings
\renewcommand\lstlistingname{\langlistingname}
\renewcommand\lstlistlistingname{\langlistlistingname}
\def\lstlistingautorefname{\langlistingautorefname}

% Abstände in Tabellen
\setlength{\tabcolsep}{\spaltenabstand}
\renewcommand{\arraystretch}{\zeilenabstand}

% Anhangsverzeichnis
\makeatletter% --> De-TeX-FAQ
% Weitergabe des folgenden Codes oder Modifikationen davon nur unter Nennung
% der Originalquelle: <http://www.komascript.de/comment/1073#comment-1073>,
% gestattet.
% Leistungsfähigere Lösung unter <https://komascript.de/comment/5578#comment-5578>.

% Inhaltsverzeichnis für den Anhang erstellen 
% \newcommand*{\maintoc}{% Hauptinhaltsverzeichnis
%   \begingroup
%     \@fileswfalse% kein neues Verzeichnis öffnen
%     \renewcommand*{\appendixattoc}{% Trennanweisung im Inhaltsverzeichnis
%       \value{tocdepth}=-10000 % lokal tocdepth auf sehr kleinen Wert setzen
%     }%
%     \tableofcontents% Verzeichnis ausgeben
%   \endgroup
% }
% \newcommand*{\appendixtoc}{% Anhangsinhaltsverzeichnis
%   \begingroup
%     \edef\@alltocdepth{\the\value{tocdepth}}% tocdepth merken
%     \setcounter{tocdepth}{-10000}% Keine Verzeichniseinträge
%     \renewcommand*{\contentsname}{% Verzeichnisname ändern
%       \langanhang}%
%     \renewcommand*{\appendixattoc}{% Trennanweisung im Inhaltsverzeichnis
%       \setcounter{tocdepth}{\@alltocdepth}% tocdepth wiederherstellen
%     }%
%     \tableofcontents% Verzeichnis ausgeben
%     \setcounter{tocdepth}{\@alltocdepth}% tocdepth wiederherstellen
%   \endgroup
% }
% \newcommand*{\appendixattoc}{% Trennanweisung im Inhaltsverzeichnis
% }
% \g@addto@macro\appendix{% \appendix erweitern
%   \if@openright\cleardoublepage\else\clearpage\fi% Neue Seite
%   \phantomsection
%   \addcontentsline{toc}{chapter}{\appendixname}% Eintrag ins Hauptverzeichnis
%   \addtocontents{toc}{\protect\appendixattoc}% Trennanweisung in die toc-Datei
% }
% \makeatother

% Kreisdiagramme
\def\printonlylargeenough#1#2{\unless\ifdim#2pt<#1pt\relax
#2\printnumbertrue
\else
\printnumberfalse
\fi}
\newif\ifprintnumber

\newcommand{\tab}{~~~~~}

\clearpairofpagestyles
\ohead[]{\textsc{\headmark}}				% Kopfzeile außen immer mit Headmark versehen
\automark[section]{chapter}		% Headmark bestehend aus Kolumnentitel
\ofoot[\pagemark]{\pagemark}	% Fußzeile mit Seitenzahl außen
\renewcommand*\chapterpagestyle{plain.scrheadings}
% \renewcommand*\partpagestyle{plain.scrheadings}		%Bei Verwendung von Parts als Überschriftenebene: Setzen des Pagestyles global

