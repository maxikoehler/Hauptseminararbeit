\chapter*{Aufgabenstellung der Arbeit}
%\vspace*{4cm}
\thispagestyle{plain.scrheadings}
{\large \textbf{Thema:} \parbox[t]{0.8\textwidth}{Critical clearing time of synchronous generators}}
\newline

The critical clearing time (CCT) is an essential parameter in power system stability
analysis. For example, in the case of synchronous generators, the CCT determines the
maximum fault-clearing time a generator can withstand without losing synchronism.
This seminar will introduce the concept of CCT computing. We will discuss the factors
influencing CCT, such as generator characteristics, system parameters, and fault type,
and explore the methods used to calculate CCT in practical power system analysis.
