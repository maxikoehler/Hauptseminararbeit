%!TEX root = ../main.tex

%%%%%%%%%%%%%%%%%%%%%%%%%%%%%%%
\chapter{Basics}
% Evtl. lieber Basics nennen; Stand der Technik ist es ja nicht wirklich
\label{chap:sota}

General sources in terms of standard literature: \autocite{oedingElektrischeKraftwerkeUnd2016,gloverPowerSystemAnalysis2017,kundurPowerSystemStability2022,machowskiPowerSystemDynamics2020}

Relevant basics:
\begin{itemize}
        \item dynamic behavior synchronous generators
        \item determination of \acs{CCT} (equal area criteria)
        \item relevant faults, their modeling and effects
        \item analytic ways to calculate the \acs{CCT}
        \item numerical methods for solving differential equations
\end{itemize}

\section{Basics synchronous generators}

\section{System stability esp. transient context}

\section{Numerical methods for TDSs and system modeling}

\section{Events harming the system stability}

%%%%%%%%%%%%%%%%%%%%%%%%%%%%%%%
\chapter{Numerical modelling}
\label{chap:methods}

\section{Object relations and classes}

\section{Algorithm and functional structure}

\section{Implementation of functions and dependencies}

\section{Implementation of numerical solvers}

\subsection*{Euler's method}

\subsection*{Heun's method}
Heun's method is implemented in Python. An example is provided in \autoref{lst:model-func-heun}

%%%%%%%%%%%%%%%%%%%%%%%%%%%%%%%
\chapter{Results}
\label{chap:results}

\section{Analytical results}

\section{Numerical results}

%%%%%%%%%%%%%%%%%%%%%%%%%%%%%%%
\chapter{Discussion}
\label{chap:discussion}

\section{Analytical vs. numerical}

\section{Single machine vs. network models}