%!TEX root = ../main.tex

%%%%%%%%%%%%%%%%%%%%%%%%%%%%%%%
%%%%%%%%%%%%%%%%%%%%%%%%%%%%%%%
\chapter{Fundamentals}
% Evtl. lieber Basics nennen; Stand der Technik ist es ja nicht wirklich
\label{chap:sota}

\commenting{Input of basic knowledge for system modelling; Maybe supplementary knowledge} \mycomment[MK]{Write Chap Fundamentals}

General sources in terms of standard literature: \autocite{oedingElektrischeKraftwerkeUnd2016,gloverPowerSystemAnalysis2017,kundurPowerSystemStability2022,machowskiPowerSystemDynamics2020}

%%%%%%%%%%%%%%%%%%%%%%%%%%%%%%%
\section{Basics synchronous generators}
\label{sec:basics-sg}
\commenting{
        \begin{itemize}
                \item characteristics of a synchronous generator; structure and types of SG's
                \item mathematical background and description of the behavior -> dynamic modelling
                \item \textbf{Swing equations}
        \end{itemize}
}

The final swing equation system can be derived to following two equations, which have to be solved in every time step to determine the pole angle $\delta$ and the rotor speed $\omega$, respectively the rotor speed change from its base value $\Delta\omega$:
\begin{align}
        \frac{d\delta}{dt}&=\Delta\omega \label{eq:swing1} \\[12pt]
        \frac{d\Delta\omega}{dt}&=\frac{1}{2 \cdot H_\mathrm{gen}} \cdot (P_\mathrm{m} - P_\mathrm{e}) \label{eq:swing2}
\end{align}
where

\begin{tabularx}{\textwidth}[H]{ll}
        $\delta$                & power angle \\
        $\Delta\omega$          & change of rotor angular speed \\
        $H_\mathrm{gen}$        & inertia constant of the \acs{SG} \\
        $P_\mathrm{m}$          & mechanical power of the turbine \\
        $P_\mathrm{e}$          & electrical power demanded transferred out of the \acs{SG}
\end{tabularx}
\vspace{12pt}

The generation of a \ac{TDS} for this equation system takes place in \autoref{sec:tds}.

%%%%%%%%%%%%%%%%%%%%%%%%%%%%%%%
\section{System stability esp. transient context}
\commenting{
        \begin{itemize}
                \item What is to be analyzed? And why? -> different stability analysis
                \item rotor angle stability,
                \item \textbf{derivation of EAC,}
                \item basic assessment models (single machine infinite bus, see \autocite{kundurPowerSystemStability2022})
        \end{itemize}
}

\begin{figure}[H]
        \centering
        \vspace{1cm}
        \begin{circuitikz}[european, scale=.9, smallR/.style={resistor,resistors/scale=.7}]
            \draw (0,0) node[oscillator, anchor=east, name=gen]{} --(.5,0)
                to[L, name=X_g] ++(2,0) coordinate(f1)
                \bushere{1}{$\underline{E}_\mathrm{T}'$}{}
                to[oosourcetrans,prim=delta,sec=wye] ++(2,0)
                \bushere{1.25}{$\underline{E}_\mathrm{T}''$}{} coordinate(b1) ++(0,-.5) -- coordinate(f2) ++(1.5,0)
                to[L, name=X_l2] ++(1,0) -- ++(1.5,0) ++(0,.5)
                \bushere{1.25}{$\underline{E}_\mathrm{ibb}$}{} ++(0,.5) coordinate(b2) ++(0,-.5) -- coordinate(f3) ++(1,0)
                node[gridnode, anchor=left, name=ib]{};
            \draw (b1) ++(0,.5) to[L, name=X_l1] (b2);
            % \draw[line width=2pt] (2.25,1) -- (2.25,-1);
            % \draw[line width=2pt] (4.75,1) -- (4.75,-1);
            % \draw[line width=2pt] (8.25,1) -- (8.25,-1);
            \node[above=6pt] at (X_g) {$X_\mathrm{g}'$};
            \node[above=6pt] at (X_l1) {$X_\mathrm{l}$};
            \node[below=6pt] at (X_l2) {$X_\mathrm{l}$};
            \path[->] (-1.2,.5) edge [bend right] node[left=6pt]{$E_\mathrm{p}~\angle~\delta$} (-1.2,-.5);
            \path[->] (ib) ++(.8,.5) edge [bend left] node[right=6pt]{$E_\mathrm{ibb}~\angle~0^{\circ}$} ++(0,-1);
            \draw[-Stealth, very thick, red] (f1) ++(0,-.5) -- ++(-.15,-.45) -- ++(.3,.2) -- ++(-.2,-.6) coordinate(f1_text);
            \node[below, red] at (f1_text) {\scriptsize fault 1};
            \draw[-Stealth, very thick, red] (f2) ++(0,.3) -- ++(-.15,-.45) -- ++(.3,.2) -- ++(-.2,-.6) coordinate(f2_text);
            \node[below, red] at (f2_text) {\scriptsize fault 2};
            \draw[-Stealth, very thick, red] (f3) ++(0,.3) -- ++(-.15,-.45) -- ++(.3,.2) -- ++(-.2,-.6) coordinate(f3_text);
            \node[below, red] at (f3_text) {\scriptsize fault 3};
        \end{circuitikz}
        \vspace{.5cm}
        \caption[Representative circuit of a \acf{SMIB} model]{Representative circuit of a \acf{SMIB} model with pole wheel voltage $E_\mathrm{p}~\angle~\delta$ and \acf{IBB} voltage $E_\mathrm{ibb}~\angle~0^{\circ}$; positions of considered faults 1 to 3 are marked with red lightning arrows}
        \label{fig:smib-model}        
\end{figure}
where

\begin{tabularx}{\textwidth}[H]{lX}
        $\delta$                        & power angle \\
        $E_\mathrm{p}$                  & pole potential of the \acf{SG} \\
        $\underline{E}_\mathrm{T}'$     & complex potential on the primary side of the transformer \\
        $\underline{E}_\mathrm{T}''$    & complex potential on the secondary side of the transformer \\
        $\underline{E}_\mathrm{ibb}$    & complex pole potential of the \acf{IBB} \\
        $X_\mathrm{g}'$                 & reactance of the \acf{SG} \\
        $X_\mathrm{l}$                  & reactance of a single line \\
\end{tabularx}

%%%%%%%%%%%%%%%%%%%%%%%%%%%%%%%
\section{Events harming the system stability}
\commenting{
        \begin{itemize}
                \item \textbf{fault types,}
                \item load-changes
                \item effects of electrical networks (esp. generator networks) vs. single machine systems -> paper \autocite{batchuComparativeStudyEqual2022}: IBB not that extremely fixed, group of critical and non-critical machines; but more of an outlook and targeting real-time calculation (for system operation)
        \end{itemize}
}

%%%%%%%%%%%%%%%%%%%%%%%%%%%%%%%
\section{Numerical methods for TDSs and system modeling}
\commenting{
        \begin{itemize}
                \item \textbf{solving second order ODEs (explicit)}
                \item Differentiation explicit/impolicit, inertial value problems, boundary value problems, ...
        \end{itemize}
}

System dynamics is a method for describing, understand, and discuss complex problems in the context of system theory \textbf{[SOURCE]}. They often can be described through a set of coupled \acfp{ODE}, most resoluted in time dimension. \commenting{How to bridge towards different boundary types, explicit and implicit methods, ...; Different solving methods, ...} 

\acsp{ODE} can be solved through numerical integration with different methods. An easy and less complex method is Euler's method. It uses a linear extrapolation to calculate the functions value at the next timestep, so following the iterable function
\begin{align}
        f_\mathrm{t+1}=f_\mathrm{t}+\left(\frac{df}{dt}\right)_\mathrm{t} \cdot \Delta t \label{eq:euler},
\end{align}
with $t$ being the time and $f$ an on $t$ dependent function. Generally a system of second order \acsp{ODE} can be rewritten as two first order equations. This often simplifies the calculation or the use of numerical methods. The presented swing equation of a \acs{SG} in \autoref{eq:swing1} and \autoref{eq:swing2} has been split up by that principle.  

%%%%%%%%%%%%%%%%%%%%%%%%%%%%%%%
%%%%%%%%%%%%%%%%%%%%%%%%%%%%%%%
\chapter{Numerical modelling}
\label{chap:methods}

Following chapter will describe the implementation of Python Code for solving the derived \acs{ODE} system (see \autoref{sec:basics-sg}). For this the Python version 3.9 was used, in combination with the packages scipy, numpy, and matplotlib.\footnote{documentation and manual can be found on \href{https://scipy.org/}{\itshape https://scipy.org/} \autocite{virtanenSciPyFundamentalAlgorithms2020}, similiar for \href{https://matplotlib.org/}{\itshape matplotlib}, and \href{https://numpy.org/}{\itshape numpy} packages} The complete code is included in the \autoref{app:code}. \mycomment[MK]{Write Chap Methods}

%%%%%%%%%%%%%%%%%%%%%%%%%%%%%%%
\section{Structure of the \acs{CCT} assessment}
\commenting{
        Program plan for determination / algorithm structure, containing:
        \begin{itemize}
                \item Pre questions:
                \begin{enumerate}
                        \item What do I want to know from the algorithm?
                        \item What do I want to see?
                \end{enumerate}
                \item Answers / Hints for the algorithm:
                \begin{enumerate}
                        \item What are needed inputs?
                        \item What are needed functions?
                        \item How do partial results interact with each other / puzzle together to the superior question?
                \end{enumerate}
        \end{itemize}
}

\begin{figure}[H]
        \centering
        \begin{tikzpicture}[node distance = 2cm, auto]
                % Place nodes
                \node [papStart] (Start1){Start};
                \node [papProcess, below of = Start1,label={[shift={(2.7,-0.6)}]\footnotesize\textit{label 1}}] (pro1){Prozess};
                \node [papProcess, below of = pro1,label={[shift={(3,-0.6)}]\footnotesize\textit{label 2}}](pro2){Prozess};
                \node [papDecision, below of = pro2, yshift= -9mm](dec1){Entscheidung};
                \node [papPredProc,  right of = dec1, xshift=25mm](predproc1){\nodepart{two}\shortstack{vordefinierter\\Prozess}};
                \node [papProcess, below of = predproc1,label={[shift={(2.3,-0.6)}]\footnotesize\textit{label 3}}](pro3){Prozess};
                \node [papEnd, below of = dec1, yshift= -20mm] (End) {Ende};
                
                % Place joins
                \coordinate [below of = dec1, yshift= -10mm] (join1);
                
                % Draw edges
                \path [papLine] (Start1) -- (pro1);
                \path [papLine] (pro1) -- (pro2);
                \path [papLine] (pro2) -- (dec1);
                \path [papLine] (dec1) -- node [right] {\papYes} (End);
                \path [papLine] (dec1) -- node [above] {\papNo} (predproc1);
                \path [papLine] (predproc1) -- (pro3);
                \path [papLine] (pro3) |- (join1);
        \end{tikzpicture}
        \caption{Program plan for determing the \acf{CCT}}
        \label{fig:program-plan}
\end{figure}

%%%%%%%%%%%%%%%%%%%%%%%%%%%%%%%
\section{Electrical simplifications and scenario setting}
\commenting{
        \begin{itemize}
                \item Simplification of all the components in SMIB network to a simple network
                \item Transforming into symmetrical components (for determination of shorts -> e.g. transformer)
        \end{itemize}
}

\missingfigure{Simplified networks, which are in interest and shall be simulated with the algorithm; ideally solved with subfigures}

%%%%%%%%%%%%%%%%%%%%%%%%%%%%%%%
\section{Implementation of the time domain solution}
\label{sec:tds}

%%%%%%%%%%%%%%%%%%%%%%%%%%%%%%%
\section{Implementation of the equal area criterion}


\section{Implementation of helping functions}


%%%%%%%%%%%%%%%%%%%%%%%%%%%%%%%
%%%%%%%%%%%%%%%%%%%%%%%%%%%%%%%
\chapter{Results}
\label{chap:results}

\mycomment[MK]{Write Chap Results}

%%%%%%%%%%%%%%%%%%%%%%%%%%%%%%%
\section{Analytical results}

%%%%%%%%%%%%%%%%%%%%%%%%%%%%%%%
\section{Numerical results}

%%%%%%%%%%%%%%%%%%%%%%%%%%%%%%%
%%%%%%%%%%%%%%%%%%%%%%%%%%%%%%%
\chapter{Discussion}
\label{chap:discussion}

\mycomment[MK]{Write Chap Discussion}