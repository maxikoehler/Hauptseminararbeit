%!TEX root = ../main.tex

%%%%%%%%%%%%%%%%%%%%%%%%%%%%%%%
\chapter{Fundamentals}
% Evtl. lieber Basics nennen; Stand der Technik ist es ja nicht wirklich
\label{chap:sota}
\todo[inline]{Input of basic knowledge for system modelling; Maybe supplementary knowledge}

General sources in terms of standard literature: \autocite{oedingElektrischeKraftwerkeUnd2016,gloverPowerSystemAnalysis2017,kundurPowerSystemStability2022,machowskiPowerSystemDynamics2020}

\section{Basics synchronous generators}
\commenting{
        \begin{itemize}
                \item \textbf{Swing equations}
                \item Characteristics of a synchronous generator
                \item types of SG's
        \end{itemize}
}

\section{System stability esp. transient context}
\commenting{
        \begin{itemize}
                \item What is to be analyzed? And why? -> different stability analysis
                \item rotor angle stability,
                \item \textbf{derivation of EAC,}
                \item basic assessment models (single machine infinite bus, see \autocite{kundurPowerSystemStability2022})
        \end{itemize}
}

\section{Numerical methods for TDSs and system modeling}
\commenting{
        \begin{itemize}
                \item \textbf{solving second order ODEs (explicit)}
        \end{itemize}
}

\section{Events harming the system stability}
\commenting{
        \begin{itemize}
                \item \textbf{fault types,}
                \item load-changes
                \item effects of electrical networks (esp. generator networks) vs. single machine systems
        \end{itemize}
}

%%%%%%%%%%%%%%%%%%%%%%%%%%%%%%%
\chapter{Numerical modelling}
\label{chap:methods}

% \section{Object relations and classes}
% Describe the classes and objects an elecrtical network is consisting of. Thus building an representation model and simplyfying it to one equation system (smib- or imeac-model). This can then be solved with the help of the further described methods.

\section{Algorithm and functional structure}
\commenting{
        Describing the basic functionality and compartements of the model.
}

\section{Implementation of functions and dependencies}
\commenting{
        Describing implementation into Python-code.
}

\section{Implementation of numerical solvers}
\commenting{
        Describing the functionality and structure of (excplicit) numerical methods. Starting from Euler (basic) to a more complex but more reliable method (Heun, predictor-corrector, ...). Main focus: Implementation into Python.
}

\subsection*{Euler's method}

\subsection*{Heun's method}
Heun's method is implemented in Python. An example is provided in \autoref{lst:model-func-heun}

%%%%%%%%%%%%%%%%%%%%%%%%%%%%%%%
\chapter{Results}
\label{chap:results}

\section{Analytical results}

\section{Numerical results}

%%%%%%%%%%%%%%%%%%%%%%%%%%%%%%%
\chapter{Discussion}
\label{chap:discussion}

\section{Analytical vs. numerical}

\section{Single machine vs. network models}