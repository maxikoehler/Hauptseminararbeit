%!TEX root = ../main.tex

\chapter{Introduction}
\label{chap:intro}

\todo[inline]{Insert some fancy introduction}

\begin{enumerate}\bfseries \singlespacing \small
    \renewcommand{\labelenumi}{\theenumi}
    \renewcommand{\theenumi}{\arabic{enumi}}
    \renewcommand{\labelenumii}{\theenumii}
    \renewcommand{\theenumii}{\theenumi.\arabic{enumii}}
    \setlength{\leftmarginii}{5.4ex}
    \item Introduction (1 page)
    \item State-of-the-art ($\sim$ 4 pages)
    \begin{enumerate} \mdseries
        \item Basics synchronous generators \\
        -> \textbf{swing-equations}
        \item System stability esp. transient context \\
        -> rotor angle stability, \textbf{derivation of EAC,} basic assessment models (single machine infinite bus, see \autocite{kundurPowerSystemStability2022})
        \item Numerical methods for \acsp{TDS} and system modelling \\
        -> \textbf{solving second order ODEs (explicit)}
        \item Events harming the system stability \\
        -> \textbf{faults,} load-changes, effects of electrical networks (esp. generator networks) vs. single machine systems
    \end{enumerate}
    \item Numerical modeling ($\sim$ 5 pages)
    \begin{enumerate} \mdseries
        {\itshape \item (Object relations and classes)}
        \item Algorithm and functional structure
        \item Implementation of functions and dependencies
        \item Implementation of numerical solvers
    \end{enumerate}
    \item Results ($\sim$ 3 pages)
    \begin{enumerate} \mdseries
        \item Analytical results
        \item Numerical results
    \end{enumerate}
    \item Discussion ($\sim$ 2 pages)
    \begin{enumerate} \mdseries
        \item Numerical vs. analytical
        {\itshape \item (Single machine vs. network models)}
        {\itshape \item ... (dependent on time and outcomes)}
    \end{enumerate}
    \item Summary and outlook (1 page)
\end{enumerate}

Total amount $\sim$ 16 pages (without appendix and supplementary pages)

\newpage
Bullet points for the thesis from Ilya:
\begin{itemize}
    \item Swing equation of synchronous generators
    \item Solving the Swing equation with the help of Python -> Solving of second order ODEs
    \item Equal-area criterion -> Derivation of the equations
    \item Simulation of a fault -> applying the equal-area criteria with the help of Python.
    \item Comparison between analytical and (numerical) simulation results
\end{itemize}

\commenting{Das ist ein Testkommentar.}

Introduction via \autocite{vdeverbandderelektrotechnikelektronikinformationstechnike.v.PerspektivenElektrischenEnergieubertragung2019} and other standard literature like \autocite{gloverPowerSystemAnalysis2017,kundurPowerSystemStability2022,machowskiPowerSystemDynamics2020,oedingElektrischeKraftwerkeUnd2016,schwabElektroenergiesystemeSmarteStromversorgung2022}. Need for understanding of Transient stability and therefore critical pole angle and fault clearing time assessment: Running and maintaining the electrical grid; Adding virtual inertia in FACTs and HVDC; Better and faster predicting, due to shorter (critical) fault clearing times; .

% \missingfigure{Insert senseful schematic picture/graph}

\begin{figure}[H]
    \centering
    \vspace{1cm}
    \begin{circuitikz}[european, scale=1, smallR/.style={resistor,resistors/scale=.7}]
        \ctikzset{grounds/scale=1.5}
        \draw (0,0) node[oscillator, anchor=east, name=gen]{} --(.5,0)
            to[L, name=X_g] ++(1,0)
            % to[tmultiwire] ++(1,0)
            to[oosourcetrans,prim=delta,sec=wye] ++(4,0)
            % to[tmultiwire] ++(1,0)
            to[L, name=X_l] ++(2,0) -- ++(2,0)
            % to[tmultiwire] ++(1,0)
            node[gridnode, anchor=left, name=ib]{};
        \draw[line width=2pt] (2.25,1) -- (2.25,-1);
        \draw[line width=2pt] (4.75,1) -- (4.75,-1);
        \draw[line width=2pt] (8.25,1) -- (8.25,-1);
        \node[above=6pt] at (X_g) {$X_\mathrm{g}'$};
        \node[above=6pt] at (X_l) {$X_\mathrm{l}$};
        \path[->] (-1.2,.5) edge [bend right] node[left=6pt]{$E_\mathrm{p}~\angle$} (-1.2,-.5);
        \path[->] (10.8,.5) edge [bend left] node[right=6pt]{$E_\mathrm{ibb}~\angle~0^{\circ}$} (10.8,-.5);
    \end{circuitikz}
    \vspace{1cm}
    \caption{Representative circuit of a \acf{SMIB} model with pole wheel voltage $E_\mathrm{p}~\angle$ and \acf{IBB} voltage $E_\mathrm{ibb}~\angle~0^{\circ}$}
    \todo[inline]{SMIB model with double OHL; different faults: near generator (both lines vs. one line not working) and far away}
    \label{fig:my_label}
\end{figure}