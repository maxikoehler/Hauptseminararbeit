%!TEX root = ../main.tex

\chapter{Introduction}
\label{chap:intro}

Bullet points for the thesis from Ilya:
\begin{itemize}
    \item Swing equation of synchronous generators
    \item Solving the Swing equation with the help of Python -> Solving of second order ODEs
    \item Equal-area criterion -> Derivation of the equations
    \item Simulation of a fault -> applying the equal-area criteria with the help of Python.
    \item Comparison between analytical and (numerical) simulation results
\end{itemize}

Introduction via \autocite{vdeverbandderelektrotechnikelektronikinformationstechnike.v.PerspektivenElektrischenEnergieubertragung2019} and other standard literature like \autocite{gloverPowerSystemAnalysis2017,kundurPowerSystemStability2022,machowskiPowerSystemDynamics2020,oedingElektrischeKraftwerkeUnd2016,schwabElektroenergiesystemeSmarteStromversorgung2022}. Need for understanding of Transient stability and therefore critical pole angle and fault clearing time assessment: Running and maintaining the electrical grid; Adding virtual inertia in FACTs and HVDC; Better and faster predicting, due to shorter (critical) fault clearing times; . \mycomment[MK]{Write Introduction}

\missingfigure{Insert senseful schematic picture/graph}

The goal of this Student Research Paper is the implementation of a \ac{CCT} determing Python algorithm for a \ac{SMIB} model. Therefore a handful of faults or fault scenarios shall be simulated with the program. In combination with a few visualizations the concepts of transient stability assessment, and therefore determing the \ac{CCT} and the critical power angle, is illustrated.

